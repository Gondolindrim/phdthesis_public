\documentclass[
cover=a4,
spinewidth=50mm,
flapwidth=10cm,
wrapwidth=5mm,
]{bookcover}
\newbookcovercomponenttype{center rotate}{
\vfill
\centering
\rotatebox[origin=c]{-90}{#1}
\vfill}
\usepackage[outline]{contour}% It doesn’t work with xelatex and lualatex
\contourlength{1pt}
\usepackage[english]{babel}
\usepackage{kantlipsum,microtype}


\usepackage{GS1,qrcode}

\usepackage{amsthm}
\usepackage[gelasio]{newtxmath}		% Loads the Times-Roman Math Fonts
\usepackage{mathtools}
\usepackage{thmtools}
\usepackage[svgnames,table]{xcolor}
\newcommand{\dpo}{\boldsymbol{\sigma}}


\usepackage{pgfplots}
\usepackage{tikz}
\usetikzlibrary{calc,patterns,angles,quotes,math,external,positioning,shapes,intersections,through,backgrounds,shadings,fadings,decorations, decorations.markings, datavisualization,cd,shapes.geometric,fit
}
\usepgfplotslibrary{external} 

\tikzset{
	testfading/.style n args={3}{
    postaction={
    decorate,
    decoration={
    markings,
    mark=between positions 0 and \pgfdecoratedpathlength step 0.5pt with {
    \pgfmathsetmacro\myval{multiply(
        divide(
        \pgfkeysvalueof{/pgf/decoration/mark info/distance from start}, \pgfdecoratedpathlength
        ),
        100
    )};
    \pgfsetfillcolor{#3!\myval!#2};
    \pgfpathcircle{\pgfpointorigin}{#1};
    \pgfusepath{fill};}
}}}}



\usepackage[siunitx]{circuitikz}
\tikzstyle{every node}=[font=\small]
\ctikzset{bipoles/thickness=1}
\ctikzset{resistors/scale=0.75}
\ctikzset{bipoles/diode/height=0.3}
\ctikzset{bipoles/diode/width=0.3}
\ctikzset{current arrow scale=24}
\ctikzset{label distance=4}

\usepackage[outline]{contour}% It doesn’t work with xelatex and lualate


\begin{document}
\begin{bookcover}
% Remark
\begin{bookcoverelement}{center}{above front}
\textcolor{blue}{A dust jacket example}
\end{bookcoverelement}

\definecolor{darkblue}{RGB}{15,35,65}
\definecolor{waveblue}{RGB}{150,200,255}
\definecolor{textwhite}{RGB}{255,255,255}


% FRONT <<<1
\begin{bookcoverelement}{tikz}{front}
\tikzset{transform shape,>={Stealth[inset=0mm,length=1.5mm,angle'=50]}}
% Define A4 cover size: 21cm wide, 29.7cm tall
\fill[darkblue] (part.south west) rectangle (\partwidth, \partheight);
% Title at the top
\node[text=white, font=\bfseries\large] at (\partwidth/2, 28) {PHD THESIS};
\node[text=white, font=\bfseries\Huge, align=center] at (\partwidth/2, 25) {DYNAMIC PHASOR THEORY OF\\ELECTRICAL CIRCUITS\\UNDER NONSTATIONARY REGIMENS};
%\node[text=white, font=\bfseries\large, align=center] at (10.5, 23) {A STUDY IN MATHEMATICAL ANALYSIS};
% Multiple sinewaves across the middle
\foreach \i in {0,...,15} {
    \draw[white, opacity=0.3, thick, samples=200, domain=0:21] 
    plot (\x, {17 + 1.5*sin(deg(0.5*\x) + \i*15)});
}
% Mathematical formulas
\node[text=white, font=\large] at (5, 11) {$\displaystyle\sum\limits_{i=0}^n \beta_i^n(t) X^{(i)} - F(t) = 0$};
\node[text=white, font=\large] at (16, 11) {$ X(t) = \dfrac{R_0(t)}{2\pi j}\displaystyle\int_{B_\alpha} \mathbf{M}{\left[X\right]}\left(\mu\right) e^{\mu  t} d\mu$};
\node[text=white, font=\large] at (10.5, 8.5) {$\dpo\left[X\right] = \dot{X} + j\omega X$};
% Author name and department at the bottom
\node[text=white, font=\bfseries\large] at (\partwidth/2, 6) {ÁLVARO A. VOLPATO};
\node[text=white, font=\bfseries\large] at (\partwidth/2, 5) {ADVISOR: LUÍS F. C. ALBERTO};
\node[text=white]                       at (\partwidth/2, 4) {DEPARTMENT OF ELECTRICAL AND COMPUTER ENGINEERING};
\node[text=white]                       at (\partwidth/2, 3) {SÃO CARLOS SCHOOL OF ENGINEERING};
\node[text=white]                       at (\partwidth/2, 2) {UNIVERSITY OF SÃO PAULO};
\begin{scope}[yshift=170mm, xshift=\partwidth/2]
\draw [fill=none, white, opacity=0.3, thick] (0,0) circle (30 mm) node [gray] {};
\draw [draw=none, fill=darkblue, opacity=0.75] (0,0) circle (29.8 mm) node [gray] {};
		\draw [->, thick, white, opacity=0.3] (   -40mm,  0   ) -- (   40mm,  0   );
		\draw [->, thick, white, opacity=0.3] (      0, -40mm ) -- (   0   ,  40mm);
		\draw [->, thick, white] (0,0) -- (35mm,0) coordinate(realvec);
		\draw [->, thick, white] (0,0) -- (0,35mm) coordinate(imagvec);
		\node [white] (realveclabel) at ([shift=({-15mm,-3mm})]realvec) {$R = 1e^{j0}$};
		\node [white, right] (realveclabel) at ([shift=({4mm,-2mm})] imagvec) {$I = 1e^{j\frac{\pi}{2}}$};
		\node [white] (reAxisLabel) at (43mm,0) {Re};
		\node [white] (imAxisLabel) at (0,43mm) {Im};
		\draw [->,white, opacity=0.5, thick] ({20mm*cos(290)},{20mm*sin(290)}) arc[start angle=290, end angle = 328, radius = 20mm];
		\node [color=white, opacity=0.5] (philabelrotated) at ({25mm*cos(310)},{25mm*sin(310)}) {$\phi(t)$};
		\node (rotX) at ({38mm*cos(330)},{38mm*sin(330)}) {};
		\node [white] (XomegatLabel) at ({43mm*cos(330)},{43mm*sin(330)}) {$Xe^{j\psi(t)}$};
		\draw [->,thick, white] (0,0) -- (rotX.center);
		\node (rotRe) at ({40mm*cos(290)},{40mm*sin(290)}) {};
		\node (rotIm) at ({40mm*cos(20)},{40mm*sin(20)}) {};
		\node [label={[white, opacity=0.5,label distance=0.0mm, rotate=290]290:$Re^{j\psi(t)}$}] (ReomgatLabel) at ({44mm*cos(290)},{27mm*sin(290)}) {};
		\node [label={[white, opacity=0.5,label distance=0.0mm, rotate=20]20:$Ie^{j\psi(t)}$}] (IomegatLabel) at ({15mm*cos(20)},{15mm*sin(20)}) {};
		\draw [->,thick, white, opacity=0.5] (0,0) -- (rotRe.center);
		\draw [->,thick, white, opacity=0.5] (0,0) -- (rotIm.center);
		\draw [->,thick,gray, dashed, line cap = round] (0,0) -- ({30mm*cos(240)},{30mm*sin(240)});
		\node [label={[gray,label distance=0.0mm]245:$m(t)$}] (mt) at ({30mm*cos(245)},{30mm*sin(245)}) {};
		%\draw [->,gray,thick] ({6mm*cos(25)},{6mm*sin(25)}) arc[start angle=25, end angle = 63, radius = 6mm];
		\node [white, opacity=0.5] (omegat) at ({20mm*cos(135)},{20mm*sin(135)}) {$\psi(t)$};
		\draw [-{Stealth[inset=0mm,length=5mm,angle'=50]}, white, opacity=0.5, line width = 2mm] (13mm,0) arc[start angle=0, end angle = 287, radius = 13mm];
\end{scope}
\end{bookcoverelement} %>>>1
% SPINE <<<1
\begin{bookcoverelement}{tikz}{spine}
% Define A4 cover size: 21cm wide, 29.7cm tall
\fill[darkblue] (part.south west) rectangle (\partwidth, \partheight);
\node[white,label={[white, label distance=5mm,text depth=-3mm, style={align=center}, rotate=-90]:\Huge DYNAMIC PHASOR THEORY OF ELECTRICAL CIRCUITS}] at (\partwidth/2, \partheight/2) {};
\node[white,label={[white, label distance=5mm,text depth=-3mm, style={align=center}, rotate=-90]:\Huge UNDER NONSTATIONARY REGIMENS}] at ({\partwidth/2 - 15mm}, {\partheight/2}) {};
\end{bookcoverelement} %>>>1
% BACK <<<1
\begin{bookcoverelement}{tikz}{back}
\tikzset{transform shape,>={Stealth[inset=0mm,length=1.5mm,angle'=50]}}
% Define A4 cover size: 21cm wide, 29.7cm tall
\fill[darkblue] (part.south west) rectangle (\partwidth, \partheight);
% Multiple sinewaves across the middle
\foreach \i in {0,...,15} {
    \draw[white, opacity=0.3, thick, samples=200, domain=0:21] 
    plot (\x, {17 + 1.5*sin(deg(0.5*\x) + \i*15)});
}
%
\node[text=white, font=\bfseries\large, align=justify, text width = 18cm] at (\partwidth/2, 24) {
	Despite the many Dynamic Phasor frameworks currently available, none have been able to capture all of the qualities that made Classical Phasors so useful: while some require truncations and approximations to be operationalizable — thus sacrificing precision for modelling convenience — some others can represent and reconstruct some signals of interest but do not offer a good theory of complex power under nonstationary regimens. This thesis comprises an endeavour of building a novel Dynamic Phasor Theory, specifically the construction of a theory that offers a representation of nonstationary sinusoids as Dynamic Phasors, as well as reconstructing those signals in time from their phasorial counterpars without the need for approximations or truncations while offering a theory of complex electrical power in nonstationary regimens, justifying the synchrophasor representation of voltages and currents and the steady-state approximation of complex power.
};
%
\node at (\partwidth/2, 5) {\includegraphics[width=5cm]{../images/uniLogo_white.pdf}};
\end{bookcoverelement} %>>>1
% FRONT FLAP <<<1
\begin{bookcoverelement}{tikz}{front flap and wrap}
\tikzset{transform shape,>={Stealth[inset=0mm,length=1.5mm,angle'=50]}}
% Define A4 cover size: 21cm wide, 29.7cm tall
\fill[darkblue] (part.south west) rectangle (\partwidth, \partheight);
\end{bookcoverelement} %>>>1
% BACK FLAP <<<1
\begin{bookcoverelement}{tikz}{back flap and wrap}
\tikzset{transform shape,>={Stealth[inset=0mm,length=1.5mm,angle'=50]}}
% Define A4 cover size: 21cm wide, 29.7cm tall
\fill[darkblue] (part.south west) rectangle (\partwidth, \partheight);
\end{bookcoverelement} %>>>1




\end{bookcover}
\end{document}
