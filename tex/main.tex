% -------------------------------------------------------------------
% APA-Style thesis LaTeX Template
% AUTHOR: Álvaro "Gondolindrim" Volpato (alvaro.volpato@usp.br)
% VERSION: 1.1
% PAGE: http://github.com/Gondolindrim/apaThesis
% LICENSE: Creative-Commons Non-commercial Share-Alike
%--------------------------------------------------------------------

% This is an example of document using the apaThesis class file,
% for a thesis that fits the American Phychology Association standards, described in 

% American Psychology Association (2009). Publication Manual of the American Psychological Association, sixth edition.

% It is supposed to be used as a template for a masters or PhD thesis. It was written in VIM and it contains
% folding data (the three "{") in the text. To enable this folding, type in vim:

% :setfoldmethod=marker

\documentclass{apaThesis}
\usepackage[utf8]{inputenc}
\hyphenation{eve-ry-where}
\hyphenation{in-ver-ti-ble}

% -----------------------------------------------
% DOCUMENT DATA {{{1
% -----------------------------------------------

% Title and author
\title{Dynamic Phasor Theory of Electrical Circuits Under Nonstationary Regimens}
\author{Álvaro Augusto Volpato}
\place{São Carlos, Brazil}
\writingdate{June 2025}
\affiliation{%
  University of São Paulo
  \par
  São Carlos School of Engineering
  \par
  Department of Electrical and Computer Engineering}
\advisor{Professor Dr. Luís Fernando Costa Alberto}
\preamble{Thesis submitted to the Electrical Engineering Graduate Program of the Department of Electrical and Computer Engineering of the São Carlos School of Engineering in partial fulfillment of the requirements for the degree of Doctor of Science in Electrical Engineering, subfield Electrical Power Systems}

\DeclareMathOperator{\Dom}{Dom}
\DeclareMathOperator{\rank}{rank}

\DeclareMathOperator{\nullity}{null}

\DeclareMathOperator{\Ln}{Ln}
\DeclareMathOperator{\Ker}{Ker}
\DeclareMathOperator{\Eig}{Eig}

\newcommand{\dpo}{\boldsymbol{\sigma}}
\newcommand{\ndpo}[1]{\dpo^{#1}}
\newcommand{\dpL}{\boldsymbol{\lambda}}
\newcommand{\apL}{\boldsymbol{\alpha}}
\newcommand{\dpS}{\boldsymbol{\Xi}}

\newcommand*{\hermconj}{{\mathsf{H}}}
\newcommand*{\transpose}{{\mathsf{T}}}
%\newtheorem{theorem}{Theorem}

% TIKZ AND PGFPLOTS CONFIGURATIONS
\usepackage{pgfplots}
\pgfplotsset{
/pgfplots/colormap={hsv}{rgb255=(255,0,0) rgb255=(255,255,0) rgb255=(0,255,0)
rgb255=(0,255,255) rgb255=(0,0,255) rgb255=(255,0,255) rgb255=(255,0,0)}
}
\pgfplotsset{
/pgfplots/colormap={hsv2}{rgb255=(0,0,0) rgb255=(128,0,128) rgb255=(0,0,230)
rgb255=(0,255,255) rgb255=(0,255,0) rgb255=(255,255,0) rgb255=(255,0,0)}
}
\usepgfplotslibrary{fillbetween}

\usepackage{tikz}
\usetikzlibrary{calc,patterns,angles,quotes,math,external,positioning,shapes,intersections,through,backgrounds,shadings,fadings,decorations, decorations.markings, datavisualization,cd,shapes.geometric,fit
}
\usepgfplotslibrary{external} 

\tikzset{
	testfading/.style n args={3}{
    postaction={
    decorate,
    decoration={
    markings,
    mark=between positions 0 and \pgfdecoratedpathlength step 0.5pt with {
    \pgfmathsetmacro\myval{multiply(
        divide(
        \pgfkeysvalueof{/pgf/decoration/mark info/distance from start}, \pgfdecoratedpathlength
        ),
        100
    )};
    \pgfsetfillcolor{#3!\myval!#2};
    \pgfpathcircle{\pgfpointorigin}{#1};
    \pgfusepath{fill};}
}}}}



\usepackage[siunitx]{circuitikz}
\tikzstyle{every node}=[font=\small]
\ctikzset{bipoles/thickness=1}
\ctikzset{resistors/scale=0.75}
\ctikzset{bipoles/diode/height=0.3}
\ctikzset{bipoles/diode/width=0.3}
\ctikzset{current arrow scale=24}
\ctikzset{label distance=4}

\tikzexternalize
% HOW TO COMPILE THIS LATEX SCRIPT
% FIRST, UNCOMMENT THESE PACKAGES AND RUN THE PICTURES:
        %\usepackage{tikz}
        %\pgfrealjobname{entry12}
        %:w | :!for JOB in entry12-pv_pv entry12-pv_iv entry12-dvdg entry12-didg entry12-dvdt entry12-didt entry12-kt_rsv entry12-mpp_resistance entry12-mpp_radius entry12-mpp_angle; do pdflatex --jobname=$JOB --output-directory=build entry12.tex ; done

% MAKE SURE ALL PDFS WERE GENERATED. THEN COMMENT THE tikz and pdfrealjobname PACKAGES AND UNCOMMENT THIS
%\long\def\beginpgfgraphicnamed#1#2\endpgfgraphicnamed{\includegraphics{#1}}

% THEN SUCCESSFULLY RUN
% :!pdflatex --output-directory=build entry12.tex

% ALTERNATIVELY, RUN

%:w | :!for JOB in <<job list separated by spaces>>; do pdflatex --jobname=$JOB --output-directory=build '\let\GenerateTikzJobs=Y\input{entry12.tex}' ; done

        
% The following code automatically includes the needed lines or not based on the 'GenerateTikzJobs' flag, which if run as Y will execute the jobs, else will just compile the pdf
\ifx\GenerateTikzJobs Y
        \usepackage{tikz}
        \pgfrealjobname{doc}
\else
        \long\def\beginpgfgraphicnamed#1#2\endpgfgraphicnamed{\includegraphics{#1}}
\fi

\usepackage{titlesec}

%\titleclass{\part}{page}
%\assignpagestyle{\part}{empty}
%\titleformat{\part}
%    [display]
%    {\centering\sffamily\Huge}
%    {\partname\ \thepart}
%    {0pt}
%    {\huge}
%    [\clearpage]

\usepackage{rotating}

\begin{document}


%\begin{customfrontmatter}\end{customfrontmatter}

\definecolor{darkblue}{RGB}{15,35,65}
\definecolor{waveblue}{RGB}{150,200,255}
\definecolor{textwhite}{RGB}{255,255,255}

\thispagestyle{empty}
% CUSTOM FRONT MATTER <<<
\begin{figure}[h]
\centering
\vspace*{-2.6cm}
\makebox[\linewidth]{

\begin{tikzpicture}[transform shape,>={Stealth[inset=0mm,length=1.5mm,angle'=50]}]

% Define A4 cover size: 21cm wide, 29.7cm tall
\fill[darkblue] (0,0) rectangle (\paperwidth, \paperheight);

% Title at the top
\node[text=white, font=\bfseries\large] at (10.5, 28) {PHD THESIS};
\node[text=white, font=\bfseries\Huge, align=center] at (10.5, 25) {DYNAMIC PHASOR THEORY OF\\ELECTRICAL CIRCUITS\\UNDER NONSTATIONARY REGIMENS};
%\node[text=white, font=\bfseries\large, align=center] at (10.5, 23) {A STUDY IN MATHEMATICAL ANALYSIS};

% Multiple sinewaves across the middle
\foreach \i in {0,...,15} {
    \draw[white, opacity=0.3, thick, samples=200, domain=0:21] 
    plot (\x, {17 + 1.5*sin(deg(0.5*\x) + \i*15)});
}


% Mathematical formulas
\node[text=white, font=\large] at (5, 11) {$\displaystyle\sum\limits_{i=0}^n \beta_i^n(t) X^{(i)} - F(t) = 0$};

\node[text=white, font=\large] at (16, 11) {$ X(t) = \dfrac{R_0(t)}{2\pi j}\displaystyle\int_{B_\alpha} \mathbf{M}{\left[X\right]}\left(\mu\right) e^{\mu  t} d\mu$};
\node[text=white, font=\large] at (10.5, 8.5) {$\dpo\left[X\right] = \dot{X} + j\omega X$};

% Author name and department at the bottom
\node[text=white, font=\bfseries\large] at (\paperwidth/2, 6) {ÁLVARO A. VOLPATO};
\node[text=white, font=\bfseries\large] at (\paperwidth/2, 5) {ADVISOR: LUÍS F. C. ALBERTO};
\node[text=white]                       at (\paperwidth/2, 4) {DEPARTMENT OF ELECTRICAL AND COMPUTER ENGINEERING};
\node[text=white]                       at (\paperwidth/2, 3) {SÃO CARLOS SCHOOL OF ENGINEERING};
\node[text=white]                       at (\paperwidth/2, 2) {UNIVERSITY OF SÃO PAULO};

\begin{scope}[yshift=170mm, xshift=\paperwidth/2]
\draw [fill=none, white, opacity=0.3, thick] (0,0) circle (30 mm) node [gray] {};
\draw [draw=none, fill=darkblue, opacity=0.75] (0,0) circle (29.8 mm) node [gray] {};
		\draw [->, thick, white, opacity=0.3] (   -40mm,  0   ) -- (   40mm,  0   );
		\draw [->, thick, white, opacity=0.3] (      0, -40mm ) -- (   0   ,  40mm);

		\draw [->, thick, white] (0,0) -- (35mm,0) coordinate(realvec);
		\draw [->, thick, white] (0,0) -- (0,35mm) coordinate(imagvec);

		\node[white] (realveclabel) at ([shift=({-15mm,-3mm})]realvec) {$R = 1e^{j0}$};
		\node[white, right] (realveclabel) at ([shift=({4mm,-2mm})] imagvec) {$I = 1e^{j\frac{\pi}{2}}$};

		\node [white] (reAxisLabel) at (43mm,0) {Re};
		\node [white] (imAxisLabel) at (0,43mm) {Im};

		\draw [->,white, opacity=0.5, thick] ({20mm*cos(290)},{20mm*sin(290)}) arc[start angle=290, end angle = 328, radius = 20mm];

		\node [color=white, opacity=0.5] (philabelrotated) at ({25mm*cos(310)},{25mm*sin(310)}) {$\phi(t)$};

		\node (rotX) at ({38mm*cos(330)},{38mm*sin(330)}) {};
		\node[white] (XomegatLabel) at ({43mm*cos(330)},{43mm*sin(330)}) {$Xe^{j\psi(t)}$};
		\draw [->,thick, white] (0,0) -- (rotX.center);

		\node (rotRe) at ({40mm*cos(290)},{40mm*sin(290)}) {};
		\node (rotIm) at ({40mm*cos(20)},{40mm*sin(20)}) {};

		\node [label={[white, opacity=0.5,label distance=0.0mm, rotate=290]290:$Re^{j\psi(t)}$}] (ReomgatLabel) at ({44mm*cos(290)},{27mm*sin(290)}) {};

		\node [label={[white, opacity=0.5,label distance=0.0mm, rotate=20]20:$Ie^{j\psi(t)}$}] (IomegatLabel) at ({15mm*cos(20)},{15mm*sin(20)}) {};
		\draw [->,thick, white, opacity=0.5] (0,0) -- (rotRe.center);
		\draw [->,thick, white, opacity=0.5] (0,0) -- (rotIm.center);

		\draw [->,thick,gray, dashed, line cap = round] (0,0) -- ({30mm*cos(240)},{30mm*sin(240)});
		\node [label={[gray,label distance=0.0mm]245:$m(t)$}] (mt) at ({30mm*cos(245)},{30mm*sin(245)}) {};

		%\draw [->,gray,thick] ({6mm*cos(25)},{6mm*sin(25)}) arc[start angle=25, end angle = 63, radius = 6mm];
		\node [white, opacity=0.5] (omegat) at ({20mm*cos(135)},{20mm*sin(135)}) {$\psi(t)$};

		\draw [-{Stealth[inset=0mm,length=5mm,angle'=50]}, white, opacity=0.5, line width = 2mm] (13mm,0) arc[start angle=0, end angle = 287, radius = 13mm];
		
\end{scope}

\end{tikzpicture}
}
\end{figure} %>>>
\cleardoublepage

% -----------------------------------------------
% PRETEXTUAL ELEMENTS {{{1
% -----------------------------------------------

% PRINTING TITLE PAGE
\printtitlepage

% PRINTING FRONT MATTER
\printfrontmatter

%--------------------------------------------------------------------------------------------------
% If you do not need a second front matter in another language, delete or comment these lines.
\cleardoublepage
% SECOND FRONT MATTER IN PORTUGUESE
\anotherfrontmatter
{Teoria de Fasores Dinâmicos de Circuitos Elétricos sob Regimes Não estacionários}%
{\theauthor}%
{São Carlos, Brasil}%
{Junho de 2025}%
{Universidade de São Paulo \par %
Escola de Engenharia de São Carlos \par%
Departamento de Engenharia Elétrica e de Computação}%
{Professor Luís Fernando Costa Alberto}%
{Tese submetida ao Programa de Pós-Graduação em Engenharia Elétrica do Departamento de Engenharia Elétrica e de Computação da Escola de Engenharia de São Carlos como requisito para obtenção do grau de Doutor em Ciências, área de Sistemas de Potência}
%--------------------------------------------------------------------------------------------------

\cleardoublepage
% CATALOGRAPHIC CARD
% I was not able to make a program for this as it can differ wildly between countries -- some won't even need one
\thispagestyle{empty}
\vspace*{\fill}
\begin{center}

This is the revised version of the thesis. The original version is available at the Graduate Program in Electrical Engineering at EESC/USP.

\vspace{1cm}

Trata-se da versão corrigida da tese. A versão original se encontra disponível na EESC/USP que aloja o Programa de Pós-Graduação em Engenharia Elétrica.
\end{center}
\newpage

\cleardoublepage
% CATALOGRAPHIC CARD
% I was not able to make a program for this as it can differ wildly between countries -- some won't even need one
\thispagestyle{empty}
\vspace*{\fill}
\begin{center}
I authorize the full or partial reproduction and dissemination of this work, by any conventional or electronic means, for study and research purposes, provided that the source is properly cited.

\vspace{1cm}

Cataloguing-in-publication data prepared by the Prof. Sérgio Rodrigues Fontes Libray and the Communication and Marketing Office at EESC-USP, with information provided by the author. 

\vspace{5mm}
\noindent\fbox{%
	\parbox{0.9\textwidth}{%
		\leftskip0.05\textwidth
		\vspace*{1cm}
		\parbox{0.8\textwidth}{%
		\ttfamily\footnotesize
		Volpato, Álvaro Augusto

\hspace{-7mm} V472d

		\begingroup\addtolength{\parindent}{2em}

			\thetitle\hphantom{ } / \theauthor; advisor Luís Fernando Costa Alberto - São Carlos, 2025.

			\thelastpage\hphantom{} p.\\

			Doctoral Thesis - Graduate Program in Electrical Engineering and Concentration Area in Electrical Power Systems - São Carlos School of Engineering, University of São Paulo - Brazil, 2025.

			1. Dynamic Phasors. 2. Nonstationary power. 3. Electrical Power Systems. 4. Linear Systems. 5. Perturbed Systems. I. Alberto, Luís Fernando Costa, advisor. II. Título.

		\endgroup
		\vspace*{1cm}
    }%
}
}
\vspace{5mm}

Responsible for the cataloguing structure of the publication according to AACR2: EESC librarians
\end{center}
\newpage

\begin{judgepage}\end{judgepage}

% -----------------------------------------------
% ABSTRACTS {{{1
% -----------------------------------------------
%\pagestyle{plain}
	
% (4.1) IN ENGLISH
\begin{newabstract}{Abstract}
	There already exists a myriad body of literature about the construction of a theory on the equivalence between nonstationary sinusoids (signals with sinusoidal ``shape'' but time-varying amplitude, frequency and phase) and time-varying complex functions, called Dynamic Phasors, with the intent of expanding the theory of static or Classical Phasors to a wider class of signals that can model sophisticated transient phenomena in differential equations — particularly those modelling electrical circuits. Because Dynamic Phasors are essentially a tool for solving time differential equations, such literature spans a large variety of subjects, including engineering, mathematics and applied sciences. Electrical engineers have been prolific to propose many such theories, aiming to model circuits, systems and controllers in a phasorial domain with the objective of constructing a mathematically solid, yet useful, way to approach modelling systems under nonstationary regimens. Power Systems engineers have been especially drawn towards such theories due to the recent penetration of power electronics and renewable energy resources into electrical grids worldwide, because such devices void the modelling hypotheses used to describe the circuit networks that make transmission systems.

	Despite many tools currently available, none have been able to capture all of the qualities that made Classical Phasors so useful: while some require truncations and approximations to be operationalizable — thus sacrificing precision for modelling convenience — some others can represent and reconstruct some signals of interest but do not offer a good theory of complex power under nonstationary regimens. This thesis comprises an endeavour of building a novel Dynamic Phasor Theory, specifically the construction of a theory that offers a representation of nonstationary sinusoids as Dynamic Phasors, as well as reconstructing those signals in time from their phasorial counterpars without the need for approximations or truncations while offering a theory of complex electrical power in nonstationary regimens, justifying the synchrophasor representation of voltages and currents and the steady-state approximation of complex power.

	The theory proposed is explored to prove a long-standing problem in the Electrical Engineering, known as the Quasi-Static Modelling or Hypothesis (QSM/QSH), by proving that if a specific circuit is excited by sinusoids which frequency varies slowly and little, then the behavior of the excited circuit can be approximated by its steady-state solution, justifying a great many results in Electrical Power Systems literature where the QSH is adopted as an underlying \textit{sine qua non} modelling pressuposition. Further, the theory proposed is also shown to be easily deployable in modelling electrical circuits because the proposed DPT transform is able to translate differentiation operators in time to specific functionals in the space of complex functions which form very powerful algebraic structures, such that the modelling in Dynamic Phasor space is greatly ameliorated. Such algebraic nature is explored in two ways: firstly, from an electrical circuits standpoint, these functionals are able to originate notions of impedances in the Dynamic Phasor space, thus leading to proofs of circuit modelling techniques (the Superposition Theorem and the Thèvenin-Norton Theorems). Second, from a signals and systems standpoint, the functionals are also able to bear a theory of linear elementary control in the Dynamic Phasor space, and a new transformation that highly resembles the Laplace transformation is proposed, such that the notions of BIBO stability and Transfer Functions are possible. Using this new theory and transform, controllers for Power Systems in nonstationary regimens based on small-signal analyses are justified. A new current setpoint controller for inverter-based systems is proposed using the proportional-integral equivalent controller in these new Transfer Functions, and shown to be a better candidate to some controllers currently used for the same purposes.

\noindent
\textbf{Keywords}: dynamic phasors, electrical power systems, nonstationary power, linear systems, control theory.
\end{newabstract}

%% (4.2) AUS DEUTSCH
%\begin{newabstract}{Zusammenfassung}
%Zusammenfassung aus Deutsch. \\
%
%\noindent
%\textbf{Stichwörter}: Stichwort 1, Stichwort 2,...
%\end{newabstract}
%\newpage

% (4.3) EM PORTUGUÊS
\begin{newabstract}{Resumo}
	Já existe um considerável corpo de literatura buscando a construção de uma teoria sobre a equivalência entre senoides não estacionárias (sinais com um ``formato senoidal'' mas com amplitude, frequência e fase variantes no tempo) e funções complexas do tempo, chamados Fasores Dinâmicos, com o intuito de expandir a teoria de Fasores Clássicos para uma classe maior de sinais que podem modelar fenômenos transientes sofisticados em equações diferenciais — particularmente aquelas que modelam circuitos elétricos. Como Fasores Dinâmicos são essencialmente uma ferramenta de solução de Equações Diferenciais, aquela literature caminha entre uma variedade de disciplinas como engenharia, matemática e ciências aplicadas. Engenheiros eletricistas têm sido prolíficos em propor tais teorias, procurando modelar ciruitos, sistemas e controladores no domínio fasorial com o objetivo de construir uma forma matematicamente sólida, mas ainda prática, de modelar sistemas em regimes não estacionários. Engenheiros de Sistemas de Potência, em especial, têm se preocupado com tais teorias devido ao crescente emprego de dispositivos baseados em eletrônica de potência e fontes de energias renováveis em redes elétricas no mundo todo, porque estes dispositivos violam as hipóteses comumente utilizadas para modelar os circuitos que formam os sistemas de transmissão.

	A despeito da propositura de muitas ferramentas, nenhuma delas é capaz de capturar todas as qualidades que fazem Fasores Clássicos tão úteis. Algumas requerem truncamentos e aproximações para serem operacionalizáveis, enquanto outras de fato representam e reconstróem alguns sinais de interesse mas não oferecem uma teoria de potência complexa sob regimes não-estacionários. Esta tese consiste da tarefa de construir uma nova Teoria de Fasores Dinâmicos, especificamente a construção de uma teoria que dispõe de uma representação de senoides não-estacionárias como Fasores Dinâmicos, bem como a reconstrução dos sinais no domínio do tempo a partir dos seus fasores, sem aproximações ou truncamentos e também dispõe de uma teoria de potência elétrica complexa em regimes não estacionários, justificando a representação de tensões e correntes como sincrofasores e a aproximação quase-estática de potência complexa.

	A teoria proposta é explorada para provar um problema de longa data na literatura de Engenharia Elétrica, conhecida como a Hipótese ou Modelagem Quase-Estática (QSH), ao provar que um circuito elétrico excitado por senoides cuja frequência varia pouco e varia lentamente pode ter seu comportamento aproximado pela sua solução de regime estacionário, justificando muitos resultados na literatura de Sistemas Elétricos de Potência que adotam a QSH como um pressuposto fundamental de modelagem. Ademais, mostra-se que a teoria proposta é facilmente utilizável na modelagem de circuitos elétricos porque a transformada DPT proposta traduz a operação de diferenciação no domínio do tempo em certos funcionais no espaço de funções complexas através de funcionais no espaço de sinais complexos. Estes funcionais formam poderosas estruturas algébricas, de forma que a modelagem no espaço de Fasores Dinâmicos é facilitada substancialmente por consistir de manipulações algébricas. Este fato é explorado de duas formas: do ponto de vista de circuitos elétricos, estes funcionais originam noções de impedâncias no espaço de Fasores Dinâmicos — levando a provas de técnicas de modelagem de circuitos como o Teorema da Dualidade, Princípio da Superposição e os Teoremas de Thèvenin e Norton. Depois, de um ponto de vista de sinais e sistemas, os funcionais também são capazes de dar luz a uma teoria de controle elementar no espaço de Fasores Dinâmicos, e uma nova transformada símil à Transformada de Laplace é proposta, tal que as noções de estabilidade entrada-saída e de Funções de Transferência são possíveis. Utilizando esta nova teoria, controladores de sistemas de potência baseados em análise de pequenos sinais em regimes não-estacionários são justificados; um novo controlador de referência de corrente para sistemas com inversores é proposto utilizando o equivalente p[roporcional-integral desta nova transformada, e mostra-se que o controlador proposto é melhor que aquele hoje utilizado na literatura.

\noindent
\textbf{Palavras-chaves}: fasores dinâmicos, sistemas elétricos de potência, potência não-estacionária, sistemas lineares, teoria de controle.
\end{newabstract}

% -----------------------------------------------

\begin{acknowledgements}

	To my advisor professor Luís, who shaped me as a professional and a researcher and serves as a role model. I could say Prof. Luís is an excellent engineer, a one-of-a-kind teacher, and a brilliant researcher. I am sure he would nevertheless prefer to be remembered as an inspiring mentor. Ironically, by trying to be like Luís I ended up figuring out how to be myself; his greatest lesson to me is not on Analysis, Differential Equations or Engineering but the gracefully simple yet humanely complex ability to be at peace with myself.

	To my parents who, beyond the deep and inextricable relationship of blood, serve as idols and templates. I feel I am the imperfect casting of perfect moulds however, as I was not able to replicate their humane excellency, even though I inherited it.
	
	To Professor Federico Bizzarri, from the Dipartimento di Elettronica, Informazione e Bioingegneria (DEIB) of the Politecnico di Milano, who extended me the opportunity to spend months with him in Milano developing research, going as far as offering financial support. Through Prof. Bizzarri I met Professor Angelo Brambilla, who showed me I was capable of more I thought I was, and also Davide del Giudice, who showed me self-worth is a quiet and unnanouced virtue. A great many thanks also go to the Politecnico di Milano and DEIB for offering me their great infrastructure, including the laboratory and computational servers. In the months I was at PoliMi I poured the entirety of my time and mind into that opportunity. My time in Milano was absolutely blissful.

	To the the funding agencies that made my academic formation possible. The São Paulo Research Foundation FAPESP, who maintains a lot of the infrastructure I worked with, such as the Laboratory of Computational Analysis of Electric Power Systems; the Coordination for the Improvement of Higher Education Personnel CAPES which supplied me with the excellency PROEX scholarship; and the National Council for Scientific and Technological Development (CNPq) who provided with travel funding for conferences. The Institute of Electrical and Electronic Engineers, IEEE, has also gracefully helped by offering student travel grants not once but twice. Finally, Santander for the finantial support, due to their Santander Mobility Program, to my international journey at the end of my doctorate.

	To the University of São Paulo (USP) and the Department of Electrical and Computer Engineering of the São Carlos School of Engineering (SEL - EESC) wherein I was able to complete both my bachelor's degree, my master's and this doctorate thesis with the utmost excellency and infrastructure. The staff and faculty at SEL has gone out of their ways, time and again, to help me, and I am grateful for all these years.

	To the custom mechanical keyboard community, particularly the brazilian community, which were a constant in my life during graduation, master's and doctorate. Especially to Felipe ``MrKeebs'' Coury who is in essence responsible for kick-starting my projects. To the members of the Advanced Input Research Institute. To Númenórien and Khazad-Durin, who are and will be, deeply missed by this Gondolindrim.

\end{acknowledgements}

% LISTS OF FIGURES, TABLES, SYMBOLS AND ACRONYMS {{{1
% -----------------------------------------------

% LIST OF FIGURES
\listoffigures
\cleardoublepage

% LIST OF TABLES
\listoftables
\cleardoublepage

% -----------------------------------------------
% LIST OF ABBREVIATIONS / ACRONYMS
% -----------------------------------------------
% This piece of code centralizes the LoA and LoS names, putting them in a HUGE and boldfaced font.
% I was not able to insert this inside a \newenvironment, because this snipped messes with the
% @makeschapter macro, which parameter #1 conflicts with the parameter number definition of the
% newenvironment giving a "Illegal number definition" error. 
\begingroup
	\makeatletter
		\def\chapter{\cleardoublepage\secdef\@chapter\@schapter}
		\def\@makeschapterhead#1{{\center\HUGE\sffamily\bfseries #1\par\nobreak\vskip 10\p@\vspace*{5mm} }}
	\makeatother

\begin{acronyms}
	\acronym{OMIB}{One Machine Infinite Bus System}
	\acronym{SM}{Synchronous Machine}
	\acronym{QSH}{Quasi-Static Hypothesis}
	\acronym{QSM}{Quasi-Static Modelling}
	\acronym{PLC}{Passive Linear Circuit}
	\acronym{DQ}{Direct-Quadrature}
	\acronym{$dq0$}{Direct-Quadrature-Zero transform (synonym with Clarke-Park Transform)}
	\acronym{3$\phi$}{Three-phase}
	\acronym{SPO}{Static Phasor Operator}
	\acronym{PE}{Phasor Equivalent}
	\acronym{EMT}{Electromagnetic Transient}
	\acronym{EPS(s)}{Electric Power System(s)}
	\acronym{PLL}{Phase Locked Loop}
	\acronym{LTI}{Linear Time Invariant}
	\acronym{(O)DE(s)}{(Ordinary) Differential Equation(s)}
	\acronym{DP(s)}{Dynamic Phasor(s)}
	\acronym{STFT}{Short-Time Fourier Transform}
	\acronym{LT}{Laplace Transform}
	\acronym{HT}{Hilbert Transform}
	\acronym{DPT}{Dynamic Phasor Transform}
	\acronym{DPF(s)}{Dynamic Phasor Functional(s)}
	\acronym{ZES}{Zero Energy Start}
	\acronym{$\mu$T}{Mu Transform}
	\acronym{$\mu$TF(s)}{Mu Transfer Function(s)}
	\acronym{BIBO}{Bounded Input Bounded Output}
\end{acronyms}
\endgroup

% -----------------------------------------------
% LIST OF SYMBOLS
% -----------------------------------------------
% Works pretty much the same as the list of abbreviations. In most graduate programs
% and university departments this LoS is not required.

%\begin{listofsymbols}
%	\item[$ \Gamma $] Letra grega Gama
%	\item[$ \Lambda $] Lambda
%	\item[$ \zeta $] Letra grega minúscula zeta
%	\item[$ \in $] Pertence
%	\item[$|\cdot|$] Complex absolute value
%	\item[$\lVert \cdot \rVert$] Complex vector or matrix euclidian norm
%\end{listofsymbols}
%
%\endgroup

% -----------------------------------------------
% TABLE OF CONTENTS {{{1
% -----------------------------------------------

\tableofcontents*
\cleardoublepage

% Redefine plain style
\pagestyle{fancy}

% -----------------------------------------------
% EPIGRAPH {{{1
% -----------------------------------------------

\begin{newepigraph}
``Iron cares not for faith or heresy. Iron is forever. From iron cometh strength. From strength cometh will. From will cometh faith. From faith cometh honor. From honor cometh iron.''

\hfill --- The Litany of Iron by Dantioch
\end{newepigraph}

%% -----------------------------------------------
% TEXT BODY {{{1
% -----------------------------------------------

% Begin page number at one and start arabic page numbering
\begintextbody

% ---------------------------------------------------------
\chapter{Introduction}\label{chapter:intro}
% ---------------------------------------------------------

	\lettrine{C}{lassical} Phasor Theory was first developed by Charles Steinmetz and published in his crucial paper \textit{Complex Quantities and their use in Electrical Engineering} \pcite{Steinmetz1893} to simplify the analysis of alternate current (AC) circuits. In this theory, Steinmetz proposes a functional operator that takes sinewaves and represents them as complex numbers. The first benefit of such operator is that, due to the simple algebra of complex numbers, operations on the space of sinewaves are reduced to operations in the complex space — much simpler and intuitive due to their geometric nature. The second benefit of the operator is that it transforms the time differential equations (DEs) of a circuit into complex algebraic equations and, due to this algebraic nature rather than differential, these new equations are much easier to solve and operate than sinewaves and differential equations; yet, the solutions of the algebraic complex equations are guaranteed to be direct representations of the steady-state solutions of the original DEs of the circuit without approximations or truncation — rendering an effective way to solve time DEs in the phasor or ``frequency'' domain. In his seminal book \textit{Theory and Calculation of Alternating Current Phenomena} \pcite{steinmetzTheoryCalculationAlternating1897}, Steinmetz proceeds to define ``true'' and ``wattless'' powers, modernly known as \textit{active} and \textit{reactive} power: complex representations of the portions of instantaneous power respectively in phase and in quadrature with voltage.

	The transformation of differential equations into algebraic ones is a major feature of the phasorial operator. The ease of operations means that circuit equations can be easily manipulated; parametric analysis of circuits, as well as their frequency response and transient behavior are greatly ameliorated. Further, several results from DC (Direct Current) regimens circuits can be imported to circuits in AC (Alternate Current) regimens — such as Kirchoff's Laws, the Superposition Theorem, the Thèvenin and Norton Theorems, further enhancing the circuit analysis theory of circuits under alternate regimens. As a consequence of Steinmetz' work and the further refinement of the theory that followed, phasor analysis of AC circuits has become a cornerstone of Electrical Engineering, permeating the whole field by becoming part of elementary circuit theory and a matter of introductory books and courses \pcite{scottElementsLinearCircuits1965,desoerBasicCircuitTheory1987}.

	Ellegant as it is, however, Classical Phasor Theory has increasingly become unsuitable for modern circuits and systems due to the growing need for the modelling of transient phenomena in \textit{nonstationary sinusoidal regimens} where amplitudes, frequencies and phases vary. Reestated, the signals and excitations involved have a certain ``sinusoidal shape'' but time-varing parameters; such new excitation signals clearly do not adhere to the Classical Phasor Theory which only embraces \textit{static sinusoidal regimens}. This prompts for the enhancement of classical phasor theory to a wider class of signals called \textbf{nonstationary sinusoids}. This problem is not a new one in Electrical Engineering literature, for it preconizes several instances where treating such signals with rigour is needed, yet no solid theory exists. For this matter, through the decades works have coined many terms like ``nonstationary regimens'', ``varying sinusoids'', ``functions shaped like sinewaves'', aiming to offer some formalization.

	Fundamentally, building a theory of time-varying phasors is a matter of Time-Frequency Analysis (TFA), an area of mathematics dealing with representing operators and signals in frequency domain \pcite{Gabor1946TheoryOC,grochenigFoundationsTimeFrequencyAnalysis2001}. But because spectral analysis is a major field of application for several subfields of Electrical Engineering like signal processing, circuit theory, and systems modelling, engineers have been notorious for proposing many such theories — especially Electrical Power Systems (EPSs) researchers. While the first works arose in the 1940s throughout the 1990s (see \cite{Venkatasubramanian1995a,Venkatasubramanian1995}), there is a modern insurgence of theories due to the fast penetration of distributed generators across electrical grids worldwide. Over the past decades \pcite{Morsi2009,Henschel1999,rupasingheAssessmentDynamicPhasor2021,stankovicDynamicPhasorsModeling2002}, electrical engineers and researchers have proposed a multitude of approaches aiming to expand the notion of classical phasors (also called \textit{static phasors} since they are defined as constant amplitude, phase and frequency) to time-varying complex functions, called \textit{dynamic phasors} (DPs) by extension, with the aim of representing nonstationary sinusoidal signals in a phasorial manner in order to accurately model circuit networks manifesting transient phenomena and behaviors more sophisticated than simple stationary sinewaves. DPs find a myriad of applications in electrical engineering, ranging from power systems stability and control to power electronics, signal modulation \pcite{rupasingheAssessmentDynamicPhasor2021}, telecommunications \pcite{stankovicDynamicPhasorsModeling2002} and information theory \pcite{Gabor1946TheoryOC}.

	Notwithstanding the many strides that have been made in the literature, the present frameworks are incomplete in the sense that they are unable to mirror the exact qualities that made static phasors such a paramount tool, and a ``complete'' theory is still sorely lacking. This thesis aims to offer such a formal theory, based on motivations and results from the literature of EPSs but has a potential for a wide application in various sub-areas of Electrical Engineering. The theory proposed is built using classical phasors theory as a template, and is shown to be an expansion of that classical theory in that is proven that classical phasors are a particular version of the dynamic phasors proposed. Further, the proposed theory opens up a wide array of theoretical results in Linear Circuit Theory, which greatly enhance our current understanding of linear circuits and considerably widen the reach of phasors by solving the problem of proving the Quasi-Static Modelling or Hypothesis, defining impedances, proving circuit modelling theorems, and offering an elementary control theory for linear systems in nonstationary sinusoidal regimens.

% ---------------------------------------------------------
\section{Motivation: the Quasi-Static Modelling}\label{subsec:intro_motivation} %<<<1
% ---------------------------------------------------------

	The so-called ``classical'' Power Systems literature is comprised of the analysis, control and simulation of large distribution systems powered by electrical machines, most commonly synchronous generators. Traditional dynamical models of Power Systems uses Phasor Equivalent (PE) models: instead of simulating Power Systems using differential equations of the voltages and currents in the time domain (called ElectroMagnetic Transient simulation or EMT), the system is transformed from the time domain to a phasorial domain where the quantities obtained are phasors that purportedly represent signals in time domain with a certain degree of accuracy.
	
	There are many advantages from a phasorial representation: first, the obviety of having quantities represented in terms of magnitudes and phases, which beget the notions of inductive or capacitive or resistive loads, as well as active, reactive and complex power — trivial and seminal concepts in Electrical Engineering. In some cases, phasorial representation is not only preferred but required: for instance, most Power System controllers control active and reactive power or power factor, eminently phasorial concepts. Second, while the time domain quantities vary (almost) sinusoidally at (or close to) the synchronous frequency (50 or 60Hz), the phasor quantities in general vary from fractions to the units of hertz, meaning that simulating the system in the phasorial domain is simpler and faster from a numerical standpoint seen as the numerical solvers can adopt larger timesteps due to the slower varying signals.

% EXAMPLE: MACHINE MODEL <<<
\begin{figure}
\centering
        \begin{tikzpicture}[american,scale=1.2,transform shape,line width=0.75, cute inductors,>={Stealth[inset=0mm,length=1.5mm,angle'=50]}]
		\draw (-2,-0.5) [vsourcesin,sources/scale=1.2,name=vin] to (-2,0.5);
		\draw (-1.5,0) to[short,current/distance=0.3,i=$I(t)$] (1,0);
		\draw[line width=1mm] (0,-1)--(0,1); % V voltage
		\node (vtlabel) at (0,1.3) {$V(t)$};
		\draw[->] ([shift=({-1.1,0.3})]vin.90) -- +(1,0) node [midway,above] {$E_{FD}$};
		\draw[->] ([shift=({-1.1,-0.3})]vin.90) -- +(1,0) node [midway,below] {$P_{m}$};
		\node [draw, minimum width=25mm, very thick, minimum height=15mm, right=1, text width=20mm, align=center] (tab_block) {Transmission system};
        \end{tikzpicture}
	\caption{Schematic of synchronous machine model \eqref{eq:machine_2a_model}.}
	\label{fig:machine_model}
\end{figure} %>>>

	Conceptually however, there is no guarantee that the phasorial quantities obtained from the Phasor Equivalent models are indeed verosimile to the time-domain quantities they represent. To make sure that phasorial quantities reconstruct the time domain signals of electrical systems, there are several hypotheses and simplifications assumed. The umbrella term of these simplifying hypotheses is called the \textbf{Quasi-Static Modelling or Hypothesis}, abbreviated QSM or QSH. The name stems from the crucial notion that, in order to obtain phasor-equivalent models of the electromagnetic transient models, one assumes that the frequency disturbances are slow and small — meaning that, albeit time-varying, the sinusoidal signals involved are ``almost static''.

% ---------------------------------------------------------
\subsection{Synchronous machine modelling}\label{subsec:synchmachine_modelling} %<<<2
% ---------------------------------------------------------

	The inception of the QSM starts at the modelling of the synchronous machine that power classical Power Systems. For instance, \cite{Ramos2000} is entirely dedicated to developing such phasorial models for Synchronous Machines; the book first starts with a time-domain (EMT) modelling that does not suppose any particular frequency signal, and uses currents and flux linkages to model the machine behavior. Once the EMT model is complete, several simplifications are made: first that the system frequency $\omega(t)$ is close to the synchronous frequency $\omega_0$ (50 or 60Hz), and that the rotor electro-rotational dynamics are much slower than the electrical dynamics of the stator, allowing disconsidering the stator transients and supposing it is in a permanent sinusoidal state. Following this, several simplifications follow and the machine is described as a phasor-equivalent dynamical, algebraic-differential model \eqref{eq:machine_2a_model} known as the ``two-axis'' model.

% MACHINE 2A MODEL <<<
\begin{equation}
	\left\{\begin{array}{l}
		\dot{E}_d = \dfrac{E_{FD} - E'_d + \left(x_q - x'_q\right)I_q}{\tau'_{q0}} \\[5mm]
		\dot{E}_q = -\dfrac{E'_q + \left(x_d - x'_d\right)I_d}{\tau'_{d0}} \\[5mm]
		\dot{\omega} = \dfrac{P_m - P_e - D\omega}{2H} \\[5mm]
		\dot{\delta} = \omega \\[5mm]
		P_e = E'_dI_d + E'_qI_q + \left(x'_q - x'_d\right)I_dI_q \\[5mm]
		V_d = E'_d - rI_d + x'_qI_q \\[5mm]
		V_q = E'_q - rI_q - x'_dI_d 
	\end{array}\right. \label{eq:machine_2a_model}
\end{equation} %>>>

	In \eqref{eq:machine_2a_model}, all quantities are noted in a per-unit measurement system. $P_m$ is the mechanical power applied to the machine shaft coming from a governor, $P_e$ active the electrical power developed by the stator, and $E_{FD}$ is a voltage supplied to the field coil by a field actuator like an AVR+PSS pair. Hence these are input quantities supplied by external devices which are generally subject to other control mechanism. $E(t)$ is the internal induced voltage in the stator, $I(t)$ the stator current supplied to the bus and $V(t)$ the terminal voltage of the machine at the point of connection. A schematic is shown in figure \ref{fig:machine_model}.

	In terms of behaviors, equations \eqref{eq:machine_2a_model} are essentially separated in three parts. The first two equations describe the ``electrical'' portion, and model the behavior of the internal voltage $E = E_d + jE_q$ induced on the stator by the  rotor across which coils is applied field voltage $E_{FD}$, generating a rotating magnetic field that interacts with the stator current $I = I_d + jI_d$. As a matter of fact, \eqref{eq:machine_2a_model} is known as ``two-axis'' due to the fact it models both $E_d$ and $E_q$.

	 The two middle equations for $\dot{\omega}$ and $\dot{\delta}$ are the electromechanical portion of the model and describe the machine rotor angle with respect to the synchronous reference; the angular frequency $\omega$ is governed by the \textit{swing equation}, which defines that the variation in frequency is given by Newton's Second Law in a rotational frame where the accelerating torque, coming from the mechanical power supplied at the shaft from a governor, is counteracted by the electrical torque $P_e$ generated by the stator current interacting with the field coil magnetic rotating field, and also a damping-friction coefficient $D$ (small enough that it is most of the times ignored) resulting from mechanical losses on the shaft like the air drag on the rotor and mechanical friction on bushings, sockets and bearings.

	Again, supposing small frequency swings, two approximations are made. First, that the electrical power $P_e$ is equivalent to the active power developed at the stator; however, there is no equivalent definition of a ``time-varying active power''. If the frequency swings are small, then the active power definition is taken as a time-varying equivalent of the static classical power $P = \left\lvert V\right\rvert\left\lvert I\right\rvert\cos\left(\phi_v - \phi_i\right)$, equalling the expression noted in \eqref{eq:machine_2a_model}.

	Second, since rotational power is torque times angular frequency, the mechanical power on the shaft is given by $P_m = \tau_m\omega$ and, since $\omega\approx 1$ in a per-unit measurement system where the synchronous frequency $\omega_0$ is the single unit, then $P_m \approx \tau_m$ — the benefit being that mechanical power is easier to measure and model than torque. The same approximation is used with the counter-accelerating electrical torque, also a modelling benefit because the electrical power is simple to calculate in terms of the voltages and currents. Finally, The last two algebraic equations describe the machine terminal voltage $V = V_d + jV_q$ as a function of the iternaln induced voltage $E$ and the stator current $I$.

	The machine model \eqref{eq:machine_2a_model} can be further approximated, leading to the better-known ``classical'' version. First, one supposes that the transient disturbances are too quick for the field and governor controllers to act upon these quantities in the same timescale as the disturbances, thus $P_m$ and $E_{FD}$ are kept constant and generally obtained from the equilibrium equations. Also, one supposes that the induced voltage $E(t)$ is such that its amplitude is constant, and that the time-domain signal $e(t)$ is a sinusoid at the constant synchronous frequency but time-varying angle $\delta$, as in

\begin{equation} e(t) = \left\lvert E\right\rvert\cos\left(\omega_0 t + \delta(t)\right) \label{eq:equivalent_emt_E} \end{equation}

	\noindent which is known as a \textit{synchrophasor}, as defined in the IEEE Standard C37.118.1-2011 for Synchrophasor Measurements for Power Systems \pcite{IEEEStandardSynchrophasor2011}. Further, one imagines that the machine used has a smooth rotor construction (as opposed to salient rotor), entailing to identical synchronous impedances $x_d = x_q = x$ and transient impedances $x'_d = x'_q = x'$. Hence the machine is approximated for a model of a phasor voltage $E(t)$ with constant amplitude and constant frequency but time-varying phase behind an impedance $r + jx'$, eliminating the differential equations for $E(t)$. Additionally it is assumed the mechanical losses on the shaft are negligible, yielding $D = 0$ leading to \eqref{eq:machine_2a_model_classical}, which is the known ``classical model'' of synchronous machines, with a schematization in figure \ref{fig:machine_model_classic}. In short, this model supposes that the induced voltage $E(t)$ has constant amplitude and frequency, and the electromechanical frequency swings are accumulated in the time-varying phase $\delta$.

% MACHINE CLASSICAL MODEL <<<
\begin{equation}
	\left\{\begin{array}{l}
		\dot{\omega} = \dfrac{P_m - P_e}{2H} \\[5mm]
		\dot{\delta} = \omega \\[5mm]
		P_e = E'_dI_d + E'_qI_q \\[5mm]
		V_d = E'_d - rI_d + x'I_q \\[5mm]
		V_q = E'_q - rI_q - x'I_d 
	\end{array}\right. \label{eq:machine_2a_model_classical}
\end{equation} %>>>

% EXAMPLE: MACHINE MODEL <<<
\begin{figure}[h]
\centering
        \begin{tikzpicture}[american,scale=1.2,transform shape,line width=0.75, cute inductors,>={Stealth[inset=0mm,length=1.5mm,angle'=50]}]
		\draw (0,0.5) [vsourcesin,sources/scale=1.2,name=vin] to(0,-0.5);
		\draw (vin.90) to[short,f=$I(t)$] ++(2,0) node(estart) {} to[short] ++(0.5,0) to[R,l=$r$] ++(1,0) to [L,l=$x'$] ++(2,0) node (vstart) {} to [short] ++(0.5,0);
		\draw[line width=1mm] ([shift=({0,0.5})]estart.center) -- ++(0,-1); % V voltage
		\node (elabel) at ([shift=({0,0.75})]estart.center) {$E(t)$};
		\node (elabel) at ([shift=({0,1.1})]vin.center) {$E(t) = \left\lvert E\right\rvert e^{j\delta(t)}$};
%
		\draw[line width=1mm] ([shift=({0,0.5})]vstart.center) -- ++(0,-1); % V voltage
		\node (vlabel) at ([shift=({0,0.75})]vstart.center) {$V(t)$};
%
		\draw[->] ([shift=({-1.1,0.3})]vin.270) node [left] {$E_{FD}$} -- +(1,0) ;
		\draw[->] ([shift=({-1.1,-0.3})]vin.270) node [left] {$P_{m}$} -- +(1,0);
        \end{tikzpicture}
	\caption{``Classical'' model approximation as per \eqref{eq:machine_2a_model_classical}.}
	\label{fig:machine_model_classic}
\end{figure} %>>>

	These equations define the open-loop synchronous machine. To achieve the model of a power system, one couples the machine to a transmission system by writing an expression for the current $I(t)$ as related to the terminal voltage $V(t)$. The simplest possible transmission system is the OMIB (One Machine Infinite Bus) depicted in Figure \ref{fig:example_omib}. This system supposes that the machine is attached to an orders-of-magnitude larger generation-transmission system — so overwhelmingly larger in fact that the particular machine under consideration has virtually no effect on it and the transmission system may be approximated by a constant voltage $V_\infty$ unwaivering to how much power that is required from, or injected into it, by the machine. Thus $V_\infty$ has constant amplitude $\left\lvert V_\infty\right\rvert$ and constant phase $\phi_\infty$, as shown in figure \ref{fig:example_omib}. The machine is attached to such Inifinte Bus through a transmission line or a particular resistive-inductive behavior. 

	If the frequency swings $\omega - \omega_0$ are kept to a minimal, the line behaves ``almost sinusoidally'', that is, as a constant impedance $R_L + jX_L$, yielding the voltage-current relationship

\begin{equation} V_\infty - V = I\left(R_L + jX_L\right) \label{eq:omib_line}\end{equation}

	\noindent thus resulting in a differential-algebraic model of the system without controllers. After the system is modelled and simulated, then it is assumed that the phasorial signals obtained from the simulations are directly related by the formula

\begin{equation} x(t) = \left\lvert X\right\rvert \cos\left(\omega_0 t + \phi_x(t)\right) \label{eq:equivalent_emt_X}\end{equation}

	\noindent where $x(t)$ denotes a synchrophasor signal in time being reconstructed, $X$ is the phasorial number obtained from simulation and $\phi_X(t)$ its argument, $\omega_0$ the synchronous frequency.

% EXAMPLE: OMIB <<<
\begin{figure}[h]
\centering
        \begin{tikzpicture}[american,scale=1.2,transform shape,line width=0.75, cute inductors,>={Stealth[inset=0mm,length=1.5mm,angle'=50]}]
		\ctikzset{sources/scale=1.2}
		\node [shape=vsourcesinshape, rotate=-90] (gen1) at (-2,0) {} ;
		\draw[->] ([shift=({-1.1,0.3})]gen1.south) -- +(1,0) node [midway,above] {$E_{FD}$};
		\draw[->] ([shift=({-1.1,-0.3})]gen1.south) -- +(1,0) node [midway,below] {$P_{m}$};

		\draw (gen1.north) to[short,current/distance=0.3,i=$I(t)$] ++(2,0) coordinate(terminalbar)
			to[L,l=$L$,-] ++(2,0) 
			to[R,l=$R$] ++(2,0) coordinate(vinfbar);

		\draw[line width=1mm] ([shift=({0,1})]terminalbar) -- ++(0,-2); % V voltage
		\node (vtlabel) at ([shift=({0,1.5})]terminalbar) {$V(t)$};

		\draw[line width=1mm] ([shift=({0,1})]vinfbar) -- ++(0,-2); % V voltage

		\node (vinfsource) [shape=vsourcesinshape, rotate=-90] at ([shift=({1,0})]vinfbar) {};
		\node[right] () at ([shift=({0.5,0})]vinfsource) {$V_\infty {=} \left\lvert V_\infty\right\rvert e^{j\phi_\infty}$};

		\draw (vinfbar.center) to (vinfsource.south);
	\end{tikzpicture}

  	\caption{One-Machine-Infinite-Bus System.}
	\label{fig:example_omib}
\end{figure} %>>>

	In general, the fact that the model equations \eqref{eq:machine_2a_model} and \eqref{eq:omib_line} suppose many modelling hypothesis leading to significant simplification is not discussed in the Power System literature — swept under the proverbial rug —  if even cited at all. Particularly for the heavily approximated classical model \eqref{eq:machine_2a_model_classical}, this fact is especially egregious due to the extensive approximations needed to achieve it, despite its wide usage in the literature. When being introduced to the literature, one (student or researcher, and certainly myself when I was introduced) cannot help but notice that the equation of the line \eqref{eq:omib_line} uses a phasorial approach given by $V = ZI$, which supposes eminently that the current and voltage signals involved are constant sinusoids of static amplitude, frequency and phase and yet the differential model of \eqref{eq:machine_2a_model} defines $E(t),V(t),I(t),\omega(t)$ as time-varying, placing a blatant contradition which is in part quenched by the supposition of small frequency swings. The very same literature also uses such models extensively in simulations of Power Systems, even under large disturbances — in contradiction with the hypotheses that made the models possible. 

	This contradiction is carried throughout the literature; in so far as the system of Figure \ref{fig:example_omib} is comprised of a single machine and the inertial approximation of a large power system (the infinite bus), it yields preliminary results and simplified analyses of the system. Nevertheless, it can already exhibit an abundance of transient behaviors — some of them of sophisticated nature, including chaos \pcite{chiangChaosSimplePower1993} and bifurcations \pcite{kwatnyLocalBifurcationPower1995}. In fact, my graduation thesis \cite{Volpato2017} deals with the specific task of finding criteria that lead the OMIB system to Hopf bifurcations — manifestly complex phenomena for such a simple system.

% ---------------------------------------------------------
\subsection{Large and multimachine Power Systems}\label{subsec:largemulti} %<<<2
% ---------------------------------------------------------

	The dissonance between the \textit{time-varying phasorial model} of electrical machines and the \textit{static phasorial model} of the transmission systems that they are attached to is even more prevalent in larger multimachine Electric Power Systems \pcite{darochaComputacaoAltoDesempenho2024}, like that of Figure \ref{fig:large_eps}. In such cases, each machine is modelled as an algebraic-differential system like \eqref{eq:machine_2a_model} or \eqref{eq:machine_2a_model_classical} and the transmission system is modelled as an \textbf{Admittance Matrix} $\mathbf{Y}$ such that the complex vector of terminal voltages and the complex vector of bus currents are related by

\begin{equation} \left[\begin{array}{c} V_1 \\[3mm] V_2 \\[3mm] V_3 \\[3mm]\vdots \\[3mm] V_n \end{array}\right] = \left[\begin{array}{ccccc} Y_{11} & Y_{12} & Y_{13} & \cdots & Y_{1n} \\[3mm] Y_{21} & Y_{22} & Y_{23} & \cdots & Y_{2n} \\[3mm] Y_{31} & Y_{32} & Y_{33} & \cdots & Y_{3n} \\[3mm] \vdots & \vdots & \vdots & \ddots & \vdots \\[3mm] Y_{n1} & Y_{n2} & Y_{n3} & \cdots & Y_{nn} \end{array}\right] \left[\begin{array}{c} I_1 \\[3mm] I_2 \\[3mm] I_3 \\[3mm] \vdots \\[3mm] I_n \end{array}\right] \label{eq:multimachine_admittance}\end{equation}

	\noindent and this accomplishes an algebraic-differential model of the system. Again, there is a contradiction that the transmission system is modelled as constant admittances while the machines are modelled as dynamic phasors. This contradiction is also justified under the assumption that the frequency swings are small in amplitude and slow in bandwidth; this means that the transmission grid is at an ``almost-static-sinusoidal'' state where the transients vanish quickly \pcite{azevedoMetodologiaFasorialPara2024}, yielding the set of complex algebraic equations \eqref{eq:multimachine_admittance}.

	For large-scale transmission systems, PE models present an immense benefit both in the methodologic and numerical aspects: if the system were modelled in an EMT framework, every capacitance and inductance element of the grid would be represented by a differential equation, and the grid model would become difficult to compute analytically but also prohivitively large to simulate; at the same time, using complex phasorial algebraic equations, a transmission system comprised of possibily thousands of nodes is reduced to a complex matrix of the same size as there are agents acting on the grid, greatly simplifying the time needed for modelling and the computational resources needed for simulation.

	In large and multimachine systems, power flow in the transmission lines is also a concern, even though as beforesaid the notions of active and reactive power in nonstationary regimens are not well-defined. Again supposing small and slow frequency swings, one uses the quasi-static modelling of the grid in \eqref{eq:multimachine_admittance} and adopts the active power $P$ and $Q$ as very close to their static or classical formulas, originating the expressions \eqref{eq:power_flow_eqs} called ``power flow equations'' describing active and reactive power transfer between two nodes $k$ and $m$:

\begin{equation} \left\{\begin{array}{l} P_{km} = V_kV_m \left[G_{km}\cos\left(\theta_k - \theta_m\right) + B_{km}\sin\left(\theta_k - \theta_m\right)\right]\\[3mm] Q_{km} = V_kV_m \left[G_{km}\sin\left(\theta_k - \theta_m\right) - B_{km}\cos\left(\theta_k - \theta_m\right)\right] \end{array}\right. \label{eq:power_flow_eqs},
\end{equation}

	\noindent where $V_ke^{j\theta_k}$ and $V_me^{j\theta_m}$ are the absolute values of the phasors of the node voltages, $B_{km}$ the susceptance between the two nodes and $G_{km}$ the conductance — both taken from the real and imaginary parts $\mathbf{Y} = \mathbf{G} +  j\mathbf{B}$, where $\mathbf{Y}$ is the admittance matrix of the ``static approximated'' transmission system \ref{eq:multimachine_admittance}. These formulas are generally simplified considering that the transmission lines have almost null resistive behavior, such that $G_{km}\approx 0$ yielding

\begin{equation} P_{km} \approx V_kV_m B_{km} \sin\left(\theta_k - \theta_m\right),\ Q_{km} \approx - V_kV_m B_{km} \cos\left(\theta_k - \theta_m\right). \label{eq:approx_power_flow_eqs}\end{equation}

	A thorough development of these power flow equations, their derivations and the simplifications involved can be found in \cite{Monticelli1999}.

% MODELLING EXAMPLE: LARGE POWER SYSTEM <<<
\begin{figure}[h]
\centering
        \begin{tikzpicture}[american,scale=1.2,transform shape,line width=0.75, cute inductors]
		\draw (-2,-0.5) [vsourcesin,sources/scale=1.2, l=$V_1(t)$] to (-2,0.5);
		\draw (-1.5,0) to[short,current/distance=0.3,i=$I_1(t)$] (0.5,0);
%

		\draw (-2,-2) [vsourcesin,sources/scale=1.2, l=$V_2(t)$] to (-2,-1);
		\draw (-1.5,-1.5) to[short,current/distance=0.3,i=$I_2(t)$] (0.5,-1.5);
%
		\foreach \n in {-0.6,-0.3,0}  \node at (-2,-3-\n)[circle,fill,inner sep=0.5pt]{};
%
		\draw (-2,-4.5) [vsourcesin,sources/scale=1.2, l=$V_n(t)$] to (-2,-3.5);
		\draw (-1.5,-4) to[short,current/distance=0.3,i=$I_n(t)$] (0.5,-4);
%
	\draw [draw=black,ultra thick] (0.5,1) rectangle (2.5,-5);
%
	\node at (1.5,-2) {$\mathbf{Y}$};
	\node at (1.5,-1.5) {Large EPS};
        \end{tikzpicture}
	\caption{Simplified ``black box'' representation of a large multimachine power system.}
	\label{fig:large_eps}
\end{figure} %>>>

%\newpage
\null
% EXAMPLE: DROOP, AVR + PSS <<<
\begin{figure}
\centering
\scalebox{0.75}{
        \begin{tikzpicture}[>={Stealth[inset=0mm,length=1.5mm,angle'=50]}, rotate=90, transform shape]
	\coordinate (origin) at (0,0);

	\node (machineblock) [draw, minimum width=20mm, very thick, minimum height=40mm] at (origin)  {};
	\node (machinelabel) [above=1mm of machineblock.north] {Machine};

	\node (efdinput) [below=12mm of machineblock.west] {};
	\node[right] (efdlabel) at (efdinput) {$E_{FD}$};

	\node (pminput) [above=12mm of machineblock.west] {};
	\node[right] (pmlabel) at (pminput) {$P_m$};

	\node (eout) [above=12mm of machineblock.east] {};
	\node[left] (elabel) at (eout) {$E'$};
	\draw[->] (elabel) -- ++(120mm,0);

	\node (omegaout) [above=3mm of machineblock.east] {};
	\node[left] (omegalabel) at (omegaout) {$\omega$};
	\draw[->] (omegalabel) -- ++(120mm,0);

	\node (deltaout) [below=3mm of machineblock.east] {};
	\node[left] (deltalabel) at (deltaout) {$\delta$};
	\draw[->] (deltalabel) -- ++(120mm,0);

	\node (vout) [below=12mm of machineblock.east] {};
	\node[left] (vlabel) at (vout) {$V$};
	\draw[->] (vlabel) -- ++(120mm,0);

	% PSS BLOCK
	\node (washout) [below=20mm of machineblock.south, draw, stewartyellow, minimum width=25mm, very thick, minimum height=15mm] {$\dfrac{sT_w}{sT_w + 1}$};

	\node at (washout) [minimum width=105mm, minimum height=3cm] (pssrounded) {};

	\node (psslabel) [above=0mm of pssrounded, stewartyellow] {\Large PSS};

	\node (delayadvance) [left=10mm of washout, draw, stewartyellow, minimum width=25mm, very thick, minimum height=15mm] {$\dfrac{sT_1 + 1}{sT_2 + 1}$};

	\node (pssinputgain) [draw, very thick, isosceles triangle, shape border rotate=180, stewartyellow, minimum height=15mm, minimum width=15mm, right=15mm of washout] {$k_{PSS}$};

	\draw[->,stewartyellow] (pssinputgain.apex) -- ([shift=({1mm,0})]washout.east);

	\draw[->,stewartyellow] (washout.west) -- ([shift=({1mm,0})]delayadvance.east);

	\draw[<-,stewartyellow,preaction={draw,white,line width=4pt}] ([shift=({1mm,0})]pssinputgain.east) -- ++(10mm,0) coordinate (omegabreak) -- (omegabreak |- omegaout);

	\draw[stewartyellow,fill] (omegabreak |- omegaout) circle (0.75mm) ;

	\node at (washout) [draw, rounded corners, stewartyellow, fill=stewartyellow, fill opacity=0.2, very thick, dashed, line cap = round, minimum width=105mm, minimum height=3cm] (pssrounded) {};

	% AVR BLOCK
	\node (avrfilter) [below=30mm of delayadvance, draw, stewartblue, minimum width=25mm, very thick, minimum height=15mm] {$\dfrac{K_e}{sT_e + 1}$};

	\coordinate (avrcircle) at ([shift=({20mm,0})]avrfilter.east);
	\node[stewartblue] at ([shift=({-3mm, 7mm})]avrcircle.north) {$-$};
	\node[stewartpurple] at ([shift=({7mm, 3mm})]avrcircle.east) {$+$};

	\draw[stewartblue, very thick] (avrcircle)  circle (5mm);

	\node at ($(avrcircle)!0.63!(avrfilter)$) [draw, rounded corners, stewartblue, fill=stewartblue, fill opacity=0.2, very thick, dashed, line cap = round, minimum width=60mm, minimum height=3cm] (avrrounded) {};
	\node[stewartblue, below=2mm of avrrounded.south] {\Large AVR};

	\draw[<-,stewartblue] ([shift=({0,6mm})]avrcircle.north) -- ++(0,16mm) -- ++(80mm,0) coordinate(avrbreak)-- (avrbreak |- vout);

	\draw[stewartblue,fill] (avrbreak |- vout) circle (0.75mm) ;

	\draw[->, stewartblue] ([shift=({-5mm,0mm})]avrcircle.west) -- ([shift=({1mm,0mm})]avrfilter.east) ;

	\node[left=30mm of delayadvance.west] (efdcircle) {};
	\draw[very thick] (efdcircle)  circle (5mm);

	\draw[->] ([shift=({0mm,3.8mm})]efdcircle.north) |- ([shift=({-1mm,0mm})] efdinput) ;
	
	\draw[->, stewartyellow] ([shift=({-0mm,0})]delayadvance.west) -- ([shift=({5mm,0mm})] efdcircle.east) node[above, near end] {$V_{PSS}$} ;

	\draw[->, stewartblue] ([shift=({-0mm,0})]avrfilter.west) -| ([shift=({0mm,-5mm})] efdcircle.south) node[left,near end] {$V_{AVR}$} ;

	\draw[->] ([shift=({-20mm,0})] efdcircle.west) node[left] (efdlabel) {$E_{FD0}$} -- ([shift=({-5mm,0mm})] efdcircle.west) ;

	% REACTIVE DROOP BLOCK
	\node (qdroopgain) [right=120mm of avrrounded.west, draw, very thick, isosceles triangle, shape border rotate=180, stewartpurple, minimum height=15mm, minimum width=15mm] {$k_Q$};

	\node[right=20mm of qdroopgain.east] (qdroopinputsum) {};
	\node[left=20mm of qdroopgain.west] (qdroopoutputsum) {};

	\node at ($(qdroopinputsum)!0.5!(qdroopoutputsum)$) [draw, rounded corners, stewartpurple, fill=stewartpurple, fill opacity=0.2, very thick, dashed, line cap = round, minimum width=90mm, minimum height=3cm] (qdrooprounded) {};
	\node[stewartpurple, below=2mm of qdrooprounded.south] {\Large Reactive Droop};

	\draw[stewartpurple, very thick] (qdroopinputsum)  circle  [radius=5mm];
	\draw[stewartpurple, very thick] (qdroopoutputsum) circle  [radius=5mm];

	\node (qcompute) [above=30mm of qdroopinputsum.north, draw, stewartpurple, minimum width=25mm, very thick, minimum height=15mm] {$Q\left(V,\delta\right)$};

	\node (qcompute_inputleft) at ([shift=({-5mm,1mm})]qcompute.north) {};
	\draw[<-,stewartpurple] (qcompute_inputleft.center) -- (qcompute_inputleft |- vout);
	\draw[stewartpurple,fill] (qcompute_inputleft |- vout) circle (0.75mm) ;

	\node (qcompute_inputright) at ([shift=({5mm,1mm})]qcompute.north) {};
	\draw[<-,stewartpurple,preaction={draw,white,line width=4pt}] (qcompute_inputright.center) -- (qcompute_inputright |- deltaout);
	\draw[stewartpurple,fill] (qcompute_inputright |- deltaout) circle (0.75mm) ;

	\draw[->,stewartpurple] (qcompute.south) -- ([shift=({0,5mm})]qdroopinputsum.north);

	\draw[->,stewartpurple] ([shift=({-3.5mm,0mm})]qdroopinputsum.west) -- ([shift=({1mm,0mm})]qdroopgain.east);
	\draw[->,stewartpurple] (qdroopgain.apex) -- ([shift=({4.5mm,0mm})]qdroopoutputsum.east);

	\draw[<-,stewartpurple] ([shift=({0,-5mm})]qdroopinputsum.south) -- ++(0mm,-12mm) node[below] {$Q_0$};
	\draw[<-,stewartpurple] ([shift=({0,-5mm})]qdroopoutputsum.south) -- ++(0mm,-12mm) node[below] {$V_0$};

	\draw[->,stewartpurple] ([shift=({-3.8mm,0})]qdroopoutputsum.west) -- ([shift=({6mm,0mm})]avrcircle.east) node[above, midway] {$V_{REF}$};
	
	% ACTIVE DROOP
	\node[above=20mm of machineblock.north, rounded corners, minimum width=80mm, minimum height=3cm] (activedrooprounded) {}; 

	\node[stewartgreen] at ([shift=({0mm,4mm})] activedrooprounded.north) {\Large Active Droop};

	\node[left=10mm of activedrooprounded.east] (pdroopinputsum) {};

	\draw[<-,stewartgreen] ([shift=({0mm,5mm})] pdroopinputsum.north) -- ++(0,12mm) node[above] (w0label) {$\omega_0$};

	\draw[stewartgreen, very thick] (pdroopinputsum)  circle  [radius=5mm];
	
	\node (pdroopgain) [left=20mm of pdroopinputsum.center, draw, very thick, isosceles triangle, shape border rotate=180, stewartgreen, minimum height=15mm, minimum width=15mm] {$k_P$};

	\draw[->,stewartgreen] ([shift=({-6mm,0mm})] pdroopinputsum.east) -- ([shift=({1mm,0})]pdroopgain.east) ;

	\node[left=18mm of pdroopgain.apex] (pdroopsum) {};
	\draw[stewartgreen, very thick] (pdroopsum) circle  [radius=5mm];

	\draw[->,stewartgreen] ([shift=({1mm,0mm})] pdroopgain.apex) -- ([shift=({5mm,0})]pdroopsum.east) ;

	\draw[<-,stewartgreen] ([shift=({0mm,5mm})] pdroopsum.north) -- ++(0,12mm) node[above] (p0label) {$P_0$};

	\node (pdroopinput) at ([shift=({10mm,0mm})] omegabreak |- omegaout) {};

	\draw[->,stewartgreen,preaction={draw,white,line width=4pt}] (pdroopinput.center) |- ([shift=({5mm,0})]pdroopinputsum.east) ;

	\draw[stewartgreen,fill] (pdroopinput) circle (0.75mm) ;

	\node[draw, above=20mm of machineblock.north, rounded corners, stewartgreen, fill=stewartgreen, fill opacity=0.2, very thick, dashed, line cap = round, minimum width=80mm, minimum height=3cm] (activedrooprounded) {};


	% TURBINE-GOVERNOR
	\node[draw, left=15mm of pminput, rounded corners, stewartpink, fill=stewartpink, fill opacity=0.2, very thick, dashed, line cap = round, minimum width=70mm, minimum height=3cm] (tgblock) {};

	\node (governorblock) [left=5mm of tgblock.center, draw, stewartpink, minimum width=25mm, very thick, minimum height=15mm] {$\dfrac{k_G}{sT_G + 1}$};
	\node[stewartpink] at ([shift=({0mm,-3mm})] governorblock.south) {\large Governor};

	\node (turbineblock)  [right=5mm of tgblock.center, draw, stewartpink, minimum width=25mm, very thick, minimum height=15mm] {$\dfrac{k_T}{sT_T + 1}$};
	\node[stewartpink] at ([shift=({0mm,-3mm})] turbineblock.south) {\large Turbine};

	\draw[<-,stewartgreen] ([shift=({0mm,1mm})] governorblock.north) |- ([shift=({-3.8mm,0})]pdroopsum.west) node[near end, above] {$P_{REF}$};
	\draw[->,stewartpink] (governorblock.east) -- ([shift=({-1mm,0})]turbineblock.west) ;
	\draw[->,stewartpink] (turbineblock.east) -- ([shift=({-1mm,0})]pminput) ;

        \end{tikzpicture}
}
	\caption{Control schematic of ``complete'' synchronous machine model with automatic control.}
	\label{fig:machine_model_controls}
\end{figure} %>>>
%\vfill
%\newpage

% ---------------------------------------------------------
\subsection{Control of Power Systems} \label{subsec:power_system_control}%<<<2
% ---------------------------------------------------------

	Beyond modelling, the representation of a power system by an algebraic equation \ref{eq:multimachine_admittance} also originates a lot of the controllers extensively used in the control of Power Systems. Such controllers are designed using the phasor-equivalent models \eqref{eq:machine_2a_model} and \eqref{eq:omib_line}, therefore using the notions of time-varying phasors generated by such models. However, these controllers also use notions of time-varying complex power based on those models, despite there not being a clear and consistent way of defining such power concepts in nonstationary regimens. As such, there is a \textit{conceptual} problem with transient controllers in phasor space, namely that there is no guarantee that the controllers in phasor space control the very time signals they intend to because they control complex phasorial quantities that only approximate the functions in time.

	Of course, for static phasors, there is a clear bijection between the phasor quantities and the time signals; thus, parametric analysis in phasor space is guaranteed to yield results in the time domain. Again, this fact is leveraged using the QSH, supposing that the system approximates its steady-state behavior once frequency swings are small and slow. However, if such is not the case, there is no guarantee that the phasorial quantities controlled by and output by these controllers indeed produce time signals that achieve the control objectives; these controllers are stable and do adhere to their objectives in phasor space, but without a clear-cut way to translate the time-varying phasors to time signals and vice-versa, it cannot be guaranteed that the control objectives are fulfilled for the time domain.

	For instance, one consequence of the QSH on frequency and voltage control in Power Systems is the decoupling between frequency-active power and voltage-reactive power. This is justified by a sensitivity analysis on the equations \eqref{eq:approx_power_flow_eqs}:

\begin{align}
        \dfrac{\partial P_{km}}{\partial \theta_k} &= V_k V_m B_{km}\cos\left(\theta_k - \theta_m\right) \\[5mm]
%
        \dfrac{\partial P_{km}}{\partial \theta_m} &= V_kV_m B_{km}\cos\left(\theta_k - \theta_m\right) \\[5mm]
%                         {km}
        \dfrac{\partial P_{km}}{\partial V_k} &= 2V_k B_{km}\sin\left(\theta_k - \theta_m\right) \\[5mm]
%                         {km}
        \dfrac{\partial P_{km}}{\partial V_m} &= V_k B_{km}\sin\left(\theta_k - \theta_m\right) \\[5mm]
%                         {km}
        \dfrac{\partial Q_{km}}{\partial\theta_k} &= V_kV_m B_{km}\sin\left(\theta_k - \theta_m\right) \\[5mm]
%                         {km}
        \dfrac{\partial Q_{km}}{\partial\theta_m} &= -V_kV_m G_{km} B_{km}\sin\left(\theta_k - \theta_m\right) \\[5mm]
%                         {km}
        \dfrac{\partial Q_{km}}{\partial V_k} &= -V_m B_{km}\cos\left(\theta_k - \theta_m\right) \\[5mm]
%                         {km}
        \dfrac{\partial Q_{km}}{\partial V_m} &= - V_k B_{km}\cos\left(\theta_k - \theta_m\right)
\end{align}

	Considering that the lines are ``strong'' enough (the values $\left\lvert G_{km}\right\rvert$ and $\left\lvert B_{km}\right\rvert$ are sufficiently large), we can consider that the angle difference $\theta_k - \theta_m$ is small enough to consider its sine approximately itself and its cosine as almost unitary; the derivative of the active power $P_{km}$ with respect to voltages becomes small, as does the derivative of $Q_{km}$ with respect to any angle. Thus, one concludes that the active power has a larger influence on the angles (thus the frequency), while the reactive power has more influence over the voltages. Therefore, in a very short simplification, this causes the ``active'' part of the circuit and the frequency behavior to be decoupled from the ``reactive'' and voltage behaviors. By virtue of these facts, one concludes that to adjust frequency one must control active power, and to adjust voltage magnitudes one must control reactive power.

	At a first glance, the simplest controllers that can be drawn from these conclusions are ones that adjust frequency linearly with active power, and voltage linearly with reactive power, called \textbf{Droop control}. The main notion is that if the grid is accelerating in frequency, then by the decoupling conclusion it has more active power offered to it than its loads can consume; therefore, the Droop controllers force generators to reduce active power injection. Conversely, a dip in frequency means more active loading than offered, thus making the generators ramp up active power injection. Similarly, one concludes that if the system has too high voltage levels, the grid is consuming less reactive power than is offered, and generators are forced to supply less reactice power, and vice-versa.

	Figure \ref{fig:machine_model_controls} shows the control block of a synchronous machine containing such controllers, augmented by other conventional control loops. In {\color{stewartgreen} green}, the active Droop control adjusts the mechanical power supplied to the machine based on variations of the frequency $\omega$ in a linear fashion, measured with respect to the synchronous frequency (that is, if the machine is at $\omega_0$ then $\omega = 0$ in the model). The active Droop control does this by a linear relationship

\begin{equation} P_{REF} - P_0 = k_P\left(\omega - \omega_0\right) \end{equation}

	\noindent where $P_{0}$ is the operating active power at the synchronous frequency, $\omega_0$ a reference frequency (generally the synchronous) and $k_P$ some gain. The signal $P_{REF}$ is a reference power that is sent to the machine governor-turbine group, represented in {\color{stewartpink} pink} in the model. These blocks model the delays and gains of governor and turbine in the form of the gains $k_G,k_T$ and time constants $T_G$ and $T_T$. This group is responsible for applying the reference power $P_{REF}$ to the machine shaft, supplying the reference power to the machine.

	On the bottom side, the machine field coil is controlled by a pair of controllers called Automatic Voltage Regulator (AVR) and a Power System Stabilizer (PSS), in what is called the excitation control group. The AVR, noted in {\color{stewartblue} blue}, is designed to adjust the field voltage $E_{FD}$ to achieve a terminal voltage reference $V_{REF}$, using a gain $K_e$ and a delay $T_e$. Indeed, it can be seen from \eqref{eq:machine_2a_model} that higher $E_{FD}$ leads to a higher $E'_q$, thus inducing a larger voltage. The reference terminal voltage $V_{REF}$ is supplied by an active Droop control, denoted in {\color{stewartpurple} purple}; this control group works in the same way as the active Droop: it adjusts the terminal voltage reference $V_{REF}$ based on swings in active power according to another linear relationship

\begin{equation} V_{REF} - Q_0 = k_Q\left(V - V_0\right) \end{equation}

	\noindent where $Q_0,V_0$ are reference values and $k_Q$ a gain.

	Finally, the system is also equipped with a PSS, noted in {\color{stewartyellow} yellow} color. This controller aims to adjust field voltage $E_{FD}$ in order to dampen frequency swings using a delay-advance controller; additionally, a washout block is used to remove low-frequency disturbances. This controller is a transient stabilizer, as opposed to the AVR which is aimed at voltage (thus mid- and long-term) stability. The PSS also has the objective of damping harmful oscillations caused by needed high AVR gains \pcite{Volpato2017}.

	All these controllers are ultimately based on the phasorial models, time-varying notions of active and reactive power, and the approximations that follow considering many simplification hypotheses. More importantly, however, is the fact that all the controllers are designed based on small-signal analysis, specifically the eigenanalysis of the linearized equations of the system around an operating point. For instance, \cite{demelloConceptsSynchronousMachine1969} first described the harmful feedback loop brought by AVRs because essentially they inject oscillations in phase with the frequency transfer function, and the PSS was designed to inject oscillations in counterphase with frequency through the advance-delay controller tuning. Modern tuning algorithms for AVRs and PSSs still rely on small-signal analysis, for instance, in \cite{kimNovelControlStrategy2023,xuSmallSignalStabilityAnalysis2024,sarkarFractionalOrderPIDPSS2025}. Since these techniques rely on small signal disturbances, they expect phasorial signals, as well as frequency, to vary little and slowly; ultimately, this means that the QSH is ingrained within the very design, evaluation and tuning of such controllers.

% -------------------------------------------------------
\subsection{The QSM beyond Power Systems} %<<<2
% ---------------------------------------------------------

	What subsections \ref{subsec:synchmachine_modelling}, \ref{subsec:largemulti} and \ref{subsec:power_system_control} intend to show is that the apparently simple supposition of small and slow frequency variations has a wide and deep reach over most aspects of Power System studies, in all its forms: signal representation, machine and transmission system modelling, and control and stability theories. This supposition is then known as the \textbf{Quasi-Static Hypothesis or Modelling} (QSH or QSM): the widely permeating hypothesis in Power System literature that the dynamic models suppose slow and small frequency variations.

	Due to its reach and depth, the literature has made efforts to justify the QSM. From a practical point of view, the qualitative and quantitative results stemming from these simulations have been shown verosimile, and widely discussed in the literature, for instance in \cite{zhuDWTBasedAggregatedLoad2018} where the quasi-static modelling is compared to a Discrete Wavelet Transform modelling and in \cite{gustafssonWavePropagationCharacteristics2015} which studies propagation of low-frequency waves in HVDC cables. I, myself, have published a paper \pcite{volpatoDynamicPhasorTransform2022} using the Short-Time Fourier Transform to theoretically support the QSH.

	From a theoretical point of view, the QSH is a intuitive way to think about ``slow-varying'' nonstationary signals. Formally, the QSH represents two facts that are supposed true in Power System literature. First, that the agents (machines) powering the electrical networks that model the transmission line are ``almost sinusoidal'', that is, they convey almost pure sinewaves with very small and very low bandwidth distortion. This allows converting the EMT models of agents into PE models, and that the phasor quantities obtained from the PE simulations reconstruct signals (like that of \eqref{eq:equivalent_emt_X}) that approximate the time domain EMT signals with a sufficient degree of precision. Second, since the excitation signals are ``slow-varying'', the network circuit modelling the grid is supposed very fast, in such a way that all its phasorial transients vanish swiftly and the grid dynamic equations reach steady-state quickly; therefore, the grid equations can be accurately approximated by their steady-state (algebraic) phasorial models at all time instants. This allows for using \eqref{eq:omib_line} as a nigh-faultless model of the transmission line, and using the admittance matrix as in \eqref{eq:multimachine_admittance} to model a big transmission system.

	It must be noted that the issue of representing signals and differential equations under nonstationary regimens is not exclusive of the Power System Literature, but permeates several other fields of Electrical Engineering and Applied Mathematics. For instance, the semicondutor literature is proficient in enhancing quasi-static models of electronic devices: \cite{crupiAnalysisQuasistaticAssumption2006} analyzes the effectiveness of a quasi-static model for a FinFET device, while \cite{allmanQuasistaticResponseMOS1981} analyzes the quasi-static behavior of a MOS FET under constant gate bias. The literature on electromagnetics is also known for studying quasi-static models of electromagnetic phenomena; for instance, \cite{mazauricGalileanCovarianceMaxwell2014} shows Galilean Electromagnetism is the equivalent of quasi-static solutions to Maxwell's Equations.

 	For a more ellaborate example, take a Frequency Modulated (FM) signal demodulator, the simplest of which is based on a first-order Phase-Locked-Loop as shown in figure \ref{fig:example_pll}. The objective of this PLL is to receive a certain frequency-modulated signal $x(t) = \sin\left(\theta(t)\right)$ and produce an estimation of the quantity $\dot{\theta}$, which is the de-modulated message or signal. To do this, an estimated signal $x_e(t) = \sin\left(\theta_e(t)\right)$ is produced, and multiplied in a mixer with $x(t)$. Using the sine product-to-sum formulas,

\begin{equation} x(t)x_e(t) = \sin\left(\theta(t)t\right)\sin\left(\theta_e(t)t\right) = \dfrac{1}{2}\left\{\raisebox{4mm}{} \sin\left[\raisebox{3mm}{} \theta(t) + \theta_e(t)\right] + \sin\left[\raisebox{3mm}{}\theta(t) - \theta_e(t)\right]\right\} . \end{equation}

	Applying a version of the Quasi-Static Hypothesis, if the swings of $\omega(t)$ are not fast nor wide, then $\omega_e(t)$ stays fairly close to $\omega(t)$ such that the sine of their sum has close to double the bandwidth of $\omega(t)$; therefore, the multiplication is passed through a Low-Pass filter which ``removes'' the sine of sum portion leaving only the sine of difference. Again, if $\omega(t)$ is slow enough, $\omega_e(t)$ will be fairly close to it and the difference will be sufficiently small that the sine of the difference is almost equal to the difference itself:

\begin{equation} e(t) \approx \dfrac{1}{2}\sin\left[\raisebox{3mm}{}\theta(t) - \theta_e(t)\right] \approx \dfrac{1}{2}\left[\raisebox{3mm}{}\theta(t) - \theta_e(t)\right] . \end{equation}

	This signal $e(t)$ then approximates the error deviation from $\theta(t)$ and the estimation $\theta_e(t)$ and is passed to a PI controller, which adjusts $\theta_e(t)$ itself to vanish the error signal. Again, supposing that the frequency variations are sufficiently slow, then the PI controller will be able to continuously track the estimation $\theta_e$ that vanishes the error. This estimation is then passed to a Voltage Controlled Oscillator (VCO) that produces $x_e(t)$.

% PLL SUBSYSTEM FIGURE <<<
\begin{figure} 
\centering
\begin{tikzpicture}[>={Stealth[inset=0mm,length=1.5mm,angle'=50]}]

% Sum shape
\node[draw, circle, minimum size=0.6cm] (sum) at (0,0){};
 
%\node at ([shift=({-3.5mm,3mm})]sum.center){\tiny $+$};
%\node at ([shift=({-3mm,-4mm})]sum.center){\tiny $-$};

\node (breaknode) at ([shift=({0,-20mm})]sum.center) {};

\draw (sum.south west) -- (sum.north east);
\draw (sum.south east) -- (sum.north west);
 
% LPF
\node [draw, very thick, minimum width=2cm, minimum height=1.2cm, right=1cm of sum]  (controller) {};
\node at ([shift=({ 0  , 1})]controller.center) {LPF};

\draw [->, very thick]  ([shift=({-0.8,-0.5})]controller.center) -- ([shift=({-0.8, 0.5})]controller.center);
\draw [->, very thick]  ([shift=({-0.9,-0.4})]controller.center) -- ([shift=({ 0.9,-0.4})]controller.center);
\draw ([shift=({-0.8, 0.2})]controller.center) -- ([shift=({ 0.2  , 0.2})]controller.center);
\draw ([shift=({ 0.2, 0.2})]controller.center) -- ([shift=({ 0.6,-0.4})]controller.center);

% PI
\node [draw, very thick, minimum width=2cm, minimum height=1.2cm, right=2cm of controller]  (picontroller) {PI};
 
% System H(s)
\node [draw, minimum width=2cm, very thick, minimum height=1.2cm, right=1.5cm of picontroller] (system) {VCO};

% Phase detector
\node [draw, thick, rounded corners, dashed, line cap = round, red, minimum width=5cm, minimum height=3cm, left=1.25cm of picontroller]  (detector) {};
\node [red] at ([shift=({ 0  , 2mm})]detector.north) {Phase detector group};
 
% Arrows with text label

\draw[->] (sum.east) -- ([shift=({-0.1,0})]controller.west) node[midway,above]{};
       
\draw[->] (controller.east) -- ([shift=({-0.1,0})]picontroller.west) node[near end,above] {$e(t)$};
       
\draw[->] (picontroller.east) -- ([shift=({-0.1,0})]system.west) node[midway](piout){}  node[midway,above]{};
      
\draw[->] (system.east) -- ++ (1.25,0) node[midway] (output) {} ;
       
\node[->] (xoeout) at ([shift=({12mm,0})]output.center) {$x_e(t)$};
                                               
\draw (output.center) |- (breaknode.center);
                                               
\draw[->] (breaknode.center) -- ([shift=({0,-0.1})]sum.south) node[near end,left]{};
 
\draw[->] ([shift=({-1.5,0})]sum.west) -- ([shift=({-0.1,0})]sum.west) node[near start,above]{$x(t)$};

\node[->] (thetaout) at ([shift=({0,20mm})]xoeout.center) {$\theta_e(t)$};

\draw[->] (piout.center) |- (thetaout);

% Integrator

\node [draw, very thick, minimum width=1.5cm, minimum height=1.5cm, above=3cm of piout] (integrator) {$\dfrac{d}{dt}$};

\draw[->] (piout.center) -- ([shift=({0,-0.1})]integrator.south);

\node[->] (omegaout) at (integrator.center -| xoeout.center) {$\omega_e(t)$};

\draw[->] (integrator.east) |- (omegaout.west);
\end{tikzpicture}
\caption{Simple first-order Phase Locked Loop synchronization subsystem.}
\label{fig:example_pll}
\end{figure}
%>>>

	The signal $\omega_e(t)$ obtained from the PLL subsystem is then the demodulated function which is later on used for the target application. Again, there is a theoretical dissonance in this modelling in that the signals $\theta(t)$ is ``slow-varying'', which in turn means that the filtering and feedback are very resemblant of a sinusoidal state; yet, this PLL system is used even when the circuit or control scheme in study is subjected to large transients.

	From an Applied Mathematics point of view, the issue of the QSH lies in the realm of Differential Equations and Functional Analysis. In general, a passive time invariant linear circuit can be modelled as a Linear Time Invariant Ordinary Differential Equation

\begin{equation} \dot{\mathbf{x}} = \mathbf{Ax + Bf}(t), \label{eq:lincircuit_ode_general}\end{equation}

	\noindent where $\mathbf{x}$ are the system states (capacitor voltages and inductor currents), $\mathbf{A}$ a matrix comprised of combinations of the R, L and C values of the circuit, $\mathbf{B}$ an adjacency matrix of the excitations and $\mathbf{f}(t)$ the vector of excitations or ``forcings''. It is known from the theory of Differential Equations that if $\mathbf{A}$ has certain characteristic (\textit{videlicet}, that it is Hurwitz Stable) then $\mathbf{x}$ will exponentially approach a stable steady-state solution; if the vector of excitations $\mathbf{f}$ is composed of sinusoidal voltages and current sources at a particular frequency $\omega$, the steady-state solution of $\mathbf{x}$ will also be made of sinusoidal signals at the excitation frequency $\omega$. This allows for developing the theory of Classical Phasors by transforming the signals $\mathbf{x,f}$ into phasor-equivalent forms and \eqref{eq:lincircuit_ode_general} is transformed into the PE model

\begin{equation} \mathbf{0} = \left(\mathbf{A} - j\omega\mathbf{I}_n \right)\mathbf{X + BF} \Leftrightarrow \mathbf{X} = -\left(\mathbf{A} - j\omega\mathbf{I}_n \right)^{-1}\mathbf{BF}, \label{eq:lincircuit_ode_phasor}\end{equation}

	\noindent where $j$ is the imaginary unit, $\mathbf{F}$ and $\mathbf{X}$ are the phasorial versions of the forcings $\mathbf{f}$ and the steady-state solution of $\mathbf{x}$, $\mathbf{I}_n$ the n-th order identity matrix, and the invertibility of the matrix $\mathbf{A} - j\omega\mathbf{I}_n$ is guaranteed by the fact that in a passive linear circuit $\mathbf{A}$ has real stable eigenvalues. In simpler terms, if the vanishing transient portions of $\mathbf{x}$ are disconsidered, the original time-domain differential equation \eqref{eq:lincircuit_ode_general} is transformed into an algebraic complex equation \eqref{eq:lincircuit_ode_phasor} which solution is, both analytically and computationally, exceptionally simple: the only ``challenge'' is the inversion of the matrix $\mathbf{A}$ which, albeit a classically computationally expensive task, is still leagues of magnitude simpler than solving the EMT model \eqref{eq:lincircuit_ode_general} in time.

	If $\mathbf{f}$ is not exactly sinusoidal but \textit{almost sinusoidal}, that is, its sinusoidal components are ``close to $\omega$'' in that their frequencies are small deviations from $\omega$, it is also known from Functional Analysis that if the forcing $\mathbf{f}$ in \eqref{eq:lincircuit_ode_general} is continuous (in the Banach Space of functions) with respect to $\omega$  then the solution of the ``almost-sinusoidally-forced'' ODE \eqref{eq:lincircuit_ode_general} will be closed to the solution of the perfectly sinusoidal one; formally, writing $\mathbf{f}$ as a vector of $k$ forcings

\begin{equation} \mathbf{f}(t) = \left[\begin{array}{c} \left\lvert f_1(t)\right\rvert\cos\left(\omega(t)t + \phi_1(t)\right) \\[3mm] \left\lvert f_2(t)\right\rvert\cos\left(\omega(t)t + \phi_2(t)\right) \\[3mm] \vdots \\[3mm] \left\lvert f_k(t)\right\rvert\cos\left(\omega(t)t + \phi_k(t)\right)\end{array}\right]\end{equation}

	\noindent and if the variations of amplitudes $\left\lvert f_i(t)\right\rvert$ and phases $\phi_i(t)$ are small and slow, and if the time-varying frequency $\omega(t) = \omega_0 + \Delta\omega(t)$ for some constant $\omega_0$, then one can conceive a ``time-varying'' sinusoidal phasorial forcing

\begin{equation} \mathbf{F}(t) = \left[\begin{array}{c} \left\lvert f_1(t)\right\rvert e^{j\phi_1(t)} \\[3mm] \left\lvert f_2(t)\right\rvert  e^{j\phi_2(t)}\\[3mm] \vdots \\[3mm] \left\lvert f_k(t)\right\rvert  e^{j\phi_k(t)} \end{array}\right]\end{equation}

	\noindent such that the phasorial signal

\begin{equation} \mathbf{X}(t) = -\left(\mathbf{A} - j\omega_0\mathbf{I}_n \right)^{-1}\mathbf{BF}(t) \label{eq:lincircuit_ode_phasor_steadystate} \end{equation}

	\noindent reconstructs the solution $\mathbf{x}$ of the time-domain differential equation with some degree of accuracy through the reconstruction formula \eqref{eq:equivalent_emt_X}. However, as it is common with mathematics, and often a source of grief between mathematicians and engineers, this procees is not able to determine ``how close'' $\mathbf{f}$ has to be to a perfect sinusoidal excitation so that $\mathbf{x}$ is ``close enough'' to its sinusoidal version, which is a major concern in engineering because in Power Systems voltage and frequency deviations are not only problematic for their potentially damaging effects in consumer and industry applications, but also heavily regulated in real world systems.

% ---------------------------------------------------------
\subsection{Modern Power Systems}\label{subsec:in} %<<<2
% ---------------------------------------------------------

	The Quasi-Static Hypothesis is a major point of fracture in the Power System literature because it depends on a very specific nature of the electrical grid and particularly of the agents that power it. In the classical EPS literature, because the majority of the agents involved are large electrical machines, the QSH becomes a reasonable modelling hypothesis for machines are large devices with significant mass and rotational inertia representing a lot of mechanical energy stored in the rotating stator, making the system inherently ``slow''. Further, the presence of strong magnetic fields generated by large and long coils also stores a large amount of magnetic energy in those fields, so that the transmission grid is inherently ``quicker'' than the sinusoidal waves injected by machines. Classical grids are also many times composed of transformers, feeders and condensers, all electromechanical in nature with huge inductances and masses, providing yet another layer of inertia. Beyond the very nature of the devices that compose the grid, most large systems have a collaborative and centralized control that monitors some key nodes in the grid and takes actions to ensure some proper functioning of the system. These constructive, inertial and controlling characteristics of the grid result in a high level of reasonability when using the Quasi-Static Hypothesis.

	More recently, the EPS literature has been growingly occupied with integrating distributed generators to modern grids, spearheaded by the growing adoption of Renewable Energy Sources (RES) like photovoltaic and wind generators, as well as the integration of battery systems. Because these more modern systems are based on electronic power devices like converters and inverters, they lack the inertial characteristics of machines and transformers; further, because the generators are distributed and generally not a part of the centralized control that large systems may have, they can take only localized actions without much information of the overall system they are connected to. Several key concepts are also inherently different from large power systems: for instance, conventional generators like machines and turbines are dispatchable, that is, there is a reasonable interval of control where the operator can reduce or enhance power output based on stability and power criteria, like for instance, varying active and reactive power outputs using Droop controllers like those of figure \ref{fig:machine_model_controls}. A diesel generator can control its fuel intake, a hydro power plant can control the aperture of watergates, a nuclear power plant can control the pressure of the circulating water, and so on. RES devices, on the other hand, are intermittent: a photovoltaic power plant can only generate power depending on how much insolation and temperature it receives, a wind generator depends on the speed of the wind through it, a seawave power generator depends primarily on currents, breaking wind, tides, water temperature.

	%Further, the economics of modern power grids introduce new challenges to the already large list of operating constraints and requirements of traditional grids. Where classical EPSs are primarily concerned with the power quality and stability of the grid, as well as maintaining proper functioning of key areas of the system (such as basic infrastructure like hospitals, public stations and emergency services) while being robust to faults, modern power systems also need to be aware of how much power they are producing and ideally maximizing power output since, in general, individual power plants generate revenue for its operators — and this revenue is obviously based on how much power is produced — and, although having considerably smaller maintenance and running costs than machines (like the cost of fuel and operation), they have high implementation initial costs, pressuring for economical returns.

	%, where the multiple control objectives can become contradictory. Famously, in 2013 the California Independent System Operator (CAISO) released, in its annual report \pcite{caiso2013AnnualReport2014}, data and analysis that outline some concern regarding the overgeneration ocurring from the increased integration of photovoltaic power plants. The now famous ``Duck Curve'' \pcite{caisoFastFactsWhat2016} shows that during the specific times when PV generation is at a day-high (the particular hours between 9AM and 3PM) it provides more energy that the system can effectively use, forcing the larger grid to operate below safe levels. Further, at the specific time interval between 3PM wnd 6PM, there is a vertiginous fall in PV power production, yet a comparable vertiginous rise in power consumption; this leads to a double-whammy mechanism where the traditional generators have to accelerate too quick, leading to instability and in some extreme cases damage to grid components. These problems, then, lead to a myriad of technical and economical challenges on power system economics and operation stemming from procedural, regulatory and operational challenges brought about by distributed, intermittent generation.

	The fact that the newer power devices cannot afford the system such inertia cracks down on every possible aspect of classical power systems discussed until here. If the frequency swings are not slow and small, the modelling of agents as phasor-equivalent models like \eqref{eq:machine_2a_model} and \eqref{eq:machine_2a_model_classical} is not possible anymore due to the consequent frailty of the supposition that the agents supply ``almost sinusoidal'' voltages and currents to the system. Further, a static modelling of the grid like in \eqref{eq:multimachine_admittance} is not possible, because the transient phenomena are now much different than pure sinusoids, and the voltage-current relationships are no longer given by simple impedance equations $V = ZI$. As a consequence, the power flow equations \eqref{eq:power_flow_eqs} are also invalid. Because not only the static admittance modelling is asunder, but also the notions of time-varying active and reactive power are not approximable from their static counterparts, the ``decoupling'' between active power and frequency, and reactive power and voltages is also not valid. Finally, this also undermines the validity of the linear controllers designed for the system, and particularly the active-reactive Droop adjustment controllers, as in figure \ref{fig:machine_model_controls}.

	Naturally, the limits of these approximations lie specifically on \textit{how quick} the system is and if the frequency swings are acceptably small and slow. In practice, albeit it being known that the approximations are conceptually invalid, it is supposed that for however imprecise they are, they become better as frequency swings subside; truthfully, such is indeed the case for the majority of systems and study cases, where the system is able to restore itself to equilibrium after faults.

	Thus, the intermittent, faster and ``less inertial'' nature of power electronics devices places modern power systems in a rather difficult state of affairs where new stability results and transient phenomena must be identified and studied, yet the underlying QSH assumption of the models used is not satisfied by the devices employed. This undermines the timescales argument made when justifying the QSH, which by its own volition undermines the phasorial theory used to represent Electrical Power Systems, by consequence putting a question mark on whether the controllers designed, simulation results obtained, and the stability theory developed using these rutted phasor theories are really reflective of the systems they model. 

	There is an already large yet still growing body of literature dedicated simply to find and analyze the new stability (in all its forms) and power balance issues stemming from the harsher and less-forgiving nature of generators that depend on the environment. Much of the current theory lies, for instance, in studying under which conditions the stability analysis results of classical Power Systems like small-signal analyses \pcite{mishraPhillipsHeffronModelPVDG2013} indirect energy methods \pcite{sauerPowerSystemDynamics2017} and even to use certain control schemes to make converter-based systems mimic the behavior of machines, like the Virtual Synchronous Machine (also called Synchroverter) controllers \pcite{moEvaluationVirtualSynchronous2017}. Further, there is an increasing preoccupation with the fact that modern power grids are in essence cyberphysical systems that communicate using modern protocols and techniques, meaning they are subject to cyber attacks and the detection and prevention of such attacks is needed \pcite{karanfilDetectionMicrogridCyberattacks2023}. From the theoretical perspective of this thesis, the problem is born at a much fundamental step: the inception of a theory that supports the phasorial models used in the nigh-entirety of Power System studies, especially those involving modern grids.

% ---------------------------------------------------------
\section{Problems this thesis aims to tackle}\label{subsec:intro_problems_tackle}
% ---------------------------------------------------------

	Given the introductory discussion, it becomes clear that the target of this thesis is the development of a Dynamic Phasors theory that justifies the classical power system literature from a theoretical point of view, but also embraces fast-responding power systems and offers a more complete framework to represent, model and control modern power systems.

	Initially, due mine and Prof. Luís' backgrounds in Power Systems, the motivations and examples we used initially were naturally aimed at that particular field. However, as I developed this research we noted that the theory that unfolded strayed ever so farther away from our initial motivation of building a Dynamic Phasor Theory for Power Systems, and we delved further and further into Linear Circuit Theory. We started asking ourselves more qualitative questions, like \textit{``how can we guarantee a circuit built of linear components is Hurwitz-stable?''} or \textit{``what does it mean for a signal to have a time-varying frequency?''}. Eventually we convinced ourselves that this was to be a study on a more fundamental, basic matter of theory and not specifically on the application of Power Systems. In reading more on the literature, we also noted that many subfields of Electrical Engineering suffered from the same affliction as we did: the lack of a complete theory for representing Nonstationary Sinusoids in a phasorial form highly resemblant of the original, or Classical, Phasors.

	As a consequence of the breadth of this problem among many fields of engineering and the depth with which it impacts Linear Circuit Theory, this thesis is built as a text on Linear Circuit Theory with the specific aim to cater to a wider audience of engineers, and obviously, electrical engineers specially, but without losing its inceptive motivational application to Power Systems.

% ---------------------------------------------------------
\subsection{Static Phasors as a template}\label{subsec:timephasor}
% ---------------------------------------------------------

	Initially, we looked at the ``static'' or ``classical phasors'' theory, as proposed by Steinmetz when he was studying stability of electrical machines connected to large systems, in order to build requirements — maybe a template if possible — for the theory we envisioned. Formally, static phasors are based on an operator that takes some sinusoidal signal $x(t) = K\cos\left(\omega t + \phi\right)$ and delivers the complex number $X = Ke^{j\phi}$. This operator has the immediate benefit that, while linearly combining sinusoids needs complicated formulas known as the Prostaph\ae resis formulas, linearly combining phasors is a matter of simple complex number geometry.

	Beyond its operational capabilities, an even bigger advantage of phasors is that the Phasor Operator has the benefit of transforming a differential equation in the time domain to an algebraic equation in the complex domain, for if $X = Ke^{j\phi}$ is the phasor of $x(t)$, then $Y = j\omega K e^{j\phi}$ is the phasor of $y(t) = \dot{x}(t)$. Therefore, consider a time ODE

\begin{equation} \sum_{k=0}^n \alpha_k x^{(k)} + M\cos\left(\omega t\right) = 0 \label{eq:steinmets_ode}\end{equation}

	\noindent and by the transform of derivative this ODE is transformed to the phasor equivalent

\begin{equation} \sum_{k=0}^n \alpha_k \left(j\omega\right)^k X + M = 0 \Leftrightarrow X = M\ \dfrac{1}{\displaystyle\sum_{k=0}^n \alpha_k \left(j\omega\right)^k} \label{eq:steinmets_ode_algebraic}\end{equation}

	Due to the simple nature of complex algebra, solving this ODE is simple:

\begin{equation} X = M\dfrac{\overbrace{\left(\alpha_0 - \alpha_2\omega^2 + ...\right)}^{\text{Even exponents}} - j\overbrace{\left(\alpha_1\omega - \alpha_3\omega^3 + ...\right)}^{\text{Odd exponents}}}{\left(\alpha_0 - \alpha_2\omega^2 + ...\right)^2 + \left(\alpha_1\omega - \alpha_3\omega^3 + ...\right)^2} \label{eq:netgrid_original_phasor_sol}\end{equation}

	\noindent and this signal reconstructs

\begin{equation} x_s(t) = K\cos\left(\omega t + \phi\right)\ \left\{\begin{array}{l} K = \dfrac{M}{\sqrt{\left(\alpha_0 - \alpha_2\omega^2 + ...\right)^2 + \left(\alpha_1\omega - \alpha_3\omega^3 + ...\right)^2}}\\[10mm] \tan\left(\phi\right) = -\dfrac{\left(\alpha_1\omega - \alpha_3\omega^3 + ...\right)}{\left(\alpha_0 - \alpha_2\omega^2 + ...\right)} \end{array}\right. ,\end{equation}

	\noindent which can be proven as being the exponentially stable steady-state solution to \eqref{eq:steinmets_ode}. Therefore, the phasor operation translates algebraic complex quantities that have a bijective representation of the time quantities they represent, such that the time signals can be reconstruced from the phasorial ones without any approximations or truncations.

% ---------------------------------------------------------
\subsection{Electrical power in AC regimen}\label{subsec:acpower}
% ---------------------------------------------------------

	 Apart from the bijective relationship between phasors and solutions of differential equations, classical phasors also offer the concept of complex power, or electrical power in Alternate Current regimens. Let $V = m_ve^{j\phi_v}$ and $I = m_ie^{j\phi_i}$ the phasors of the voltage over and current through a bipole. Then the instantaneous power can be shown to be calculated as

\begin{equation} p(t) = P\left[1 + \cos\left(2\omega t + 2\phi_v\right)\right] + Q\sin\left(2\omega t + 2\phi_v\right) \label{eq:time_inst_power}\end{equation}

	\noindent where $P$ and $Q$ are calculated as

\begin{equation} P = \dfrac{m_vm_i}{2}\cos\left(\phi_v - \phi_i\right), Q = \dfrac{m_vm_i}{2}\sin\left(\phi_v - \phi_i\right) . \label{eq:static_pq}\end{equation}

	Now observe that the number $S = \frac{1}{2}\left<V,I\right> = \frac{1}{2}V\overline{I}$ (``$<>$'' denoting the complex internal product), called \textit{complex power}, is such that the real part of $S$ is exactly $P$ and its imaginary part is exactly $Q$. This means that there is a direct bijection between $S$ and the instantaneous power \eqref{eq:time_inst_power}. The physical interpretations of $P$ and $Q$ become clear in two ways: first, integrating $p(t)$ over a period $T = 2\pi/\omega$ results that $P$ is the average power over $T$ while the sine part $Q$ fades on the integral — meaning $P$ is the power spent by the active elements of the circuits over a period while $Q$ is a power cyclically stored in the reactive elements, originating their namesakes. Second, one can easily prove that

\begin{equation} i(t) = \dfrac{2 P}{m_v} \cos\left(\omega t + \phi_v \right) + \dfrac{2 Q}{m_v}\sin\left(\omega t + \phi_v\right), \label{eq:active_reactive_current} \end{equation}

	\noindent meaning that the active power $P$ corresponds to the component of the current that is in phase with the voltage, whilst the reactive power $Q$ corresponds to the component in quadrature with the voltage. The biunivocity between phasors and steady-state solutions of the time LTI ODEs that model the network grid and the complex power representation for instantaneous power mean that the entire analysis of the circuit can be undertaken in the phasor domain, while the time-domain counterparts are accurately represented by the phasorial quantities.

% ---------------------------------------------------------
\subsection{The current literature}\label{subsec:currentlit}
% ---------------------------------------------------------

	These characteristics of classical phasors delineate the initial duty of this thesis: that of constructing a functional transform, defined in the space of nonstationary sinusoids, that produces a phasorial model in the same fashion and with the same results as the classical model. More specifically, considering an ODE

\begin{equation} \sum_{k=0}^n \alpha_k x^{(k)} + M(t)\cos\left(\omega(t) t + \phi(t)\right) = 0, \label{eq:steinmets_ode_timevar} \end{equation}

	\noindent then the primary objective is to build a functional transform that takes $x(t)$ and delivers a time-varying complex function $X(t)$ that transforms \eqref{eq:steinmets_ode_timevar} into a differential equation in the phasor domain like the classical operator transforms \eqref{eq:steinmets_ode} into \eqref{eq:steinmets_ode_algebraic}, such that the phasorial quantities accurately reconstruct the time domain signals. 

	Further, the theory proposed aims to offer a theory of complex power under nonstationary regimens, that is, achieve notions of active and reactive power as in \eqref{eq:static_pq} such that the instantaneous power can be reconstructed from these quantities, like \eqref{eq:time_inst_power} is reconstructed from $P$ and $Q$ of \eqref{eq:static_pq}. Moreover, these new notions of complex power should have the same or similar physical interpretations: $P$ should be the average power over some interval where $Q$ fades, and the current decomposed in some form through $P$ and $Q$.

	Finally the theory developed must generalize the Classical Phasor Theory, as in, classical phasors have to be a particularization of the Dynamic Phasors proposed.

	Several works dealt with this matter, yet none fulfills all these requirements. The literature lacks a solid framework that represents nonstationary sinusoidal signals as time-varying complex functions, keeping intact desirable phasor characteristics familiar to engineers like phase, amplitude and angular frequency. The most widely used framework to represent such signals, the Short-Time Fourier Transform (STFT), presents a major setback by generating a model composed of several (possibly infinite) complex differential systems to solve in order to reconstruct a certain signal in time. Engineers tackle this issue by considering the majority of the signal power is concentrated on the first harmonic, truncating the modelling to the first term only \pcite{veeramrajuDynamicModelACAC2024}, abdicating higher order harmonics and therefore accuracy. Nevertheless, STFT DPs have been extensively used in the literature due to their proximity with Fourier Analysis, as engineers are used to transient impedances and power formul\ae brought by this framework, despite knowledge of their modelling inaccuracy.

	Other approaches have been proposed to represent nonstationary signals while maintaining the idea of phasors, like the Gabor-Wigner Transform \pcite{Cho2010} and the S-Transform \pcite{dashPowerQualityAnalysis2003}. More recently, some researchers have proposed abandoning the idea of phasors altogether in favour of the Hilbert Transform \pcite{derviskadicPhasorsModelingPower2020}, which is able to accurately represent some signals of interest in a frequency domain as long as the signal has limited bandwidth and certain specific characteristics, making the transform limited in scope. The wavelet transform has also been used in power system studies \pcite{Morsi2009} to represent nonstationary signals with varying degrees of success due to the plethora of available wavelet transforms.

 	It was upon reading \cite{Mendes2020} and \cite{Henschel1999}, very detailed works in Dynamic Phasor Theory, that a common point among the theoretical frameworks available became apparent: they almost solely on integral transforms which, while certainly powerful, bring their own set of challenges. \cite{Henschel1999}, for instance, shows that the numerical integration process required to solve systems of complex differential equations built using such transforms is a particularly problematic one when it comes to numerical simulation because integrals inherently need to be differentiated at some point, but numerical differentiation is always reliant on approximations and invariably generate numerical artifacts especially when discontinuous disturbances like steps and impulses are involved.

	\cite{Mendes2020} is particularly concerned with integral trasforms for the specific purpose of reaching a Dynamic Phasor Theory of Power Systems, and makes an argument about the fact that integral transforms have a problem when dealing with the issue of complex power representation since integral transforms generally transform a multiplication into a convolution. As such, extracting specific components like the active or reactive power from the convoluted signal is rather difficult, not to say impossible: it is hard to obtain analytical results from any convolution. The Laplace Transform, in particular, requires the convolution to be calculated at a \textit{stable contour} in the complex space, known as a Brömwich contour, making analytical computation impossible for arbitrary signals. The matter of complex power in nonstationary regimens has its own niche in the literature and has been the target of many discussion over the years: the most used theories used rely heavily on the Quasi-Static Hypothesis and lay heavy hold in approximations and truncations. Further, the current transforms do not offer a theory of complex power under nonstationary regimens, which the literature has been sorely lacking for decades: while the classical concepts of active, reactive and complex power in AC regimen are well understood and widely used, there is no unified representation of such quantities for systems under nonstationary conditions \pcite{Kukacka2016}. This means that there is no standard definition of active and reactive power in nonstationary regimens, despite the fact the literature features several theories \pcite{Kusters1979,emanuelSummaryIEEEStandard2004,Kukacka2016}, including an IEEE Standard cataloguing definitions \pcite{ieeepowerandenergysocietyIEEEStandard421520162016}. The available proposed theories are often contradictory or simply prolix, bringing several concepts like distortion power, fundamental power, nonactive power \pcite{Emanuel1996} \textit{et cetera}, some suggesting complex power should be interpreted as a three or even four-dimensional quantity, a notion close to hypercomplex algebras like quaternion numbers \pcite{eisaNewNotionsSuggested2008}. None of these theories have been widely adopted, consequence of their inadequacy to cater to the natural meaning or significant physical notion of components of the instantaneous power \pcite{eisaPhysicalInterpretationElectric2016} and build nonstationary alternatives to the well-understood active, reactive and apparent power in AC systems, which have clear physical and theoretical interpretations.

	As a consequence, engineers and researchers make up for this using the static phasor definitions to make quasistationary approximations of the classical concepts for complex power \pcite{zhaoDynamicAnalysisUniformity2024}, that is, adopting

\begin{equation} P(t) = \dfrac{m_v(t)m_i(t)}{2}\cos\left(\phi_v(t) - \phi_i(t)\right), Q(t) = \dfrac{m_v(t)m_i(t)}{2}\sin\left(\phi_v(t) - \phi_i(t)\right) . \label{eq:static_pq}\end{equation}

	\noindent which obviously do not reconstruct instantaneous power, requiring again the QSH to justify them. This justifies, for instance, the the power flow formulas \eqref{eq:power_flow_eqs}; for the control of Power Systems, the highly approximated nature of these equations is underwhelming, because the analysis of complex power in nonstationary regimens is a seminal concept for the real-time monitoring of power systems where power quality and harmonics must be assessed in real time (like the Droop controllers of figure \ref{fig:machine_model_controls}), as well as power flow analysis, dynamical state estimation and power system stability where active and reactive power are used proheminently in frequency and voltage control.

	There are some works in the literature that have tried to define notions of Dynamic Phasors without resorting to integral transforms. For instance, \cite{darochaComputacaoAltoDesempenho2024,danielSimuladorTransitoriosEletromagneticos2018,azevedoMetodologiaFasorialPara2024,almeidaModelagemFasorialTrifasica2024} argumented that because sine and cosine are orthogonal functions, a signal of the form

\begin{equation} x(t) = x_d(t)\cos\left(\omega t \right) - x_q(t)\sin\left(\omega t\right) \end{equation}

	\noindent can be represented by some phasor $x_d(t) + jx_q(t)$, akin to the Shifted Frequency Analysis debuted by \cite{zhangSynchronousMachineModeling2007}. \cite{Venkatasubramanian1994} defines that a signal $e_o(t) = E(t)\cos\left(\omega_0 t + \phi(t)\right)$ can be \textit{associated} to a phasor $\hat{e}_0(t) = E(t)e^{j\phi(t)}$, and proceeds to develop ``phasor calculus'', in an approach called \textit{linear operator approach} because such association is linear.

	However, all these strategies fundamentally require that the signal under consideration is limited in its spectrum; for the Shifted Frequency Analysis method, it is supposed that the signal has ``\textit{frequencies within a band centered around a fundamental frequency}''.  The linear operator approach requires that ``(...) we restrict the choice of phasors to those with bandwidths less than the carrier frequency $\omega_0$''. Therefore, ultimately, these tools also limit the set of signals that they can operate.


\section{This text} %<<<1
% ------------------------------------------------

%-------------------------------------------------
\subsection{To whom and for what this text is intended}
%-------------------------------------------------

	The text is meant as an self-contained theory of Linear Circuits using particular applications. It is however natural that, since both I and Professor Luís are researchers of Power Systems, the motivations, examples and discussions are biased towards that particular field. ``Self-contained'' means that the text should be readable as an entire theory without the need of big dives into further literature, if anything to check a theorem or a concept that is unfamiliar to engineers. Even then, the text is not meant as an introductory course on Linear Circuits; as a matter of fact, it is expected the reader has undergone a basic course on the subject. It is supposed that the reader is acquainted with Kirchoff's Laws and Graph Theory to describe circuits. Knowledge of Differential Equations and Linear Algebra are also needed. Due to the nature of this text — a doctorate thesis — it is first and foremost aimed to a graduate-level reader, although someone in final years of undergraduate studies should not have issues. For chapters \ref{chapter:dpos} and \ref{chapter:control_theory}, it is desirable that the reader is acquainted with Complex Analysis, Abstract Algebra, and Functional Analysis. For chapter \ref{chapter:control_theory}, which deals with elementary control theory in the Dynamic Phasor space, the reader is also expected to have undergone courses on Linear Control Theory and Signals and Systems.

%-------------------------------------------------
\subsection{Objective, contributions and thesis overview}
%-------------------------------------------------

	The progression of the thesis and its contributions is as follows:

\begin{enumerate}
	\item A theory on linear systems is introduced with specific results in Linear Differential Equations that support the thesis throughout;
	\item Hence the theory of Classical Phasors is presented as a natural consequence of the Linear Systems theory presented, building the template for the Dynamic Phasor Theory proposed;
	\item The proposed Dynamic Phasors Theory is shown, as motivated by Classical Phasors and the shortcomings of the current theories;
	\item An adaptation of the theory for three-phase circuits is developed;
	\item Using this theory, the formal justification and proof of the Quasi-Stationary Hypothesis is shown, as well as some analysis on multi-frequency systems;
	\item A definition of impedances under nonstationary regimens using Dynamic Phasor Functionals (DPFs), a specific set of functional transforms in Dynamic Phasor space;
	\item Proofs of circuit modelling theorems (Kirchoff's Laws, Voltage-current source duality, Superposition Principle, Thèvenin-Norton Theorems) in their Dynamic Phasor equivalents are shown using DPFs;
	\item An elementary control theory of linear systems under nonstationary regimens is developed using the DPFs and linear systems analysis.
\end{enumerate}

	The text is separated into three parts. Part \ref{part:linearsys_phasor_theory} deals with what could be called as ``classical'' theory, that is, the theory of Linear Systems and the theory of Classical Phasors that stems from it. This part debuts with chapter \ref{chapter:linear_systems} presenting a solid mathematical background on the theory of linear algebra and linear dynamical systems. More specifically, this chapter develops Linear Algebra and Linear Differential Equations in a straightforward way so as to mathematically support the definitions and theorems that come later. This first part has the primary objective to develop the theory of linear algebra from the ground up, starting from the very definitions of vector spaces and linear combinations, then defining matrices as tabular representations of linear maps under a basis. While this is seemingly too elementary for a doctorate thesis, it is precisely these definitions in the specific sequence and construction they are presented in that allow the building of matrices and polynomials of Dynamic Phasor Functionals, which will later expand into an entire theory of network analysis in nonstationary regimen. Further, the definitions of norms of linear maps as well as inner products allow for the development of the Functional Analysis in the Banach Space $L^2$ that originates the fundamental control theory in generalized sinusoidal regimens of chapter \ref{chapter:control_theory}.

	Thence, chapter \ref{chapter:linear_systems} continues with the aim to develop the general solution to a Linear Differential Equation by presenting the matrix exponential as a natural consequence of Jordan Decomposition and the construction of a Jordan Chain of solutions; with a slight introduction to Dynamical Systems, the chapter finishes by proving any stable homogeneous linear system is exponentially stable, and shows definitions of Hurwitz and Lyapunov Stability.

	Further, chapter \ref{chapter:classical_phasors} presents the theory of Classical Phasors a natural consequence of the Hurwitz Stability of linear electrical circuits. The phasor mapping is shown to be an operator in the space of static sinusoids, and several properties are shown like its linearity and complexification of linear differential equations. The theory on complex power under Alternate Current regimen is also presented, followed by some small network analysis section that will be expanded in the Dynamic Phasor domain. This chapter serves as a quick recap on phasor theory in order to build the template and requirements set out for the upcoming a Dynamic Phasors Theory. With the goal of straightforward rememberance rather than comprehensive development of the classical theory, this chapter is purposefully thin and quick; for instance, circuiy analysis techniques in the phasor domain are left unproven, but cited, because their generalized Dynamic Phasor counterparts will be shown and proven in detail later.

	Part \ref{part:dynphasor_theory} deals specifically with Dynamic Phasor Theory. In the first chapter of this part, chapter \ref{chapter:dynamic_phasor_theory}, the motivation and problem of Dynamic Phasors is presented, and the current techniques and frameworks for Dynamic Phasors are presented. The two main techniques modernly used — Short-Time Fourier Transform and the Hilbert Transform — are presented to some detail, with the intent to make a critical review of these techniques, pinpointing exactly what characteristics they lack or cannot provide, this asserting why a new Dynamic Phasor Theory is needed. Ultimately, understanding the shortcomings of these current techniques is the main motivator for the development of the proposed Dynamic Phasors Theory and all that comes next.

	Thence the development of the proposed theory of Dynamic Phasors is presented, by means of what was called the Dyamic Phasor Transform (DPT). Dynamic Phasors are constructed as the result of a specific class of differential operators in the space of complex functions of time. As such, this first chapter of part \ref{part:dynphasor_theory} shows novel results and comprise the fundamental contribution of this thesis. This chapter is the cornerstone of the thesis and should be read more carefully. First it is shown that this proposed theory is a direct mirror of the Classical Phasors theory as it offers the same results for the generalized class of sinusoids It is shown that this theory can construct complex time-varying functions, known as Dynamic Phasors, that directly mirror the nonstationary signals such that one can be reconstructed from the other; in other words, the Dynamic Phasors proposed reconstruct the time-signals they represent without any losses, approximation or truncation. Then, it is shown that this technique can transform a linear differential equation in time to a complex differential equation in the space of Dynamic Phasors, just like classical phasors transform an equation \eqref{eq:steinmets_ode} into an algebraic equation in complex space \eqref{eq:steinmets_ode_algebraic}.

	Further, it is shown that this framework also achieves nonstationary notions of active, reactive and complex power, that have the same expressions \eqref{eq:static_pq} and physical meaning as their static counterparts: $P$ and $Q$ reconstruct instantaneous power just like \eqref{eq:time_inst_power}, the active power is the average power over some interval where the reactive power vanishes and the current can be decomposed in the same way as \eqref{eq:active_reactive_current}, that is, the active power relates to a component of current in phase with voltage while reactive power corresponds to a portion of current in quadrature with voltage.

	Chapter \ref{chapter:dynamic_phasor_theory} also deals with Three-Phase Dyamic Phasors. The idea is to carry the results from single-phase quantities to three-phase, thus keeping this three-phase section shorter and quicker. The main challenge with three-phase Dynamic Phasors is dealing with the added dimension — the zero-sequence component — and asking what kinds of excitations lead to balanced three-phase waves. It is shown that a linear system does not necessarily need to be excited by a balanced three-phase voltage to yield balanced behavior; thus a larger and more permissive condition for balanced behavior is developed.

	Chapter \ref{chapter:choice_apparent_frequency} studies the effects of the choice of the time-varying frequency in the models and results produced by the Dynamic Phasor Theory proposed. This chapter shows a proof that frequency swings in nonstationary sinusoidal excitations add certain dynamic contribution to the circuit response which naturally cannot be ignored. In short, it is shown that if the circuit network is ``much faster'' than the frequency signal adopted, then the circuit differential equations achieve steady-state before the frequency swings happen, meaning steady-state approximation is the more accurate the ``faster'' the circuit is. This consists essentially of a formal proof of the the Quasi-Static Approximation under the Dynamic Phasor Theory proposed, thus justifying classical phasor equivalent models like those of \eqref{eq:machine_2a_model} and \eqref{eq:machine_2a_model_classical} The Dynamic Phasors Theory proposed. It also shows that under such conditions, the impedance relationships although time-varying become essentially their static counterparts, justifying admittance models for large grids like \eqref{eq:multimachine_admittance}.

	Chapter \ref{chapter:choice_apparent_frequency} also investigates what happens if a particular system is modelled using different frequency references; the main result is that if the two frequency signals are ``close enough'' (integrable), then there is a diffeomorphism between the complex systems of differential equations that they produce, meaning that it does not really matter in which frequency reference the system is modelled in, for as long as the solution exists on one of them, it exists for all other ones. Here more important results are shown, for instance, that if a linear circuit is excited by nonstationary sinusoids at a particular time-varying frequency, all voltages and currents will also be nonstationary sinusoids at that particular frequency. This again shows that the theory proposed generalizes Classical Phasor Theory: it is very well known that if a linear circuit is excited by static sinusoids at a particular fixed frequency, voltages and currents are also sinusoids at that frequency, thus a particularization of the larger result if the frequency adopted is fixed. This justifies ``fixed-frequency but time-varying phase models'' like \eqref{eq:equivalent_emt_E} and \eqref{eq:equivalent_emt_X}.

	Following part \ref{part:dynphasor_theory}, part \ref{part:applications} expands the Dynamic Phasor Theory proposed with the idea of Dynamic Phasor Functionals, first presented in chapter \ref{chapter:dpos}. These transforms are the attempt to operationalize the Dynamic Phasor Transform to allow a swifter and more intuitive modelling of linear systems and circuit networks under nonstationary regimens. They are built a special class of complex functional transforms that form powerful algebraic structures, such that differentials in the time domain become algebraic manipulations in Dynamic Phasor space — a notion close to modelling circuits in more mainstream tehcniques like Laplace Transforms. It is proven that a notion of Dynamic Impedances is defineable, and that the paramount theorems of Kirchoff's Laws, the Superposition Principle and the Thèvenin-Norton Theorems also find Dynamic Phasor counterparts. In essence, this contribution means that Dynamic Phasors have the exact same properties as the complex functions obtained from those commonplace frameworks: transforming derivatives and integrals into algebraic complex equations that can be much more easily operated yet are biunivocal and complete representations.

	Because impedance relationships become algebraic, this chapter also shows that a matrix representation of large grids like \eqref{eq:multimachine_admittance} is also possible in the Dynamic Phasor domain without and approximations or QSH.

	Further, chapter \ref{chapter:control_theory} shows that from the Dynamic Phasor Transform a notion of a ``Laplace-like'' transform can be built, which is called the $\mu$ Transform or just ``$\mu$T''' for short. This transform can be used to build Dynamic Phasor Transfer Functions (DPFTs) and the elementary Control Theory in Dynamic Phasor space. In this chapter it is shown that very important control results are also carried to the Dynamic Phasor Space, like the fact that in $\mu$Ts, the system is \textit{Bounded Input Bounded Output} stable (sometimes called BIBO stability or input-output stability) if the roots of the denominator lie in the open left half complex plane. It is shown that tese results allow building more intuitive and better define control structures for systems under nonstationary regimens, like Power Systems.

	This chapter essentially proves that there can be controllers made specifically for systems in nonstationary regimens where controlling the phasor quantities does indeed reflect a control on the time domain, and guaranteedly so because the Dynamic Phasor Transform is biunivocal ans lossless. Thus, this chapter validades controllers like those of subsection \ref{subsec:power_system_control}; further, the chapter gives an example on how to design these controllers for linear systems — effectively solving the ``conceptual issue'' with phasor-domain controllers mentioned in subsection \ref{subsec:power_system_control}.

	Finally, part \ref{part:ending} finishes the thesis with some applications, discussion and conclusion. Chapter \ref{chapter:applications} shows three applications of the entire theory, discussed and developed in detail, showing how this theory can be used in Electric Power Systems and Electronic Circuits to produce phasorial models of these systems with relative ease and high resemblance to current techniques. Chapter \ref{chapter:discussion_conclusion} shows the discussion and conclusion, where some critic view of the theory presented is shown as well as the capabilities of the theory developed. Some further investigations are also discussed.

%-------------------------------------------------
\section{Associated papers} %<<<2
%-------------------------------------------------

	As of the writing of this thesis and its submission (june of 2025), several journal and conference papers were written, all of which were authored by me and Professor Luís:

\begin{itemize}
	\item ``Towards a New Dynamic Phasor Theory for Modeling IBG Penetrated Power Grids'', presented at the International Symposium on Circuit and Systems 2025 \pcite{volpatoNewDynamicPhasor2025};
	\item ``Dynamic Phasor and Nonstationary Power Theory as an extension of Classical Phasor Theory'', published in the Transactions on Circuits and Systems I \pcite{volpatoDynamicPhasorNonstationary2025};
	\item ``Dynamic Phasor Functionals for Modelling and Simulating Circuits and Systems in Nonstationary Sinusoidal Regimens'' submitted for publication \pcite{volpatoDynamicPhasorFunctionals2025};
	\item ``A Rigorous Approach to Quasistationary and Phasor-Equivalent Modelling of Power Systems'', manuscript \pcite{volpatoRigorousApproachQuasistationary2025};
	\item ``Effects of Apparent Frequency Choice in Dynamic Phasor Transformations'', manuscript \pcite{volpatoEffectsApparentFrequency2025};
	\item ``Representation of Dynamic Phasor Operators as Transfer Functions in Control Systems under Nonstationary Sinusoidal Regimens'', manuscript \pcite{volpatoRepresentationDynamicPhasor2025}.
\end{itemize}

	In the construction of this theory, two papers were published still during my master's degree dealing with some investigations which led to the development of this theory:

\begin{itemize}
	\item ``The Dynamic Phasor Transform Applied to Simulation and Control of Grid-Connected Inverters'', published in the Journal of Control, Automation and Electrical Systems \pcite{volpatoDynamicPhasorTransform2022};
	\item ``Grid-connected Inverters per-unit Dynamic Phasor Modelling, Simulation And Control'', presented at the VIII Brazilian Simposium on Electrical Systems SBSE \pcite{A.Volpato2021}.
\end{itemize}


\part{Linear Systems and Classical Phasor Theory}\label{part:linearsys_phasor_theory}

\chapter{Theory of Linear Dynamical Systems}\label{chapter:linear_systems}

%-------------------------------------------------
\section{Introduction} %<<<1

%-------------------------------------------------
\subsection{Objectives} %<<<2

	The objective of this chapter is to develop a theory on Linear Dynamical Systems, and particularly, to show that these systems are ``inherently exponential''. Naturally, such theory is highly dependent on Linear Algebra and simple stability definitions, hence why the chapter starts as a Linear Algebra chapter to evolve into Dynamical Systems. This is primarily used as the cornerstone to classical phasors, specifically to show that the differential equations that model Passive Linear Circuits, namely Linear Time Invariant Differential Equations, follow very specific patterns that allow drawing the simple, yet difficult to prove, characteristic that in such circuits the homogeneous solution vanishes exponentially as time grows. This is used in the inception of Classical Phasor Theory, by showing that a PLC when excited by sinusoids experiences sinusoidal responses (voltages and currents) in steady-state because the homogeneous transients fade exponentially.

	The sequence of these facts, however, is not simple to prove. There is a lot of theory regarding Linear Systems and Algebra that need to be undertaken to arrive at the conclusions needed. As such, the objective of this chapter, in a more lengthened explanation, is to introduce the theory of linear systems needed to model and understand Passive Linear Circuits, in a sequence of theorems and developments that, to the best of the author's knowledge, is not found in the literature of Electrical Engineering with the objective of exploring the theory specifically for electrical circuits.

	Furthermore, several constructions and results from this chapter are used in the text. For instance, the definitions of a field and vector space are used in chapter \ref{chapter:dpos} to show that Dynamic Phasor Functionals (the Dynamic Phasor equivalent of derivatives in time domain) form a vector space and a field; also, the same chapter uses the inception of matrices as representations of linear mappings with respect to a particular basis in the definitions of matrices of these Dynamic Phasor Functionals, so that an admittance matrix representation of circuits is possible under nonstationary regimens.

%-------------------------------------------------
\subsection{Notation} %<<<

	Due to the the mathematical-theoretic nature of this thesis, mathematical rigour is needed and with it comes the mathematical notation. The notation used in this thesis is derived from \cite{lamportTLA2002} and \cite{Perko2001}, and is explained as follows.

	The set of natural numbers is denoted $\mathbb{N}$, while the set of integers is $\mathbb{Z}$, the set of real numbers is $\mathbb{R}$ and the complex numbers are $\mathbb{C}$. Complex conjugation is denoted as the overline $\overline{z}$. As in Electrical Engineering the letter ``i'' is generally used for current, we denote the imaginary number that satisfies $x^2 + 1 = 0$ in the complex domain as $j$, and its opposite-conjugate as $\overline{j} = -j$. We consider that zero is a part of the naturals, and $\mathbb{N}_k$ (for a natural $k$) denotes the naturals up to $k$, that is, the set $\left\{0,1,2,...,k\right\}$; conversely, $\mathbb{Z}_k$ represents the integers between and including $-k$ and $k$, that is, $\left\{-k,-(k-1),...,-1,0,1,2,...,k-1,k\right\}$. The sets with superscripts or subscripts represent specific cases: the asterisk $\mathbb{N}^*$ represents a version without zero, the plus sign $\mathbb{R}_+$ represents the non-negative elements (zero is included) and $\mathbb{R}_-$ the non-positive elements.

	Numbers are denoted in simple letters, such as $t$, $x$, and so on. Vectors and matrices are denoted in bold letters: lowercase $\mathbf{x}$ denotes a vector and uppercase $\mathbf{A}$ denotes a matrix. The set of vectors of n dimensions of a particular set $X$ is denoted with a power notation, that is, $X^n$, while the matrices of size n-by-m of $X$ are denoted $X^{(n\times m)}$.  For real and complex numbers, the \textbf{amplitude} or \textbf{absolute value} is denoted with simple vertical brackets as in $\left\lvert a + jb \right\rvert = \sqrt{a^2 + b^2}$. For vectors and matrices the \textbf{norms} are denotes using double vertical bars, as in $\left\lVert \mathbf{x}\right\rVert$ and $\left\lVert \mathbf{A}\right\rVert$. Unless specifically noted, the vector and matrix norms are the $p$-norm for an arbitrary $p$. This will be defined thoroughly in the text.

	Matrix simple transposition is denoted with a stylized ``T'', as in, $\mathbf{A}^\transpose$, and the hermitian transpose (conjugate transpose) is denoted with a stylized ``H'' as in $\mathbf{A}^\hermconj$
.

	An \textbf{application}, or mapping, is a set-theoretic relationship that maps elements from a domain $D$ (denoted $D = \Dom\left(f\right)$) and assumes values in a certain other set $T$, denoted in simple letters with a bracket as in $f\left[x\right]$. The collection of all possible outputs of $f$ is called the image of $f$, denoted $\Im\left(f\right)$, defined as

\begin{equation} \Im\left(f\right) = \left\{ f\left[x\right]: x\in\Dom\left(f\right)\right\}. \label{eq:def_image}\end{equation}

	If the argument $x$ does not belong to $D$ then $f$ applied to $x$ is unspecified. A \textbf{function} is a mapping such that $f\left[x\right]$ is unique to $x$ (that is, a certain $x$ can only have one $f\left[x\right]$) while the converse is not necessarily true; if it is (that is, each $x$ maps uniquely to $f\left[x\right]$) then $f$ is called \textit{injective}. Naturally, $\Im\left(f\right)\subset T$; if equality holds, $f$ is called \textit{surjective}. A function that is both inective and surjective is such that an element of $\Dom\left(f\right)$ is biunivocally related to another element (``one-to-one'') in the image, and this image spans the entire space in which $f$ takes values; such a function is called \textit{bijective}. Many equivalent definitions for injections, surjections and bijections exist, and any book on analysis will present a definition that suits its text.

	 In most cases it is convenient to associate to $f$ some expression $e(x)$ such that $f\left[x\right] = e(x)$ (note that the brackets denote a function while the parenthesis denote a syntax). In this case, $f$ is defined as $f\coloneq \left[x\in D\mapsto e(x)\right]$, the symbol ``$\coloneq$'' denoting a definition. In this definition, the target set $T$ is supposed to be the largest one where the expression $e(x)$ takes values on. It might be interesting to state and define $D$, $T$ and $e(x)$ clearly; in this case, the longer notation

\begin{equation} f: \left\{\begin{array}{rcl} D &\to& T \\[3mm] x &\mapsto& e(x) \end{array}\right. \end{equation}

	\noindent is used. The notation $\left[D\to T\right]$ denotes the \textbf{set of all functions} from the set $D$ that take value in the set $T$, such that $f\in\left[D\to T\right]$ reads ``$f$ is a function with domain $D$ that takes values in a set $T$''. Particularly, functions of real numbers are called \textbf{signals}, that is, $f\in\left[\mathbb{R}\to X\right]$ is a signal onto the set $X$; $\left[\mathbb{R}\to\mathbb{R}\right]$ are the \textbf{real signals} and $\left[\mathbb{R}\to\mathbb{C}\right]$ are the \textbf{complex signals}. In general, signals are called so to represent quantities that vary in time, that is, most signals are functions of time.

	A function is called a \textbf{transform} if it is a self-map, that is, takes elements from a domain $X$ into $X$ itself. Indeed, the usual functional transforms (Fourier, Laplace, Hilbert and so on) transform complex signals into complex signals.

	A \textbf{sequence} is a function from either the naturals or the integers. The notation $X^{\left[\mathbb{N}\right]}$ denotes a natural sequence in a set $X$, that is, an ennumerated collection $A = \left(a_0,a_1,a_2,...\right)$ such that all $a_k\in X$. This can be shortly denoted as $A = \left(a_k\right)_{k=0}^\infty$. In the same way, $X^{\left[\mathbb{Z}\right]}$ denotes an integer sequence in $X$, that is, an ordered set $A = \left(...,a_{(-2)},a_{(-1)},a_0,a_1,a_2,a_3,...\right)$ with all $a_k\in X$, which can be deonted in short as $A = \left(a_k\right)_{k\in\mathbb{Z}}$. In some cases it might be interesting to have finite sequences — also called \textbf{tuples} — say, from index 0 to a finite index $m$, which can be denoted $\left(a_k\right)_{k=0}^m$. It must be noted that sequences, by being ennumerated collections, define unique relations with respect to their elements, that is, a sequence $A$ is only equal to a sequence $B$ if $a_k = b_k$ for all indexes $k$.

	\textbf{Operators and functionals} are maps defined in spaces of functions, that is, ``functions of functions'', and are denoted with brackets and in bold letters. \textbf{Operators} are denoted in lowercase bold letters with brackets and transform functions into numbers; for instance, the norm of a function in the Banach $L^2$ space is defined as

\begin{equation} \mathbf{n}_2\left[\cdot\right] : \left\{\begin{array}{rcl} L^2\left(\mathbb{R}\right) &\to& \mathbb{R}^+ \\[3mm] f(x) &\mapsto& \sqrt{\raisebox{0mm}[7mm][6mm]{} \displaystyle\int_{-\infty}^{\infty} f(x)^2 dx}\end{array}\right.  .\end{equation}

	Sometimes, however, some special operators have specific notations; for instance, norms are generally noted in vertical brackets like $\left\lVert f \right\rVert_2$. \textbf{Functionals}, on the other hand, deliver functions into functions, and are denoted in uppercase bold letters; for instance, the Laplace Transform of a function $f$ is denoted $\mathbf{L}\left[f\right]$, the Fourier Transform $\mathbf{F}\left[f\right]$, the Hilbert Transform $\mathbf{H}\left[f\right]$ and so on. It is on purpose that both matrices and functionals are denoted in uppercase bold, for the multiplication of a matrix $\mathbf{A}$ and a vector function $\mathbf{x}(t)$ is a linear functional $\mathbf{A}\left[\mathbf{x}\right]$, while any linear functional can be expressed as a matrix given some basis.

	It must be noted that the distinction between operators, functionals and transforms made here is not standardized in mathematics and may vary among the literature and authors.

	Derivatives are noted with apostrophes, as in, $f'(x)$ represents the first derivative, $f''(x)$ represents the second, $f'''(t)$ represets the third and so on. Particularly for time derivatives, the over-dot notation $\dot{f}(t), \ddot{f}(t), \dddot{f}(t)$ is used. For higher-order or generic-order derivatives, the exponent with parenthesis $f^{(n)}$ is used. Differentials, on the other hand, are denoted in two ways: either as functionals or in the ``fraction-like'' Leibnitz notation. For the first notation, the usual $d$ is used for single-variable functions:

\begin{equation} \dfrac{df(x)}{dx} \text{, for the first derivative and } \dfrac{d^nf(x)}{dx^n} \text{ for the n-th order} .\end{equation}

%	\noindent while the \textit{del} $\partial$ is used for multivariate functions
%
%\begin{equation} \begin{array}{l} \dfrac{\partial f(\mathbf{x})}{\partial x_k} \text{ for the first derivative with respect to the k-th coordinate} \\[5mm] \dfrac{\partial^n f(\mathbf{x})}{\partial x_k^n} \text{ for the n-th derivative with respect to the k-th coordinate, and } \\[5mm] \dfrac{\partial^n f(\mathbf{x})}{\partial x_1\partial x_2 \cdots \partial x_k} \text{ for mixed or sequential derivatives} \end{array} \end{equation}

	For normed vector spaces, and particularly Banach Spaces such as functional spaces, the delta $\delta$ is used to denote the Frechèt derivative as in

\begin{equation} \left.\dfrac{\delta \mathbf{F}\left[x\right]}{\delta x}\right\rvert_{x=x_0} \end{equation}

	\noindent denotes the derivative of the functional $F$ with respect to $x$ calculated at $x_0$. This derivative is defined as the bounded linear map $\mathbf{A}$ that satisfies

\begin{equation} \mathbf{A} = \left.\dfrac{\delta \mathbf{F}\left[x\right]}{\delta x}\right\rvert_{x=x_0} \Leftrightarrow \lim\limits_{\left\lVert \Delta x\right\rVert\to 0} \dfrac{\left\lVert \mathbf{F}\left[x_0 + \Delta x\right] - \mathbf{F}\left[x_0\right] - \mathbf{A}\left[x_0\right] \Delta x \right\rVert_W}{\left\lVert \Delta x\right\rVert_V} = 0 \label{eq:def_frechet}\end{equation}

	\noindent called the Frechèt Derivative \pcite{gelfandCalculusVariations1963}, where $\mathbf{A}\left[x\right] \Delta x$ is $\mathbf{A}$ calculated at the operating point $x_0$ applied onto $\Delta x$. It is obvious that $\mathbf{A}$ depends on the point $x_0$ it is calculated against; yet, the notation $\mathbf{A}\left[x_0\right]$ denotes the operator \textit{calculated at} $x_0$ and not operating on it, which can become confusing. Thus, the shorter notation $\mathbf{A}$ will be used when $x_0$ is understood while the fraction notation will be used to highlight the operating element. It is also obvious that this definition depends on the specific norms adopted for the domain space $\left\lVert\cdot\right\rVert_V$ and the image space $\left\lVert\cdot\right\rVert_W$; in most cases these norms are tacitly understood.

	For the differential operators notation, the small caps bold $\mathbf{d}$ is used. For sequential differentiation, $\mathbf{d}^k$ denotes the k-th order differential operator. Because the differentiation operation varies in definition among the many sets involved, generally the functional will be noted with a subscript. For instance, the differentiation of real signals is the operator that takes a function and an operating point and delivers the derivative of that function at that point

\begin{equation}\mathbf{d}_\mathbb{R}: \left\{\begin{array}{rcl} \mathbb{R}\times\left[\mathbb{R}\to\mathbb{R}\right] &\to& \mathbb{R} \\[3mm] \left(x_0,f(x)\right) &\mapsto& \left.\dfrac{df(x)}{dx}\right\rvert_{x=x_0} \end{array}\right.\end{equation}

	\noindent and when the operating point $x_0$ is understood, the shorter notation $f'(t)$ is used. At the same time, the differentiation of complex signals is

\begin{equation}\mathbf{d}_\mathbb{C}: \left\{\begin{array}{ccc} \mathbb{R}\times\left[\mathbb{R} \to\mathbb{C}\right] &\to& \mathbb{C} \\[3mm] \left(x_0,u(x) + jv(x)\right) &\mapsto& \mathbf{d}_\mathbb{R}\left[x_0,u\right] + j \mathbf{d}_\mathbb{R}\left[x_0,v\right]\end{array}\right.\end{equation}

	\noindent and so on. From these definitions one can define differential functionals, denoted in uppercase bold, where the differential operators are evaluated continuously, as in

\begin{equation}\mathbf{D}_\mathbb{R}: \left\{\begin{array}{rcl} \left[\mathbb{R}\to\mathbb{R}\right] &\to& \left[\mathbb{R}\to\mathbb{R}\right] \\[3mm] f(x) &\mapsto& f'(t) = \mathbf{d}_\mathbb{R}\left[t,f\right] \end{array}\right.\end{equation}

	\noindent and, conversely,

\begin{equation}\mathbf{D}_\mathbb{C}: \left\{\begin{array}{ccc} \left[\mathbb{R} \to\mathbb{C}\right] &\to& \left[\mathbb{R}\to\mathbb{C}\right] \\[3mm] f(t) = u(x) + jv(x) &\mapsto& f'(t) = \mathbf{d}_\mathbb{R}\left[t,u\right] + j \mathbf{d}_\mathbb{R}\left[t,v\right]\end{array}\right.\end{equation}

	Finally, given a function $f\in\left[X\to Y\right]$ and the differential operator, $f$ is said to be \textbf{class n smooth at} $x_0$ if $\mathbf{d}^n\left[x_0,f\right]$ exists. Conversely, defining a functional $\mathbf{D}_X$ of the space $X$, then $f$ is said to be \textbf{class n smooth} if $\mathbf{D}^n\left[f\right]$ exists in $X$. The set of class n smooth functions in $X$ is denoted $C_X^n$ (or simply $C^n$ when $X$ and $\mathbf{D}_X$ are tacitly understood), so that $f\in C^n$ reads ``$f$ is class $n$ smooth''.

%-------------------------------------------------
\section{Linear Circuits as Linear Functionals} %<<<1

	The theory of linear operators is generally defined in terms of vector fields and scalars, which is explained swiftly below.

\begin{definition}[Field]\label{def:field}
	A \textbf{field} is a set $F$ wherein two binary operations are defined: the sum ``$+$'' and multiplication ``$\cdot$'', following certain rules:

\begin{itemize}
	\item \textbf{Associativity}: for three $a,b,c$ in $F$, $a+(b+c) = (a+b)+c$ and $a\cdot (b\cdot c) = (a\cdot b) \cdot c$;
	\item \textbf{Commutativity}: for any $a,b$ in $F$, $a+b = b+a$ and $a\cdot b = b\cdot a$;
	\item \textbf{Identities}: there exist two elements in $F$, 0 and 1, such that $a + 0 = a$ and $a\cdot 1 = a$ for all $a\in F$;
	\item \textbf{Additive inverse}: for any $a\in F$, there exists the additive inverse $-a$ such that $a + (-a) = 0$ ;
	\item \textbf{Multiplicative inverse}: for any $a\in F$ except $0$ there exists the multiplicative inverse $a^{-1}$ such that $a\cdot a^{-1} = 1$ ;
	\item \textbf{Distributivity} of multiplication over sum: for $a,b,c\in F$, $a\cdot (b + c) = a\cdot b + a\cdot c$.
\end{itemize}
\end{definition}

	Classically, in linear algebra the field of reals and complex numbers is used; it is simple to prove that the usual complex sum and multiplication adhere to the properties of fields. Concurrently, one can define the notion of a vector space.

\begin{definition}[Vector space]\label{def:vector_space}%<<<
	A \textbf{vector space} $V$ over a field $F$ is a set wherein two binary operations can be defined: vector sum, or simply sum, and scalar multiplication. In this thesis, vectors are denoted in boldface to distinguish them from the elements of the field $F$, called scalars. These operations again have certain properties

\begin{itemize}
	\item \textbf{Associativity}: for three $\mathbf{u,v,w}$ in $V$, $\mathbf{u+(v+w)} = \mathbf{(u+v)+w}$;
	\item \textbf{Commutativity}: for any $\mathbf{u,v}$ in $V$, $\mathbf{u+v=v+u}$;
	\item \textbf{Identity of vector sum}: there exists an element called the zero vector $\mathbf{0}$ in $V$ such that $\mathbf{v+0 = 0}$ for all $\mathbf{v}\in V$;
	\item \textbf{Additive inverse}: for any $\mathbf{v}\in V$, there exists the additive inverse $-\mathbf{v}$ such that $\mathbf{v + (-v) = 0}$ ;
	\item \textbf{Scalar multiplication compatibility}: for any two scalars $a,b\in F$ and any vector $\mathbf{v}\in V$, $a(b\mathbf{v}) = (ab)\mathbf{v}$ ;
	\item \textbf{Identity element of scalar multiplication}: there exists an element $1\in F$ such that $1\cdot \mathbf{v} = \mathbf{v}$ for any $\mathbf{v}\in V$ ;
	\item \textbf{Distributivity}: for any two scalars $a,b$ and two vectors $\mathbf{v,u}$, $a\mathbf{v} + b\mathbf{v} = (a+b)\mathbf{v}$ and $a(\mathbf{v+u}) = a\mathbf{v} + b\mathbf{u}$.
\end{itemize}
\end{definition}

	Although the definitions seem somewhat prolix or tedious, the reach of these definitions is immense. In general, the most common vector space is the space of complex vectors of length $n$, denoted $\mathbb{C}^n$, that is, numbers of the form $\mathbf{z} = \left[z_1,z_2,...,z_n\right]^\transpose$ where all $z_k$ are complex numbers, or particularly, the space of real vectors of length $n$. It is immediate to notice that $\mathbb{C}^n$ and $\mathbb{R}^n$ are vector fields over their unidimensional counterparts, and a introductory level in linear algebra will deal in such realms.

	It is however, less obvious to notice that the space $\left[\mathbb{R}\to\mathbb{C}^n\right]$ of complex vector functions of one real variable of length $n$ a vector space over the field of complex numbers. Indeed, by adopting the vector sum operation as

\begin{equation} (+): \left\{\begin{array}{rcl} \left[\mathbb{R}\to\mathbb{C}^n\right]^2 &\to& \left[\mathbb{R}\to\mathbb{C}^n\right] \\[5mm] \left(\left[\begin{array}{c} x_1\left(t\right) \\[3mm] x_2(t) \\[3mm] \vdots \\[3mm] x_n(t) \end{array}\right],\left[\begin{array}{c} y_1\left(t\right) \\[3mm] y_2(t) \\[3mm] \vdots \\[3mm] y_n(t) \end{array}\right]\right) &\mapsto& \left[\begin{array}{c} x_1\left(t\right) + y_1\left(t\right) \\[3mm] x_2\left(t\right) + y_2\left(t\right) \\[3mm] \vdots \\[3mm] x_n\left(t\right) + y_n\left(t\right) \end{array}\right] \end{array}\right. \end{equation}

	\noindent and the scalar multiplication as

\begin{equation} (\cdot): \left\{\begin{array}{rcl} \mathbb{C}\times\left[\mathbb{R}\to\mathbb{C}^n\right] &\to& \left[\mathbb{R}\to\mathbb{C}^n\right] \\[5mm] \left(z,\left[\begin{array}{c} x_1\left(t\right) \\[3mm] x_2(t) \\[3mm] \vdots \\[3mm] x_n(t) \end{array}\right]\right) &\mapsto& \left[\begin{array}{c} zx_1\left(t\right) \\[3mm] zx_2\left(t\right) \\[3mm] \vdots \\[3mm] zx_n\left(t\right) \end{array}\right] \end{array}\right. \end{equation}

	\noindent then one can prove that these operations fulfill of the definitions of a vector space over a field. A function $\alpha$ defined from a vector space to another vector space maintaining the linear structure is called a \textbf{linear map}. More precisely, for two vector spaces $V$ and $U$, $\alpha \in\left[V\to U\right]$ is a linear operator if $\alpha\left(\mathbf{x} + \mathbf{y}\right) = \alpha\mathbf{x} + \alpha\mathbf{y}$  for any two $\mathbf{x,y}\in V$ and $\alpha\left(z\mathbf{x}\right) = z\alpha\left(\mathbf{x}\right)$ for any scalar $z$ and any vector $\mathbf{x}\in V$.  In this thesis, the two vector spaces of interest are either $\left[\mathbb{R}\to\mathbb{C}^n\right]$ or $\mathbb{C}^n$, both over $\mathbb{C}$. 

	For the purposes of electrical circuits analysis and electrical network theory, linear circuits — circuits composed of resistances, inductances and capacitances — are generally modelled as differential equations of the form

\begin{equation} \dot{\mathbf{x}}(t) = \mathbf{A}\left[\mathbf{x}\right](t) + \mathbf{f}(t), \label{eq:matrix_lti_ode_eq_def} \end{equation}

	\noindent where $\mathbf{A}\left[\mathbf{x}\right]$ is a transform in the space $\left[\mathbb{R}\to\mathbb{C}^n\right]$. Additionally, $\mathbf{A}\left[\cdot\right]$ is \textit{linear}, that is,

\begin{equation} \mathbf{A}\left[\mathbf{x} + \alpha\mathbf{y}\right] = \mathbf{A}\left[\mathbf{x}\right] + \alpha\mathbf{A}\left[\mathbf{y}\right] \end{equation}

	\noindent for any $\mathbf{x,y}\in\left[\mathbb{R}\to\mathbb{C}^n\right]$ and $\alpha$ a scalar.

	The fact that a linear circuit can indeed be modelled as an equation of the form \eqref{eq:matrix_lti_ode_eq_def} will be shown in section \ref{chapter:classical_phasors}, hence this fact is for now assumed. In such circuits, the coefficients $a_{ik}$ are certain combinations of the $RLC$ parameters of the circuit derived by manipulating and modelling the circuit using Kirchoff's Laws.

	As for nomenclature, $\mathbf{x}(t)\in\left[\mathbb{R}\to\mathbb{C}^n\right]$ is the system \textbf{output} or \textbf{response}, and $\mathbf{f}(t)\in\left[\mathbb{R}\to\mathbb{C}^n\right]$ is some \textbf{forcing, excitation} or \textbf{input}. An equation in the form of \eqref{eq:matrix_lti_ode_eq_def} is described as the \textbf{state-space equation} of the system, where the components of $\mathbf{x}$ are called \textbf{states}. Loosely, states are certain quantities within the system such that \textit{their collection suffices to describe the system completely}, as in, any other quantity can be obtained from the states. For linear circuits, it will be proven in section \ref{chapter:classical_phasors} that inductor currents and capacitor voltages are states of the system for any node voltage or branch current is obtainable from this list. More precisely, suppose that the system has $n$ nodes and $b$ branches and adopt

\begin{equation} \mathbf{v} = \left[ \overbrace{v_1,v_2,...,v_n}^{\text{Node voltages}}\right],\mathbf{i} = \left[ \overbrace{i_1, i_2 , ... i_b}^{\text{Branch currents}}\right]^\transpose .\end{equation}

	Now suppose the circuit has $q$ capacitors and $p$ inductors. Then the sate vector is

\begin{equation} \mathbf{x} = \left[ \overbrace{v_{C_1},v_{C_2},...,v_{C_q}}^{\text{Capacitor voltages}}, \overbrace{i_{L_1}, i_{L_2} , ... i_{L_p}}^{\text{Inductor currents}}\right]^\transpose .\end{equation}

	\noindent then $\mathbf{v}$ and $\mathbf{i}$ can be obtained as some linear combinations of $\dot{\mathbf{x}}$ and $\mathbf{x}$ . The converse is also true as a direct consequence of Kirchoff's Laws.

%	And there exist two matrices $\mathbf{N,M}$ such that 
%
%\begin{equation} \left[\begin{array}{c} \mathbf{v} \\[3mm] \mathbf{i} \end{array}\right] = \mathbf{N}\dot{\mathbf{x}} + \mathbf{Mx} . \end{equation}
%	
%	The converse is also true as a direct consequence of Kirchoff's Laws, that is, there exists a matrix $\mathbf{P}$ such that 

%\begin{equation} \mathbf{x} = \mathbf{P} \left[\begin{array}{c} \mathbf{v} \\[3mm] \mathbf{i} \end{array}\right] . \end{equation}


\begin{example}\label{example:rlc_circuit_lti_ode_example} % EXAMPLE OF LTI ODE CIRCUIT MODELLING <<<
	Consider the figure \ref{fig:ltiode_modelling_example_circuit} where an RLC circuit is shown. This circuit has an excitation $u(t)$, given by a controlled voltage source, and an input $v(t)$, given by the voltage across the resistor load $R$. The circuit has two nodes and two loops are shown, a red and a green one.

% MODELLING EXAMPLE: RLC CIRCUIT <<<
\begin{figure}[htb!]
\centering
\scalebox{0.75}{
        \begin{tikzpicture}[american,scale=1.2,transform shape,line width=0.75, cute inductors]
		\draw (0,0)
			to[vsource,sources/scale=1.25, v>=$u(t)$,invert] (0,4)
			to[L,l=$L$,f>^=$i_L$,v>=$v_L(t)$,-*] (4,4) 
			to[C, l_=$C$,f>_=$i_C$, v^=$v_C(t)$, voltage shift = 0.5mm] (4,0)
			to[short] (0,0); 
		\node[shape=circle,draw,inner sep=1pt] at (4,4.5) {$1$};
		\draw (4,4) to[short] (5,4) to[short] (8,4) to[R,l_=$R$,f>_=$i_R$, v^=$v(t)$,voltage shift = 1mm] (8,0) to[short] (4,0);
		\draw[rounded corners=10,loop, draw opacity=0.3,->,color=red] (0.5,0.5) -- (0.5,3.5) -- (3.5,3.5) -- (3.5,0.5) -- (1,0.5) ;
		\draw[rounded corners=10,loop, draw opacity=0.3,->, color=blue] (4.5,0.5) -- (4.5,3.5) -- (7.5,3.5) -- (7.5,0.5) -- (5,0.5) ;
        \end{tikzpicture}
}
	\caption{RLC circuit as modelling example for linear circuit as an LTI ODE.}
	\label{fig:ltiode_modelling_example_circuit}
\end{figure} %>>>

	First, apply the KVL to the red loop and blue loops to yield

\begin{equation}
	\left\{\begin{array}{l}
		-u(t) + v_L(t) + v_C(t) = 0 \\[3mm]
		-v_C(t) + v(t) = 0
	\end{array}\right. \label{fig:ltiode_modelling_example_circuit_kvl}
\end{equation}

	Then apply the KCL to the node $1$:

\begin{equation} i_L(t) - i_C(t) - i_R(t) = 0 \label{fig:ltiode_modelling_example_circuit_kcl} \end{equation}

	Therefore, equations \eqref{fig:ltiode_modelling_example_circuit_kvl} and \eqref{fig:ltiode_modelling_example_circuit_kcl} form a three-equation system with six states $v_C,v_L,v,i_R,i_C,i_L$. The remaning three equations come from the equations of the circuit elements:

\begin{equation}
	\left\{\begin{array}{l}
		i_R(t) = Rv(t) \\[3mm]
		i_C(t) = C \dfrac{dv_C(t)}{dt} \\[3mm]
		v_L(t) = L\dfrac{di_L(t)}{dt}
	\end{array}\right. \label{fig:ltiode_modelling_example_circuit_components}
\end{equation}

	Now let

\begin{equation} \mathbf{x} = \left[\begin{array}{c} v_C \\[3mm] i_L \end{array}\right] \end{equation}

	Then the first equation of \eqref{fig:ltiode_modelling_example_circuit_kvl} and \eqref{fig:ltiode_modelling_example_circuit_kcl} yield

\begin{equation} \left\{ \begin{array}{l} \dot{v}_C = \dfrac{1}{C} \left(i_L - \dfrac{v_C}{R}\right) = -\dfrac{1}{RC}v_c + \dfrac{1}{C}i_L \\[5mm] \dot{i}_L = \dfrac{1}{L} \left(u - v_C\right) = - \dfrac{1}{L}v_C + \dfrac{1}{L}u \end{array}\right.\end{equation}

	\noindent meaning this circuit is such that

\begin{equation} \mathbf{A}\left[ \begin{array}{l} v_C \\[3mm] i_L \end{array}\right] = \left[\begin{array}{c} -\dfrac{1}{RC}v_c + \dfrac{1}{C}i_L \\[5mm] - \dfrac{1}{L}v_C \end{array}\right]\end{equation}

	\noindent which is clearly linear, thus this circuit defines an equation like \eqref{eq:matrix_lti_ode_eq_def}.

\examplebar
\end{example} %>>>

	In general, the objective of discussions of differential equations in applied sciences is to \textbf{solve} the equation, that is, find a \textbf{solution} to \eqref{eq:matrix_lti_ode_eq_def} — a particular signal $\mathbf{x}(t)$ that satisfies the equation given a set of initial conditions $\mathbf{x}\left(t_0\right),\mathbf{x}'\left(t_0\right),...,\mathbf{x}^{(n-1)}\left(t_0\right)$ for some time $t_0$. Such a process is called \textbf{solving} or \textbf{integrating} the differential equation. In most cases, a introductory Differential Equations course is concerned with teaching algorithms and techniques to solve differential equations analytically; in this text, it is assumed the reader is acquainted with such processes. Here we are more interested with the fact that, in a deeper sense, solving \eqref{eq:matrix_lti_ode_eq_def} is equivalent to finding the root of the functional equation

\begin{equation} \mathbf{D}_{\mathbb{C}^n}\left[\mathbf{x}\right] - \mathbf{A}\left[\mathbf{x}\right] = \mathbf{f}(t),  \label{eq:equivalent_lti_ode_operator} \end{equation}

	\noindent where $\mathbf{D}_{\mathbb{C}^n}$ denotes the differential functional in $\mathbb{C}^n$.  Ideally, one can find the set of all solutions, that is, an expression that defines the entire class of functions that solve the equation. Such expression is known as the \textbf{general solution}. 

% which will be described in the form of a n-th order differential equation
%
%\begin{equation} \sum\limits_{k=0}^n \alpha_k x^{\left(k\right)} - {f}(t) = 0, \label{eq:vector_lti_ode_eq_def} \end{equation}
%
%	\noindent where the $\alpha_k$ are complex numbers with $\alpha_n\neq 0$, $x\in\left[\mathbb{R}\to\mathbb{C}\right]$ and ${f}(t)\in\left[\mathbb{R}\to\mathbb{C}\right]$ output and input respectively. The difference between the matrix equation \eqref{eq:matrix_lti_ode_eq_def} and the n-th order equation \eqref{eq:vector_lti_ode_eq_def} is a practical one, because mathematically, \eqref{eq:vector_lti_ode_eq_def} are special cases of matrix ODE where a vector form is equivalent to a space-state equation on $n$ states. Indeed, write
%
%\begin{equation} \mathbf{x} = \left[\begin{array}{c} x(t) \\[3mm] x^{(1)}(t)\\[3mm] x^{(2)}(t) \\[3mm] \vdots \\[3mm] x^{(n-1)} (t) \end{array}\right] \end{equation}
%
%	\noindent and \eqref{eq:vector_lti_ode_eq_def} is equivalent to the matrix LTI ODE
%
%\begin{equation} \dot{\mathbf{x}} = \mathbf{A}\mathbf{x} \label{eq:theo_matrix_equivalence_matrix_ode} \end{equation}
%
%	\noindent where
%
%\begin{equation} \mathbf{A} = \left[\begin{array}{ccccc} 0 & 1 & 0 & ... & 0 \\[3mm] 0 & 0 & 1 & ... & 0  \\[3mm] \vdots & \vdots & \vdots & \ddots & \vdots \\[3mm] 0 & 0 & 0 & ... & 1 \\[3mm] -\dfrac{\alpha_0}{\alpha_n} & -\dfrac{\alpha_1}{\alpha_n} & -\dfrac{\alpha_2}{\alpha_n} & ... & -\dfrac{\alpha_{(n-1)}}{\alpha_n} \end{array}\right] \label{eq:equivalent_frobenius_matrix}\end{equation}
%
%	For engineers, however, the n-th order differential form is easier to work with because they are also somewhat easier to solve: it will be shown that its solutions can be found through the polynomial
%
%\begin{equation} H(x) = \sum\limits_{k=0}^n \alpha_k x^k \end{equation}
%
%	\noindent which is easily obtained by substituting the derivative orders of the original ODE \eqref{eq:vector_lti_ode_eq_def} into exponents. Albeit both forms being somewhat equivalent, however, the state-space form is easier to work with in a proof context because its qualitative and quantitative behavior can be drawn as direct consequences of the matrix $\mathbf{A}$ and its characteristics, specifically its eigenvectors and eigenvalues. Therefore, in what follows, the theorems and proofs will be shown for the state-space form \eqref{eq:matrix_lti_ode_eq_def} and then particularized for the n-dimensional case later

	The nomenclature of ``linear`` and ``time invariant`` for \eqref{eq:matrix_lti_ode_eq_def} comes from the ability to combine inputs and predict the outputs basisd on the individual response of each input. Let 

\begin{equation} \mathbf{G}\left[\mathbf{f}\right] = \mathbf{x}(t) , \label{eq:lti_system_def} \end{equation}

	\noindent be a shorthand notation (generally called ``input-output notation'') for the differential equation \eqref{eq:matrix_lti_ode_eq_def}, and reads ``the system $\mathbf{G}$ maps the input $\mathbf{f}$ to the output $\mathbf{x}$''. Note that $\mathbf{G}\left[\cdot\right]$ relates a function $\mathbf{f}$ to another function $\mathbf{x}$, thence $\mathbf{G}\left[\cdot\right]$ is a functional. Exploring the properties of \eqref{eq:matrix_lti_ode_eq_def} we prove that $\mathbf{G}$ is \textbf{linear}, that is, a linear combination of inputs is equivalent to the same linear combination of responses:
	
\begin{equation} \mathbf{G}\left[\mathbf{f}_1 + \alpha\mathbf{f}_2\right] = \mathbf{G}\left[\mathbf{f}_1\right] + \alpha\mathbf{G}\left[\mathbf{f}_2\right],\ \alpha\in\mathbb{C} \label{eq:lti_scaling}.\end{equation}

	The map $\mathbf{G}$ is also \textbf{time invariant}, because a delay $\tau$ in the input causes a delay $\tau$ in the output:

\begin{equation} \mathbf{G}\left[\mathbf{f}\left(t - \tau\right)\right] = \mathbf{x}\left(t-\tau\right) \label{eq:lti_delay}\end{equation}

	These properties are easily drawn from the linearity of the operator $\mathbf{A}$ and the time invariancy of the derivative operator. Therefore, a state-space equation of the form $\dot{\mathbf{x}} = \mathbf{Ax} + \mathbf{f}\left(t\right)$ is called a \textbf{Linear Time Invariant} system, or LTI for short. Linearity and time invariancy are rather intuitive concepts: linear systems are those that follow the superposition principle — meaning superposing inputs is equivalent to superposing their respective outputs — whereas time invariant systems are ones that do not ``change in time'' or do not ``age'', that is, the input-to-output mapping does not change as time grows, such that the same input applied at a delay will yield the same output, only delayed by the same amount.

	With clarity in the best interest, and because this thesis is concerned only with systems of the form \eqref{eq:matrix_lti_ode_eq_def}, these LTI systems will be thenceforth called simply \textbf{linear systems}.
	
\begin{example}\label{example:rlc_circuit_lti_ode_example} % EXAMPLE OF LTI ODE CIRCUIT MODELLING <<<
	Consider again the figure \ref{fig:ltiode_modelling_example_circuit} and adopt  $L = 10$mH, $C = 1\mu$F and $R = 1$k$\Omega$. Suppose the system is subject to three different excitations: $u_1(t) = \theta\left(t\right)$, $u_2(t) = \theta\left(t-5ms\right)$ and $u_3(t) = 2\theta\left(t\right)$ with $\theta(t)$ the heaviside step function. That is, $u_2$ and $u_3$ are identical to $u_1$, except $u_2$ is a delayed version with a delay of 5 miliseconds and $u_3$ is scaled by a factor of two. Figure \ref{fig:delayed_example_rlc_ltiode} shows the system load voltage response $v(t)$ to all three excitations, and it is clear that the responses $v_2(t)$ and $v_3(t)$ are copies of $v_1$, albeit $v_2$ being delayed by the same 5 miliseconds and $v_3$ scaled by the same factor of two. This shows that, indeed, the circuit is linear and time-invariant system: a scaling of input caused the same scaling of output, and a delay in the input caused the same delay in the output.

% TIME INVARIANCY PLOT EXAMPLE <<<
\begin{figure}[htb]
\centering
	\leavevmode\beginpgfgraphicnamed{3harm_yhat}
                \begin{tikzpicture}
                        \begin{axis}[
                                width = 0.9\textwidth,
                                height = 0.9/1.618*\textwidth,
                                xlabel={$t$},
                                ylabel={$v(t)$},
                                xmin=0, xmax=0.015,
                                ymin=0, ymax=4,
                                xtick={0,0.001,...,0.015},
				xticklabel style={
				        /pgf/number format/fixed,
				        /pgf/number format/precision=3,
				},
				scaled x ticks={real:0.001},
                                ytick={0,0.5,...,4},
				samples=1000,
                                legend pos=north east,
                                ymajorgrids=true,
                                xmajorgrids=true,
                                every axis plot/.append style={thick, no marks},
                        ]    
                        \addplot[domain=0:0.015,thick,color=blue] {ceil(x)*( 1 - exp(-500*x)*( sin(10000*x*180/pi)/20 + cos(10000*x*180/pi)))};
                        \addplot[domain=0:0.015,thick,color=red ] {ceil(x-0.005)*( 1 - exp(-500*(x-0.005))*( sin(10000*(x-0.005)*180/pi)/20 + cos(10000*(x-0.005)*180/pi)))};
                        \addplot[domain=0:0.015,thick,color=green] {2*ceil(x)*( 1 - exp(-500*x)*( sin(10000*x*180/pi)/20 + cos(10000*x*180/pi)))};
			\legend{$v_1$,$v_2$,$v_3$}
                        \end{axis}
                \end{tikzpicture}
	\endpgfgraphicnamed
        \caption{Delayed response example of RLC circuit modelled.}
        \label{fig:delayed_example_rlc_ltiode}
\end{figure}
% >>>

\examplebar
\end{example} %>>>

%-------------------------------------------------
\section{Natural response of a LTI ODE} %<<<1

	The linearity and time invariance of Linear, Time Invariant Ordinary Differential Equations can be widely exploited to draw many properties; one of the many important aspects of LTI differential equations is that their response can be broken into two components: a ``natural'' component and a ``forced'' one. This fact plays a major role in the theory of Linear ODEs; particularly for this thesis, this fact is the main motivator for Phasor Theory as we want to show that, for a PLC, the homogeneous solution will always vanish in time, meaning that the particular solution will dominate over time.

\begin{theorem}[Homogeneous and particular solutions of LTI ODE]\label{theo:homogeneous_particular_ltiode} %<<<

	Consider the LTI ODE 

\begin{equation} \dot{\mathbf{x}} = \mathbf{A}\left[\mathbf{x}\right] + \mathbf{f}(t) . \label{eq:homo_theorem_homogeneous_ode_def_original}\end{equation}

	Let $\mathbf{X}_p$ be a known particular solution. Then the sum $\mathbf{x} = \mathbf{x_h} (t) + \mathbf{x_p}(t)$, where $\mathbf{x}_h$ a solution to the homogeneous (also called natural or non-forced) ODE

\begin{equation} \dot{\mathbf{x}}_\mathbf{h} = \mathbf{A}\left[\mathbf{x_h}\right] \label{eq:homo_theorem_homogeneous_ode_def},\end{equation}

	\noindent is also a solution to \eqref{eq:homo_theorem_homogeneous_ode_def_original}.

\end{theorem}
\noindent\textbf{Proof:} by definition the particular solution satisfies

\begin{equation} \dot{\mathbf{x}}_p = \mathbf{A}\left[\mathbf{x}_p\right] + \mathbf{f}(t) .\end{equation}

	Adopt $\mathbf{x}_h$ the solution to \eqref{eq:homo_theorem_homogeneous_ode_def} ; then

\begin{equation} \dot{\mathbf{x}}_p + \dot{\mathbf{x}}_h = \mathbf{A}\left[\mathbf{x}_p\right] + \mathbf{A}\left[\mathbf{x}_h\hphantom{x_P}\right] \mathbf{f}(t) .\end{equation}

	Using the linearity of the derivative and of $\mathbf{A}\left[\cdot\right]$,

\begin{equation} \dfrac{d}{dt} \left(\mathbf{x}_p + \mathbf{x}_h\right) = \mathbf{A}\left[\mathbf{x}_p + \mathbf{x}_h\right] + \mathbf{f}(t) .\end{equation}
\hfill$\blacksquare$
\vspace{5mm}
\hrule
\vspace{5mm} %>>>

	Another way of understanding theorem \ref{theo:homogeneous_particular_ltiode} is to write the excitation $\mathbf{f}$ can be written as $\mathbf{f} + \beta\mathbf{0}_n$, where $\mathbf{0}_n$ is the null vector of dimension $n$, for some scalar matrix $\beta\in\mathbb{C}^{\left(n\times m\right)}$; then

\begin{equation} \mathbf{G}\left[\mathbf{f} + \beta^\transpose\mathbf{0}\right] = \mathbf{G}\left[\mathbf{f}\right] + \beta^\transpose \mathbf{G}\left[\mathbf{0}\right] \end{equation}

	Therefore call $\mathbf{x}_h$ as 

\begin{equation} \mathbf{x}_h(t) = \mathbf{G}\left[\mathbf{0}\right]. \end{equation}

	\noindent that is, the response of the system with no forcing, or rather, ``natural'' response, and call

\begin{equation} \mathbf{x}_p(t) = \mathbf{G}\left[\mathbf{f}\right] \end{equation}

	\noindent as the excited or forced response. Then

\begin{equation} \mathbf{G}\left[\mathbf{f} + \beta^\transpose\mathbf{0}\right] = \mathbf{G}\left[\mathbf{f}\right] + \beta^\transpose\mathbf{G}\left[\mathbf{0}\right] = \mathbf{x}_p +\beta^\transpose\mathbf{x}_h , \end{equation}

	The beauty of this fact is that the general solution of a particular LTI ODE can be found through only two functions: a particular solution and the homogeneous solution. If these two are found, then for any solution $\mathbf{x}$ of this ODE there is a $\beta$ such that

\begin{equation} \mathbf{x}(t) = \mathbf{x}_p(t) + \beta^\transpose\mathbf{x}_h(t) \end{equation}

	\noindent and, because of the differential nature of the system, $\beta$ is only determined by the initial conditions:

\begin{equation} \mathbf{x}(0) = \mathbf{x}_p(0) + \beta^\transpose\mathbf{x}_h(0) \Leftrightarrow \left(\mathbf{x}(0) - \mathbf{x}_p(0)\right) = \beta^\transpose\mathbf{x}_h(0) \end{equation}

	The benefit of separating the response of an LTI ODE into natural and forced behaviors is that, as equation \eqref{eq:homo_theorem_homogeneous_ode_def} shows, the natural homogeneous response does not depend on the forcing because, by definition, it is calculated when the system is not forced. For this reason, $\mathbf{x}_h$ is also called the system \textit{natural response}, and $\mathbf{x}_p$ the system \textit{forced response}.

\begin{example}\label{example:rlc_circuit_natural_example} % EXAMPLE OF LTI ODE CIRCUIT MODELLING <<<
	Consider again the RLC circuit of figure \ref{fig:ltiode_modelling_example_circuit} in example \ref{example:rlc_circuit_lti_ode_example} where an RLC circuit is shown and which modelling is given by

\begin{equation} LC\dfrac{d^2v(t)}{dt^2} + \dfrac{L}{R}\dfrac{dv(t)}{dt} + v(t) - u(t) = 0 \end{equation}

	With $L = 10$mH, $C = 1\mu$F and $R = 1$k$\Omega$. Suppose that the system is excited by a cosine:

\begin{equation} LC\dfrac{d^2v(t)}{dt^2} + \dfrac{L}{R}\dfrac{dv(t)}{dt} + v(t) - M\cos\left(\omega t\right) = 0 \label{eq:example2_original_ode}\end{equation}

	\noindent then, solving the homogeneous ODE yields the natural response

\begin{equation} LC\dfrac{d^2v_h(t)}{dt^2} + \dfrac{L}{R}\dfrac{dv_h(t)}{dt} + v_h(t) = 0 \end{equation}

	\noindent by inspection it can be shown that $e^{kt}$ is a solution where

\begin{equation} k = \dfrac{-\dfrac{L}{R} \pm \sqrt{\left(\dfrac{L}{R}\right)^2 - 4LC} }{2LC} = -\dfrac{1}{2RC} \pm \sqrt{\left(\dfrac{1}{2RC}\right)^2 - \dfrac{1}{LC}} .\end{equation}

	For a numerical example, consider $L = 10$mH, $C = 1\mu$F and $R = 1$k$\Omega$, $M = 1V$ and $\omega = 500$ rad.s$^{-1}$. These values yield a pair of conjugate complex solutions. Because $v_h$ is real, these solutions amount to

\begin{equation} x_h(t) = c_he^{-k_R t} \cos\left(k_I t\right) \label{example:rlc_circuit_natural_example_homogeneous} \end{equation}

	\noindent where $c_h$ is a constant that can be drawn from initial conditions and

\begin{equation} k_R = \dfrac{1}{2RC},\ k_I = \pm j\ \sqrt{\dfrac{1}{LC} - \left(\dfrac{1}{2RC}\right)^2} .\end{equation}

	For the particular solution, suppose $v_p = A\sin\left(\omega t\right) + B\cos\left(\omega t\right)$, yielding

\begin{gather} -LC\omega^2\left[A\sin\left(\omega t\right) + B\cos\left(\omega t\right) \right] + \nonumber\\[5mm]\hspace{1cm} \dfrac{L}{R}\omega \left[A\cos\left(\omega t\right) - B\sin\left(\omega t\right) \right] + \nonumber\\[5mm]\hspace{3cm} A\sin\left(\omega t\right) + B\cos\left(\omega t\right) - M\cos\left(\omega t\right) = 0 \end{gather}

	Grouping the terms,

\begin{equation} \sin\left(\omega t\right) \left( -LC\omega^2 A - \dfrac{L\omega}{R}B + A\right) + \cos\left(\omega t\right)\left(-LC\omega^2 B + \dfrac{L\omega}{R}A + B - M\right) = 0 \end{equation}

	Because sine and cosine are orthogonal, this can only be possible if

\begin{equation}\left\{\begin{array}{l} -LC\omega^2 A - \dfrac{L\omega}{R}B + A = 0 \\[5mm] -LC\omega^2 B + \dfrac{L\omega}{R}A + B - M = 0 \end{array}\right. \Leftrightarrow \left\{\begin{array}{l} \left(1 -LC\omega^2\right) A - \dfrac{L\omega}{R}B = 0 \\[5mm] \left(1 - LC\omega^2 \right) B + \dfrac{L\omega}{R}A - M = 0 \end{array}\right. \end{equation}
	
	From the first equation,

\begin{equation}  A = \dfrac{L\omega}{R\left(1 -LC\omega^2\right)}B  \end{equation}

	\noindent and substituting on the second,

\begin{gather}
	\left(1 - LC\omega^2 \right) B + \dfrac{L\omega}{R}\dfrac{L\omega}{R\left(1 -LC\omega^2\right)}B - M = 0 \nonumber\\[5mm]
	\left[\left(1 - LC\omega^2 \right)^2 + \left(\dfrac{L\omega}{R}\right)^2 \right] B - M\left(1 - LC\omega^2 \right) = 0 \nonumber\\[5mm]
	B = M\xfrac{2mm}{5mm}{\left(1 - LC\omega^2 \right)}{\left[\left(1 - LC\omega^2 \right)^2 + \left(\dfrac{L\omega}{R}\right)^2 \right]} \Leftrightarrow A = M\xfrac{2mm}{5mm}{\left( \dfrac{\omega L}{R} \right)}{\left[\left(1 - LC\omega^2 \right)^2 + \left(\dfrac{L\omega}{R}\right)^2 \right]}
\end{gather}

	Therefore let $\sqrt{A^2 + B^2} = M$ and $\phi$ such that

\begin{equation} \tan\left(\phi\right) = \dfrac{A}{B} = \xfrac{3mm}{2mm}{\dfrac{\omega L}{R}}{ 1 - LC\omega^2} \end{equation}

	\noindent then the general solution of the particular solution is

\begin{equation} x_p = c_p\cos\left(\omega t + \phi\right) \label{example:rlc_circuit_natural_example_particular} \end{equation}

	\noindent with $c_p$ another constant that can be obtained from initial conditions. Therefore, a solution to the excited ODE \eqref{eq:example2_original_ode} is a the sum of \eqref{example:rlc_circuit_natural_example_homogeneous} and \eqref{example:rlc_circuit_natural_example_particular}:

\begin{equation} x(t) = c_he^{-k_R t} \cos\left(k_I t\right) + c_p\cos\left(\omega t + \phi\right) \label{eq:example2_original_ode_homo}\end{equation}

	\noindent where the $c_p$ and $c_h$ are constants respective to initial conditions. At this point of the text, this solution is still \textit{a solution}, but it will be shown later that it is actually the \textit{general solution}, that is, any solution to the original forced ODE \eqref{eq:example2_original_ode} is obtained by varying $c_p$ and $c_h$ on \eqref{eq:example2_original_ode_homo}. For a specific example, supposing $x(0) = M$ and $x'(0) = 0$,

\begin{equation} \left\{\begin{array}{l} M = c_h + c_p\cos\left(\phi\right) \\[5mm] 0 = -c_hk_R - c_p\omega\sin\left(\phi\right) \end{array}\right. \end{equation}

	Multiply the first equation by $k_R$ and sum to the second:

\begin{equation} c_p = \dfrac{Mk_R}{k_R\cos\left(\phi\right) - \omega\sin\left(\phi\right)} \end{equation}

	\noindent therefore

\begin{equation} c_h = M - \cos\left(\phi\right)\left[\dfrac{Mk_R}{k_R\cos\left(\phi\right) - \omega\sin\left(\phi\right)}\right] = -\dfrac{M\omega \sin\left(\phi\right)}{k_R\cos\left(\phi\right) - \omega\sin\left(\phi\right)}\end{equation}

\examplebar
\end{example} %>>>

	Because the natural response of an LTIODE is independent of the excitation, one might ask what is its shape and the characteristics; the first step in the discussion is to show that the general solution to the homogeneous part $\mathbf{x}_h$ is easily achievable when a certain equivalent matrix $\mathbf{A}$ is diagonalizable. If such is not the case, then a more refined discussion is made under the framework of linear algebra and linear differential equations to deal with the case that $\mathbf{A}$ is not diagonalizable, also called defective.

	The ultimate objective of this subsection is to show that it is possible to define a complex matrix exponential function such that the general solution to the natural part can be written as $\mathbf{x} = e^{\mathbf{A}t}\mathbf{x}_0$; this is motivated by the fact that the general solution to a one-dimensional ODE $\dot{x}(t) = ax(t)$ is $x(t) = e^{at}x_0$. Then, it is shown that if this equivalent matrix has certain properties, namely that its eigenvalues all have negative real part, then the natural part of the general solution vanishes in time in a strong exponential sense, leaving only the particular solution as time grows. In Classical and Dynamic Phasor Theory, this is of utmost importance because the particular solution $\mathbf{x_p}$ is somehwat simple to find, and because $\mathbf{x_h}$ vanishes as time grows, the solution $\mathbf{x_p}$ dominates over time. In other words, phasors are a particularization of the solution of the LTIODE when the transient behavior is disconsidered and the steady-state dominates over time.

	In order to achieve this, we must first define what a diagonalizable operator is, which needs the ideas of fields, matrices and bases.

%-------------------------------------------------
\section{Bases, matrices and operations}\label{sec:bases_matrices_operations} %<<<1

	Let $\mathbf{U} = \left(\mathbf{u}_1,\mathbf{u}_2,...,\mathbf{u}_k\right)$ be a sequence of $k$ arbitrary vectors. Adopt a collection of scalars $\left(z_1,z_2,...,z_k\right)$ and the expression

\begin{equation} \mathbf{m} = z_1\mathbf{u}_1 + z_2\mathbf{u}_2 + \cdots + z_k\mathbf{u}_k \end{equation}

	\noindent is called a \textbf{linear combination} of the $\mathbf{u}_i$. Further, the \textbf{span} of the set is defined as the collection of all vectors that can be written as a linear combination of the $\mathbf{u}_i$:

\begin{equation} \mathspan\left(\mathbf{U}\right) = \left\{\sum\limits_{i=1}^k z_i\mathbf{u}_i : z_i\text{ is a scalar}\right\}\end{equation}
	
	\noindent and if a certain set $\mathbf{W}$ is the span of some set $\mathbf{U}$ we say that $\mathbf{U}$ \textbf{generates} $\mathbf{W}$, or is a \textbf{generating set} of $\mathbf{W}$. In that case, a particular vector $\mathbf{x}\in\mathbf{W}$ is expressed as a linear combination of the vectors in this generating set, that is,

\begin{equation} \mathbf{x} = \sum_{i=1}^k x_i\mathbf{u}_i\end{equation}

	\noindent where the $x_k$ are scalars in the field $F$ that the vector space is defined over. Naturally, if the generating set changes, then the $x_i$ also change; in this case, $\mathbf{x}$ can be represeted by the tuple $\left(x_i\right)_{i=1}^k$ with respect to the specific generating set $\mathbf{U}$. We define this as a columnar arrangement known as coordinates:

\begin{equation} \left[\mathbf{x}\right]_\mathbf{U} = \sum\limits_{i=1}^k x_i \mathbf{u}_i \vcentcolon = \left[\begin{array}{c} x_1 \\[3mm] x_2 \\[3mm] \vdots \\[3mm] x_k \end{array}\right]_{\mathbf{U}}\end{equation}

	\noindent meaning $\left[\mathbf{x}\right]_{\mathbf{U}} = \left[x_1,x_2,\cdots,x_k\right]^\transpose$ is the representation of $\mathbf{x}$ on (or against) $\mathbf{U}$. It must be noted that because the coordinates $x_i$ are unique, the relationship $\left(x_1,x_2,\cdots,x_n\right) \leftrightarrow \mathbf{x}$ is bijective and uniquely defined, therefore an \textbf{isomorphism}.

	The space of $n$ coordinates in $F$ is denoted $F^n$, an allusion to the fact that formally this space is defined as a cartesian product $F\times F\times ... \times F$, $n$ times. In some cases it is interesting to display this arrangement in a horizontal matter, so we define the \textbf{transposition} operation, denoted with a superscript $\transpose$, that transforms a columnar arrangement into a horizontal one and vice-versa:

\begin{equation} \left[\begin{array}{c} x_1 \\[3mm] x_2 \\[3mm] \vdots \\[3mm] x_k \end{array}\right]^\transpose = \left[x_1,x_2,\cdots,x_k\right] \text{, and } \left[y_1,y_2,\cdots,y_k\right]^\transpose =  \left[\begin{array}{c} y_1 \\[3mm] y_2 \\[3mm] \vdots \\[3mm] y_k \end{array}\right] .\end{equation}

	Hereforth, we assume that that the domain of $\mathbf{A}$ can be defined over the complex numbers, that is, $F = \mathbb{C}$, meaning every vector in that domain can be represented as a set of $n$ complex coordinates. Naturally the question becomes what is the minimum or ``smallest'' generating set that a certain space can have. A \textbf{basis} (plural \textbf{bases}) of a space is a collection $\mathbf{V} = \left(\mathbf{v}_1,\mathbf{v}_2,...,\mathbf{v}_k\right)$ of \textbf{linearly independent} vectors, that is, a set such that no single component can be written as a linear combination of the others, and $\mathbf{V}$ generates that particular set. Linear independence means that for a collection of scalars $z_k$, the linear combination of the $\mathbf{v}_k$ is anihilated if and only if the scalars are null:

\begin{equation} \mathbf{0} = z_1 \mathbf{v}_1 + z_2\mathbf{v}_2 + ... + z_k \mathbf{v}_n \Leftrightarrow z_1 = z_2 = ... = z_k = 0.\end{equation}

	It follows that any vector $\mathbf{x}$ can be written as a unique linear combination of the vectors in $\mathbf{V}$: write

\begin{equation} \mathbf{x} = \beta_1 \mathbf{v}_1 + \beta_2\mathbf{v}_2 + ... + \beta_k \mathbf{v}_k \end{equation}

	\noindent where the $\beta_k$ are the \textbf{coordinates of} $\mathbf{x}$ in the basis $\mathbf{V}$. Suppose $\mathbf{x}$ has a second representation of coordinates $\gamma_k$:

\begin{equation} \mathbf{x} = \gamma_1 \mathbf{v}_1 + \gamma_2\mathbf{v}_2 + ... + \gamma_k \mathbf{v}_k \end{equation}

	\noindent subtracting both equations yields

\begin{equation} \mathbf{0} = \left(\gamma_1 - \beta_1\right) \mathbf{v}_1 + \left(\gamma_2 - \beta_2\right)\mathbf{v}_2 + ... + \left(\gamma_k - \beta_k\right) \mathbf{v}_k \end{equation}

	\noindent which can only be possible if $\beta_i = \gamma_i$, for all $i$, due to the linear independency of the vectors in the basis.

	A vector space $V$ has dimension $n$, denoted $\dim\left(V\right)$, if such is the smallest number of vectors a basis needs to have in that space, or equivalently, the least number of vectors a generating set has to have to be a basis of that space. It is a direct consequence of their definitions, and a paramount property of bases, that their span is the whole of the vector space they are immersed in; saying a vector space has dimension $n$ means exactly $n$ linearly independent vectors are needed to form a basis for it. It is obvious that $\mathbb{C}^n$ has dimension $n$.

	Naturally one asks what is the dimension of the space $\left[\mathbb{R}\to\mathbb{C}\right]$. Unfortunately such a basis does not exist; there does not exists a finite (or infinite, for this matter) set of elements that generates the entire space. There do exist, however, bases for specific subspaces. (Famously, the space of square-Lebesgue-integrable functions $L^2$ has an inner product that induces a basis which itself induces famous transforms as its inner product like Fourier and Laplace, which will be used in the later chapters of this text). For example, if we restrict the space of functions to those that solve the linear ODE being studied, theorem \ref{theo:ode_sol_space_dim} shows that the restricted space has dimension of the order of the ODE.

\begin{theorem}[Dimension of the space of solutions of an ODE]\label{theo:ode_sol_space_dim} %<<<
	The space of solutions of the linear ODE $\dot{\mathbf{x}} = \mathbf{A}\left[\mathbf{x}\right]$, in $\left[\mathbb{R}\to\mathbb{C}^n\right]$ has dimension $n$.
\end{theorem}
\noindent\textbf{Proof.} Let $\mathbf{V} = \left(\mathbf{v}_k\right)_{k=1}^n$ a set of $n$ linearly independent vectors in this space. This means that for any time $t$,

\begin{equation} \sum_{k=1}^n \alpha_k \mathbf{v}_k(t) = \mathbf{0}(t)\Leftrightarrow \alpha_1 = \alpha_2 = ... = \alpha_n .\end{equation}

	\noindent where $\mathbf{0}(t)$ here represents the null function, that is, $\mathbf{0}(t) = \left[0(t),0(t),0(t),...,0(t)\right]^\transpose$ for all times. Consider a vector $\mathbf{u}$ in the span of $\mathbf{V}$

\begin{equation} \mathbf{u} = \sum_{k=1}^n \alpha_k \mathbf{v}_k.\end{equation}

	Then

\begin{equation} \dfrac{d\mathbf{u}}{dt} = \dfrac{d}{dt}\left(\ \sum_{k=1}^n \alpha_k \mathbf{v}_k\right) \end{equation}

	\noindent and using that the differential transform is linear,

\begin{equation} \dfrac{d}{dt}\left(\ \sum_{k=1}^n \alpha_k \mathbf{v}_k\right) = \sum_{k=1}^n \alpha_k \dot{\mathbf{v}}_k \end{equation}

	\noindent but since each $\mathbf{v}_k$ is a solution of the ODE,

\begin{equation} \sum_{k=1}^n \alpha_k \dot{\mathbf{v}}_k = \sum_{k=1}^n \alpha_k \mathbf{A}\left[\mathbf{v}_k\right]   \end{equation}

	\noindent and using the linearity of $\mathbf{A}\left[\cdot\right]$,

\begin{equation} \sum_{k=1}^n \alpha_k \mathbf{A}\left[\mathbf{v}_k\right] = \mathbf{A}\left[\sum_{k=1}^n \alpha_k \mathbf{v}_k\right] = \mathbf{A}\left[\mathbf{u}\right] \end{equation}

	\noindent meaning $\mathbf{u}$ is a solution to $\dot{\mathbf{u}} = \mathbf{A}\left[\mathbf{u}\right]$, that is, $\mathbf{V}$ generates some solutions of the ODE. We now prove it actually generates all solutions. Suppose the contrary, that it does not generate all of them and that there exists a $\mathbf{v}_{(n+1)}$ that is also a solution of the ODE but is linearly independent of all the $\mathbf{v}_k$, meaning

\begin{equation} \sum_{k=1}^{(n+1)} \alpha_k \mathbf{v}_k = \mathbf{0}(t)\Leftrightarrow \alpha_k = 0,\ k\in\mathbb{N}_{(n+1)}^* .\end{equation}

	We now ``freeze'' this equation in time: for each time $t = a$,

\begin{equation} \sum_{k=1}^{(n+1)} \alpha_k \mathbf{v}_k(a) = \mathbf{0},\ k\in\mathbb{N}_{(n+1)}^* . \end{equation}

	\noindent where $\mathbf{0}$ is the null complex vector. This cannot be true because the $\mathbb{C}^n$ has dimension $n$; yet this equation dictates that there exist $n+1$ linearly independent vectors in $\mathbb{C}^n$. Therefore, at all times $t$ there must be some linear combination among the complex vectors $\mathbf{v}_k(t=a)$, which then means there must be a linear combination among the $\mathbf{v}_k$, contradicting the supposition.
\hfill$\blacksquare$
\vspace{5mm}
\hrule
\vspace{5mm} %>>>

	Therefore, notwithstanding the fact that $\left[\mathbb{R}\to\mathbb{C}^n\right]$ does not have a basis (because there needs to be an uncountable infinite number of vectors), the set of solutions of the ODE $\dot{\mathbf{x}} = \mathbf{A}\left[\mathbf{x}\right]$ defined by $\mathbf{A}$ in that space does have dimension $n$. Due to this fact, in the specific case of this differential equation we define $\Dom\left(\mathbf{A}\right)$ as the space of solutions of that ODE defined by $\mathbf{A}$, as opposed to its proper domain.

	It will be shown that bases are somewhat equivalent in the sense a vector representation in a particular base can be changed to a representation on another base and that, despite this process being possible, linear functionals have certain properties that are unwaivering to a base change. It is however natural to assume a certain fixed or natural basis can be adopted to establish common grounds of representation; for instance, adopting the canonical basis $\mathbf{I}_n$ for $\mathbb{C}^n$ composed of the vectors $\mathbf{e}_k$, $1\leq k \leq n$, which contain a unitary element on the k-th positions and zero everywhere else:

\begin{equation} \mathbf{I}_n = \left(\mathbf{e}_1,\mathbf{e}_2,\mathbf{e}_3,...,\mathbf{e}_n\right) = \left( \left[\begin{array}{c} 1 \\[3mm] 0 \\[3mm] 0 \\[3mm] \vdots \\[3mm] 0 \end{array}\right], \left[\begin{array}{c} 0 \\[3mm] 1 \\[3mm] 0 \\[3mm] \vdots \\[3mm] 0 \end{array}\right], \left[\begin{array}{c} 0 \\[3mm] 0 \\[3mm] 1 \\[3mm] \vdots \\[3mm] 0 \end{array}\right], ..., \left[\begin{array}{c} 0 \\[3mm] 0 \\[3mm] 0 \\[3mm] \vdots \\[3mm] 1 \end{array}\right] \right) \label{eq:canonical_basis_complex}\end{equation}

	\noindent and let us define the canonical  basis of the solutions of the ODE as $\left(\mathbf{v}_1,\mathbf{v}_2,\mathbf{v}_3,...,\mathbf{v}_n\right)$ where each $\mathbf{v}_k$ is the canonical vector $\mathbf{e}_k$ at an initial time $t_0$ that is, $\mathbf{v}_k$ is defined as the vector function that satisfies

\begin{equation} \left\{\begin{array}{l} \dot{\mathbf{v}}_k = \mathbf{A}\left[\mathbf{v}_k\right] \\[3mm] \mathbf{v}_k\left(t_0\right) = \mathbf{e}_k \end{array}\right. .\end{equation}

	One asks whether the $\mathbf{v}_k$ exist and are unique, and the answer is yes: this is easily provable using the Banach-Cacciopoli Fixed Point Theorem. For now we do not prove this fact since we are more interested in the qualities of the $\mathbf{v}_k$ as a basis. Now establish the bijection

\begin{equation} \phi: \left\{\begin{array}{rcl} \Dom\left(\mathbf{A}\right) &\to& \mathbb{C}^n \\[3mm] \mathbf{v}_k &\mapsto& \mathbf{e}_k \end{array}\right. \end{equation}

	\noindent (the notation $\Dom\left(\mathbf{A}\right)$ here is not understood as the proper domain $\left[\mathbb{R}\to\mathbb{C}^n\right]$ but the restriction of this space to the subspace of functions that satisfy the ODE defined by $\mathbf{A}\left[\cdot\right]$). We want to use this bijection to show that any vector in $\Dom\left(\mathbf{A}\right)$ admits a representation in $\mathbb{C}^n$ using the chosen basis $\mathbf{V}$. First we note that this bijection is a morphism between $\Dom\left(\mathbf{A}\right)$ and $\mathbb{C}^n$, that is, it preserves the algebraic structures and operations of sum and multiplication by scalar; particularly, any linear combination in $\Dom\left(\mathbf{A}\right)$ remains in $\mathbb{C}^n$, that is, for any $\mathbf{v},\mathbf{w}\in\Dom\left(\mathbf{A}\right)$ and any $\alpha\in\mathbb{C}$,

\begin{equation} \phi\left(\mathbf{v} + \alpha\mathbf{w}\right) = \phi\left(\mathbf{v}\right) + \alpha\phi\left(\mathbf{w}\right) .\end{equation}

	Further, we can show that because $\phi$ maps each $\mathbf{v}_k$ specifically to the $k$-th vector $\mathbf{e}_k$ of the canonical basis of $\mathbb{C}^n$, then $\phi$ is unique. This fact allows us to represent any $\mathbf{x}\in\Dom\left(\mathbf{A}\right)$ as coordinates on the chosen basis $\mathbf{V}$,

\begin{equation} \left[\mathbf{x}\right]_{\mathbf{V}} = \sum\limits_{k=1}^n x_k \mathbf{v}_k .\end{equation}

	\noindent then we can establish a representation of $\mathbf{x}$ into the $\mathbb{C}^n$ by using $\phi$ and its linearity:

\begin{equation} \phi\left(\left[\mathbf{x}\right]_{\mathbf{V}}\right) = \phi\left(\sum\limits_{k=1}^n x_k \mathbf{v}_k\right) = \sum\limits_{k=1}^n x_k \phi\left(\mathbf{v}_k\right) = \sum\limits_{k=1}^n x_k \mathbf{e}_k .\end{equation}

	Therefore $\phi\left(\left[\mathbf{x}\right]_{\mathbf{V}}\right)$ has a set of coordinates in $\mathbb{C}^n$ identical to the coordinates of $\mathbf{x}$ in $\mathbf{V}$ but using the canonical basis $\mathbf{I}_n$ for $\mathbb{C}^n$. Thus we can adopt the representation of $\mathbf{x}$ with respect to the canonical basis of $\mathbb{C}^n$ as

\begin{equation} \left[\mathbf{x}\right]_{\mathbf{I}_n} = \phi\left(\left[\mathbf{x}\right]_{\mathbf{V}}\right) = \sum\limits_{k=1}^n x_k \mathbf{e}_k .\end{equation}

	In other words, $\mathbf{x}$ is represented as a point in $\mathbb{C}^n$, meaning that an analysis on $\Dom\left(\mathbf{A}\right)$ is equivalent to an analysis on $\mathbb{C}^n$. Further, the basis $\mathbf{V}$ also induces a canonical matrix form for the linear map $\mathbf{A}\left[\cdot\right]$, denoted $\left[\mathbf{A}\right]_{\mathbf{I}_n}$: exploring the linearity of $\mathbf{A}\left[\cdot\right]$ one yields

\begin{equation}
	\mathbf{A}\left[\mathbf{x}\right]_{\mathbf{V}} = \mathbf{A}\left[\begin{array}{c} x_1 \\[3mm] x_2 \\[3mm] \vdots \\[3mm] x_n\end{array} \right]_\mathbf{V} = \mathbf{A}\left[\sum\limits_{k=1}^n x_k\mathbf{v}_k \right] = \sum_{k=1}^n x_k\mathbf{A}\left[\mathbf{v}_k\right] \label{eq:Aappliedtox} .% =  \left[\begin{array}{c} a_{11}x_1(t) + a_{12}x_2(t) + ... + a_{1n}x_n(t) \\[3mm] a_{21}x_1(t) + a_{22}x_2(t) + ... + a_{2n}x_n(t) \\[3mm] \vdots \\[3mm]  a_{n1}x_1(t) + a_{n2}x_2(t) + ... + a_{nn}x_n(t) \end{array}\right]
\end{equation}

	Therefore, operating a vector $\mathbf{x}$ through $\mathbf{A}\left[\mathbf{x}\right]$ can be (instead of direct calculation) found simply if one knows the coordinates of $\mathbf{x}$ with respect to $\mathbf{V}$ (which are the same as with repect to $\mathbf{I}_n$) and how $\mathbf{A}\left[\cdot\right]$ transforms each of the canonical vectors $\mathbf{v}_k$. But seen as each $\mathbf{A}\left[\mathbf{v}_k\right]$ is a vector itself, it has coordinates on $\mathbf{V}$:

\begin{equation} \mathbf{A}\left[\mathbf{v}_k\right] = \left[\begin{array}{c} a_{1k} \\[3mm] a_{2k} \\[3mm] \vdots \\[3mm] a_{nk} \end{array}\right]_\mathbf{V}\end{equation}

	\noindent and because $\phi$ is bijective and unique, we can represent $\mathbf{A}\left[\cdot\right]$ uniquely by the $\mathbf{A}\left[\mathbf{v}_k\right]$; therefore, we can arrange the $\mathbf{A}\left[\mathbf{v}_k\right]$ into a tabular arrangement that we will call a \textbf{matrix}, by using these vectors as the columns. This matrix is denoted $\mathbf{A}$:

\begin{equation} \mathbf{A} = \left[\raisebox{15mm}{} \begin{array}{cccc} \left[\begin{array}{c} \vdots \\[3mm] \mathbf{A}\left[\mathbf{v}_1\right] \\[3mm] \vdots \end{array}\right] & \left[\begin{array}{c} \vdots \\[3mm] \mathbf{A}\left[\mathbf{v}_2\right] \\[3mm] \vdots \end{array}\right] & ... & \left[\begin{array}{c} \vdots \\[3mm] \mathbf{A}\left[\mathbf{v}_n\right] \\[3mm] \vdots \end{array}\right]\end{array}\right] = \left[\begin{array}{cccc} a_{11} & a_{12} & ... & a_{1n} \\[3mm] a_{21} & a_{22} & ... & a_{2n} \\[3mm] \vdots & \vdots & \ddots & \vdots \\[3mm]  a_{n1} & a_{n2} & ... & a_{nn} \end{array}\right]  . \label{eq:tabular_arrangement}\end{equation}

	\noindent or we can see a matrix as an arrangement of $n$ vectors of dimension $n$ as its rows:

\begin{equation} \mathbf{A} = \left[\begin{array}{c} \left[\cdots\ \mathbf{r}_1\ \cdots \right] \\[5mm] \left[\cdots\ \mathbf{r}_2\  \cdots \right] \\[5mm]  \vdots  \\[5mm] \left[\cdots\ \mathbf{r}_n\ \cdots \right] \end{array}\right] \end{equation}

	\noindent such that rows and columns can be ``rotated'' through a version of the \textbf{transposition} operation for matrices, that is, the transpose of $\mathbf{A}$, denoted $\mathbf{A}^\transpose$, is the matrix which rows and the columns of $\mathbf{A}$ and vice-versa:

\begin{equation} \mathbf{A}^\transpose = \left[\raisebox{15mm}{} \begin{array}{cccc} \left[\begin{array}{c} \vdots \\[3mm] \mathbf{r}_1 \\[3mm] \vdots \end{array}\right] & \left[\begin{array}{c} \vdots \\[3mm] \mathbf{r}_2 \\[3mm] \vdots \end{array}\right] & ... & \left[\begin{array}{c} \vdots \\[3mm] \mathbf{r}_n \\[3mm] \vdots \end{array}\right]\end{array}\right] = \left[\begin{array}{c} \left[\cdots\ \mathbf{c}_1\ \cdots \right] \\[5mm] \left[\cdots\ \mathbf{c}_2\  \cdots \right] \\[5mm]  \vdots  \\[5mm] \left[\cdots\ \mathbf{c}_n\ \cdots \right] \end{array}\right] .\end{equation}

	\noindent and it can be shown that the relationship between matrix and functional is bijective, meaning that the matrix $\mathbf{A}$ is a particular matrix representation of the linear operator $\mathbf{A}\left[\cdot\right]$ when the canonical basis is adopted. The underlying implication is that the matrix $\mathbf{A}$, called the \textbf{canonical representation} of the linear functional $\mathbf{A}\left[\cdot\right]$, is such that they can be interpreted as the same entity in some sense.

	Equation \eqref{eq:Aappliedtox} then defines that the linear transform $\mathbf{A}\left[\cdot\right]$ applied to a particular vector $\mathbf{x}$, becomes a linear combination of the column vectors of the canonical matrix representation $\mathbf{A}$ where the coefficients of the combination are the coordinates of $\mathbf{x}$. Because of this, we can define a matrix-by-vector multiplication as the linear combination of the column vectors.

\begin{definition}[Matrix-by-vector multiplication]\label{def:matrixbyvector} Let $\mathbf{A}\in\mathbb{C}^{(n\times n)}$, $\mathbf{c}_k,\ k\in\mathbb{N}^*_n$ its column vectors, and $\mathbf{x} = \left[x_1,x_2,...,x_n\right]^\transpose\in\mathbb{C}^n$. Then the multiplication $\mathbf{Ax}$ is defined as

\begin{equation} \mathbf{Ax} = \sum_{k=1}^n x_k\mathbf{c}_k \end{equation}
\end{definition}

	And the idea is that defining such multiplication this way makes the application $\mathbf{A}\left[\mathbf{x}\right]$ a simple multiplication in $\mathbb{C}^n$, while retaining algebraic structions and retaining the bijection $\phi$ between $\Dom\left(\mathbf{A}\right)$ and $\mathbb{C}^n$, as denoted in theorem \ref{theo:bij_v_cnmaintained}, easily proven by inspection.

\begin{theorem}\label{theo:bij_v_cnmaintained} Let $\mathbf{A}\left[\cdot\right]$ some linear map with, $\mathbf{x}\in\Dom\left(\mathbf{A}\right)$ and $\mathbf{V}$ the canonical basis of the domain. Then the coordinates of the vector $\mathbf{A}\left[\mathbf{x}\right]$ in $\mathbf{V}$ are the same coordinates than the multiplication $\left[\mathbf{A}\right]_{\mathbf{I}_n} \left[\mathbf{x}\right]_{\mathbf{I}_n}$, that is,

\begin{equation} \left[\raisebox{1mm}[2mm][2mm]{} \mathbf{A}\left[\mathbf{x}\right]\right]_\mathbf{V} = \left[\mathbf{A}\right]_{\mathbf{I}_n} \left[\mathbf{x}\right]_{\mathbf{I}_n}\end{equation}
\end{theorem}
\hrule
\vspace{3mm}

	Using definition \ref{def:matrixbyvector} we can simplify the application $\mathbf{A}\left[\mathbf{x}\right]$ to a multiplication $\mathbf{Ax}$. The linearity property of this multiplication is immediately provable. If we repeat this same line of thought for horizontal vectors, we achieve a similar result in horizontal form, the idea being that instead of producing column vectors we can produce row vectors by transposing the multiplication.

\begin{definition}[Vector-by-matrix multiplication]\label{def:vectorbymatrix} Let $\mathbf{A}\in\mathbb{C}^{(n\times n)}$, $\mathbf{r}_k,\ k\in\mathbb{N}^*_n$ its row vectors, and $\mathbf{x} = \left[x_1,x_2,...,x_n\right]\in\mathbb{C}^n$. Then the multiplication $\mathbf{xA}$ is defined as

\begin{equation} \mathbf{xA} = \sum_{k=1}^n x_k\mathbf{r}_k \end{equation}
\end{definition}
\begin{theorem}\label{theo:vector_trasnp} Let $\mathbf{A}\in\mathbb{C}^{(n\times n)}$ and $\mathbf{x} = \left[x_1,x_2,...,x_n\right]^\transpose\in\mathbb{C}^n$. Then $\left(\mathbf{Ax}\right)^\transpose = \mathbf{x}^\transpose\mathbf{A}^\transpose$.
\end{theorem}
\hrule
\vspace{3mm}

	It is clear that in order for the matrix-by-vector operation to be feasible, the vector $\mathbf{x}$ has to have as many elements as the matrix $\mathbf{A}$ has rows, whereas for vector-by-matrix, $\mathbf{x}$ has to have as many elements as $\mathbf{A}$ has columns. Thence, the ODE \eqref{eq:matrix_lti_ode_eq_def} can be written as

\begin{equation} \dfrac{d}{dt} \left[\mathbf{x} \right]_{\mathbf{I}_n} = \left[\mathbf{A}\right]_{\mathbf{I}_n}\left[\mathbf{x}\right]_{\mathbf{I}_n} + \left[\mathbf{f}\right]_{\mathbf{I}_n} . \label{eq:equivalent_lti_ode_operator} \end{equation}

	\noindent where it must be understood that $\left[\mathbf{x} \right]_{\mathbf{I}_n}$ is a time-varying quantity. In order to simplify notation, and seen as due to the bijection $\phi$ the representation in $\mathbf{V}$ and in $\mathbf{I}_n$ are the same, then hereforth we just write


\begin{equation} \dot{\mathbf{x}} = \mathbf{Ax} + \mathbf{f} . \label{eq:equivalent_lti_ode_operator} \end{equation}

	For completude, we can also define a matrix-by-matrix multiplication as an operation induced by the matrix-by-vector multiplication.

\begin{definition}[Matrix-by-matrix multiplication]\label{def:matrixbymatrix}  Let $\mathbf{A,B}\in\mathbb{C}^{(n\times n)}$, $\mathbf{b}_k,\ k\in\mathbb{N}^*_n$ the column vectors of $\mathbf{B}$. Then the multiplication $\mathbf{AB}$ is defined as the matrix which columns are the multiplications of $\mathbf{A}$ by the columns of $\mathbf{B}$:

\begin{equation} \mathbf{AB} = \left[\raisebox{15mm}{} \begin{array}{cccc} \left[\begin{array}{c} \vdots \\[3mm] \mathbf{A}\mathbf{b}_1 \\[3mm] \vdots \end{array}\right] & \left[\begin{array}{c} \vdots \\[3mm] \mathbf{A}\mathbf{b}_2 \\[3mm] \vdots \end{array}\right] & ... & \left[\begin{array}{c} \vdots \\[3mm] \mathbf{A}\mathbf{b}_n \\[3mm] \vdots \end{array}\right]\end{array}\right] .\end{equation}

\end{definition}

	From this definition many properties of this multiplication can be drawn, such as its notorious non-commutativity and the fact that it can only be defined if $\mathbf{A}$ has as many columns as $\mathbf{B}$ has rows. Proving all such properties is not the scope of this text and will thenceforth be assumed. It is simple to prove that joining definitions \ref{def:matrixbyvector}, \ref{def:vectorbymatrix} and \ref{def:matrixbymatrix} yields the ``transpose of product'' rule of theorem \ref{theo:matrix_trasnp}.

\begin{theorem}\label{theo:matrix_trasnp} Let $\mathbf{A}\in\mathbb{C}^{(n\times m)},\mathbf{B}\in\mathbb{C}^{(m\times n)}$, where $n,m\in\mathbb{N}_1$. Then $\left(\mathbf{AB}\right)^\transpose = \mathbf{B}^\transpose\mathbf{A}^\transpose$.
\end{theorem}
\hrule
\vspace{3mm}

	It is also simple to prove that the basis $\mathbf{I}_n$ induces the matrix which columns are the canonical vectors

\begin{equation} \mathbf{I}_n = \left[\begin{array}{ccccc} 1 & 0 & 0 & \cdots & 0 \\[3mm] 0 & 1 & 0 & \cdots & 0 \\[3mm] 0 & 0 & 1 & \cdots & 0 \\[3mm] \vdots & \vdots & \vdots & \ddots & \vdots \\[3mm] 0 & 0 & 0 & \cdots & 1 \end{array}\right]_{(n\times n)}\end{equation}

	\noindent called the \textbf{identity matrix}. This matrix has a fundamental role in matrix algebra because it is the neutral element of matrix multiplication: $\mathbf{AI} = \mathbf{IA} = \mathbf{A}$ for any matrix $\mathbf{A}$ of $n$ rows. One can see this by the fact that the columns of $\mathbf{AI}$ are linear combinations of the columns of $\mathbf{A}$ where the coefficients are the elements of the columns of $\mathbf{I}_n$; but since the columns of $\mathbf{I}_n$ are simply the canonical vectors, each column of $\mathbf{AI}$ is just a copy of the columns of $\mathbf{A}$. Due to this, we can define the inverse operation of the multiplication, that is, matrix invertibility, as the property of \textit{some} matrices to have a multiplicative inverse.

\begin{definition}[Invertible matrix]\label{def:invertible_matrix} A matrix $\mathbf{A}$ is said to be \textbf{left invertible} if there is a matrix $\mathbf{B}$ such that $\mathbf{BA = I}_n$, and \textbf{right invertible} if there is a matrix $\mathbf{C}$ such that $\mathbf{AC = I}_n$. If a matrix is left and right invertible, then it is said to be simply \textbf{invertible}, and its inverse is denoted $\mathbf{A}^{-1}$.

	Matrices that are not invertible are called \textbf{singular}, or simply, \textbf{non-invertible}.
\end{definition}

	In essence, left-invertibility of a matrix $\mathbf{A}$ means that its associated linear mapping is injective, while right-invertibility is equivalent to surjection. Full invertibility, then, means that the linear map is bijective. It can be shown that invertible matrices are always square (have the same number of columns and rows) and that $\mathbf{A}^{-1}$ is both left and right invertible, because any matrix of size $m\times n$ where $m\neq n$ can be left or right invertible but never fully invertible because the left and right inverses are naturally distinct because they have different sizes.

	One result of the representation of matrices under bases is the conclusion that any matrix that has linearly independent columns is invertible.

\begin{theorem}[Matrix invertibility]\label{theo:invertiblematrix} %<<<
	A square matrix is invertible if and only if its columns are linearly independent. \end{theorem}
\noindent\textbf{Proof:} we first prove that $\mathbf{A}$ is left-invertible if and only if its columns are linearly independent, and then proving that for a square matrix left invertibility means right invertibility.

	We first prove the forward implication that linearly independent columns imply left invertibility. Take a matrix $\mathbf{A}$ which suffices this property. First we prove that a left inverse exists, that is, there exists a matrix $\mathbf{B}$ such that $\mathbf{AB} = \mathbf{I}_n$. But

\begin{equation} \mathbf{AB} = \left[\raisebox{15mm}{} \begin{array}{cccc} \left[\begin{array}{c} \vdots \\[3mm] \mathbf{A}\mathbf{b}_1 \\[3mm] \vdots \end{array}\right] & \left[\begin{array}{c} \vdots \\[3mm] \mathbf{A}\mathbf{b}_2 \\[3mm] \vdots \end{array}\right] & ... & \left[\begin{array}{c} \vdots \\[3mm] \mathbf{A}\mathbf{b}_n \\[3mm] \vdots \end{array}\right]\end{array}\right] = \mathbf{I}_n \end{equation}

	\noindent with $\mathbf{b}_k$ the columns of $\mathbf{B}$. Dividing this equation column by column, this means

\begin{equation} \mathbf{Ab}_k = \mathbf{e}_k, \end{equation}

	\noindent that is, finding $\mathbf{B}$ means finding each column $\mathbf{b}_k$; but since $\mathbf{Ab}_k$ is in essence some linear combination of the columns of $\mathbf{A}$, $\mathbf{b}_k$ exists if and only if a linear combination of the columns exists for any of the canonical vectors $\mathbf{e}_k$. Because these columns are linearly idependent, then $\mathbf{A}$ forms a basis, meaning that such linear combination indeed exists for any $\mathbf{e}_k$. Thence we can find each $\mathbf{b}_k$ and build $\mathbf{B}$.

	And now we prove the backwards implication that right invertibility means linearly independent columns. If a right-inverse $\mathbf{B}$ exists, pick an arbitrary vector $\mathbf{x}$ and

\begin{gather}
	\mathbf{ABx} = \mathbf{I}_n \mathbf{x} \\[3mm]
	\left[\raisebox{15mm}{} \begin{array}{cccc} \left[\begin{array}{c} \vdots \\[3mm] \mathbf{A}\mathbf{b}_1 \\[3mm] \vdots \end{array}\right] & \left[\begin{array}{c} \vdots \\[3mm] \mathbf{A}\mathbf{b}_2 \\[3mm] \vdots \end{array}\right] & ... & \left[\begin{array}{c} \vdots \\[3mm] \mathbf{A}\mathbf{b}_n \\[3mm] \vdots \end{array}\right]\end{array}\right]\mathbf{x} = \mathbf{x} \\[3mm]
	\sum\limits_{k=1}^n \mathbf{Ab}_k x_k = \mathbf{x} \\[3mm]
	\mathbf{A}\left(\sum\limits_{k=1}^n \mathbf{b}_k x_k \right) = \mathbf{x}
\end{gather}

	\noindent which means $\mathbf{A}$ multiplied by some vector yields the arbitrary $\mathbf{x}$. Since this works for any $\mathbf{x}$, the only way for this to be possible is if $\mathbf{A}$ forms a basis, otherwise it cannot express any arbitrary $\mathbf{x}$.

	Finally, we prove that right invertibility yields left invertibility. Note that we have proven that a right inverse $\mathbf{B}$ exists; by definition, it is left-invertible and its left-inverse is $\mathbf{A}$.  Now, let us multiply $\mathbf{AB} = \mathbf{I}_n$ by $\mathbf{A}$ on the left, yielding

\begin{equation} \mathbf{ABA} = \mathbf{A} \Leftrightarrow \mathbf{ABA} - \mathbf{A} = \mathbf{0}_n \Leftrightarrow \left(\mathbf{BA} - \mathbf{I}_n\right)\mathbf{A} = \mathbf{0}_n .\end{equation}

	\noindent (here we are assuming the associativity of matrix multiplication). Multiply this equation on the right by $\mathbf{B}$; but because $\mathbf{AB} = \mathbf{I}_n$, this yields

\begin{equation} \left(\mathbf{BA} - \mathbf{I}_n\right)\mathbf{AB} = \mathbf{0}_n \Leftrightarrow \left(\mathbf{BA} - \mathbf{I}_n\right)\mathbf{I}_n = \mathbf{0}_n \Leftrightarrow \mathbf{BA} - \mathbf{I}_n = \mathbf{0}_n .\end{equation}
\hfill$\blacksquare$
\vspace{5mm}
\hrule
\vspace{5mm} %>>>

	Another result stemming from the representation under bases is that bases can themselves be seen as matrices — hence why bases are noted in bold capital letters like matrices in this thesis. Indeed, pick a basis $\mathbf{V} = \left(\mathbf{v}_1,\mathbf{v}_2,...,\mathbf{v}_k\right),\ k\leq n$. Then each $\mathbf{v}_i$ admits a representation under $\mathbf{I}_n$, and we can write

\begin{equation} \left[\mathbf{V}\right]_{\mathbf{I}_n} = \left[\raisebox{15mm}{} \begin{array}{cccc} \left[\begin{array}{c} \vdots \\[3mm] \left[\mathbf{v}_1\right]_{\mathbf{I}_n} \\[3mm] \vdots \end{array}\right] & \left[\begin{array}{c} \vdots \\[3mm] \left[\mathbf{v}_2\right]_{\mathbf{I}_n} \\[3mm] \vdots \end{array}\right] & ... & \left[\begin{array}{c} \vdots \\[3mm] \left[\mathbf{v}_k\right]_{\mathbf{I}_n} \\[3mm] \vdots \end{array}\right]\end{array}\right] . \label{eq:complete_basis_bn}\end{equation}

	\noindent meaning a basis forms a matrix with linearly independent rows and columns and, conversely, a matrix with linearly independent rows and columns forms a matrix.

\begin{theorem}[Bases as matrices]\label{theo:bases_as_matrices} %<<< 
	Any square matrix with linearly independent columns or rows forms a basis, and the converse is also true.
\end{theorem}

	Finally, given the properties of generating sets and especially bases, the objective is now to explore these properties to find certain specific bases where the characteristics of the mapping $\mathbf{A}\left[\cdot\right]$ are convenient to help solving the differential equation it defines.

%-------------------------------------------------
\section{Base changes} %<<<1

	Bases are not unique; in fact it might be that for some reason it is useful to represent a certain vector $\mathbf{x}$ in another basis other than $\mathbf{V}$, say a new basis $\mathbf{W}$, in order to explore the properties of this particular basis. It is immediate to see, and natural to grasp, that a certain vector $\mathbf{x}$ will have different coordinates in different basis. Given the coordinates of $\mathbf{x}$ in a first basis, the process of finding the coordinates of $\mathbf{x}$ in a different basis is called a \textit{change of basis}. Let $\mathbf{W} = \left(\mathbf{w}_1,\mathbf{w}_2,...,\mathbf{w}_n\right)$ denote a second basis of $\mathbb{C}^n$. Then each $\mathbf{w}_k$ is given by a particular combination of the $\mathbf{v}_i$:

\begin{equation} \mathbf{w}_k = \sum\limits_{i=1}^n p_{ik} \mathbf{v}_i \end{equation}

	\noindent where the $p_{\left(i,k\right)}$ are the coordinates of $\mathbf{w}_k$ against the first basis $\mathbf{V}$. Then denote

\begin{equation} \mathbf{P} = \left[\begin{array}{cccc} p_{11} & p_{12} & ... & p_{1n} \\ [5mm] p_{21} & p_{22} & ... & p_{2n} \\[5mm] \vdots & \vdots & \ddots & \vdots \\[5mm] p_{n1} & p_{n2} & ... & p_{nn} \end{array}\right] \end{equation}

	\noindent as the \textit{transition} matrix pertaining to the change of basis $\mathbf{V}$ to $\mathbf{W}$. It stands to reason that in order for this process to make sense, $\mathbf{P}$ must be invertible: the $\mathbf{v}_k$ and the $\mathbf{w}_i$ must be biunivocally related. Then this equation is equivalent to

\begin{equation}\left[\begin{array}{c} \left[\cdots\ \mathbf{w}_1\ \cdots \right] \\[5mm] \left[\cdots\ \mathbf{w}_2\  \cdots \right] \\[5mm]  \vdots  \\[5mm] \left[\cdots\ \mathbf{w}_n\ \cdots \right] \end{array}\right] = \mathbf{P}\left[\begin{array}{c} \left[\cdots\ \mathbf{v}_1\ \cdots \right] \\[5mm] \left[\cdots\ \mathbf{v}_2\  \cdots \right] \\[5mm]  \vdots  \\[5mm] \left[\cdots\ \mathbf{v}_n\ \cdots \right] \end{array}\right]. \end{equation}

	Now consider an arbitrary vector $\mathbf{x}$ with coordinates $z_k$ on $\mathbf{V}$ and $y_k$ on $\mathbf{W}$. Then

\begin{gather}
	\sum\limits_{k=1}^n x_k\mathbf{v}_k = \sum\limits_{k=1}^n y_k\mathbf{w}_k \nonumber\\[5mm]
	\sum\limits_{k=1}^n x_k\mathbf{v}_k = \sum\limits_{k=1}^n y_k \left( \sum\limits_{i=1}^n p_{\left(i,k\right)}\mathbf{v}_k\right) \nonumber\\[5mm]
	\sum\limits_{k=1}^n x_k\mathbf{v}_k = \sum\limits_{k=1}^n \left( \sum\limits_{i=1}^n y_k p_{\left(i,k\right)}\right) \mathbf{v}_k
\end{gather}

	\noindent due to the linear independency of the $\mathbf{v}_k$ this yields

\begin{equation}
	x_k = \sum\limits_{i=1}^n y_k p_{\left(i,k\right)}
\end{equation}

	\noindent which is equivalent to

\begin{equation} \left[\begin{array}{c} x_1 \\[3mm] x_2\\[3mm] \vdots \\[3mm] x_n\end{array}\right] = \mathbf{P}\left[\begin{array}{c} y_1 \\[3mm] y_2\\[3mm] \vdots \\[3mm] y_n\end{array}\right] \Leftrightarrow \left[\mathbf{x}\right]_\mathbf{V} = \mathbf{P}\left[\mathbf{x}\right]_\mathbf{W}\Leftrightarrow \left[\mathbf{x}\right]_\mathbf{W} = \mathbf{P}^{-1}\left[\mathbf{x}\right]_\mathbf{V} \label{eq:coordinate_transform}\end{equation}

	Naturally, a linear map $\mathbf{A}\left[\cdot\right]$ will also change matrix forms in different basis, meaning that the operation $\mathbf{A}\left[\mathbf{x}\right]$ stays the same but has different representation. Let

\begin{equation} \left[\mathbf{y}\right]_\mathbf{W} = \left[\mathbf{A}\right]_\mathbf{W} \left[\mathbf{x}\right]_\mathbf{W}, \left[\mathbf{y}\right]_\mathbf{V} = \left[\mathbf{A}\right]_\mathbf{V} \left[\mathbf{x}\right]_\mathbf{V} \end{equation}

	\noindent denote the results of the linear operator $\mathbf{A}\left[\cdot\right]$ on $\mathbf{x}$ in both basis. Then

\begin{gather}
	\left[\mathbf{y}\right]_\mathbf{V} = \mathbf{P}\left[\mathbf{y}\right]_\mathbf{W} \nonumber\\[5mm]
	\left[\mathbf{A}\right]_\mathbf{V}\left[\mathbf{x}\right]_\mathbf{V} = \mathbf{P}\left[\mathbf{A}\right]_\mathbf{W}\left[\mathbf{x}\right]_\mathbf{W} \nonumber\\[5mm]
	\left[\mathbf{A}\right]_\mathbf{V}\left[\mathbf{x}\right]_\mathbf{V} = \mathbf{P}\left[\mathbf{A}\right]_\mathbf{W} \mathbf{P}^{-1} \left[\mathbf{x}\right]_\mathbf{V} \nonumber\\[5mm]
	\mathbf{0} = \left(\left[\mathbf{A}\right]_\mathbf{V} - \mathbf{P}\left[\mathbf{A}\right]_\mathbf{W} \mathbf{P}^{-1}\right) \left[\mathbf{x}\right]_\mathbf{V}
\end{gather}

	\noindent which can only be true for any arbitrary $\mathbf{x}$ if the matrix in parenthesis is the null matrix:

\begin{equation} \left[\mathbf{A}\right]_\mathbf{V} = \mathbf{P}\left[\mathbf{A}\right]_\mathbf{W} \mathbf{P}^{-1}. \end{equation}

	Therefore, the matrices $\left[\mathbf{A}\right]_\mathbf{V}$ and $\left[\mathbf{A}\right]_\mathbf{W}$ represent the same linear operator $\mathbf{A}\left[\cdot\right]$ in two different basis related through $\mathbf{P}$. Because of this, the concept of \textit{similarity} is drawn as an equivalence relation between two matrices such that two similar matrices represent the same linear operator in different basis.

\begin{definition}[Matrix similarity] \label{def:matrix_similarity}
	Two complex matrices $\mathbf{X,Y}\in\mathbb{C}^{\left(n\times n\right)}$ are \textbf{similar} if there is an invertible $\mathbf{P}$ such that $\mathbf{X} = \mathbf{PYP}^{-1}$, where $\mathbf{P}$ is called a similarity matrix. ``$\mathbf{X}$ is similar to $\mathbf{Y}$'' is  denoted $\mathbf{X}\sim\mathbf{Y}$.
\end{definition}

	It is simple to prove that matrix similarity is an equivalence relationship (it is reflexive, symmetric and transitive). The notion of base changes is paramount to the analysis of linear mappings. In what follows, we strive to achieve specific base changes, that is, specific similarity relationships, that allow for better understanding the properties of linear mappings and matrices, in order to apply this to the solution of LTI ODEs.

\begin{theorem}[Similarity of bases]\label{theo:bases_similarity} %<<< 
	Any two bases are always similar.
\end{theorem}
\noindent\textbf{Proof:} pick two bases $\mathbf{V}$ and $\mathbf{W}$. Then each $\mathbf{v}_k$ admits a representation under $\mathbf{W}$ and

\begin{align}
	\left[\mathbf{V}\right]_{\mathbf{W}} &= \left[\raisebox{15mm}{} \begin{array}{cccc} \left[\begin{array}{c} \vdots \\[3mm] \left[\mathbf{v}_1\right]_{\mathbf{W}} \\[3mm] \vdots \end{array}\right] & \left[\begin{array}{c} \vdots \\[3mm] \left[\mathbf{v}_2\right]_{\mathbf{W}} \\[3mm] \vdots \end{array}\right] & ... & \left[\begin{array}{c} \vdots \\[3mm] \left[\mathbf{v}_k\right]_{\mathbf{W}} \\[3mm] \vdots \end{array}\right]\end{array}\right] = \nonumber\\[3mm]
	&=\left[\raisebox{15mm}{} \begin{array}{cccc} \left[\begin{array}{c} \vdots \\[3mm] \left[\mathbf{W}\right]_{\mathbf{I}_n} \left[\mathbf{v}_1\right]_{\mathbf{I}_n} \\[3mm] \vdots \end{array}\right] & \left[\begin{array}{c} \vdots \\[3mm] \left[\mathbf{W}\right]_{\mathbf{I}_n}\left[\mathbf{v}_2\right]_{\mathbf{I}_n} \\[3mm] \vdots \end{array}\right] & ... & \left[\begin{array}{c} \vdots \\[3mm] \left[\mathbf{W}\right]_{\mathbf{I}_n}\left[\mathbf{v}_k\right]_{\mathbf{I}_n} \\[3mm] \vdots \end{array}\right]\end{array}\right] = \nonumber\\[3mm]
	&= \left[\mathbf{W}\right]_{\mathbf{I}_n} \left[\raisebox{15mm}{} \begin{array}{cccc} \left[\begin{array}{c} \vdots \\[3mm]  \left[\mathbf{v}_1\right]_{\mathbf{I}_n} \\[3mm] \vdots \end{array}\right] & \left[\begin{array}{c} \vdots \\[3mm] \left[\mathbf{v}_2\right]_{\mathbf{I}_n} \\[3mm] \vdots \end{array}\right] & ... & \left[\begin{array}{c} \vdots \\[3mm] \left[\mathbf{v}_k\right]_{\mathbf{I}_n} \\[3mm] \vdots \end{array}\right]\end{array}\right] = \left[\mathbf{W}\right]_{\mathbf{I}_n}\left[\mathbf{V}\right]_{\mathbf{I}_n} .
\end{align}

	Using the same equation we get $\left[\mathbf{W}\right]_{\mathbf{V}} = \left[\mathbf{V}\right]_{\mathbf{I}_n}\left[\mathbf{W}\right]_{\mathbf{I}_n}$, meaning

\begin{equation} \left[\mathbf{V}\right]_{\mathbf{W}} = \left[\mathbf{W}\right]_{\mathbf{I}_n}\left[\mathbf{W}\right]_{\mathbf{V}}\left[\mathbf{W}\right]_{\mathbf{I}_n}^{-1} \end{equation}

	\noindent and the fact that the representation of any basis under $\mathbf{I}_n$ is taken for its simplicity. It must be noted that $\left[\mathbf{W}\right]_{\mathbf{I}_n}^{-1}$ exists because $\mathbf{W}$ is a basis, therefore it has linearly independent columns.
\hfill$\blacksquare$
\vspace{5mm}
\hrule
\vspace{5mm} %>>>


	Seen as solving a linear differential equation is, in essence, finding subspaces of functions unchanged by differentiation and a particular functional alike, and we know which functions are unchanged by differentiation, we want to know what functions are unchanged by the functional being studied so we can find the intersection. Pick a basis $\mathbf{V}$ in $\left[\mathbb{R}\to\mathbb{C}^n\right]$, and pick a vector $\mathbf{x} = \left[x_1,x_2,...,x_n\right]^\transpose_\mathbf{V}$. Then

\begin{equation} \mathbf{A}\left[\mathbf{x}\right] = \mathbf{A}\left[\sum\limits_{k=1}^n x_k\mathbf{v}_k\right] = \sum\limits_{k=1}^n x_k \mathbf{A}\left[\mathbf{v}_k\right] \label{eq:application_decomp_1}\end{equation}

	\noindent therefore, for an arbitrary vector $\mathbf{x}$, finding $\mathbf{A}\left[\mathbf{x}\right]$ can be made easier if we just know the vectors $\mathbf{A}\left[\mathbf{v}_k\right]$, that is, how $\mathbf{A}\left[\cdot\right]$ acts on the basis of vectors. Since this application is a vector itself, each $\mathbf{A}\left[\mathbf{v}_k\right]$ has a coordinate with respect to $\mathbf{V}$, say, 

	\begin{equation} \mathbf{A}\left[\mathbf{v}_k\right] = \sum\limits_{i=1}^n y_{\left(i,k\right)} \mathbf{v}_i , k\in\mathbb{N}_n^*\label{eq:application_decomp_2}\end{equation}

	\noindent which means that the coordinates of $\mathbf{A}\left[\mathbf{v}_k\right]$ on the basis $\mathbf{V}$ is

\begin{equation} \left[\raisebox{1mm}[2mm][2mm]{} \mathbf{A}\left[\mathbf{v}_k\right]\right]_\mathbf{V} = \left[\begin{array}{c} y_{\left(1,k\right)} \\[3mm] y_{\left(1,k\right)} \\[3mm] \vdots \\[3mm] y_{\left(n,k\right)} \end{array}\right] \end{equation}

	and that the representation of the operator $\mathbf{A}\left[\cdot\right]$ on $\mathbf{V}$ is 

\begin{equation} \left[\raisebox{0mm}[1mm][1mm]{} \mathbf{A}\right]_\mathbf{V} =  \left[\raisebox{15mm}{} \begin{array}{cccc} \left[\begin{array}{c} \vdots \\[3mm] \mathbf{A}\left[\mathbf{v}_1\right] \\[3mm] \vdots \end{array}\right] & \left[\begin{array}{c} \vdots \\[3mm] \mathbf{A}\left[\mathbf{v}_2\right] \\[3mm] \vdots \end{array}\right] & ... & \left[\begin{array}{c} \vdots \\[3mm] \mathbf{A}\left[\mathbf{v}_n\right] \\[3mm] \vdots \end{array}\right] \end{array}\right] = \left[\begin{array}{cccc} y_{1,1} & y_{1,2} & \cdots & y_{1,n} \\[3mm] y_{2,1} & y_{2,2} & \cdots & y_{2,n} \\[3mm] \vdots & \vdots & \ddots & \vdots \\[3mm] y_{n,1} & y_{n,2} & \cdots & y_{n,n}\end{array}\right]. \label{eq:application_decomp_3}\end{equation}

	Then \eqref{eq:application_decomp_1} becomes

\begin{align} \mathbf{A}\left[\mathbf{x}\right] &= \mathbf{A}\left[\sum\limits_{k=1}^n x_k\mathbf{v}_k\right] = \sum\limits_{k=1}^n x_k \left(\sum\limits_{i=1}^n y_{\left(i,k\right)} \mathbf{v}_i\right) = \nonumber\\[3mm]
%
	&= \left[\raisebox{15mm}{} \begin{array}{cccc} \left[\begin{array}{c} \vdots \\[3mm] \mathbf{v}_1 \\[3mm] \vdots \end{array}\right] & \left[\begin{array}{c} \vdots \\[3mm] \mathbf{v}_2 \\[3mm] \vdots \end{array}\right] & ... & \left[\begin{array}{c} \vdots \\[3mm] \mathbf{v}_n \\[3mm] \vdots \end{array}\right]\end{array}\right]\left[\begin{array}{cccc} y_{1,1} & y_{1,2} & \cdots & y_{1,n} \\[3mm] y_{2,1} & y_{2,2} & \cdots & y_{2,n} \\[3mm] \vdots & \vdots & \ddots & \vdots \\[3mm] y_{n,1} & y_{n,2} & \cdots & y_{n,n}\end{array}\right]\left[\begin{array}{c} x_1 \\[3mm] x_2 \\[3mm] \vdots \\[3mm] x_n\end{array}\right] = \nonumber \\[3mm]
%
	&= \mathbf{V} \left[\mathbf{A}\right]_\mathbf{V}\left[\mathbf{x}\right]_\mathbf{V}
\end{align}

	\noindent meaning that the application of $\mathbf{A}$ on any vector $\mathbf{x}$ can be found if we just know the representation of $\mathbf{A}$ through $\mathbf{V}$, \textbf{given that for every} $\mathbf{A}\left[\mathbf{v}_k\right]$ a representation \eqref{eq:application_decomp_2} can be found, that is, if $\mathbf{A}$ when operated through a set of linearly independent vectors generates another set of linearly independent vectors. Equivalently, this means that the image of the entire space $\left[\mathbb{R}\to\mathbb{C}^n\right]$ through $\mathbf{A}\left[\cdot\right]$ is n-dimensional, that is, the entirety of the space.

	It might be that such is not the case — maybe $\mathbf{A}\left[\cdot\right]$ applied to a basis $\mathbf{V}$ generates a set of $d\leq n$ independent vectors, meaning the image of the entire space $\left[\mathbb{R}\to\mathbb{C}^n\right]$ through $\mathbf{A}\left[\cdot\right]$ has a smaller dimension $i$ (it is ``smaller'' than the original space). In some sense, the functional ``shrinks'' the original space as it ``squashes'' or ``vanishes'' some dimensions.

	One can wonder if this process depends on the basis chosen, that is, if $\mathbf{A}\left[\cdot\right]$ generates a set of linearly independent vectors for different bases. One direct consequence of theorem \ref{theo:bases_similarity} is that if the generating set chosen is a basis, then this characteristic remains for any other complete basis chosen; consequently, the capacity of a linear mapping to produce linearly independent vector is unwaivering to the chosen basis. By choosing the canonical basis we conclude that if $\mathbf{A}$ has $d\leq n$ linearly independent columns (or rows) then it generates a subspace of dimension $d\leq n$. This is known as \textbf{rank}, denoted $\rank\left(\mathbf{A}\right)$, that is, when $\mathbf{A}\left[\cdot\right]$ operates a basis it generates a basis of dimension $d$ which may (if $d=n$) or may not (if $d < n$) be a basis. More deeply, when $\mathbf{A}\left[\cdot\right]$ operates the entire space it is defined in, it generates a space of dimension $d$. If $d=n$, we say that $\mathbf{A}$ is of \textbf{complete rank}.

	One asks then ``what happened'' to the other dimensions, or rather, what does $\mathbf{A}\left[\cdot\right]$ causes to those extra dimensons. To answer this, we first define the concept of a Kernel, that is the pre-image of the null vector by $\mathbf{A}\left[\cdot\right]$.

\begin{definition}[Kernel of a mapping] The \textbf{kernel} of a mapping $\mathbf{A}\left[\cdot\right]$ is the counter-image of zero, that is, set of vectors that are mapped to the null vector:
	
\begin{equation} \Ker\left(\mathbf{A}\right) = \left\{\mathbf{x}\in\Dom\left(\mathbf{A}\right): \mathbf{A}\left[\mathbf{x}\right] = \mathbf{0}\right\} .\end{equation}
\end{definition}

	Particularly for finite dimensional linear maps, the kernel is also called the \textbf{nullspace} because for this class of maps the kernel is itself a vector space, that is, a subspace of $\Dom\left(\mathbf{A}\right)$. The dimension of the Kernel is called the \textbf{nullity} of $\mathbf{A}$, denoted $\nullity\left(\mathbf{A}\right)$. As per \eqref{eq:def_image}, the subspace generated by the vectors produced by $\mathbf{A}\left[\cdot\right]$ is the image of $\mathbf{A}$, denoted

\begin{equation} \Im\left(\mathbf{A}\right) = \left\{ \mathbf{A}\left[\mathbf{x}\right]: \mathbf{x}\in\Dom\left(\mathbf{A}\right)\right\}\end{equation}

	\noindent which is essentially the collection of all possible outputs of $\mathbf{A}$. What theorem \ref{theo:rank_nullity} states is that, basically, that if the space generated by $\mathbf{A}$ has dimension less than $n$, than this is because some of these dimensions are brought to the null vector, that is, they are ``squished'' into that vector.

\begin{theorem}[Rank-nullity theorem]\label{theo:rank_nullity} %<<<
	For a linear mapping $\mathbf{A}\left[\cdot\right]$, 

\begin{equation} \text{Im}\left(\mathbf{A}\right) \cup \Ker\left(\mathbf{A}\right) = \text{Dom}\left(\mathbf{A}\right) \end{equation}

	\noindent which is equivalent to

\begin{equation} \rank\left(\mathbf{A}\right) + \nullity\left(\mathbf{A}\right) = \text{dim}\left(\text{Dom}\left(\mathbf{A}\right)\right).\end{equation}
\end{theorem}
\noindent\textbf{Proof.} If $\mathbf{A}$ is of complete rank, the proof is done because since it forms a basis, the only vector that maps to the null vector is the null vector itself, meaning that the kernel has a single element (the null vector) hence the nullity of $\mathbf{A}$ is zero. Let us assume then that $\mathbf{A}$ has $d<n$ linearly independent columns, that is, its image has dimension $d<n$.

	It is simple to see that the kernel of $\mathbf{A}$ is also a subspace, meaning there is a basis for it. Let $\mathbf{V} = \left\{\mathbf{v}_1,...,\mathbf{v}_k\right\}$ be such a base. Then choose any collection of linearly independent $n-k$ vectors $\mathbf{W} = \left\{\mathbf{w}_1,...,\mathbf{w}_{(n-k)}\right\}\in\Dom\left(\mathbf{A}\right)\setminus\Ker\left(\mathbf{A}\right)$, so that $\mathbf{V}\cup\mathbf{W}$ is a basis for $\Dom\left(\mathbf{A}\right)$. The existence of $\mathbf{W}$ is guaranteed by the fact that the domain of $\mathbf{A}$ has dimension $n$, therefore $n-k$ vectors linearly independent among themselves and from the $\mathbf{v}_k$ can be found. The theorem claims that $\mathbf{W}$ generates the image of $\mathbf{A}$.

	Indeed, because $\mathbf{V}\cup\mathbf{W}$ is a basis of the domain, then any $\mathbf{x}$ can be written as some linear combination of the $\mathbf{v}_k$ and the $\mathbf{w}_k$: 

\begin{equation} \mathbf{x} = \sum_{k=1}^k \alpha_k \mathbf{v}_k + \sum_{k=1}^{n-k} \beta_k \mathbf{w}_k\label{eq:theo_ranknull_1}\end{equation}

	\noindent but since the $\mathbf{v}_k$ are in the kernel,

\begin{equation} \mathbf{A}\left[\mathbf{x}\right] = \mathbf{A}\left[\sum_{k=1}^k \alpha_k \mathbf{v}_k + \sum_{k=1}^{n-k} \beta_k \mathbf{w}_k\right] = \sum_{k=1}^k \alpha_k \mathbf{A}\left[\mathbf{v}_k\right] + \sum_{k=1}^{n-k} \beta_k \mathbf{A}\left[\mathbf{w}_k\right] = \sum_{k=1}^{n-k} \beta_k \mathbf{A}\left[\mathbf{w}_k\right].\end{equation}

	Meaning that the vector produced by $\mathbf{A}\left[\mathbf{x}\right]$ is a linear combination of the $\mathbf{w}_k$, that is, $\mathbf{W}$ generates the image of $\mathbf{A}$.
\hfill$\blacksquare$
\vspace{5mm}
\hrule
\vspace{5mm} %>>>

%-------------------------------------------------
\section{Invariant subspaces and eingenstuff} %<<<1

	Let us adopt $\mathbf{U}_\mathbf{A} = \left\{\mathbf{u}_1,\mathbf{u}_2,...,\mathbf{u}_d\right\}$ as a basis of this space. Then \eqref{eq:application_decomp_2} has to be adjusted as

\begin{equation} \mathbf{A}\left[\mathbf{u}_k\right] = \sum\limits_{i=1}^d y_{\left(ik\right)} \mathbf{u}_i, k\in\mathbb{N}_d^* \Rightarrow \left[\raisebox{0mm}[1mm][1mm]{} \mathbf{A}\right]_\mathbf{U} = \left[\begin{array}{cccc} y_{11} & y_{12} & \cdots & y_{1d} \\[3mm] y_{21} & y_{22} & \cdots & y_{2d} \\[3mm] \vdots & \vdots & \ddots & \vdots \\[3mm] y_{n1} & y_{n2} & \cdots & y_{nd}\end{array}\right].\label{eq:application_decomp_4}\end{equation}

	Also, any vector $\mathbf{z}$ in this particular space can be written as some combination of the vectors of $\mathbf{U}$:

\begin{equation} \mathbf{z} = \sum\limits_{i=1}^d z_k \mathbf{u}_k \Rightarrow \mathbf{A}\left[\mathbf{z}\right] = \sum\limits_{i=1}^d z_i\left(\sum\limits_{i=1}^d  y_{\left(i,k\right)} \mathbf{u}_k\right) = \mathbf{U} \left[\mathbf{A}\right]_\mathbf{U}\left[\mathbf{z}\right]_\mathbf{U} .\label{eq:application_decomp_3}\end{equation}


	But note that this equation implies $\mathbf{A}\left[\mathbf{z}\right]$ is $\mathbf{U}$ multiplied by a vector $\left[\mathbf{A}\right]_\mathbf{U}\left[\mathbf{z}\right]_\mathbf{U}$, that is, $\mathbf{A}\left[\mathbf{z}\right]$ admits a set of coordinates in $\mathbf{U}$ because this equation is simply a change of coordinates (see \eqref{eq:coordinate_transform}). This yields that $\mathbf{A}\left[\mathbf{z}\right]$ belongs to $E\left(\mathbf{A}\right)$. Reestated, every vector in $E\left(\mathbf{A}\right)$, when operated through the mapping, is still inside that subspace — or conversely, this subspace is an \textbf{invariant subspace} under $\mathbf{A}\left[\cdot\right]$.

	Invariant subspaces are a major point of concern in mathematics and span multiple areas of applied sciences. One of the reasons for this is because the idea of such subspaces begets the process of \textbf{spectral decomposition} of a linear map, or a matrix. Because $E\left(\mathbf{A}\right)$ is of dimension $d\leq n$, then any set of $d$ linearly independent vectors in this subset generates the entirery of the subset. Particularly, let us pick the basis $\mathbf{V}$ such that the $\mathbf{v}_k$ fulfill

\begin{equation} \mathbf{A}\left[\mathbf{v}_k\right] = \lambda\mathbf{v}_k\end{equation}

	\noindent for some non-null complex $\lambda$. This means that the representation of $\mathbf{A}\left[\cdot\right]$ through $\mathbf{V}$ is

\begin{equation} \left[\raisebox{0mm}[1mm][1mm]{} \mathbf{A}\right]_\mathbf{V} = \left[\begin{array}{cccccc} \lambda_1 & 0 & \cdots & 0 & \cdots & 0 \\[3mm] 0 & \lambda_2 & \cdots & 0 & \cdots & 0 \\[3mm] \vdots & \vdots & \ddots & \vdots  & \cdots & 0\\[3mm] 0 & 0 & \cdots & \lambda_d & \cdots & 0\end{array}\right]_{\left(d\times n\right)} \end{equation}

% TWO DIMENSIONAL PARALLELOGRAM <<<
\begin{figure}[t]
\centering
\scalebox{1}{
	\begin{tikzpicture}[scale=1.5,>={Stealth[inset=0mm,length=1.5mm,angle'=50]}]

		\clip (-2mm,-2mm) rectangle (105mm,53mm); % For some reason this needs to be here, otherwise the caption is spaced lower from the picture?
		\node (origin) at (0,0) {};
	
		\foreach \p in {5,10,...,35} {
			\draw [gray] (\p mm,-2mm) -- (\p mm, 40mm);
			\draw [gray] (-2mm, \p mm) -- (40mm, \p mm);
		}
		
		\draw [->, thick] (   -2mm,  0   ) -- (   40mm,  0   ) node (xaxis) {};
		\draw [->, thick] (      0, -2mm ) -- (   0   ,  40mm) node (yaxis) {};


		\node[above] at (yaxis) {$\mathbf{e}_2$};
		\node[right] at (xaxis) {$\mathbf{e}_1$};

		\node (v2) at (15mm,35mm) {};
		\draw[fill=white,draw=none] (v2) circle(2mm);
		\node[stewartgreen!50!black] (v2label) at (v2) {$\mathbf{v}_2$};

		\node (v1) at (25mm,10mm) {};
		\draw[fill=white,draw=none] (v1) circle(2mm);
		\node[red] (v1label) at (v1) {$\mathbf{v}_1$};

		\node (sum) at ($(v2) + (v1)$) {};

		\draw [white, line width = 1mm] ($(origin)!0.05!(v2)$) -- (v2);
		\draw [->, thick, stewartgreen!50!black] (0,0) -- (v2);

		\draw [white, line width = 1mm] ($(origin)!0.05!(v1)$) -- (v1);
		\draw [->, thick, red] (0,0) -- (v1);


		\node (u) at (20mm,25mm) {};
		\draw[fill=white,draw=none] (u) circle(2mm);
		\node[black] (ulabel) at (u) {$\mathbf{u}$};

		\draw [white, line width = 1mm] ($(origin)!0.05!(u)$) -- (u);
		\draw [->, thick, black] (0,0) -- (u);

		\node [label={[font=\Huge]90:$\mathbb{R}^2$}] at(20mm,45mm) {};

		\begin{scope}[xshift=60mm]
			\node (origin) at (0,0) {};
			\node (v2) at (15mm,35mm) {};
			\node (v1) at (25mm,10mm) {};

			\begin{scope}
				\clip (-2mm,-2mm) rectangle (42mm,40mm);
				\foreach \p in {-10,-5,...,35} {
					\node (node\p) at (\p mm,0mm) {};
					\node (nodev\p) at (0mm, \p mm) {};
					\draw [gray] ($(node\p) - 0*(v2)$) -- ($(node\p) + 2*(v2)$);
					\draw [gray] ($(nodev\p) - 0*(v1)$) -- ($(nodev\p) + 2*(v1)$);
				}

			\end{scope}

			\node[red] (v2label) at ($1.15*(v2)$) {};
			\node (v2) at (v2label) {};
			\draw[fill=white,draw=none] (v2) circle(2.5mm);
			\node[stewartgreen!50!black] (v2labelagain) at (v2label) {$\lambda_1\mathbf{v}_2$};

			\node[red] (v1label) at ($1.7*(v1)$) {};
			\node (v1) at (v1label) {};
			\draw[fill=white,draw=none] (v1) circle(2.5mm);
			\node[red] (v1labelagain) at (v1label) {$\lambda_1\mathbf{v}_1$};

			\draw [line width = 1mm, white] ($(origin)!0.05!(v2)$) -- (v2labelagain);
			\draw [->, thick, stewartgreen!50!black, name path=v2path] (0,0) -- (v2labelagain);
        		
			\draw [line width = 1mm, white] ($(origin)!0.05!(v1)$) -- (v1labelagain);
			\draw [->, thick, red, name path=v1path] (0,0) -- (v1labelagain);

			\node (u) at (35mm,35mm) {};
			\draw[fill=white,draw=none] (u) circle(2mm);
			\node[black] (ulabel) at (u) {$\mathbf{u}$};

			\draw [white, line width = 1mm] ($(origin)!0.05!(u)$) -- (u);
			\draw [->, thick, black] (0,0) -- (u);

			\path[name path=utov1] (u) -- ($(u) - 1.25*(v2)$);
			\draw[dashed, line cap = round, name intersections={of=utov1 and v1path, by={uintv1}}] (u) -- (uintv1);
 			\node (uintv1label) at ($(uintv1) - 0.05*(v2)$) {};
			\draw[fill=white,draw=none] (uintv1label) circle(1.5mm) {};
			\node[black,red] at (uintv1label.center) {$\alpha_1$};

			\path[name path=utov2] (u) -- ($(u) - 1.5*(v1)$);
			\draw[dashed, line cap = round, name intersections={of=utov2 and v2path, by={uintv2}}] (u) -- (uintv2);
 			\node (uintv2label) at ($(uintv2) - 0.05*(v1)$) {};
			\draw[fill=white,draw=none] (uintv2label) circle(1.5mm) {};
			\node[black,stewartgreen!50!black] at (uintv2label.center) {$\alpha_2$};

			\node [label={[font=\Huge]90:$\mathbf{A}\mathbb{R}^2$}] at(20mm,45mm) {};

		\end{scope}

			\draw[-{Stealth[inset=0mm,length=6mm,angle'=50]}, line width=3mm] (45mm,20mm) -- (55mm,20mm);
			\node [label={[font=\Huge]90:$\mathbf{A}$}] at(50mm,22mm) {};
	\end{tikzpicture}
	}
	\caption{Eigendecomposition of the $\mathbb{R}^2$ space through eigenvectors of a matrix.}
	\label{fig:eigendecomposition}
\end{figure} %>>>

	\noindent we want to find the $d$ linearly independent vectors which, when operated through $\mathbf{A}\left[\cdot\right]$, are simply multiplied by some non-null factor, that is, they are simply ``stretched'' or ``compressed'' but they are not eliminated nor ``change directions''. Finding such vectors has the immediate benefit in that operating $\mathbf{A}\left[\cdot\right]$ on an arbitrary vector $\mathbf{x} = \left[x_1,x_2,...,x_d\right]^\transpose_\mathbf{V}$ is made much simpler: whereas the computation of $\mathbf{A}\left[\mathbf{x}\right]$ is worksome, if $\mathbf{x}$ is written with respect to the basis of vectors $\mathbf{V}$, then

\begin{equation} \mathbf{x} = \sum\limits_{i=1}^d x_i\mathbf{v}_i \Rightarrow \mathbf{A}\left[\mathbf{x}\right] = \mathbf{A}\left(\sum\limits_{i=1}^d x_i\mathbf{v}_i\right) = \sum\limits_{i=1}^d x_i\mathbf{A}\left[\mathbf{v}_i\right] = \sum\limits_{i=1}^d x_i\lambda_i\mathbf{v}_i \ ,\end{equation}

	\noindent that is, the operation is made to be a simple scaling of the $\mathbf{v}_k$ through the $\lambda_k$ and the components of $\mathbf{x}$. Such vectors $\mathbf{v}$ are then called the \textit{eigenvectors} of $\mathbf{A}$ (from the german prefix \textit{eigen}, meaning ``specific'', ``inherent''). Equivalently, finding $\mathbf{v}$ means finding the kernel of the linear mapping $\left(\mathbf{A}\left[\mathbf{x}\right] - \lambda\mathbf{x}\right)$. Once a canonical representation for $\Dom\left(\mathbf{A}\right)$ is adopted, this is equivalent to finding the kernel of the matrix $\left(\mathbf{A} - \lambda\mathbf{I}\right)$. The invariant subspace of $\mathbf{A}$, which is generated by applying it to a basis of $n$ linearly independent vectors, is called the \textbf{eigenspace} of $\mathbf{A}\left[\cdot\right]$, denoted $\Eig\left(\mathbf{A}\right)$, and the reason is simple: it is the space generated by its eigenvectors. Such eigenvectors can be calculated through

\begin{equation} \mathbf{Av} - \lambda\mathbf{v} = \mathbf{0} \Leftrightarrow \left(\mathbf{A} - \lambda\mathbf{I}\right)\mathbf{v} = \mathbf{0} \label{eq:eigenvector_def} \end{equation}

	\noindent where $\mathbf{v}\neq \mathbf{0}$ because the trivial solution does not generate any space. The question now becomes how to compute eigenvectors and eigenvalues; for this, we introduce the \textbf{determinant}. Let us imagine a real matrix $\mathbf{A}$ on the space $\mathbb{R}^2$, as in figure \ref{fig:determinant}. The canonical vectors, $\mathbf{e}_1$ and $\mathbf{e}_2$, make a square of side 1. When $\mathbf{A}$ operates these vectors it generates two other vectors that form a parallelogram $P$. It must be noted that, by the definition of matrix-by-vector multiplication, $\mathbf{A}\left[\mathbf{e}_1\right]$ is the first column of $\mathbf{A}$ and $\mathbf{A}\left[\mathbf{e}_2\right]$ is its second column. The determinant is the area of this paralellogram of none or both vectors are inverted; if one of the vectors is inversed, the determinant is the negative area. Therefore, the determinant is the measure of ``how much'' the matrix $\mathbf{A}$ distorts (``expands'' or ``shrinks'') the original space, and how much its operator $\mathbf{A}\left[\cdot\right]$ expands its own original space. The determinant is negative if $\mathbf{A}$ ``reverses'' the space.

% TWO DIMENSIONAL PARALLELOGRAM <<<
\begin{figure}[htb!]
\centering
\scalebox{0.8}{
	\begin{tikzpicture}[scale=1.5,>={Stealth[inset=0mm,length=1.5mm,angle'=50]}]
		\node (origin) at (0,0) {};
		\draw [->] (   -2mm,  0   ) -- (   40mm,  0   ) node (xaxis) {};
		\draw [->] (      0, -2mm ) -- (   0   ,  40mm) node (yaxis) {};

		\node[above] at (yaxis) {$\mathbf{e}_2$};
		\node[right] at (xaxis) {$\mathbf{e}_1$};

		\node[stewartgreen!50!black] (v2) at (15mm,35mm) {};
		\node[red] (v1) at (25mm,10mm) {};

		\node (sum) at ($(v2) + (v1)$) {};

		\fill [opacity=0.2,black](origin.center) -- (v1.center) -- (sum.center) -- (v2.center) -- (origin.center);
        
		\draw [->, stewartgreen!50!black] (0,0) -- (v2);
		\draw [->, red] (0,0) -- (v1);
		\node[label={[color = stewartgreen!50!black, label distance=0.5mm]120:$\mathbf{A\left[e_2\right]}$}] at ($(origin)!0.8!(v2)$) {};
		\node[label={[color = red, label distance=0.5mm]300:$\mathbf{A\left[e_1\right]}$}] at ($(origin)!0.8!(v1)$) {};

		\node at ($(origin)!0.5!(sum)$) {$P$};

		\draw[stewartgreen!50!black,dashed,line cap = round] (v2) -- (sum) {};
		\draw[red,dashed,line cap = round] (v1) -- (sum) {};
	\end{tikzpicture}
	}
	\caption{representation of the two-dimension parlalelogram generated by a transformation $\mathbf{a}$.}
	\label{fig:determinant}
\end{figure} %>>>

	In $\mathbb{R}^3$, take the vectors $\mathbf{A}\left[\mathbf{e}_1\right]$, $\mathbf{A}\left[\mathbf{e}_2\right]$ and $\mathbf{A}\left[\mathbf{e}_3\right]$, which are by definition the first, second and third columns of $\mathbf{A}$. Then the determinant is the positive volume of the cube defined by these vectors if none or two vectors are inverted, and the negative of the volume if one or three vectors are inverted. Generically, in the $\mathbb{R}^n$, the determinant is the hypervolume of the hypercube defined by

\begin{equation} P_n = \left\{\mathbf{x}\in\mathbb{R}^n: \mathbf{x} = t_1\mathbf{A}\left[\mathbf{e}_1\right] + t_2\mathbf{A}\left[\mathbf{e}_2\right] + ... + t_n\mathbf{A}\left[\mathbf{e}_n\right],\ t_k\in\left[0,1\right]\right\} \end{equation}

	\noindent where the $\mathbf{c}_k$ are the columns of $\mathbf{A}$. The determinant is negative if $\mathbf{A}$ reverses the orientation of the space, and positive if not. The common formulas for determinants, as in

\begin{equation} \det\left(\left[\begin{array}{cc} a & b \\[3mm] c & d\end{array}\right]\right) = ad - bc \label{eq:2d_det}\end{equation}

	\noindent and

\begin{equation} \det\left(\left[\begin{array}{ccc} a & b & c \\[3mm] d & e & f \\[3mm] g & h & i\end{array}\right]\right) = aei + bfg + cdh - gec - dbi - afh \label{eq:3d_det}\end{equation}

	\noindent can be calculated from the discussion on $\mathbb{R}^2$ and $\mathbb{R}^3$ and are supposed. This definition becomes nonsensical, however, once the analysis goes from real spaces to complex spaces because a ``complex volume'' is not a well-defined idea. However, it is simple to see that the formulas \eqref{eq:2d_det} and \eqref{eq:3d_det} themselves hold even if the entries are complex, hinting at the fact that while the geometric motivation is not suitable for complex spaces, the resulting formulas remain. Therefore, the notion of a determinant can be more formally placed as an operator on complex matrix space (a function that takes a matrix and delivers a number) that has several operational properties.

\begin{definition}[Determinant of a complex matrix]\label{def:determinant} Let $\mathbf{A}\in\mathbb{C}^{(n\times n)}$ a complex matrix with columns $\mathbf{c}_k\ k\in\mathbb{N}_n^*$. Then the determinant is a matrix function defined with the following properties.

\noindent\textbf{(P1)} Switching two columns changes the sign of the determinant;

\noindent\textbf{(P2)} The determinant is multilinear on the columns: suppose any column, say $\mathbf{c}_1$, is the linear combination of two over vectors, say $\mathbf{c}_1 = a\mathbf{v} + b\mathbf{w}$ for two complex numbers $a,b$. Then

\begin{gather}
	\det\left(\left[\raisebox{15mm}{} \begin{array}{cccc} \left[\begin{array}{c} \vdots \\[3mm] a\mathbf{v} + b\mathbf{w} \\[3mm] \vdots \end{array}\right] & \left[\begin{array}{c} \vdots \\[3mm] \mathbf{c}_2 \\[3mm] \vdots \end{array}\right] & ... & \left[\begin{array}{c} \vdots \\[3mm] \mathbf{c}_n \\[3mm] \vdots \end{array}\right]\end{array}\right]\right) = \nonumber \\[3mm]
%
	a\det\left(\left[\raisebox{15mm}{} \begin{array}{cccc} \left[\begin{array}{c} \vdots \\[3mm] \mathbf{v} \\[3mm] \vdots \end{array}\right] & \left[\begin{array}{c} \vdots \\[3mm] \mathbf{c}_2 \\[3mm] \vdots \end{array}\right] & ... & \left[\begin{array}{c} \vdots \\[3mm] \mathbf{c}_n \\[3mm] \vdots \end{array}\right]\end{array}\right]\right) 
	+ b\det\left(\left[\raisebox{15mm}{} \begin{array}{cccc} \left[\begin{array}{c} \vdots \\[3mm] \mathbf{w} \\[3mm] \vdots \end{array}\right] & \left[\begin{array}{c} \vdots \\[3mm] \mathbf{c}_2 \\[3mm] \vdots \end{array}\right] & ... & \left[\begin{array}{c} \vdots \\[3mm] \mathbf{c}_n \\[3mm] \vdots \end{array}\right]\end{array}\right]\right) 
\end{gather}

\noindent\textbf{(P3)} The determinant is invariant to transposition, that is, $\det\left(\mathbf{A}\right) = \det\left(\mathbf{A}^\transpose\right)$; and

\noindent\textbf{(P4)} The determinant of the indentity matrix $\mathbf{I}_n$ is $1$.
\end{definition}

	It can be shown that due to property \textbf{(P3)} then properties \textbf{(P1)} and \textbf{(P2)} are also valid for rows, that is, switching two rows also changes the sign of the determinant and that the determinant is multilinear on the rows. Further, it can also be shown that this definition \ref{def:determinant} defines a unique function in the space of square complex matrices, and from this definition all properties of determinants can be drawn.

	One of the most important properties of determinants is that if a certain matrix is such that its columns (or its rows) are not linearly independent (that is, there is some non-trivial linear combination of the columns that results the null vector) then its determinant is null; intuitively, this means that because the columns are linearly dependent they form a subspace of dimension lower than $n$, that is, the matrix degenerates the space into a lower-sized one by ``squashing'' dimensions. Therefore, in the $n$-th dimensional space, the hypercube defined has volume zero.

	However, because the matrix does not have linearly independent columns it does not form a basis over the space — therefore by theorem \ref{theo:invertiblematrix} the matrix is not invertible. At the same time, if the matrix is not invertible, by the theorem this can only mean it is not a basis, therefore its eigenvectors do not form an n-dimensional space, therefore its determinant is zero. Theorem \ref{theo:nulldet} shows this result.

\begin{theorem}[Null determinant of invertible matrices]\label{theo:nulldet} %<<<
	A matrix is invertible if and only if it has non-null determinant. \end{theorem}
\noindent\textbf{Proof.} First we must note that by the property \textbf{(P1)} of determinants, switching two columns inverts the signal of the determinant. Hence, if a matrix has two equal columns it must have zero determinant, since swapping those columns inverts the determinant but the matrix remains the same; therefore the same determinant. This means the determinant is its own opposite, therefore it must be zero.

	Now take 

\begin{equation}
	\mathbf{A} = \left[\raisebox{15mm}{} \begin{array}{cccc} \left[\begin{array}{c} \vdots \\[3mm] \mathbf{c}_1 \\[3mm] \vdots \end{array}\right] & \left[\begin{array}{c} \vdots \\[3mm] \mathbf{c}_2 \\[3mm] \vdots \end{array}\right] & ... & \left[\begin{array}{c} \vdots \\[3mm] \mathbf{c}_n \\[3mm] \vdots \end{array}\right]\end{array}\right]
\end{equation}

	\noindent such that there is a linear combination among the columns, say,

\begin{equation} \mathbf{c}_1 = \sum\limits_{k=2}^n \alpha_k\mathbf{c}_k .\end{equation}

	Then use the property \textbf{(P2)} of determinants to write 

\begin{align}
	\det\left(\mathbf{A}\right) &= \det\left( \left[\raisebox{15mm}{} \begin{array}{cccc} \left[\begin{array}{c} \vdots \\[3mm] \sum\limits_{k=2}^n \alpha_k\mathbf{c}_k \\[3mm] \vdots \end{array}\right] & \left[\begin{array}{c} \vdots \\[3mm] \mathbf{c}_2 \\[3mm] \vdots \end{array}\right] & ... & \left[\begin{array}{c} \vdots \\[3mm] \mathbf{c}_n \\[3mm] \vdots \end{array}\right]\end{array}\right]\right) = \\[3mm]
%
	&= \sum\limits_{k=2}^n \alpha_k\det\left( \left[\raisebox{15mm}{} \begin{array}{cccc} \left[\begin{array}{c} \vdots \\[3mm] \mathbf{c}_k \\[3mm] \vdots \end{array}\right] & \left[\begin{array}{c} \vdots \\[3mm] \mathbf{c}_2 \\[3mm] \vdots \end{array}\right] & ... & \left[\begin{array}{c} \vdots \\[3mm] \mathbf{c}_n \\[3mm] \vdots \end{array}\right]\end{array}\right]\right) 
\end{align}

	\noindent and note that each matrix in the summation has two identical columns: the first and the k-th. Therefore all determinants are zero, therefore $\det\left(\mathbf{A}\right) = 0$. Now we prove that a null determinant yields non-invertibility. Assuming a null determinant, then consider the determinant of the matrix $\mathbf{B}_1$ which first column is some arbitrary linear combination of the columns of $\mathbf{A}$:

\begin{equation} \det\left(\mathbf{B}_1\right) = \det\left( \left[\raisebox{15mm}{} \begin{array}{cccc} \left[\begin{array}{c} \vdots \\[3mm] \sum\limits_{k=1}^n \alpha_k\mathbf{c}_k \\[3mm] \vdots \end{array}\right] & \left[\begin{array}{c} \vdots \\[3mm] \mathbf{c}_2 \\[3mm] \vdots \end{array}\right] & ... & \left[\begin{array}{c} \vdots \\[3mm] \mathbf{c}_n \\[3mm] \vdots \end{array}\right]\end{array}\right]\right) \end{equation}

	\noindent then this determinant is clearly equal to

\begin{equation}  \det\left(\mathbf{B}\right) = \sum\limits_{k=1}^n \alpha_k \det\left( \left[\raisebox{15mm}{} \begin{array}{cccc} \left[\begin{array}{c} \vdots \\[3mm] \mathbf{c}_k \\[3mm] \vdots \end{array}\right] & \left[\begin{array}{c} \vdots \\[3mm] \mathbf{c}_2 \\[3mm] \vdots \end{array}\right] & ... & \left[\begin{array}{c} \vdots \\[3mm] \mathbf{c}_n \\[3mm] \vdots \end{array}\right]\end{array}\right]\right) \end{equation}

	\noindent and the determinant in the sum is zero for all $k\geq 2$ for having two equal columns thence

\begin{equation}  \det\left(\mathbf{B}\right) =  \alpha_1\det\left( \left[\raisebox{15mm}{} \begin{array}{cccc} \left[\begin{array}{c} \vdots \\[3mm] \mathbf{c}_1 \\[3mm] \vdots \end{array}\right] & \left[\begin{array}{c} \vdots \\[3mm] \mathbf{c}_2 \\[3mm] \vdots \end{array}\right] & ... & \left[\begin{array}{c} \vdots \\[3mm] \mathbf{c}_n \\[3mm] \vdots \end{array}\right]\end{array}\right]\right) = \alpha_1\det\left(\mathbf{A}\right) = 0. \end{equation}

	Meaning any linear combiation on the first column will yield a null determinant. Taking a linear combination on the second column,

\begin{equation} \det\left(\mathbf{B}_2\right) = \det\left( \left[\raisebox{15mm}{} \begin{array}{cccc} \left[\begin{array}{c} \vdots \\[3mm] \mathbf{c}_1 \\[3mm] \vdots \end{array}\right] & \left[\begin{array}{c} \vdots \\[3mm] \sum\limits_{k=1}^n \alpha_k\mathbf{c}_k \\[3mm] \vdots \end{array}\right] & ... & \left[\begin{array}{c} \vdots \\[3mm] \mathbf{c}_n \\[3mm] \vdots \end{array}\right]\end{array}\right]\right) \end{equation}

	\noindent and for the same arguments $\det\left(\mathbf{B}_2\right) = 0$. Therefore, for any linear combination of the k-th column $\mathbf{B}_k$, $\det\left(\mathbf{B}_k\right) = 0$. Ultimately, this implies that for any matrix of coefficients $\mathbf{B}$, the product $\mathbf{C = AB}$ will have null determinant, because the matrix $\mathbf{P}$ is composed of linear combinations of the matrix $\mathbf{A}$ weighted by the coefficients of the columns of $\mathbf{B}$.

	We prove by contradiction: if $\mathbf{A}$ is invertible, then its columns are linearly independent; therefore, for any matrix $\mathbf{C}$ we can find a matrix of coefficients $\mathbf{B}$ such that $\mathbf{AB = C}$ and $\det\left(\mathbf{C}\right) = 0$. In other words, if $\mathbf{A}^{-1}$ exists, then $\mathbf{B} = \mathbf{A}^{-1}\mathbf{C}$ reconstructs a chosen matrix $\mathbf{C}$ from the columns of e$\mathbf{A}$, and $\mathbf{C}$ will have null determinant. This is obviously contradictory because this would imply any matrix $\mathbf{C}$ chosen has null determinant. Particularly, we can choose $\mathbf{I}_n$ which by definition has determinant 1 but, if $\mathbf{A}^{-1}$ existed, would have determinant zero — a contradiction.
\hfill$\blacksquare$
\vspace{5mm}
\hrule
\vspace{5mm} %>>>

	This can be seen directly in the definition of eigenvectors \eqref{eq:eigenvector_def}: because the multiplication of a matrix by a vector is equivalent to a linear combination of its columns (by the definition of a matrix-by-vector multiplication), the nullity of the product of a matrix by a non-null vector implies that this matrix is singular because there is a linear combination among the matrix columns; this in turn means

\begin{equation} \text{det}\left(\mathbf{A} - \lambda\mathbf{I}\right) = 0. \label{eq:eigenvalue_def}\end{equation}

	It can be further proven that the determinant \eqref{eq:eigenvalue_def} yields a monic polynomial of degree $n$; this polynomial is called the matrix \textit{characteristic polynomial} of $\mathbf{A}$, denoted

\begin{equation} P_\mathbf{A} (x) = \text{det}\left(x\mathbf{I} - \mathbf{A}\right) .\end{equation}

	Therefore the values of $\lambda$ that satisfy \eqref{eq:eigenvector_def} are also roots of $P_\mathbf{A}$ and are called \textit{eigenvalues} of $\mathbf{A}$. It must be noted that the characteristic polynomial and the eigenvalues are invariant to a basis change: indeed, take $\mathbf{A} = \mathbf{PBP}^{-1}$, $\lambda$ an eigenvalue of the operator $\mathbf{A}$ and $\mathbf{v}$ an eigenvector. Then

\begin{gather}
	\text{det}\left(\mathbf{A} - x\mathbf{I}\right) = 0\nonumber \\[5mm]
	\text{det}\left(\mathbf{PBP}^{-1} - x\mathbf{I}\right) = 0\nonumber \\[5mm]
	\text{det}\left(\mathbf{PBP}^{-1} - x\mathbf{PP}^{-1}\right) = 0\nonumber \\[5mm]
	\text{det}\left(\mathbf{PBP}^{-1} - \mathbf{P}x\mathbf{P}^{-1}\right) = 0\nonumber \\[5mm]
	\text{det}\left[\mathbf{P}\left(\mathbf{B} - x\mathbf{I}\right)\mathbf{P}^{-1}\right] = 0\nonumber \\[5mm]
	\text{det}\left(\mathbf{P}\right)\text{det}\left(\mathbf{B} - x\mathbf{I}\right) \text{det}\left(\mathbf{P}^{-1}\right)  = 0\nonumber \\[5mm]
	\text{det}\left(\mathbf{P}\right)\text{det}\left(\mathbf{B} - x\mathbf{I}\right) \dfrac{1}{\text{det}\left(\mathbf{P}\right)}  = 0\nonumber \\[5mm]
	\text{det}\left(\mathbf{B} - x\mathbf{I}\right) = 0
\end{gather}

	\noindent (here we assume the property of determinant of product and determinant of the inverse). This ultimately means that eigenvalues are uniquely related to an operator regardless of the matrix representation or basis adopted. As for the eigenvectors, if $\mathbf{v}$ is an eigenvector of $\mathbf{A}$ then

\begin{gather}
	\left(\mathbf{A} - \lambda\mathbf{I}\right)\mathbf{v} = \mathbf{0} \nonumber\\[5mm]
	\left(\mathbf{PBP}^{-1} - \lambda\mathbf{I}\right)\mathbf{v} = \mathbf{0} \nonumber\\[5mm]
	\left(\mathbf{PBP}^{-1} - \lambda\mathbf{PP}^{-1}\right)\mathbf{v} = \mathbf{0} \nonumber\\[5mm]
	\mathbf{P}\left(\mathbf{B} - \lambda \mathbf{I}\right)\mathbf{P}^{-1}\mathbf{v} = \mathbf{0} \nonumber\\[5mm]
	\left(\mathbf{B} - \lambda \mathbf{I}\right)\mathbf{P}^{-1}\mathbf{v} = \mathbf{0}
\end{gather}

	\noindent therefore $\mathbf{P}^{-1}\mathbf{v}$ is an eigenvector of $\mathbf{B}$, meaning there is a one-to-one relationship between the eigenvectors of $\mathbf{A}$ and those of $\mathbf{B}$.	Figure \ref{fig:eigendecomposition} shows the process of an \textit{eigendecomposition} of $\mathbb{R}^2$. In the figure, $\mathbf{v}_1$ and $\mathbf{v}_2$ are the eigenvectors of an arbitrary operator $\mathbf{A}$ and $\mathbf{u}$ is an arbitrary vector. On the left, the real plane is first presented in the canonical basis $B = \left\{\mathbf{e}_1,\mathbf{e}_2\right\}$, $\mathbf{e}_1 = \left[1,0\right]^\transpose$ and $\mathbf{e}_2 = \left[0,1\right]^\transpose$. The right side shows the space $\mathbf{A}\mathbb{R}^2$, that is, $\mathbb{R}^2$ operated through $\mathbf{A}$; this new space is topologically equivalent to $\mathbb{R}^2$ but it is ``stretched and bent'' as it is operated. In this new space, the basis equivalent to the canonical is $B' = \left\{\mathbf{Av}_1,\mathbf{Av}_2\right\} = \left\{\lambda_1\mathbf{v}_1,\lambda_2\mathbf{v}_2\right\}$. Suppose that in this new basis the coordinates of $\mathbf{u}$ are $\alpha_1,\alpha_2$; then $\mathbf{u}$ can be written in $\mathbf{A}\mathbb{R}^2$ as $\mathbf{Au} = \alpha_1\lambda_1\mathbf{v}_1 + \alpha_2\lambda_2\mathbf{v}_2$, showing again that the operation of $\mathbf{A}$ onto $\mathbf{u}$ is made much simpler, yielding a sum of vectors.

%-------------------------------------------------
\section{Diagonalizable operators}\label{sec:diagonalizable_operators} %<<<1

	The simplest case of linear mapping is that which eigenspace is the entirety of its original space. This property has many definitions, which are all equivalent and immediate from the discussion of the last section.

\begin{definition}[Diagonal matrix] A diagonal matrix $\mathbf{A}$ is such that all elements but the main diagonal are zero, that is, $a_{ij} = 0$ if $j\neq i$:

\begin{equation} \mathbf{A} = \left[\begin{array}{cccc} a_1 & 0 & ... & 0 \\[3mm] 0 & a_2 & ... & 0 \\[3mm] \vdots & \vdots & \ddots & \vdots \\[3mm] 0 & 0 & ... & a_n \end{array}\right]. \end{equation}
\end{definition}
\begin{definition}[Diagonalizable mapping] A diagonalizable mapping $\mathbf{A}$ is that such that there is a basis $\mathbf{V}$ of $\Dom\left(\mathbf{A}\right)$ such that $\left[\mathbf{A}\right]_\mathbf{V}$ is a diagonal matrix. \end{definition}

\begin{theorem}[Definitions of diagonalizable mapping] The following statements are equivalent: for a linear mapping $\mathbf{A}\left[\cdot\right]$,

\begin{itemize}
	\item $\mathbf{A}\left[\cdot\right]$ is a diagonalizable mapping, that is, there exists a basis $\mathbf{V}$ such that $\left[\mathbf{A}\right]_\mathbf{V}$ is a diagonal matrix;
	\item For any basis $\mathbf{V}$, $\left[\mathbf{A}\right]_\mathbf{V}$ also forms a basis, that is, it has linearly independent columns and linearly independent rows;
	\item For any basis $\mathbf{V}$, the vectors $\mathbf{A}\left[\mathbf{v}_k\right]$ are linearly independent;
	\item The kernel of $\mathbf{A}$ has dimension zero;
	\item The eigenspace of $\mathbf{A}$ has dimension $n$.
\end{itemize}
\end{theorem}

	Since diagonalizing $\mathbf{A}$ means finding a new basis of vectors such that, in this new basis of vectors, $\mathbf{A}$ is diagonal, it becomes clear that the only basis that can achieve this is the basis composed of the eigenvectors of $\mathbf{A}$: because by definition an eigenvector is unwaivering to $\mathbf{A}$, a new basis constructed from the eigenvectors of $\mathbf{A}$ yields a diagonal matrix. Such basis is called an \textbf{eigenbasis} of $\mathbf{A}$.

\begin{theorem}[Diagonalization of a complex matrix] \label{theo:diagonalization} %<<<
	A linear operator $\mathbf{A}$ is diagonalizable if and only if it has $n$ distinct eigenvalues. In this case, its matrix on the canonical basis $\mathbf{A}$ is similar to a diagonal matrix $\mathbf{D}$ given by $\mathbf{A} = \mathbf{P}\mathbf{D}\mathbf{P}^{-1}$, where the columns of $\mathbf{P}$ are the eigenvectors of $\mathbf{A}$

\begin{equation} \mathbf{P} = \left[\raisebox{15mm}{} \begin{array}{cccc} \left[\begin{array}{c} \vdots \\[3mm] \mathbf{v}_1 \\[3mm] \vdots \end{array}\right] & \left[\begin{array}{c} \vdots \\[3mm] \mathbf{v}_2 \\[3mm] \vdots \end{array}\right] & ... & \left[\begin{array}{c} \vdots \\[3mm] \mathbf{v}_n \\[3mm] \vdots \end{array}\right] \end{array}\right], \end{equation}

	and the diagonal of $\mathbf{D}$ is the list of eigenvalues of $\mathbf{A}$

\begin{equation} \mathbf{D} = \left[\begin{array}{ccccc}\lambda_1 & 0 & 0 & ... & 0 \\[3mm] 0 & \lambda_2 & 0 & ... & 0 \\[3mm] \vdots & \vdots & \vdots & \ddots & \vdots \\[3mm] 0 & 0 & 0 & ... & \lambda_n \end{array}\right] \end{equation}

	where the $\lambda_k$ are the $n$ not necessarily different eigenvalues. 

\end{theorem}
\textbf{Proof:} take $\mathbf{P}$ as defined. Then

\begin{align}
	\mathbf{AP} = \mathbf{A}& \left[\raisebox{15mm}{} \begin{array}{cccc} \left[\begin{array}{c} \vdots \\[3mm] \mathbf{v}_1 \\[3mm] \vdots \end{array}\right] & \left[\begin{array}{c} \vdots \\[3mm] \mathbf{v}_2 \\[3mm] \vdots \end{array}\right] & ... & \left[\begin{array}{c} \vdots \\[3mm] \mathbf{v}_n \\[3mm] \vdots \end{array}\right] \end{array}\right] = \left[\raisebox{15mm}{} \begin{array}{cccc} \left[\begin{array}{c} \vdots \\[3mm] \mathbf{A}\mathbf{v}_1 \\[3mm] \vdots \end{array}\right] & \left[\begin{array}{c} \vdots \\[3mm] \mathbf{A}\mathbf{v}_2 \\[3mm] \vdots \end{array}\right] & ... & \left[\begin{array}{c} \vdots \\[3mm] \mathbf{A}\mathbf{v}_n \\[3mm] \vdots \end{array}\right] \end{array}\right] = \nonumber\\[5mm]
%
	= &\left[\raisebox{15mm}{} \begin{array}{cccc} \left[\begin{array}{c} \vdots \\[3mm] \lambda_1\mathbf{v}_1 \\[3mm] \vdots \end{array}\right] & \left[\begin{array}{c} \vdots \\[3mm] \lambda_2\mathbf{v}_2 \\[3mm] \vdots \end{array}\right] & ... & \left[\begin{array}{c} \vdots \\[3mm] \lambda_n\mathbf{v}_n \\[3mm] \vdots \end{array}\right] \end{array}\right] = \nonumber\\[5mm]
%
	= & \left[\raisebox{15mm}{} \begin{array}{cccc} \left[\begin{array}{c} \vdots \\[3mm] \mathbf{v}_1 \\[3mm] \vdots \end{array}\right] & \left[\begin{array}{c} \vdots \\[3mm] \mathbf{v}_2 \\[3mm] \vdots \end{array}\right] & ... & \left[\begin{array}{c} \vdots \\[3mm] \mathbf{v}_n \\[3mm] \vdots \end{array}\right] \end{array}\right]\left[\begin{array}{ccccc}\lambda_1 & 0 & 0 & ... & 0 \\[3mm] 0 & \lambda_2 & 0 & ... & 0 \\[3mm] \vdots & \vdots & \vdots & \ddots & \vdots \\[3mm] 0 & 0 & 0 & ... & \lambda_n \end{array}\right] = \mathbf{PD}
\end{align}

\hfill$\blacksquare$
\vspace{5mm}
\hrule
\vspace{5mm} %>>>
\begin{remark}\label{remark:spectral_matrix} A matrix $\mathbf{P}$ which columns are eigenvectors of $\mathbf{A}$ is called a \textbf{spectral matrix} of $\mathbf{A}$. \end{remark}

	Therefore, it is a direct consequence of the definition of eigenvectors that only a basis of $n$ distinct eigenvectors fulfills the requirement that $\mathbf{A}$ is diagonal under it, which explains why the similarity matrix $\mathbf{P}$ is the collection of eigenvalues of $\mathbf{A}$. Because a basis for a $n$-dimensional vector space needs to have $n$ linearly independent vectors, this is only possible if $\mathbf{A}$ has $n$ linearly independent eigenvalues, then this new basis exists and the diagonalization is possible.

	From the point of view of linear algebra, the process of diagonalizing a linear mapping is, in essence, a change of basis — diagonalization is the process of finding a new basis of vectors under which the map is diagonal. This is only possible if the eigenspace of $\mathbf{A}$ is the entirety of the space it is immersed in. From a differential equations point of view, the diagonalization process is the transformation of the space of the vectors $\mathbf{x}\in\left[\mathbb{R}\to \mathbb{C}^n\right]$ that are solutions to $\dot{\mathbf{x}} = \mathbf{Ax}$ into a new space of functions $\mathbf{z}$ by means of a change of basis; in this new space, $\mathbf{z}$ are solutions to

\begin{equation} \dot{\mathbf{z}} = \mathbf{Dz} = \left[\begin{array}{cccc} \lambda_1 & 0 & ... & 0 \\[3mm] 0 & \lambda_2 & ... & 0 \\[3mm] \vdots & \vdots & \ddots & \vdots \\[3mm] 0 & 0 & ... & \lambda_n \end{array}\right] \end{equation}

	\noindent that is, each component of $\mathbf{z}$ is defined by a simple one-dimensional HLTI ODE $\dot{z}_i = \lambda_i z_i$, which general solution is simple: $z_i(t) = c_ie^{\lambda_i t}$, where $c_i$ is a scalar. In other words, each one-dimensional LTI ODE is self-contained and does not depend on other indexes: the equations are not \textit{coupled}, making what is called a \textit{decoupled} system of differential equations. This decoupling is therefore a direct consequence of the diagonalization process: through the similarity, operating $\mathbf{A}$ onto a vector $\mathbf{x}$ is transformed into a new equation where the components of $\mathbf{z}$ are operated onto themselves. Therefore, the new system $\dot{\mathbf{z}} = \mathbf{Dz}$ is composed of $n$ decoupled single-dimensional ODEs that are simple to solve.

\begin{theorem}[General solution to a diagonalizable homogeneous linear system]\label{theo:diagonalizable_lti_sol} %<<<
	Consider the homogeneous linear system $\dot{\mathbf{x}} = \mathbf{Ax}$. If $\mathbf{A}$ is diagonalizable, then the general solution to this system is

\begin{equation} \mathbf{x}(t) = \sum\limits_{k=1}^n c_ke^{\lambda_k t} \mathbf{v}_k \end{equation}

	\noindent where the $\lambda_k$ are the eigenvalues of $\mathbf{A}$ and $\mathbf{v}_k$ are its eigenvectors. The $c_k$ are calculated using the initial condition

\begin{equation} \left[\begin{array}{c} c_1 \\[3mm] c_2 \\[3mm] \vdots \\[3mm] c_n \end{array}\right] = \mathbf{P}^{-1}\mathbf{x}_0 \end{equation}

	\noindent where $\mathbf{P}$ is the matrix which columns are the eigenvectors of $\mathbf{A}$ and $\mathbf{x}_0 = \mathbf{x}\left(0\right)$.

\end{theorem}
\textbf{Proof.} Two proofs are shown; they are somewhat similar but presented here for completion sake.

\textbf{Proof 1}. Since $\mathbf{A}$ has $n$ linearly independent eigenvectors, then $\mathbf{x}(t)$ can be written as a linear combination of the eigenvectors with time-varying coefficients $z_i$:

\begin{equation} \mathbf{x}(t) = \sum_{k=1}^n z_i(t)\mathbf{v}_i \end{equation}

	and using this definition onto the original HLTI ODE,

\begin{gather}
	\dot{\mathbf{x}} = \mathbf{Ax} \nonumber\\[5mm]
	\dfrac{d}{dt}\left( \sum_{k=1}^n z_k(t)\mathbf{v}_k\right) = \mathbf{A}\left( \sum_{k=1}^n z_k(t)\mathbf{v}_k\right) =  \sum_{k=1}^n z_k(t)\lambda_k\mathbf{v}_k \nonumber\\[5mm]
	\sum_{k=1}^n \dot{z}_k(t) \mathbf{v}_k =  \sum_{k=1}^n z_k(t)\lambda_i\mathbf{v}_k \nonumber\\[5mm]
	\sum_{k=1}^n \left[ \dot{z}_k(t) -z_k(t)\lambda_i\right] \mathbf{v}_k = \mathbf{0}
\end{gather}

	But because the $\mathbf{v}_k$ are linearly independent this can only be true if $\dot{z}_k(t) =  z_k(t)\lambda_k$ is satisfied for each $k$, $1\leq k \leq n$. The solution to this atomized ODE is trivial: $z_k = c_ke^{\lambda_k t}$, with $c_k$ a scalar, yielding

\begin{equation} \mathbf{x}(t) = \sum_{k=1}^n c_ke^{\lambda_k t} \mathbf{v}_k . \end{equation}

\textbf{Proof 2.} Because $\mathbf{A}$ has $n$ different eigenvalues, it is diagonalizable. Therefore, through lemma \ref{lemma:frob_companion_matrix}, there exists an invertible matrix $\mathbf{P}$ such that $\mathbf{D} = \mathbf{P}^{-1}\mathbf{A}\mathbf{P}$ is diagonal and its entries are its eigenvalues, denoted $\lambda_k$, which are the same eigenvalues as $\mathbf{A}$. Not only that, the columns of $\mathbf{P}$ are the eigenvectors $\mathbf{v}_k$ of $\mathbf{A}$:

\begin{equation} \mathbf{P} = \left[\raisebox{15mm}{} \begin{array}{cccc} \left[\begin{array}{c} \vdots \\[3mm] \mathbf{v}_1 \\[3mm] \vdots \end{array}\right] & \left[\begin{array}{c} \vdots \\[3mm] \mathbf{v}_2 \\[3mm] \vdots \end{array}\right] & ... & \left[\begin{array}{c} \vdots \\[3mm] \mathbf{v}_n \\[3mm] \vdots \end{array}\right] \end{array}\right]
\end{equation}

	Now consider the change of variables $\mathbf{z} = \mathbf{P}^{-1}\mathbf{x}$. Then

\begin{equation} \dot{\mathbf{x}} = \mathbf{Ax} \Leftrightarrow \dfrac{d}{dt}\left( \mathbf{P}\mathbf{z}\right) = \mathbf{A} \mathbf{P}\mathbf{z}\end{equation}

	\noindent multiplying both sides on the right by $\mathbf{P}^{-1}$,

\begin{equation} \dot{\mathbf{z}} = \mathbf{P}^{-1}\mathbf{A} \mathbf{P}\mathbf{z} = \mathbf{Dz}\end{equation}

	\noindent and because the matrix $\mathbf{D}$ is diagonal, this system is trivial to solve because it is a system of decoupled LTI ODEs:

\begin{equation} \dot{\mathbf{z}} = \mathbf{Dz} = \left[\begin{array}{ccccc}\lambda_1 & 0 & 0 & ... & 0 \\[3mm] 0 & \lambda_2 & 0 & ... & 0 \\[3mm] \vdots & \vdots & \vdots & \ddots & \vdots \\[3mm] 0 & 0 & 0 & ... & \lambda_n \end{array}\right] \mathbf{z} \label{theo:homogenous_solutions_ltiode_diagonalz}\end{equation}

	which solutions are $z_i = c_i e^{\lambda_i t}$, with the $c_i$ calculated using initial conditions. In vector form,

\begin{equation} \mathbf{z} = \left[\begin{array}{cccc} c_1 & 0 & ... & 0 \\[3mm] 0 & c_2 & ... & 0 \\[3mm] \vdots & \vdots & \ddots & \vdots \\[3mm] 0 & 0 & ... & c_n \end{array}\right] \left[\begin{array}{c} e^{\lambda_1 t} \\[3mm] e^{\lambda_2 t} \\[3mm] \vdots \\[3mm] e^{\lambda_n t} \end{array}\right] \Leftrightarrow \mathbf{x} = \mathbf{P}\left[\begin{array}{cccc} c_1 & 0 & ... & 0 \\[3mm] 0 & c_2 & ... & 0 \\[3mm] \vdots & \vdots & \ddots & \vdots \\[3mm] 0 & 0 & ... & c_n \end{array}\right] \left[\begin{array}{c} e^{\lambda_1 t} \\[3mm] e^{\lambda_2 t} \\[3mm] \vdots \\[3mm] e^{\lambda_n t} \end{array}\right]\end{equation}

	and since the columns of $\mathbf{P}$ are the eigenvectors of $\mathbf{A}$, the product of $\mathbf{P}$ by the diagonal matrix of initial conditions yields

\begin{equation} \mathbf{x}(t) = \sum\limits_{k=1}^n c_ke^{\lambda_k t} \mathbf{v}_k \end{equation}

	which is the same result as proof 1.

\textbf{Calculating the coefficients $c_k$}. Let $\mathbf{x}_0 \vcentcolon = \mathbf{x}(0)$:

\begin{equation} \mathbf{x}_0 = \sum\limits_{k=1}^n c_k \mathbf{v}_k \label{eq:diagonal_initial_cond_calc_1}\end{equation}

	and because the $\mathbf{v}_k$ are linearly independent, the $c_k$ are uniquely defined; another way to see this is to note that \eqref{eq:diagonal_initial_cond_calc_1} is equivalent to

\begin{equation} \left[\raisebox{15mm}{} \begin{array}{cccc} \left[\begin{array}{c} \vdots \\[3mm] \mathbf{v}_1 \\[3mm] \vdots \end{array}\right] & \left[\begin{array}{c} \vdots \\[3mm] \mathbf{v}_2 \\[3mm] \vdots \end{array}\right] & ... & \left[\begin{array}{c} \vdots \\[3mm] \mathbf{v}_n \\[3mm] \vdots \end{array}\right] \end{array}\right] \left[\begin{array}{c} c_1 \\[3mm] c_2 \\[3mm] \vdots \\[3mm] c_n \end{array}\right] = \mathbf{x}_0 \Leftrightarrow \mathbf{P} \left[\begin{array}{c} c_1 \\[3mm] c_2 \\[3mm] \vdots \\[3mm] c_n \end{array}\right] = \mathbf{x}_0 \label{eq:diagonal_initial_cond_calc_2}\end{equation}

	but because the $n$ eigenvectors are distinct and $\mathbf{P}$ is invertible,

\begin{equation} \left[\begin{array}{c} c_1 \\[3mm] c_2 \\[3mm] \vdots \\[3mm] c_n \end{array}\right] = \mathbf{P}^{-1}\mathbf{x}_0 \end{equation}
\hfill$\blacksquare$
\vspace{5mm}
\hrule
\vspace{5mm} %>>>

	In a deeper context, because the operator $\mathbf{A}$ has $n$ distinct eigenvectors, then the set

\begin{equation} \boldsymbol{\Lambda} = \left\{e^{\lambda_1 t}\mathbf{v}_1,e^{\lambda_2 t}\mathbf{v}_2,...,e^{\lambda_n t}\mathbf{v}_n\right\} \end{equation}

	 \noindent is a basis of the space of complex signals that are solutions of $\dot{\mathbf{x}} = \mathbf{Ax}$. The components of $\boldsymbol{\Lambda}$ are called \textbf{modes} of the linear differential equation. Because $\boldsymbol{\Lambda}$ is a basis, any solution to the ODE is a linear combination of the vectors in it, therefore the general solution is given by some arbitrary combination. Reestated, the solutions of an LTI ODE are decomposed into a linear combination of its \textbf{modes} $e^{\lambda t}$: the components of $\mathbf{x} = \left[x_1,x_2,...,x_n\right]$ are decomposed into a new basis $\mathbf{z} = \left[z_1,z_2,...,z_n\right]$ such that the solutions are decoupled in this new basis. Because the eigenvectors of $\mathbf{A}$ form a basis over the vector space, then the solutions of $\dot{\mathbf{z}} = \mathbf{Dz}$ form a basis over the space of its solutions, which is the same space than that of the solutions of $\dot{\mathbf{x}} = \mathbf{Ax}$ but expressed in a different basis. Therefore, the general solution found for the system in $\mathbf{z}$, when transformed back into the space $\mathbf{x}$, is also a general solution to the differential equation in $\mathbf{x}$.

	It cannot be understated that the appearance of the exponential function is only natural: take $\mathbf{D}_{\mathbb{C}^n} \left[\mathbf{x}\right]$ as the differential functional on $\mathbb{C}^n$. It is simple to see that it is linear. Further, calculating the eigenvectors of this operator yields

\begin{gather}
	\mathbf{D}_{\mathbb{C}^n}\left[\mathbf{x}\right] = \lambda\mathbf{x}\nonumber\\[5mm]
	\dot{\mathbf{x}} = \lambda\mathbf{x}\nonumber\\[5mm]
	\dot{\mathbf{x}} = \left(\lambda \mathbf{I}\right)\mathbf{x}
\end{gather}

	\noindent because the identity matrix $\mathbf{I}$ has the canonical basis vectors $\mathbf{e}_k$ as its eigenvectors, then for any complex $\lambda$ the signal $e^{\lambda t}\mathbf{e}_k$ is an eigenvector of the differential operator, which is to say that the exponential function is the eigenvector of the differential operator, explaining why its appearance in the solutions of the linear ODEs. Remarkably, because every complex number generates an eigenvector, this means that the differential operator has as many eigenvectors as there are complex numbers, that is, the eigenspace of the differential operator has uncountably infinite eigenvectors. Formally, if we take the cardinality of the real numbers as $\aleph_0$ (which is also the same cardinality as the complex numbers), then the cardinality of the eigenspace of $\mathbf{D}_{\mathbb{C}^n}$ is also $\aleph_0$. Furthermore, each eigenvector is linearly independent from the other.

 	Therefore, the functional equation $\mathbf{D}\left[\mathbf{x}\right] = \mathbf{A}\left[\mathbf{x}\right]$ is then the restriction of the vector space $\left[\mathbb{R}\to\mathbb{C}^n\right]$ onto the subspace where $\lambda$ are also eigenvalues of $\mathbf{A}$, that is, finding particular functions unchanged by $\mathbf{A}$ and $\mathbf{D}$ alike, that is, the intersection of their invariant subspaces.

%-------------------------------------------------
\section{Defective LTI ODEs} %<<<1

	The gist of theorem \ref{theo:diagonalizable_lti_sol} is that any homogeneous diagonalizable LTI ODE can be written, by its diagonalization, as a simpler ODE \eqref{theo:homogenous_solutions_ltiode_diagonalz} where the variables are decoupled and which solution is easy to gather. The obvious gap in the theorem is that not always will $\mathbf{A}$ be diagonalizable; in other words, not always will an LTI ODE have $n$ linearly independent eigenvectors.

	The consideration of the case where the original LTI ODE \eqref{theo:homogenous_solutions_ltiode_originalode} is not diagonalizable needs a more detailed discussion into the nature of eigendecomposition. If $\mathbf{A}$ has less than $n$ eigenvectors, then the span of these eigenvectors is not $n$-dimensional and a decomposition is impossible. Then the operator $\mathbf{A}\left[\cdot\right]$ and its matrix $\mathbf{A}$ are \textbf{non-diagonalizable} or \textbf{defective}. It must be said that defectiveness is a rare phenomena for complex matrices: almost every complex matrix is diagonalizable, for the subset of $\mathbb{C}^{(n\times n)}$ of non-diagonalizable matrices is meagre, that is, it is nowhere dense on $\mathbb{C}^{(n\times n)}$. Intuitively, this can mean a lot of notions. First, that if a non-diagonalizable matrix is found, all matrices ``close to it'' are surely diagonalizable. Second, that defective matrices are ``spaced out''; finally, that statistically, in a process where matrices are picked randomly, there is zero probability of picking a diagonalizable one.

	It needs to be asserted that there is a distinction between the number of distinct eigenvalues a matrix may have and the number of linearly independent eigenvectors. Fundamentally, there is a difference between the dimension of the invariant subspaces under a linear transformation and its number of eigenvalues. This causes a gap between how many distinct eigenvalues there are, and the capacity of the eigenvectors to generate a vector space. Because two eigenvectors pertaining to two distinct eigenvalues are linearly independent by nature, if a matrix has $n$ distinct eigenvalues then there are necessarily $n$ linearly independent eigenvectors; therefore, if an operator is diagonalizable, the invariant subspaces are easy to find by means of the spans of its eigenvectors. On the other hand, if an operator is not diagonalizable it fails to offer enough independent eigenvectors to form an eigenbasis that covers the whole field it acts upon, which is to say that the eigenvectors of its matrix do not span the whole $\mathbb{C}^n$.

	In order for a linear operator to not have $n$ distinct eigenvalues, at least one eigenvalue $\lambda$ has an \textbf{algebraic multiplicity} $\mu\left(\lambda\right)$ greater than one, that is, it is a higher root (double, triple \textit{et cetera}) of the characteristic polynomial $P_\mathbf{A}(x)$. If the number of eigenvectors of $\lambda$ does not match its algebraic multiplicity — the eigenvectors of $\lambda$ do not form a basis over a space that has dimension $\mu\left(\lambda\right)$ — then the eigenvectors of $\mathbf{A}$ do not form an $n$-dimensional basis, which is to say $\mathbf{A}$ does not have complete rank. Because of this, a distinction is made: the number of linearly independent eigenvectors pertaining to a certain eigenvalue $\lambda$ is called the \textbf{geometric multiplicity} of $\lambda$, noted $\gamma\left(\lambda\right)$. This value certainly coincides with the dimension of the span of the eigenvectors pertaining to $\lambda$. Hence, while $\mu\left(\lambda\right)$ represents the multiplicity of $\lambda$ as a root of the characteristic polynomial, $\gamma\left(\lambda\right)$ represents how many dimensions the eigenvectors of $\lambda$ can express, or rather, the dimension of their span.

	The span of the eigenvectors related to a eigenvalue $\lambda$ is called the \textbf{eigenspace} of $\lambda$, denoted $E_{\mathbf{A}}\left(\lambda\right)$. Because the span of the eigenvectors is the set of all linear combinations of these eigenvalues, and a linear combination of eigenvectors is an eigenvector itself, then $E_\mathbf{A}\left(\lambda\right)$ can be defined as the set of all eigenvectors related to $\lambda$:

\begin{equation} E_\mathbf{A}\left(\lambda\right) = \left\{\mathbf{v}\in\Dom\left(\mathbf{A}\right): \mathbf{A}\left[\mathbf{v}\right] - \lambda\mathbf{v} = \mathbf{0}\right\} \end{equation}

	\noindent therefore, the geometric multiplicity of an eigenvalue can also be defined as the dimension of its eigenspace; in matrix form,

\begin{equation} E_\mathbf{A}\left(\lambda\right) = \left\{\mathbf{v}\in\mathbb{C}^n : \left(\mathbf{A} - \lambda\mathbf{I}\right)\mathbf{v} = \mathbf{0}\right\} \end{equation}

	\noindent which is precisely $\Ker\left(\mathbf{A} - \lambda\mathbf{I}\right)$; $\gamma\left(\lambda\right)$ is the dimension of this nullspace, or nullity of $\left(\mathbf{A} - \lambda\mathbf{I}\right)$ denoted $\mathnull\left(\mathbf{A} - \lambda\mathbf{I}\right)$. And the eigenspace of $\mathbf{A}$ is defined as the union of all the eigenspaces:

\begin{equation} E\left(\mathbf{A}\right) = \bigcup_{\lambda\in\rho\left(\mathbf{A}\right)} E_\lambda\left(\mathbf{A}\right), \end{equation}

	\noindent where $\rho\left(\mathbf{A}\right)$ represents the \textit{spectrum} of $\mathbf{A}$, that is, the list of its eigenvalues. Because two eigenvectors associated with distinct eigenvalues are linearly independent (which is simple to prove), the eigenspaces are mutually disjoint (up to the null vector) by definition and this relation can also be expressed in terms of the direct sum

\begin{equation} E\left(\mathbf{A}\right) = E\left(\lambda_1\right)\oplus E\left(\lambda_2\right) \oplus ... \oplus E\left(\lambda_j\right) = \bigoplus_{\lambda\in \rho\left(\mathbf{A}\right)} E\left(\lambda\right). \end{equation}

	\noindent and naturally $E\left(\mathbf{A}\right)$ is the largest subspace that is invariant to $\mathbf{A}$. Saying $\mathbf{A}$ has a rank $k$ means that $E\left(\mathbf{A}\right)$ has dimension $k$. Because the eigenspaces are mutually disjoint, it is simple to see that the dimension of the entire eigenspace of the operator is the sum of the dimensions of the eigenspaces:

\begin{equation} \text{dim}\left(E\left(\mathbf{A}\right)\right) = \sum_{\lambda\in \rho\left(\mathbf{A}\right)} \text{dim}\left(E\left(\lambda\right)\right). \end{equation}

	Ideally, $\mathbf{A}$ has $n$ distinct eigenvectors — equivalent to saying it is of complete rank — which means $E\left(\mathbf{A}\right)$ is the whole $\mathbb{C}^n$. This in turn allows us to express any vector in $\mathbb{C}^n$ as a linear combination of eigenvectors, thus generating the entire image of $\mathbf{A}$ through eigenvectors. As shown in section \ref{sec:diagonalizable_operators}, this greatly simplifies analysis but can only be done if $\mathbf{A}$ has $n$ distinct eigenvectors — thus being diagonalizable — and particularly, it holds if all eigenvalues have unitary algebraic multiplicity.

	The differentiation between algebraic and geometric multiplicities causes diverse phenomena in linear transformations; constructing interesting examples is a difficult matter because, as said earlier, the set of defective matrices of $\mathbb{C}^{(n\times n)}$ is meagre, that is, it is statistically impossible to pick a defective matrix randomly and examples must be manufactured spefically. In a trivial example, the null matrix $\mathbf{0}$ of order $n$ has an eigenvalue $0$ of algebraic multiplicity $n$ and geometric multiplicity $n$, because the canonical vectors are eigenvectors of $\mathbf{0}$. For a non-trivial example, consider

\begin{equation} \mathbf{M} = \left[\begin{array}{ccc} -1 & -16 & 20 \\[3mm] 1 & -9 & 5 \\[3mm] 4 & -16 & 15\end{array}\right] \label{eq:nonsimple_matrix_example}\end{equation}

	\noindent which has an eigenvalue $\lambda_1 = -5$ with algebraic multiplicity two, and a simple eigenvalue $\lambda_2 = 15$. Calculating the eigenvectors related to $\lambda_1$ yields

\begin{equation} \left[\begin{array}{ccc} 4 & -16 & 20 \\[3mm] 1 & -4 & 5 \\[3mm] 4 & -16 & 20\end{array}\right]\mathbf{v} = \mathbf{0}\end{equation}

	\noindent which yields three equivalent equations

\begin{equation} v_1 - 4v_2 + 5v_3 = 0. \end{equation}

	This equation gives the insight that there are two degrees of freedom pertaining to $\lambda_1$: choosing $v_2 = 0$ yields one eigenvector $\left[-5,0,1\right]^\transpose$ and choosing $v_3 = 0$ yields $\left[4,1,0\right]^\transpose$. Indeed, $\lambda_1$ has two associated eigenvectors, therefore its geometrical multiplicity is two. As for $\lambda_2 = 15$, its eigenvector is $\left[4,1,4\right]^\transpose$. This means that, although $\mathbf{M}$ does not have $n$ distinct eigenvalues, it does have $n$ distinct eigenvectors, making diagonalization possible; indeed, adopting a similarity matrix $\mathbf{P}$ which columns are the eigenvectors yields

\begin{equation} \mathbf{P} = \left[\begin{array}{ccc} -5 & 4 & 4 \\[3mm] 0 & 1 & 1 \\[3mm] 1 & 0 & 4\end{array}\right] \end{equation}

	\noindent therefore $\mathbf{P}^{-1}\mathbf{MP}$ is a diagonal matrix which entries are the eigenvalues of $\mathbf{P}$:

\begin{equation} \mathbf{P}^{-1}\mathbf{MP} = \left[\begin{array}{ccc} -5 & 0 & 0 \\[3mm] 0 & -5 & 0 \\[3mm] 0 & 0 & 15\end{array}\right] \end{equation}

	In general, for any eigenvalue $\lambda$ the inequality $1 \leq \mu\left(\lambda\right) \leq \gamma\left(\lambda\right) \leq n$ is always true. In the case that $\mu\left(\lambda\right) > \gamma\left(\lambda\right)$, $\lambda$ is called a \textit{defective eigenvalue}; if a certain matrix has at least one defective eigenvalue, it is surely defective. However, if $\mu\left(\lambda\right) = \gamma\left(\lambda\right) = 1$, $\lambda$ is called a \textit{simple eigenvalue}. If a certain matrix eigenvalues are all simple, then the matrix has $n$ distinct eigenvalues and $n$ distinct eigenvectors, therefore it is diagonalizable; this is the expected case for most matrices. Finally, if $\mu\left(\lambda\right) = \gamma\left(\lambda\right) \geq 1$, $\lambda$ is called \textit{semisimple eigenvalue}; if a certain matrix has a combination of simple and semisimple eigenvalues it still has $n$ distinct eigenvectors — therefore being diagonalizable — despite not having $n$ distinct eigenvalues.

	For differential equations, the defectiveness of a matrix $\mathbf{A}$ means that there is no linear combination of the components of $\mathbf{x}$ such that the transformation of $\dot{\mathbf{x}} = \mathbf{Ax}$ yields an uncoupled LTI ODE $\dot{\mathbf{z}} = \mathbf{Dz}$ or, in simpler terms, the ODE does not have enough modes to decompose its space of solutions. Again, the difference between algebraic and geometric multiplicities of eigenvalues causes interesting phenomena in linear operators, which translates into equally interesting phenomena in differential equations. Particularly, even if not all eigenvalues of an LTI ODE are simple, this does not necessarily mean that the diagonalization is impossible, because the other eigenvalues can be semisimple. For instance, take the trivial system $\dot{\mathbf{x}} = \mathbf{0x}$, which solution is a constant vector. This is the general solution because the null matrix $\mathbf{0}$ is diagonalizable, despite being blatantly singular. For a non-trivial example, consider

\begin{equation} \dot{\mathbf{x}} = \mathbf{Mx}, \end{equation}

	\noindent where $\mathbf{M}$ is the matrix in \eqref{eq:nonsimple_matrix_example}. The fact that $\mathbf{M}$ is still diagonalizable but has only two eigenvalues (natural modes), one being semisimple, means that the solution $x(t) = c_1e^{-5t} + c_2e^{15t}$ has only two modes, in spite of the expected three modes of a three-dimensional system — yet these two modes are sufficient to fully describe the space of solutions of ODE; hence this solution is general.

%-------------------------------------------------
\section{Jordan decomposition and generalized eigenvectors}

	Seen as the invariant subspaces of a diagonalizable operator are simple to find as the direct sum of eigenspaces, it becomes a direct consequence that the invariant subspace of an otherwise defective operator are not equal to the union of its eigenspaces due to the simple fact that the eigenspaces are not able to amount to the entire invariant subspace under the transformation, immediately bringing the question of how to find such invariant subspaces for defective operators. Furthermore, it is also clear that a diagonalization is in essence the ``simplest form'' an operator can have, and it is also natural to question what is the closest thing any non-diagonalizable operator may have, that is, what is the ``simplest possible'' matrix $\mathbf{J}$ that is similar to the operator matrix $\mathbf{A}$.

	In short, supposing a certain matrix $\mathbf{A}$ of some linear map $\mathbf{A}\left[\cdot\right]$ is diagonalizable, then

\begin{equation} \Dom\left(\mathbf{A}\right) = \bigoplus_{\lambda\in\rho\left(\mathbf{A}\right)} E_\mathbf{A}\left(\lambda\right)\label{eq:dom_bigosum_eigenspace}\end{equation}

	\noindent where $E_\mathbf{A}\left(\lambda\right)$ is the eigenspace pertaining to the eigenvalue $\lambda$ and is defined as the span of all eigenvectors associated with that eigenvalue, which is $E_\mathbf{A}\left(\lambda\right) = \Ker\left(\mathbf{A} - \lambda\mathbf{I}_n\right)$. This happens if all eigenvalues have the same algebraic and geometric multiplicities, that is, every eigenvector has the capacity to generate an $\mu_\mathbf{A}\left(\lambda\right)$-dimensional space.

	If the matrix is defective, the algebraic and geometric multiplicities are not equal — this means that there is a ``gap'' between the algebraic multiplicity $\mu\left(\lambda\right)$ of a certain eigenvalue and its capacity to generate a space with that dimension, which is $\gamma\left(\lambda\right)$. In short, there are not enough distinct eigenvectors to make a basis over $\mathbb{C}^n$ and \eqref{eq:dom_bigosum_eigenspace} is not true anymore, that is, the direct sum of eigenspaces is ``smaller'' than the domain. Therefore some more vectors are needed to complete the basis. This section aims to prove that the vectors sought are obtainable through a notion of \textit{generalized eigenvectors} such that

\begin{equation} \Dom\left(\mathbf{A}\right) = \Im\left(\mathbf{A}\right) = \bigoplus_{\lambda\in\rho\left(\mathbf{A}\right)} G_\mathbf{A}\left(\lambda\right)\label{eq:dom_bigosum_eigenspace_general}\end{equation}

	\noindent where $G_\mathbf{A}\left(\lambda\right)$ are the generalized eigenspaces generated by these generalized eigenvectors. Further, these generalized eigenspaces are produced by the kernels of powers of $\mathbf{A} - \lambda\mathbf{I}_n$ up to the power of the multiplicity of $\lambda$, that is, 

\begin{equation} G_\mathbf{A}\left(\lambda\right) = \bigcup_{k=1}^{\mu_\mathbf{A}\left(\lambda\right)} \Ker\left(\mathbf{A} - \lambda\mathbf{I}_n\right)^k. \end{equation}

%	Generalizing eigenvectors means generating new vectors that augment the defective list of ordinary eigenvectors to make a new basis; the ``broader'' eigenvectors are found through a sequence of vectors called a Jordan Chain. Such chains are constructed by calculating eigenvectors of the powers of $\left(\mathbf{A} - \lambda\mathbf{I}\right)$, that is, a second-order generalized eigenvector $\mathbf{u}_2$ is the solution to $\left(\mathbf{A} - \lambda\mathbf{I}\right)^2\mathbf{u}_2 = \mathbf{0}$; a third order eigenvector is the solution to $\left(\mathbf{A} - \lambda\mathbf{I}\right)^3\mathbf{u}_3 = \mathbf{0}$ and so on.
%
%	Using this chain, it can be shown that the generalized eigenvectors obtained are linearly independent, which is the required property in order to build a basis of vectors. Using the similarity matrix $\mathbf{P}$ which columns are the ordinary and generalized eigenvectors, it can be shown any complex matrix is similar to an ``almost diagonal'' matrix $\mathbf{J}$, called the Jordan Canonical Form, meaning there exists some similarity matrix $\mathbf{P}$ such that $\mathbf{A} = \mathbf{PJP}^{-1}$. The matrix $\mathbf{J}$ is a block diagonal matrix
%
%\begin{equation} \mathbf{J} = \left[\begin{array}{cccc} \mathbf{J}_1 & & & \\[3mm] & \mathbf{J}_2 & & \\[3mm] & & \ddots  & \\[3mm] & & & \mathbf{J}_k \end{array}\right] \label{eq:jordan_form}\end{equation}
%
%	where each block $\mathbf{J}_i$ is called a Jordan Block and is a square matrix of the form
%
%\begin{equation} \mathbf{J}_i = \left[\begin{array}{cccccc} \lambda_i & 1 & 0 & 0 & ... & 0 \\[3mm] 0 & \lambda_i & 1 & 0 & ... & 0 \\[3mm] 0 & 0 & \lambda_i & 1 &... & 0 \\[3mm] \vdots & \vdots & \vdots & \ddots & \ddots & 1 \\[3mm] 0 & 0 & 0 & 0 & ... & \lambda_i \end{array}\right]. \label{eq:jordan_form_i}\end{equation}
%
%	Proving that $\mathbf{J}$ has the form of \eqref{eq:jordan_form} and that the $\mathbf{J}_i$ are of the form \eqref{eq:jordan_form_i} is beyond this text; a good proof is found in \cite{limaAlgebraLinear2021}.
%
%	To every eigenvalue $\lambda$, its geometric multiplicity is the number of blocks that correspond to it; the sum of the sizes of the blocks correspond to the algebraic multiplicity of $\lambda$. The columns of $\mathbf{J}$ are the generalized eigenvectors, and they are a set of $n$ linearly independent vectors; therefore they form a basis over the $n$-dimensional vector space. The columns of $\mathbf{J}$ are the generalized eigenvectors, and they are a set of $n$ linearly independent vectors; therefore they form a basis over the $n$-dimensional vector space.
%
%	It becomes obvious that the Jordan Canonical Form of a diagonalizable matrix yields a diagonal $\mathbf{J}$ and all $\mathbf{J}_i$ are of the form $\left[\lambda_i\right]$, meaning that the process of diagonalizing a matrix is a particular case of building the Jordan Canonical Form. This process is also called a Jordan Decomposition.
%
%	This development means, in practicality, that even if a system $\dot{\mathbf{x}} = \mathbf{Ax}$ is defective, there is still a basis of functions which span is the space of all solutions of the system. As an example, consider the system where
%
%\begin{equation} \mathbf{A} = \left[\begin{array}{cccc} 0 & 1 & 0 & 0\\[3mm] 0 & 0 & 1 & 0 \\[3mm] 0 & 0 & 0 & 1 \\[3mm] 12 & 25 & 13 & -1 \end{array}\right] \end{equation}
%
%	The eigenvalues of $\mathbf{A}$ are a double eigenvalue $\lambda_1 = -1$ and two eigenvalues $\lambda_2 = -3,\ \lambda_3 = 4$ that have single multiplicity; the eigenvectors are
%
%\begin{equation} \mathbf{v}_1 = \left[\begin{array}{c} -1 \\ 1 \\ -1 \\ 1 \end{array}\right],\ \mathbf{v}_2 = \left[\begin{array}{c} 1 \\ -3 \\ 9 \\ -27 \end{array}\right],\ \mathbf{v}_3 = \left[\begin{array}{c} 1 \\ 4 \\ 16 \\ 64 \end{array}\right]\end{equation}
%
%	Clearly $\mathbf{A}$ because only three linearly independent eigenvectors can be found; this trio is not able to make a basis over $\mathbb{C}^4$ and the diagonalization is therefore impossible. Then we need a new vector, $\mathbf{u}$, that is linearly independent to all three ordinary eigenvectors; let us build $\mathbf{u}$ from a Jordan Chain starting with $\mathbf{v}_1$. Then, the vector $\mathbf{u}$ is a second-order generalized eigenvector that satisfies
%
%\begin{equation} \left(\mathbf{A} - \lambda_1\mathbf{I}\right)\mathbf{u}_1 = \mathbf{v}_1\end{equation}
%
%	Multiplying both sides by $\left(\mathbf{A} - \lambda_1\mathbf{I}\right)$,
%
%\begin{equation} \left(\mathbf{A} - \lambda_1\mathbf{I}\right)^2\mathbf{u}_1 = \overbrace{\left(\mathbf{A} - \lambda_1\mathbf{I}\right)\mathbf{v}_1}^{=\mathbf{0}\text{, by definition}} = \mathbf{0}\end{equation}
%
%	And computing $\mathbf{u}_1$ yields
%
%\begin{equation} \mathbf{u}_1 = \left[\begin{array}{c} -3 \\ 2 \\ -1 \\ 0 \end{array}\right] \end{equation}
%
%	Finally, it can be easily inspected that the vectors $\mathbf{u}_1,\ \mathbf{v}_1,\ \mathbf{v}_2,\ \mathbf{v}_3$ are linearly independent, meaning they form a basis $E = \left\{\mathbf{u}_1,\ \mathbf{v}_1,\ \mathbf{v}_2,\ \mathbf{v}_3\right\}$ over $\mathbb{C}^4$, whereas the trio of original eigenvectors could not. Therefore, consider the matrix
%
%\begin{equation}
%	\mathbf{P} = \left[\raisebox{15mm}{} \begin{array}{cccc} \left[\begin{array}{c} \vdots \\[3mm] \mathbf{v}_1 \\[3mm] \vdots \end{array}\right] & \left[\begin{array}{c} \vdots \\[3mm] \mathbf{u}_1 \\[3mm] \vdots \end{array}\right] & \left[\begin{array}{c} \vdots \\[3mm] \mathbf{v}_2 \\[3mm] \vdots \end{array}\right] & \left[\begin{array}{c} \vdots \\[3mm] \mathbf{v}_3 \\[3mm] \vdots \end{array}\right] \end{array}\right] = \left[\begin{array}{cccc} -1 & -3 & 1 & 1 \\[3mm] 1 & 2 & -3 & 4 \\[3mm] -1 & -1 & 9 & 16 \\[3mm] 1 & 0 & -27 & 64 \end{array}\right] 
%\end{equation}
%
%	And it is simple to inspect that the Jordan Canonical Form of $\mathbf{A}$ is given by
%
%\begin{equation} \mathbf{J} = \mathbf{P}^{-1}\mathbf{AP} = \left[\begin{array}{cccc} -1 & 1 & 0 & 0 \\[3mm] 0 & -1 & 0 & 0 \\[3mm] 0 & 0 & -3 & 0 \\[3mm] 0 & 0 & 0 & 4 \end{array}\right] = \left[\begin{array}{ccc} \mathbf{J}_1 & & \\[3mm] & \mathbf{J}_2 & \\[3mm] & & \mathbf{J}_3 \end{array}\right] \end{equation}
%
%	\noindent where $\mathbf{J}_i$ is the Jordan Block of the $i$-th eigenvalue:
%
%\begin{equation} \mathbf{J}_1 = \left[\begin{array}{cc} -1 & 1 \\[3mm] 0 & -1 \end{array}\right],\ \mathbf{J}_2 = \left[-3\right],\ \mathbf{J}_3 = \left[4\right] \end{equation}
%
%	\noindent and, just as proposed, the Jordan Block of the double eigenvalue $\lambda_1 = -1$ has size 2. Therefore, a variable change $\mathbf{z} = \mathbf{P}^{-1}\mathbf{x}$ can be made on $\dot{\mathbf{x}} = \mathbf{Ax}$ to yield
%
%\begin{equation} \dot{\mathbf{z}} = \mathbf{Jz}.\end{equation}
%
%	Under this variable change, $z_2$, $z_3$ and $z_4$ are simple to find as $z_2 = c_2e^{-t}$, $z_3 = c_3e^{-3t}$ and $z_4 = c_4e^{4t}$. $z_1$, on the other hand, has a more complicated ODE:
%
%\begin{equation} \dot{z}_1 = -z_1 + z_2 = -z_1 + c_2e^{-t} \end{equation}
%
%	\noindent and it can be asserted that $z_1 = c_1te^{-t}$ is a solution. Therefore, the general solution to $\dot{\mathbf{x}} = \mathbf{Ax}$ is
%
%\begin{equation} x(t) = c_1 te^{-t}\mathbf{u}_1 + c_2e^{-t}\mathbf{v}_1 + c_3e^{-3t}\mathbf{v}_2 + c_4e^{4t}\mathbf{v}_3 \label{eq:homogenous_solutions_ltiod_double_eigen_ltiode_gensol}\end{equation}
%
%	Further, because the basis $E = \left\{\mathbf{u}_1,\ \mathbf{v}_1,\ \mathbf{v}_2,\ \mathbf{v}_3\right\}$ is a basis over the vector space $\mathbb{C}^4$, then $z_1,z_2,z_3,z_4$ form a basis over the solutions of $\dot{\mathbf{z}} = \mathbf{Jz}$; therefore \eqref{eq:homogenous_solutions_ltiod_double_eigen_ltiode_gensol} is guaranteedly the general solution of \eqref{eq:homogenous_solutions_ltiod_double_eigen_ltiode}.

\begin{definition}[Generalized eigenvectors] \label{def:generalized_eigenvectors} %<<<
	Let $\mathbf{A}\in\mathbb{C}^{\left(n\times n\right)}$. A \textbf{generalized eigenvector} of order or rank $\mathbf{m}$ corresponding to the eigenvalue $\lambda$ of $\mathbf{A}$ is a vector $\mathbf{v}_k$ such that

\begin{equation} \left(\mathbf{A} - \lambda\mathbf{I}\right)^m \mathbf{v}_k = \mathbf{0} \label{def:generalized_eigenvectors_def}\end{equation}

	\noindent but

\begin{equation} \left(\mathbf{A} - \lambda\mathbf{I}\right)^{\left(m-1\right)} \mathbf{v}_k \neq \mathbf{0} \end{equation}
\end{definition} %>>>

\begin{definitionremark} If $\mathbf{v}$ satisfies \eqref{def:generalized_eigenvectors_def} for $m=1$ then it is an ordinary eigenvector. Reestated, a generalized eigenvector of rank 1 is an ordinary eigenvector. \end{definitionremark}

	The next theorems show that the vectors generated by Jordan Chains are linearly independent and that they are generalized eigenvectors; finally, the algorithm for finding a Jordan Chain of Solutions is found, and it is shown that these solutions are indeed linearly independent solutions to the HLTI ODE considered.

\begin{definition}[Jordan Chain] \label{def:jordan_chain} %<<<
	Let $\mathbf{A}\in\mathbb{C}^{\left(n\times n\right)}$, and $\lambda$ one of its eigenvalues with algebraic multiplicity $\mu_\mathbf{A}\left(\lambda\right) = m$. Let $\mathbf{v}$ be a generalized eigenvector of rank $m$. A \textbf{Jordan Chain} of the eigenvector $\mathbf{v}$ is a sequence $\left\{\mathbf{v}_{1},\mathbf{v}_{2},...,\mathbf{v}_{m}\right\}$, with $\mathbf{v}_{1} = \mathbf{v}$, that satisfies

\begin{equation}
\left\{\begin{array}{l}
	\left(\mathbf{A} - \lambda\mathbf{I}\right) \mathbf{v}_{1} = \mathbf{0} \\[3mm]
	\left(\mathbf{A} - \lambda\mathbf{I}\right) \mathbf{v}_{2} = \mathbf{v}_{1} \\[3mm]
	\left(\mathbf{A} - \lambda\mathbf{I}\right) \mathbf{v}_{3} = \mathbf{v}_{2} \\[3mm]
	\hspace{1cm} \vdots \\[3mm]
	\left(\mathbf{A} - \lambda\mathbf{I}\right) \mathbf{v}_{m} = \mathbf{v}_{\left(m-1\right)}
\end{array}\right. \label{eq:def_jordan_chain_seq}
\end{equation}
\end{definition} %>>>

\begin{theorem}[Jordan Chains form generalized eigenvectors] \label{theo:jordan_chains_def2} %<<<
	A Jordan Chain of an eigenvalue $\lambda$ with algebraic multiplicity $m$ of a complex matrix $\mathbf{A}$ is a sequence of sequentially ranked generalized eigenvectors of $\mathbf{A}$ related to the eigenvalue $\lambda$.
\end{theorem}
\textbf{Proof:} from the definition \ref{def:jordan_chain} of Jordan Chains, it can be drawn that for any $p,q$, $1 \leq p \leq q \leq m$, 

\begin{equation} \left(\mathbf{A} - \lambda\mathbf{I}\right)^q \mathbf{v}_{p} = \mathbf{0} \end{equation}

	Therefore, $\mathbf{v}_p$ satisfies the conditions of a $p$-ranked generalized eigenvector, as in

\begin{equation} \left(\mathbf{A} - \lambda\mathbf{I}\right)^p \mathbf{v}_{p} = \mathbf{0} \end{equation}

	but

\begin{equation} \left(\mathbf{A} - \lambda\mathbf{I}\right)^{(p-1)}\mathbf{v}_{p} = \mathbf{v}_1 \neq \mathbf{0} \end{equation}
\hfill$\blacksquare$

\vspace{5mm}
\hrule
\vspace{5mm} %>>>

\begin{theorem}[Linear independence of a Jordan Chain] \label{theo:jordan_chain_li} %<<<
	The vectors of a Jordan Chain are linearly independent.
\end{theorem}
\textbf{Proof:} let $\left\{\mathbf{v}_1,\mathbf{v}_2,...,\mathbf{v}_k\right\}$ be a Jordan Chain pertaining to an eigenvalue $\lambda$, and consider the equation

\begin{equation} \sum\limits_{i=1}^{k} \alpha_i \mathbf{v}_i = \mathbf{0} \label{eq:theo_jordan_chain_li_original} \end{equation}

	where the $\alpha$ coefficients are scalars. The $\mathbf{v}_i$ are linearly independent if this can only be true if all $\alpha$ are zero. Multiply this equation by $\left(\mathbf{A} - \lambda\mathbf{I}\right)$:

\begin{equation} \sum\limits_{i=1}^{k} \alpha_i \left(\mathbf{A} - \lambda\mathbf{I}\right)\mathbf{v}_i = \mathbf{0}. \end{equation}

	By their own definition,

\begin{equation} \sum\limits_{i=2}^{k} \alpha_i \mathbf{v}_{(i-1)} + \overbrace{\left(\mathbf{A} - \lambda\mathbf{I}\right)\mathbf{v}_1}^{=\mathbf{0}} = \mathbf{0} \Rightarrow 	\sum\limits_{i=2}^{k} \alpha_i \mathbf{v}_{(i-1)} = \mathbf{0} \end{equation}

	And multiply again,

\begin{equation} \sum\limits_{i=3}^{k} \alpha_i \mathbf{v}_{(i-2)} + \overbrace{\left(\mathbf{A} - \lambda\mathbf{I}\right)\mathbf{v}_1}^{=\mathbf{0}} = \mathbf{0} \Rightarrow \sum\limits_{i=3}^{k} \alpha_i \mathbf{v}_{(i-2)} = \mathbf{0}\end{equation}

	And this process can be taken recursively until

\begin{equation} \alpha_k \mathbf{v}_{1} = \mathbf{0} \end{equation}

	which can only happen if $\alpha_k = 0$ because the ordinary eigenvalue $\mathbf{v}_1$ is defined as non-null. Therefore, substituting $\alpha_k = 0$ into \eqref{eq:theo_jordan_chain_li_original}

\begin{equation} \sum\limits_{i=1}^{k-1} \alpha_i \mathbf{v}_i = \mathbf{0} \end{equation}

	and the same process can be done recursively to yield $\alpha_{(k-1)} = \alpha_{(k-2)} = ... = \alpha_1 = 0$.

\hfill$\blacksquare$
\vspace{5mm}
\hrule
\vspace{5mm} % >>>

	It stems from theorem \ref{theo:jordan_chain_li} that a Jordan Chain is an effective way to generate a basis for a $\mu\left(\lambda\right)$-dimensional space. However, imagine a certain matrix has an eigenvalue of algebraic multiplicity 5, but geometric multiplicity 3, that is, it relates to three linearly independent eigenvectors $\mathbf{v}_1,\mathbf{v}_2,\mathbf{v}_3$. Of course, the objective is to generate two more generalized eigenvectors so that the trio of ordinary ones can become five. But several questions rise: are the new generalized eigenvectors linearly independent of the ordinary ones? Because, if an ordinary eigenvector $\mathbf{v}_1$ generates two generalized eigenvectors $\mathbf{u}_{1},\mathbf{u}_{2}$, but these are linearly dependent of $\mathbf{v}_2$ and $\mathbf{v}_3$, then the collection $\left\{\mathbf{v}_1,\mathbf{v}_2,\mathbf{v}_3,\mathbf{u}_1,\mathbf{u}_2\right\}$ does not have a span of dimension 5. Or, what if one of the $\mathbf{u}_1$ or $\mathbf{u}_2$ is linearly dependent with eigenvectors of other eigenvalues than $\lambda$?

	Therefore, it needs now to be proven that the eigenspaces generated by the vectors in a Jordan Chain are not only linearly independent from themselves, but also from the chains of other eigenvalues, and also from chains stemming from different eigenvectors.

\begin{theorem} \label{theo:jordan_chain_different_eigenvalues} %<<<
	The generalized eigenvectors of the Jordan Chains of two distinct eigenvalues $\lambda$ are linearly independent.

\end{theorem}
\textbf{Proof:} let $\mathbf{A}$ be the complex matrix in question. Take two distinct eigenvalues $\lambda_1$ and $\lambda_2$, and take $m$ as the smallest algebraic multiplicity between them. Let $\left\{v_1^1,v_1^2,...,v_1^m\right\}$ and $\left\{v_2^1,v_2^2,...,v_2^m\right\}$ be Jordan Chains of $\lambda_1$ and $\lambda_2$ respectively. Consider the equation

\begin{equation} \sum\limits_{i=0}^m \alpha_i \mathbf{v}_1^i + \sum\limits_{i=0}^m \beta_i\mathbf{v}_2^i = \mathbf{0} \label{eq:jordan_chain_different_eigenvalues_example} \end{equation}

	First, multiply it by $\left(\mathbf{A} - \lambda_2\mathbf{I}\right)^m$:

\begin{equation} \sum\limits_{i=0}^m \alpha_i \left(\mathbf{A} - \lambda_2\mathbf{I}\right)^m\mathbf{v}_1^i + \sum\limits_{i=0}^m \beta_i \left(\mathbf{A} - \lambda_2\mathbf{I}\right)^m\mathbf{v}_2^i = \mathbf{0}\end{equation}

	\noindent because $\left(\mathbf{A} - \lambda_2\mathbf{I}\right)^m\mathbf{v}_2^i = \mathbf{0}$ for all $i$ (theorem \ref{theo:jordan_chains_def2}), then this equation is equivalent to 

\begin{equation} \sum\limits_{i=0}^m \alpha_i \left(\mathbf{A} - \lambda_1\mathbf{I}\right)^m\mathbf{v}_1^i = \mathbf{0}\end{equation}

	\noindent and because $\left(\mathbf{A} - \lambda_2\mathbf{I}\right)^m\mathbf{v}_1^i \neq \mathbf{0}$ for all $i$ (by definition), then this equation can only be true for $\alpha_i = 0$. Substituting this into \eqref{eq:jordan_chain_different_eigenvalues_example} yields that all $\beta_i$ need to also be null because the elements of the same Jordan Chain are linearly independent, as per theorem \ref{theo:jordan_chain_li}.
\hfill$\blacksquare$
\vspace{5mm}
\hrule
\vspace{5mm}
% >>>

%\begin{theorem} \label{theo:jordan_chain_different_eigenvectors} %<<<
%	The generalized eigenvectors of the Jordan Chains of two distinct ordinary eigenvectors are linearly independent.
%
%\end{theorem}
%\textbf{Proof:} let $\mathbf{A}$ be the complex matrix in question. Take two distinct eigenvectors $\mathbf{v}_1$ and $\mathbf{v}_2$, $\lambda_1$ and $\lambda_2$ their respective eigenvalues (not necessarily distinct). Let $\left\{v_1^1,v_1^2,...,v_1^{(m_1)}\right\}$ and $\left\{v_2^1,v_2^2,...,v_2^{(m_2)}\right\}$ be Jordan Chains of $\mathbf{v}_1$ and $\mathbf{v}_2$ respectively. Suppose that one of the vectors in the second Jordan Chain can be attained by the linear combination of the vectors in the first chain:
%
%\begin{equation} \sum\limits_{i=1}^{m_1} \alpha_i \mathbf{v}_1^i = \mathbf{v}_2^k \label{eq:jordan_chain_different_eigenectors_example} \end{equation}
%
%	First, multiply it by $\left(\mathbf{A} - \lambda_2\mathbf{I}\right)^k$:
%
%\begin{equation} \sum\limits_{i=1}^{m_1}\alpha_i \left(\mathbf{A} - \lambda_2\mathbf{I}\right)^k\mathbf{v}_1^i = \mathbf{0} \end{equation}
%
%	If $\lambda_1 \neq \lambda_2$, this can opnly happen if
%
%\begin{equation} \left(\mathbf{A} - \lambda_2\mathbf{I}\right)^k\left(\sum\limits_{i=1}^{m_1}\alpha_i \mathbf{v}_1^i\right) = \mathbf{0} \Leftrightarrow \sum\limits_{i=1}^{m_1}\alpha_i \mathbf{v}_1^i = \mathbf{0}\end{equation}
%
%	but because the vectors in a Jordan Chain are linearly independent, this immediately implies all $\alpha_i$ are null, implying $\mathbf{v}_2$ is null, which is a paradox because ordinary eigenvectors are non-null.
%
%	If $\lambda_1 = \lambda_2$, \eqref{eq:jordan_chain_different_eigenectors_example} implies
%
%\begin{equation} \sum\limits_{i=k+1}^{m_1}\alpha_i \left(\mathbf{A} - \lambda\mathbf{I}\right)^k\mathbf{v}_1^i = \mathbf{0} \Leftrightarrow \left(\mathbf{A} - \lambda\mathbf{I}\right)^k\left(\sum\limits_{i=k+1}^{m_1}\alpha_i \mathbf{v}_1^i\right) = \mathbf{0} \Leftrightarrow \sum\limits_{i=k+1}^{m_1}\alpha_i \mathbf{v}_1^i = \mathbf{0}\end{equation}
%
%	Applying this to \eqref{eq:jordan_chain_different_eigenectors_example},
%
%\begin{equation} \sum\limits_{i=1}^{k} \alpha_i \mathbf{v}_1^i = \mathbf{v}_2^k \label{eq:jordan_chain_different_eigenectors_example_2} \end{equation}
%
%	Multiply this by $\left(\mathbf{A} - \lambda_2\mathbf{I}\right)^{(k-1)}$:
%
%\begin{equation} \sum\limits_{i=1}^{k} \alpha_i \left(\mathbf{A} - \lambda_2\mathbf{I}\right)^{(k-1)}\mathbf{v}_1^i = \left(\mathbf{A} - \lambda_2\mathbf{I}\right)^{(k-1)}\mathbf{v}_2^k \label{eq:jordan_chain_different_eigenectors_example_3} \end{equation}
%
%	By definition, this implies
%
%\begin{equation} \alpha_1 \mathbf{v}_1 = \mathbf{v}_2 \end{equation}
%
%	which is impossible because by hypothesis $\mathbf{v}_1$ and $\mathbf{v}_2$ are linearly independent.
%
%\hfill$\blacksquare$
%\vspace{5mm}
%\hrule
%\vspace{5mm}
%% >>>

	Therefore, let us consider that the generalized eigenvectors of a Jordan Chain related to a single eigenvalue $\lambda_k$ are linearly independent; therefore they generate a space of dimension $\mu_\mathbf{A}\left(\lambda_k\right)$, and that space is by definition the generalized eigenspace

\begin{equation} G_{\mathbf{A}} \left(\lambda_k\right) = \bigcup_{i=1}^{\mu_\mathbf{A}\left(\lambda_k\right)} \text{span}\left(\mathbf{v}_i^k\right) =  \bigcup_{i=1}^{\mu_\mathbf{A}\left(\lambda_k\right)} \Ker\left(\mathbf{A} - \lambda_k\mathbf{I}_n\right)^i \end{equation}

	Furthermore, the Jordan Chains of different eigenvectors are linearly independent themselves; therefore not only they form the n-dimentional space $\Dom\left(\mathbf{A}\right)$, the generalized eigenspaces of two distinct eigenvalues are mutually disjoint up to the null vector; therefore, 

\begin{equation} \Dom\left(\mathbf{A}\right) = \bigoplus_{\lambda\in\rho\left(\mathbf{A}\right)} G_{\mathbf{A}} \left(\lambda\right)\end{equation}

%%-------------------------------------------------
\section{Jordan Decomposition} %<<<1

	Theorem \ref{theo:diagonalization} states that for a diagonalizable matrix, the basis comprised of its eigenvectors yields that the representation of that matrix on that basis is a diagonal matrix. What happens then if the operator is not diagonalizable and we choose the basis of the generalized eigenvectors?

	Theorem \ref{theo:jordandecomp} proves what is known as the \textit{Jordan Decomposition}, a direct parallel to theorem \ref{theo:diagonalization}: if the ``generalized eigenbasis'' is taken, that is, a basis of the generalized eigenvectors of a matrix, then that matrix will be similar to its \textit{Jordan Canonical Form}, sometimes simply denoted the ``Jordan Form'', which is basically an ``almost diagonal'' matrix.

\begin{theorem}[Jordan Canonical Form]\label{theo:jordandecomp} %<<<
	Let $\mathbf{A}\in\mathbb{C}^{(n\times n)}$, $\lambda_1,\lambda_2,...\lambda_c$ the list is its eigenvalues, for $c\leq n$ and for each eigenvalue $\lambda_k$ and $m = \mu_\mathbf{A}\left(\lambda_k\right) - 1$. Let

\begin{equation} \mathbf{G}_k = \left[\raisebox{15mm}{} \begin{array}{cccc} \left[\begin{array}{c} \vdots \\[3mm] \mathbf{v}_1^k \\[3mm] \vdots \end{array}\right] & \left[\begin{array}{c} \vdots \\[3mm] \mathbf{v}_2^k \\[3mm] \vdots \end{array}\right] & ... & \left[\begin{array}{c} \vdots \\[3mm] \mathbf{v}_{m}^k \\[3mm] \vdots \end{array}\right]\end{array}\right] \label{eq:jordandecomp}\end{equation}

	\noindent as the basis of generalized eigenvectors of sequential order pertaning to a certain eigenvalue $\lambda_k$, and let 

\begin{equation} \mathbf{G} = \left[ \mathbf{G}_{\lambda_1},\mathbf{G}_{\lambda_2},\cdots,\mathbf{G}_{\lambda_c}\right]\end{equation}
	
	\noindent as the sequential union of such bases. Then $\mathbf{A}$ is similar to its \textbf{Jordan Canonical Form}:

\begin{equation} \mathbf{J} = \left[\begin{array}{cccc} \mathbf{J}_1 & 0 & \cdots & 0 \\[3mm] 0 & \mathbf{J}_2 & \cdots & 0 \\[3mm] \vdots & \vdots & \ddots & \vdots \\[3mm] 0 & 0 & \cdots & \mathbf{J}_c \end{array}\right], \end{equation}

	\noindent where each $\mathbf{J}_i$ is the \textbf{Jordan Block} of the eigenvalue $\lambda_k$:

\begin{equation}
	\mathbf{J}_k = \left[\raisebox{15mm}{} \begin{array}{cccccc} \lambda_k & 1 & 0 & \cdots & 0 & 0 \\[3mm] 0 & \lambda_k & 1 & \cdots & 0 & 0 \\[3mm] 0 & 0 &  \lambda_k & \cdots & 0 & 0 \\[3mm] \vdots & \vdots & \vdots & \ddots & \vdots & \vdots \\[3mm] 0 & 0 & 0 & \cdots & \lambda_k & 1 \\[3mm] 0 & 0 & 0 & \cdots & 0 & \lambda_k\end{array}\right]_{(m\times m)}
\end{equation}

	\noindent and the similarity matrix is $\mathbf{G}$, that is, $\mathbf{A} = \mathbf{GJG}^{-1}$.
\end{theorem}
\noindent\textbf{Proof.} take $\mathbf{G}_k$ as in \eqref{eq:jordandecomp}; then the representation of $\mathbf{A}$ in this base is given by

\begin{equation} \mathbf{A}\mathbf{G}_k = \left[\raisebox{15mm}{} \begin{array}{cccc} \left[\begin{array}{c} \vdots \\[3mm] \mathbf{A}\mathbf{v}_1^k \\[3mm] \vdots \end{array}\right] & \left[\begin{array}{c} \vdots \\[3mm] \mathbf{A}\mathbf{v}_2^k \\[3mm] \vdots \end{array}\right] & ... & \left[\begin{array}{c} \vdots \\[3mm] \mathbf{A}\mathbf{v}_{m}^k \\[3mm] \vdots \end{array}\right]\end{array}\right] .\end{equation}

	And we heavily use the fact that a matrix-by-vector multiplication is essentially the linear combination of the matrix columns with coefficients that are the vector coordinates: because $\mathbf{v}_1^k$ is an ordinary eigenvector, then the first column of $\mathbf{J}_k$ is $\mathbf{A}\mathbf{v}_1^k = \lambda_k \mathbf{v}_1$, that is, $\mathbf{G}_k\left[\lambda_k,0,0,...,0\right]^\transpose $ because it is $\lambda_k\mathbf{v}_1^k + 0\mathbf{v}_2^k + ... + 0\mathbf{v}_{m}^k$. Then, the second column of $\mathbf{J}_k$ is $\mathbf{A}\mathbf{v}_2^k$. But by the Jordan Chain construction,

\begin{equation} \left(\mathbf{A} - \lambda\mathbf{I}_n\right)\mathbf{v}_2^k = \mathbf{v}_1^k \Leftrightarrow \mathbf{A} \mathbf{v}_2^k = \mathbf{v}_1^k + \lambda^k\mathbf{v}_2^k \end{equation}

	\noindent then this second column is $\mathbf{G}_k\left[1,\lambda_k,0,...,0\right]^\transpose $ because it is $\mathbf{v}_1^k + \lambda_k\mathbf{v}_2^k + ... + 0\mathbf{v}_{m}^k$. Taking this process further, we have that the j-th column of $\mathbf{AG}_k$ is $\mathbf{A}\mathbf{v}_j^k$. But by the Jordan Chain construction,

\begin{equation} \left(\mathbf{A} - \lambda\mathbf{I}_n\right)\mathbf{v}_j^k = \mathbf{v}_{(j-1)}^k \Leftrightarrow \mathbf{A} \mathbf{v}_j^k = \mathbf{v}_j^k + \lambda_k\mathbf{v}_{(j-1)}^k .\end{equation}

	Therefore let $\mathbf{J}_k$ such that $\mathbf{AG}_k = \mathbf{G}_k\mathbf{J}_k$; this means that the j-th column of $\mathbf{J}_k$ is compose of an element $1$ in the $j$ position and $\lambda_k$ in the $(j-1)$-th position:

\begin{equation}
	\mathbf{J}_k = \left[\raisebox{15mm}{} \begin{array}{cccccc} \lambda_k & 1 & 0 & \cdots & 0 & 0 \\[3mm] 0 & \lambda_k & 1 & \cdots & 0 & 0 \\[3mm] 0 & 0 &  \lambda_k & \cdots & 0 & 0 \\[3mm] \vdots & \vdots & \vdots & \ddots & \vdots & \vdots \\[3mm] 0 & 0 & 0 & \cdots & \lambda_k & 1 \\[3mm] 0 & 0 & 0 & \cdots & 0 & \lambda_k\end{array}\right]_{(m\times m)}
\end{equation}

	\noindent such that $\mathbf{A}\mathbf{G}_k = \mathbf{G}_k\mathbf{J}_k$. Now let 

\begin{equation} \mathbf{G} = \left[ \mathbf{G}_{\lambda_1},\mathbf{G}_{\lambda_2},\cdots,\mathbf{G}_{\lambda_c}\right]\end{equation}

	\noindent ($c$ being the number of eigenvalues). Then

	\begin{equation} \mathbf{AG} = \mathbf{G}\left[\begin{array}{cccc} \mathbf{J}_1 & 0 & \cdots & 0 \\[3mm] 0 & \mathbf{J}_2 & \cdots & 0 \\[3mm] \vdots & \vdots & \ddots & \vdots \\[3mm] 0 & 0 & \cdots & \mathbf{J}_c \end{array}\right] .\end{equation}

	Call the block matrix as $\mathbf{J}$ and this eqation yields $\mathbf{AG = GJ}$. Because $\mathbf{G}$ has linearly independent columns (since they are made by a Jordan Chain of $\mathbf{A}$), it is invertible; therefore,

\begin{equation} \mathbf{A = GJG}^{-1} \end{equation}

	\noindent meaning $\mathbf{A}$ is similar to $\mathbf{J}$ with a similarity matrix $\mathbf{G}$.
\hfill$\blacksquare$
\vspace{5mm}
\hrule
\vspace{5mm} %>>>
\begin{remark} In line with remark T\ref{remark:spectral_matrix}, a matrix $\mathbf{G}$ which columns are generalized eigenvectors of $\mathbf{A}$ is called a \textbf{generalized spectral matrix} of $\mathbf{A}$. \end{remark}

\begin{corollary}\label{corollary:jordan_nk} %<<<
	A Jordan Block of size $k$ denoted $\mathbf{J}_k$ is equal to

\begin{equation} \mathbf{J}_k = \lambda\mathbf{I}_k + \mathbf{N}_k,\end{equation}

	\noindent where $\mathbf{N}_k$ has ones on the supra-diagonal ($n_{ij} = 1$ if $j = i + 1$ and zero elsewhere), and $\mathbf{N}_k$ is k-th order nilpotent, that is, $\left(\mathbf{N}_k\right)^m = \mathbf{0}$ for $m\geq k$.
\end{corollary}
\noindent\textbf{Proof:} take $\mathbf{N}_k$ 

\begin{equation}
	\mathbf{N}_k = \left[\raisebox{15mm}{} \begin{array}{ccccccc} 0 & 1 & 0 & 0 & \cdots & 0 & 0 \\[3mm] 0 & 0 & 1 & 0 & \cdots & 0 & 0 \\[3mm] 0 & 0 &  0 & 1 & \cdots & 0 & 0 \\[3mm] \vdots & \vdots & \vdots & \vdots & \ddots & \vdots & \vdots \\[3mm] 0 & 0 & 0 & 0 & \cdots & 0 & 1 \\[3mm] 0 & 0 & 0 & 0 & \cdots & 0 & 0 \end{array}\right]_{(k\times k)}
\end{equation}

	Denote $\mathbf{n}_i$ as the i-th column. Now note that $\mathbf{N}_k^2$ is the matrix such that its first column is $\mathbf{N}_k\mathbf{n}_1$, that is, zero; the second column is $\mathbf{N}_k\mathbf{n}_2$, which is one times $\mathbf{n}_1$ and zero elsewhere, meaning it is also null. The third column however is equal to $\mathbf{n}_2$, the fourth is equal to $\mathbf{n}_3$, and so on. Therefore, $\mathbf{N}_k^2$ is a ``copy'' of $\mathbf{N}_k$ but where the columns are ``pushed'' to the right:

\begin{equation}
	\mathbf{N}_k^2 = \left[\raisebox{15mm}{} \begin{array}{ccccccc} 0 & 0 & 1 & 0 & \cdots & 0 & 0 \\[3mm] 0 & 0 & 0 & 1 & \cdots & 0 & 0 \\[3mm] 0 & 0 & 0 & 0 & \cdots & 0 & 0 \\[3mm] \vdots & \vdots & \vdots & \vdots & \ddots & \vdots & \vdots \\[3mm] 0 & 0 & 0 & 0 & \cdots & 0 & 0 \\[3mm] 0 & 0 & 0 & 0 & \cdots & 0 & 0 \end{array}\right]_{(k\times k)}.
\end{equation}

	To obtain $\mathbf{N}_k^3$ the same happens: $\mathbf{N}_k^3$ is a version of $\mathbf{N}_k^2$ where the columns are pushed to the right. And the successive multiplications of $\mathbf{N}_k$ then define an algorithm where the columns are sequentially shifted rightwards; eventually, at the k-th multiplication all columns are shifted and only null elements remain.
\hfill$\blacksquare$
\vspace{5mm}
\hrule
\vspace{5mm} %>>>

	Interestingly, the nilpotency of $\mathbf{J}_k - \lambda\mathbf{I}_k$ implies that a Jordan Chain of an eigenvalue $\lambda$ can have at most $\mu_\mathbf{A}\left(\lambda\right)$ vectors, so that every Jordan Chain is limited.

\begin{corollary}\label{corollary:jordan_nk_at_most} %<<<
	A Jordan Chain of an eigenvalue $\lambda$ has at most $\mu_\mathbf{A}\left(\lambda\right)$ vectors.
\end{corollary}
\noindent\textbf{Proof:} by contradiction. Suppose $\mu_\mathbf{A}\left(\lambda\right)+1$ vectors are possible; then from \eqref{eq:def_jordan_chain_seq}, let $\mathbf{v}_1$ be an ordinary eigenvector of $\lambda$ and

\begin{equation} \left(\mathbf{A} - \lambda\mathbf{I}\right)^{\left(\mu_\mathbf{A}\left(\lambda\right)\right)}\mathbf{v}_1 = \mathbf{0} \label{eq:jordanchain_nilpotent}\end{equation}

	\noindent but since $\left(\mathbf{A} - \lambda\mathbf{I}\right)$ is nilpotent of degree $\mu_\mathbf{A}\left(\lambda\right)$,

\begin{equation} \left(\mathbf{A} - \lambda\mathbf{I}\right)^{\left(\mu_\mathbf{A}\left(\lambda\right)\right)} = \mathbf{0}_{(n\times n)} \end{equation}

	\noindent so that \eqref{eq:jordanchain_nilpotent} would imply any vector at all can be an ordinary eigenvector of $\lambda$, which is false.\hfill$\blacksquare$
\vspace{5mm}
\hrule
\vspace{5mm} %>>>

%-------------------------------------------------
\section{Generalized Jordan Chains} %<<<1

	Up until now, the developments relied on the fact that, to every eigenvalue of a matrix there are linearly independent eigenvectors of a matrix, and for each of these eigenvectors there exists a Jordan Chain can be generated such that the vectors in this chain form a subspace which dimension is the algebraic multiplicity of the eigenvalue, thus completing the needed set required to build the general solution to an LTI ODE. Indeed, theorem \ref{theo:jordan_chain_li_solutions_2} shows that a Jordan Chain generates a set of linearly independent solutions of $\dot{\mathbf{x}} = \mathbf{Ax}$.

\begin{theorem}[Jordan Chains generate solutions] \label{theo:jordan_chain_li_solutions_2} %<<<
	Consider the ODE $\dot{\mathbf{x}} = \mathbf{Ax}$, and let $\left\{\mathbf{v}_1,\mathbf{v}_2,...,\mathbf{v}_m\right\}$ be a Jordan Chain of an eigenvector $\lambda$ of algebraic multiplicity $m$. Then the sequence of functions 

\begin{equation} \mathbf{x}_k(t) =  \left[\displaystyle\sum\limits_{i=0}^{\mu_\mathbf{A}\left(\lambda\right)-1} \dfrac{t^i}{i!}\left(\mathbf{A} - \lambda\mathbf{I}\right)^{i}\right] \mathbf{v}_ke^{\lambda t} \end{equation}

	\noindent is a sequence of linearly independent solutions to $\dot{\mathbf{x}} = \mathbf{Ax}$.
\end{theorem}
\textbf{Proof:} let $\mathbf{J}$ the Jordan Decomposition of $\mathbf{A}$ with $\mathbf{G}$ a generalized spectral matrix of $\mathbf{A}$. And consider the differential equation

\begin{equation}\dot{\mathbf{y}} = \mathbf{Jy}. \end{equation}

	Naturally, the solution of $\dot{\mathbf{x}} = \mathbf{Ax}$ is such that $\mathbf{x} = \mathbf{Gy}$. Let us focus on the $k$-th Jordan Block, that is, the solutions $\mathbf{y}$ in the subspace of the eigenvectors of a particular eigenvalue $\lambda_k$. Let $m = \mu_\mathbf{A}\left(\lambda\right)$ and the ODE on $\mathbf{y}$ yields

\begin{equation}\left\{\begin{array}{l} \dot{y}_1 = \lambda y_1 \\[3mm] \dot{y}_2 = \lambda y_2 + y_1 \\[3mm] \hspace{1cm}\vdots \\[3mm] \dot{y}_{n} = \lambda y_n + y_{(n-1)} \end{array}\right. .\end{equation}

	Thus we can solve for $y_1$ as $y_1 = k_1e^{\lambda t}$ for some scalar $k_1$. Substituting this into the equation of $y_2$ yields $y_2 = k_2te^{\lambda t} + k_1e^{\lambda t}$ for some $k_2,k_1$ scalars. Therefore, we get

\begin{equation} y_j = \left[\displaystyle\sum\limits_{i=1}^{j} \dfrac{k_it^{(i-1)}}{(i-1)!}\right] e^{\lambda t} \end{equation}

	\noindent for a collection of scalars $\left(k_i\right)_{i=1}^j$, for all components $y_j$ for $1\leq j\leq m$. Reconstructing the solution $\mathbf{x}$, we know that $\mathbf{x = Gy}$; but since the matrix-by-vector is a linear combination of the columns of the matrix where the coefficients are the vector coordinates, and the columns of $\mathbf{G}$ are sequentially-ordered generalized eigenvectors of $\mathbf{A}$, the eigenvalue $\lambda_k$ generates a portion of $\mathbf{x}$:

\begin{equation} \mathbf{x} = \sum_{i=1}^m \mathbf{v}_iy_i \end{equation}

	\noindent and we have that this portion of the solution is generated by a linear combination of the sequence

\begin{equation}
\left\{\begin{array}{l}
	\mathbf{x}_1(t) = \mathbf{v}_1 e^{\lambda t} \\[3mm]
	\mathbf{x}_2(t) = \left(t\mathbf{v}_1 + \mathbf{v}_2\right) e^{\lambda t} \\[3mm]
	\mathbf{x}_3(t) = \left(\dfrac{t^2}{2}\mathbf{v}_1 + t\mathbf{v}_2 + \mathbf{v}_3\right) e^{\lambda t} \\[3mm]
	\hspace{2cm} \vdots \\[3mm]
	\mathbf{x}_m(t) = \left[\displaystyle\sum\limits_{i=1}^{m} \dfrac{t^{(m-i)}}{(m-i)!}\mathbf{v}_i \right] e^{\lambda t}
\end{array}\right. \label{eq:jordan_chain_sols_seq}
\end{equation}

	\noindent and it is simple to prove by inspection that these vectors are linearly independent and solve $\dot{\mathbf{x}} = \mathbf{Ax}$.
\hfill$\blacksquare$
\vspace{5mm}
\hrule
\vspace{5mm} %>>>

	Naturally, the general solution of $\dot{\mathbf{x}} = \mathbf{Ax}$ is a linear combination of all of the components of all sequences of the form \eqref{eq:jordan_chain_sols_seq} calculated for each eigenvalue. The issue is that computing every sequence for every ordinary eigenvector can be a very workful task, which prompts us to ask what is the most generalized way one can define a Jordan Chain. Reestated, what is the most relaxed condition for a certain set of generalized eigenvectors to generate a $\mu_\lambda$-dimensional space?

\begin{theorem} \label{theo:jordan_group} %<<<
	Let $\mathbf{A}$ be a complex matrix, $\lambda$ one of its eigenvalues, $\mu_\lambda = m$. Then there exists a set $\left\{\mathbf{v}_1,\mathbf{v}_2,...,\mathbf{v}_m\right\}$ of $m$ linearly independent solutions to

\begin{equation} \left(\mathbf{A} - \lambda\mathbf{I}\right)^m\mathbf{v} = \mathbf{0} \label{eq:jordain_chain_general_equation} \end{equation}

 \end{theorem}
\textbf{Proof: } take $\gamma$ as the geometric multiplicity of $\lambda$ and let $W = \left\{\mathbf{w}_1,\mathbf{w}_2,...,\mathbf{w}_\gamma\right\}$ be the set of $\gamma$ linearly independent ordinary eigenvectors related to $\lambda$. For each of the $\mathbf{w}_i$, generate a Jordan Chain $W_i = \left\{\mathbf{w}_i^1,\mathbf{w}_i^2,...,\mathbf{w}_i^m\right\}$ with $\mathbf{w}_i^1 = \mathbf{w}_i$.

\begin{equation}
\left\{\begin{array}{cccc}
	\mathbf{w}_1 = \mathbf{w}_1^1 &  \mathbf{w}_2 = \mathbf{w}_2^1 & ... & \mathbf{w}_m = \mathbf{w}_\gamma^1 \\[5mm]
	\mathbf{w}_1^2 & \mathbf{w}_2^{m} & ... & \mathbf{w}_\gamma^{2} \\[5mm]
	\mathbf{w}_1^{3} & \mathbf{w}_2^{3} & ... & \mathbf{w}_\gamma^{3} \\[5mm]
	\vdots & \vdots & \vdots & \vdots \\[5mm]
	\mathbf{w}_1^{(m-1)} & \mathbf{w}_2^{(m-1)} & ... & \mathbf{w}_\gamma^{(m-1)} \\[5mm]
	\mathbf{w}_1^{m} & \mathbf{w}_2^{m} & ... & \mathbf{w}_\gamma^{m}
\end{array}\right. \label{theo:generalized_jordan_chain_list}
\end{equation}

	From theorem \ref{theo:jordan_group}, any two vectors chosen from any two different columns (that is, from different $\mathbf{w}$) are linearly independent; from theorem \ref{theo:jordan_chain_li}, any two vectors chosen from any two different rows are linearly independent. Therefore, any collection of $m$ different vectors in any of the chains in \eqref{theo:generalized_jordan_chain_list} is linearly independent; in fact, any linear combination of this collection satisfies \eqref{eq:jordain_chain_general_equation}.
\hfill$\blacksquare$
\vspace{5mm}
\hrule
\vspace{5mm} %>>>

	From theorem \ref{theo:jordan_group} one can define a Generalized Jordan Chain as a set of linearly independent vectors that satisfy \eqref{eq:jordain_chain_general_equation}, which existence is guaranteed by the theorem.

\begin{definition}[Generalized Jordan Chains (GJCs)]\label{def:jordan_group} %<<<
	A Generalized Jordan Chain of an eigenvalue $\lambda$ with algebraic multiplicity $m$ is a set of $m$ linearly independent solutions to the equation

\begin{equation} \left(\mathbf{A} - \lambda\mathbf{I}\right)^m\mathbf{v} = \mathbf{0} \end{equation}

	\noindent which always exists as proven by theorem \ref{theo:jordan_group}.
\end{definition} %>>>

	From theorem \ref{theo:jordan_group}, a GJC is a set of $\mu\left(\lambda\right)$ linearly independent vectors; therefore it fulfills the role of an ordinary Jordan Chain, albeit with an easier construction. However, the downside of using GJCs instead of an ordinary Jordan Chain is that the groups do not generate a chain of solutions so easily. One might think that, through an implication of theorem \ref{theo:jordan_chain_li_solutions_2}, a Generalized Jordan Chain $\left\{\mathbf{v}_1,\mathbf{v}_2,...,\mathbf{v}_m\right\}$ would generate a sequence of functions like an ordinary counterpart, in the form of \eqref{eq:jordan_chain_sols_seq}. Such is not always the case, because if $\mathbf{v}_1$ is not an ordinary eigenvector, $\mathbf{x}_1$ is not a solution; nor is $\mathbf{x}_2$ if $\mathbf{v}_1$ is not ordinary and $\mathbf{v}_2$ is not of rank 2. Generalistically, for $\mathbf{x}_k$, $1\leq k < m$ to be a solution, all the $\mathbf{x}_i$ must be of rank $i$. It is however either impractical for this to happen — by theorem \ref{theo:jordan_group}, a Generalized Jordan Chain can even contain several eigenvectors of the same rank — or simply impossible because a Generalized Jordan Chain might not have eigenvectors of a particular rank $i$; for instance, one can choose $m$ vectors of rank $3$ and above.

	However, it is simple to see that $\mathbf{x}_m$ is a solution indeed; therefore, for each generalized eigenvector of a Generalized Jordan Chain, the set of linearly independent solutions are, in fact, the last solutions of the sequence generated by taking each of these eigenvectors.

\begin{theorem} \label{theo:jordan_chain_li_solutions_3} %<<<
	Consider the ODE $\dot{\mathbf{x}} = \mathbf{Ax}$, and let $\left\{\mathbf{v}_1,\mathbf{v}_2,...,\mathbf{v}_m\right\}$ be a Generalized Jordan Chain of an eigenvector $\lambda$ of algebraic multiplicity $m$. Then the sequence of functions 

\begin{equation} \mathbf{x}_k(t) =  \left[\displaystyle\sum\limits_{i=0}^{m-1} \dfrac{t^i}{i!}\left(\mathbf{A} - \lambda\mathbf{I}\right)^{i}\right] \mathbf{v}_ke^{\lambda t} \end{equation}

	for $1\leq k\leq m$ is a sequence of linearly independent solutions to $\dot{\mathbf{x}} = \mathbf{Ax}$.
\end{theorem}
\textbf{Proof:} pick an arbitrary $\mathbf{v}_k$ from the Generalized Jordan Chain, and suppose it is of $p$ rank. Then there is a Jordan Chain $\left\{\mathbf{u}_1,\mathbf{u}_2,...,\mathbf{u}_p\right\}$ with $\mathbf{v}_k = \mathbf{u}_p$. Write the Jordan Chain of solutions

\begin{equation}
\left\{\begin{array}{l}
	\mathbf{y}_1(t) = \mathbf{u}_1 e^{\lambda t} \\[3mm]
	\mathbf{y}_2(t) = \left(t\mathbf{u}_1 + \mathbf{u}_2\right) e^{\lambda t} \\[3mm]
	\mathbf{y}_3(t) = \left(\dfrac{t^2}{2}\mathbf{u}_1 + t\mathbf{u}_2 + \mathbf{u}_3\right) e^{\lambda t} \\[3mm]
	\hspace{2cm} \vdots \\[3mm]
	\mathbf{y}_p(t) = \left[\displaystyle\sum\limits_{i=0}^{p-1} \dfrac{t^i}{i!}\mathbf{u}_{(p-i)} \right] e^{\lambda t}
\end{array}\right.
\end{equation}

	By theorem \ref{theo:jordan_group}, the $\mathbf{u}$ are related through the recursion $\mathbf{u}_i = \left(\mathbf{A} - \lambda\mathbf{I}\right)^{(m-i-1)}\mathbf{u}_m = \left(\mathbf{A} - \lambda\mathbf{I}\right)^{(m-i-1)}\mathbf{v}_k$:

\begin{equation} \mathbf{y}_p(t) =  \left[\displaystyle\sum\limits_{i=0}^{p-1} \dfrac{t^i}{i!}\left(\mathbf{A} - \lambda\mathbf{I}\right)^{i}\mathbf{v}_k\right] e^{\lambda t} = \left[\displaystyle\sum\limits_{i=0}^{p-1} \dfrac{t^i}{i!}\left(\mathbf{A} - \lambda\mathbf{I}\right)^{i}\right] \mathbf{v}_k e^{\lambda t} \end{equation}

	\noindent and it stems from the linearly independency of Generalized Jordan Chains that the $\mathbf{y}_p$ generated from each eigenvector in the GJC are all linearly independent. Now consider the function

\begin{equation} \mathbf{x}_k(t) =  \left[\displaystyle\sum\limits_{i=0}^{m-1} \dfrac{t^i}{i!}\left(\mathbf{A} - \lambda\mathbf{I}\right)^{i}\right] \mathbf{v}_ke^{\lambda t} \end{equation}

	\noindent and separate the sum into the indexes from $1$ to $p-1$ and from $p$ to $m$:

\begin{equation} \mathbf{x}_k(t) =  \left[\displaystyle\sum\limits_{i=0}^{p-1} \dfrac{t^i}{i!}\left(\mathbf{A} - \lambda\mathbf{I}\right)^{i}\right] \mathbf{v}_ke^{\lambda t} + \left[\displaystyle\sum\limits_{i=p}^{m-1} \dfrac{t^i}{i!}\left(\mathbf{A} - \lambda\mathbf{I}\right)^{i}\right] \mathbf{v}_ke^{\lambda t}\end{equation}

	\noindent but because $\mathbf{v}_k$ is of rank $p$, the second sum is null because by definition $\left(\mathbf{A} - \lambda\mathbf{I}\right)^{i}\mathbf{v}_k = \mathbf{0}$ for all $i \geq p$, meaning $\mathbf{x}_k = \mathbf{y}_p$. Therefore the $\mathbf{x}_k$ for a linearly independent set of solutions to $\dot{\mathbf{x}} = \mathbf{Ax}$.
\hfill$\blacksquare$
\vspace{5mm}
\hrule
\vspace{5mm} %>>>

	The immediate consequence of theorem \ref{theo:jordan_chain_li_solutions_3} is, finally, that a collection of all generalized eigenvectors of $\mathbf{A}$ is able to generate a general solution to $\dot{\mathbf{x}} = \mathbf{Ax}$.

\begin{corollary} \label{theo:homogeneous_solutions_ltiode} %<<<
	Let $\mathbf{A}\in\mathbb{C}^{(n\times n)}$ and consider the homogeneous or natural LTI ODE

\begin{equation} \dot{\mathbf{x}} = \mathbf{Ax} \label{eq:general_solution_ltiode_generalized} .\end{equation}

	Let $\lambda_k$ denote the eigenvalues of $\mathbf{A}$, $m_k = \mu\left(\lambda_k\right)$ the algebraic multiplicity of $\lambda$. For each $\lambda_k$ take a Generalized Jordan Chain $\left\{\mathbf{v}_1,\mathbf{v}_2,...\mathbf{v}_m\right\}$, and to each of the generalized eigenvectors $\mathbf{v}_p$ associate a function

\begin{equation} \mathbf{x}_p(t) =  \left[\displaystyle\sum\limits_{i=0}^{m_k-1} \dfrac{t^i}{i!}\left(\mathbf{A} - \lambda\mathbf{I}\right)^{i}\right]\mathbf{v}_p e^{\lambda t}. \label{eq:general_solution_ltiode_generalized_sol}\end{equation}

	Then all functions $\mathbf{x}_p$ are linearly independent solutions of \eqref{eq:general_solution_ltiode_generalized}. In other words, the general solution of \eqref{eq:general_solution_ltiode_generalized} is a linear combination of the $\mathbf{x}_p$.
\end{corollary} %>>>

%-------------------------------------------------
\section{Matrix exponentials and the general solution of a LTI ODE} %<<<1

	Although theorem \ref{theo:homogeneous_solutions_ltiode} does offer a complete solution to any LTI ODE, it requires computing all generalized eigenvectors using a matrix power equation, which can be troublesome and not fit to work for some proofs.

	Consider a one-dimensional LTI ODE $\dot{x} = a x$, with $a\in\mathbb{C}$, which solution is immediate $x(t) = e^{\lambda t}x_0$ where $x_0 = x\left(0\right)$. This section aims to prove that the notion of an exponential matrix is possible, and that the general solution to an LTI ODE system $\dot{\mathbf{x}} = \mathbf{Ax}$ is, in some sense, the same solution as the one-dimensional counterpart: $\mathbf{x}(t) = e^{\mathbf{A}t}\mathbf{x}_0$. This development allows for the concise representation of the solution to an LTI ODE and, due to several properties of the matrix exponential, makes proofs and conclusions much easier than using generalized eigenvectors.

	An intuitive introduction is as follows. Remember that the scalar exponential $e^{\lambda t}$ can be expanded as

\begin{equation} e^{\lambda t} = \sum\limits_{i\in\mathbb{N}} \dfrac{t^i}{i!} \lambda^i . \end{equation}

	At the same time, from theorem \ref{theo:jordan_chain_li_solutions_3}, to every generalized eigenvector $\mathbf{v}_k$ pertaining to an eigenvalue $\lambda$ of a matrix $\mathbf{A}$ there can be associated a solution $\mathbf{x}_k$ of the LTI ODE $\dot{\mathbf{x}} = \mathbf{Ax}$ given by

\begin{equation} \mathbf{x}_k(t) =  \left[\displaystyle\sum\limits_{i=0}^{m-1} \dfrac{t^i}{i!}\left(\mathbf{A} - \lambda\mathbf{I}\right)^{i}\right] \mathbf{v}_ke^{\lambda t}. \label{eq:matrix_exp_geneigen_solution_k}\end{equation}

	\noindent and this expression is very similar to the exponential expansion of the scalar exponential, prompting the definition of an \textbf{exponential function} of a complex matrix which can be defined as

\begin{equation} e^{\mathbf{A}} = \sum\limits_{i\in\mathbb{N}} \dfrac{1}{i!} \mathbf{A}^i\ , \label{eq:matrix_exp_def} \end{equation}

	\noindent so that the term $\left[\sum\limits_{i=0}^{m-1} \frac{t^i}{i!}\left(\mathbf{A} - \lambda\mathbf{I}\right)^{i}\right]$ can be roughly expressed as $e^{\left(\mathbf{A} - \lambda\mathbf{I}\right)t}$. Then, write $e^{\lambda t}\mathbf{v}_k = e^{\lambda t \mathbf{I}} \mathbf{v}_k$ and

\begin{equation} \mathbf{x}_k(t) = e^{\left(\mathbf{A} - \lambda\mathbf{I}\right)t} e^{\lambda t \mathbf{I}} \mathbf{v}_k = e^{\left[\left(\mathbf{A} - \lambda\mathbf{I}\right) + \lambda t \mathbf{I}\right]}\mathbf{v}_k =  e^{\mathbf{A}t}\mathbf{v}_k\end{equation}

	To make this proof solid, we first define the matrix operation, prove it ``makes sense'', that is, it exists and converges, and that the operational propreties of exponentiation — like that fact that a product of exponents becomes the exponent of the product. Formally, the exponential function for matrices is defined as in definition \ref{def:matrix_exp}.

\begin{definition}[Matrix exponential operation]\label{def:matrix_exp}
	Let $\mathbf{A}\in \mathbb{C}^{\left(n\times n\right)}$; then the exponential operation is a transform in the space of complex matrices defined as

\begin{equation} e^{\left(\cdot\right)}: \left\{\begin{array}{rcl} \mathbb{C}^{\left(n\times n\right)} &\to& \mathbb{C}^{\left(n\times n\right)} \\[3mm] \mathbf{A} &\mapsto&  \displaystyle\sum\limits_{k\in\mathbb{N}} \dfrac{1}{k!} \mathbf{A}^k\end{array}\right.\end{equation}

	\noindent where $\mathbf{A}^0$ is the identity matrix for any matrix.
\end{definition}

	However, to prove that the infinite sum \eqref{eq:matrix_exp_def} converges always, we have to define the norm of a matrix, which stems from the definitions of norms of linear maps, which themselves stem from norms of vector spaces, begetting some theory of norms in vector spaces and the spectral theorem.

%-------------------------------------------------
\subsection{Inner product and norms}\label{subsec:inner_prod_norms}

\begin{definition}[Norm of a vector space]\label{def:norm_vecspaece} In a vector space $V$ over the complex numbers, a \textbf{norm} is a real-valued function $\left\lvert \mathbf{x}\right\rvert_V\in\left[V\to\mathbb{R}_+\right]$ that satisfies:
\begin{itemize}
	\item \textbf{Triangle inequality}: for $\mathbf{x,y}\in V,\ \left\lvert \mathbf{x} + \mathbf{y}\right\rvert \leq \left\lvert\mathbf{x}\right\rvert + \left\lvert\mathbf{y}\right\rvert$ ;
	\item \textbf{Absolute homogeneity}: $\left\lvert z\mathbf{x}\right\rvert = \left\lvert z\right\rvert \left\lvert\mathbf{x}\right\rvert$ for any $z\in\mathbb{C}$;
	\item \textbf{Positive definiteness}: $\left\lvert \mathbf{x}\right\rvert = 0 \Leftrightarrow \mathbf{x} = \mathbf{0}$.
\end{itemize}
\end{definition}

	Notably, there can be many (infinite actually) functions that satisfy these properties, meaning the norm function is not unique. It can be shown however \pcite{schaeferTopologicalVectorSpaces1999} that in a finite-dimensional vector space all norms are equivalent; for any two norms $\left\lvert\ \cdot\ \right\rvert^1_V, \left\lvert\ \cdot\ \right\rvert^2_V $ defined for the same vector space $V$, there always exists two non-zero reals $k,K$ such that for any $\mathbf{x}\in V$

\begin{equation} k\left\lvert\mathbf{x}\right\rvert^1_V \leq \left\lvert\mathbf{x}\right\rvert^1_V \leq K\left\lvert\mathbf{x}\right\rvert^1_V .\end{equation}

	It is said that \textit{a norm induces a topology} in a vector space, that is, a notion of distance among the vectors as $d\left(\mathbf{x},\mathbf{y}\right) = \left\lvert \mathbf{x - y}\right\rvert$; this means it is (under certain circumstances) possible to define properties like limits, derivatives and integrals in these spaces. In complex vector spaces, the most used norm is the one born from the complex inner product.

\begin{definition}[Complex inner product]\label{def:complex_inner_prod} The inner product of a complex vector space $V$ is a binary operation $\left<\cdot,\cdot\right>\in\left[ V\times V\to \mathbb{C}\right]$ that satisfies, for any $\mathbf{x,y,v,w}\in V$,

\begin{itemize}
	\item \textbf{Conjugate symmetry}: $\left<\mathbf{v,w}\right> = \overline{\left<\mathbf{w,v}\right>}$; and
	\item \textbf{Linearity on the first argument:} $\left<\mathbf{\alpha x + \beta y,\mathbf{w}}\right> = \alpha\left<\mathbf{x,w}\right> + \beta\left<\mathbf{y,w}\right>$ for complex $\alpha,\beta$; and
	\item \textbf{Conjugate linearity on the second argument:} $\left<\mathbf{w},\alpha \mathbf{x} + \beta \mathbf{y}\right> = \overline{\alpha}\left<\mathbf{x,w}\right> + \overline{\beta}\left<\mathbf{y,w}\right>$;
	\item \textbf{Positive definiteness:} $\left<\mathbf{v,v}\right> = 0$ if $\mathbf{v}\neq \mathbf{0}$.
\end{itemize}

	Specifically, the \textbf{complex inner product} is defined as
\begin{equation} \left<\mathbf{v,w}\right> = \sum_{k=1}^n \overline{w_k}v_k\end{equation}
\end{definition}

	The inner product has many desirable properties that make it a very useful tool for a myriad of purposes. Particularly, we call two vectors $\mathbf{w,v}$ as \textbf{orthogonal} if $\left<\mathbf{w,v}\right> = 0$ and a \textbf{orthogonal basis} of a vector space $V$ as a basis which vectors are all orthogonal among themselves. Also, from the inner product we can define a norm function, a notion of ``sizes'' of vectors.

\begin{definition}[Norm of a complex vector space]\label{def:norm_complex} Let $\mathbf{v}\in\mathbb{C}^n$. Then the \textbf{norm} $\left\lvert \mathbf{v}\right\rvert$ is defined as 
\begin{equation} \left\lvert \mathbf{v}\right\rvert = \sqrt{\left<\mathbf{v,v}\right>}  = \sqrt{\sum_{k=1}^n \left\lvert v_k\right\rvert^2 }\end{equation}
\end{definition}

	One of the many benefits of orthogonality is that a certain basis which elements are orthogonal among themselves — called an \textbf{orthogonal basis} — is that the decomposition of any vector $\mathbf{v}$ with respect to the basis can be easily found by the inner product of the vector and the basis constituents, as shown in theorem \ref{theo:orthobasis_decomp}.

\begin{theorem}[Orthogonal basis decomposition]\label{theo:orthobasis_decomp} %<<<
	Let $V$ a vector space with $\dim\left(V\right) = n$, $W = \left\{\mathbf{w}_i\right\}_{i\in\mathbb{N}_n^*}$ an orthogonal basis of $V$, that is, $\left<w_i,w_j\right> = 0$ if $i\neq j$. Then the coordinates of any vector $\mathbf{v}$ in $V$ can be found by the inner product of $\mathbf{v}$ and the elements of the basis, that is,

\begin{equation} \mathbf{v} = \sum_{k\in\mathbb{N}_n^*} v_k\mathbf{w}_k \Leftrightarrow v_k = \dfrac{\left<\mathbf{v},\mathbf{w}_k\right>}{\left\lvert\mathbf{w}_k \right\rvert^2} \end{equation}
\end{theorem}
\textbf{Proof:} by simple computation. Given that the basis adopted is orthonormal, using the linearity of the inner product yields

\begin{equation} \left<\mathbf{v},\mathbf{w}_k\right> = \left<\sum_{i\in\mathbb{N}_n^*} v_i\mathbf{w}_i,\mathbf{w_k}\right> = \sum_{i\in\mathbb{N}_n^*} v_i \left<\mathbf{w}_i,\mathbf{w}_k\right> \end{equation}

	\noindent and the inner product vanishes for any $i\neq k$, but is nonzero if $i=k$ and

\begin{equation} \left<\mathbf{v},\mathbf{w}_k\right> = v_k \left<\mathbf{w}_k,\mathbf{e_k}\right> = v_i\left\lvert \mathbf{w}_k\right\rvert^2 \end{equation}

\hfill$\blacksquare$\vspace{3mm}\hrule\vspace{3mm} %>>>

	Particularly, if the elements of a orthogonal basis also have a unit norm then the basis is called \textbf{orthonormal} and the decomposition becomes simply

\begin{equation} v_k = \left<\mathbf{v},\mathbf{w}_k\right> . \label{eq:coordinate_extraction}\end{equation}

	The idea of a inner product is a generalization for the fact that in the real space $\mathbb{R}^2$, the dot-product is used to determine the idea of angles between vectors, of which orthogonality is a paramount notion. It can also be shown that there are many (infinite in fact) functions that satisfy the properties of an inner product, hence the definition of \textit{the} complex inner product as the one to be henceforth used; the proof that this inner product satisfies all the properties is left to the reader. One of its main properties is the fact that this inner product defines an adjoint operator for complex matrices, known as the Hermitian adjoint, or simply hermitian.

\begin{definition}[Hermitian adjoint] For any $\mathbf{A}\in\mathbb{C}^{(n\times m)}$ and any two vectors $\mathbf{x,y}\in\mathbb{C}^n$,

\begin{equation} \left<\mathbf{Ax},\mathbf{y}\right> = \left<\mathbf{x},\mathbf{A}^\hermconj\mathbf{y}\right> \end{equation}

	\noindent where the superscript $\hermconj$ denotes the \textbf{Hermitian adjoint} or simply ``hermitian'' of the matrix $\mathbf{A}$, defined as its transpose-conjugate

\begin{equation} \mathbf{A}^\hermconj = \overline{\mathbf{A}^\transpose} .\end{equation}

\end{definition}

	The property of hermitianism allows us to define the complex inner product as $\left<\mathbf{w,v}\right> = \mathbf{v}^\hermconj \mathbf{w}$. It also begets one of the most famous theorems of linear algebra: the spectral theorem.

\begin{theorem}[Spectral Theorem]\label{theo:spectral_theorem} %<<<
	If $\mathbf{A}$ is hermitian, that is, $\mathbf{A} = \mathbf{A}^\hermconj$, then its eigenvalues are real and there exists an orthogonal basis consisiting of its eigenvectors, or equivalently, its eigenvectors are orthogonal — linearly independent and orthogonal among themselves.
\end{theorem}
\noindent\textbf{Proof.} Take an eigenvalue $\lambda_1$ of $\mathbf{A}$ corresponding to an eigenvector $\mathbf{v}$. Then

\begin{equation} \lambda_1\left<\mathbf{v}_1,\mathbf{v}_1\right> = \left<\lambda_1\mathbf{v}_1,\mathbf{v}_1\right> = \left<\mathbf{A}\mathbf{v}_1,\mathbf{v}_1\right> = \left<\mathbf{v}_1,\mathbf{A}^\hermconj\mathbf{v}_1\right> = \left<\mathbf{v}_1,\mathbf{A}\mathbf{v}_1\right> = \overline{\lambda_1}\left<\mathbf{v}_1,\mathbf{v}_1\right> . \label{eq:spectral_eq1}\end{equation}

	But because by definition $\mathbf{v}_1\neq\mathbf{0}$, this means $\lambda_1 = \overline{\lambda_1}$, therefore it is real. Now pick $K_1$ as the space generated by all vectors orthogonal to $\mathbf{v}_1$; it is simple to see that $K_1$ is invariant to $\mathbf{A}$ since for any $\mathbf{k_1}$ in this space, $\mathbf{Ak}_1$ is orthogonal to $\mathbf{v}_1$:

\begin{equation} \left<\mathbf{Ak}_1,\mathbf{v}\right> = \left<\mathbf{k}_1,\mathbf{Av}\right> = \left<\mathbf{k}_1,\lambda_1\mathbf{v}\right> = \overline{\lambda_1}\left<\mathbf{k}_1,\mathbf{v}\right> = 0 . \end{equation}

	Because of this, there exists some eigenvector of $\mathbf{A}$ in $K_1$, say $\mathbf{v}_2$. Applying \eqref{eq:spectral_eq1} to this $\mathbf{v}_2$ and $\lambda_2$ shows $\lambda_2$ is also real. Now let $K_3$ the space generated by all vectors orthogonal to both $\mathbf{v}_1$ and $\mathbf{v}_2$, and by the same line of thought $K_2$ is invariant to $\mathbf{A}$, therefore there is some $\mathbf{v}_3$ in $K_2$. By induction, all eigenvectors of $\mathbf{A}$ are orthogonal and all eigenvalues are real.
\hfill$\blacksquare$
\vspace{5mm}
\hrule
\vspace{5mm} %>>>

	In so far as there exist infinite norms for complex spaces, the one of definition \ref{def:complex_inner_prod} will be adopted in this text and will be called simple as \textit{the norm} for complex vectors. It is simple to see that this norm in fact satisfies the properties of a norm.

%-------------------------------------------------
\subsection{Norms of maps and matrices}

	The adoption of a particular norm in two vector spaces $V,W$ induces the notion of a norm for maps between such spaces , defined as the maximum ratio between the norms of the output and input of the map.

\begin{definition}[Norm of a map] \label{def:mapping_norm}
	Let $V,W$ be two metric spaces, $\phi\left(\cdot\right)$ a mapping from $V$ to $W$. Denote $\left\lVert \cdot\right\rVert_V$ as the norm in $V$, $0_V$ as the null element of $V$, and $\left\lVert \cdot\right\rVert_W$ the norm in $W$. Then the norm of $\phi$ is the number such that

\begin{equation} \left\lVert \phi \right\rVert  = \inf\left\{ \alpha\in\mathbb{R}^+\cup\left\{\infty\right\}:\ \left\lVert \phi\left(\mathbf{v}\right)\right\rVert_W \leq \alpha\left\lVert \mathbf{v}\right\rVert_V \forall \mathbf{v}\in V\right\} \label{eq:norm_map_def}\end{equation}
\end{definition}
\begin{definitionremark}\label{remark:equiv_def_map}
	The following definitions for a map are equivalent to \eqref{eq:norm_map_def}:

\begin{align}
	\left\lVert \phi \right\rVert &= \inf\left\{ \alpha\in\mathbb{R}^+\cup\left\{\infty\right\}:\ \left\lVert \phi\left(\mathbf{v}\right)\right\rVert_W \leq \alpha\left\lVert \mathbf{v}\right\rVert_V \forall \mathbf{v}\in V\right\} \label{eq:norm_map_def_1} \\[3mm]
	&= \sup\left\{ \dfrac{\left\lVert \phi\left(\mathbf{v}\right)\right\rVert_W}{\left\lVert \mathbf{v}\right\rVert_V}:\ \mathbf{v}\in V\wedge \mathbf{v}\neq 0_V \right\} \text{ if } V\neq\left\{0_V\right\} \label{eq:norm_map_def_2}\\[3mm]
	&= \sup\left\{ \left\lVert \phi\left(\mathbf{v}\right)\right\rVert_W:\ \left\lVert \mathbf{v}\right\rVert_V = 1 \wedge \mathbf{v}\in V \right\} \text{ if } V\neq\left\{0_V\right\} \label{eq:norm_map_def_3}
\end{align}
\end{definitionremark}

	Intiutively, for some generic map, take all the positive real numbers $\alpha$, together with infinity, such that the norm of the resulting operation is never greater than that of the argument times $\alpha$, that is, the mapping never ``streches'' the argument for more than $\alpha$. The norm is the infimum of such set, that is, the ``least amount'' by which the operator ``streches'' its input.

	It can be proven (\cite{rudin1991functional}) that a linear map is continuous if and only if it is bounded, that is, there is some non-infinite positive $M$ such that 

\begin{equation} \left\lvert \mathbf{A}\left[\mathbf{x}\right]\right\rvert_W \leq M\left\lvert \mathbf{x}\right\rvert_V \forall \mathbf{x}\in V , \label{eq:matrix_norm_def}\end{equation}

	\noindent meaning that the set of all the $\alpha$ of \eqref{eq:norm_map_def} is closed, nonempty and bounded below; hence the infimum exists and is not infinite. For matrices operating in the space of complex signals, this definition yields a similar definition, but with some caveats. It is clear that the norm of a mapping, as defined in definition \ref{def:mapping_norm}, depends on the norms of the spaces $V$ and $W$; this means that there is a plethora of norms available, and the norm of the maps are in general induced by the norms of the vectors.

	For complex matrices, we can show that their norms are always bounded. Because any two norms in a finite vector space are equivalent, we must only show this for a single norm, and we choose the complex norm.

\begin{theorem}[Complex matrices are bounded linear maps]%<<<
	For any $\mathbf{A}\in\mathbb{C}^{(n\times m)}$ and $\mathbf{x}\in\mathbb{C}^{n}$,

\begin{equation} \left\lvert\mathbf{Ax}\right\rvert \leq c\left\lvert \mathbf{x}\right\rvert \end{equation}

	where $c$ is the maximum of the norms of the column vectors of $\mathbf{A}$.
\end{theorem}
\noindent\textbf{Proof:} pick $x$ as defined:
\begin{equation} c = \max_{1\leq k\leq m} \left\lvert\mathbf{Ae}_k\right\rvert\end{equation}

	\noindent and note that $c$ can not be inifnite and is always positive. Then

\begin{equation} \left\lvert \mathbf{Ax}\right\rvert^2 = \left\lvert \sum_{k=1}^n x_k \mathbf{c}_k\right\rvert^2 \leq \left(\sum_{k=1}^n \left\lvert x_k \mathbf{c}_k\right\rvert\right)^2 \leq \sum_{k=1}^n \left\lvert x_k \mathbf{c}_k\right\rvert^2 = \sum_{k=1}^n \left\lvert x_k\right\rvert^2 \left\lvert\mathbf{c}_k\right\rvert^2 \leq c^2 \sum_{k=1}^n \left\lvert x_k\right\rvert^2  = c^2\left\lvert \mathbf{x}\right\rvert^2\end{equation}
\hfill$\blacksquare$
\vspace{5mm}
\hrule
\vspace{5mm} %>>>

	Therefore, for complex matrices, the infimum of \eqref{eq:norm_map_def} becomes a minimum and for norm defined on the vector space $V$

\begin{equation}  \left\lVert \mathbf{A}\right\rVert_{V} = \max_{\mathbf{v}\neq 0} \dfrac{\left\lvert \mathbf{Av}\right\rvert_V}{\left\lvert \mathbf{v}\right\rvert_V} = \max_{\left\lvert \mathbf{v} \right\rvert_V = 1} \left\lvert \mathbf{Av}\right\rvert_V . \label{eq:matrix-norm_def} \end{equation}

	By definition, the vector matrix that it induces are \textit{consistent}, that is,

\begin{equation} \left\lvert \mathbf{Av}\right\rvert_V \leq \left\lVert \mathbf{A}\right\rVert_V \left\lvert \mathbf{v}\right\rvert_V, \label{eq:consistent_norm} \end{equation}

	\noindent which basically states that if $\left\lvert \mathbf{Av}\right\rvert_V$ achieves a value $c$ for some $\mathbf{v}$, then $\left\lVert A\right\rVert_V$ is at least $c/\left\lvert\mathbf{v}\right\rvert$. Thence, it is clear that $\left\lVert\mathbf{A}\right\rVert_V$ depends on the vector norm adopted. For the complex vector norm of definition \ref{def:norm_complex}, the matrix norm induced is henceforth denoted $\left\lVert \cdot\right\rVert_2$. This norm is known as the \textbf{spectral norm}, due to theorem \ref{theo:spectral_norm}.

\begin{theorem}[Matrix spectral norm]\label{theo:spectral_norm} %<<<
	For $\mathbf{A}\in\mathbb{C}^{(n\times n)}$, $\left\lVert\mathbf{A}\right\rVert_2$ is the largest singular value of $\mathbf{A}$, that is, the absolute value of the largest eigenvalue of $\mathbf{A^\hermconj}\mathbf{A}$.
\end{theorem}
\noindent\textbf{Proof.} Take $B = \mathbf{A}^\hermconj\mathbf{A}$; then $\mathbf{B}$ is clearly hermitian, that is, $\mathbf{B} = \mathbf{B}^\hermconj$. Therefore, by the Spectral Theorem (theorem \ref{theo:spectral_theorem}), the eigenvectors of $\mathbf{B}$ are orthogonal. In particular, let us pick the eigenvectors $\left\{\mathbf{v}_1,...,\mathbf{v}_n\right\}$ with a unitary norm and let their set be the basis $\mathbf{V}$. Let $\lambda_k$ the eigenvalue of the eigenvector $\mathbf{v}_k$ and  $\left[\mathbf{x}\right]_\mathbf{V} = \left[x_1,...,x_n\right]^\transpose$ the coordinates of some vector $\mathbf{x}$ in the basis $\mathbf{V}$. Then

\begin{equation} \mathbf{Bx} = \mathbf{B}\left(\sum_{k=1}^n x_k\mathbf{v}_k\right) = \sum_{k=1}^n x_k\mathbf{Bv}_k = \sum_{k=1}^n x_k\lambda_k\mathbf{v}_k\end{equation}

	So that $\left\lvert \mathbf{Ax}\right\rvert = \sqrt{\left<\mathbf{Ax},\mathbf{Ax}\right>}$. By the definition of a Hermitian adjoint, this is equal to

\begin{equation} \left\lvert \mathbf{Ax}\right\rvert = \sqrt{\left<\mathbf{x},\mathbf{A}^\hermconj\mathbf{Ax}\right>} = \sqrt{\left<\mathbf{x},\mathbf{Bx}\right>} = \sqrt{  \left< \sum_{k=1}^nx_k\lambda_k\mathbf{v}_k, \sum_{i=1}^n x_i \mathbf{v}_i\right>} = \sqrt{\sum_{k=1}^n \sum_{i=1}^n x_k\overline{x_i\lambda_k}\left<\mathbf{v}_k \mathbf{v}_i\right>} .\end{equation}

	But because the $\mathbf{v}_i$ are orthogonal, $\left<\mathbf{v}_i,\mathbf{v}_k\right> = 0$ if $i\neq k$ and equal to $\left\lvert \mathbf{v}_k\right\rvert^2 = 1$ if $k=i$. Also because the $\lambda_k$ are real, they are equal to their conjugates. Then

\begin{equation} \left\lvert \mathbf{Ax}\right\rvert = \sqrt{\sum_{k=1}^n x_k\overline{x_k}\lambda_k} = \sqrt{\sum_{k=1}^n \left\lvert x_k\right\rvert^2 \lambda_k} .\end{equation}

	Now take

\begin{equation} \lambda = \max_{\lambda_k\in\rho\left(\mathbf{A}\right)} \left\lvert\lambda_k\right\rvert\end{equation}

	\noindent and $\mathbf{v}$ the eigenvector respective to the $\lambda_k$ that achieves $\lambda$. Then

\begin{equation} \left\lvert \mathbf{Ax}\right\rvert \leq \sqrt{\sum_{k=1}^n \left\lvert x_k\right\rvert^2 \lambda} = \sqrt{\lambda} \sqrt{\sum_{k=1}^n \left\lvert x_k\right\rvert^2} = \sqrt{\lambda} \left\lvert\mathbf{x}\right\rvert\end{equation}

	\noindent which proves

\begin{equation} \left\lVert\mathbf{A}\right\rVert_2 \leq \dfrac{\left\vert \mathbf{Ax}\right\rvert}{\left\lvert\mathbf{x}\right\rvert} =  \sqrt{\lambda}. \label{eq:spectral_theo_less_than}\end{equation}

	But for the specific $\mathbf{v}$,

\begin{equation} \left\lvert \mathbf{Av}\right\rvert = \sqrt{\left<\mathbf{v},\mathbf{A}^\hermconj\mathbf{Av}\right>} = \sqrt{\left<\mathbf{v},\mathbf{Bv}\right>} = \sqrt{\left<\mathbf{v}, \lambda \mathbf{v}\right>} = \sqrt{\lambda \left<\mathbf{v}, \mathbf{v}\right>} = \sqrt{\lambda}\end{equation}

	\noindent meaning $\left\lvert \mathbf{Ax}\right\rvert$ achieves $\sqrt{\lambda}$ at $\mathbf{v}$. But since $\left\lvert\mathbf{v}\right\rvert = 1$, this means $\left\lVert\mathbf{A}\right\rVert_2 \geq \sqrt{\lambda}$ by the consistency of matrix norm \eqref{eq:consistent_norm}. Together with \eqref{eq:spectral_theo_less_than}, this yields $\left\lVert\mathbf{A}\right\rVert_2 = \sqrt{\lambda}$.
\hfill$\blacksquare$
\vspace{5mm}
\hrule
\vspace{5mm} %>>>

	The simpler $\left\lVert\cdot\right\rVert$ with a omitted $V$ denotes the matrix norm induced by any vector norm $\left\lvert\ \cdot\ \right\rvert_V$. It is also left to the reader to show that the matrix norm \eqref{eq:matrix-norm_def} satisfies the norm definitions, that is, for any two complex matrices $\mathbf{A}$ and $\mathbf{B}$ and any complex scalar $z$,

\begin{itemize}
	\item \textbf{Triangle inequality}: $\left\lVert\mathbf{A} + \mathbf{B}\right\rVert \leq \left\lVert\mathbf{A}\right\rVert + \left\lVert\mathbf{B}\right\rVert$ ;
	\item \textbf{Absolute homogeneity}: $\left\lVert z\mathbf{A}\right\rVert = \left\lvert z\right\rvert \left\lVert\mathbf{A}\right\rVert$ ;
	\item \textbf{Positive definiteness}: $\left\lVert \mathbf{A}\right\rVert = 0 \Leftrightarrow \mathbf{A} = \mathbf{0}$ ;
	\item \textbf{Non-negativity}: $\left\lVert \mathbf{A}\right\rVert \geq 0$ for all $\mathbf{A}$.
\end{itemize}

	It is also left left to the reader to show that the matrix norm is \textbf{sub-multiplicative}: $\left\lVert \mathbf{A}\mathbf{B}\right\rVert \leq \left\lVert \mathbf{A} \right\rVert\left\lVert \mathbf{B}\right\rVert$. Using the notion of a matrix norm, we can prove that the matrix exponential is a convergent series.

\begin{theorem}[Convergence of matrix exponential] \label{theo:matrix_exponential_powerconvergence}%<<<
The series

\begin{equation} e^{\mathbf{A}} = \sum\limits_{i\in\mathbb{N}} \dfrac{1}{i!} \mathbf{A}^i \end{equation}

	converges absolutely for all $\mathbf{A}\in\mathbb{C}^{(n\times n)}$, where $\mathbf{A}^0 \equiv \mathbf{I}$ for any matrix $\mathbf{A}$. Additionally,

\begin{equation} \left\lVert e^{\mathbf{A}} \right\rVert \leq e^{\left\lVert \mathbf{A}\right\rVert}\end{equation}
\end{theorem}
\textbf{Proof:} let the partial sum

\begin{equation} \mathbf{S}_n = \sum\limits_{i=0}^n \dfrac{1}{i!} \mathbf{A}^i \end{equation}

	\noindent therefore

\begin{equation} \left\lVert e^{\mathbf{A}} - \mathbf{S}_n\right\rVert = \left\lVert \sum\limits_{i=n+1}^\infty \dfrac{1}{i!} \mathbf{A}^i\right\rVert \leq \sum\limits_{i=n+1}^\infty \dfrac{1}{i!} \left\lVert \mathbf{A} \right\rVert^i \end{equation}

	However, this term is a part of the expansion of

\begin{equation} \sum\limits_{i\in\mathbb{N}} \dfrac{1}{i!} \left\lVert \mathbf{A} \right\rVert^i = e^{\left\lVert \mathbf{A}\right\rVert} \end{equation}

	\noindent but because the exponential of any scalar is absolutely convergent, then for all $n$ there exists a decreasing $\varepsilon_n$ such that

\begin{equation} \sum\limits_{i=n+1}^\infty \dfrac{1}{i!} \left\lVert \mathbf{A} \right\rVert^i \leq \varepsilon_n \end{equation}

	\noindent meaning

\begin{equation} \left\lVert e^{\mathbf{A}} - \mathbf{S}_n\right\rVert \leq \varepsilon_n \end{equation}

	\noindent and this proves that the power series is convergent. Furthermore,

\begin{equation} e^{\mathbf{A}} = \sum\limits_{i\in\mathbb{N}} \dfrac{1}{i!} \mathbf{A}^i \Rightarrow \left\lVert e^{\mathbf{A}} \right\rVert \leq \sum\limits_{i\in\mathbb{N}} \dfrac{1}{i!} \left\lVert\mathbf{A}\right\rVert^i = e^{\left\lVert \mathbf{A}\right\rVert}\end{equation}

\hfill$\blacksquare$
\vspace{5mm}
\hrule
\vspace{5mm} %>>>

%-------------------------------------------------
\subsection{Matrix exponential properties} %<<<2

	Having now proved that the notion of a matrix exponential \textit{makes sense}, that is, it exists and is well-defined, we can assert some properties of this operation.

\begin{theorem}[Matrix exponential derivative]
	Let $\mathbf{A}\in\mathbb{C}^{(n\times n)}$ and consider the function $\mathbf{G}(t) = e^{\mathbf{A}t}$. Then $\mathbf{G}'(t) = \mathbf{AG}(t)$.
\end{theorem}
\noindent\textbf{Proof:} by definition,

\begin{equation} \mathbf{G} = \sum\limits_{k\in\mathbb{N}} \dfrac{1}{k!}\mathbf{A}^k t^k .\end{equation}

	Taking the derivative,

\begin{equation} \mathbf{G}' = \sum\limits_{k\in\mathbb{N}_1} \dfrac{1}{(k-1)!}\mathbf{A}^k t^{(k-1)} .\end{equation}

	\noindent (here we assume the simplicity of noting that the matrix power can extrapolate the derivative because it is constant with respect to $t$). But

\begin{equation} \mathbf{G}' = \sum\limits_{k\in\mathbb{N}_1} \dfrac{1}{(k-1)!}\mathbf{A}\mathbf{A}^{(k-1)} t^{(k-1)} = \mathbf{A}\left(\sum\limits_{k\in\mathbb{N}_1} \dfrac{1}{(k-1)!}\mathbf{A}^{(k-1)} t^{(k-1)}\right) = \mathbf{AG}.\end{equation}

	The manipulation that $\mathbf{A}$ can multiply the entire infinite summation is possible because that summation is equal to $e^{\mathbf{A} t}$, meaning it converges.

\begin{theorem}[Matrix exponential inverse] \label{theo:matrix_exponential_inverse} %<<<
	Let $\mathbf{A}\in\mathbb{C}^{(n\times n)}$. Then $\left(e^{\mathbf{A}}\right)^{-1} = e^{-\mathbf{A}}$.
\end{theorem}
\noindent \textbf{Proof:} take $\mathbf{G}(t) = e^{\mathbf{A}t}e^{-\mathbf{A}t}$. Then

\begin{equation} \mathbf{G}'(t) = \mathbf{A}e^{\mathbf{A}t}e^{-\mathbf{A}t} + e^{\mathbf{A}t}\left(-\mathbf{A}\right)e^{-\mathbf{A}t} \end{equation}

	\noindent (here we assume the rule of derivation of product in matrix calculus). From the definition of the matrix exponential, it is simple to see that $\mathbf{A}$ and $e^{\mathbf{A}}$ commute:

\begin{equation} \mathbf{G}'(t) = \mathbf{A}e^{\mathbf{A}t}e^{-\mathbf{A}t} + \left(-\mathbf{A}\right)e^{\mathbf{A}t}e^{-\mathbf{A}t} = \mathbf{0} \end{equation}

	Therefore $\mathbf{G}(t)$ must be a constant matrix; taking $t=0$ yields $\mathbf{G}(t) = e^{\mathbf{0}}e^{\mathbf{0}} = \mathbf{II} = \mathbf{I}$. Therefore $e^{-\mathbf{A}}$ is the right inverse of $e^{\mathbf{A}}$. Take $\mathbf{G}(t) = e^{-\mathbf{A}t}e^{\mathbf{A}t}$ and do the same steps to prove that $e^{-\mathbf{A}}$ is also the left inverse of $e^{\mathbf{A}}$; therefore $e^{-\mathbf{A}}$ is the inverse of $e^{\mathbf{A}}$.
\hfill$\blacksquare$
\vspace{5mm}
\hrule
\vspace{5mm} %>>>

	Theorem \ref{theo:matrix_exponential_inverse} is special because, in short, it defines that any matrix exponential is invertible. This means that the definition \ref{def:matrix_exp} of the matrix exponential can be further refined as

\begin{equation} e^{\left(\cdot\right)}: \left\{\begin{array}{rcl} \mathbb{C}^{\left(n\times n\right)} &\to& \text{GL}\left(n,\mathbb{C}\right) \\[3mm] \mathbf{A} &\mapsto&  \displaystyle\sum\limits_{k\in\mathbb{N}} \dfrac{1}{k!} \mathbf{A}^k\end{array}\right. \label{eq:matrix_exp_gln_def} ,\end{equation}

	\noindent where $\text{GL}\left(n,\mathbb{C}\right)$ repreents the General Linear Group of degree $n$, that is, the collection of invertible complex matrices of size $n$. Particularly, given some invertible matrix $\mathbf{B}$, then the equation $e^{\mathbf{A}} = \mathbf{B}$ has at least one solution, probably infinite in fact since the complex logarithm is multi-valued. By choosing a particular branch of the complex logarithm then the definition \eqref{eq:matrix_exp_gln_def} becomes bijective and a matrix logarithm can be defined as $\ln\left(\mathbf{A}\right)$ is the matrix such that 

\begin{equation} e^{\ln\left(\mathbf{A}\right)} = \mathbf{A}, \end{equation}

	\noindent and it can be proven that if this matrix exists then this logarithm operation holds the famous logarithm properties: $\ln\left(\mathbf{A}\right)\ln\left(\mathbf{B}\right) = \ln\left(\mathbf{AB}\right)$ if $\mathbf{A}$ and $\mathbf{B}$ commute, $\ln\left(\mathbf{A}^{-1}\right) = -\ln\left(\mathbf{A}\right)$ and $\ln\left(\mathbf{A}^k\right) = k\ln\left(\mathbf{A}\right)$ for $k\in\mathbb{Z}$, as well as derivative and integration properties.

\begin{theorem}[Matrix exponential sum] \label{theo:matrix_exponential_sum} %<<<
	Let $\mathbf{A},\mathbf{B}\in\mathbb{C}^{(n\times n)}$ commute, that is, $\mathbf{AB} = \mathbf{BA}$. Then

\begin{equation} e^{\mathbf{A} + \mathbf{B}} = e^{\mathbf{A}}e^{\mathbf{B}} \end{equation}
\end{theorem}
\textbf{Proof:} take $\mathbf{G}(t) = e^{\left(\mathbf{A}+\mathbf{B}\right)t}e^{-\mathbf{A}t}e^{-\mathbf{B}t}$. Then

\begin{equation} \mathbf{G}'(t) = \left(\mathbf{A}+\mathbf{B}\right)e^{\left(\mathbf{A}+\mathbf{B}\right)t}e^{-\mathbf{A}t}e^{-\mathbf{B}t} + e^{\left(\mathbf{A}+\mathbf{B}\right)t}\left(-\mathbf{A}\right)e^{-\mathbf{A}t}e^{-\mathbf{B}t} + e^{\left(\mathbf{A}+\mathbf{B}\right)t}e^{-\mathbf{A}t}\left(-\mathbf{B}\right)e^{-\mathbf{B}t} \end{equation}

	It follows directly from the power series definition that if $\mathbf{A}$ and $\mathbf{B}$ commute, then $\mathbf{A}$ and $e^{\mathbf{B}t}$ commute; therefore $\mathbf{A}$ and $e^{\mathbf{A}}$ also always commute. Therefore

\begin{align}
	\mathbf{G}'(t) &= \left(\mathbf{A}+\mathbf{B}\right)e^{\left(\mathbf{A}+\mathbf{B}\right)t}e^{-\mathbf{A}t}e^{-\mathbf{B}t} + e^{\left(\mathbf{A}+\mathbf{B}\right)t}\left(-\mathbf{A}\right)e^{-\mathbf{A}t}e^{-\mathbf{B}t} + e^{\left(\mathbf{A}+\mathbf{B}\right)t}\left(-\mathbf{B}\right)e^{-\mathbf{A}t}e^{-\mathbf{B}t} \nonumber\\[5mm]
	&= \left(\mathbf{A}+\mathbf{B}\right)e^{\left(\mathbf{A}+\mathbf{B}\right)t}e^{-\mathbf{A}t}e^{-\mathbf{B}t} - e^{\left(\mathbf{A}+\mathbf{B}\right)t}\left(\mathbf{A+B}\right) e^{-\mathbf{A}t}e^{-\mathbf{B}t}\nonumber\\[5mm]
	&= \left(\mathbf{A}+\mathbf{B}\right)e^{\left(\mathbf{A}+\mathbf{B}\right)t}e^{-\mathbf{A}t}e^{-\mathbf{B}t} - \left(\mathbf{A+B}\right)e^{\left(\mathbf{A}+\mathbf{B}\right)t} e^{-\mathbf{A}t}e^{-\mathbf{B}t}\nonumber\\[5mm]
	&= \mathbf{0}
\end{align}

	meaning $\mathbf{G}(t)$ is constant; taking $t=0$ yields $\mathbf{G}(t) = e^{\mathbf{0}}e^{\mathbf{0}} = \mathbf{II} = \mathbf{I}$. Then,

\begin{equation} \mathbf{I} = e^{\left(\mathbf{A}+\mathbf{B}\right)}e^{-\mathbf{A}}e^{-\mathbf{B}} \end{equation}

	Multiply this on the right by $e^{\mathbf{B}}e^{\mathbf{A}}$ and 

\begin{equation} e^{\mathbf{B}}e^{\mathbf{A}} = e^{\left(\mathbf{A}+\mathbf{B}\right)} \end{equation}

	But this equation implies that $e^{\mathbf{B}}$ and $e^{\mathbf{A}}$ also commute, because $e^{\left(\mathbf{A}+\mathbf{B}\right)} = e^{\left(\mathbf{B}+\mathbf{A}\right)}$, and the proof is complete.

\hfill$\blacksquare$
\vspace{5mm}
\hrule
\vspace{5mm} %>>>

\begin{theorem}[Identity scaling] \label{theo:matrix_exponential_identity_scaling} %<<<
	Let $z,w\in\mathbb{C}$. Then $e^{z\mathbf{I}w} = e^{zw}\mathbf{I}$.
\end{theorem}
\textbf{Proof:} from the definition,

\begin{equation} e^{z\mathbf{I}w} = \sum\limits_{i\in\mathbb{N}} \dfrac{1}{i!} \left(z\mathbf{I}w\right)^i = \sum\limits_{i\in\mathbb{N}} \dfrac{\left(zw\right)^i}{i!} \mathbf{I}^i = \sum\limits_{i\in\mathbb{N}} \dfrac{\left(zw\right)^i}{i!} \mathbf{I} = \left[\sum\limits_{i\in\mathbb{N}} \dfrac{\left(zw\right)^i}{i!}\right] \mathbf{I} = e^{zw}\mathbf{I} \end{equation}

\hfill$\blacksquare$
\vspace{5mm}
\hrule
\vspace{5mm} %>>>

	Finally, having asserted the computational aspects of the matrix exponential, we can prove a short but deep theorem that sums up this entire chapter.
 	
\begin{theorem}[Exponential solution of a LTI ODE] \label{theo:homogeneous_solutions_ltiode_matrix} %<<<
	The general solution of the homogeneous LTI ODE $\dot{\mathbf{x}} = \mathbf{Ax}$, $\mathbf{A}\in\mathbb{C}^{(n\times n)}$, is

\begin{equation} \mathbf{x} = e^{\mathbf{A}t}\mathbf{x}_0 \end{equation}
\end{theorem}
\textbf{Proof:} according to theorem \ref{theo:homogeneous_solutions_ltiode}, the union of all Generalized Jordan Chains of $\mathbf{A}$ generates a set of $n$ linearly independent solutions

\begin{equation} \mathbf{x}_p(t) =  \left[\displaystyle\sum\limits_{i=0}^{m_k-1} \dfrac{t^i}{i!}\left(\mathbf{A} - \lambda\mathbf{I}\right)^{i}\right]\mathbf{v}_p e^{\lambda_k t}. \end{equation}

	where $\mathbf{v}_p$ is a generalized eigenvector of the eigenvalue $\lambda_k$ and $m_k = \mu\left(\lambda_k\right)$. 	But note that

\begin{align}
	e^{\left(\mathbf{A}-\lambda\mathbf{I}\right)t}\mathbf{v}_p &= \left[\displaystyle\sum\limits_{i=0}^{\infty} \dfrac{t^i}{i!}\left(\mathbf{A} - \lambda_k\mathbf{I}\right)^{i}\right]\mathbf{v}_p = \left[\sum\limits_{i=0}^{m_k-1} \dfrac{t^i}{i!}\left(\mathbf{A} - \lambda_k\mathbf{I}\right)^{i} + \displaystyle\sum\limits_{i=m_k}^{\infty} \dfrac{t^i}{i!}\left(\mathbf{A} - \lambda_k\mathbf{I}\right)^{i}\right]\mathbf{v}_p  \nonumber\\[5mm]
%
	&= \left[\sum\limits_{i=0}^{m_k-1} \dfrac{t^i}{i!}\left(\mathbf{A} - \lambda_k\mathbf{I}\right)^{i}\right]\mathbf{v}_p + \left[\sum\limits_{i=m_k}^{\infty} \dfrac{t^i}{i!}\left(\mathbf{A} - \lambda_k\mathbf{I}\right)^{i}\right]\mathbf{v}_p \nonumber\\[5mm]
%
	&= \left[\sum\limits_{i=0}^{m_k-1} \dfrac{t^i}{i!}\left(\mathbf{A} - \lambda_k\mathbf{I}\right)^{i}\right]\mathbf{v}_p + \left[\sum\limits_{i=m_k}^{\infty} \dfrac{t^i}{i!}\left(\mathbf{A} - \lambda_k\mathbf{I}\right)^{i}\mathbf{v}_p\right]
\end{align}

	However, $\left(\mathbf{A} - \lambda_k\mathbf{I}\right)^{i}\mathbf{v}_p = \mathbf{0}$ for all $i \geq m_k$ by the very definition of a Generalized Jordan Chain; therefore

\begin{equation} e^{\mathbf{A}t}\mathbf{v}_p = \left[\sum\limits_{i=0}^{m_k-1} \dfrac{t^i}{i!}\left(\mathbf{A} - \lambda_k\mathbf{I}\right)^{i}\right]\mathbf{v}_p \end{equation}

	Meaning

\begin{equation} \mathbf{x}_p(t) = e^{\left(\mathbf{A}-\lambda_k\mathbf{I}\right)t}e^{\lambda_k t}\mathbf{v}_p\ . \end{equation}

	Write $\mathbf{v}_p = \mathbf{Iv}_p$ and use $e^{\lambda_k t}\mathbf{I} = e^{\lambda_k\mathbf{I} t}$ (theorem \ref{theo:matrix_exponential_identity_scaling}):

\begin{equation} \mathbf{x}_p(t) = e^{\left(\mathbf{A}-\lambda_k\mathbf{I}\right)t}e^{\lambda_k \mathbf{I} t}\mathbf{v}_p . \end{equation}

	Now use the fact that $\mathbf{I}$ commutes with any matrix of the same order and $e^{\left(\mathbf{A}-\lambda_k\mathbf{I}\right)t}e^{\lambda_k \mathbf{I} t} = e^{\left(\mathbf{A}-\lambda_k\mathbf{I}\right)t + \lambda_k \mathbf{I} t}$ (theorem \ref{theo:matrix_exponential_sum}):

\begin{equation} \mathbf{x}_p(t) = e^{\left[\left(\mathbf{A}-\lambda_k\mathbf{I}\right)t + \lambda_k \mathbf{I} t\right]}\mathbf{v}_p = e^{\mathbf{A}t}\mathbf{v}_p . \end{equation}

	The implication this equation is that, given a list of $n$ generalized eigenvectors $\left\{\mathbf{v}_1,\mathbf{v}_2,...,\mathbf{v}_n\right\}$ of $\mathbf{A}$, then the set of functions $\left\{e^{\mathbf{A}t}\mathbf{v}_1,e^{\mathbf{A}t}\mathbf{v}_2,...,e^{\mathbf{A}t}\mathbf{v}_n\right\}$ is a set of linearly independent functions of $\dot{\mathbf{x}} = \mathbf{Ax}$, meaning that the general solution of this LTI ODE is given by some linear combination of these vectors:

\begin{equation} \mathbf{x}(t) = \sum\limits_{k=1}^n \alpha_k e^{\mathbf{A}t}\mathbf{v}_k = e^{\mathbf{A}t}\left(\sum\limits_{k=1}^n \alpha_k \mathbf{v}_k\right) \label{eq:theo_homogeneous_solution_ltiode_matrix_lineardepsol} \end{equation}

	that is, $\mathbf{x}$ is $e^{\mathbf{A}t}$ multiplied by a constant vector that is a linear combination of the $\mathbf{v}_k$. Let $\mathbf{x}_0 = \mathbf{x}(0)$:

\begin{equation} \mathbf{x}_0 = e^{\mathbf{0}}\left(\sum\limits_{k=1}^n \alpha_k \mathbf{v}_k\right) = \mathbf{I}\left(\sum\limits_{k=1}^n \alpha_k \mathbf{v}_k\right) = \left(\sum\limits_{k=1}^n \alpha_k \mathbf{v}_k\right) \end{equation}

	and substituting this into \eqref{eq:theo_homogeneous_solution_ltiode_matrix_lineardepsol},

\begin{equation} \mathbf{x}(t) = e^{\mathbf{A}t}\mathbf{x}_0 \end{equation}

\hfill$\blacksquare$
\vspace{5mm}
\hrule
\vspace{5mm}
%>>>

\begin{corollary}[Existence and uniqueness of the solutions of LTI ODEs]\label{theo:existence_uniqueness} %<<<
	The linear ODE $\dot{\mathbf{x}} = \mathbf{Ax}$ always has solutions for any time instant $t\in\mathbb{R}$. Furthermore, these solutions are unique: given $\mathbf{A}$, the initial condition $\mathbf{x}_0$ and the initial time $t_0$, the solution $\mathbf{x}(t)$ is unique.
\end{corollary}
\noindent\textbf{Proof.} Given that $\left\lVert e^{\mathbf{A}t}\right\rVert$ always exists for any $t$, given $\mathbf{x}_0$ then

\begin{equation} \left\lvert \mathbf{x}(t)\right\rvert = \left\lvert e^{\mathbf{A}t} \mathbf{x}_0\right\rvert \leq \left\lVert e^{\mathbf{A}t} \right\rVert\left\lvert \mathbf{x}_0\right\rvert \leq e^{\left\lVert\mathbf{A}\right\rVert t} \left\lvert \mathbf{x}_0\right\rvert\end{equation}

	\noindent which proves that $\mathbf{x}(t)$ exists. For uniqueness, suppose two distinct $\mathbf{x}_1(t)$, $\mathbf{x}_2(t)$ satisfy the ODE with the same initial condition $\mathbf{x}_0$. Then let $\mathbf{z} = \mathbf{x}_1 - \mathbf{x}_2$ and it is simple to see that $\mathbf{z}$ satisfies the ODE with null initial condition. But this means

\begin{equation} \mathbf{z}(t) = e^{\mathbf{A}t}\mathbf{z}_0 = e^{\mathbf{A}t}\mathbf{0} = \mathbf{0}(t)\end{equation}

	\noindent therefore $\mathbf{x}_1 = \mathbf{x}_2$ for any time $t$.
\hfill$\blacksquare$
\vspace{5mm}
\hrule
\vspace{5mm}
%>>>

%-------------------------------------------------
\section{Line and matrix ODEs}\label{sec:line_matrix_ode} %<<<1

	We now turn our concern towards ordinary equations of the form 	

\begin{equation} \sum\limits_{k=0}^n \alpha_k x^{\left(k\right)} + f(t) = 0, \label{eq:line_ode_def}\end{equation}

	\noindent which we may call ``line ODEs'' in contrast to the ``matrix ODEs'' $\dot{\mathbf{x}} = \mathbf{Ax + f}(t)$. Line ODEs are used when, instead of finding the behavior of all the states of the system at once, one wants to focus on a particular state; in the case of electrical circuits, some particular voltage or current.

\begin{example}[RLC circuit matrix ODE and line ODE equivalence]\label{example:rlc_circuit_matrix_ode} % EXAMPLE OF LTI ODE CIRCUIT MODELLING <<<
	Consider the figure \ref{fig:rlc_matrixline_example} where an RLC circuit is shown. This circuit has an excitation $u(t)$, given by a controlled voltage source, and an input $v(t)$, given by the voltage across the resistor load $R$. The circuit has two nodes and two loops are shown, a red and a green one.

% MODELLING EXAMPLE: RLC CIRCUIT <<<
\begin{figure}[htb!]
\centering
\scalebox{0.75}{
        \begin{tikzpicture}[american,scale=1.2,transform shape,line width=0.75, cute inductors]
		\draw (0,0)
			to[vsource,sources/scale=1.25, v>=$u(t)$,invert] (0,4)
			to[L,l=$L$,f>^=$i_L$,v>=$v_L(t)$,-*] (4,4) 
			to[C, l_=$C$,f>_=$i_C$, v^=$v_C(t)$, voltage shift = 0.5mm] (4,0)
			to[short] (0,0); 
		\node[shape=circle,draw,inner sep=1pt] at (4,4.5) {$1$};
		\draw (4,4) to[short] (5,4) to[short] (8,4) to[R,l_=$R$,f>_=$i_R$, v^=$v(t)$,voltage shift = 1mm] (8,0) to[short] (4,0);
		\draw[rounded corners=10,loop, draw opacity=0.3,->,color=red] (0.5,0.5) -- (0.5,3.5) -- (3.5,3.5) -- (3.5,0.5) -- (1,0.5) ;
		\draw[rounded corners=10,loop, draw opacity=0.3,->, color=blue] (4.5,0.5) -- (4.5,3.5) -- (7.5,3.5) -- (7.5,0.5) -- (5,0.5) ;
        \end{tikzpicture}
}
	\caption{RLC circuit as modelling example for ``matrix'' and ``line'' ODEs.}
	\label{fig:rlc_matrixline_example}
\end{figure} %>>>

	First, apply the KVL to the red loop and blue loops to yield

\begin{equation}
	\left\{\begin{array}{l}
		-u(t) + v_L(t) + v_C(t) = 0 \\[3mm]
		-v_C(t) + v(t) = 0
	\end{array}\right. \label{fig:ltiode_modelling_example_circuit_kvl_matrix}
\end{equation}

	Then apply the KCL to the node $1$:

\begin{equation} i_L(t) - i_C(t) - i_R(t) = 0 \label{fig:ltiode_modelling_example_circuit_kcl_matrix} \end{equation}

	Therefore, equations \eqref{fig:ltiode_modelling_example_circuit_kvl_matrix} and \eqref{fig:ltiode_modelling_example_circuit_kcl_matrix} form a three-equation system with six states $v_C,v_L,v,i_R,i_C,i_L$. The remaning three equations come from the equations of the circuit elements:

\begin{equation}
	\left\{\begin{array}{l}
		i_R(t) = Rv(t) \\[3mm]
		i_C(t) = C \dfrac{dv_C(t)}{dt} \\[3mm]
		v_L(t) = L\dfrac{di_L(t)}{dt}
	\end{array}\right. \label{fig:ltiode_modelling_example_circuit_components_matrix}
\end{equation}

\textbf{Matrix form:} let

\begin{equation} \mathbf{x} = \left[\begin{array}{c} v_C \\[3mm] i_L \end{array}\right] \end{equation}

	Then the first equation of \eqref{fig:ltiode_modelling_example_circuit_kvl} and \eqref{fig:ltiode_modelling_example_circuit_kcl} yield

\begin{equation} \left[\begin{array}{c} \dot{v}_C \\[3mm] \dot{i}_L \end{array}\right] = \left[\begin{array}{c} \dfrac{1}{C} \left(i_L - \dfrac{v_C}{R}\right) \\[3mm] \dfrac{1}{L}\left(u(t) - v_C\right) \end{array}\right] = \left[\begin{array}{cc} -\dfrac{1}{RC} & \dfrac{1}{C} \\[3mm] -\dfrac{1}{L} & 0 \end{array}\right]\left[\begin{array}{c} v_C \\[3mm] i_L \end{array}\right] + \dfrac{1}{L}\left[\begin{array}{c} 0 \\[3mm] u(t)\end{array}\right] \end{equation}

\textbf{``Line'' form:} now suppose we only want the ODE that models the load voltage $v(t)$. Differentiate the resistor equation to yield

\begin{equation} \dfrac{di_R(t)}{dt} = R\dfrac{dv(t)}{dt} \end{equation}

	and substitute this and \eqref{fig:ltiode_modelling_example_circuit_components_matrix} into \eqref{fig:ltiode_modelling_example_circuit_kcl} to yield

\begin{equation} \dfrac{v_L(t)}{L} - C\dfrac{d^2v_C(t)}{dt^2} - \dfrac{1}{R}\dfrac{dv(t)}{dt} = 0 \end{equation}

	Now use \eqref{fig:ltiode_modelling_example_circuit_kvl} to yield $v_C(t) = v(t)$ and $v_L(t) = u(t) - v(t)$:

\begin{equation} \dfrac{\left(u(t) - v(t)\right)}{L} - C\dfrac{d^2v(t)}{dt^2} - \dfrac{1}{R}\dfrac{dv(t)}{dt} = 0 \end{equation}

	and reorganizing this equation,

\begin{equation} LC\dfrac{d^2v(t)}{dt^2} + \dfrac{L}{R}\dfrac{dv(t)}{dt} + v(t) - u(t) = 0 \end{equation}

\examplebar
\end{example}

	In a mathematics setting, ``line ODEs'' mean we turn the interest of study from a system-wide perspective to a single-state perspective, allowing to develop transforms and results for a single state and then replicating the results for the system. This is exactly what will be done in the Classical and Dynamic Phasor theories of this text.

	We want to show that all of the results developed here can be transported to ``line ODEs'' in a simple manner. We also want to show that, in the case of line ODEs, eigenanalysis is made easier by the fact that eigenvalues can be calculated in a simpler manner by finding the roots of the polynomial 

\begin{equation} H(z) = \sum\limits_{k=0}^n \alpha_k z^{k}, \end{equation}

	\noindent which is obtained obviously by substituting the derivatives of \eqref{eq:line_ode_def} into exponentials. We can also obtain the modes and eigenvectors of the matrix easily this way.

	The next two theorems prove that any line ODE can be transformed into a matrix ODE, and that a matrix ODE of size $n$ can be transformed into $n$ line ODEs, effectively showing that there is some equivalente between these two types of ODEs. First, proving the equivalence from a line to a matrix ODE is simple: as shown in the next theorem,a line ODE can produce a matrix ODE by means of a \textit{Companion Matrix}.

\begin{theorem}[Line-to-matrix ODEs equivalence] \label{theorem:line_to_matrixode_equiv} %<<<

	Consider the line ODE

\begin{equation} \sum\limits_{k=0}^n \alpha_k x^{\left(k\right)} + f(t) = 0, \label{eq:linode_line2matrix}\end{equation}

	\noindent where $\alpha_k\in\mathbb{C},\ k\in\mathbb{N}_n$ is a sequence of complex numbers and $\alpha_n\neq 0$, and $x,f \in\left[\mathbb{R}\to\mathbb{C}\right]$. Let the Frobenius Companion Matrix, or simply companion matrix, of this ODE be the square matrix defined as

\begin{equation} \mathbf{C}_M = \left[\begin{array}{ccccc} 0 & 1 & 0 & ... & 0 \\[3mm] 0 & 0 & 1 & ... & 0  \\[3mm] \vdots & \vdots & \vdots & \ddots & \vdots \\[3mm] 0 & 0 & 0 & ... & 1 \\[3mm] - \dfrac{\alpha_0}{\alpha_n} & - \dfrac{\alpha_1}{\alpha_n} & -\dfrac{\alpha_2}{\alpha_n} & ... & -\dfrac{\alpha_{(n-1)}}{\alpha_n} \end{array}\right] . \end{equation}

	Then the solution $x(t)$ to \eqref{eq:linode_line2matrix} satisfies

\begin{equation} \dfrac{d}{dt}\left[\begin{array}{c} x \\[3mm] \dot{x} \\[3mm] \ddot{x} \\[3mm] \vdots \\[3mm] x^{(n-1)}\end{array}\right] = \mathbf{C}_M \left[\begin{array}{c} x \\[3mm] \dot{x} \\[3mm] \ddot{x} \\[3mm] \vdots \\[3mm] x^{(n-1)}\end{array}\right] + \left[\begin{array}{c} 0 \\[3mm] 0 \\[3mm] 0 \\[3mm] \vdots \\[3mm] -\dfrac{f(t)}{\alpha_n} \end{array}\right] . \end{equation}

\end{theorem}
\noindent\textbf{Proof:} first note that for $0\leq k \leq n$, $x^k = d/dt\left(x^{(k-1)}\right)$. This means that if we build the vector

\begin{equation} \mathbf{y} = \left[\begin{array}{c} x \\[3mm] \dot{x} \\[3mm] \ddot{x} \\[3mm] \vdots \\[3mm] x^{(n-1)}\end{array}\right], \end{equation}

	\noindent that is, the vector $\mathbf{y}$ such that $y_k = x^{(k-1)}$, then the first $n-1$ elements of $\mathbf{x}$ are such that $\dot{y}_k = x^{(k-1)} = y_{(k-1)}$. This generates the ``right-shifted identity'' block of $\mathbf{C}_M$. However, the last element $y_{(n-1)} = x_{(n-1)}$ can be written as a linear combination of the other elements due to the ODE itself:

\begin{equation} \sum\limits_{k=0}^n \alpha_k x^{\left(k\right)} + f(t) = 0 \Leftrightarrow \alpha_n x^{\left(n\right)} = -\sum\limits_{k=0}^{(n-1)} \alpha_k x^{\left(k\right)} - f(t) \Leftrightarrow y_n = \sum_{k=1}^{(n-1)} -\dfrac{\alpha_k}{\alpha_n}y_k - \dfrac{f(t)}{\alpha_n} \end{equation}

	\noindent which generates the last row of $\mathbf{C}_M$ and the excitation vector.
\hfill$\blacksquare$
\vspace{5mm}
\hrule
\vspace{5mm} %>>>

%-------------------------------------------------
\subsection{Matrix-to-line equivalence and the Cayley-Hamilton Theorem} %<<<2

	To prove the reverse equivalence between a matrix ODE and a line ODE, one needs the Cayley-Hamilton Theorem, a seminal theorem in Linear Algebra.

\begin{lemma}[Schur Decomposition]\label{lemma:schur_decomp} %<<<
	Any square complex matrix is \textbf{triangularizable}, that is, it is similar to an upper triangular matrix.\end{lemma}
\noindent\textbf{Proof:} an upper triangular matrix is a matrix $\mathbf{T}$ such that $t_{ij} = 0$ if $j > i$, that is, all elements below the diagonal are null. The theorem proves that any $\mathbf{A}\in\mathbb{C}^{(n\times n)}$ is equivalent to an upper triangular matrix but with small adaptations one can prove this is also the case for some lower triangular matrix.

	Pick an eigenvalue-eigenvector $\lambda_1,\mathbf{v}_1$ pair of $\mathbf{A}$; then there exists a basis $\mathbf{V} = \left\{\mathbf{v}_1,\mathbf{v}_2,...,\mathbf{v}_n\right\}$ where $\mathbf{v}_2,\mathbf{v}_3,...,\mathbf{v}_n$ are not yet chosen. Then $\mathbf{B}^{-1}\mathbf{AB}$ is a matrix such that

\begin{equation} \mathbf{B}^{-1}\mathbf{AB} = \left[\begin{array}{ccccc} \lambda_1 & x_{12} & x_{13} & \cdots & x_{1n} \\[3mm] 0 & x_{22} & x_{23} & \cdots & x_{2n} \\[3mm] 0 & x_{32} & x_{33} & \cdots & x_{3n} \\[3mm] \vdots & \vdots & \vdots & \ddots & \vdots \\[3mm] 0 & x_{n2} & x_{n3} & \cdots & x_{nn} \end{array}\right]. \end{equation}

	Now pick the bottom-right sub-matrix

\begin{equation} \mathbf{A}_2 = \left[\begin{array}{cccc} x_{22} & x_{23} & \cdots & x_{2n} \\[3mm] x_{32} & x_{33} & \cdots & x_{3n} \\[3mm] \vdots & \vdots & \ddots & \vdots \\[3mm] x_{n2} & x_{n3} & \cdots & x_{nn} \end{array}\right]_{(n-1)\times (n-1)}. \end{equation}

	\noindent and repeat this process: let $\mathbf{v}_2 = \left[v_{21}, \mathbf{u}_2^\transpose\right]^\transpose$, with $\lambda_2, \mathbf{u}_2$ an eigenvalue-eigenvector pair of $\mathbf{A'}$ and $v_{21}$ an adjustment complex number so as to make $\mathbf{v}_2$ orthogonal to $\mathbf{v}_1$. Then $\mathbf{A}'$ is reduced down to another upper-triangular matrix. Take the bottom-right submatrix

\begin{equation} \mathbf{A}_3 = \left[\begin{array}{cccc} y_{33} & y_{34} & \cdots & y_{3n} \\[3mm] y_{43} & y_{44} & \cdots & y_{4n} \\[3mm] \vdots & \vdots & \ddots & \vdots \\[3mm] y_{n3} & y_{n4} & \cdots & y_{nn} \end{array}\right]_{(n-2)\times (n-2)}. \end{equation}

	Now adopt $\lambda_3, \mathbf{u}_3$ an eigenpair of this new matrix and $\mathbf{v}_3 = \left[v_{31},v_{32},\mathbf{u}_3^\transpose\right]^\transpose$, $v_{31}$ and $v_{32}$ adjusted to make $\mathbf{v}_3$ orthogonal to $\mathbf{v}_2$ and $\mathbf{v}_1$. Do this process until the decomposition of $\mathbf{A}$ is exhausted, and the base built $\mathbf{V}$ will be lower-triangular and orthogonal.
\hfill$\blacksquare$
\vspace{5mm}
\hrule
\vspace{5mm} %>>>
\begin{lemma}\label{lemma:matrix_sim_powers} %<<<
	Matrix similarity is closed to the power operation, that is, if two square matrices $\mathbf{A}$ and $\mathbf{B}$ are similar with a similarity matrix $\mathbf{P}$, then $\mathbf{A}^k$ is similar to $\mathbf{B}^k$ with the same similarity matrix $\mathbf{P}$ for any positive $k$. If $\mathbf{A}$ is invertible (meaning $\mathbf{B}$ is also invertible) the relationship is also valid for negative powers.
\end{lemma}
\noindent\textbf{Proof:}  from the hypothesis, $\mathbf{A}$ and $\mathbf{B}$ are related by $\mathbf{A} = \mathbf{P}^{-1}\mathbf{BP}$ with $\mathbf{P}$ invertible. Then

\begin{equation} \mathbf{A}^2 = \left(\mathbf{P}^{-1}\mathbf{BP}\right)^2 = \mathbf{P}^{-1}\mathbf{BP}\mathbf{P}^{-1}\mathbf{BP} = \mathbf{P}^{-1}\mathbf{B}^2\mathbf{P} .\end{equation}

	The result for the k-th power can be proven by induction: if the proposition is true for a power $k$, then

\begin{equation} \mathbf{A}^{k+1} = \mathbf{A}^k \left(\mathbf{P}^{-1}\mathbf{BP}\right) = \mathbf{P}^{-1}\mathbf{B}^k\mathbf{P} \mathbf{P}^{-1}\mathbf{BP} = \mathbf{P}^{-1}\mathbf{B}^{k+1}\mathbf{P} .\end{equation}

	If $\mathbf{A}$ is invertible, then it is simple to see that $\mathbf{B}^{-1}$ exists and $\mathbf{A}^{-1} = \mathbf{P}^{-1}\mathbf{B}^{-1}\mathbf{P}$, and the same induction process yields the results for negative powers.
\hfill$\blacksquare$
\vspace{5mm}
\hrule
\vspace{5mm} %>>>
\begin{lemma}\label{lemma:gl_fund_algebra} %<<<
	The general linear group of complex matrices of order $n$ $\text{GL}\left(\mathbb{C},n\right)$ (the set of all invertible complex square matrices of size $n$) is a closed ring (it is closed to multiplication) and adheres to the Fundamental Theorem of Algebra, that is, all polynomials in this space can be broken down into a multiplication of the monomials of its roots.
\end{lemma}
\noindent\textbf{Proof:} it is simple to prove that if two matrices $\mathbf{A,B}$ are invertible, then $\left(\mathbf{AB}\right)^{-1} = \mathbf{B}^{-1}\mathbf{A}^{-1}$: take two arbitrary vectors $\mathbf{x,y}$ such that

\begin{equation} \mathbf{ABx} = \mathbf{y} .\label{eq:mult_inv_init}\end{equation}

	Let $\mathbf{Bx} = \mathbf{z}$; due to the invertibility of $\mathbf{A}$ for every $\mathbf{y}$ a unique $\mathbf{z}$ can be found, and due to the invertibility of $\mathbf{B}$ for every $\mathbf{z}$ an $\mathbf{y}$ can be found; therefore, for every $\mathbf{y}$ there exists a unique $\mathbf{x}$. This proves $\mathbf{AB}$ is invertible. Now first multiply \eqref{eq:mult_inv_init} on the left by $\mathbf{A}^{-1}$ and then by $\mathbf{B}^{-1}$ to yield 

\begin{equation} \mathbf{y} = \mathbf{B}^{-1}\mathbf{A}^{-1}\mathbf{x} \end{equation}

	\noindent yielding the identity $\left(\mathbf{AB}\right)^{-1} = \mathbf{B}^{-1}\mathbf{A}^{-1}$ sought, proving that the General Linear group is unchanged to multiplication, therefore it is a closed ring. Now consider a polynomial 

\begin{equation} \mathbf{P}\left(\mathbf{X}\right) = \mathbf{A}\mathbf{X} + \mathbf{B},\end{equation}

	\noindent where $\mathbf{A,B}$ are invertible. Then clearly the only root of $\mathbf{P}\left(\mathbf{X}\right) = \mathbf{0}$ is $\mathbf{X} = -\mathbf{A}^{-1}\mathbf{B}$. This entails to the fact that $\text{GL}\left(\mathbb{C},n\right)$ adheres to the Fundamental Theorem of Algebra; pick a polynomial

\begin{equation} \mathbf{P}\left(\mathbf{X}\right) = \sum_{k=0}^n \mathbf{A}_k\mathbf{X}^k .\end{equation}

	Then because the minimal polynomials have unique solutions, take one root $\mathbf{X}_1$ of $\mathbf{P}$ and divide the polynomial, yielding

\begin{equation} \mathbf{P}\left(\mathbf{X}\right) = \left(\mathbf{X} - \mathbf{X}_1\right) \mathbf{Q}_1\left(\mathbf{X}\right) .\end{equation}

	And now take a root of $\mathbf{Q}_1$ and factorize it; by induction, there are $n$ solutions $\left(\mathbf{X}_k\right)_{k=1}^n$ of $\mathbf{P}$ and

\begin{equation} \mathbf{P}\left(\mathbf{X}\right) = \prod_{k=1}^n \left(\mathbf{X} - \mathbf{X}_k\right) .\end{equation}
\hfill$\blacksquare$
\vspace{5mm}
\hrule
\vspace{5mm} %>>>
\begin{lemma}\label{lemma:triangular_inveritibility} %<<<
	Any upper triangular matrix $\mathbf{T}\in\mathbb{C}^n{(n\times n)}$ is invertible. Its eigenvalues are its diagonal values and the eigenvectors $\left(\mathbf{u}_k\right)_{k=1}^n$ of $\mathbf{T}$ are such that $\mathbf{u}_k$ has null components until the $k-1$ coordinate.
\end{lemma}
\noindent\textbf{Proof:} it is simple to see that any triangular matrix is invertible because its columns are inherently linearly independent. Take an arbitrary vector $\mathbf{u} = \left[u_1,u_2,...,u_n\right]$ and calculating the eigenvalues of $\mathbf{T}$ yields

\begin{equation} \mathbf{Tu} = \left[\begin{array}{ccccc} t_{11} & t_{12} & t_{13} & \cdots & t_{1n} \\[3mm] 0 & t_{22} & t_{23} & \cdots & t_{2n} \\[3mm] 0 & 0 & t_{33} & \cdots & t_{3n} \\[3mm] \vdots & \vdots & \vdots & \ddots & \vdots \\[3mm] 0 & 0 & 0 & \cdots & t_{nn} \end{array}\right]\left[\begin{array}{c} u_1 \\[3mm] u_2 \\[3mm] u_3 \\[3mm] \vdots \\[3mm] u_n \end{array}\right]\ \left\{\begin{array}{l} t_{(n,n)}u_n = \lambda u_n \\[3mm] t_{(n-1,n)}u_{n} + t_{(n-1,n-1)}u_{(n-1)} = \lambda u_{(n-1)} \\[3mm] t_{(n-2,n)}u_{n} + t_{(n-2,n-1)}u_{(n-1)} + t_{(n-2,n-2)}u_{(n-2)} = \lambda u_{(n-2)} \\[3mm] \hspace{5mm} \vdots \end{array}\right. \end{equation}

	\noindent yielding $n$ equations of the form

\begin{equation} \sum_{k=i}^n t_{(i,k)} u_k = \lambda u_i .\end{equation}

	Now, we can take the first equation $t_{(n,n)}u_n = \lambda u_n$ and one of the solutions is $\lambda = t_{(n,n)}$; adopt some non-null value for $u_n$ and all the other components of $\mathbf{u}$ can be calculated from the following equations. Or we can assume $u_n = 0$. In this case, the next equation yields

\begin{equation} t_{(n-1,n)}u_{n} + t_{(n-1,n-1)}u_{(n-1)} = \lambda u_{(n-1)} \Rightarrow t_{(n-1,n-1)}u_{(n-1)} = \lambda u_{(n-1)} \end{equation}

	\noindent which can mean $\lambda = t_{(n-1,n-1)}$, and adopt some $u_{(n-1)}$, and all the other components of $\mathbf{u}$ can be calculated. Or one can take $u_{(n-1)} = 0$. And so on.

	This process means that the eigenvectors of $\mathbf{T}$ will be such that $\mathbf{u}_i$ has null elements from $u_{(i+1)}$ through $u_n$ and that $t_{ii}$ is the corresponding eigenvalue.
\hfill$\blacksquare$
\vspace{5mm}
\hrule
\vspace{5mm} %>>>
\begin{lemma}\label{lemma:null_matrix} %<<<
	The only matrix which kernel is the entire $\mathbb{C}^n$ is the null matrix. Equivalently, a matrix vanishes a basis if and only if it is the null matrix.
\end{lemma}
\noindent\textbf{Proof:} it is simple to see that if a matrix has at least one non-null element, say in column $k$, then its multiplication by the canonical vector $\mathbf{e}_k$ is that column, that is, a non-null vector. Therefore, the kernel of this matrix is not the entire $\mathbb{C}^n$. Equivalently, if a matrix $\mathbf{A}$ is such that $\mathbf{Ae}_k$ is not the null vector for one of the canonical vectors, it is not null because $\mathbf{Ae}_k$ is its k-th column. At the same time, if the kernel of a matrix is the entire space, this can only mean it is the null matrix because $\mathbf{Ae}_k$ is always zero. Therefore, a matrix is null if and only if it vanishes the canonical basis. Elementary, this means that $\mathbf{AI}_n = \mathbf{0}$ if and only if $\mathbf{A}$ is the null matrix, which is obvious.

	For the second claim, pick a basis $\mathbf{U} = \left(\mathbf{u}_k\right)_{k=1}^n$ such that $\mathbf{AU} = 0$, that is, $\mathbf{A}$ vanishes every element in the basis. Because $\mathbf{U}$ is a basis, every canonical vector $\mathbf{e}_k$ has a coordinate in this basis, that is, $\mathbf{U}$ and $\mathbf{I}_n$ are similar (a consequence of theorem \ref{theo:bases_similarity}). Because the similarity matrix is invertible, $\mathbf{AI_n} = \mathbf{P}^{-1}\mathbf{AUP} = \mathbf{P}^{-1}\mathbf{0P} = \mathbf{0}$. Therefore $\mathbf{A}$ can only be the null matrix.
\hfill$\blacksquare$
\vspace{5mm}
\hrule
\vspace{5mm} %>>>
\begin{theorem}[Cayley-Hamilton Theorem] \label{theo:cayley_hamilton} %<<<
	Any complex matrix vanishes its own characteristic polynomial, that is: take some $\mathbf{A}\in\mathbb{C}^{(n\times n)}$ and $P_\mathbf{A}\left(x\right)$ its characteristic polynomial

\begin{equation} P_\mathbf{A}\left(x\right) = \det\left(x\mathbf{I}_n - \mathbf{A}\right) = \sum_{k=0}^n \alpha_k x^k. \label{theo:def_char_poly}\end{equation}

	Then $\mathbf{A}$ vanishes $P_\mathbf{A}$, that is,

\begin{equation} P_\mathbf{A}\left(\mathbf{A}\right) = \sum_{k=0}^n \alpha_k \mathbf{A}^k = \mathbf{0}.\end{equation}
\end{theorem}
\noindent\textbf{Proof:} on the complex numbers, the characteristic polynomial of $\mathbf{A}$ is defined as in \eqref{theo:def_char_poly}. But we can extend this definition to a polynomial over the space of complex matrices as

\begin{equation} \mathbf{P_\mathbf{A}}\left(\mathbf{X}\right):\ \left\{\begin{array}{rcl} \mathbb{C}^{(n\times n)} &\to& \mathbb{C}^{(n\times n)}\\[3mm] \mathbf{X} &\mapsto& \displaystyle\sum_{k=0}^n \alpha_k \mathbf{X}^k \end{array}\right. .\label{eq:char_long}\end{equation}

	The theorem claims that $\mathbf{A}$ vanishes this matrix polynomial, that is, it is a root of the polynomial: $\mathbf{P_\mathbf{A}}\left(\mathbf{X}\right) = \mathbf{0}$, where $\mathbf{0}$ is the null square matrix. From lemma \ref{lemma:schur_decomp}, $\mathbf{A}$ can be decomposed into a triangular matrix $\mathbf{T}$ through some orthogonal basis $\mathbf{V}$. The theorem then shortens to proving that $\mathbf{P}_\mathbf{A}\left(\mathbf{T}\right)$ vanishes all the vectors in $\mathbf{V}$ of $\mathbf{T}$, which by lemma \ref{lemma:null_matrix} means $\mathbf{P}_\mathbf{A}\left(\mathbf{T}\right) = \mathbf{0}$. Then we use the fact that similarity is closed to power (lemma \ref{lemma:matrix_sim_powers}) and because $\mathbf{A}$ and $\mathbf{T}$ are similar,

\begin{equation} P_\mathbf{A}\left(\mathbf{A}\right) = \sum_{k=0}^n \alpha_k \mathbf{A}^k = \sum_{k=0}^n \alpha_k \mathbf{V}^{-1}\mathbf{T}^k\mathbf{V} = \mathbf{V}^{-1}\left(\sum_{k=0}^n \alpha_k \mathbf{T}^k\right)\mathbf{V} = \mathbf{V}^{-1}\mathbf{P_A}\left(\mathbf{T}\right)\mathbf{V} = \mathbf{0} \end{equation}

	\noindent and the result is proven. To prove $\mathbf{P}_\mathbf{A}\left(\mathbf{T}\right)$ vanishes a base we start with the fact that similarity keeps eigenvalues, $\mathbf{T}$ will have the eigenvalues of $\mathbf{A}$. Using lemma \ref{lemma:gl_fund_algebra}, the characteristic polynomial of $\mathbf{A}$ can be defined as 

\begin{equation} \mathbf{P_\mathbf{A}}\left(\mathbf{X}\right) = \sum_{k=0}^n \alpha_k \mathbf{X}^k = \prod_{k=1}^n \left(\mathbf{X} - \lambda_k\mathbf{I}_n\right)\label{eq:m_vanish}\end{equation}

	\noindent and one can notice that $\mathbf{A}$ and $\mathbf{T}$ have the same eigenvalues, therefore the same characteristic polynomial, that is, $\mathbf{P_A}\left(\mathbf{X}\right) \equiv \mathbf{P_T}\left(\mathbf{X}\right)$. Therefore, the proof of the general case of the theorem — for an arbitrary complex matrix $\mathbf{A}$ — reduces to proving a special case for triangular matrices.

	We note that, since any matrix $\mathbf{X}$ commutes with itself and the identity, then the monomials $\mathbf{X} - \lambda_k\mathbf{I}_n$ commute amongst themselves. Therefore, it does not matter the order in which the monomials of $\mathbf{P_T}\left(\mathbf{X}\right)$ of \eqref{eq:m_vanish} are written. Due to lemma \ref{lemma:triangular_inveritibility}, any triangular matrix $\mathbf{T}$ is invertible; then take an eigenbasis $\left(\mathbf{u}_k\right)_{k=1}^n$ of $\mathbf{T}$ and, because the order of the monomials does not matter, we smartly push the k-th eigenvalue monomial to the rightmost position:

\begin{equation} \mathbf{P_\mathbf{T}}\left(\mathbf{T}\right)\mathbf{u}_k  = \left(\mathbf{T} - \lambda_1\mathbf{I}_n\right)\left(\mathbf{T} - \lambda_k\mathbf{I}_2\right)\cdots\overbrace{\left(\mathbf{T} - \lambda_k\mathbf{I}_k\right)\mathbf{u}_k}^{=\mathbf{0}} = \mathbf{0} \end{equation}

	\noindent which formally is written

\begin{equation} \mathbf{P_\mathbf{T}}\left(\mathbf{T}\right)\mathbf{u}_k = \left[\prod_{\substack{i=1 \\ i\neq k} }^n \left(\mathbf{T} - \lambda_i\mathbf{I}_n\right)\right]\left(\mathbf{T} - \lambda_k\mathbf{I}_k\right)\mathbf{u_k} = \left[\prod_{\substack{i=1 \\ i\neq k} }^n \left(\mathbf{T} - \lambda_i\mathbf{I}_n\right)\right]\mathbf{0} = \mathbf{0}.\end{equation}

	Therefore, $\mathbf{P_\mathbf{T}}\left(\mathbf{T}\right)$ vanishes a basis of $\mathbb{C}^n$ meaning it is the null matrix by lemma \ref{lemma:null_matrix}.
\hfill$\blacksquare$
\vspace{5mm}
\hrule
\vspace{5mm} %>>>

\begin{theorem}[Matrix-to-line ODEs equivalence] \label{theorem:ode_matrix_equiv} %<<<

	Consider the matrix ODE

\begin{equation} \dot{\mathbf{x}} = \mathbf{Ax} + \mathbf{B}\mathbf{f}(t), \label{eq:theo_mtl_originalode}\end{equation}

	\noindent where $\mathbf{A}\in\mathbb{C}^{(n\times n)},\ \mathbf{B}\in\mathbb{C}^{(n\times n)},\ \mathbf{f}\in\left[\mathbb{R}\to\mathbb{C}^m\right]$. If $\mathbf{f}\in C^n$, then this ODE is equivalent to $n$ ``line'' ODEs where each $x_i$ satisfies 

\begin{equation} \sum_{k=0}^n \alpha_k x_i^{(k)} + g_i(t) = 0, \label{eq:theo_nthnonst_end}\end{equation}

	\noindent where the $\alpha_k$ are the coefficients of the characteristic polynomial of $\mathbf{A}$ and  $g_i$ is the i-th line of

\begin{equation} \mathbf{g}(t) = -\sum_{k=0}^n \alpha_k\left(\ \sum_{j=0}^{k-1} \mathbf{A}^j\mathbf{B}\mathbf{f}^{(k-j)}\right) . \label{eq:theo_nthnonst_g}\end{equation}

\end{theorem}
\noindent\textbf{Proof:} take the original ode \eqref{eq:theo_mtl_originalode} and taking several derivatives yields

\begin{gather}
\ddot{\mathbf{x}} = \mathbf{A}\left(\mathbf{Ax} + \mathbf{Bf}\right) + \dot{\mathbf{f}} = \mathbf{A}^2\mathbf{x} + \mathbf{ABf} + \mathbf{B}\dot{\mathbf{f}}\\\nonumber
\dddot{\mathbf{x}} = \mathbf{A}^2\left(\mathbf{Ax} + \mathbf{Bf}\right) + A\dot{\mathbf{f}} + \ddot{\mathbf{f}} = \mathbf{A}^3\mathbf{x} + \mathbf{A}^2\mathbf{Bf} + \mathbf{AB}\dot{\mathbf{f}} + \mathbf{B}\ddot{\mathbf{f}} \\\nonumber
\vdots
\end{gather} 

	\noindent and induction yields

\begin{equation} \mathbf{x}^{(k)} = \mathbf{A}^k \mathbf{x} + \sum_{j=0}^{k-1} \mathbf{A}^j \mathbf{Bf}^{(k-j)}, k\geq 1\end{equation}

        Now take an arbitrary linear combination of all these derivatives up to $n$ with coefficients $\alpha_i$:

\small \begin{equation} \sum_{k=0}^n \alpha_k  \mathbf{x}^{(k)} = \left(\sum_{k=0}^n \alpha_k \mathbf{A}^k \right) \mathbf{x} + \sum_{k=1}^n \alpha_k \left(\ \sum_{j=0}^{k-1} \mathbf{A}^j \mathbf{Bf}^{(k-j)}\right) \end{equation}\normalsize

        Choose the $\alpha_k$ as the coefficients of the characteristic polynomial of $A$. By the Cayley-Hamilton Theorem, the term in parenthesis vanishes:

\begin{equation} \sum_{k=0}^n \alpha_k \mathbf{x}^{(k)} = \sum_{k=1}^n \alpha_k \left(\ \sum_{j=0}^{k-1} \mathbf{A}^j \mathbf{Bf}^{(k-j)}\right) \label{eq:theo_nthnonst_beg}\end{equation}

        The i-th line of \eqref{eq:theo_nthnonst_beg} yields \eqref{eq:theo_nthnonst_end} and \eqref{eq:theo_nthnonst_g}.
\hfill$\blacksquare$
\vspace{5mm}
\hrule
\vspace{5mm} %>>>

%-------------------------------------------------
\subsection{The Frobenius Matrix and Hurwitz Polynomial} %<<<2

	Having proven that a matrix ODE is equivalent to $i$ line ODEs, we now study the homogeneous behavior of line ODEs. The next theorem shows that the modes of the ODE are easily obtainable by solving the polynomial given by the Companion matrix characteristic polynomial; and this one is also easily obtainable by substituting the derivatives of the equation by powers.

\begin{theorem}[Hurwitz Polyomial] \label{lemma:hurwitz_polynomial} %<<<

	Consider a homogeneous line ODE

\begin{equation} \sum\limits_{k=0}^n \alpha_k x^{\left(k\right)}  = 0, \end{equation}

	\noindent where $\alpha_k\in\mathbb{C},\ k\in\mathbb{N}_n$ is a sequence of complex numbers and $\alpha_n\neq 0$, and $x \in\left[\mathbb{R}\to\mathbb{C}\right]$. Then the characteristic polynomial of the Frobeniuns Companion Matrix of this ODE is given by

\begin{equation} \mathbf{P_C}\left(x\right) = \left(-1\right)^{(n-1)} H(x).\end{equation}

	\noindent where $H(x)$, called the Hurwitz Polynomial of the ODE, is given by substituting the derivatives by powers:

\begin{equation} H(z) = \sum\limits_{k=0}^n \alpha_k z^k, \end{equation}

	\noindent such that the roots of $H(x)$ are the eigenvalues of $\mathbf{C_M}$ and the modes of the line ODE. Further, if $a_0\neq 0$, the geometric multiplicities of the eigenvectors are equal to their algebraic multiplicity.
\end{theorem}
\noindent\textbf{Proof:} the fact that the roots of $\mathbf{P_C}$ are the modes of the linear ODE are a direct consequence of theorem \ref{theorem:line_to_matrixode_equiv}. Without loss of generality suppose $\alpha_n = 1$; calculating the eigenvalues of $\mathbf{C}_P$:

\begin{equation} \det\left(x \mathbf{I} - \mathbf{C}_P\right) = \det\left(\left[\begin{array}{ccccc} x & -1 & 0 & ... & 0 \\[3mm] 0 & x & -1 & ... & 0  \\[3mm] \vdots & \vdots & \vdots & \ddots & \vdots \\[3mm] 0 & 0 & 0 & ... & -1 \\[3mm] a_0 & a_1 & a_2 & ... & x + a_{(n-1)} \end{array}\right]\right) \end{equation}

	and compute this determinant through Laplace expansion on the last row:

\begin{align} \det\left(x \mathbf{I} - \mathbf{C}_P\right) &= a_0 \det\left(\left[\begin{array}{cccccc} -1 & 0 & ... & 0 & 0 \\[3mm] x & -1 & ... & 0 & 0 \\[3mm] 0 & x & ... & 0 & 0 \\[3mm] \vdots & \vdots & \ddots & \vdots & \vdots \\[3mm] 0 & 0 & ... & -1 & 0 \\[3mm] 0 & 0 & ... & x & -1 \end{array}\right]\right) + \nonumber  \\[3mm] 
%
& \hspace{1cm} - a_1 \det\left(\left[\begin{array}{cccccc} x & 0 & ... & 0 & 0 \\[3mm] 0 & -1 & ... & 0 & 0 \\[3mm] 0 & x & ... & 0 & 0 \\[3mm] \vdots & \vdots & \ddots & \vdots & \vdots \\[3mm] 0 & 0 & ... & -1 & 0 \\[3mm] 0 & 0 & ... & x & -1 \end{array}\right]\right) + \nonumber\\[5mm]
%
%
%
& \hspace{2cm} + a_2 \det\left(\left[\begin{array}{cccccc} x & 0 & 0 & ... & 0 & 0 \\[3mm] 0 & x & 0 &... & 0 & 0 \\[3mm] 0 & 0 & -1 & ... & 0 & 0 \\[3mm] \vdots & \vdots & \vdots & \ddots & \vdots & \vdots \\[3mm] 0 & 0 & 0 & ... & -1 & 0 \\[3mm] 0 & 0 & 0 & ... & x & -1 \end{array}\right]\right) + ... + \nonumber\\[5mm] 
%
& \hspace{3cm} + \left(-1\right)^{\left(n-1\right)}\left(a_{(n-1)} + x\right) \det\left(\left[\begin{array}{cccccc} x & 0 & 0 & ... & 0 & 0 \\[3mm] 0 & x & 0 &... & 0 & 0 \\[3mm] 0 & 0 & x & ... & 0 & 0 \\[3mm] \vdots & \vdots & \vdots & \ddots & \vdots & \vdots \\[3mm] 0 & 0 & 0 & ... & x & 0 \\[3mm] 0 & 0 & 0 & ... & 0 & x \end{array}\right]\right)
\end{align}

	Therefore the determinant will be composed of $n$ terms made up of the coefficients $a_k$ and the concatenation of two triangular matrices plus a term for the last cofactor matrix:

\begin{equation} \det\left(x \mathbf{I} - \mathbf{C}_P\right) = \left(-1\right)^{(n-1)}x\det\left(\mathbf{\Lambda}_{(n-1)}\right) + \sum\limits_{k=0}^{n-1} a_k \left(-1\right)^{k}\det\left(\left[\begin{array}{cc} \mathbf{\Lambda}_k & \mathbf{0} \\[3mm] \mathbf{0} & \mathbf{T}_k \end{array}\right]\right) \end{equation}

	\noindent where $\mathbf{\Lambda}_k$ is the k-th order square diagonal matrix composed of only $x$ terms, and $\mathbf{T}_k$ is a triangular square matrix which diagonal is composed of $-1$ terms and its subdiagonal is composed of $x$:

\begin{equation}
\mathbf{\Lambda}_k = \left[\begin{array}{cccccc} \lambda & 0 & ... & 0 & 0 \\[3mm] 0 & \lambda & ... & 0 & 0 \\[3mm] \vdots & \vdots & \ddots & \vdots & \vdots \\[3mm] 0 & 0 & ... & 0 & \lambda \end{array}\right]_{\left(k\times k\right)},\
%
\mathbf{T}_k = \left[\begin{array}{cccccc} -1 & 0 & 0 & ... & 0 & 0 \\[3mm] x & -1 & 0 & ... & 0 & 0  \\[3mm] 0 & x & -1 & ... & 0 & 0 \\[3mm] \vdots & \vdots & \vdots & \ddots & \vdots & \vdots \\[3mm] 0 & 0 & 0 & ... & x & -1 \end{array}\right]_{\left(\raisebox{3mm}{} \left(n-1-k\right)\times \left(n-1-k\right)\right)}
\end{equation}

	But because both of these matrices are triangular, their determinants are easy to calculate as the multiplication of the diagonal elements:

\begin{equation} \det\left(\mathbf{\Lambda}_k\right) = x^k ,\ \det\left(\mathbf{T}_k\right) = \left(-1\right)^{\left(n-1-k\right)} \end{equation}

	which yields

\begin{align}
	\det\left(x \mathbf{I} - \mathbf{C}_M\right) &= \left(-1\right)^{(n-1)}x^n + \sum\limits_{k=0}^{n-1} a_k \left(-1\right)^{k}x^k \left(-1\right)^{\left(n-1-k\right)} \nonumber\\[5mm]
%
	 &= \left(-1\right)^{\left(n-1\right)}x^n + \left(-1\right)^{\left(n-1\right)}\sum\limits_{k=0}^{n-1} a_k x^k \nonumber\\[5mm]
%
	 &= \left(-1\right)^{\left(n-1\right)}\left(x^n + \sum\limits_{k=0}^{n-1} a_k x^k\right) = \left(-1\right)^{\left(n-1\right)}H(x) .
\end{align}

	Finally, this also means that the algebraic multiplicity of the eigenvalue $\lambda$ must be equal to its geometric multiplicity. If $a_0$ is not null then the columns of $\mathbf{C_M}$ are linearly independent, that is, the matrix is certainly invertible, which can only happen if the multiplicities match. If it is null, then the first column is null, therefore $0$ is an eigenvalue. Therefore attribute to $0$ an eigenvector linearly independent of all the other $(n-1)$. Even if more coefficients are zero, eigenvectors can be found: for instance if $a_1$ is also zero, due to the structure of $\mathbf{C_M}$ an eigenvector of the eigenvalue $0$ can still be found and need not be chosen because the column of $a_1$ is still linearly independent of all others; therefore, in this case, $0$ is a double root (algebraic multiplicity 2) but has geometric multiplicity two.
\hfill$\blacksquare$
\vspace{5mm}
\hrule
\vspace{5mm} %>>>

	And one of the main results of this theorem is that the diagonalization of the companion matrix is a direct consequence of the multiplicity of the roots of the Hurwitz Polynomial.

\begin{corollary}[Frobenius Companion Matrix] \label{lemma:frob_companion_matrix} %<<<
	The Frobenius Companion Matrix $\mathbf{C}_M$ is diagonalizable if and only if $H(z)$ has only simple roots — that is, it has $n$ distinct roots $\lambda_1,\lambda_2,...,\lambda_n$. In this case, the $n$ distinct eigenvectors of $\mathbf{C}_M$ are given by

\begin{equation} v_k = \left[1,\lambda_k^1,\lambda_k^2,...,\lambda_k^{(n-1)}\right]^\intercal \end{equation}

 	and $\mathbf{C}_M$ can be diagonalized as $\mathbf{C}_M = \mathbf{P}^{-1}\mathbf{D}\mathbf{P}$ where

\begin{equation}
	\mathbf{D} = \left[\begin{array}{ccccc} \lambda_1 & 0 & 0 & ... & 0 \\[3mm] 0 & \lambda_2 & 0 & ... & 0  \\[3mm] \vdots & \vdots & \vdots & \ddots & \vdots \\[3mm] 0 & 0 & 0 & ... & \lambda_n  \end{array}\right],\
%
	\mathbf{P} = \left[\begin{array}{ccccc} 1 & 1 & 1 & ... & 1 \\[3mm] \lambda_1 & \lambda_2 & \lambda_3 & ... & \lambda_n \\[3mm] \lambda_1^2 & \lambda_2^2 & \lambda_3^2 & ... & \lambda_n^2 \\[3mm] \vdots & \vdots & \vdots & \ddots & \vdots \\[3mm] \lambda_1^{(n-1)} & \lambda_2^{(n-1)} & \lambda_3^{(n-1)} & ... & \lambda_n^{(n-1)}  \end{array}\right],\
\end{equation}
\end{corollary}
\textbf{Proof:}  Now, calculate the eigenvectors: let $\mathbf{v} = \left[v_1,v_2,...,v_n\right]^\intercal$ and $\mathbf{C}_M\mathbf{v} = \lambda\mathbf{v}$ yields

\begin{equation}
	\left[\begin{array}{ccccc} 0 & 1 & 0 & ... & 0 \\[3mm] 0 & 0 & 1 & ... & 0  \\[3mm] \vdots & \vdots & \vdots & \ddots & \vdots \\[3mm] 0 & 0 & 0 & ... & 1 \\[3mm] -a_0 & -a_1 & -a_2 & ... & -a_{(n-1)} \end{array}\right] \left[\begin{array}{c} v_1 \\[3mm] v_2\\[3mm] \vdots \\[3mm] v_n \end{array}\right] = \lambda \left[\begin{array}{c} v_1 \\[3mm] v_2\\[3mm] \vdots \\[3mm] v_n \end{array}\right]
\end{equation}

	and writing this as a system of equations,

\begin{equation}
\left\{\begin{array}{c}
	v_2 = \lambda v_1 \\[3mm]
	v_3 = \lambda v_2 \\[3mm]
	\vdots \\[3mm]
	v_{(k+1)} = \lambda v_{(k)} \\[3mm]
	\vdots \\[3mm]
	v_{n} = \lambda v_{(n-1)} \\[3mm]
	-a_0v_1 -a_1v_2 + ... - a_{(n-1)}v_n = \lambda v_n
\end{array}\right. \label{eq:frob_companion_matrix_recursion}
\end{equation}

	The first $n-1$ equations form a recurrence that can be written as $v_k = \lambda^{(k-1)} v_1$ for $k$ from $2$ to $n$; the last one is equivalent to

\begin{equation} \lambda^n v_1 + \sum\limits_{k=0}^{n-1} a_k v_{(k+1)} + \lambda v_n = 0 \end{equation}

	and using the recurrence,

\begin{equation} \lambda^n v_1 + \sum\limits_{k=0}^{n-1} a_k \lambda^k v_1 = 0 \Leftrightarrow v_1 M\left(\lambda\right) = 0 \end{equation}

	This last equation is fulfilled for any $v_1$ because $M\left(\lambda\right)$ is null by definition. Therefore, adopt $v_1 = 1$ and the recurrence yields

\begin{equation} \mathbf{v} = \left[1,\lambda,\lambda^2,...,\lambda^{(n-1)}\right]^\intercal \label{eq:frob_companion_matrix_eigenvector} \end{equation}

	Equation \eqref{eq:frob_companion_matrix_eigenvector} shows that each $\lambda$ can have only a single eigenvalue pertaining to it, because, if $\mathbf{v}_1$ and $\mathbf{v}_2$ are different and pertain to the same $\lambda$, they must be related by some scaling in order to satisfy \eqref{eq:frob_companion_matrix_recursion}. Therefore, if $\mathbf{C}_m$ has $n$ distinct eigenvectors then it must have $n$ distinct eigenvalues, which is to say $M(x)$ has only simple roots.
\hfill$\blacksquare$
\vspace{5mm}
\hrule
\vspace{5mm} %>>>

%-------------------------------------------------
\subsection{General solution of a line ODE} %<<<2

	Finally, we can obtain the general solution of a line ODE simply by the eigenvalues of its companion matrix (the roots of the Hurwitz polynomial) and its multiplicities.

\begin{theorem}[General solution of a LTI ODE] \label{theo:homogeneous_solutions_ltiode_hurwitz} %<<<
	The general solution of the homogeneous LTI ODE

\begin{equation} \sum\limits_{k=0}^n \alpha_k x^{\left(k\right)} = 0 \label{theo:homogenous_solutions_ltiode_originalode}\end{equation}

	is given by

\begin{equation} x(t) = \sum\limits_{H\left(\lambda_k\right) = 0} e^{\lambda_k t}\left[c_{(k,0)},c_{(k,1)},...,c_{(k,\left(\mu\left(\lambda_k\right)-1\right))}\right]\left[\begin{array}{c} 1 \\ t \\ t^2 \\ \vdots \\t^{(\mu\left(\lambda_k\right)-1)} \end{array}\right] \label{theo:homogenous_solutions_ltiode_originalode_ithsol}  \end{equation}

	\noindent where the $c_{(k,a)}$ are complex scalars calculated using initial time conditions, the $\lambda_k$ are the roots of the \textit{Hurwitz Polynomial} of equation \eqref{theo:homogenous_solutions_ltiode_originalode}, $\mu\left(\lambda_k\right)$ is the algebraic multiplicity of $\lambda_k$. 
\end{theorem}
\noindent \textbf{Proof:} take an arbitrary component $x_i(t)$ of $\mathbf{x}$:

\begin{equation} \sum\limits_{k=0}^n \alpha_k x_i^{\left(k\right)} = 0 \label{theo:homogenous_solutions_ltiode_originalode_ith}\end{equation}

	and write this equation in matrix form $\dot{\mathbf{y}} = \mathbf{Ay}$, where $\mathbf{A}$ is the companion matrix of $H(x)$. From corollary \ref{theo:homogeneous_solutions_ltiode_matrix}, the general solution to this LTI ODE in $\mathbf{y}$ will be linear combinations of the

\begin{equation} \mathbf{y}_p(t) =  e^{\mathbf{A}t} \mathbf{v}_p =  \left(\displaystyle\sum\limits_{i=0}^{m_k-1} \dfrac{t^i}{i!}\mathbf{A}^i\right)\mathbf{v}_p e^{\lambda t} \end{equation}

	where the $\lambda_k$ are the eigenvalues of $\mathbf{A}$, that is, the distinct roots of $H(x)$, $m_k$ the algebraic multiplicity of $\lambda_k$, and the $\mathbf{v}_p$ are vectors in some Generalized Jordan Chain of $\lambda_k$. Then, from the fact that the comanion matrix $\mathbf{A}$ has unitary entries in the first row, the general solution to \eqref{theo:homogenous_solutions_ltiode_originalode_ith} will be the sum of the first components of each $\mathbf{y}_p$:

\begin{equation} x_i(t) =  \sum\limits_{k=1}^j e^{\lambda_k t}\left(\displaystyle\sum\limits_{a=0}^{m_k-1} c_{(k,a)}^i t^a\right) = \sum\limits_{k=1}^j e^{\lambda_k t}\left[c_{(k,0)}^i,c_{(k,1)}^i,...,c_{(k,(m_k-1))}^{i}\right]\left[\begin{array}{c} 1 \\ t \\ t^2 \\ \vdots \\t^{(m_k-1)} \end{array}\right]  \end{equation}

%	and grouping the equations into rows for all $i$,
%
%\begin{equation} \mathbf{x}(t) = \sum\limits_{k=j}^k e^{\lambda_k t}\left[\begin{array}{cccc} c_{(k,0)}^1 & c_{(k,1)}^1 &... & c_{(k,(m_k-1))}^{1}\\[3mm] c_{(k,0)}^2 & c_{(k,1)}^2 & ... & c_{(k,(m_k-1))}^{2} \\ \vdots & \vdots & \ddots & \vdots \\[3mm] c_{(k,0)}^n & c_{(k,1)}^n & ... & c_{(k,(m_k-1))}^{n}\end{array}\right]\left[\begin{array}{c} 1 \\ t \\ t^2 \\ \vdots \\t^{(m_k-1)} \end{array}\right] = \sum\limits_{k=1}^j e^{\lambda_k t}\mathbf{C}_k \mathbf{p}_{(m_k)}(t) \label{theo:homogenous_solutions_ltiode_originalode_total_sol} \end{equation}
\hfill$\blacksquare$
\vspace{5mm}
\hrule
\vspace{5mm}
%>>>

%-------------------------------------------------
\section{Stability} %<<<1

	Given that we now know the general solution to the linear DE $\dot{\mathbf{x}} = \mathbf{Ax}$, we start analyzing quantitatively how this equation behaves at some specific points and how these behaviors manifest in the most general, excited ODE $\dot{\mathbf{x}} = \mathbf{Ax + f}(t)$. We are interested in two behaviors: the equilibrium and the steady-state behavior. 

\begin{definition}[Dynamical System]\label{def:dynamical_system} A \textbf{Dyamical System} is a triad $\left(T,x,\varphi\right)$ where $T\subset\mathbb{R}$ is a closed time interval called the time domain, $x\in U$ is the state ($U$ being the state space) and $\varphi\in\left[T\times U\right]$ is an evolution function that satisfies

\begin{itemize}
	\item The system can start from any point in $U$, that is, $\varphi\left(0,x_0\right) = x_0$ for any $x_0\in U$ and
	\item The system time evolution is consistent, that is, in a time interval $t_1 + t_2$ the system evolves the ``same amount'' than it would if started from $t_1$ to $t_1 + t_2$, that is,
	\begin{equation} \varphi\left(t_1 + t_2,x\right) = \varphi\left(t_2,\varphi\left(t_1,x\right)\right).\end{equation}
\end{itemize}
\end{definition}

	In a mathematical context, a Dynamical System can be defined in a myriad of ways; intutively, a Dynamical System is a system that ``evolves in time'', that is, it describes the dependence of the state of a system with respect to time. We are interested in the specific class of continuous (or differential) Dynamical Systems, that is, the class of systems described by $\dot{\mathbf{x}} = \mathbf{f(x)}$ where $\mathbf{f}\in\left[\mathbb{C}^n\to\mathbb{C}^n\right]$ is continuous. Hereforth, ``Dynamical System'' refers to this class of system; it can be proven that this class of system indeed fulfills definition \ref{def:dynamical_system} by defining a certain function called a trajectory.

\begin{definition}[Orbit or trajectory] Consider the system defined by $\dot{\mathbf{x}} = \mathbf{f(x)},\ \mathbf{f}\in\left[\mathbb{C}^n\to\mathbb{C}^n\right]$. An \textbf{orbit} or \textbf{trajectory} of the system is a function $\varphi\left(t,\mathbf{x}_0\right)\in\left[\mathbb{R}\times\mathbb{C}^n\to\mathbb{C}^n\right]$ that represents the time evolution of the system starting from some initial point $\mathbf{x}_0$, that is, 

\begin{equation}\left\{\begin{array}{l} \dfrac{d}{dt} \varphi\left(t,\mathbf{x}_0\right) = \mathbf{f}\left(\varphi\left(t,\mathbf{x}_0\right)\right) \\[3mm] \varphi\left(t_0,\mathbf{x}_0\right) = \mathbf{x}_0\end{array}\right. , \end{equation}

	\noindent for some initial time $t_0$.
\end{definition}

	Existence of the trajectory $\varphi$ is a fundamental aspect of Dynamical Systems. In a general case, for an arbitrary $\mathbf{f}$, the Picard-Lindelöf Theorem \pcite{Perko2001} shows that for real systems, a solution to the initial value problem exists and is unique in some open interval containing $t_0$ if $\mathbf{f}$ is locally Lipschitz continuous, and under certain conditions the solution can be extended to a maximal interval. The requirements of this extension, however, are not simple for nonlinear systems: taking for example the ODE $\dot{x} = x^2$, one initially guesses this is an ideal candidate for continuation on the entire reals because $f = x^2$ is not only continuous but infinitely so, everywhere. However, the general solution to the ODE is

\begin{equation} x(t) = \dfrac{x_0}{1 - x_0 t} \end{equation}

	\noindent which explodes for $t\to x_0^{-1}$. Global variations of the Picard-Lindelöf Theorems do exist, with the obvious tradeoff that the requirements on $\mathbf{f}$ need to be harder; for instance, if $\mathbf{f}$ is globally Lipschitz, then a solution exists for all $t \geq t_0$. More forgiving theorems on the existence of $\varphi$ are also available, but most prove existence and not uniqueness, another major point of concern which generally is proven using the Banach-Cacioppoli Fixed Point Theorem.

	Luckily, for a a linear system of the type $\mathbf{f}\left(\mathbf{x}\right) = \mathbf{Ax}$ the existence of the orbit is guaranteed by corollay \ref{theo:existence_uniqueness} which states that the trajectory $\varphi$ exists for all $t\geq t_0$ and is unique to $\mathbf{x}_0$ and $t_0$.

	Therefore, a trajectory or orbit then defines the evolution function of the continuous Dynamical System $\dot{\mathbf{x}} = \mathbf{f(x)}$. Intuitively, a trajectory is the sequence of states that the system takes once it starts from a given initial point. The most trivial trajectory is an equilibrium, that is, a point $\mathbf{x}^*$ such that $\varphi\left(t,\mathbf{x}^*\right) = \mathbf{x}^*$ for all times $t$. 

\begin{definition}[Equilibrium] An \textbf{equilibrium} of the system $\dot{\mathbf{x}} = \mathbf{f(x)}$ is a point $\mathbf{x}^*$ such that $\mathbf{f}\left(\mathbf{x}^*\right) = \mathbf{0}$. \end{definition}

	It is simple to see that if a system starts at the particular point $\mathbf{x}^*$, then $\dot{\mathbf{x}} = \mathbf{0}$, that is, $\varphi\left(t,\mathbf{x}^*\right) = \mathbf{x}^*$ for all time $t\geq t_0$ — the system ``stays at that point'' because it ``does not move''. The most important aspect of equilibria is how the system behaves around them, that is, if the system is ``attracted to them'', ``repulsed by them'' or if the system somehow orbits \textit{around} them. 

\begin{definition}[Assymptotic stability of Dynamical Systems] A Dynamical System $\dot{\mathbf{x}} = \mathbf{f}\left(\mathbf{x}\right)$, $\mathbf{x}\in\left[\mathbb{R}\to\mathbb{C}^n\right]$ is \textbf{assymptotically stable} at an equilibrium point $\mathbf{x}^*$ if a trajectory $\mathbf{x}(t)$ starting from a sufficiently close initial condition $\mathbf{x_0}$ tends to $\mathbf{x}^*$ the origin at infinity, that is,

\begin{equation} \lim_{t\to\infty} \left\lvert \varphi\left(t,\mathbf{x}_0\right) - \mathbf{x}^*\right\rvert = \mathbf{0} \end{equation}

	\noindent where $\mathbf{x}_0$ belongs to some neighborhood of an equilibrium $\mathbf{x}^*$.
\end{definition}

	Intuitively, a system is stable at an equilibrium point if it tends to that point when started from an initial condition sufficiently close to it. ``Sufficiently close'' here means that if the system starts from an initial point too far away, it might be that it falls to another entirely different equilibrium or even behavior; nonlinear systems, in special, can manifest a myriad of different transient behaviors. This begets the notion that $\mathbf{x}_0$ must be in some vicinity, or neighborhood, of the equilibrium; this vicinity is called the \textbf{stability or attraction region or basin}  of $\mathbf{x}^*$. 
	
	Estimating the attraction region of equilibria for particular systems is the objective of much literature in applied sciences. Particularly for Electrical Power Systems, there is a wide body of literature regarding estimation methods for finding stability regions: \cite{Chiang_Alberto_2015} is a book wholly dedicated to this matter, showing many methods like energy methods, brute-force methods and so on; \cite{10838245,EstimatingTransientStability2025} show a method based on order reduction through Koopman operators, and \cite{10038021} shows an estimation based on a square-method approximation.

	Perhaps the most important aspect of equilibria is how the system behaves around them, that is, if the system is ``attracted to them'', ``repulsed by them'' or if the system somehow orbits them. The discussion on equilibria, qualities of equilibria, the characteristics of the stability boundaries of these points and the system behavior around them is a major subject in Dynamical Systems — especially nonlinear ones where a wide plethora of behaviors can manifest. For obvious reasons will not be discussed here: yet again, luckily, linear systems do not subscribe to those uncertainties that befall nonlinear ones. It is simple to see that any equilibrium point $\mathbf{x}^*$ is such that $\mathbf{Ax}^* = \mathbf{0}$, that is, $\mathbf{x}^*\in\Ker\left(\mathbf{A}\right)$. If $\mathbf{A}$ is invertible, then the kernel is comprised of only the origin. The objective in this text is to explore the specific class of continuous Dynamical Systems that can be defined by some equation $\dot{\mathbf{x}} = \mathbf{Ax} + \mathbf{f}(t)$ (which we can loosely call ``linear systems'') and its non-forced version $\dot{\mathbf{x}} = \mathbf{Ax}$. We want to assert under which conditions the linear system is stable, that is, what are the minimum requirements on $\mathbf{A}$ such that the linear system $\dot{\mathbf{x}} = \mathbf{Ax}$ it defines is stable. To this wise, theorem \ref{def:exp_charac} shows that the response of this system is inherently exponential.

\begin{lemma}\label{lemma:matrix_exp_eigen}
	If $\lambda$ is an eigenvalue of $\mathbf{A}$ with eigenvector $\mathbf{v}$, then $e^{\lambda}$ is an eigenvalue of $e^{\mathbf{A}}$ with eigenvector $\mathbf{v}$.
\end{lemma}
\noindent\textbf{Proof: } let $\lambda$ the eigenvalue, $\mathbf{v}$ the associated eigenvector; then

\begin{equation}
	\left(\mathbf{A} - \lambda\mathbf{I}_n\right)\mathbf{v} = \mathbf{0} .
\end{equation}

	Now consider

\begin{equation}
	e^{\left(\mathbf{A} - \lambda\mathbf{I}_n\right)}\mathbf{v} = \left[\sum_{k=0}^\infty \dfrac{1}{k!} \left(\mathbf{A} - \lambda\mathbf{I}_n\right)^k\right]\mathbf{v} =  \sum_{k=0}^\infty \dfrac{1}{k!} \left(\mathbf{A} - \lambda\mathbf{I}_n\right)^k\mathbf{v} = \mathbf{v}
\end{equation}

	\noindent because all powers of $\left(\mathbf{A} - \lambda\mathbf{I}_n\right)$ multiplied by $\mathbf{v}$ are the null vector, except for the 0-th power, which is the identity. Therefore

\begin{equation}
	\left[e^{\left(\mathbf{A} - \lambda\mathbf{I}_n\right)} - \mathbf{I}_n\right]\mathbf{v} = \mathbf{0}.
\end{equation}

	Multiply the equation on the left by $e^{\left(\lambda\mathbf{I}_n\right)}$, and remember that any matrix exponent is invertible as per theorem \ref{theo:matrix_exponential_inverse}. Also remember that $\mathbf{A}$ and $\mathbf{I}_n$ always commute, hence by theorem \ref{theo:matrix_exponential_sum} the multiplication of matrix exponentials is the exponential of matrix sum

\begin{equation}
	\left(e^{\mathbf{A}} - e^{\lambda\mathbf{I}_n}\right)\mathbf{v} = \mathbf{0} .
\end{equation}

	Finally, by theorem \ref{theo:matrix_exponential_identity_scaling}, $e^{\lambda\mathbf{I}_n} = e^{\lambda}\mathbf{I}_n$ and 

\begin{equation}
	\left(e^{\mathbf{A}} - e^{\lambda}\mathbf{I}_n\right)\mathbf{v} = \mathbf{0} .
\end{equation}

	which concludes the proof.

\begin{lemma}\label{lemma:exponential_matrix_norm} %<<<
	Let $\lambda$ be the eigenvalues of $\mathbf{A}$. Then the norm of the exponential of $\mathbf{A}$ is of exponential order, that is,

\begin{equation} \left\lVert e^{\mathbf{A}t} \right\rVert \leq \sum_{\lambda\in\rho\left(\mathbf{A}\right)} \left[ \sum_{k=0}^{\mu_\mathbf{A}\left(\lambda\right)} O\left(t^k\right)\right] e^{\text{Re}\left(\lambda\right) t}. \end{equation}

\end{lemma}
\noindent\textbf{Proof: } let $\mathbf{J}$ be the Jordan Canonical Form of $\mathbf{A}\in\mathbb{C}^{(n\times n)}$ and consider the Jordan Blocks. First we prove that 

\begin{equation} \left\lVert \mathbf{A}\right\rVert \leq \sum_{i=1}^n \left\lVert \mathbf{J}_i\right\rVert . \label{eq:jordan_ji_A_ineq}\end{equation}

	We first note that for any invertible matrix $\mathbf{A}$,

\begin{equation}
\left.\begin{array}{l}
	\mathbf{I}_n = \mathbf{AA}^{-1} \Rightarrow \left\lVert \mathbf{I}_n\right\rVert =  \left\lVert \mathbf{AA}^{-1} \right\rVert \leq \left\lVert \mathbf{A}\right\rVert\left\lVert \mathbf{A}^{-1}\right\rVert \Rightarrow \left\lVert\mathbf{A}^{-1}\right\rVert \geq \dfrac{1}{\left\lVert\mathbf{A}\right\rVert} \\[7mm]
	\mathbf{I}_n = \mathbf{A}^{-1}\mathbf{A} \Rightarrow \left\lVert \mathbf{I}_n\right\rVert =  \left\lVert \mathbf{A}^{-1}\mathbf{A} \right\rVert \leq \left\lVert \mathbf{A}^{-1}\right\rVert\left\lVert \mathbf{A}\right\rVert \Rightarrow \left\lVert\mathbf{A}^{-1}\right\rVert \leq \dfrac{1}{\left\lVert\mathbf{A}\right\rVert}
\end{array}\right\}\ 
	\left\lVert\mathbf{A}^{-1}\right\rVert = \dfrac{1}{\left\lVert\mathbf{A}\right\rVert} \end{equation}

	Therefore, let $\mathbf{J}$ be the Jordan Canonical Form of $\mathbf{A}$. Then

\begin{equation} \left\lVert \mathbf{A}\right\rVert = \left\lVert \mathbf{GJG}^{-1}\right\rVert \leq \left\lVert \mathbf{G}\right\rVert\left\lVert \mathbf{J}\right\rVert\left\lVert \mathbf{G}^{-1}\right\rVert = \left\lVert \mathbf{J}\right\rVert . \label{eq:jordan_A_ineq}\end{equation}

	Now consider $\mathbf{J}_k$ the k-th Jordan block, and pick the vector $\mathbf{x} = \left[\mathbf{x}_1,\mathbf{x}_2,...,\mathbf{x}_m\right]$ where $\mathbf{x}_k$ has the same size as the k-th Jordan block and has unitary size. Then

\begin{align}
	\left\lvert \mathbf{Jx}\right\rvert &=
	\left\lvert \left[\begin{array}{ccccc} \mathbf{J}_1 & 0 & 0 & \cdots & 0 \\[3mm] 0 & \mathbf{J}_2 & 0 & \cdots & 0 \\[3mm] 0 & 0 & \mathbf{J}_3 & \cdots & 0 \\[3mm] \vdots & \vdots & \vdots & \ddots & \vdots \\[3mm] & 0 & 0 & 0 \cdots & \mathbf{J}_m \end{array}\right] \left[\begin{array}{c} \mathbf{x}_1 \\[3mm] \mathbf{x}_2 \\[3mm] \mathbf{x}_3 \\[3mm] \vdots \\[3mm] \mathbf{x}_m \end{array}\right] \right\rvert =
%
	\left\lvert \left[\begin{array}{c} \mathbf{J}_1\mathbf{x}_1 \\[3mm] \mathbf{J}_2\mathbf{x}_2 \\[3mm] \mathbf{J}_3\mathbf{x}_3 \\[3mm] \vdots \\[3mm] \mathbf{J}_m\mathbf{x}_m \end{array}\right] \right\rvert =  \nonumber\\[3mm]
%
	&= \left\lvert \left[\begin{array}{c} \mathbf{J}_1\mathbf{x}_1 \\[3mm] \mathbf{0} \\[3mm] \mathbf{0} \\[3mm] \vdots \\[3mm] \mathbf{0} \end{array}\right] \right\rvert + \left\lvert \left[\begin{array}{c} \mathbf{0} \\[3mm] \mathbf{J}_2\mathbf{x}_2 \\[3mm] \mathbf{0} \\[3mm] \vdots \\[3mm] \mathbf{0} \end{array}\right] \right\rvert + \left\lvert \left[\begin{array}{c} \mathbf{0} \\[3mm] \mathbf{0} \\[3mm] \mathbf{J}_3\mathbf{x}_3 \\[3mm] \vdots \\[3mm] \mathbf{0} \end{array}\right] \right\rvert + \cdots + \left\lvert \left[\begin{array}{c} \mathbf{0} \\[3mm] \mathbf{0} \\[3mm] \mathbf{0} \\[3mm] \vdots \\[3mm] \mathbf{J}_m\mathbf{x}_m \end{array}\right] \right\rvert \leq \sum_{i=1}^m \left\lvert \mathbf{J}_i\mathbf{x}_i\right\rvert.
\end{align}

	But clearly $\left\lvert \mathbf{x} \right\rvert \geq \left\lvert \mathbf{x}_i \right\rvert$, so dividing the inequality by $\left\lvert\mathbf{x}\right\rvert$ yields

\begin{equation}
	\dfrac{\left\lvert \mathbf{Jx}\right\rvert}{\left\lvert \mathbf{x}\right\rvert} \leq \sum_{i=1}^m \dfrac{\left\lvert \mathbf{J}_i\mathbf{x}_i\right\rvert}{\left\lvert \mathbf{x}\right\rvert} \leq \sum_{i=1}^m  \dfrac{\left\lvert\mathbf{J}_i\mathbf{x}_i\right\rvert}{\left\lvert \mathbf{x}_i\right\rvert} \leq  \sum_{i=1}^m \left\lVert\mathbf{J}_i\right\rVert . \label{eq:jordan_ineq}
\end{equation}

	This already implies

\begin{equation} \left\lVert \mathbf{J}\right\rVert \leq  \sum_{i=1}^m \left\lVert\mathbf{J}_i\right\rVert , \end{equation}

	\noindent because \eqref{eq:jordan_ineq} holds for any $\mathbf{x}$; with \eqref{eq:jordan_A_ineq} this implies \eqref{eq:jordan_ji_A_ineq}. Therefore

\begin{equation} \left\lVert e^\mathbf{A}\right\rVert \leq e^{\left\lVert\mathbf{A}\right\rVert}, \end{equation}

	\noindent and because the real exponential function is strictly increasing,

\begin{equation} e^{\left\lVert\mathbf{A}\right\rVert} \leq  e^{\sum_{i=1}^m \left\lVert\mathbf{J}_i\right\rVert} = \prod_{i=1}^m e^{\left\lVert\mathbf{J}_i\right\rVert} \end{equation}

	Now we find an upper limit for the norm of each Jordan Block. Pick a particular block $\mathbf{J}_k$ of size $m$. Then by corollary \ref{corollary:jordan_nk}, $\mathbf{J}_k = \lambda \mathbf{I}_k + \mathbf{N}_k$ where $\mathbf{N}_k$ is nilpotent of order $m$. Therefore

\begin{equation} e^{\mathbf{J}_kt} = e^{\left(\lambda_k\mathbf{I}_m + \mathbf{N}_k\right)t} = e^{\lambda_k\mathbf{I}_m t}e^{\mathbf{N}_k t} = e^{\lambda_k t} \mathbf{I}_m \left[\sum_{i=0}^\infty \dfrac{1}{i!} \mathbf{N}_k^i t^i\right] .\end{equation} 

	But because $\mathbf{N}_k$ is m-th degree nilpotent, all powers above $m$ vanish:

\begin{equation} e^{\mathbf{J}_kt} = e^{\lambda_k t} \mathbf{I}_m \left[\sum_{i=0}^m \dfrac{1}{i!} \mathbf{N}_k^i t^i\right] .\end{equation} 

	Now taking the norm

\begin{equation}\left\lVert e^{\mathbf{J}_kt}\right\rVert \leq \left\lvert e^{\lambda_k t} \right\rvert \sum_{i=0}^m \dfrac{1}{i!} \left\lVert \mathbf{N}_k \right\rVert^i t^i,\end{equation} 

	\noindent and note that

\begin{equation} \left\lvert e^{\lambda_k t} \right\rvert  = \left\lvert e^{\left[\text{Re}\left(\lambda_k\right) + j\text{Im}\left(\lambda_k\right) \right]  t} \right\rvert = \left\lvert e^{\text{Re}\left(\lambda_k\right) t}\right\rvert \cancelto{1}{\left\lvert e^{j\text{Im}\left(\lambda_k\right)  t} \right\rvert} = \left\lvert e^{\text{Re}\left(\lambda_k\right) t}\right\rvert = e^{\text{Re}\left(\lambda_k\right) t}. \end{equation}

	Meaning

\begin{equation}\left\lVert e^{\mathbf{J}_kt}\right\rVert \leq e^{\text{Re}\left(\lambda_k\right) t} \sum_{i=0}^m \dfrac{1}{i!} \left\lVert \mathbf{N}_k \right\rVert^i t^i,\end{equation} 

	\noindent so that

\begin{equation} \left\lVert e^{\mathbf{A}t} \right\rVert \leq \sum_{\lambda\in\rho\left(\mathbf{A}\right)} \left[ \sum_{k=1}^{\mu_\mathbf{A}\left(\lambda\right)} O\left(t^k\right)\right] e^{\text{Re}\left(\lambda\right) t}. \end{equation}
\hfill$\blacksquare$
\vspace{5mm}
\hrule
\vspace{5mm}
%>>>

\begin{theorem}[Exponential characteristic of Linear ODEs] \label{def:exp_charac} %<<<
	In a complex linear system $\dot{\mathbf{x}} = \mathbf{Ax}$,

\begin{equation} \left\lVert \varphi\left(t,\mathbf{x}_0\right) \right\rVert \leq \sum_{\lambda\in\rho\left(\mathbf{A}\right)} \left[ \sum_{k=0}^{\mu_\mathbf{A}\left(\lambda\right)} O\left(t^k\right)\right] e^{\text{Re}\left(\lambda\right) t}, \label{theo:hurwitz_stable_exponential_stable_almost_exp} \end{equation}

\noindent for any initial condition $\mathbf{x}_0$.
\end{theorem}
\noindent\textbf{Proof:} by theorem \ref{theo:homogeneous_solutions_ltiode_matrix},

\begin{equation} \varphi\left(t,\mathbf{x}_0\right) = e^{\mathbf{A}t}\mathbf{x}_0 \end{equation}

	\noindent meaning that

\begin{equation} \left\lvert\varphi\left(t,\mathbf{x}_0\right)\right\rvert = \left\lvert e^{\mathbf{A}t}\mathbf{x}_0\right\rvert \leq \left\lVert e^{\mathbf{A}t}\right\rVert\left\lvert\mathbf{x}_0\right\rvert .\end{equation}

	Now use lemma \ref{lemma:exponential_matrix_norm}:

\begin{equation} \left\lvert\varphi\left(t,\mathbf{x}_0\right)\right\rvert \leq \left[\sum_{\lambda\in\rho\left(\mathbf{A}\right)} \left[ \sum_{k=0}^{\mu_\mathbf{A}\left(\lambda\right)} O\left(t^k\right)\right] e^{\text{Re}\left(\lambda\right) t}\right]\left\lvert\mathbf{x}_0\right\rvert \label{eq:exponential_char_theorem_final}\end{equation}

	\noindent and because $\left\lvert\mathbf{x}_0\right\rvert$ is constant, the proof is complete.
\hfill$\blacksquare$
\vspace{5mm}
\hrule
\vspace{5mm}
%>>>

	It follows from this theorem that if all the eigenvalues $\lambda$ are in the left open half plane, then the linear system $\dot{\mathbf{x}} = \mathbf{Ax}$ is stable. It is left to show that this is an unambiguous condition, that is, if the system is stable then all eigenvalues have negative real part. This is simple to prove: suppose that a particular eigenvalue $\lambda_e$ has either null or positive real part. Then, by theorem \ref{theo:homogeneous_solutions_ltiode}, $\varphi\left(t,\mathbf{x}_0\right)$ is a linear combination

	\begin{equation} \varphi\left(t,\mathbf{x}_0\right) = \sum_{\lambda_k\in\rho\left(\mathbf{A}\right)} c_k\mathbf{x}_k,\ \mathbf{x}_k(t) =  \left[\displaystyle\sum\limits_{i=0}^{\mu_\mathbf{A}\left(\lambda_k\right)-1} \dfrac{t^i}{i!}\left(\mathbf{A} - \lambda_k\mathbf{I}\right)^{i}\right]\mathbf{v}_k e^{\lambda_k t}, \end{equation}

	\noindent where the combination coefficients $c_k$ are calculated through the initial condition. Let $x_e$ be the solution pertaining to the eigenvalue $\lambda_e$; then removing $\mathbf{x}_e$ to the other side,

\begin{equation} \varphi\left(t,\mathbf{x}_0\right) - c_e\mathbf{x}_e = \sum_{\substack{\lambda_k\in\rho\left(\mathbf{A}\right) \\ \lambda_k \neq \lambda_e} } c_k\mathbf{x}_k, \end{equation}

	\noindent and taking the triangular inequality,

\begin{equation} \left\lvert\varphi\left(t,\mathbf{x}_0\right) - c_e\mathbf{x}_e\right\rvert \leq \sum_{\substack{\lambda_k\in\rho\left(\mathbf{A}\right) \\ \lambda_k \neq \lambda_e} } \left\lvert c_k\mathbf{x}_k \right\rvert. \end{equation}

	Theorem \ref{def:exp_charac} shows that, because all other $\mathbf{x}_k$ pertain to stable eigenvalues, then this difference falls exponentially, that is, $\varphi\left(t,\mathbf{x}_0\right)$ approaches $c_e\mathbf{x}_e$. If the initial condition $\mathbf{x}_0$ is arbitrary, that is, $c_e$ cannot be guaranteed to be zero, then if $\lambda_e$ is null and simple, $\varphi\left(t,\mathbf{x}_0\right)$ tends to a constant norm in time; if $\lambda_e$ is null but not simple, $\varphi\left(t,\mathbf{x}_0\right)$ explodes in a polynomial fashion. Finally, if $\lambda_e$ has positive real part, $\varphi\left(t,\mathbf{x}_0\right)$ explodes exponentially. Therefore, the only way for an orbit $\varphi\left(t,\mathbf{x}_0\right)$ with arbitrary initial conditions to stabilize to the origin is if all eigenvalues are stable, that is, have negative real part. Particularly, if the initial conditions of the system as chosen specifically so that the coefficients $c_k$ of unstable or oscillatory eigenvalues are zero, then the system is also assymptotically stable.

	Therefore, the position of the eigenvalues of $\mathbf{A}$ is of high importance when stability is concerned. Because of this, a linear system is said to be \textit{Hurwitz Stable}, in honor of the mathematician Adolf Hurwitz and related to the Hurwitz Polynomial of section \ref{sec:line_matrix_ode}, if the eigenvalues of $\mathbf{A}$ are all in the open half left semiplane — the conclusion of corollary \ref{corollary:hurwitz_lti_odes}. By force of a metonym, in this case these eigenvalues are said to be \textit{stable eigenvalues} and $\mathbf{A}$ is said to be Hurwitz.

\begin{corollary}[Hurwitz stability of Linear ODEs] \label{corollary:hurwitz_lti_odes} %<<<
	A linear system $\dot{\mathbf{x}} = \mathbf{Ax}$ is stable if and only if it is Hurwitz stable, or equivalently, if $\mathbf{A}$ is Hurwitz, which is to say $\mathbf{A}$ has only stable eigenvalues.
\end{corollary}
%>>>

	Qualitatively, the main consequence of corollary \ref{corollary:hurwitz_lti_odes} is that, in some sense, linear stability and Hurwitz stability are one and the same, which a direct consequence of the fact that linear systems are inherently ``exponential'', as stated in theorem \ref{def:exp_charac}.

	A big consequence of this fact is that if a linear system is stable, then surely it is \textit{exponentially stable}, that is, $\mathbf{x}$ tends to $\mathbf{0}$ under an exponential curve, which is a very strong form of stability.

\begin{corollary}[Exponential stability of linear systems] \label{corollary:exp_stability} %<<<
	If a linear system $\dot{\mathbf{x}} = \mathbf{Ax}$ is stable then it is \textbf{globally exponentially stable}, that is, for any initial point $\mathbf{x}_0$ there exist two positive numbers $K$ and $\alpha$ such that

\begin{equation} \left\lvert \varphi\left(t,\mathbf{x}_0\right)\right\rvert \leq Ke^{-\alpha t} \label{eq:theo_exp_stab_result}\end{equation}
\end{corollary}
\noindent\textbf{Proof:} following the resulting equation \eqref{eq:exponential_char_theorem_final} of theorem \ref{def:exp_charac}, all that is left to prove that, supposing all eigenvalues are stable (have negative real part) then \eqref{eq:exponential_char_theorem_final} implies \eqref{eq:theo_exp_stab_result}. The proof is then just to show that for any (finite) polynomial $P(x)$ and any $\beta > 0$, there exist $\alpha,K > 0$ such that

\begin{equation} P(x)e^{-\beta x} \leq Ke^{-\alpha x}. \end{equation}

	If the polynomial is constant this is immediate. Then suppose it has degree $m \geq 1$. If we prove the proposition for a singleton $P(x) = x^m$, then

\begin{equation} \left\lvert \left(\sum_{k=0}^m a_mx^m\right)e^{-\beta x}\right\rvert \leq \sum_{k=0}^m \left\lvert a_m \right\rvert\left\lvert x^me^{-\beta x} \right\rvert \leq \sum_{k=0}^m \left\lvert a_m \right\rvert e^{-\alpha x} = \left(\sum_{k=0}^m \left\lvert a_m \right\rvert\right) e^{-\alpha x}. \end{equation}

	\noindent and the proposition is proven for $P(x)$. Starting with the definition of exponential,

\begin{equation} e^{\beta x} = \sum_{k=0}^\infty \dfrac{1}{k!} \left(\beta x\right)^k > \sum_{k=m}^\infty \dfrac{1}{k!} \left(\beta x\right)^k \end{equation}

	\noindent and divide by some $x^m$:

\begin{equation} \dfrac{e^{\beta x}}{x^m} > \beta^m \sum_{k=m}^\infty \dfrac{1}{k!} \left(\beta x\right)^{k-m} = \beta^m \sum_{k=0}^\infty \dfrac{1}{(k+m)!} \left(\beta x\right)^{k}\end{equation}

	\noindent and inverting the inequality yields

\begin{equation} x^m e^{-\beta x} < \dfrac{1}{\raisebox{1mm}[3mm][0mm]{} \beta^m \displaystyle\sum_{k=0}^\infty \dfrac{1}{(k+m)!} \left(\beta x\right)^{k} } = \dfrac{1}{\beta^m} \left[\dfrac{1}{\raisebox{1mm}[3mm][0mm]{} \displaystyle\sum_{k=0}^\infty \dfrac{1}{(k+m)!} \left(\beta x\right)^{k} }\right]. \end{equation}

	Noticeably, the denominator is always positive and increases with $x$; therefore it is always smaller than the quantity at $x = 0$ and

\begin{equation} x^m e^{-\beta x} < \dfrac{1}{\beta^m} \dfrac{1}{\dfrac{1}{m!}} = \dfrac{m!}{\beta^m}. \label{eq:boudned_exp_poly}\end{equation}

	\noindent which proves that $x^me^{-\beta t}$ is bounded. But 

\begin{equation} (k+m)! = k! (k+1)(k+2)...(k+m-1)(k+m) < k! (k+m)^m, \end{equation}

	\noindent so 

\begin{equation} x^m e^{-\beta x} < \dfrac{1}{\beta^m} \left[\dfrac{1}{\raisebox{1mm}[3mm][0mm]{} \displaystyle\sum_{k=0}^\infty \dfrac{1}{k! \left(k+m\right)^m} \left(\beta x\right)^{k} }\right] = \dfrac{1}{\beta^m} \left[\dfrac{1}{\raisebox{1mm}[3mm][0mm]{} \displaystyle\sum_{k=0}^\infty \dfrac{1}{k!}\left(\dfrac{\beta^k}{ \left(k+m\right)^m} \right) x^{k} }\right]. \end{equation}

	\noindent and now we want to prove that an $\alpha > 0$ exists satisfying

\begin{equation} \alpha^k \leq \dfrac{\beta^k m^m }{\left(k+m\right)^m},\ \forall k\geq 0 \end{equation}

	\noindent because this would mean 

\begin{equation} x^m e^{-\beta x} < \dfrac{1}{\beta^m} \left[\dfrac{1}{\raisebox{1mm}[3mm][0mm]{} \displaystyle\sum_{k=0}^\infty \dfrac{1}{k!} m^m \alpha^k x^k }\right] = \dfrac{1}{\left(\beta m\right)^m}e^{-\alpha x}, \end{equation}

	\noindent and this would conclude the proof. Consider the function on $k$ 

\begin{equation} f(k) = \left(\dfrac{\alpha}{\beta}\right)^k \dfrac{\left(k + m\right)^m}{m^m} \end{equation}

	\noindent and we want to show that there exists some $\alpha$ such that $f(k)\leq 1$ for all $k\geq 0$. Manipulating $f(k)$,

\begin{equation} f(k) = \left(\dfrac{\alpha}{\beta}\right)^k \left(1 + \dfrac{k}{m}\right)^m . \label{theo:exp_limit_f_k_seq}\end{equation}

	Now fix $m$ and take the sequence

\begin{equation} \left[\left(1 + \dfrac{k}{m}\right)^m\right]_{k=0}^\infty \end{equation}

	\noindent and note that this sequence is strictly increasing and that at the limit

\begin{equation} \lim_{m\to\infty} \left(1 + \dfrac{k}{m}\right)^m = e^k \end{equation}
	
	\noindent meaning \eqref{theo:exp_limit_f_k_seq} implies

\begin{equation} f(k) < \left(\dfrac{\alpha}{\beta}\right)^k e^k .\end{equation}

	Therefore, choose $\alpha$ such that

\begin{equation} \dfrac{\alpha}{\beta} < \dfrac{1}{e} \Leftrightarrow  \alpha < \dfrac{\beta}{e} \end{equation}

	\noindent then $f(z) \leq 1$ for all $k\geq 0$.
\hfill$\blacksquare$
\vspace{5mm}
\hrule
\vspace{5mm}
%>>>

	Intuitively, exponential stability means that the system goes to its equilibrium very fast, and also that disturbances are almost always not reflective on stability because the exponential function dominates most of the practical signals.
	
	When considering a forced system, one has to naturally include the particular or non-forced component of the solution. However, because the general solution of the homogeneous part is independent from the forcing, it will be kept the same, that is, vanishing in time. This, in turn, means that the general solution approaches the particular solution exponentially. which is a very strong form of approaching. Finally, this is to say that the particular solution $\mathbf{x}_p$ is a \textbf{exponentially stable steady-state solution} of the ODE.

\begin{theorem}[Stable steady-state orbits of forced LTI ODEs]\label{theorem:stable_sol_lti_systems} %<<<

	Consider the LTI ODE 

\begin{equation} \dot{\mathbf{x}} = \mathbf{Ax} + \mathbf{f}(t) \label{corollary:stable_sol_lti_systems_original_ltiode} \end{equation}

	\noindent where $\mathbf{A}\in\mathbb{C}^{(n\times n)}$ is Hurwitz stable. Let $\varphi\left(t,\mathbf{x}_0\right)$ be an orbit of the system with an arbitrary initial position $\mathbf{x}_0$ at $t_0$, and suppose a particular orbit $\mathbf{x}_p = \varphi\left(t,\mathbf{x}_0^p\right)$ is known. Then 

\begin{equation} \left\lvert \varphi\left(t,\mathbf{x}_0\right) - \varphi\left(t,\mathbf{x}_0^p\right) \right\rvert \leq Ke^{-\alpha t} \end{equation}

	\noindent for any $\mathbf{x}_0$. 
\end{theorem}
\noindent\textbf{Proof:} a revisitation of theorem \ref{theo:homogeneous_particular_ltiode}. Consider a known orbit, called the ``particular trajectory'' or solution, $\varphi\left(t,\mathbf{x}_0^p\right)$, and consider a trajectory for an arbitrary initial condition $\varphi\left(t,\mathbf{x}_0\right)$. Then both solutions satisfy the system differential equation, that is,

\begin{equation} \dfrac{d}{dt}\varphi\left(t,\mathbf{x}_0\right) = \mathbf{A} \varphi\left(t,\mathbf{x}_0\right) + \mathbf{f}(t),\ \dfrac{d}{dt}\varphi\left(t,\mathbf{x}_0^p\right) = \mathbf{A} \varphi\left(t,\mathbf{x}_0^p\right) + \mathbf{f}(t). \end{equation}

	The difference of these equations yields

\begin{equation} \dfrac{d}{dt} \left[ \varphi\left(t,\mathbf{x}_0\right) - \varphi\left(t,\mathbf{x}_0^p\right)\right] = \mathbf{A}\left[ \varphi\left(t,\mathbf{x}_0\right) - \varphi\left(t,\mathbf{x}_0^p\right)\right]. \end{equation}

	Now let $\varepsilon\left(t,\mathbf{x}_0 - \mathbf{x}_0^p\right)  = \varphi\left(t,\mathbf{x}_0\right) - \varphi\left(t,\mathbf{x}_0^p\right)$; then this equation is equivalent to

\begin{equation} \dfrac{d}{dt} \left[ \varepsilon\left(t,\mathbf{x}_0 - \mathbf{x}_0^p\right)\right] = \mathbf{A}\varepsilon\left(t,\mathbf{x}_0 - \mathbf{x}_0^p\right), \end{equation}

	\noindent that is, $\varepsilon$ is the orbit of the nonforced system with initial condition $\mathbf{x}_0 - \mathbf{x}_0^p$. Because $\mathbf{A}$ is Hurwitz, $\varepsilon$ tends to the origin exponentially for any initial condition; because $\mathbf{x}_0^p$ is fixed, this means for an arbitrary $\mathbf{x}_0$, yielding the result.
\hfill$\blacksquare$
\vspace{5mm}
\hrule
\vspace{5mm} %>>>

	Theorem \ref{theorem:stable_sol_lti_systems} shows a very strong result that if a particular orbit is known, then all orbits of the forced linear system will tend to it in infinite time, that is, the particular solution is a \textbf{stable steady state orbit}. Reestated, \textbf{for any initial condition}, the system will converge towards the particular solution. In essence, this is because the differences in orbits are essentially the initial conditions and, after enough time, the system vanishes the differences. In some sense, the initial conditions of the system describe a certain state or ``energy'' from which the system departs, and that this energy is inevitably spent as the system progresses in time.

	In short, this theorem will be of great use because it allows us to, in some way, disconsider the initial conditions if we know a particular orbit if the system. As it will be shown, this has a great many benefits to applied sciences because it allows us to benefit from the fact that the general solution to the system does not need be known if some particular solution is within grasp, supposing that we are willing to discard transient disturbances.

%-------------------------------------------------
\section{Lyapunov stability} %<<<1

	Lyapunov stability is a strong method of proving the stability of Dynamical Systems. The core idea, which is actually Lyapunov's second method of stability, is that if a function of the states of the system, called ``energy function'' can be found, the properties of this function can be explored instead of directly solving the system differential equations, i.e., obtaining the orbits. An energy function is a function $V\left(\mathbf{x}\right)$ of the states that is positive definite ($V > 0$ at any point but the origin and zero at the origin) and which time derivative is semidefinite negative that ($\dot{V} \leq 0$). Lyapunov proves that if such a function can be found, then the system is assymptotically stable.

	Finding Lyapunov functions however is schockingly not trivial. No such function exists for an arbitrary nonlinear system, and finding energy functions that fit specific classes of systems is the goal of a vast body of literature. Startingly, in some cases it can be proven certain systems cannot admit such a function: case in point, particularly for Electrical Power Systems, it is widely known that while there exists a single continuous energy function for EPSs with lossless transmission \pcite{narasimhamurthiExistenceEnergyFunction1984}, an analogous one for lossy systems does not exist \pcite{chiangDirectMethodsStability2011}, and there is no smooth transformation between the energy function of a lossless system into that for a lossy one, that is, there is no way to ``adapt'' the models of lossless systems to obtain models for lossy ones. Further, it was also shown in \cite{albertoDevelopmentGeneralizedEnergy2012} that there is no energy function for a general model of EPS, although a less rectrictive notion of \textit{generalized energy function} does exist.

	Luckily, for linear systems, the definitions become simpler due to the linearity property; theorem \ref{theo:lyapunov_linear} shows that in linear systems assymptotic stability and Lyapunov stability are equivalent.

\begin{definition}[Definite matrices] A hermitian square matrix $\mathbf{M}\in\mathbb{C}^{(n\times n)}$ is said to be:
\begin{itemize}
	\item \textbf{Positive definite}, denoted $\mathbf{M} > 0$, if $\mathbf{x}^\hermconj\mathbf{Mx} > 0$ for any $\mathbf{x}\in\mathbb{C}^n$ but the null vector and $\mathbf{x}^\hermconj\mathbf{Mx} = 0$ only if $\mathbf{x=0}$;
	\item \textbf{Positive semi-definite}, denoted $\mathbf{M} \geq 0$, if  $\mathbf{x}^\hermconj\mathbf{Mx} \geq 0$, that is, it can be zero at other vectors;
	\item \textbf{Negative definite}, denoted $\mathbf{M} < 0$, if $\mathbf{x}^\hermconj\mathbf{Mx} < 0$ for any $\mathbf{x}\in\mathbb{C}^n$ but the null vector and $\mathbf{x}^\hermconj\mathbf{Mx} = 0$ only if $\mathbf{x=0}$;
	\item \textbf{Negative semi-definite}, denoted $\mathbf{M} \leq 0$, if  $\mathbf{x}^\hermconj\mathbf{Mx} \leq 0$, that is, it can be zero at other vectors.
\end{itemize}
\end{definition}

\begin{definition}[Lyapunov Stability for Linear Systems]\label{def:lyapunov_linear} %<<<
	Let $\dot{\mathbf{x}} = \mathbf{Ax},\ \mathbf{A}\in\mathbb{C}^{(n\times n)},\ \mathbf{x}\in\left[\mathbb{R}\to\mathbb{C}^n\right]$ a complex linear system. Then the system is \textit{Lyapunov Stable} if there exist a positive definite matrix $\mathbf{M}\in\mathbb{C}^{(n\times n)}$ that defines a function called ``energy function''

\begin{equation} V:\left\{\begin{array}{rcl} \mathbb{C}^n &\to& \mathbb{R}^+ \\[3mm] \mathbf{x} &\mapsto& \mathbf{x}^\hermconj\mathbf{Mx} \end{array}\right.\end{equation}

	\noindent such that $\dot{V}\left(\mathbf{x}\right) = \mathbf{x}^\hermconj\mathbf{Qx}$, where $\mathbf{Q}\in\mathbb{C}^{(n\times n)}$ is a negative definite matrix.
\end{definition}%>>>
\begin{theorem}[Lyapunov Stability Theorem for Linear Systems]\label{theo:lyapunov_linear} %<<<
	If a system $\dot{\mathbf{x}} = \mathbf{Ax},\ \mathbf{A}\in\mathbb{C}^{(n\times n)},\ \mathbf{x}\in\left[\mathbb{R}\to\mathbb{C}^n\right]$ is Lyapunov Stable, then it is assymptotically stable.
\end{theorem}
\noindent\textbf{Proof:} by definition there exist two matrices $\mathbf{M} > 0,\ \mathbf{Q} < 0$ such that $V\left(\mathbf{x}\right) = \mathbf{x}^\hermconj\mathbf{Mx}$ and $\dot{V}\left(\mathbf{x}\right) = \mathbf{x}^\hermconj\mathbf{Qx}$ . Then $V\left(\mathbf{x}\right) = 0$ at $\mathbf{x = 0}$ and only at that point. Then consider the set of vectors

\begin{equation} U\left(\varepsilon\right) = \left\{\mathbf{x}\in\mathbb{C}^n:\ V\left(\mathbf{x}\right) < \varepsilon\right\} .\end{equation}

	The existence of $U\left(\varepsilon\right)$ is guaranteed because $\left[0,\varepsilon\right]$ is closed and compact, and $V$ is continuous. Pick a starting point $\mathbf{x}_0$ in $U\left(\varepsilon\right)$ and we conclude that the trajectory that follows can only stay inside $U\left(\varepsilon\right)$; if we suppose that after some time $T$ the trajectory escapes this set, that is, at some point, $V\left(\mathbf{x}\left(T\right)\right) = \varepsilon$, then this contradicts the fact that $\dot{V}\left(\mathbf{x}\right)$ is negative semidefinite. Therefore, the trajectory $\mathbf{x}(t)$ stays at least in $U\left(\varepsilon\right)$. Because $\mathbf{Q}$ is supposed negative semi definite, it may be that the trajectory does not tend to $\mathbf{0}$; it can, for instance, stay at some constant value $V$ in a particular equilibrium or closed periodic orbit, as long as this orbit fulfills $\dot{V} = 0$. This proves that if $\mathbf{Q}$ is negative semi-definite, any trajectory starting in $U\left(\varepsilon\right)$ stays in that set, with no particular assymptote; $V\left(\mathbf{x}\right)$ can only remain constant or be reduced, but never grow.

	If additionally $\mathbf{Q}$ is strictly negative definite, meaning $\dot{V} < 0$ for any point that is not the null point, then $\mathbf{x}$ inevitably ``falls down'' to zero. Suppose that the system reaches some set $V\left(\mathbf{x}\right) = \varepsilon' < \varepsilon$, which may be a particular equilibrium $\mathbf{x}^*$ or a closed orbit. By definition, in this set, $\dot{V} = 0$, contradicting the negative definiteness of $\mathbf{Q}$. But since $\dot{V}\left(\mathbf{x}^*\right) < 0$ by definition, then the trajectory cannot remain at constant positive $V$; in some sense, because ``energy must be spent'' the system is forced to zero energy, that is, it is forced into the set $V\left(\mathbf{x}\right) = 0$, which is only the null vector.
\hfill$\blacksquare$
\vspace{5mm}
\hrule
\vspace{5mm} %>>>

	It is simple to notice that if an energy function can be found, which is equivalent to say that if two matrices $\mathbf{M} > 0$ and $\mathbf{Q} < 0$ can be found, then the system is assymptotically stable, that is, Lyapunov Stability implies assymptotic stability. The gist of theorem \ref{theo:lyapunov_linear} is that if a positively defined function $V\left(\mathbf{x}\right)$, called \textit{Lyapunov Function}, can be found such that its derivative is negative, then the system has no choice but wane to the equilibrium point that is the origin. The idea behind such function is to generalize the physical idea of energy; for ``common'' systems, like electrical circuits and mechanical pendular or mass-spring systems, the energy functions are (generally) Lyapunov Functions of the systems they represent.

\begin{theorem}[Lyapunov Equation for linear systems] \label{theo:lyapunov_linear} %<<<
	Let $\dot{\mathbf{x}} = \mathbf{Ax}$ with $\mathbf{A}\in\mathbb{C}^{(n\times n)}$ be a linear system. Then this system is Lyapunov stable if and only if it is Hurwitz stable, that is, if $\mathbf{A}$ is Hurwitz.
\end{theorem}
\textbf{Proof:} consider the function $V\left(\mathbf{x}\right) = \mathbf{x}^\hermconj\mathbf{M}\mathbf{x}$; if $\mathbf{M}$ is positive definite, then $V$ is only zero at the origin. Then calculate $\dot{V}$:

\begin{equation} \dfrac{dV}{dt} = \dfrac{d}{dt} \left(\mathbf{x}^\hermconj\mathbf{M}\mathbf{x}\right) = \mathbf{x}^\hermconj \mathbf{M} \dot{\mathbf{x}} + \dot{\mathbf{x}}^\hermconj\mathbf{M}^\hermconj \mathbf{x} =\mathbf{x}^\hermconj \mathbf{M} \dot{\mathbf{x}} + \dot{\mathbf{x}}^\hermconj\mathbf{M} \mathbf{x} \end{equation}

	\noindent (this identity follows from matrix calculus). Using the definition of the linear system,

\begin{equation} \dfrac{dV}{dt} = \mathbf{x}^\hermconj \mathbf{M} \mathbf{Ax} + \mathbf{x}^\hermconj\mathbf{A}^\hermconj \mathbf{M}\mathbf{x}  = \mathbf{x}^\hermconj\left(\mathbf{M} \mathbf{A} + \mathbf{A}^\hermconj\mathbf{M}\right)\mathbf{x} .\end{equation} 

	Therefore, if $\mathbf{Q} = \mathbf{M} \mathbf{A} + \mathbf{A}^\hermconj\mathbf{M}$ is negative definite, then $\dot{V} < 0$. Therefore, if for some given negative definite $\mathbf{Q}$, the matrix $\mathbf{M}$ that is the solution to the Lyapunov Matrix Equation
	
\begin{equation}  -\mathbf{Q} + \mathbf{M} \mathbf{A} + \mathbf{A}^\hermconj\mathbf{M} = \mathbf{0} \label{eq:lyapunov_matrix_equation}\end{equation}

	\noindent is positive definite. The proof now aims to show that if $\mathbf{A}$ is Hurwitz, a solution to this equation can be found (that is, Hurwitz Stability implies Lyapunov Stability) and that, if a solution can be found, then the system is Hurwitz (Lyapunov Stability implies Hurwitz stability). This latter implication is a direct consequence of corollary \ref{corollary:hurwitz_lti_odes}; if the system is Lyapunov Stable it is stable nonetheless, and since stability of linear system is Hurwitz, then the system is Hurwitz stable.

	Now we need to prove that if the system is Hurwitz stable then for at least one negative definite $\mathbf{Q}$ there exists a positive definite $\mathbf{M}$ that satisfies \eqref{eq:lyapunov_matrix_equation}. Choose a negative definite $\mathbf{Q}$ and consider the function

\begin{equation} \mathbf{F}\left(\mathbf{Q},t\right) = -e^{\mathbf{A}^\hermconj t}\mathbf{Q}e^{\mathbf{A} t} \label{eq:def_f_func}\end{equation}

	\noindent which is notably hermitian, that is, $\mathbf{F} = \mathbf{F}^\hermconj$. Then

\begin{equation} -\dfrac{d\mathbf{F}\left(\mathbf{Q},t\right)}{dt} = \mathbf{A}^\hermconj e^{\mathbf{A}^\hermconj t}\mathbf{Q}e^{\mathbf{A} t} + e^{\mathbf{A}^\hermconj t}\mathbf{Q}e^{\mathbf{A} t}\mathbf{A} = \mathbf{F}\left(\mathbf{Q},t\right)\mathbf{A} + \mathbf{A}^\hermconj\mathbf{F}\left(\mathbf{Q},t\right) \end{equation}

	\noindent which is suspiciously close to the Lyapunov Equation \eqref{eq:lyapunov_matrix_equation}. Due to lemma \ref{lemma:exponential_matrix_norm}, because $\mathbf{A}$ is Hurwitz, both $e^{\mathbf{A}t}$ and its hermitian $e^{\mathbf{A}^\hermconj t}$ have a norm that is less than $Ke^{-\alpha t}$ for some positive $K,\alpha$ (here we assume the property that the eigenvalues of the hermitian conjugate of $\mathbf{A}$ are the complex conjugates of its eigenvalues, which is simple to prove). Then 

\begin{equation} \left\lVert \mathbf{F}\left(\mathbf{Q},t\right) \right\rVert \leq  \left\lVert e^{\mathbf{A} t} \right\rVert \left\lVert \mathbf{Q} \right\rVert \left\lVert e^{\mathbf{A}^\hermconj t} \right\rVert \leq K^2 e^{-2\alpha t}\left\lVert\mathbf{Q}\right\rVert \end{equation}

	This then implies

\begin{equation} \lim\limits_{t\to\infty} \mathbf{F}\left(\mathbf{Q},t\right) = \mathbf{0} \end{equation}


	\noindent meaning

\begin{equation} \int_0^\infty \dfrac{d\mathbf{F}\left(\mathbf{Q},s\right)}{ds} = -\left[e^{\mathbf{A}^\hermconj s}\mathbf{Q}e^{\mathbf{A} s}\right]_0^\infty = \mathbf{Q} .\end{equation}

	\noindent but this implies

\begin{equation} \mathbf{Q} = \int_0^\infty \left[ \mathbf{F}\left(\mathbf{Q},t\right)\mathbf{A} + \mathbf{A}^\hermconj\mathbf{F}\left(\mathbf{Q},t\right)\right]ds = \left[ \int_0^\infty \mathbf{F}\left(\mathbf{Q},s\right)ds\right]\mathbf{A} + \mathbf{A}^\hermconj \left[ \int_0^\infty \mathbf{F}\left(\mathbf{Q},s\right)ds\right] \end{equation}

	\noindent meaning 

\begin{equation} \mathbf{M} = -\int_0^\infty \mathbf{F}\left(\mathbf{Q},s\right)ds = -\int_0^\infty e^{\mathbf{A}^\hermconj s}\mathbf{Q}e^{\mathbf{A} s}ds \label{eq:lyapunov_matrix_equation_solution} \end{equation}

	\noindent is a unique solution to the Lyapunov Matrix Equation. Now all that is left is to prove this solution is positive definite. Take an arbitrary $\mathbf{u}$ and

\begin{equation} \mathbf{u}^\hermconj\mathbf{M}\mathbf{u} = \int_0^\infty -\mathbf{u}^\hermconj e^{\mathbf{A}^\hermconj s}\mathbf{Q}e^{\mathbf{A} s} \mathbf{u} ds = \int_0^\infty -\left(e^{\mathbf{A} s} \mathbf{u}\right)^\hermconj \mathbf{Q} \left(e^{\mathbf{A} s} \mathbf{u}\right) ds \end{equation}

	\noindent but since $\mathbf{Q}$ is chosen negative definite, the scalar integrand is always positive, therefore the integral is positive for any $\mathbf{u}$, yielding $\mathbf{M}$ positive definite.
\hfill$\blacksquare$
\vspace{5mm}
\hrule
\vspace{5mm} %>>>

\begin{corollary} \label{theo:lyapunov_linear_hurwitz_corollary_1} %<<<
	A matrix $\mathbf{A}\in\mathbb{C}^{(n\times n)}$ is Hurwitz stable if and only if for any negative defined $\mathbf{Q}$, there is a positive definite matrix $\mathbf{M}$ such that $\mathbf{Q} = \mathbf{M} \mathbf{A} + \mathbf{A}^\hermconj\mathbf{M}^\hermconj$.
\end{corollary} %>>>

\begin{corollary} \label{theo:lyapunov_linear_hurwitz_corollary_2} %<<<
	$\mathbf{A}$ is Hurwitz stable if and only if a Lyapunov Function $V\left(\mathbf{x}\right)$ can be found, that is, for linear systems, Hurwitz Stability is equivalent to Lyapunov Stability.
\end{corollary} %>>>

	In so far as corollaries \ref{theo:lyapunov_linear_hurwitz_corollary_1} and \ref{theo:lyapunov_linear_hurwitz_corollary_2} seem to be mere rewritings of the conclusions of theorem \ref{theo:lyapunov_linear}, they have a profound consequence in the theory of linear passive electrical circuit networks, because the matrix $\mathbf{Q}$ is not unique. For a chosen negative defined $\mathbf{Q}$, there is a uniquely defined $\mathbf{M}$ that solves the Lyapunov matrix equation \eqref{eq:lyapunov_matrix_equation}, in the form of \eqref{eq:lyapunov_matrix_equation_solution}. However, nothing is said about $\mathbf{Q}$ — the theorem imposes no restrictions on this matrix except for negative definiteness — meaning that it can be arbitrarily chosen. This is the express result of corollary \ref{theo:lyapunov_linear_hurwitz_corollary_1}. This, in turn, means that in a broader context if a Lyapunov energy function $V\left(\mathbf{x}\right)$ can be found for a passive electrical circuit modelled as $\dot{\mathbf{x}} = \mathbf{Ax}$ then $\mathbf{A}$ is Hurwitz stable. Nothing needs to be said about $V\left(\mathbf{x}\right)$ particularly; in fact, because the matrix $\mathbf{M}$ can be found for any negative definite $\mathbf{Q}$, $V$ is not unique. Therefore if a single energy function can be found, the linear system is Hurwitz stable.

	In a linear dynamical systems context, this means that the only assymptotic stability available to linear systems is the Hurwitz one. If, on the one hand, nonlinear systems can exhibit a plethora of different stability types, linear systems are restrained to exponential stability. Due to this fact, we can reduce terminology: a linear system can be said to be only ``stable'', and this implies it is stable in a wide reach of senses: assymptotically, exponentially, Hurwitz and Lyapunov. On the other hand, a linear system is ``unstable'' if it is not assymptotically, exponentially, Hurwitz nor Lyapunov stable.


%--------------------------------------------------------------------------------------------------
\chapter{Classic Phasors Theory} \label{chapter:classical_phasors}
%--------------------------------------------------------------------------------------------------

%-------------------------------------------------
\section{Introduction to phasors} %<<<1

	Classical Phasor Theory is proheminently based on the Classical Phasors Operator, which is a bijection that takes a sinusoid and represents it as a point in the complex plane. The paramount property of this operator, which prompted its inception by Steinmetz, is the fact that while adding sinusoids requires formul\ae s known as the Prostaph\ae resis Formul\ae s, adding two complex numbers is a simple matter of complex number operations which are geometrical.

 Despite these useful operational properties, the most useful result stemming from the transform $p_S$ is that it transforms time differential equations defined by linear circuits (such as the grids of Electrical Power Systems) into algebraic equations in the complex domain — a process called ``complexification''. This process allows for a much easier and simpler analysis of electric power grids because the phasor representation of sinusoidal waves allows for the development of the usual phasorial alternating current electrical analysis theory.

	The theoretical challenge is to prove that the complex phasorial quantities obtained from solving the algebraic complex equations are indeed representative of the time signals that solve the original time differential equations of the circuit. Reestated, as simple the operator definition might be, it still is as of now only that — a definition or a representation of some arbitrary operator with nice operational properties. The proof of this fact is absolutely not trivial; in short, it needs first to be proven that Passive Linear Circuits are Hurwitz-stable, that is, if a particular solution can be found then the homogeneous part vanishes exponentially such that the particular solution dominates. Then, it must be shown that the particular solution to a linear circuit excited by sinusoids is also a sinusoid.

	In a broader Electrical Engineering sense, this complexification process is of primary importance for is permeating effects, especially because these results beget the notion of impedances: defining capacitors and inductors as ``complex resistances'', defined as $X_C = \left(j\omega C\right)^{-1}$ and $Z_L = j\omega L$, and this in turn allows representing alternate current networks by phasorial equivalents of direct current circuits, extending the properties of resistive direct current circuits to alternating current ones, such that seminal theorems and laws — like Kirchoff's Voltage and Current laws and Thèvenin's and Norton's Theorems — can be easily ported to alternating current equivalents.

	In the narrower field of Power Systems, when capacitors and inductors are substituted by impedances and applied to an electrical grid — with the assumption that the machine and inverter dynamics are supposed much slower than the grid dynamics —, the exponential transient behaviors of the grid dissipate rapidly, allowing modelling the grid as a set of algebraic complex equations. Owing to this, the modelling of the grid itself is greatly simplified and the only dynamical models needed are those of the agents that act upon the grid. Moreover, the transformation of the electrical equations of the grid into complex algebraic equations, often called ``complexifying'' or the ``complexification'' of the eletrical grid has a great many benefits for power system analysis. First, it allows engineers, researchers and designers to obtain voltages and currents without the need to directly solve the time differential equations of the grid; second, it allows for the representation of voltages and currents in a phasor complex diagram, which begets the notions of angle lags, active and reactive power (therefore complex power). Finally, the complex power $S = V \overline{I}$ is shown to be a direct representation of the instantaneous AC power of a circuit, and its real part the average power developed by that circuit.

	In short, the establishment of Classical Phasors need the following steps: first, show that Passive Linear Circuis are stable. Then, show that the operator $p_S$ allosw the establishment of a bijection between a sinusoid $A\cos\left(\omega t + \phi\right)$ with constant amplitude and phase and the complex number $Ae^{j\phi}$, such that the differential equation in time is transformed into an algebraic equation. Then, this complexification process is justified because the sinusoid is the stable solution to the time Differential Equations of an electrical grid, disconsidering transient vanishing behavior of the grid.

	Then, is shown that the ratios between the phasor of the voltage and the phasor of current of bipoles in the phasor domain are algebraic quantities called \textit{impedances}, which act as ``complex resistances'', allowing for the modelling of sinusoidally-excited electrical circuits directly in the phasor domain instead of modelling in time domain to later transport the model to phasors.

	Finally, it is shown that the instant power of an electrical device operating under a sinusoidal voltage and current is also bijective to a complex number called the complex apparent power $S$, and that the real part of $S$ pertains to the average power developed by the device in half a time period.

	These facts are the basis of alternate current grid analysis theory, and are largely taught in engineering schools in the first years of undergraduate courses. After this, the issues with this approach will be shown, motivating the need for Dynamic Phasors. Most importantly, it will be shown that this complexification process is unable to translate more useful signals of a sinusoidal shape where the amplitude and phase angle are variant in time.

%-------------------------------------------------
\section{Linear Circuits as Linear Systems (again)} %<<<1

	We first must define the precise target of our analysis: a linear, passive circuit. Theorem \ref{theo:generic_rlc_modelling} shows that any RLC circuit can be modelled as a linear differential equation. Even though this is a well-known fact in the literature, the gist of this particular theorem is that the result is shown in a very specific form that preserves the circuit structure.

	Proving theorem \ref{theo:generic_rlc_modelling} requires a deep introduction to circuits in graph theory (for concepts like incidence matrices), falling outside of the scope of this thesis. Therefore, the theorem is not proven; for the same reason, the example \ref{example:node_analysis_time} that follows the theorem does not apply the theorem directly due to the lack of precise definitions of incidence matrices. Rather, the example proves the simpler assertion that the circuit under study does define a linear system.

\begin{theorem}[Structure-preserving generic modelling of an RLC circuit \pcite{Freund2008,Huang2022,Antoniadis2019}]\label{theo:generic_rlc_modelling} %<<<
	For a given RLC circuit, denote the incidence matrix

\begin{equation} \mathbf{A}_0 =  \left[\mathbf{A}_R,\mathbf{A}_L,\mathbf{A}_C,\mathbf{A}_V,\mathbf{A}_I\right], \end{equation}

	\noindent composed of $-1,1$ and $0$, where nodes are numbered accordingly. Let the $\mathbf{R,L,C}$ parameter matrices be the matrices of the resistance, inductance and capacitance components ($R$ and $C$ diagonal and $L$ will not be diagonal if mutual inductances are present). Let $x(t) = \left[v^\intercal,i^\intercal\right]^\intercal$, where $v$ is the vector of node voltages, $i$ the branch currents and $f(t) = i_f^\intercal$ the excitation currents from current sources (voltage excitation sources can be transformed into current sources with Norton's Theorem). Then $x$ satisfies

\begin{equation} \mathbf{E}\dot{\mathbf{x}} = \left(\mathbf{J-K}\right)\mathbf{x}(t) + \mathbf{Gf}(t) \label{eq:linear_circuit_linear_system}\end{equation}

	where

\begin{equation} \mathbf{E} = \left[\begin{array}{cc} \mathbf{A}_C C \mathbf{A}_C^\intercal & \mathbf{0}\\[1mm] \mathbf{0} & L \end{array}\right],\ \mathbf{G} = \left[\begin{array}{c} \mathbf{A}_i \\ \mathbf{0} \end{array}\right],\ \mathbf{J} = \left[\begin{array}{cc} \mathbf{0} & -\mathbf{A}_L \\[1mm] \mathbf{A}_L^\intercal & \mathbf{0} \end{array}\right],\ \mathbf{K} = \left[\begin{array}{cc} \mathbf{A}_R \mathbf{R}^{-1}  \mathbf{A}_R^\intercal & \mathbf{0} \\[1mm] \mathbf{0} & \mathbf{0} \end{array}\right] ,\end{equation}

	\noindent and $\mathbf{A}_i$ is the input-to-node connectivity matrix.
\end{theorem} %>>>

\begin{example}[Node analysis of a second-order circuit]\label{example:node_analysis_time} %<<<

	Consider the second-order circuit of figure \ref{fig:nodeanalysis_example}, which we use as an example of node analysis. First, start with the current laws: from the nodes,

% MODELLING EXAMPLE: RLC CIRCUIT <<<
\begin{figure}[htb!]
\centering
        \begin{tikzpicture}[american,scale=1,transform shape,line width=0.75, cute inductors, voltage shift = 1,>={Stealth[inset=0mm,length=1.5mm,angle'=50]}]
	\ctikzset{/tikz/circuitikz/voltage/distance from node=10mm}
		% DRAWING VOLTAGE LOOPS
		\draw[thick, blue  , <-]  ({2+cos(30)},2.5) arc (30:320:1);% syntax (starting point coordinates) arc (starting angle:ending angle:radius)
		\draw[thick, red   , <-]  ({6+cos(30)},2.5) arc (30:320:1);% syntax (starting point coordinates) arc (starting angle:ending angle:radius)
		\draw[thick, green , <-] ({10+cos(30)},2.5) arc (30:320:1);% syntax (starting point coordinates) arc (starting angle:ending angle:radius)

		\draw[thick, stewartyellow, <-]  (7.75,4.75) arc (0:320:0.5);% syntax (starting point coordinates) arc (starting angle:ending angle:radius)

		\draw (0,0)
			to[vsource,sources/scale=1.25,f<^=$i_{V1}$, v>=$v_1$,invert] (0,4)
			to[R,l=$R_1$,f>^=$i_{R1}$,v>=$v_{R1}$,-*] (4,4) 
			to[C,l=$C_1$,f>^=$i_{C1}$,v>=$v_{C1}$,-*] (4,0) 
			to[short] (0,0); 
		\draw (4,4)
			to[short]  (4,6)
			to[isource,sources/scale=1.25, l=$i_1$, v>=$v_{I1}$] (8,6)
			to[short]  (8,4);
		\draw (4,4)
			to[R,l=$R_2$,f>^=$i_{R2}$,v>=$v_{R2}$,-*] (8,4) 
			to[R,l=$R_3$,f>^=$i_{R3}$,v>=$v_{R3}$,-*] (8,0)
			to[short]  (4,0);
		\draw (8,4)
			to[L,l=$L_1$,f>=$i_{L1}$,v>=$v_{L1}$] (12,4) 
			to[R,l=$R_4$,f>^=$i_{R4}$,v>=$v_{R4}$] (12,0) 
			to[short]  (8,0);

		% DRAWING NODE LABELS
		\node[shape=circle,draw,inner sep=1pt] at (  0,4.5) {$1$};
		\node[shape=circle,draw,inner sep=1pt] at (3.5,4.5) {$2$};
		\node[shape=circle,draw,inner sep=1pt] at (8.5,4.5) {$3$};
		\node[shape=circle,draw,inner sep=1pt] at ( 12,4.5) {$4$};
		\node[shape=circle,draw,inner sep=1pt] at ( 6,-0.5) {$5$};
		
		% DRAWING LOOP LABELS

		\node[color=blue] at (2,2) {$L1$} ;
		\node[color=red ] at (6,2) {$L2$} ;
		\node[color=stewartyellow] at (7.25,4.75) {$L3$} ;
		\node[color=green] at (10,2) {$L4$} ;
        \end{tikzpicture}
	\caption{Second-order circuit for node analysis example.}
	\label{fig:nodeanalysis_example}
\end{figure} %>>>

\begin{equation} %<<<
	\left\{\begin{array}{l}
		(1):\ -i_{V1} - i_{R1} = 0 \\[3mm]
		(2):\ i_{R1} - i_{R2} - i_{C1} - i_{I1} = 0 \\[3mm]
		(3):\ i_{R2} - i_{R3} + i_{I1} - i_{L1} = 0 \\[3mm]
		(4):\ i_{L1} - i_{R4} = 0 \\[3mm] 
		(5):\ i_{V1} + i_{C1} + i_{R3} + i_{R4} = 0 
	\end{array}\right.
\end{equation} %>>>

	But since $i_{V1} = i_{R1}$, we eliminate the former:

\begin{equation} %<<<
	\left\{\begin{array}{l}
		i_{R1} - i_{R2} - i_{C1} - i_1 = 0 \\[3mm]
		i_{R2} - i_{R3} + i_1 - i_{L1} = 0 \\[3mm]
		i_{L1} - i_{R4} = 0 \\[3mm] 
		i_{R1} + i_{C1} + i_{R3} + i_{R4} = 0 
	\end{array}\right.
\end{equation} %>>>

	In matrix form,

\begin{equation} %<<<
%
	\left[\begin{array}{ccccccc}
	-1 &  0 &  1 &-1 & 0 & 0 \\[3mm]
	 0 & -1 &  0 & 1 &-1 & 0 \\[3mm]
	 0 &  1 &  0 & 0 & 0 &-1 \\[3mm]
	 1 &  0 &  1 & 0 & 0 & 1 
	\end{array}\right]
%
	\left[\begin{array}{c}
		i_{C1} \\[3mm] i_{L1} \\[3mm] i_{R1} \\[3mm] i_{R2} \\[3mm] i_{R3} \\[3mm] i_{R4}
	\end{array}\right] =
%
	\left[\begin{array}{c}
		0 \\[3mm] 1 \\[3mm] -1 \\[3mm] 0 \\[3mm] 0
	\end{array}\right]
%
	\left[\begin{array}{c}
		i_1
	\end{array}\right] \label{eq:example_KCL}
\end{equation} %>>>

	Now apply Kirchoff's Voltage Law on the loops:

\begin{equation} %<<<
	\left\{\begin{array}{l}
		(L1):\ -v_1 + v_{R1} = 0 \\[3mm]
		(L2):\ -v_{C1} + v_{R2} + v_{R1} = 0 \\[3mm]
		(L3):\ v_{I1} + v_{R2} = 0 \\[3mm]
		(L4):\ -v_{R3} + v_{L1} + v_{R4} = 0 
	\end{array}\right.
\end{equation} %>>>

	But since $V_{I1} = -v_{R2}$, we eliminate the former:

\begin{equation} %<<<
	\left\{\begin{array}{l}
		-v_1 + v_{R1} = 0 \\[3mm]
		-v_{C1} + v_{R2} + v_{R1} = 0 \\[3mm]
		-v_{R3} + v_{L1} + v_{R4} = 0 
	\end{array}\right.
\end{equation} %>>>

	In matrix form,

\begin{equation} %<<<
%
	\left[\begin{array}{ccccccc}
	 0 &  0 &  1 & 0 & 0 & 0 \\[3mm]
	-1 &  0 &  0 & 1 & 1 & 0 \\[3mm]
	 0 &  1 &  0 & 0 &-1 & 1 
	\end{array}\right]
%
	\left[\begin{array}{c}
		v_{C1} \\[3mm] v_{L1} \\[3mm] v_{R1} \\[3mm] v_{R2} \\[3mm] v_{R3} \\[3mm] v_{R4}
	\end{array}\right] =
%
	\left[\begin{array}{c}
		1 \\[3mm] 0 \\[3mm] 0
	\end{array}\right]
%
	\left[\begin{array}{c}
		v_1
	\end{array}\right]\label{eq:example_KVL}
\end{equation} %>>>

	Now using the capacitor, inductor and resistor relationships on \eqref{eq:example_KCL} and \eqref{eq:example_KVL},

\begin{equation} %<<<
\left\{\begin{array}{rcl}
	\left[\begin{array}{ccccccc}
	-C_1 &  0 & R_1 & -R_2 &  0   & 0    \\[3mm]
	   0 & -1 &   0 &  R_2 & -R_3 & 0    \\[3mm]
	   0 &  1 &   0 &    0 &  0   & -R_4 \\[3mm]
	 C_1 &  0 & R_1 &    0 &  0   &  R_4 
	\end{array}\right]
%
	\left[\begin{array}{c}
		\dot{v}_{C1} \\[3mm] i_{L1} \\[3mm] v_{R1} \\[3mm] v_{R2} \\[3mm] v_{R3} \\[3mm] v_{R4}
	\end{array}\right] &=&
%
	\left[\begin{array}{c}
		0 \\[3mm] 1 \\[3mm] -1 \\[3mm] 0 
	\end{array}\right]
%
	\left[\begin{array}{c}
		i_1
	\end{array}\right]  \\[30mm]
%
	\left[\begin{array}{ccccccc}
	 0 &  0 &  1 & 0 & 0 & 0 \\[3mm]
	-1 &  0 &  0 & 1 & 1 & 0 \\[3mm]
	 0 &L_1 &  0 & 0 &-1 & 1
	\end{array}\right]
%
	\left[\begin{array}{c}
		v_{C1} \\[3mm] \dot{i}_{L1} \\[3mm] v_{R1} \\[3mm] v_{R2} \\[3mm] v_{R3} \\[3mm] v_{R4}
	\end{array}\right] &=&
%
	\left[\begin{array}{c}
		1 \\[3mm] 0 \\[3mm] 0
	\end{array}\right]
%
	\left[\begin{array}{c}
		v_1
	\end{array}\right]
\end{array}\right.
\end{equation} %>>>

	Isolating $v_{C1},\dot{v}_{C1},i_{L1},\dot{i}_{L1}$ and grouping them in a vector,

\begin{equation} %<<<
\left\{\begin{array}{rcl}
	\left[\begin{array}{ccccccc}
	-C_1 & 0 \\[3mm]
	   0 & 0\\[3mm]
	   0 & 0\\[3mm]
	 C_1 & 0 
	\end{array}\right]
%
	\left[\begin{array}{c}
		\dot{v}_{C1} \\[3mm] \dot{i}_{L1}
	\end{array}\right]
+
	\left[\begin{array}{ccccccc}
	 0 &  0 \\[3mm]
	 0 & -1 \\[3mm]
	 0 &  1 \\[3mm]
	 0 &  0 
	\end{array}\right]
%
	\left[\begin{array}{c}
		v_{C1} \\[3mm] i_{L1}
	\end{array}\right]
+
	\left[\begin{array}{ccccccc}
	R_1 & -R_2 &  0   & 0    \\[3mm]
	  0 &  R_2 & -R_3 & 0    \\[3mm]
	  0 &    0 &  0   & -R_4 \\[3mm]
	R_1 &    0 &  0   &  R_4 
	\end{array}\right]
%
	\left[\begin{array}{c}
		v_{R1} \\[3mm] v_{R2} \\[3mm] v_{R3} \\[3mm] v_{R4}
	\end{array}\right] &=&
%
	\left[\begin{array}{c}
		0 \\[3mm] 1 \\[3mm] -1 \\[3mm] 0
	\end{array}\right]
%
	\left[\begin{array}{c}
		i_1
	\end{array}\right]  \\[30mm]
%
\left[\begin{array}{ccccccc}
	 0 &  0 \\[3mm]
	 0 &  0 \\[3mm]
	 0 &L_1 
	\end{array}\right]
%
	\left[\begin{array}{c}
		\dot{v}_{C1} \\[3mm] \dot{i}_{L1}
	\end{array}\right]
+
	\left[\begin{array}{ccccccc}
	 0 & 0 \\[3mm]
	-1 & 0 \\[3mm]
	 0 & 0 
	\end{array}\right]
%
	\left[\begin{array}{c}
		v_{C1} \\[3mm] i_{L1}
	\end{array}\right]
+
	\left[\begin{array}{ccccccc}
		1 & 0 & 0 & 0 \\[3mm]
		0 & 1 & 1 & 0 \\[3mm]
		0 & 0 &-1 & 1
	\end{array}\right]
%
	\left[\begin{array}{c}
		v_{R1} \\[3mm] v_{R2} \\[3mm] v_{R3} \\[3mm] v_{R4}
	\end{array}\right] &=&
%
	\left[\begin{array}{c}
		1 \\[3mm] 0 \\[3mm] 0
	\end{array}\right]
%
	\left[\begin{array}{c}
		v_1
	\end{array}\right]
\end{array}\right.
\end{equation} %>>>

	\noindent and the vector $\left[v_{R1},\ v_{R2},\ v_{R3},\ v_{R4}\right]$ can be obtained as a function of $\left[v_{C1},i_{L1}\right]^\transpose$ and its derivatives from the first equation because the matrix that multiplies it is invertible. Substituting into the second equation eliminates $\left[v_{R1},\ v_{R2},\ v_{R3},\ v_{R4}\right]$, leading to an equation of the form \eqref{eq:linear_circuit_linear_system} where $\mathbf{x} = \left[v_{C1},i_{L1}\right]^\transpose$.

\examplebar
\end{example}%>>>

%-------------------------------------------------
\section{Hurwitz stability of Passive Linear Circuits} %<<<1

	Having shown that a PLC defines a linear system, we now prove that this linear system will be stable, by proving they are Lyapunov stable thus Hurwitz stable by corollary \ref{corollary:hurwitz_lti_odes}.

\begin{lemma}\label{lemma:positive_def_diag} A diagonal matrix of positive coefficients is positive definite. \end{lemma}
\noindent\textbf{Proof:} by simple inspection. Take $\mathbf{A}$ such matrix with positive diagonal elements $a_i$, and suppose it has size $n$. Consider an arbitrary complex vector $\mathbf{x}$ of size $n$. Then

\begin{equation} \mathbf{x}^\hermconj \mathbf{A} = \left[\begin{array}{ccccc} a_1\overline{x}_1 & 0 & 0 & \cdots & 0\\[3mm] 0 & a_2\overline{x}_2 & 0 & \cdots & 0 \\[3mm]  0 & 0 & a_3\overline{x}_3 & \cdots & 0\\[3mm] \vdots & \vdots & \vdots & \ddots & \vdots \\[3mm] 0 & 0 & 0 & \cdots & a_n\overline{x}_n \end{array}\right]. \end{equation}

\begin{equation} \mathbf{x}^\hermconj \mathbf{Ax} =  a_1\overline{x}_1x_1 + a_2\overline{x}_2x_2 + a_3\overline{x}_3x_3 + \cdots +  a_n\overline{x}_nx_n = \sum_{k=1}^n a_k \left\lvert x_k\right\rvert^2 , \end{equation}

	\noindent and because all $a_k$ are positive, this is always positive, except at the origin.
\begin{lemma} \label{lemma:positive_def_congruent} Any matrix that is congruent to another positive definite matrix is also positive definite. \end{lemma}
\noindent\textbf{Proof:} here, congruency means that a matrix $\mathbf{A}$ is congruent to another $\mathbf{B}$ is there is a matrix $\mathbf{C}$ such that $\mathbf{A} = \mathbf{C}^\hermconj\mathbf{BC}$; then

\begin{equation} \mathbf{x}^\hermconj\mathbf{Ax} = \mathbf{x}^\hermconj\mathbf{C}^\hermconj\mathbf{BCx} = \left(\mathbf{Cx}\right)^\hermconj\mathbf{B}\left(\mathbf{Cx}\right) \end{equation}

	\noindent and, if $\mathbf{B} > 0$, this is always positive, hence $\mathbf{x}^\hermconj\mathbf{Ax}$ is also always positive, therefore $\mathbf{A} > 0$.

\begin{theorem}[PLCs are stable]\label{theo:plcs_stable} %<<<
	Any non-excited Passive Linear Circuit, that is, a circuit comprised of only inductances, capacitances and resistances with at least one resistance, is stable.
\end{theorem}
\noindent\textbf{Proof:} by finding a Lyapunov Function. Suppose that the linear circuit in question has $p$ inductors, $q$ capacitors and $w$ resistors and write the state space $\mathbf{x}$ as follows: first the capacitor voltages, then the inductor currents

\begin{equation} \mathbf{x} = \left[ \overbrace{v_1,v_2,...,v_q}^{\text{Capacitor voltages}}, \overbrace{i_1, i_2 , ... i_p}^{\text{Inductor currents}}\right] \end{equation}

	\noindent and this system (with no excitations) is modelled as $\dot{\mathbf{x}} = \mathbf{Ax}$. Also write $\mathbf{i}_R = \left[i_{R1},...,i_{Rw}\right]^\transpose$ as the vector of currents on resistors. Now consider the energy functions:

\begin{itemize}
	\item For inductors, adopt the energy stored in the magnetic field $E_L(t) = \dfrac{1}{2} Li_L^2(t)$;
	\item For capacitors, adopt the energy stored in the electric field $E_C(t) = \frac{1}{2} Cv_C^2(t)$;
	\item And for resistors, adopt the total energy expenditure at time $t$: $E_R(t) = \displaystyle R\int_{-\infty}^t i_R^2(s)ds$.
\end{itemize}

	\noindent and $U(t)$ as the total energy developed by the system at a given time:

\begin{equation}
	U(t) = \mathbf{x}^\transpose\dfrac{1}{2}\left[\begin{array}{cccccccc} C_1 & 0 & ... & 0 & 0 & 0 & ... & 0 \\[3mm]  0 & C_2 & ... & 0 & 0 & 0 & ... & 0 \\[3mm] \vdots & \vdots & \ddots & \vdots & \vdots & \vdots & \ddots & \vdots \\[3mm] 0 & 0 & ... & C_q & 0 & 0 & ... & 0 \\[3mm] 0 & 0 & ... & 0 & L_1 & 0 & ... & 0 \\[3mm] 0 & 0 & ... & 0 & 0 & L_2 & ... & 0 \\[3mm] \vdots & \vdots & \ddots & \vdots & \vdots & \vdots & \ddots & \vdots \\[3mm] 0 & 0 & ... & 0 & 0 & 0 & ... & L_p \end{array}\right]\mathbf{x} + \int_{-\infty}^{t} \mathbf{i_R}^\transpose(s) \left[\begin{array}{cccc} R_1 & 0 & ... & 0 \\[3mm] 0 & R_2 & ... & 0 \\[3mm] \vdots & \vdots & \ddots & \vdots \\[3mm] 0 & 0 & ... & R_w \end{array}\right] \mathbf{i_R}(s)ds .
\end{equation}

	Now denote the capacitance, inductance and resistance matrices

\begin{equation}
	\mathbf{C} = \left[\begin{array}{ccccc} C_1 & 0 & ... & 0 \\[3mm]  0 & C_2 & ... & 0 \\[3mm] \vdots & \vdots & \ddots & \vdots \\[3mm] 0 & 0 & ... & C_q \end{array}\right],\
	\mathbf{L} = \left[\begin{array}{ccccc} L_1 & 0 & ... & 0 \\[3mm]  0 & L_2 & ... & 0 \\[3mm] \vdots & \vdots & \ddots & \vdots \\[3mm] 0 & 0 & ... & L_p \end{array}\right],\
	\mathbf{R} = \left[\begin{array}{ccccc} R_1 & 0 & ... & 0 \\[3mm]  0 & R_2 & ... & 0 \\[3mm] \vdots & \vdots & \ddots & \vdots \\[3mm] 0 & 0 & ... & R_w \end{array}\right]
\end{equation}

	Then

\begin{equation} U(t) = \dfrac{1}{2} \mathbf{x}^\transpose\left[\begin{array}{cc} \mathbf{C} & \mathbf{0} \\[3mm] \mathbf{0} & \mathbf{L} \end{array}\right]\mathbf{x} + \int_{-\infty}^{t} \mathbf{i_R}^\transpose (s) \mathbf{R} \mathbf{i_R}(s)ds = \dfrac{1}{2} \mathbf{x}^\transpose\mathbf{Z}\mathbf{x} + \int_{-\infty}^{t} \mathbf{i_R}^\transpose (s) \mathbf{R} \mathbf{i_R}(s)ds \end{equation}

	\noindent that is, this function $U$ represents the stored energy on capacitors and inductors plus the energy dissipated by resistors. By Tellegen's Theorem \cite{desoerBasicCircuitTheory1987}, this function is constant in time, that is, $\dot{U} = \mathbf{0}$; intuitively, since the circuit is a closed system, no energy gets in or out. 

	We now want to write $U$ as a function of the states $\mathbf{x}$, which is achieved by writing $\mathbf{i_R}$ as a function of $\mathbf{x}$. According to Kirchoff's Current Law, the current through the k-th resistor $i_{Rk}$ is given by a direct sum of the currents of all other elements of the circuit:

\begin{equation} i_{Rk} = \mathbf{i_k}^\transpose \left[i_{C1},i_{C2},...,i_{Cq},i_{L1},i_{L2},...,i_{Lp},i_{R1},i_{R2},...,i_{Rw}\right]^\transpose. \end{equation}

	where $\mathbf{i}_k$ is a column vector composed of elements that are $1$, $0$ or $-1$ and necessarily $0$ at the $i_{Rk}$ position. Then use the capacitor models to write

\begin{equation} i_{Rk} = \mathbf{i_k}^\transpose \left[C_1\dot{v}_{C1},C_2\dot{v}_{C2},...,C_q\dot{v}_{Cq},i_{L1},i_{L2},...,i_{Lp},i_{R1},i_{R2},...,i_{Rw}\right]^\transpose .\end{equation}

	Arranging the resistor currents as rows of the vector $\mathbf{i_R}$,

\begin{equation} \mathbf{i_R} = \left[\begin{array}{c} i_{R1} \\[3mm] i_{R2} \\[3mm] \vdots \\[3mm] i_{Rw} \end{array}\right] = \left[\begin{array}{c} \left[\begin{array}{ccc} ... & \mathbf{i_1}^\transpose & ... \end{array}\right] \\[3mm] \left[\begin{array}{ccc} ... & \mathbf{i_2}^\transpose & ... \end{array}\right] \\[3mm] \vdots \\[3mm] \left[\begin{array}{ccc} ... & \mathbf{i_w}^\transpose & ... \end{array}\right] \end{array}\right] \left[\begin{array}{c} C_1\dot{v}_{C1} \\[3mm] C_2\dot{v}_{C2} \\[3mm] \vdots \\[3mm] C_q\dot{v}_{Cq} \\[3mm] i_{L1} \\[3mm] i_{L2} \\[3mm] \vdots \\[3mm] i_{Lp} \\[3mm] i_{R1} \\[3mm] i_{R2} \\[3mm] \vdots \\[3mm] i_{Rw}\end{array} \right] = \left[\begin{array}{ccc} \mathbf{A^I_CC} & \mathbf{A^I_L} & \mathbf{A^I_R} \end{array}\right] \left[\begin{array}{c} \dot{v}_{C1} \\[3mm] \dot{v}_{C2} \\[3mm] \vdots \\[3mm] \dot{v}_{Cq} \\[3mm] i_{L1} \\[3mm] i_{L2} \\[3mm] \vdots \\[3mm] i_{Lp} \\[3mm] i_{R1} \\[3mm] i_{R2} \\[3mm] \vdots \\[3mm] i_{Rw}\end{array} \right], \end{equation}

	\noindent where the $\mathbf{A^I}$ matrices are called called current adjacency matrices. $\mathbf{A^I_C}$ is of size $w\times q$, $\mathbf{A^I_L}$ of size $w\times p$ and $\mathbf{A^I_R}$ of size $w\times w$ with null diagonal, and all are composed of $-1,1,0$ elements. Then

\begin{equation} \mathbf{i_R} = \mathbf{A^I_CC} \left[\begin{array}{c} \dot{v}_{C1} \\[3mm] \dot{v}_{C2} \\[3mm] \vdots \\[3mm] \dot{v}_{Cq} \end{array}\right] + \mathbf{A^I_L}\left[\begin{array}{c} i_{L1} \\[3mm] i_{L2} \\[3mm] \vdots \\[3mm] i_{Lp} \end{array}\right] + \mathbf{A^I_R}\left[\begin{array}{c} i_{R1} \\[3mm] i_{R2} \\[3mm] \vdots \\[3mm] i_{Rw} \end{array}\right] = \mathbf{A^I_CC}\dot{\mathbf{v}}_\mathbf{C} + \mathbf{A^I_Li_L} + \mathbf{A^I_Ri_R}. \label{eq:current_adjacency}\end{equation}

	Doing the same process with the voltages across resistors, by Kirchoff's Voltage Law, these voltages are direct combinations of the capacitor and inductor voltages:

\begin{equation} \mathbf{v_R}= \mathbf{A^V_C} \left[\begin{array}{c} v_{C1} \\[3mm] v_{C2} \\[3mm] \vdots \\[3mm] v_{Cq} \end{array}\right]  + \mathbf{A^V_L}\mathbf{L} \left[\begin{array}{c} \dot{i}_{L1} \\[3mm] \dot{i}_{L2} \\[3mm] \vdots \\[3mm] \dot{i}_{Lp} \end{array}\right] + \mathbf{A^V_R}\mathbf{R}\mathbf{i}_R , \label{eq:voltage_adjacency}\end{equation}

	\noindent where the $\mathbf{A^V_C},\ \mathbf{A^V_L},\ \mathbf{A^V_R}$ are voltage adjacency matrices comprised of $-1,0,1$ and $\mathbf{A^V_R}$ has null diagonal. Then substituting \eqref{eq:voltage_adjacency} into \eqref{eq:current_adjacency} and noting that $\mathbf{v_R} = \mathbf{Ri_R}$, and that $\mathbf{R}$ is diagonal hence invertible and

\begin{equation}
	\mathbf{i_R} = \mathbf{A^I_C}\mathbf{C} \left[\begin{array}{c} \dot{v}_{C1} \\[3mm] \dot{v}_{C2} \\[3mm] \vdots \\[3mm] \dot{v}_{Cq} \end{array}\right]  + \mathbf{A^I_L} \left[\begin{array}{c} i_{L1} \\[3mm] i_{L2} \\[3mm] \vdots \\[3mm] i_{Lp} \end{array}\right] + \mathbf{A^I_R}\mathbf{R}^{-1}\left\{\mathbf{A^V_C} \left[\begin{array}{c} v_{C1} \\[3mm] v_{C2} \\[3mm] \vdots \\[3mm] v_{Cq} \end{array}\right]  + \mathbf{A^V_L}\mathbf{L} \left[\begin{array}{c} \dot{i}_{L1} \\[3mm] \dot{i}_{L2} \\[3mm] \vdots \\[3mm] \dot{i}_{Lp} \end{array}\right] + \mathbf{A^V_R}\mathbf{R}\mathbf{i}_R \right\}
\end{equation}

	And reorganizing,

\begin{equation}
	\left(\mathbf{I} - \mathbf{A^I_RR^{-1}A^V_RR}\right)\mathbf{i_R} = \left[\begin{array}{cc} \mathbf{A^I_CC} & \mathbf{A^I_RR^{-1}A^V_LL} \end{array}\right]\dot{\mathbf{x}} + \left[\begin{array}{cc} \mathbf{A^I_RR^{-1}A^V_C} & \mathbf{A^I_L} \end{array}\right]\mathbf{x}
\end{equation}

	Therefore 

\begin{equation}
	\mathbf{i_R} = \mathbf{K}\dot{\mathbf{x}} + \mathbf{Lx}
\end{equation}

	\noindent and using the system differential model $\dot{\mathbf{x}} = \mathbf{Ax}$,
	
\begin{equation} \mathbf{i_R} = \mathbf{K}\mathbf{Ax} + \mathbf{Lx} = \left(\mathbf{KA + L}\right)\mathbf{x}\end{equation}

	Then

\begin{equation} U\left(\mathbf{x}\right) = \dfrac{1}{2} \mathbf{x}^\transpose\mathbf{Z}\mathbf{x} + \int_{-\infty}^{t} \mathbf{x}^\transpose(s)\left(\mathbf{KA + L}\right)^\transpose \mathbf{R}\left(\mathbf{KA + L}\right)\mathbf{x}(s)ds \end{equation}

	Now let us take a closer look at the matrices involved. By definition,

\begin{equation} \mathbf{Z} = \left[\begin{array}{cc} \mathbf{C} & \mathbf{0} \\[3mm] \mathbf{0} & \mathbf{L} \end{array}\right] \end{equation}

	\noindent and by lemma \ref{lemma:positive_def_diag} $\mathbf{Z}$ is positive definite because it is a diagonal matrix with positive entries; at the same time, the matrix

\begin{equation} \mathbf{T} = \left(\mathbf{KA + L}\right)^\transpose \mathbf{R}\left(\mathbf{KA + L}\right) \end{equation}

	\noindent is congruent to $\mathbf{R}$. Hence, by lemma \ref{lemma:positive_def_congruent}, $\mathbf{T}$ is also positive definite because $\mathbf{R}$ is positive definite. This also makes $\mathbf{T}$ hermitian. Using $\dot{U} = 0$, 

\begin{align}
	0 &= \dfrac{d}{dt}\left[\mathbf{x}^\transpose\dfrac{1}{2}\mathbf{Zx} + \int_{-\infty}^{t} \mathbf{x}^\transpose(s)\mathbf{T}\mathbf{x}(s)ds\right] \nonumber\\[3mm]
	0 &= \mathbf{x}^\transpose\mathbf{ZAx} + \mathbf{x}^\transpose\mathbf{T}\mathbf{x} \nonumber\\[3mm]
	\mathbf{x}^\transpose\mathbf{ZAx} &= - \mathbf{x}^\transpose\mathbf{T}\mathbf{x} \label{theo:lincircuits_are_lyapunov}
\end{align}

	Now, consider the function

\begin{equation} V\left(\mathbf{x}\right) = \dfrac{1}{2}\mathbf{x}^\transpose \mathbf{Zx} \end{equation}

	\noindent as the candidate for Lyapunov Energy Function of this system. Notably, $V$ represents only the energy stored in the capacitors and inductors; this function is always positive and can be zero only at the origin, since $\mathbf{Z}$ is positive definite. Then

\begin{equation} \dot{V} = \dfrac{1}{2}\dfrac{d}{dt}\left(\mathbf{x}^\transpose \mathbf{Zx}\right) =  \mathbf{x}^\transpose \mathbf{ZAx}\end{equation}

	but by equation \eqref{theo:lincircuits_are_lyapunov} this implies

\begin{equation} \dot{V} = -\mathbf{x}^\transpose\mathbf{T}\mathbf{x} \end{equation}

	and because $\mathbf{T}$ is positive definite, this function is always negative.
\hfill$\blacksquare$
\vspace{5mm}
\hrule
\vspace{5mm}
%>>>
\begin{remark} The stability result dictates that the circuit \textbf{needs} at least one resistance; if no resistances are present, then $\mathbf{T = 0}$, meaning the function $\mathbf{V}$ is not positive definite but semi-positive definite and that the circuit will stay in the manifold defined by some $V\left(\mathbf{x}\right) = k > 0$. Intuitively, this means that if the circuit is not \textbf{passive} (it does not ``consume'' energy), then this energy is constantly exchanged between the capacitors and inductors without ``being spent'', that is, the system is not forced to $V\left(\mathbf{x}\right) = 0$. \end{remark}

	Due to the properties of linear systems, PLCs being Lyapunov stable mean they are also Hurwitz stable, hence exponentially stable. This fact is of great uses in the theory of electrical circuits, notably the fact that if such is the case, the transient (``natural'') behavior of the system inevitably vanishes exponentially, and the assymptotic behavior is solely described by the particular (``forced'') behavior. In the theory of alternating current circuits, Hurwitz Stability plays a major role as a simplifying characteristic of linear passive networks. Theorem \ref{theo:phasors_solutions} proves that any LTI ODE, when excited sinusoidally, has a exponential homogeneous response and a sinusoidal forced response; if the system is Hurwitz-stable, then the natural response vanishes and only the sinusoidal response remains, meaning that the exponentially stable steady-state response of the system is also sinusoidal.

\begin{theorem}[Steady-state solutions of sinusoidally-forced LTI ODEs]\label{theo:phasors_solutions} %<<<
Consider the linear n-th order LTI Ordinary Differential Equation

\begin{equation} \sum\limits_{k=0}^n \alpha_k x^{(k)}(t) - M\cos\left(\omega t\right) = 0,\label{eq:linear_ode_phasor_solution_1}\end{equation}

	\noindent where $y^{(k)}$ represents the k-th derivative of $y$ with $y^{(0)} \equiv y$; the $\alpha_k$ are real numbers with $\alpha_n \neq 0$, and $M,\omega$ are positive real numbers. If the associated Hurwitz Polynomial

\begin{equation} H\left(z\right) = \sum\limits_{k=0}  \alpha_k z^k \label{theo:linear_ode_phasor_solution_3}\end{equation}

	\noindent is stable, that is, has only roots with negative real part, then the globally exponentially stable steady-state solution of \eqref{eq:linear_ode_phasor_solution_1} is given by

\begin{equation} x_s(t) = K\cos\left(\omega t + \phi\right) \label{eq:linear_ode_phasor_solution_2}\end{equation}

	\noindent where 

\begin{align}
	K &= \sqrt{A^2 + B^2} = \sqrt{\left(\alpha_0 - \alpha_2\omega^2 + ...\right)^2 + \left(\alpha_1 - \alpha_3\omega^3 + ...\right)^2}\\[3mm]
	\tan\left(\phi\right) &= \dfrac{\left(\alpha_1 - \alpha_3\omega^3 + ...\right)}{\left(\alpha_0 - \alpha_2\omega^2 + ...\right)}
\end{align}

\end{theorem}
\textbf{Proof: } because the system is linear and Hurwitz-stable, its homogenous non-forced equivalent is surely globally exponentially assymptotic. This means that $x(t)$ will tend, globally and exponentially, to a particular solution $x_p$, and the only challenge is to find one such particular solution. Because the space of sinusoids at a particular frequency $\omega$ is invariant to differentiation, then surely a linear combination of both is a solution to the original ODE. Because of this, suppose

\begin{equation} x_p(t) = A\cos\left(\omega t\right) + B\sin\left(\omega t\right) \end{equation}

	then calculate $A$ and $B$: applying $x_p$ into the original ODE,

\begin{gather}
	\sum\limits_{k=0}^n \alpha_k\left[A\cos\left(\omega t\right) + B\sin\left(\omega t\right)\right]^{(k)} - M\cos\left(\omega t\right) = 0 \nonumber\\[3mm]
	\left. \begin{array}{l}
		\alpha_0\left[A\cos\left(\omega t\right) + B\sin\left(\omega t\right)\right] + \\[3mm]
		\hspace{5mm}\alpha_1\left[-A\omega\sin\left(\omega t\right) + B\omega\cos\left(\omega t\right)\right] + \\[3mm]
		\hspace{10mm}\alpha_2\left[-A\omega^2\cos\left(\omega t\right) - B\omega^2\sin\left(\omega t\right)\right] + \\[3mm]
		\hspace{30mm} \vdots\\[3mm]
		\hspace{15mm} - M\sin\left(\omega t\right) = 0 
	\end{array}\right.  \nonumber\\[3mm]
%
	\left(-M + \alpha_0A + \alpha_1\omega B - \alpha_2\omega^2A + ...\right)\cos\left(\omega t\right) +  \left(\alpha_0 B - \alpha_1\omega A - \alpha_2\omega^2B + ...\right)\sin\left(\omega t\right) = 0
\end{gather}

	But since the sine and cosine functions are orthogonal, this can only be true if $A$ and $B$ satisfy

\begin{equation}
\left\{\begin{array}{l}
	\alpha_0A + \alpha_1 B\omega - \alpha_2\omega^2A + ... - M = 0 \\[3mm]
	\alpha_0B - \alpha_1 A\omega - \alpha_2\omega^2B + ... = 0
\end{array}\right.
\end{equation}

	Developing this system,

\begin{gather}
\left\{\begin{array}{l}
	B\overbrace{\left(\alpha_0 - \alpha_2\omega^2 + ...\right)}^{\text{Even exponents}} - A\overbrace{\left(\alpha_1 - \alpha_3\omega^3 + ...\right)}^{\text{Odd exponents}} = 0 \\[3mm]
	A\left(\alpha_0 - \alpha_2\omega^2 + ...\right) + B\left(\alpha_1 - \alpha_3\omega^3 + ...\right) - M= 0
\end{array}\right. \nonumber\\[5mm]
	\Rightarrow\left\{\begin{array}{l}
	B = M \left[\dfrac{\left(\alpha_0 - \alpha_2\omega^2 + ...\right)}{\left(\alpha_0 - \alpha_2\omega^2 + ...\right)^2 + \left(\alpha_1 - \alpha_3\omega^3 + ...\right)^2} \right] \\[10mm]
	A = M \left[\dfrac{-\left(\alpha_1 - \alpha_3\omega^3 + ...\right)}{\left(\alpha_0 - \alpha_2\omega^2 + ...\right)^2 + \left(\alpha_1 - \alpha_3\omega^3 + ...\right)^2} \right]
	\end{array}\right.
\end{gather}

	Therefore $A$ and $B$ are calculated. Adopt

\begin{align}
	K &= \sqrt{A^2 + B^2} = \sqrt{\left(\alpha_0 - \alpha_2\omega^2 + ...\right)^2 + \left(\alpha_1 - \alpha_3\omega^3 + ...\right)^2}\\[3mm]
	\tan\left(\phi\right) &= \dfrac{\left(\alpha_1 - \alpha_3\omega^3 + ...\right)}{\left(\alpha_0 - \alpha_2\omega^2 + ...\right)}
\end{align}

	And $x_p = K\cos\left(\omega t + \phi\right)$. \hfill$\blacksquare$
\vspace{5mm}
\hrule
\vspace{5mm}
% >>>

	 In other words, a Hurwitz-stable LTI system when excited sinusoidally responds with a natural homogeneous response of exponential decay added by a second part corresponding to the excited response, and it is also sinusoidal. Consequently, after enough time the excited sinusoidal part of its response dominates. If the time constants are small enough compared to the sinusoid, that is, if the initial transient timescale is disregarded, then a timescale argument can be made that the system can be modelled in its steady-state by a purely sinusoidal response. In short, the particular solution

\begin{equation} x_p(t) = K\sin\left(\omega t + \phi\right) \end{equation}

	\noindent is the exponentially stable steady-state solution to the original LTI ODEs \eqref{eq:linear_ode_phasor_solution_1}. Because of this, denote $x_\infty = x_p$, where the infinity symbol denotes the fact that $x(t)$ tends to $x_\infty$ as $t\to\infty$. 

	Here we can already see a glimpse of an argument for quasistatic modelling. The fact that passive electrical grids are Hurwitz (and therefore exponentially) stable has profound consequences in the study of Electrical Power Systems. Supposing the electrical machines and inverter systems connected to the grid impose to it perfectly sinusoidal voltages, theorem \eqref{theo:phasors_solutions} proves that the solution to the linear ODEs defined by the grid will also be sinusoidal signals added by vanishing exponential terms. Thus if the timescales of these vanishing terms are significantly slower than the sinusoidal forcing period, they can be disregarded for effects of transient analysis without much loss in accuracy. To some effect, this means that if the circuit network is ``quicker'' than the excitations, then the exponentials vanish quickly enough that the steady-state sinusoidal behavior can be considered to be the transient solution.

%-------------------------------------------------
\section{Static Phasors} %<<<1

	One fortunate result of the linearity and Hurwitz Stability of PLCs and the fact that their steady-state response to sinusoidal excitations are sinusoids themselves is the fact that the combined excitations lead to combined responses, that is, when subject to a combination of two sinusoids, the reponse is the combination of the individual response of the sinusoids. The problem now lies in the fact that the algebra of sinusoids is problematic, and takes a lot of calculations to be done, as shown in theorem \ref{theo:closed}, which also shows that the class of sinusoids is, in fact, closed to addition.

\begin{theorem}[The class of sinusoids is closed to addition] \label{theo:closed}%<<<
	The sum of two sinusoids of the same frequency is a sinusoid at that frequency.
\end{theorem}
\noindent\textbf{Proof:} take two arbitrary sinusoids $A\cos\left(\omega t + \alpha\right)$ and $B\cos\left(\omega t + \beta\right)$ at the frequency $\omega$. Then

\begin{align}
	S(t) &= A\cos\left(\omega t + \alpha\right) + B\cos\left(\omega t + \beta\right) = \nonumber\\[3mm]
%
	&= A\left[\cos\left(\omega t\right)\cos\left(\alpha\right) - \sin\left(\omega t\right)\sin\left(\alpha\right)\right] + B\left[\cos\left(\omega t\right)\cos\left(\beta\right) - \sin\left(\omega t\right)\sin\left(\beta\right)\right] = \nonumber\\[3mm]
%
	&= \cos\left(\omega t\right)\left[A\cos\left(\alpha\right) + B\cos\left(\beta\right)\right] - \sin\left(\omega t\right)\left[A\sin\left(\alpha\right) + B\sin\left(\beta\right)\right]
\end{align}

	\noindent now let $C \geq 0$ and $\phi$ that satisfy

\begin{align}
	C &= \sqrt{\left[A\cos\left(\alpha\right) + B\cos\left(\beta\right)\right]^2 + \left[A\sin\left(\alpha\right) + B\sin\left(\beta\right)\right]^2} = \\[3mm] &= \sqrt{\raisebox{3mm}[1mm][1mm]{} A^2 + B^2 + 2AB\cos\left(\alpha - \beta\right)}, \label{eq:sinesum_c} \\[3mm]
	& \left\{\begin{array}{l} C\sin\left(\phi\right) = A\sin\left(\alpha\right) + B\sin\left(\beta\right) \\[3mm]
				  C\cos\left(\phi\right) = A\cos\left(\alpha\right) + B\cos\left(\beta\right)\end{array}\right. \label{eq:sinesum_phi}
\end{align}

	\noindent and noting that $\phi = \pm \pi/2$ if the cosine expression is null. Then

\begin{equation} S(t) = C\left[\cos\left(\omega t\right)\cos\left(\phi\right) - \sin\left(\omega t\right)\sin\left(\phi\right)\right] = C\cos\left(\omega t + \phi\right) \label{eq:sinesum_final} \end{equation}

\hfill$\blacksquare$
\vspace{5mm}
\hrule
\vspace{5mm}
% >>>

	As shown by the proof of the theorem, the algebra of sinusoids is contrived and worksome. Consider the proposition: if we represent a sinusoid $x(t) = K\cos\left(\omega t + \phi\right)$ as the point $X = Ke^{j\phi}$ in the complex plane, as in Figure \ref{fig:static_phasor_representation}, then adding the complex numbers is considerably simpler than sinusoids. The figure shows the number $X$ and the graph of $x(t)$; if the number $X$ is rotated by a quantity $\omega t$ (that is, multiplied by $e^{j\omega t}$), then $x(t)$ is the real projection of the the rotated $Xe^{j\omega t}$.

% STATIC PHASOR DIAGRAM <<<
\begin{figure}[htb]
\centering
	\begin{tikzpicture}[scale=2,>={Stealth[inset=0mm,length=1.5mm,angle'=50]}]
		\draw [fill=none,gray, thick] (0,0) circle (10 mm) node [gray] {};
		\draw [->, thick, black!30] (   -20mm,  0   ) -- (   20mm,  0   );
		\draw [->, thick, black!50] (      0, -15mm ) -- (   0   ,  15mm);

		\draw [->, thick, black] (0,0) -- (14mm,0) coordinate(realvec);

		\node (realveclabel) at ([shift=({0,-2mm})]realvec) {$R = 1e^{j0}$};

		\node [black!50] (reAxisLabel) at (22mm,0) {Re};
		\node [black!50] (imAxisLabel) at (0,17mm) {Im};

		\node [label={[label distance=0.1mm]30:$X = Ke^{j\phi}$}] (X) at ({10mm*cos(30)},{10mm*sin(30)}) {};
		\draw [->,thick] (0,0) -- (X.center);
		\draw [->,stewartblue,thick] ({8mm*cos(0)},{8mm*sin(0)}) arc[start angle=0, end angle = 28, radius = 8mm];
		\draw [->,stewartblue,thick] ({8mm*cos(290)},{8mm*sin(290)}) arc[start angle=290, end angle = 318, radius = 8mm];

		\node [color=stewartblue] (philabel) at ({6.5mm*cos(13)},{6.5mm*sin(13)}) {$\phi$};
		\node [color=stewartblue] (philabelrotated) at ({6.5mm*cos(304)},{6.5mm*sin(304)}) {$\phi$};

		\node (rotX) at ({cos(320)},{sin(320)}) {};
		\node (XomegatLabel) at ({13mm*cos(330)},{13mm*sin(330)}) {$Xe^{j\omega t}$};
		\draw [->,thick] (0,0) -- (rotX.center);

		\node (rotRe) at ({13mm*cos(290)},{13mm*sin(290)}) {};
		\node (RomegatLabel) at ({14mm*cos(290)},{14mm*sin(290)}) {$Re^{j\omega t}$};
		\draw [->,thick] (0,0) -- (rotRe.center);

		%\draw [->,gray,thick] ({6mm*cos(25)},{6mm*sin(25)}) arc[start angle=25, end angle = 63, radius = 6mm];
		\node [green!50!black] (omegat) at ({6mm*cos(135)},{6mm*sin(135)}) {$\omega t$};

		\draw [-{Stealth[inset=0mm,length=3mm,angle'=50]},green!50!black, line width = 1mm] (4mm,0) arc[start angle=0, end angle = 287, radius = 4mm];
		
		% SINEWAVE PLOT AXES
		\draw [->, gray, thick]  (   -15mm, -20mm  ) -- (   15mm,  -20mm  );
		\draw [->, gray, thick]  (      0,  -20mm  ) -- (    0  ,  -50mm  );

		\node[gray] (axistlabel) at ( 0mm,-52mm) {$t$};

		\begin{axis}[color=stewartblue, at={(-12.5mm,-52mm)}, rotate=-90, width=35mm, height=25mm, scale only axis, yticklabel=\empty, xticklabel=\empty, axis line style={draw=none}, tick style={draw=none}]
			\addplot[domain=30:325, smooth, samples=100] {cos(x)};
			\addplot[dashed, domain=325:720, smooth, samples=100] {cos(x)};
		\end{axis}

		\node (rotXaxis) at ([shift=({0,-25.7mm})]rotX.center) {};
		\draw [color=stewartblue,fill] (rotXaxis) circle (0.5mm);

		\draw[dashed,color=stewartblue, thick] (rotX) -- (rotXaxis) ;

		\node [stewartblue] (xsignal) at (25mm, -40mm) {$x(t)  = K\cos\left(\omega t + \phi\right)$};

	\end{tikzpicture}
	\caption{Sinusoidal signal as the real projection of a rotated stationary phasor.}
	\label{fig:static_phasor_representation}
\end{figure} %>>>

	Confusingly, the single-dimensional $x(t)$ has become a two-dimensional quantity. The key concept is that if the frequency $\omega$ is fixed, a sinusoidal signal — albeit real — needs two dimensions to be described: the phase $\phi$ and the magnitude $K$. Conversely, in order to reconstruct the real sinusoid, two quantities are required. As a phase reference (a phasor with zero phase) is adopted, because the cosine angle $\omega t + \phi$ grows linearly with time, the angle difference between $x$ and the reference is kept constant at all times -- hence the angle of $x$ at $t = 0$ is enough to describe $x$. Therefore, there is a bijection between the pair $\left(K,\phi\right)\in\mathbb{R}^2$ and $x_\infty(t) = K\cos\left(\omega t + \phi\right)\in\left[\mathbb{R}\to\mathbb{R}\right]$. Because $\mathbb{C}$ is homeomorphic to $\mathbb{R}^2$; this allows for representing $x_\infty(t) = K\cos\left(\omega t + \phi\right)$ as its complexification $X = Ke^{j\phi}$.
\par
	Figure \ref{fig:complexification_process} shows this process: a function $K\cos\left(\omega t + \phi\right)$ is picked from the space of real signals $\left[\mathbb{R}\to\mathbb{R}\right]$ (in green) and, by taking its value at $t = 0$, it is associated with a pair $\left(K,\phi\right)$ in $\mathbb{R}^2$ (in yellow). Because there is an isomorphism between $\mathbb{R}^2$ and $\mathbb{C}$, it is easy to associate $\left(K,\phi\right)$ to a complex $Ke^{j\phi}$ in $\mathbb{C}$ (in blue). Therefore, the tandem process of representing the real funcion by the complex number, called \textit{complexification}, is justified. 

% COMPLEXIFICATION <<<
\begin{figure}[htb]
\centering
	\begin{tikzpicture}[scale=1.5,>={Stealth[inset=0mm,length=2.5mm,angle'=50]}]
		\draw [fill=none,gray, thick,stewartgreen,fill=stewartgreen!30,line width=0.5mm]   ({20mm*cos(150)},{20mm*sin(150)}) circle(10mm) node[stewartgreen] (kcos) {$K\cos\left(\omega t + \phi\right)$};
		\draw [fill=none,gray, thick,stewartyellow,fill=stewartyellow!30,line width=0.5mm] ({20mm*cos( 30)},{20mm*sin( 30)}) circle( 7mm)  node[stewartyellow] (kphi) {$\left(K,\phi\right)$};
		\draw [fill=none,gray, thick,stewartblue, fill=stewartblue!30,line width=0.5mm]    ({20mm*cos(260)},{20mm*sin(260)}) circle( 7mm)  node[stewartblue] (kphicomp) {$Ke^{j\phi}$};

		\draw [->,stewartblue   ,testfading={0.25mm}{stewartyellow}{stewartblue}] ([shift=({{7mm*cos( -95)},{7mm*sin( -95)}})]kphi.center) to [bend left]  ([shift=({{7mm*cos( 20)},{7mm*sin( 20)}})]kphicomp.center); 
		\draw [->,stewartblue   ,testfading={0.25mm}{stewartgreen}{stewartblue}]  ([shift=({{10mm*cos(250)},{10mm*sin(250)}})]kcos.center) to [bend right] ([shift=({{7mm*cos(160)},{7mm*sin(160)}})]kphicomp.center);
		\draw [->,stewartyellow ,testfading={0.25mm}{stewartgreen}{stewartyellow}]([shift=({{10mm*cos( 45)},{10mm*sin( 45)}})]kcos.center) to [bend left]  ([shift=({{7mm*cos(150)},{7mm*sin(150)}})]kphi.center);

		\node (label1) at (-1mm,21mm) {$t = 0$};
		\node (label2) at (22mm,-9mm) {Isomorphism};
		\node (label3) at (-29mm,-9mm) {``Complexification''};
	\end{tikzpicture}
	\caption{The process of \textit{complexification} of a sinusoid $K\cos\left(\omega t  + \phi\right)$ into a complex number $Ke^{j\phi}$.}
	\label{fig:complexification_process}
\end{figure} %>>>

	Therefore, one can define a Phasor representation based on these results as a bijection between the signal $x(t)$ and its phasorial counterpart $X$.

\begin{definition}[Static Phasor Operator (SPO)]\label{def:static_phasor_transform} %<<<
Let $x(t) = A\cos\left(\omega t + \phi\right)$, where $A,\omega$ and $\phi$ are constant real numbers. Then there is a bijection $\mathbf{p_S}\left[x\right]$, which we call Static Phasor Operator, defined as 

\begin{equation}
	\mathbf{p_S}\left[x\right] \vcentcolon \left\{\begin{array}{rcl}
	\left[\mathbb{R}\to \mathbb{R}\right] &\to& \mathbb{C} \\[3mm]
	x = A\cos\left(\omega t + \phi\right) &\mapsto& X = A e^{j\phi}
\end{array}\right.
\end{equation}

	The inverse operator is defined as

\begin{equation}
	\mathbf{p_S^{-1}}\left[X\right] \vcentcolon \left\{\begin{array}{rcl}
	\mathbb{C} &\to& \left[\mathbb{R}\to \mathbb{R}\right]  \\[3mm]
	X &\mapsto& \Re\left(Xe^{j\omega t}\right)
\end{array}\right.
\end{equation}
\end{definition} %>>>
\begin{definitionremark} The Static Phasor Operator relates a function to a complex number, thus being an operator. Therefore it is denoted with a lowercase notation $\mathbf{p_S}$. \end{definitionremark}
%-------------------------------------------------
\subsection{Operational properties of Static Phasors} %<<<2

	Operationally, the most obvious benefit of this operator is that it simplifies the algebra of sinusoids involved, as shown in theorem \ref{theo:ps_morphism}.

\begin{theorem}[The Static Phasor Operator and its inverse are linear morphisms] \label{theo:ps_morphism}%<<<
	The Static Phasor Operator maintains the summation operation of sinusoids, that is, if $X = \mathbf{p_S}\left[x\right]$ and $Y = \mathbf{p_S}\left[y\right]$, then $X + Y = \mathbf{p_S}\left[x + y\right]$. At the same time, if $x(t) = \mathbf{p_S^{-1}}\left[X\right]$ and $y(t) = \mathbf{p_S^{-1}}\left[Y\right]$, then $x(t) + y(t) = \mathbf{p_S^{-1}}\left[X + Y\right]$.
\end{theorem}
\noindent\textbf{Proof.} Take the sinusoids from theorem \ref{theo:closed}. The first sinusoid is related to $Ae^{j\alpha}$, the second to $Be^{j\beta}$ and

\begin{equation} Ae^{j\alpha} + Be^{j\beta} = \left[A\cos\left(\alpha\right) + B\cos\left(\beta\right)\right] + j\left[A\sin\left(\alpha\right) + B\sin\left(\beta\right)\right]\end{equation}

	\noindent and note that the absolute value of this number is $C$ as in \eqref{eq:sinesum_c} and its argument is $\phi$ as in \eqref{eq:sinesum_phi}, that is,

\begin{equation} Ae^{j\alpha} + Be^{j\beta} = Ce^{j\phi}\end{equation}

	\noindent meaning a notably simpler process than summing sinusoids. For the inverse, if $x(t) = \mathbf{p_S^{-1}}\left[X\right] = \text{Re}\left(Xe^{j\omega t}\right)$ and $y(t) = \mathbf{p_S^{-1}}\left[Y\right] = \text{Re}\left(Ye^{j\omega t}\right)$ then one uses the linearity of the real part and

\small\begin{equation} x(t) + y(t) = \text{Re}\left(Xe^{j\omega t}\right) + \text{Re}\left(Ye^{j\omega t}\right) = \text{Re}\left(Xe^{j\omega t} + Xe^{j\omega t}\right) = \text{Re}\left[\left(X + Y\right)e^{j\omega t}\right] = \mathbf{p_S^{-1}}\left[X + Y\right]\end{equation}\normalsize

\hfill$\blacksquare$
\vspace{5mm}
\hrule
\vspace{5mm}
% >>>

	As beforementioned, the first engineer to notice the direct bijection between a sinusoid to a complex number as a useful result in engineering was Steinmetz, who noticed that the combinations (addition or subtraction) of sinewaves as solutions to the Differential Equations of electrical networks was vastly superior to directly solving the differential equations of the system or using trigonometric formulas to add sinusoids:

\vspace{3mm}
\begin{quotation}
\textit{``The sine-wave is completely determined and characterized by intensity and phase. It is obvious that the phase is of interest only as a difference of phase, where several waves of different phases are under consideration. [...] The representation of sine-waves by their rectangular components is very useful in so far as it avoids the use of trigonometric functions. To combine sinewaves, we have simply to add or subtract their rectangular components.''} \hfill\pcite{Steinmetz1893} 
\end{quotation}
\vspace{3mm}

	When theory is concerned, in general \pcite{desoerBasicCircuitTheory1987,scottElementsLinearCircuits1965}, the operational properties of Static Phasors are taught from a strictly sinusoidal point of view as in theorem \ref{theo:ps_morphism}, which makes it clear that for any two sinusoids $x(t) = M\cos\left(\omega t + \alpha\right)$  and $y(t) = N\cos\left(\omega t + \beta\right)$ their linear combination $z(t) = x(t) + cy(t)$ yields an equivalent phasor $Z = X + cY$ ($X$ and $Y$ being the phasors of $x(t)$ and $y(t)$) for any real $c$. Further, it is also simple to see that the properties of derivatives are obeyed.

	In a deeper and more complete setting, the concept of a Static Phasor is born from steady-state solutions of sinusoidally-forced linear ODEs, that is, these proofs also prove that the linearity of the static phasor transform also allows to obtain the combination of responses for linear circuits, as shown in theorem \ref{theo:spo_linear}.

\begin{theorem}[The Static Phasor Operator is linear]\label{theo:spo_linear} %<<<
	Let $\left(\alpha_k\right)_{k=0}^n$ define a Hurwitz-stable system and $x_\infty(t)$ and $y_\infty(t)$ be the steady-state solutions to two different sinusoids at the same frequency, that is,

\begin{equation} \left\{ \begin{array}{l} \sum\limits_{k=0}^n \alpha_k x^{(k)}(t) - A\cos\left(\omega t + \alpha\right) = 0 \\[3mm] \sum\limits_{k=0}^n \alpha_k y^{(k)}(t) - B\cos\left(\omega t + \beta\right) = 0 \end{array}\right. , \label{eq:input_linearity}\end{equation}

	\noindent such that

\begin{equation} \lim\limits_{t\to\infty} \left[x(t) - x_\infty(t)\right] = 0,\ \lim\limits_{t\to\infty} \left[y(t) - y_\infty(t)\right] = 0 \end{equation}

	\noindent and the existence of $x_\infty$ and $y_\infty$ is guaranteed by theorem \ref{theo:phasors_solutions}. Denote $\mathbf{p_S}\left[x_\infty\right] = X,\ \mathbf{p_S}\left[y_\infty\right] = Y$. Now consider the response $z(t)$ of the combined forcing, that is,

\begin{equation} \sum\limits_{k=0}^n \alpha_k z^{(k)}(t) - \left[A\cos\left(\omega t + \alpha\right) + c B\cos\left(\omega t + \beta\right)\right] = 0,\ c\in\mathbb{C}, \end{equation}

	\noindent then 

\begin{equation} \lim\limits_{t\to\infty} \left[z(t) - \left(x_\infty(t) + cy_\infty(t)\right)\right] = 0 \end{equation}

	\noindent so the phasor of $z_\infty(t)$ is $Z = X + cY$. Succintly, 

\begin{equation} \mathbf{p_S}\left[x_\infty(t) + c y_\infty(t)\right] = \mathbf{p_S}\left[x_\infty\right] + c \mathbf{p_S}\left[y_\infty\right],\ \forall c \in\mathbb{C}. \end{equation}
\end{theorem}
\noindent\textbf{Proof:} a direct consequence from theorems \ref{theo:closed}, \ref{theo:ps_morphism} and the fact that the differential equation is LTI. Add the first equation of \eqref{eq:input_linearity} to the second one \eqref{eq:input_linearity} multiplied by some complex $c$, and using the linearity of derivatives:

\begin{equation} \sum\limits_{k=0}^n \alpha_k \left[x + cy\right]^{(k)}(t) - \left[ A\cos\left(\omega t + \alpha\right) + c B\cos\left(\omega t + \beta\right)\right] . \label{eq:input_linearity_2}\end{equation}

	Now let $z = x + cy$ and

\begin{equation} \sum\limits_{k=0}^n \alpha_k z^{(k)}(t) - \left[ A\cos\left(\omega t + \alpha\right) + c B\cos\left(\omega t + \beta\right)\right] , \label{eq:input_linearity_3}\end{equation}

	\noindent and due to the linearity of both ODEs, the solution $z(t)$ of \eqref{eq:input_linearity_3} is equal to $x(t) + cy(t)$ with $x(t)$ the solution of the first ODE of \eqref{eq:input_linearity}  and $y(t)$ the solution of the second equation. This means

\begin{align}
	\left\lvert z(t) - \left[x_\infty(t) + cy_\infty(t)\right]\right\rvert &= \left\lvert x(t) + cy(t) - \left[x_\infty(t) + cy_\infty(t)\right]\right\rvert = \nonumber\\[3mm] &= \left\lvert x(t) - x_\infty(t) + c\left[y(t) - y_\infty(t)\right]\right\rvert \leq \nonumber\\[3mm] &\leq \left\lvert x(t) - x_\infty(t)\right\rvert + \left\lvert c \right\rvert\left\lvert y(t) - y_\infty(t)\right\rvert
\end{align}

	\noindent and, by hypothesis, the right side vanishes at infinity, and it is immediate from this that

\begin{equation} \lim_{t\to\infty} \left[z(t) - \left(x_\infty(t) + cy_\infty(t)\right)\right] = 0 .\end{equation}

	Therefore, denote $z_\infty(t) = x_\infty(t) + cy_\infty(t)$ and by the linearity of $\mathbf{p_S}$,

\begin{equation} \mathbf{p_S}\left[x_\infty(t) + cy_\infty(t)\right] = \mathbf{p_S}\left[x_\infty\right] + c\mathbf{p_S}\left[cy_\infty\right] = X + cY.\end{equation}


\hfill$\blacksquare$
\vspace{5mm}
\hrule
\vspace{5mm}
% >>>

% STATIC PHASOR DIAGRAM <<<
\begin{figure}[htb]
\centering
	\begin{tikzpicture}[scale=1.5,>={Stealth[inset=0mm,length=1.5mm,angle'=50]}]
		\draw [->, thick, black!50] (   -30mm,  0   ) -- (   30mm,  0   );
		\draw [->, thick, black!50] (      0, -5mm ) -- (   0   ,  15mm);

		\node [gray] (imlabel) at ( 0mm, 17mm) {Im};
		\node [gray] (relabel) at (32mm,  0mm) {Re};

		\node (O) at (0,0)       {};
		\node (X) at (9mm,12mm)  {};
		\node [stewartblue] (Xlabel) at ([shift=({0,2mm})]X)  {$X$};
		\node (Y) at (7mm,2mm)   {};
		\node (2Y) at (14mm,4mm) {};
		\node [stewartgreen] (Ylabel) at ([shift=({-2mm,-3.5mm})]Y)  {$Y$};
		\node (2Y) at (14mm,4mm) {};
		\node [stewartyellow] (2Ylabel) at ([shift=({2mm,-1mm})]2Y)  {$2Y$};

		\node (SUM) at (23mm,16mm) {};
		\node [stewartpurple] (SUMlabel) at ([shift=({2mm, 2mm})]SUM)  {$X + 2Y$};

		\draw [->, thick, stewartblue]  (O.center) -- (X.center);
		\draw [->, thick, stewartyellow]  (O.center) -- (2Y.center);
		\draw [->, thick, stewartgreen]  (O.center) -- (Y.center);
		\draw [thick, dashed, stewartblue!50]  (2Y.center) -- (SUM.center);
		\draw [thick, dashed, stewartgreen!50] (X.center) -- (SUM.center);

		\draw [->, thick, stewartpurple] (O.center) -- (SUM.center);

		\draw [->, gray, thick]  (   -30mm, -10mm  ) node (beginXaxis) {} -- (   30mm,  -10mm  ) node (endXaxis) {};
		\draw [->, gray, thick]  (      0,  -10mm  ) node (beginYaxis) {} -- (    0  ,  -70mm  ) node (endYaxis) {} ;

		\node [gray] (tlabel) at ( 0mm,-72mm) {$t$};

		\draw [thick,dashed,stewartblue!50]     (X) -- (beginXaxis -|   X);
		\draw [thick,dashed,stewartgreen!50]    (Y) -- (beginXaxis -|   Y);
		\draw [thick,dashed,stewartyellow!50]   (2Y) -- (beginXaxis -|  2Y);
		\draw [thick,dashed,stewartpurple!50] (SUM) -- (beginXaxis -| SUM);

		\begin{axis}[at={(-33.5mm,-71.5mm)}, rotate=-90, width=67mm, height=67mm, scale only axis, yticklabel=\empty, xticklabel=\empty, axis line style={draw=none}, tick style={draw=none}]
			\addplot[color = stewartblue, domain=0:720, smooth, samples=100] {15mm*cos(x+53.13)};
			\addplot[color = stewartgreen, domain=0:720, smooth, samples=100] {7.28mm*cos(x+15)};
			\addplot[color = stewartyellow, domain=0:720, smooth, samples=100] {2*7.28mm*cos(x+15)};
			\addplot[color = stewartpurple, domain=0:720, smooth, samples=100] {28.02mm*cos(x+34.82)};
		\end{axis}

		\node [label={[color=stewartblue]right:$x(t)$}] at           (33mm, -40mm) {};
		\node [label={[color=stewartgreen]right:$y(t)$}] at          (33mm, -43mm) {};
		\node [label={[color=stewartyellow]right:$2y(t)$}] at        (33mm, -46mm) {};
		\node [label={[color=stewartpurple]right:$x(t) + 2y(t)$}] at (33mm, -49mm) {};

	\end{tikzpicture}
	\caption{Static Phasor Operator linearity schematic.}
	\label{fig:spo_linearity}
\end{figure} %>>>

	Figure \ref{fig:spo_linearity} shows a schematization of the linearity property. In that figure, two vectors $X$ (in blue) and $Y$ (in green) are operated and the vector $X + 2Y$, in purple, is generated. Below, the real projection of these vectors are shown as sinusoids, showing how the sinusoidal signals and the complex vectors are linearly operated.

	Meanwhile, theorem \ref{theo:spo_der} proves a bigger result: that the SPO is not only linear, but it also transforms derivatives into algebraic equations.

\begin{theorem}[Static Phasor Operator of derivatives]\label{theo:spo_der} %<<<
	Let $\left(\alpha_k\right)_{k=0}^n$ define a Hurwitz-stable system and $x(t)$ its response to a particular sinusoid at a frequency $\omega$:

\begin{equation} \sum\limits_{k=0}^n \alpha_k x^{(k)}(t) - A\cos\left(\omega t + \alpha\right) = 0 , \label{eq:spo_diff_1}\end{equation}

	\noindent such that $\mathbf{p_S}\left[x_\infty\right] = X$. Consider $y_i(t) = x^{(i)}(t)$; then:

\begin{enumerate}
	\item $y_i(t)$ is the solution to $\sum\limits_{k=0}^n \alpha_k y_i^{(k)}(t) - \left(\omega\right)^i A\cos\left(\omega t + \alpha + \dfrac{i\pi}{2}\right) = 0$;

	\item Further, $y_i(t)$ has a stable sinusoidal steady-state solution $y_{i,\infty}(t)$ at the frequency $\omega$;

	\item Finally, $\mathbf{p_S}\left[y_{i,\infty}\right] = \left(j\omega\right)^i X$, which is to say $\mathbf{p_S} \circ \mathbf{D^i} = \left(j\omega\right)^i \mathbf{p_S}$ or

\begin{equation} \mathbf{p_S}\left[ \dfrac{d^ix_\infty(t)}{dt^i}\right] = \left(j\omega\right)^i \mathbf{p_S}\left[ x_\infty \right] .\end{equation}
\end{enumerate}
\end{theorem} 
\noindent\textbf{Proof:} take the system and consider \eqref{eq:spo_diff_1}. Then $x(t)$ has a sinusoidal steady-state solution $x_\infty = M_x \cos\left(\omega t + \phi_x\right)$. Then write $z = x'(t)$ and differentiate \eqref{eq:spo_diff_1} with respect to time:

\begin{equation} \sum\limits_{k=0}^n \alpha_k z^{(k)}(t) - \left[- A\omega\sin\left(\omega t + \alpha\right)\right] = 0 \end{equation}

	\noindent writing the sine as a de-phased cosine:

\begin{equation} \sum\limits_{k=0}^n \alpha_k z^{(k)}(t) - A\omega\cos\left(\omega t + \alpha + \dfrac{\pi}{2}\right) = 0. \label{eq:spo_diff_2}\end{equation}

	Because the equation is linear, the phasor $Z$ corresponding to $z(t)$ is scaled at the same rate that the input of \eqref{eq:spo_diff_2} is scaled with respec to the input of \eqref{eq:spo_diff_1}, as per \eqref{eq:lti_scaling}. Therefore $z(t)$ has modulus $\omega M$. Because this equation is also time-invariant, a delay in the excitation causes the same delay in the solution, that is, the phase of $z$ is the same phase as $x(t)$ but shifted $\pi/2$, as per \eqref{eq:lti_delay}. Hence  

\begin{equation} z(t) = \omega x\left(t + \dfrac{\pi}{2}\right) \Rightarrow z_\infty = \omega x_\infty \left(t + \dfrac{\pi}{2}\right) = \omega M_x \cos\left(\omega t + \phi_x + \pi/2\right)\end{equation}

	\noindent and taking the phasor operator,

\begin{equation} \mathbf{p_S}\left[z_\infty\right] = \omega M_x e^{j\left(\phi_x + \frac{\pi}{2}\right)} = \omega Xe^{j\frac{\pi}{2}} = j\omega X . \end{equation}

	For the i-th derivative, we can iterate this process using induction. Alternatively, knowing it is true for $i=1$ and noting that

\begin{equation} \dfrac{d^i}{dt^i}\left[A\cos\left(\omega t + \alpha\right)\right] = A\omega^i\cos\left(\omega t + \alpha + \dfrac{i\pi}{2}\right) .\end{equation}

	\noindent then, by denoting $x^{(i)} = z_i(t)$ and differentiating \eqref{eq:spo_diff_1} $i$ times we obtain

\begin{equation} \sum\limits_{k=0}^n \alpha_k z_i^{(k)}(t) - A\omega^i\cos\left(\omega t + \alpha + \dfrac{i\pi}{2}\right) = 0 , \label{eq:spo_diff_i}\end{equation}

	\noindent yielding

\begin{equation} \mathbf{p_S}\left[z_{i,\infty}\right] = \omega^i M_x e^{j\left(\phi_x + \frac{i\pi}{2}\right)} = \omega^i Xe^{j\frac{i\pi}{2}} = \left(j\omega\right)^i X . \end{equation}

\hfill$\blacksquare$
\vspace{5mm}
\hrule
\vspace{5mm}
% >>>

	----- Finally, proving the transformation of integral property is the same process but with a small caveat: adopting $z = \int x dt$ needs the adoption of the bottom integration limit in order to remove the integration constant. This integration limit is generally defined as $-\infty$, as long as the integral of $x$ in $\left(-\infty,t\right]$ converges for all time instants $t$; special considerations can be made when convergence is not guaranteed or $x(t)$ has some discontinuity, which can be the case because signals are generally defined starting from time $t = 0$. In this case the Cauchy principal value of the integral can be used or a different integration limit altogether and the results will largely remain.

	These properties greatly simplify the solution of linear differential equations of linear circuits; for instance, one can re-prove theorem \ref{theo:phasors_solutions}.

\begin{theorem}[Phasors as solutions to sinusoidally-forced LTI ODEs (reproof)]\label{theo:phasors_solutions_reproof} %<<<
Consider the linear n-th order LTI Ordinary Differential Equation

\begin{equation} \sum\limits_{k=0}^n \alpha_k x^{(k)}(t) - M\cos\left(\omega t\right) = 0,\label{eq:linear_ode_phasor_solution_reproof_1}\end{equation}

	\noindent where $y^{(k)}$ represents the k-th derivative of $y$ with $y^{(0)} \equiv y$; the $\alpha_k$ are real numbers with $\alpha_n \neq 0$ such that \eqref{eq:linear_ode_phasor_solution_reproof_1} is Hurwitz, and $M,\omega$ are positive real numbers. Then the globally exponentially stable steady-state solution of \eqref{eq:linear_ode_phasor_solution_1} is given by

\begin{equation} x_s(t) = K\cos\left(\omega t + \phi\right)\end{equation}

	\noindent where 

\begin{gather}
	K = \dfrac{M}{\left(\alpha_0 - \alpha_2\omega^2 + ...\right) + j\left(\alpha_1 - \alpha_3\omega^3 + ...\right)} \\[5mm]
	\tan\left(\phi\right) = \dfrac{\left(\alpha_1 - \alpha_3\omega^3 + ...\right)}{\left(\alpha_0 - \alpha_2\omega^2 + ...\right)}
\end{gather}

\end{theorem}
\textbf{Proof: } due to Hurwitz stability this system admits an exponentially stable sinusoidal solution. Using theorems \ref{theo:spo_linear} and \ref{theo:spo_der}, \eqref{eq:linear_ode_phasor_solution_reproof_1} is transformed into

\begin{equation} \sum\limits_{k=0}^n \alpha_k \left(j\omega\right)^k X - M = 0 \label{eq:linear_ode_phasor_solution_reproof_2} \end{equation}

	\noindent and solving this equation yields

\begin{equation} X\left[\sum\limits_{k=0}^n \alpha_k \left(j\omega\right)^k \right] - M = 0 \Rightarrow X = \dfrac{M}{\left[\sum\limits_{k=0}^n \alpha_k \left(j\omega\right)^k \right]} = \dfrac{M}{\left(\alpha_0 - \alpha_2\omega^2 + ...\right) + j\left(\alpha_1 - \alpha_3\omega^3 + ...\right)} .\end{equation}

	Therefore

\begin{gather}
	\left\lvert X\right\rvert = \dfrac{M}{\left\lvert\left(\alpha_0 - \alpha_2\omega^2 + ...\right) + j\left(\alpha_1 - \alpha_3\omega^3 + ...\right)\right\rvert} = \dfrac{M}{\sqrt{\left(\alpha_0 - \alpha_2\omega^2 + ...\right)^2 + \left(\alpha_1 - \alpha_3\omega^3 + ...\right)^2}} \\[3mm]
	\tan\left(\text{arg}\left(X\right)\right) = \dfrac{\left(\alpha_1 - \alpha_3\omega^3 + ...\right)}{\left(\alpha_0 - \alpha_2\omega^2 + ...\right)}
\end{gather}
\hfill$\blacksquare$
\vspace{5mm}
\hrule
\vspace{5mm}
% >>>

	Remarkably, the reproof \ref{theo:phasors_solutions_reproof} of theorem \ref{theo:phasors_solutions} is strikingly simpler. This shows that the SPO is not only a great way to simplify the algebra of sinusoids, but also a great way to simplify the solution of differential equations by transforming differential operators into algebraic ones, as shown by transforming \eqref{eq:linear_ode_phasor_solution_reproof_1} into \eqref{eq:linear_ode_phasor_solution_reproof_2}.

%-------------------------------------------------
\section{Impedances and Kirchoff's Laws in the Phasor domain} %<<<1

	The last two properties are especially useful in the development of phasorial electrical analysis theory for their capability of easening the solution of differential equations. More specifically, these properties allow for the definitions of capacitive conductance and inductive impedances as algebraic quantities; indeed, consider a voltage $v = m_v\cos\left(\omega t + \phi_v\right)$ over a capacitor of value $C$ and $V$ the corresponding phasor of $v(t)$; then

\begin{equation} i(t) = C\dfrac{dv(t)}{dt} = -C\omega m_v \sin\left(\omega t + \phi_v\right) = C\omega m_v\cos\left(\omega t + \phi_v + \dfrac{\pi}{2}\right) \end{equation}

	Therefore the phasor of $i$ can be calculated as

\begin{equation} I = C\omega m_ve^{\displaystyle j\left(\phi_v + \dfrac{\pi}{2}\right)} = V C\omega e^{\displaystyle j\left(\dfrac{\pi}{2}\right)} = V \left(j\omega C\right) \Leftrightarrow \dfrac{V}{I} = \dfrac{1}{j\omega C}. \end{equation}

	Now consider a current $i = m_i\cos\left(\omega t + \phi_i\right)$ through an inductor $L$; then

\begin{equation} v(t) = L\dfrac{di(t)}{dt} = -L\omega m_i \sin\left(\omega t + \phi_i\right) = L\omega m_i\cos\left(\omega t + \phi_i + \dfrac{\pi}{2}\right) \end{equation}

	Therefore the phasor of $v$ can be calculated as

\begin{equation} V = L\omega j\left(\phi_i + \dfrac{\pi}{2}\right) = I \omega L e^{\displaystyle j\left(\dfrac{\pi}{2}\right)} = I \left(j\omega I\right) \Leftrightarrow \dfrac{V}{I} = j\omega L . \end{equation}

	These identities are then applied to an electrical grid with the assumption that the excitations (machine and inverter dynamics) are much slower than the grid dynamics, leading to the fact that the exponential transient behaviors of the grid dissipate rapidly and allowing to consider the grid as a set of algebraic complex equations. The benefit of representing sinusoidal waves as complex numbers is that complex algebra is much simpler than the algebra of sinusoidal signals which requires contrived formulas to be undertaken. Instead, two-dimensional vectorial algebra is used in the complex space, and the complex number pertaining to voltages and currents are obtained; the bijection $p_S$ as defined in \ref{def:static_phasor_transform} combined with theorem \ref{theo:phasors_solutions} guarantee that the complex numbers obtained are bijective representations of the exponentially stable steady-state sinusoidal solutions to the electrical grid differential equations.

	Furthermore, using the linearity of the Phasor Operator one can prove the phasorial counterparts to Kirchoff's Laws.

\begin{theorem}[Kirchoff's Current Law in the Phasor domain] \label{theo:kirchoff_current_phasor}
Let $i_p(t)$, $p = 1,...,q$ be the sinusoidal currents of a certain network meeting at a node, $I_p$ their phasors. Then

\begin{equation} \sum\limits_{p=1}^q I_p = 0 \end{equation}

\end{theorem}
\noindent \textbf{Proof.} By Kirchoff's Current Law in time domain, $\sum i_p(t) = 0$. Applying the phasor operator and using its linearity yields $\sum I_p = 0$. \hfill$\blacksquare$
	
\begin{theorem}[Kirchoff's Voltage Law in the Phasor domain] \label{theo:kirchoff_voltage_phasor}
Let $v_p(t)$, $p = 1,...,q$ be the sinusoidal voltages of a certain network around a certain closed loop, $V_p$ their phasors. Then

\begin{equation} \sum\limits_{p=1}^q V_p = 0 \end{equation}

\end{theorem}
\noindent \textbf{Proof:} akin to theorem \ref{theo:kirchoff_current_phasor}. \hfill$\blacksquare$

	It can be shown \pcite{scottElementsLinearCircuits1965,desoerBasicCircuitTheory1987} that one can also prove phasorial equivalents of the Superposition Theorem and the Thèvenin-Norton Theorems; these will not be proven now, but later in the broader context of Dynamic Phasors which generalize Classical Phasors.

	These results process makes AC network analysis much easier than, for instance, directly solving their time differential equations. This process of ``solving'' an AC network is as follows:

\begin{enumerate}
	\item Substitute inductances as impedances $j\omega L$, capacitances as conductances $j\omega C$ and resistances as impedances $R$;
	\item Substitute voltage and current sources by their phasor equivalents;
	\item Write the complex algebraic equations of the network;
	\item In the frequency domain, solve the complex algebraic equations of the node voltages and branch currents obtaining their equivalent phasors;
	\item Apply the inverse transform to obtain their equivalent steady-state time responses.
\end{enumerate}

\begin{example}[Phasorial analysis of a second-order circuit] \label{example:phasorial_analysis}%<<<

	Consider the second-order circuit of figure \ref{fig:nodeanalysis_example} with sinusoidal forcings $v_1 = V\cos\left(\omega t + \phi_v\right)$ and $i_1(t) = I\cos\left(\omega t + \phi_i\right)$, yielding the phasors $V_1 = Ve^{j\phi_v}$ and $I_1 = Ie^{j\phi_i}$. Then substituting the inductance by $j\omega L$ and the capacitance by $1/j\omega C$ one arrives at the phasorial version of the circuit, depicted in figure \ref{fig:nodeanalysis_example_phasorial}

% MODELLING EXAMPLE: RLC CIRCUIT <<<
\begin{figure}[htb!]
\centering
        \begin{tikzpicture}[american,scale=1,transform shape,line width=0.75, cute inductors, voltage shift = 1]
	\ctikzset{/tikz/circuitikz/voltage/distance from node=10mm}
		\draw (0,0)
			to[vsource,sources/scale=1.25,f<^=$I_{V1}$, v>=$V_1$,invert] (0,4)
			to[R,l=$R_1$,f>^=$I_{R1}$,v>=$V_{R1}$,-*] (4,4) 
			to[C,l=$\dfrac{1}{j\omega C_1}$,f>^=$I_{C1}$,v>=$V_{C1}$,-*] (4,0) 
			to[short] (0,0); 
		\draw (4,4)
			to[short]  (4,7)
			to[isource,sources/scale=1.25, l=$I_1$, v>=$V_{I1}$] (8,7)
			to[short]  (8,4);
		\draw (4,4)
			to[R,l=$R_2$,f>^=$I_{R2}$,v>=$V_{R2}$,-*] (8,4) 
			to[R,l=$R_3$,f>^=$I_{R3}$,v>=$V_{R3}$,-*] (8,0)
			to[short]  (4,0);
		\draw (8,4)
			to[L,l=$j\omega L_1$,f>^=$I_{L1}$,v>=$V_{L1}$] (12,4) 
			to[R,l=$R_4$,f>^=$I_{R4}$,v>=$V_{R4}$] (12,0) 
			to[short]  (8,0);
		% DRAWING VOLTAGE LOOPS
		\draw[rounded corners=10,loop, draw opacity=0.3,->, color=blue] (0.5,0.5) -- (0.5,3.5) -- (3.5,3.5) -- (3.5,0.5) -- (1,0.5) ;
		\draw[rounded corners=10,loop, draw opacity=0.3,->, color=red] (4.5,0.5) -- (4.5,3.5) -- (7.5,3.5) -- (7.5,0.5) -- (5,0.5) ;
		\draw[rounded corners=10,loop, draw opacity=0.3,->, color=green] (8.5,0.5) -- (8.5,3.5) -- (11.5,3.5) -- (11.5,0.5) -- (9,0.5) ;
		\draw[rounded corners=10,loop, draw opacity=0.3,->, color=stewartyellow] (4.5,4.5) -- (4.5,6.5) -- (7.5,6.5) -- (7.5,4.5) -- (5,4.5) ;
		% DRAWING NODE LABELS
		\node[shape=circle,draw,inner sep=1pt] at (  0,4.5) {$1$};
		\node[shape=circle,draw,inner sep=1pt] at (3.5,4.5) {$2$};
		\node[shape=circle,draw,inner sep=1pt] at (7.5,4.5) {$3$};
		\node[shape=circle,draw,inner sep=1pt] at ( 12,4.5) {$4$};
		\node[shape=circle,draw,inner sep=1pt] at ( 6,-0.5) {$5$};
		
		% DRAWING LOOP LABELS

		\node[color=blue] at (2,2) {$L1$} ;
		\node[color=red ] at (6,2) {$L2$} ;
		\node[color=stewartyellow] at (6,5.3) {$L3$} ;
		\node[color=green] at (10,2) {$L4$} ;
        \end{tikzpicture}
	\caption{Second-order circuit for node analysis example, in the phasorial domain.}
	\label{fig:nodeanalysis_example_phasorial}
\end{figure} %>>>

	First, start with the current laws: from the nodes,

\begin{equation} %<<<
	\left\{\begin{array}{l}
		(1):\ -I_{V1} - I_{R1} = 0 \\[3mm]
		(2):\  I_{R1} - I_{R2} - I_{C1} - I_1 = 0 \\[3mm]
		(3):\  I_{R2} - I_{R3} + I_1 - I_{L1} = 0 \\[3mm]
		(4):\  I_{L1} - I_{R4} = 0 \\[3mm] 
		(5):\  I_{V1} + I_{C1} + I_{R3} + I_{R4} = 0 
	\end{array}\right.
\end{equation} %>>>

	But since $I_{V1} = -I_{R1}$, eliminate the former:

\begin{equation} %<<<
	\left\{\begin{array}{l}
		 I_{R1} - I_{R2} - I_{C1} - I_1 = 0 \\[3mm]
		 I_{R2} - I_{R3} + I_1 - I_{L1} = 0 \\[3mm]
		 I_{L1} - I_{R4} = 0 \\[3mm] 
		 -I_{R1} + I_{C1} + I_{R3} + I_{R4} = 0 
	\end{array}\right.
\end{equation} %>>>

	In matrix form,

\begin{equation} %<<<
%
	\left[\begin{array}{ccccccc}
	-1 &  0 &  1 &-1 & 0 & 0\\[3mm]
	 0 & -1 &  0 & 1 &-1 & 0\\[3mm]
	 0 &  1 &  0 & 0 & 0 &-1\\[3mm]
	 1 &  0 & -1 & 0 & 1 & 1
	\end{array}\right]
%
	\left[\begin{array}{c}
		I_{C1} \\[3mm] I_{L1} \\[3mm] I_{R1} \\[3mm] I_{R2} \\[3mm] I_{R3} \\[3mm] I_{R4}
	\end{array}\right] =
%
	\left[\begin{array}{c}
		0 \\[3mm] 1 \\[3mm] -1 \\[3mm] 0
	\end{array}\right]
%
	\left[\begin{array}{c}
		I_1
	\end{array}\right] \label{eq:example_KCL_phasor}
\end{equation} %>>>

	Now apply Kirchoff's Voltage Law on the loops:

\begin{equation} %<<<
	\left\{\begin{array}{l}
		(L1):\ -V_1 + V_{R1}          = 0 \\[3mm]
		(L2):\ -V_{C1} + V_{R2} + V_{R1} = 0 \\[3mm]
		(L3):\  V_{I1} + V_{R2}          = 0 \\[3mm]
		(L4):\ -V_{R3} + V_{L1} + V_{R4} = 0 
	\end{array}\right.
\end{equation} %>>>

	But since $V_{I1} = -V_{R2}$, eliminate the former:

\begin{equation} %<<<
	\left\{\begin{array}{l}
		-V_1 + V_{R1}          = 0 \\[3mm]
		-V_{C1} + V_{R2} + V_{R1} = 0 \\[3mm]
		-V_{R3} + V_{L1} + V_{R4} = 0 
	\end{array}\right.
\end{equation} %>>>

	In matrix form,

\begin{equation} %<<<
%
	\left[\begin{array}{ccccccc}
	 0 &  0 &  1 & 0 & 0 & 0 \\[3mm]
	-1 &  0 &  0 & 1 & 1 & 0 \\[3mm]
	 0 &  1 &  0 & 0 &-1 & 1 
	\end{array}\right]
%
	\left[\begin{array}{c}
		V_{C1} \\[3mm] V_{L1} \\[3mm] V_{R1} \\[3mm] V_{R2} \\[3mm] V_{R3} \\[3mm] V_{R4}
	\end{array}\right] =
%
	\left[\begin{array}{c}
		1 \\[3mm] 0 \\[3mm] 0
	\end{array}\right]
%
	\left[\begin{array}{c}
		V_1
	\end{array}\right]\label{eq:example_KVL_phasor}
\end{equation} %>>>

	Now using the capacitor, inductor and resistor relationships on \eqref{eq:example_KCL_phasor} and \eqref{eq:example_KVL_phasor},

\begin{gather} %<<<
%
	\left[\begin{array}{ccccccc}
	 j\omega C_1 & 0 & 0 & 0 & 0 & 0 \\[3mm]
	           0 & 1 & 0 & 0 & 0 & 0 \\[3mm]
	           0 & 0 & 1 & 0 & 0 & 0 \\[3mm]
	           0 & 0 & 0 & 1 & 0 & 0 \\[3mm]
	           0 & 0 & 0 & 0 & 1 & 0 \\[3mm]
	           0 & 0 & 0 & 0 & 0 & 1
	\end{array}\right]
%
	\left[\begin{array}{c}
		V_{C1} \\[3mm] V_{L1} \\[3mm] V_{R1} \\[3mm] V_{R2} \\[3mm] V_{R3} \\[3mm] V_{R4}
	\end{array}\right] =
%	
	\left[\begin{array}{ccccccc}
	  1 &  0           &  0   & 0   & 0   & 0   \\[3mm]
	  0 &  j\omega L_1 &  0   & 0   & 0   & 0   \\[3mm]
	  0 &  0           &  R_1 & 0   & 0   & 0   \\[3mm]
	  0 &  0           &  0   & R_2 & 0   & 0   \\[3mm]
	  0 &  0           &  0   & 0   & R_3 & 0   \\[3mm]
	  0 &  0           &  0   & 0   & 0   & R_4
	\end{array}\right]
%
	\left[\begin{array}{c}
		I_{C1} \\[3mm] I_{L1} \\[3mm] I_{R1} \\[3mm] I_{R2} \\[3mm] I_{R3} \\[3mm] I_{R4}
	\end{array}\right]
\end{gather} %>>>

	Solving for the voltages,

\begin{equation} %<<<
	\left[\begin{array}{c}
		V_{C1} \\[3mm] V_{L1} \\[3mm] V_{R1} \\[3mm] V_{R2} \\[3mm] V_{R3} \\[3mm] V_{R4}
	\end{array}\right] =
%
	\left[\begin{array}{ccccccc}
	\left(j\omega C_1\right)^{-1} &  0           &  0   & 0   & 0   & 0   \\[3mm]
	                            0 &  j\omega L_1 &  0   & 0   & 0   & 0   \\[3mm]
	                            0 &  0           &  R_1 & 0   & 0   & 0   \\[3mm]
	                            0 &  0           &  0   & R_2 & 0   & 0   \\[3mm]
	                            0 &  0           &  0   & 0   & R_3 & 0   \\[3mm]
	                            0 &  0           &  0   & 0   & 0   & R_4
	\end{array}\right]
%
	\left[\begin{array}{c}
		I_{C1} \\[3mm] I_{L1} \\[3mm] I_{R1} \\[3mm] I_{R2} \\[3mm] I_{R3} \\[3mm] I_{R4}
	\end{array}\right] \label{eq:example_KVL_phasor_1}
\end{equation} %>>>

	Thus substituting \eqref{eq:example_KVL_phasor_1} into \eqref{eq:example_KVL_phasor},

\begin{equation} %<<<
%
	\left[\begin{array}{ccccccc}
	 0 &  0 &  1 & 0 & 0 & 0 \\[3mm]
	-1 &  0 &  0 & 1 & 1 & 0 \\[3mm]
	 0 &  1 &  0 & 0 &-1 & 1 
	\end{array}\right]
%
	\left[\begin{array}{ccccccc}
	\left(j\omega C_1\right)^{-1} &  0           &  0   & 0   & 0   & 0   \\[3mm]
	                            0 &  j\omega L_1 &  0   & 0   & 0   & 0   \\[3mm]
	                            0 &  0           &  R_1 & 0   & 0   & 0   \\[3mm]
	                            0 &  0           &  0   & R_2 & 0   & 0   \\[3mm]
	                            0 &  0           &  0   & 0   & R_3 & 0   \\[3mm]
	                            0 &  0           &  0   & 0   & 0   & R_4
	\end{array}\right]
%
	\left[\begin{array}{c}
		I_{C1} \\[3mm] I_{L1} \\[3mm] I_{R1} \\[3mm] I_{R2} \\[3mm] I_{R3} \\[3mm] I_{R4}
	\end{array}\right]
	=
%
	\left[\begin{array}{c}
		1 \\[3mm] 0 \\[3mm] 0
	\end{array}\right]
%
	\left[\begin{array}{c}
		V_1
	\end{array}\right]\label{eq:example_KVL_phasor}
\end{equation} %>>>

	Note that with combined with \eqref{eq:example_KCL_phasor} this system forms a seven-equations-by-six-variable system, meaning one equation is redundant. From the third equation of \eqref{eq:example_KCL_phasor} we note that $I_{L1} = I_{R4}$, and that equation can be removed, leaving a defined system where the remaining currents can be obtained and, from them, all of the rest of the voltages.

\examplebar
\end{example}%>>>

%-------------------------------------------------
\section{Complex and Average Power of Static Phasors} %<<<1

	A very convenient and useful result of the complexification $p_S$ is that the current and voltage phasors can also be used to calculate the instantaneous power developed by a particular circuit, as shown in theorem \ref{theo:sfp_complex_apparent_power}; more specifically, the inner product of the complex phasor space $S = \left< V,I\right> = V\overline{I}$ is called the Complex or Apparent Power is equivalent to $S = P + jQ$, where $P$ is called the Active Power and $Q$ called the Reactive Power, and the instantaneous power $p(t) = v(t)i(t)$ is a combination of $P$ and $Q$.

\begin{theorem}[Phasorial Complex Power]\label{theo:sfp_complex_apparent_power} % <<<
	Let $i = m_i\cos\left(\omega t + \phi_i\right)$ be current through an AC network and $v = m_v\cos\left(\omega t + \phi_v\right)$ be the voltage across the same circuit. Denote $I$ and $V$ as the corresponding phasors of $i$ and $v$. Then the inner product of $V$ and $I$, denoted as the Complex Apparent Power $S \in \mathbb{C}$ calculated as

\begin{equation} S = \dfrac{1}{2}\left<V,I\right> = V\overline{I} = P + jQ, \label{eq:complex_power_def}\end{equation}

	where

\begin{equation}
\left\{\begin{array}{l}
	P = \dfrac{1}{2}m_im_v\cos\left(\phi_v-\phi_i\right) = \left\lvert V\right\rvert\left\lvert I\right\rvert\cos\left(\phi_v-\phi_i\right) \\[5mm]
	Q = \dfrac{1}{2}m_im_v\sin\left(\phi_v-\phi_i\right) = \left\lvert V\right\rvert\left\lvert I\right\rvert\sin\left(\phi_v-\phi_i\right)
\end{array}\right.
\end{equation}

	is such that the instantaneous power performed by the circuit can be calculated as

\begin{equation} p(t) = P \left\{1 + \cos\left[2\left(\omega t + \phi_v\right)\right] \right\} + Q \sin\left[2\left(\omega t + \phi_v \right)\right] . \end{equation}

\end{theorem}
\textbf{Proof:} the instantaneous power is calculated as

\begin{equation} p(t) = v(t)i(t) = m_i m_v \cos\left( \omega t + \phi_v \right)\cos\left(\omega t + \phi_i\right) \end{equation}

	Using that

\begin{equation} \cos(a)\cos(b) = \dfrac{1}{2}\left[\cos(a+b) + \cos(a-b)\right], \label{eq:cos_identity}\end{equation}

	Then

\begin{equation} p(t) = m_i m_v \dfrac{1}{2} \left[ \cos\left(2\omega t + \phi_v + \phi_i\right) + \cos\left(\phi_v - \phi_i\right)\right] \end{equation}

	Denote $\Delta\phi = \phi_v - \phi_i$. Then $\phi_v + \phi_i = 2\phi_v - \Delta\phi$; therefore,

\begin{equation} p(t) = \dfrac{m_i m_v}{2} \left\{\cos\left[2\left(\omega t + \phi_v\right) - \Delta\phi\right] + \cos\left[\Delta\phi\right]\right\} \end{equation}

	Using $\cos(a-b) = \cos(a)\cos(b) + \sin(a)\sin(b)$,

\begin{equation} p(t) = \dfrac{m_i m_v}{2} \left\{\raisebox{5mm}{} \cos\left(\Delta\phi\right)\left\{\raisebox{3mm}{} 1 + \cos\left[2\left(\omega t + \phi_v\right)\right]\right\} + \sin\left(\Delta\phi\right)\sin\left[2\left(\omega t + \phi_v\right)\right] \right\} . \label{eq:sfp_complex_apparent_power_eq1} \end{equation}

	Let

\begin{align}
	P = \dfrac{m_i m_v}{2} \cos\left(\Delta\phi\right) \\[3mm]
	Q = \dfrac{m_i m_v}{2} \sin\left(\Delta\phi\right)
\end{align}

	Then

\begin{equation} p(t) = P\left\{1 + \cos\left[2\left(\omega t + \phi_v\right) \right]\right\} + Q\sin\left[2\left(\omega t + \phi_v \right)\right] . \label{eq:sfp_complex_apparent_power_eq2} \end{equation}

	Now, calculating $S$,
	
\begin{align}
S(t)
	&= \dfrac{1}{2}\left<V,I\right> = V\overline{I} \nonumber\\[3mm]
	&= \dfrac{m_vm_i}{2} e^{j\phi_v} e^{-j\phi_i} = \dfrac{m_v m_i}{2} e^{j\Delta\phi} \nonumber\\[3mm]
	&= \dfrac{m_i m_v}{2} \left[\cos\left(\Delta\phi\right) + j\sin\left(\Delta\phi\right) \right] = P + jQ \label{eq:sfp_complex_apparent_power_eq3}
\end{align}

	Finally, it is immediate to note the bijection between equations \eqref{eq:sfp_complex_apparent_power_eq1} and \eqref{eq:sfp_complex_apparent_power_eq2}: given $p(t)$, one can construct the complex value $S$; on the other hand, given $S$, one can reconstruct $p(t)$.  \hfill$\blacksquare$

\vspace{5mm}
\hrule
\vspace{5mm}
% >>>

	One note to be made is about the RMS value of sinusoids. More often than not, in the literature a term $\sqrt{2}$ appears in the definition \ref{def:static_phasor_transform}, that is, phasors are defined with an amplitude divided by $\sqrt{2}$. This stems directly from the fact that equation \eqref{eq:complex_power_def} needs a halving of the inner product of voltage and current, which ultimately stems from the $\frac{1}{2}$ of identity \eqref{eq:cos_identity}. Therefore, if we define the SPO as relating $x(t) = K\cos\left(\omega t + \phi\right)$ to 

\begin{equation} X_{\text{RMS}} = \dfrac{K}{\sqrt{2}}e^{j\phi} \end{equation}

	\noindent then the complex power becomes $S = \left\langle V, I\right\rangle$. This is known as the \textbf{power invariant} version of the operator because without this term, the inner product $V\overline{I}$ equates to double the instantaneous power. This fact can also be seen through the non-coincidence that the RMS value of a sinusoid of magnitude $K$ is $K/\sqrt{2}$.

	Theorem \ref{theo:sfp_complex_apparent_power} is seminal in the understanding of how electrical power works in AC grids; yet, as it is presented, not much insight is given as to what exactly are the physical interpretations of the active and reactive components of power. To this extent, there are two ways to give meaning to these componsnets. First, corollary \ref{corollary:direct_quad_current} shows that the active power accounts for a component of current that is in phase with voltage, whereas the reactive power accounts for the component in quadrature to voltage.

\begin{corollary}[Direct and quadrature components of AC currents]\label{corollary:direct_quad_current} %<<<
	Let $v,i,P,Q$ as defined in theorem \ref{theo:sfp_complex_apparent_power}. Then $i$ can be written as

\begin{equation} i(t) = \dfrac{2P}{m_v}\cos\left(\omega t + \phi_v\right) + \dfrac{2Q}{m_v}\sin\left(\omega t + \phi_v\right) .\end{equation}
\end{corollary}
\textbf{Proof:} write

\begin{align}
	i(t)
	&= m_i\cos\left(\omega t + \phi_i\right) \nonumber\\[3mm]
	&= m_i\cos\left(\omega t + \phi_v - \Delta\phi\right) \nonumber\\[3mm]
	&= m_i\left[\cos\left(\omega t + \phi_v\right)\cos\left(\Delta\phi\right) + \sin\left(\omega t + \phi_v\right)\sin\left(\Delta\phi\right) \right] \nonumber\\[3mm]
	&= \dfrac{2P}{m_v}\cos\left(\omega t + \phi_v\right) + \dfrac{2Q}{m_v}\sin\left(\omega t + \phi_v\right)
\end{align} \hfill$\blacksquare$

\vspace{5mm}
\hrule
\vspace{5mm} %>>>

	Corollary \ref{corollary:direct_quad_current} implies that the current can also be written in a sum of two components, one in phase with voltage and another one in quadrature. Therefore, $P$ represents the component of the current that is in phase with the voltage; corollary \ref{corollary:sfp_active_average_power} shows that, because of this, the average power over a half-period $T/2 = \pi/\omega$ is exactly $P$; this fact justifies the naming of ``active power'' for $P$. As for $Q$, it is generated by the component of the current that is in quadrature with the voltage; when integrated over time to obtain the average power, this component vanishes. This means $Q$ is a purely oscillatory power flow that is periodically exchanged between capacitances and inductances; this happens most notably in the LC circuit, also called a ``tank'' circuit. Therefore, $Q$ is a power component which energy is deposited on the storing elements (inductances and capacitances) in half a cycle of the sinusoid, but then retrieved on the following half cycle — meaning $Q$ is not a spent power, rather oscillatory, justifying its naming of ``reactive power''. Despite $Q$ not generating any effectively used power, it is nevertheless important because it still generates a current component, meaning it needs to be accounted for in the dimensioning and power spenditure of the grid.

	Another way to give meaning to active and reactive power is

\begin{corollary}[Active power as average power]\label{corollary:sfp_active_average_power} %<<<
	Let $v,i,P,Q$ as defined in theorem \ref{theo:sfp_complex_apparent_power}. Then for any time instant $t$, the average power in the interval $\left[t,t+T/2\right]$ with $t = 2\pi/\omega$ is equal to

\begin{equation} \dfrac{2}{T} \int_{t}^{t + \frac{T}{2}} v(x)i(x)dx = P .\end{equation}
\end{corollary}
\textbf{Proof:} a direct consequence of equation \eqref{eq:sfp_complex_apparent_power_eq2}. Compute the average power:

\begin{gather}
	\dfrac{2}{T} \int_{t}^{t + \frac{T}{2}} p(x)dx = \dfrac{2}{T}\int_{t}^{T + \frac{T}{2}} \left(\raisebox{5mm}{} P\left\{1 + \cos\left[2\left(\omega x + \phi_v\right) \right]\right\} + Q\sin\left[2\left(\omega x + \phi_v \right)\right]\right)dx \nonumber\\[3mm]
	= P \left(\dfrac{2}{T}\int_{t}^{t + \frac{T}{2}} \left\{1 + \cos\left[2\left(\omega x + \phi_v\right) \right]\right\}dx\right) + Q\left[ \dfrac{2}{T} \int_{t}^{t + \frac{T}{2}} \sin\left[2\left(\omega x + \phi_v \right)\right]dx \right] .
\end{gather}

	The integrals of the sine and the cosine vanish in the interval $\left[t,t +T/2\right]$, leaving

\begin{equation} \dfrac{2}{T} \int_{t}^{t + \frac{T}{2}} p(x)dx = P \left(\dfrac{2}{T}\int_{t}^{t + \frac{T}{2}} 1dx\right) = P .\end{equation}

\hfill$\blacksquare$

\vspace{5mm}
\hrule
\vspace{5mm} %>>>

	Finally, the fact that $P$ and $Q$ can be calculated directly through the phasors $V$ and $I$ mean that the phasor analysis is sufficient to not only describe the Electrical Grid in time, but also to describe how power is distributed along it; this effectively means that the entirety of the analysis can be carried out in phasorial form, and when a time representation is needed, a simple inverse transform yields the time signals pertaining to voltages, current and power.



\part{Dynamic Phasors Theory}\label{part:dynphasor_theory}

%--------------------------------------------------------------------------------------------------
\chapter{Dynamic Phasors Theory}\label{chapter:dynamic_phasor_theory}
%--------------------------------------------------------------------------------------------------

	 In a direct language, the essence of phasors is that if one is amenable to disregarding the transient response of the system, the steady-state solution of the states of the system can be found as some particular orbit of the differential equation. When the excitation is comprised of static sinusoids of frequency $\omega$, static sinusoids of frequency $\omega$ comprise a particular solution of the differential equations defined by the system; due to the exponentially stable nature of passive linear circuits, this sinusoidal particular orbit is also the exponentially stable steady-state behavior of the system. In simpler terms, phasors are a ``sneaky'' way to solve sinusoidally excited LTI differential equations, given one is willing to discard transient phenomena.

	In the past chapter, some very important tools of circuit analysis — Kirchoff's Voltage and Current laws, the Superposition Theorem, and the Thèvenin-Norton Theorems — were left out because they will be proven for the more generalized Dynamic Phasor case. Using these theorems, the capacitive conductance and inductive impedances are defined as $Y_C = j\omega C$ and $Z_L = j\omega L$, and these entities are applied to an electrical grid making the phasorial analysis self-sufficient in the sense that the time-domain analysis does not need to be undertaken first before using phasors. Finally, the complex power $S = V \overline{I}$ is shown to be a direct representation of the instantaneous AC power of a circuit, and its real part the average power developed by that circuit.

	For all its ellegance, however, the Classical Phasor Theory only embraces a very specific type of signals: static sinusoids — meaning constant amplitude, frequency and phase. However, in most Electrical Engineering studies, the sinusoidal voltages of agents and nodes are not ``static'', in the sense that they exhibit transient effects such as time-varying amplitudes, phases and even frequencies. Particularly in Electric Power Systems, such phenomena are ubiquituous and a common occurence after disturbances like loads or faults.

%-------------------------------------------------
\section{Nonstationary sinusoidal signals: the current theory of Dynamic Phasors} %<<<2

	We want to introduce the theory of Dynamic Phasors, which can be summarized as being the time-varying alternatives to classical phasors. This means that the objective is to embrace a larger, more general class signals of a certain ``sinusoidal shape'', like in definition \ref{def:sinusoid} .

\begin{definition}[Sinusoid] \label{def:sinusoid} %<<<
	A signal $x(t)\in\left[\mathbb{R}\to\mathbb{R}\right]$ is a \textbf{sinusoid} if there are two functions $m(t)$ called a modulus or amplitude (\textit{moduli} in the plural) and $\theta(t)$ called the angle such that $x(t) = m(t)\cos\left(\theta(t)\right)$. Furthermore, $x$ is a \textbf{stationary sinusoid} or phase if $m$ and $\dot{\theta}$ are constant, and \textbf{nonstationary} if else.
\end{definition} %>>>

	Particularly, we are interested in sinusoids which angle can be decomposed in a time-varying notion of frequency and phase. By correlation, because in static phasors the frequency multiplies the time $t$, consider first the signals

\begin{equation} x(t) = m\left(t\right)\cos\left[\omega(t) t + \phi(t)\right], \label{eq:dynamic_sinusoid_example}\end{equation}

	\noindent where the amplitude $m$, frequency $\omega$ and phase $\phi$ are time-varying correlative quantities of the amplitude, frequency and phase of static phasors.

	Notably, the Static Phasor Operator is only applicable to the signal \eqref{eq:dynamic_sinusoid_example} if m, $\omega$ and $\phi$ are constants, meaning that the Classical Phasor Theory is unnaplicable otherwise, that is, there is no phasor representation possible for signal \eqref{eq:dynamic_sinusoid_example} that can allow for the solution of the linear ODEs. In this scenario, a natural question is wether there is some extended idea of phasor, variant in time, to denote such a signal; the intuitive candidate would be

\begin{equation} X(t) = m\left(t\right)e^{j\phi(t)}. \end{equation}

	Despite being elementary to note how this phasor reconstructs $x(t)$, this is not a bijective transformation that can take a signal in time to translate it into a complex number in the frequency-space algebraic domain that, when solved, can reconstruct the original solution to the original differential equations, like the Static Fourier Phasors can.

	In summary, the classic idea of phasor, while allowing for phasorial representation of signals, can only do so for static sinusoidal signals and fails to give a mathematical tool that can solve more sophisticated nonstationary signals. As shown in the introduction, the justification of applying the CPT to phasorial dynamical systems requires the Quasi-Static Modelling, that is, the supposition that the grid dynamics need to be supposed much faster than the transients of the agents that act upon it, allowing modelling the grid in its static purely sinusoidal behavior, thus allowing for its complexification. This assumption is broken when switched power systems like inverters are at play, because the timescale of their dynamics are comparable to the timescales of grid dynamics.

	To develop Dynamic Phasors, the literature by default escalates Classical Phasors to integral transforms, inspired by a branch of mathematics called Time-Frequency Analysis, founded with the intent of expressing nonstationary time signals in the frequency domain. The most used strategies revolve around integral transformations of some form, that is, to decompose a signal $x(t)\in\left[\mathbb{R}\to\mathbb{C}\right]$ into a combination of translations (dilations, contractions and shifts) of a base or reference function that has a defined frequency spectrum. Call this base function $\mu\left(t\right)$; then the transformation of $x(t)$ with respect to $\mu$ is given by the inner product

\begin{equation} \left< x \right>_{\left(a,b\right)} = \left<x,\mu \left(\dfrac{\tau-a}{b}\right)\right> = \int_{-\infty}^{\infty} x\left(\tau\right)\overline{\raisebox{5mm}{} \mu \left(\dfrac{\tau-a}{b}\right)} d\tau, \end{equation}

	The idea is to adopt a base function $\mu$ that when translated forms a basis over the Hilbert Space of square-integrable functions $L^2\left(\mathbb{R}\right)$, that is, the base function $\mu$ needs to have finite energy (equivalent to being square integrable), the translations must be orthogonal and the span of the translations must be $L^2\left(\mathbb{R}\right)$. The result $\left< x \right>_{\left(a,b\right)}$ is then called the \textbf{component} or \textbf{harmonic} of $x(t)$ at the parameters $a$ and $b$. Famous examples of these integral transforms are the Short-Time Fourier Transform (STFT) and the Wavelet Transform (WT). In the STFT, the base function $\mu$ is the exponential $e^{-j\omega \tau}$ multiplied by a windowing function $w\left(\tau\right)$:

\begin{equation} \mu_{\left(\text{STFT}\right)}\left(\omega,\tau\right) = w\left(\tau\right)e^{j\omega\tau}.\end{equation}
	
	Typical cases of the windowing function are the gaussian distribution or a simple rectangular function. These windowings mean that the STFT is in essence a Fourier Transform of the original signal $x(t)$ limited by the window; as the time $t$ slides, so does the window. Therefore, the STFT corresponds to the time-sliding Fourier Transform of the windowed signal $x(t)$. It can be further shown that a rectangular  windowing represents a convolution with the sinc function in the frequency domain, which introduces ringing artifacts and oscillations in short time windows; this can be seen as a consequence of the Nyquist Theorem, which states that as the sampling frequency window gets narrower, the convolution yields alising phenomena. To mitigate this, some windows as the Gaussian Function or the Hann Window act as filters that avoid such behavior.

	In the case of the Wavelet Transform, the base function is called the \textit{mother wavelet}, and there are several ones to choose from in the literature — such as the Ricker Wavelet, the Morlet Wavelet, Butterworth Wavelet, Ormsby Wavelet as compared in \cite{ryanRickerOrmsbyKlander1994} — each option featuring benefits and disadvantages in signal processing. The benefit of the Wavelet Transform is that, unlike the STFT, it provides high-frequency resolution at lower frequency by means of its bidimensional transform \pcite{Guo2022}. Multidimensional wavelet transformations are also available in the literature, first conceived by \cite{zouDiscreteOrthogonalMband1992}.

	In this thesis, the theoretical bases for Dynamic Phasors based on the rectangular windowing of Short Time Fourier Transforms will be presented as the most common candidate for the theoretical basis of Dynamical Phasors in the Electric Power System literature. In the next subsections, the STFT will be presented and defined rigorously; it will be proven that the STFT does offer the notion of a time-frequency representation of nonstationary sinusoids and it does expand on the notion of impedances. Because the developments are based on the Fourier Transform, the Dynamic Phasors resulting from this analysis are thenceforth called STFT Dynamic Phasors (STFT-DPs).

	It will be further shown that there are two fundamental shortcomings with the STFT, namely the fact that it fails to offer a reconstruction of the original time signals of an electrical grid, and it fails to offer a well-defined notion of a complex power. It will also be shown that, unlike their static phasor counterpart, Fourier Dynamic Phasors lack the ability to transform time differential equations into algebraic ones; rather, they produce infinite sets of complex differential equations. Consequently, the STFT framework offers little solace when it comes to the representation of non-static sinusoidal signals.

	Further, we also take a look at the Hilbert Transform (HT), which while able to produce \textit{some} notion of time-varying phasors, it has very specific characteristics that make it very limited in scope. Moreover, it is really only applicable to a limited class of signals — namely those which Fourier Transform of the amplitude has a support limited by the transform of the angle — and that it also fails to produce dynamic counterparts to active and reactive power.

	In other words, neither the STFT or the HT offer alternatives to theorems \ref{theo:phasors_solutions} and \ref{theo:sfp_complex_apparent_power}, that is, the complex signals obtained through those transforms are not guaranteed to be losslessly reconstruct the solutions to the original time differential equations of the electrical system, and the instantaneous power cannot be obtained from the inner product of the complex space induced by the Dynamic Phasors of neither STFT or HT. Furthermore, the solution of the equivalent system in the frequency domain is significantly more difficult than in the time domain, rendering it questionable when it comes to practicality and usefulness.

%------------------------------------------------
\section{Short-Time Fourier Transform of nonstationary signals} %<<<1

	Take the signal $x(t)$ of Figure \ref{fig:stft_schematic}, sampled through a window $w(t)$ of size $T$ (which can be time-varying as long as it is positive), generating a windowed signal $y(s) = x(s)w(s-T)$; then the Fourier Transform of this sampled signal is taken:

\begin{equation} \mathbf{F}\left[y\right] = \int_{\ \mathbb{R}} x\left(s\right)w\left(s - t\right) e^{-j\omega s}ds . \label{eq:stft_generic}\end{equation}

% GAUSS STFT DIAGRAM <<<
\begin{figure}[t]
\centering
	\begin{tikzpicture}[scale=1,>={Stealth[inset=0mm,length=1.5mm,angle'=50]}]
	\draw [black!30, line width=5mm, -{Stealth[inset=0mm,length=10mm,angle'=50]}] (-12mm,17mm) -- (7mm,17mm);
	\node[black] at (-5mm, 17mm)  (stftlabel) {STFT};

		\begin{axis}[
			at={(-75mm,0mm)},
			width=75mm,
			height=50mm,
			tick style={draw=none},
			axis lines=middle,
			xmin = -2, xmax = 9,
			ymin = -0.3, ymax = 3,
 			xticklabel={\empty}, yticklabel={\empty},
			xlabel = {$s$}
			]
			\addplot  [thick, stewartblue, domain=-2:9, smooth, samples=100] {0.1*((0.9*x - 1.4)^3 - 3*(x - 1.4)^2 - 0.4*(x-1.4) + 15)};
			\addplot  [thick, stewartpink, domain=-2:9, smooth, samples=100] {1.8*exp(-(x/2-1.5)^2)};
			\addplot +[thick, mark=none,stewartpink, dashed] coordinates {(3, 0) (3, 1.8)};

			\node[stewartblue] at (axis cs:7.5,2.2)  (xlabel) {$x(s)$};
			\node[stewartpink] at (axis cs:3,2.2)  (wlabel) {$w(s-t)$};
			\node[stewartpink] at (axis cs:3,-0.2)  (wlabel) {$t$};
		\end{axis}

		\begin{axis}[
			at={(10mm,0mm)},
			width=75mm,
			height=50mm,
			tick style={draw=none},
			axis lines=middle,
			xmin = -2, xmax = 9,
			ymin = -0.25, ymax = 2,
 			xticklabel={\empty}, yticklabel={\empty},
			xlabel = {$s$}
			]
			\addplot  [thick, stewartgreen, domain=-2:9, smooth, samples=100] { 1.8*exp(-(x/2-1.5)^2)*(0.1*((0.9*x - 1.4)^3 - 3*(x - 1.4)^2 - 0.4*(x-1.4) + 15))};
			\node[stewartgreen] at (axis cs:6.3,1.5)  (xlabel) {$y(s) = x(s)w(s-t)$};
		\end{axis}

	\end{tikzpicture}
	\caption
[Schematization of the STFT through a gaussian window to produce a windowed signal.]
{Schematization of the STFT: a signal $x(t)$ is sampled through a window (in this case a gaussian) to produce a windowed signal which is subject to a Fourier Transform.}
	\label{fig:stft_schematic_generic}
\end{figure} %>>>

	As the time $t$ grows, the translated window $w(s-t)$ ``slides'', such that at each time $t$ the periodic signal resulting changes. This causes $\mathbf{F}\left[y\right] = Y\left(\omega,t\right)$ to be a complex time function in both frequency $\omega$ and the time $t$. Therefore, these harmonics become time-varying. Figure \ref{fig:stft_schematic_generic} shows a schematic of this process using a gaussian window, that is, a contraction-translation of $w(x) = e^{-x^2}$.

	Different windows yield particular sampling characteristics — for instance, the gaussian window naturally yields statistical properties of minimizing the standard deviation for a generic sampled signal. It is clear to see that this process makes it considerably difficult to model generic signals seen as it results two-dimensional complex functions, and it requires $x(t)$ to be defined everywhere the window is also defined (in the case of the gaussian window, the entirety of reals). Therefore, for the purposes of modelling, the most used window is the ``boxcar'' or simply rectangular window; the windowed signal produced is a periodic restriction of the original signal $x(t)$, therefore it can be written as a Fourier Series at the frequency $\omega$. This ``substitutes'' the frequency component $\omega$ for integer indexes, that is, harmonics at the multiple frequencies $k 2\pi/T(t)$. Again, as the time $t$ grows, the window ``slides'', such that at each time $t$ the periodic signal resulting changes, and so do its harmonics; therefore, these harmonics become time-varying.

% STFT DIAGRAM <<<
\begin{figure}[t]
\centering
	\begin{tikzpicture}[scale=1,>={Stealth[inset=0mm,length=1.5mm,angle'=50]}]
	\draw [black!30, line width=5mm, -{Stealth[inset=0mm,length=10mm,angle'=50]}] (-12mm,17mm) -- (7mm,17mm);
	\node[black] at (-5mm, 17mm)  (stftlabel) {STFT};

		\begin{axis}[
			at={(-50mm,0mm)},
			width=50mm,
			height=50mm,
			tick style={draw=none},
			axis lines=middle,
			xmin = -0.25, xmax = 7,
			ymin = -0.3, ymax = 2.2,
 			xticklabel={\empty}, yticklabel={\empty},
			xlabel = {$s$}
			]
			\addplot  [thick, stewartblue, domain=-0.25:7, smooth, samples=100] {0.1*((0.9*x - 1.4)^3 - 3*(x - 1.4)^2 - 0.4*(x-1.4) + 15)};
			\addplot  [thick, stewartpink, dashed, domain=1:5, smooth, samples=100] {1.6};
			\addplot +[thick, mark=none,stewartpink, dashed] coordinates {(1, 0) (1, 1.6)};
			\addplot +[thick, mark=none,stewartpink, dashed] coordinates {(5, 0) (5, 1.6)};

			\node[stewartblue] at (axis cs:5.6,2)  (xlabel) {$x(s)$};
			\node[stewartpink] at (axis cs:3,1.8)  (wlabel) {$w(s-t)$};
			\node[stewartpink] at (axis cs:1,-0.2)  (tlabel) {$t-T$};
			\node[stewartpink] at (axis cs:5,-0.2)  (tTlabel) {$t$};
		\end{axis}

		\begin{axis}[
			at={(10mm,0mm)},
			width=100mm,
			height=50mm,
			tick style={draw=none},
			axis lines=middle,
	%		axis line style={black!50},
			xmin = -9, xmax = 9,
			ymin = -0.25, ymax = 1.8,
 			xticklabel={\empty}, yticklabel={\empty},
			xlabel = {$s$}
			]
			\foreach \x  [evaluate=\x as \xeval] in {-3,...,2}
			{
				\addplot[thick, stewartblue, domain=4*\xeval:4*(\xeval + 1), smooth, samples=100] {0.1*((0.9*(x+1-4*\xeval) - 1.4)^3 - 3*((x+1-4*\xeval) - 1.4)^2 - 0.4*((x+1-4*\xeval)-1.4) + 15)};
			}
			\foreach \x  [evaluate=\x as \xeval] in {-3,-2,-1,1,2} {
				\addplot +[thick, mark=none,stewartpink, dashed] coordinates {(4*\xeval, 0) (4*\xeval, 1.8)};
				\edef\temp{\noexpand \node[thick, stewartpink,below] at (axis cs:4*\xeval,0)  (\x T) {$\x T$};} \temp
			}
		\end{axis}

	\end{tikzpicture}
	\caption
[Schematization of the STFT using a rectangular window.]
{Schematization of the STFT using a rectangular window: a signal $x(t)$ is sampled to produce a periodic restriction which is subject to a Fourier Series.}
	\label{fig:stft_schematic}
\end{figure} %>>>

	Thus, the process is as follows: take a signal $x(t)$ and a period signal $T(t)$ equivalent to a frequency $\omega(t) = 2\pi/T(t)$, and define a periodic signal as the restriction of $x(t)$ on $\left[t-T(t),t\right]$, that is,

\begin{equation} y: \left\{\begin{array}{rcl} \left[0,T\right] &\to& \mathbb{C} \\[3mm] s &\mapsto& x\left(s + T - t\right) \end{array}\right. \end{equation}

	\noindent then the harmonics are calculated as the Fourier Series of this periodic restriction:

\begin{equation} \left\langle y \right\rangle_k = \dfrac{1}{T}\int\limits_{0}^{T}y\left(s\right) e^{-jk\omega s}ds  \end{equation}

	\noindent and Theorem \ref{theo:stft_analysis} proves that this transformation reconstructs $x(t)$.

\begin{theorem}[Short-Time Fourier Transform Analysis \pcite{volpatoDynamicPhasorTransform2022}] \label{theo:stft_analysis}%<<<
	Let $x(t)\in\left[\mathbb{R}\to\mathbb{C}\right]$ square-integrable in some window interval $\left(t-T,t\right]$, with $T\in\mathbb{R}_+^*$ the window period (that may be time-varying) and $\omega = 2\pi T^{-1}$ the corresponding window angular frequency. Let $\left\langle x \right\rangle_k\hspace{-0.5mm} \left(t\right)\in\left[\mathbb{R}\to\mathbb{C}\right]$ be the k-th order harmonic of the Fourier Series of $x(t)$ limited to the interval, that is,

\begin{equation} \left\langle x \right\rangle_k\hspace{-0.5mm} \left(t\right) = \dfrac{1}{T}\int\limits_{t-T}^{t}x\left(\lambda\right) e^{-jk\omega\lambda}d\lambda .\label{eq:harmonics} \end{equation}

	Then, for every $\tau$ in $\left(t-T,t\right]$,

\begin{equation} x\left(\tau\right) = \sum\limits_{k\in\mathbb{Z}} \left\langle x \right\rangle_k\hspace{-0.5mm} \left(t\right)\ e^{jk\omega\tau}. \label{eq:fourierSeries} \end{equation}

\end{theorem}

\textbf{Proof}. For every instant $t$, define the periodic limitation $y\left(\phi\right)$ of $x(t)$, such that

\begin{equation} y\left(\phi\right) = x\left(s + T - t\right), s\in\left[0,T\right]. \label{eq:ySubstitution} \end{equation}

        Thence the Fourier Analysis of $y$ yields

\begin{equation} y\left(s\right) = \sum\limits_{k\in\mathbb{Z}} \left\langle y \right\rangle_k\hspace{-1mm} \ e^{jk\omega s}, \label{eq:ydef} \end{equation}

where

\begin{equation} \left\langle y \right\rangle_k = \dfrac{1}{T}\int\limits_{0}^{T}y\left(u\right) e^{-jk\omega u}d u \label{eq:yfour} \end{equation}

        Manipulating \eqref{eq:ydef},

\begin{equation} y\left(s\right) = \sum\limits_{k\in\mathbb{Z}} \left[\left\langle y \right\rangle_ke^{-jk\omega\left(t-T\right)}\right] e^{jk\omega\left(s - t + T\right)} \end{equation}

        And, from the definition, of $\left\langle y \right\rangle_k$ \eqref{eq:yfour},

\begin{equation} \left\langle y \right\rangle_k e^{-jk\omega\left(t-T\right)} = \dfrac{1}{T}\int\limits_{0}^{T}y\left(u\right) e^{-jk\omega\left(u - t +T\right)}du .\end{equation}

        Using \eqref{eq:ySubstitution}, adopt $\lambda = u - t + T$:

\begin{equation} \left\langle y \right\rangle_k e^{-jk\omega\left(t-T\right)} = \dfrac{1}{T}\int\limits_{t-T}^{t}x\left(\lambda\right) e^{-jk\omega\lambda}d\lambda \label{eq:ySubs2} \end{equation}

        Then define

\begin{equation} \left\langle x \right\rangle_k \coloneqq \left\langle y \right\rangle_ke^{-jk\omega\left(t-T\right)} \label{eq:ySubs3} \end{equation}

        Hence \eqref{eq:ySubs2} and \eqref{eq:ySubs3} imply

\begin{equation} \left\langle x \right\rangle_k = \dfrac{1}{T}\int\limits_{t-T}^{t}x\left(\lambda\right) e^{-jk\omega\lambda}d\lambda \end{equation}

        And, from \eqref{eq:ydef},

\begin{equation} y\left(s\right) = \sum\limits_{k\in\mathbb{Z}} \left\langle x \right\rangle_k e^{jk\omega\left(s - t + T\right)} \end{equation}

        Finally, using $\tau = s - t + T$,

\begin{equation} x\left(\tau\right) = \sum\limits_{k\in\mathbb{Z}} \left\langle x \right\rangle_k e^{jk\omega\tau} \end{equation}

\hfill$\blacksquare$

\vspace{5mm}
\hrule
\vspace{5mm}
% >>>

	It is trivial to prove that the sequence of harmonics is unique, that is: two signals $y(t)$ and $x(t)$ can only share the same harmonics for all time instants if and only if $x(t)=y(t)$ for all time instants.

\begin{theorem}[Uniqueness of Fourier Dynamic Phasors]\label{theo:fdp_uniqueness}%<<<
	Let $x(t)$ and $y(t)$ be two real signals. Then $x(t) = y(t)$ for all time instants if and only if $\left\langle x \right\rangle_k\hspace{-0.5mm} \left(t\right) = \left\langle y \right\rangle_k\hspace{-0.5mm} \left(t\right)$ for all $t$ and for all $k$. \end{theorem}
\textbf{Proof:} it is easy to prove that if $x(t) = y(t)$ for all time instants $t$, then the harmonics follow:

\begin{equation}
        \left\langle x \right\rangle_k\hspace{-0.5mm} \left(t\right) - \left\langle y \right\rangle_k\hspace{-0.5mm} \left(t\right) = \dfrac{1}{T}\int\limits_{t-T}^{t} \left[x\left(\lambda\right) - y\left(\lambda\right)\right]e^{-jk\omega\lambda}d\lambda = \dfrac{1}{T}\int\limits_{t-T}^{t} \left(0\right) e^{-jk\omega\lambda}d\lambda = 0
\end{equation}

	For the other direction, suppose $x$ and $y$ share the same harmonics for all time instants $t$. Then, for all $\tau\in\left(t-T,t\right]$,

\begin{equation}
        x\left(\tau\right) - y\left(\tau\right) = \sum\limits_{k\in\mathbb{Z}} \overbrace{\left[\left\langle x \right\rangle_k\hspace{-0.5mm} \left(t\right) - \left\langle y \right\rangle_k\hspace{-0.5mm} \left(t\right)\right]}^{=0\ \forall t,k}\ e^{jk\omega\tau} = 0
\end{equation}
        \hfill$\blacksquare$
\vspace{5mm}
\hrule
\vspace{5mm}
%>>>

	Theorem \ref{theo:fdp_uniqueness} implies that the transformation is bijective; therefore it can be defined as a functional operator $\mathbf{STFT}\left[\cdot\right]$.

\begin{definition}[STFT Dynamic Phasor Transform]%<<<
	Let $x(t)\in\left[\mathbb{R}\to\mathbb{C}\right]$ be a complex signal and $T(t)$ a positive time window parameter; then its STFT Dynamic Phasor Transform over period $T$ (or over frequency $\omega = 2\pi T^{-1}$), denoted $\mathbf{STFT}\left[x\right]$, maps $x$ to its series of harmonics, that is, a series of complex signals, such that

\begin{equation}
        \mathbf{STFT}\left[\cdot\right]: \left\{\begin{array}{rcl}
	\left[\mathbb{R}\to\mathbb{C}\right] &\to& \left[\mathbb{R}\to\mathbb{C}\right]^{\left[\mathbb{Z}\right]} \\[5mm]
%
        x\left(t\right) &\mapsto& \left\{ \left\langle x \right\rangle_k\hspace{-0.5mm}\left(t\right) = \displaystyle\dfrac{1}{T}\int\limits_{t-T}^{t}x\left(\lambda\right) e^{-jk\omega\lambda}d\lambda\right\}_{k\in\mathbb{Z}}
\end{array}\right.
\end{equation}

        Conversely, the inverse transformation over $T$ takes a series of complex harmonic signals and reconstructs a time signal:

\begin{equation}
        \mathbf{STFT}^{-1}\left[\cdot\right]: \left\{\begin{array}{rcl}
	\left[\mathbb{R}\to\mathbb{C}\right]^{\left[\mathbb{Z}\right]} &\to& \left[\mathbb{R}\to\mathbb{C}\right] \\[5mm]
%
        \left\{\raisebox{4mm}{} \left\langle x \right\rangle_k\hspace{-0.5mm}\left(t\right) \right\}_{k\in\mathbb{Z}} &\mapsto& x(t) = \displaystyle\sum\limits_{k\in\mathbb{Z}} \left\langle x \right\rangle_k e^{jk\omega\tau}
\end{array}\right. \label{def:inverse_fdp}
\end{equation}

\end{definition} %>>>

%------------------------------------------------
\subsection{Operational properties of STFT Dynamic Phasors} %<<<2

	Much the same way like Static Phasors, STFT DPs inherit the same niceties and propeties of the Fourier Transform: for all integers $k$ and any complex $\alpha$, $\left\langle x + \alpha y\right\rangle_k = \left\langle x\right\rangle_k + \alpha\left\langle y\right\rangle_k$, meaning the transform is linear. For the derivative, using the multiplication rule for integrals,

\begin{align}
	\left\langle \dfrac{dx(t)}{dt}\right\rangle_k &= \int^{t}_{t-T} x'(s)e^{-jk\omega s} ds = \left[x(s) e^{-jk\omega s}\right]_{t-T}^t - \int^{t}_{t-T} x(s) \dfrac{d}{ds} e^{-jk\omega s} ds = \nonumber\\[3mm]
	&= \left[x(s) e^{-jk\omega s}\right]_{t-T}^t - \int^{t}_{t-T} x(s) \left(-jk\omega\right) e^{-jk\omega s} ds = \left[x(s) e^{-jk\omega s}\right]_{t-T}^t + jk\omega \int^{t}_{t-T} x(s) e^{-jk\omega s} \nonumber\\[3mm]
	&= \left[x(s) e^{-jk\omega s}\right]_{t-T}^t + jk\omega \left< x\right>_k \label{eq:stft_derivative_1}
\end{align}

	Now note that by Leibnitz' Rule for Integrals,

\begin{align}
	\dfrac{d \left\langle x\right\rangle_k}{dt} = \dfrac{d}{dt} \int^{t}_{t-T} x(s)e^{-jk\omega s} ds &= \left[x(s)e^{-jk\omega s}\right]_{s=t}\dfrac{dt}{dt} - \left[x(s)e^{-jk\omega s}\right]_{\left(s=t-T\right)}\dfrac{d\left(t-T\right)}{dt} \nonumber\\[3mm]
	&= \left[x(s)e^{-jk\omega s}\right]_{s=t} - \left[x(s)e^{-jk\omega s}\right]_{\left(s=t-T\right)} = \left[x(s)e^{-jk\omega s}\right]^t_{t-T} \label{eq:stft_derivative_2}
\end{align}

	Therefore join \eqref{eq:stft_derivative_1} and \eqref{eq:stft_derivative_2},

\begin{equation} \left\langle \dfrac{dx(t)}{dt}\right\rangle_k = \dfrac{d\left\langle x\right\rangle_k}{dt} + jk\omega \left\langle x\right\rangle_k \label{eq:stft_derivative_3} \end{equation}

	\noindent meaning that the harmonics of derivatives can be obtained by the harmonics themselves. Thus much the same way as the Static Fourier Phasors, these properties allow for establishing relationships between the harmonics of voltage and current of linear devices. In the STFT-DPs, however, these relationships are not exactly impedances: because the relationship of the transformation of the derive must be written for every single order $k$, each harmonic order represents its own equation. Consider a voltage $v = m_v\cos\left(\omega t + \phi_v\right)$ over a capacitor of value $C$; then

\begin{equation} i(t) = C\dfrac{dv(t)}{dt} \Rightarrow \left\langle i\right\rangle_k = C\left[ \dfrac{d\left\langle v\right\rangle_k}{dt} + jk\omega\left\langle v\right\rangle_k\right] \label{eq:fdp_capacitive_equation}\end{equation}

	Now consider a current $i = m_i\cos\left(\omega t + \phi_i\right)$ through an inductor $L$; then

\begin{equation} v(t) = L\dfrac{di(t)}{dt} \Rightarrow \left\langle v\right\rangle_k = L\left[ \dfrac{d\left\langle i\right\rangle_k}{dt} + jk\omega\left\langle i\right\rangle_k\right] \label{eq:fdp_inductive_equation} \end{equation}

	If the phasors involved are static the differentials of the harmonics are null and these equations are equivalent to $\left\langle i\right\rangle_k = jk\omega C\left\langle v\right\rangle_k$ for the capacitor and  $\left\langle v\right\rangle_k = jk\omega L\left\langle i\right\rangle_k$ for the inductor, which coincide with the impedance equations of the static phasors. However, this new frame defines differential equations, meaning that the complexification of an electrical grid yields infinitely many complex differential systems, one for each harmonic order; the process of solving an electrical grid would then become an interative process for every single order $k$:

\begin{enumerate}
	\item Substitute inductances as differential equations of the form \eqref{eq:fdp_inductive_equation}, capacitances as differential equations of the form \eqref{eq:fdp_capacitive_equation}  and resistances as impedances $R$;
	\item Write the complex differential equations of the network;
	\item Substitute voltage and current sources by their STFT-DP equivalents;
	\item In the frequency domain, solve the complex differential equations of the node voltages and branch currents obtaining their equivalent phasors;
\end{enumerate}

	After this process, a sequence of harmonics as time functions indexed by $k$ is obtained; the inverse transform as defined in \eqref{def:inverse_fdp} is applied to obtain the equivalent signals in time.

%------------------------------------------------
\subsection{Shortcomings of Short-Time Fourier Dynamic Phasors} %<<<2

%-------------------------------------------------
\subsubsection{Gabor's Inequality} %<<<3

	One of the glaring questions pertaining to the effectiveness of the STFT is the choice $T$ of the length of the window. As an example, take the signal

\begin{equation} x(t) = u(-t) \cos\left(2\pi\times 2\times 10^3 \times t\right) + u(t) \cos\left(2\pi\times 4\times 10^3 \times t\right), \label{eq:stft_signal} \end{equation}

	\noindent where $u(t)$ is the heaviside step, which is a signal comprised of a sinusoid at the frequency 2kHz for $t < 0$, but at $t = 0$ changes to 4kHz. Figure \ref{fig:stft_heatmap} shows the \textit{heatmap} of the STFT of the signal \eqref{eq:stft_signal} using two sampling frequencies, a ``high frequency'' $f_H = 1kHz$ (top plot, pertaining to a shorter window length $T_h = f_h^{-1} = 1ms$), a ``low frequency'' $f_s = 100Hz$ (bottom plot, pertaining to a longer window length $T_s = f_s^{-1} = 10ms$) and a third ideal scenario (bottom plot).

% STFT DIFFERENT FREQUENCIES COMPARISON <<<
\definecolor{cfefefe}{RGB}{254,254,254}
\definecolor{cfdfdfd}{RGB}{253,253,253}
\definecolor{cfcfcfc}{RGB}{252,252,252}
\definecolor{cfbfbfb}{RGB}{251,251,251}
\definecolor{cfafafa}{RGB}{250,250,250}
\definecolor{cf9f9f9}{RGB}{249,249,249}
\definecolor{cf8f8f8}{RGB}{248,248,248}
\definecolor{cf7f7f7}{RGB}{247,247,247}
\definecolor{whitesmoke}{RGB}{245,245,245}
\definecolor{cf3f3f3}{RGB}{243,243,243}
\definecolor{cf2f2f2}{RGB}{242,242,242}
\definecolor{ce5e5e5}{RGB}{229,229,229}
\definecolor{ce4e4e4}{RGB}{228,228,228}
\definecolor{silver}{RGB}{192,192,192}
\definecolor{cbfbfbf}{RGB}{191,191,191}
\definecolor{ccccccc}{RGB}{204,204,204}
\definecolor{grey}{RGB}{128,128,128}
\definecolor{c7f7f7f}{RGB}{127,127,127}
\definecolor{ccccccc}{RGB}{204,204,204}
\definecolor{ccbcbcb}{RGB}{203,203,203}
\definecolor{ce3e3e3}{RGB}{227,227,227}
\definecolor{cf1f1f1}{RGB}{241,241,241}
\definecolor{cf6f6f6}{RGB}{246,246,246}
\definecolor{cf7f7f7}{RGB}{247,247,247}
\definecolor{ce4e4e4}{RGB}{228,228,228}
\definecolor{cf2f2f2}{RGB}{242,242,242}
\definecolor{cf4f4f4}{RGB}{244,244,244}
\begin{figure}[htb!]
\centering
\scalebox{0.975}{
	\input{data/stft/100}
}
	\caption
[Heatmap of the STFT transform of the example signal at two sampling frequencies and the ``ideal scenario''.]
{Heatmap of the STFT transform of signal \eqref{eq:stft_signal} at two sampling frequencies of $f_H = 1kHz$ (top plot) and at $f_L = 100Hz$ (middle plot). Bottom plot shows an ``ideal scenario'' composed of initely fine, instantly changing lines.}
	\label{fig:stft_heatmap}
\end{figure} %>>>

	The heatmap is a color plot where the absolute value of the STFT is shown in time in the band of frequencies; darker colors mean a higher absolute value. Thus it shows how the frequency spectrum is concentrated energy-wise as time passes. Reestated, taking a vertical line at $t = t_0$, each vertical point in the line shows the energy distribution along the frequency axis at the particular time instant $t_0$. Conversely, taking a horizontal line at $f = f_0$, each point in the horizontal line shows the time distribution of that particular frequency, that is, the evolution of the contribution of that particular frequency to the signal energy spectrum.

	At a first glance, because the signal is monotonic at a frequency of $2kHz$ at $t<0$ and at $4kHz$ at $t > 0$, one intuitively expects the STFT to show a harmonic concentrated at $2kHz$ for $t<0$ and $4kHz$ for $t\geq 0$. The heatmap would start as an infinitely thin black line at 2kHz, which at t = 0 immediately changes to another infinitely thin line at 4kHz; this is the ideal case depicted in figure \ref{fig:stft_heatmap}. Nevertheless such is not the case when the STFT is computed; as Figure \ref{fig:stft_heatmap} shows, at the higher sampling frequency, the heatmap is scattered vertically, but the horizontal scatter is small, that is, the frequency lines are thick and exists over a wide band of frequencies but the transition around t = 0 is quick. However, at the slower sampling frequency, the vertical scatter is small, but the horizontal scatter is large, that is, the the frequency lines are thinner but they linger for very long.

	In essence, what is happening is that the higher sampling frequency has a shorter window length; thus it can accurately detect the time intervals when frequency swings happen. However, since the time window is shorter, the window stretches high and varies too quickly, multiplying the signal and capturing multiple frequency bands; therefore, while there is precision in estimating the time instants where frequency swings happen, the amplitude of these frequency swings is not so well captured. The inverse happens when a shorter window length is used. This phenomenon is known as Gabor's Inequality.

\begin{theorem}[Gabor's Inequality or the Fundamental Principle of Communication \pcite{Gabor1946TheoryOC}]\label{theo:principle_comm} %<<<

	Let $\psi(t)\in\left[\mathbb{R}\to\mathbb{C}\right]$ some square-integrable complex signal, $\phi(\omega) = \mathbf{F}\left[\psi\right]$ its Fourier Transform. Define the quantities

\begin{equation} \omega^* = \sqrt{ \dfrac{\displaystyle\int \overline{\phi(\omega)} \omega^2 \phi(\omega) d\omega}{\displaystyle\int\overline{\phi(\omega)}\phi(\omega) d\omega }} = \sqrt{\dfrac{\displaystyle\int \overline{\dfrac{d\psi(t)}{dt}} \dfrac{d\psi(t)}{dt} dt}{\displaystyle\int \psi(t)\overline{\psi(t)}dt} } \end{equation}

	\noindent called \textbf{effective frequency} and

\begin{equation} t^* = \sqrt{\dfrac{\displaystyle\int \overline{\psi(t)}t^2\psi(t) dt}{\displaystyle\int \psi(t)\overline{\psi(t)}dt}} \end{equation}

	\noindent called the \textbf{mean epoch}, and the deviations

\begin{equation} \left\{\begin{array}{l} \Delta t = \sqrt{2\pi\left\langle\left(t - t^*\right)^2\right\rangle} \\[3mm] \Delta \omega = \sqrt{2\pi\left\langle\left(\omega - \omega^*\right)^2\right\rangle}\end{array}\right.  \end{equation}

	\noindent as the \textbf{effective duration} and \textbf{effective frequency width}, where $\left\langle\cdot\right\rangle$ represents average value. In short, these deviations are the mean RMS average of the signal with respect to the mean epoch, and the mean RMS average deviation of the signal spectrum with respect to the effective frequency, multiplied by $\sqrt{2\pi}$. Then

\begin{equation} \Delta t\ \Delta \omega \geq \pi . \label{eq:gabor_inequality}\end{equation}

\hrule
\end{theorem}
\vspace{5mm}
%>>>

	In short, for an arbitrary signal, the ``variation in frequency swings'' and the ``variation in interval'' that these frequency swings occur are closely related; reducing one means enlarging the other, and vice-versa. Further, \cite{Gabor1946TheoryOC} asks what is the signal that achieves the minimum value, that is, the signal that satisfies $\Delta t\Delta \omega = \pi$. That signal is

\begin{equation} \psi(t) = e^{-\alpha^2\left(t - t_0\right)^2}e^{j\left(\omega_0 t + \phi\right)} \label{eq:optimal_signal}\end{equation}

	\noindent where $\alpha,\omega,\phi$ are fixed, called \textbf{Gabor's Wavelet}. The Fourier Transform of this signal is

\begin{equation} \mathbf{F}\left[\psi\right] = \phi\left(\omega\right) = e^{-\left(\frac{2}{\alpha}\right)^2 \left(\omega - \omega_0\right)^2}e^{\left[-t_0\left(\omega - \omega_0\right) + \phi\right]} \label{eq:optimal_signal_fourier}\end{equation}

	Thus the shapes of the signal amplitude and its spectrum are gaussian curves, with $t_0$and $\omega_0$ means as shown in figure \ref{fig:gabor_minimum_signal}. The mean epoch and effective frequency are

\begin{equation} \Delta t = \sqrt{\dfrac{\pi}{2}}\dfrac{1}{\alpha},\ \Delta\omega = \sqrt{2\pi} \alpha \end{equation}

	\noindent and, indeed, one notices that this signal achieves the equality of \eqref{eq:gabor_inequality}.

% MINIMUM SIGNAL CURVES <<<
\begin{figure}[htb]
\centering
	\begin{tikzpicture}[scale=1,>={Stealth[inset=0mm,length=1.5mm,angle'=50]}]
		\begin{axis}[
			at={(0mm,0mm)},
			width= 75mm,
			height=75mm,
			tick style={draw=none},
			axis lines=middle,
			xmin = -0.2, xmax = 2.2,
			ymin = -0.2, ymax = 1.1,
 			xticklabel={\empty}, yticklabel={\empty},
			xlabel = {$t$},
			ylabel = {$\left\lvert\psi(t)\right\rvert$}
			]
			\addplot  [thick, stewartblue, domain=0:2, smooth, samples=100] {e^(-16*(x-1)^2)};
			\addplot  [thick, stewartblue, domain=0:2, dashed, line cap=round, samples=100] coordinates {(1,0)(1,1)};

			\addplot  [thick, black, dashed, line cap=round, samples=100] coordinates {(   1+(1/8)     ,  { e^(-16*( 1 + (1/8) - 1)^2) + 0.05})( 1 + (1/8) ,1.09)};
			\addplot  [thick, black, dashed, line cap=round, samples=100] coordinates {(   1-(1/8)     ,  { e^(-16*( 1 - (1/8) - 1)^2) + 0.05})( 1 - (1/8) ,1.09)};
			\draw[<-] (axis cs: {1 + 1/8} ,1.04) -- (axis cs: {1 + 1/8 + 0.1} ,1.04);
			\draw[<-] (axis cs: {1 - 1/8} ,1.04) -- (axis cs: {1 - 1/8 - 0.1} ,1.04);
			\node at(axis cs:1,1.05) {$\Delta t$};
			\node at(axis cs:1,-0.1) {$t_0$};
		\end{axis}
		\begin{axis}[
			at={(75mm,0mm)},
			width= 90mm,
			height=75mm,
			tick style={draw=none},
			axis lines=middle,
			xmin = -0.2, xmax = 3.2,
			ymin = -0.2, ymax = 1.1,
 			xticklabel={\empty}, yticklabel={\empty},
			xlabel = {$\omega$},
			ylabel = {$\left\lvert\phi(\omega)\right\rvert$}
			]
			\addplot  [thick, stewartgreen, domain=0:3, smooth, samples=100] {0.8*e^(-4*(x-1.2)^2)};
			\addplot  [thick, black, dashed, line cap=round, samples=100] coordinates {(   1.2+(0.25)/(0.8^2)     ,  { 0.8*e^(-4*( 1.2 + (0.25)/(0.8^2) - 1.2)^2) + 0.05})( 1.2 + (0.25)/(0.8^2) ,1.05)};
			\addplot  [thick, black, dashed, line cap=round, samples=100] coordinates {(   1.2-(0.25)/(0.8^2)     ,  { 0.8*e^(-4*( 1.2 - (0.25)/(0.8^2) - 1.2)^2) + 0.05})( 1.2 - (0.25)/(0.8^2) ,1.05)};
			\draw[<->] (axis cs: {1.2 + (0.25)/(0.8^2)} ,1) -- (axis cs: {1.2 - (0.25)/(0.8^2)} ,1);
			\addplot  [thick, stewartgreen, line cap=round, domain=0:2, dashed, samples=100] coordinates {(1.2,0)(1.2,0.8)};
			\node at(axis cs:1.2,-0.1) {$\omega_0$};
			\node at(axis cs:1.2,1.05) {$\Delta\omega$};
		\end{axis}
	\end{tikzpicture}
	\caption
[Amplitude and spectrum of the ``optimal signal'' \eqref{eq:optimal_signal}.]
{Amplitude and spectrum of the ``optimal signal'' \eqref{eq:optimal_signal} showing the mean epoch and effective frequency as well as the mean deviations.}
	\label{fig:gabor_minimum_signal}
\end{figure} %>>>

	Due to this statistical property of the wavelet, it can work as a generator for a kernel called Gabor Kernel using a reference version of \eqref{eq:optimal_signal} where $\alpha = \sqrt{\pi}$ and $\phi = 0$:

\begin{equation} \mu_G\left(s, t,\omega\right) = e^{-\pi\left(s - t_0\right)^2}e^{j\omega_0 s} \label{eq:gabor_kernel}\end{equation}

	\noindent incepting the Gabor Transform, defined as

\begin{equation} \mathbf{G}\left[x\right] = X\left(t,\omega\right) = \int_\mathbb{R} x(s)\overline{\mu_G\left(s, t,\omega\right)}ds = \int_\mathbb{R} x(s)e^{-\pi\left(s - t\right)^2}e^{-j\omega s}ds \label{eq:gabor_transform} ,\end{equation}

	\noindent which one can recognize as a STFT \eqref{eq:stft_generic} with a gaussian window. This transform has been used extensively in the literature, for instance in Power Quality assessment \pcite{szmajdaGaborTransformGaborWigner2010}, detection of radar signals \pcite{shu-longjiDetectionRadarSignals1992}, representation of time systems \pcite{rotsteinGaborTransformTimevarying1999} and image processing \pcite{jieyaoGeneralizedGaborTransform1995}.

	Thus, in short, it is the nature of the Fourier Transform process — which is obviously underlying to the STFT decomposition — that there is a tradeoff between the ``frequency resolution'' and the ``time resolution'' acquired from the transform. This happens even in ``simple signals'', like that of \eqref{eq:stft_signal}, and it means that this transform is inexorably inefficient in capturing transient nonstationary phenomena in various timescales: a great many details are needed to find a particular window length (or frequency signal) that adheres to a satisfactory compromise between both. Even when the optimal windowing is used, in the for of Gabor's Transform, the issue persists. This is a particular problem for Power Systems, because transient effects in such systems can manifest in various bandwidths and timescales, meaning that the choice of a particular frequency signal means certain phenomena will probably be neglected for precision in time resolution in a particular frequency band.

%-------------------------------------------------
\subsubsection{Infinite complex systems}\label{subsec:infinite_complex_systems} %<<<2

	Another downside of the STFT framework is that the voltage-current relationships it implies define infinite complex differential systems, one for each harmonic; this can be seen as both a downside and a benefit — the capability of modelling harmonics in realtime, while being useful for the analysis of power quality during operation of electrical grids, makes it impractical to model the grid in simulations. It is immediate to notice that this process represents too much a sophistication to be considered practical: the solution of infinite complex differential systems is most certainly computationally impossible. In order to solve this, the simplifying hypotheses that the signals involved are mainly concentrated in the first harmonic, or fundamental, is made; under this assumption all higher-order harmonics can be ignored for practical purposes, summarizing the analysis to a single differential system. To this regard, the accuracy of the transform is be sacrificed, as the higher harmonics supposed innocuous and only the fundamental harmonic dominating as per equations \eqref{sys:fdp_sys_fundamental_harmonic}. Thus the voltage-current relationships become

\begin{equation} \left\{\begin{array}{l} I = C\dfrac{dV}{dt} + j\omega CV \\[5mm] V = L\dfrac{dI}{dt} + j\omega LI \end{array}\right. \label{sys:fdp_sys_fundamental_harmonic} \end{equation}

	\noindent where the capital $V$ and $I$ are notations for the fundamental harmonics $\left\langle v\right\rangle_1$ and $\left\langle i\right\rangle_1$, producing the ``approximated'' Dynamic Phasors of voltage and current $V$ and $I$, which will be called the ``STFT Dynamic Phasors'' or simply STFT-DPs. This process then abdicates precision for conveniency, that is, is gives approximations to the solution of the time Differential Equations to the Electrical Grid being modelled; inasmuch as they allow traditional phasorial representation, the phasors they represent guaranteed not to mirror signals in time, but only approximately — underwhelmingly so, for while Static Phasors can be proven to perfectly reconstruct the stable steady-state solution of the time ODEs of the grid. 

	It is natural to ask what is the effectiveness of this approximation — how close the signals reconstructed from STFT-DPs are from the signals that solve the time differential equations of the grid. In \cite{volpatoDynamicPhasorTransform2022}, I and Prof. Luís prove that this approximation is valid under the Quasi-Stationary Hypothesis, that is, if the magnitude and frequency signals are ``slow'', the signals reconstructed from these phasors approximate the solution of these ODEs. The proof uses a concept of bandwidth of the modulus and phase angle; once these bandwidths get smaller and the signals get ``slower'', the Dynamic Phasor of the original signal $x$ gets arbitrarily close to an ``averaged'' signal $x_A$ which amplitude is the average amplitude of $x(t)$ and which phase is the average phase. This proof is shown in theorem \ref{theo:fdp_quasi_static}. Further, in \cite{volpatoDynamicPhasorTransform2022} we then prove that this implies that $x$ gets arbitrarily close to the static phasor that represents the solutions to the original time differential equations of the system.

\begin{theorem}[Quasi-static modelling of STFT-DPs \pcite{volpatoDynamicPhasorTransform2022}]\label{theo:fdp_quasi_static}%<<<
Let 

\begin{equation} x(t) = m(t)\cos\left(\omega(t) t + \phi(t)\right) \end{equation}

	\noindent be a nonstationary sinusoid where $m$ and $\phi$ are $C^1$-class. Let the approximated Dynamic Phasor $x_A$ be calculated as

\begin{equation} x_A(t) = \sum_{k\in\left\{-1,1\right\}} \dfrac{1}{2} m_A (t) e^{j\phi_A(t)} e^{jk\omega t} = m_A(t)\cos\left(\omega(t) t + \phi_A(t)\right), \end{equation}

	\noindent where $m_A$ and $\phi_A$ are the averaged approximations of the modulus and phase angle during $\left[t-T,t\right]$:

\begin{equation}
\left\{\begin{array}{l}
	m_A(t) = \dfrac{1}{T} \displaystyle\int_{t-T}^t m(\tau)d\tau \\[5mm]
	\phi_A(t) = \dfrac{1}{T} \displaystyle\int_{t-T}^t \phi(\tau)d\tau
\end{array}\right. .
\end{equation}

	Then 

\begin{equation} \left\lvert x(t) - x_A(t) \right\lvert  < \dfrac{4\pi \left(B_m + B_\phi \left\lvert m_A\right\rvert\right)}{\omega} \left(1 + 4\pi \dfrac{B_\phi}{\omega}\right) \left(1 + \dfrac{\pi^2}{3}\right), \end{equation}

	where $B_m$ and $B_\phi$ are the bandwidths of the modulus and phase angle signals defined as


\begin{equation} B_z = \sup_{\left[t-T,t\right]} \left\lvert \dfrac{dz(t)}{dt}\right\rvert \end{equation}
\end{theorem}
\hrule
\vspace{5mm}
%>>>

%-------------------------------------------------
\subsubsection{Power signals} %<<<2

	Finally, the Fourier Dynamic Phasors are unable to give an alternative to theorem \ref{theo:sfp_complex_apparent_power}, that is, the Dynamic Phasors are unable to prove that the complex power induced by their inner product does not reflect the instantaneous power developed by the circuit being studied. Indeed, if $\left(\left\langle v\right\rangle_k\right)_{k\in\mathbb{Z}}$ and $\left(\left\langle i\right\rangle_k\right)_{k\in\mathbb{Z}}$ represent the time-varying harmonics of voltage and current, then the harmonics of the power (their product) are calculated by a convolution

\begin{equation} \left\langle p\right\rangle_k = \sum_{m\in\mathbb{Z}} \left\langle v\right\rangle_m\left\langle i\right\rangle_{(k-m)} \label{eq:stft_power_def}\end{equation}

	\noindent and it is obvious that extracting components like active and reactive power from that is not possible. Using the single-harmonic approximation, that is, supposing  $\left\lvert\left\langle x\right\rangle_k\right\rvert \ll \left\lvert\left\langle x\right\rangle_1\right\rvert$ for both $v$ and $i$, then using the approximations on \eqref{eq:stft_power_def} yields that for the k-th power harmonic, 

\begin{equation} \left\lvert\left\langle p\right\rangle_k\right\rvert \ll \left\lvert\left\langle p\right\rangle_1\right\rvert \text{ and } \left\langle p\right\rangle_1 \approx \left\langle v\right\rangle_1\left\langle i\right\rangle_1 . \end{equation}

	\noindent and one can define 

\begin{equation} S = \frac{1}{2}\left\langle p\right\rangle_1,\ P = \text{Re}\left(S\right),\ Q = \text{Im}\left(S\right) .\end{equation}

	However, it is obvious that the first-harmonic approximation is particularly problematic here, because the errors commited by approximating both current and voltage propagate throughout all harmonics of $p$. This is particularly grievous in Power Systems because virtually all modern electric power systems are equipped with active and reactive power control units, like Droop control or Maximum Power Tracking Point algorithms, meaning complex power must be a direct reflection of the actual instantaneous power otherwise these power controls are not guaranteed to work properly.

	If another window that is not the boxcar window is used, then the situation becomes more difficult. Supposing a continuous arbitrary window, then the convolution becomes the integral on the frequency space of the Dynamic Phasors $V\left(t,\omega\right)$  for voltage and $I\left(t,\omega\right)$ for current:

\begin{equation} P\left(t,\omega\right) = \int_{\mathbb{R}} V\left(t,\omega - \kappa\right) I\left(t,\kappa\right) d\kappa ,\label{eq:stft_continuous_power_def}\end{equation}

	\noindent making the process of obtaining an expression for $P$ more difficult, let alone even extracting the active and reactive components. As discussed in the introduction of this thesis, the literature features several attempts to solve this conundrum, most of which involve quite contrived representations of complex power as higher complex algebras or employing distortion as a component of power — none of which theories have been been quite adopted in the literature in the literature.

%--------------------------------------------------------------------------------------------------
\section{Representation of sinusoidal signals using the Hilbert Transform}\label{chapter:hilbert_transform} %<<<1
%--------------------------------------------------------------------------------------------------

%-------------------------------------------------
\subsection{Analytical Representation of real signals} %<<<2

%	In Complex Analysis, an analytic complex function $f(z)$ at some particular open set $D$ is that which for every $z_0\in D$, $f(z)$ is locally given by some convergent power series, that is, there exists a complex sequence $\left(a_k\right)_{k\in\mathbb{N}}$ such that
%
%\begin{equation} f(z) = \sum_{k\in\mathbb{N}} a_k\left(z-z_0\right)^k .\end{equation}
%
%	It is a fundamental result of Complex Analysis that any complex-differentiable (holomorphic) function in an open set $D$ is analytic and vice-versa; moreover, another big result is that a complex-differentiable function in an open set is also infinitely differentiable. As such, in the space of complex functions, holomorphicity and analicity are equivalent.
%
%\begin{definition}[Holomorphic complex functions] %<<<
%	For an arbitrary $f\in\left[\mathbb{C}\to\mathbb{C}\right]$, the following definitions are equivalent:
%
%\begin{itemize}
%	\item $f$ is analytic in an open set $D$, that is, for every $z_0\in D$ there exists a convergent power series for $f$;
%	\item $f$ is holomorphic in $D$, that is, it is infinitely differentiable in a vicinity of any $z_0\in D$;
%	\item $f$ and its derivatives are polynomically limited, that is, for every compact $K\subset D$ there exists some constant $k$ such that for any $z_0\in K$ and $i\in\mathbb{N}$,
%	\begin{equation} \left\lvert \dfrac{d^if}{dz^i}\left(z_0\right)\right\rvert \leq i!k^i \end{equation}
%\end{itemize}
%\end{definition} %>>>
%
%	It is clear from the definition that analicity is a very strong condition for a complex function, giving it many properties.

	It is known, and simple to inspect, that the Fourier Transform of an arbitrary real signal $x(t)$ has negative frequency components \pcite{smithMathematicsDiscreteFourier2007} that are symmetric with respect to conjugation, that is, the Fourier Transform $X\left(\omega\right) = \mathbf{F}\left[x\right]$ is Hermitian symmetric:

\begin{equation} X\left(-\omega\right) = \overline{X\left(\omega\right)} \end{equation}

	\noindent meaning that the negative frequency parts can be safely discarded because they can be reconstructed from the positive frequency components. Thus a signal with only positive frequency spectrum can be built as $S\left(\omega\right)$:

\begin{equation} S\left(\omega\right) = X\left(\omega\right) \left[ 1 + \text{sgn}\left(\omega\right)\right] = \left\{\begin{array}{l} 2X\left(\omega\right), \text{ if } \omega > 0 \\[3mm] X\left(\omega\right), \text{ if } \omega = 0 \\[3mm] 0, \text{ if } \omega < 0 \\[3mm]\end{array}\right. \end{equation}

	\noindent which contains only the positive frequency components of $X\left(\omega\right)$. The operation is revertible because $X\left(\omega\right)$ can be obtained from $S\left(\omega\right)$ through

\begin{equation} X\left(\omega\right) = \left\{\begin{array}{l} \frac{1}{2} S\left(\omega\right), \text{ if } \omega > 0 \\[3mm] S\left(\omega\right), \text{ if } \omega = 0 \\[3mm] \frac{1}{2}\overline{ S\left(-\omega\right)}, \text{ if } \omega < 0 \\[3mm]\end{array}\right. = \dfrac{ S\left(\omega\right) + \overline{S\left(-\omega\right)} }{2} .\end{equation}

	Naturally one wonders what signal does $S\left(\omega\right)$ reconstruct. This signal is

\begin{align} \mathbf{F}^{-1}\left[S\right] &= \mathbf{F}^{-1}\left[X\left(\omega\right) + X\left(\omega\right) \text{sgn}\left(\omega\right)\right] = \mathbf{F}^{-1}\left[X\right] + \mathbf{F}^{-1}\left[X\left(\omega\right) \text{sgn}\left(\omega\right)\right] = \nonumber\\[3mm] &= x(t) + \overbrace{\mathbf{F}^{-1}\left[X\right] * \mathbf{F}^{-1}\left[\text{sgn}\left(\omega\right)\right]}^{\text{Convolution}} = x(t) + x(t) * \left(j\dfrac{1}{\pi t} \right) = x(t) + j \left[x(t) * \dfrac{1}{\pi t}\right]\end{align}

	\noindent and by definition this convolution is given by

\begin{equation} x(t) * \dfrac{1}{\pi t} = \dfrac{1}{\pi}\int_\mathbb{R} \dfrac{x\left(\tau\right)}{t-\tau} d\tau \label{eq:hilb_transf_def}\end{equation}

	\noindent which is the naïve definition of the \textbf{Hilbert Transform} of the signal $x(t)$, denoted $\mathbf{H}\left[x\right]$. Therefore, the \textbf{Analytic Representation} of $x(t)$, defined as the signal reconstructed from $S\left(\omega\right)$ is given by

\begin{equation} x_a(t) = x(t) + j\mathbf{H}\left[x\right] .\end{equation}

	This process is known as \textbf{Hilbert Filtering} , that is, ``removing'' the negative frequencies. The objective now becomes to show that this representation yields desirable properties; most importantly, it allows for easily obtaining modulation and de-modulation techniques.

	It is simple to see that this process yields some idea of a time-varying phasor; for instance, one can adopt $m_x(t) = \left\lvert x_a(t)\right\rvert$ as the time-varying amplitude of $x(t)$, also called an \textbf{envelope}. Further, one can adopt $\phi_x(t) = \arg\left(x_a(t)\right)$ as the time-varying phase and $\omega_x = \dot{\phi_x}$ as the equivalent time-varying frequency in radians or $f_x = \omega_a/2\pi$ as its value in hertz.

\begin{example}[Analytical signal of $e^{-x^2}\cos\left(20\pi x\right)$]\label{example:hilbert_analytical}

	The analytical signal $x_a(t)$ corresponding to

\begin{equation} x(t) = e^{-t^2}\cos\left(2\pi\times 10\times t\right) \end{equation}

	\noindent is such that $x_a = x(t) + j\mathbf{H}\left[x\right] = e^{-t^2}e^{j2\pi\times 10\times t}$, which is a Gabor Wavelet. The plots of $x(t)$, $\left\lvert x_a\right\rvert$ and $f_a = \omega_a/2\pi$ are shown in Figure \ref{fig:analytical_example}.

% HILBERT TRANSFORM ENVELOPE <<<
\begin{figure}[htb]
\centering
	\begin{tikzpicture}
	\begin{axis}[
		at = {(0,65mm)},
		width=150mm,
		height=75mm,
		%tick style={draw=none},
		axis lines=middle,
		xtick = {-4,-6,...,4},
		ytick = {-1,1},
		x tick style={color=black},
		y tick style={color=black},
		xmin = -4.2, xmax = 4.2,
		ymin = -1.1, ymax = 1.1,
		xlabel = {$t\ \left(s\right)$},
		%xlabel style = {at={(axis cs:50,0.5)},anchor=north},
		ylabel = {$x(t) = e^{-t^2}\cos\left(2\pi\times 2\times t\right)$},
		ylabel style = {at={(axis cs:0.2,1)},anchor=west},
		trig format=rad
		]
		%\addplot  [thick, stewartblue, domain=-0.25:7, smooth, samples=100] {0.1*((0.9*x - 1.4)^3 - 3*(x - 1.4)^2 - 0.4*(x-1.4) + 15)};
		\addplot[color=stewartblue, domain=-4:4, samples=2000, smooth, thick] {exp(-x^2)*cos(2*pi*2*x)};
	\end{axis}

	\begin{axis}[
		at = {(0,0)},
		width=150mm,
		height=75mm,
		%tick style={draw=none},
		axis lines=middle,
		xtick = {-4,-6,...,4},
		ytick = {1},
		x tick style={color=black},
		y tick style={color=black},
		xmin = -4.2, xmax = 4.2,
		ymin = -0.2, ymax = 1.1,
		xlabel = {$t\ \left(s\right)$},
		%xlabel style = {at={(axis cs:50,0.5)},anchor=north},
		ylabel = {$\left\lvert x_a(t)\right\rvert$},
		ylabel style = {at={(axis cs:0.2,1)},anchor=west},
		trig format=rad
		]
		\addplot[color=stewartgreen,smooth, samples=1000, thick] {exp(-x^2)};
	\end{axis}
	\begin{axis}[
		at = {(0,-65mm)},
		width=150mm,
		height=75mm,
		%tick style={draw=none},
		axis lines=middle,
		xtick = {-4,-6,...,4},
		ytick = {0,0.5,...,2},
		x tick style={color=black},
		y tick style={color=black},
		xmin = -4.2, xmax = 4.2,
		ymin = 0, ymax = 2.2,
		xlabel = {$t\ \left(s\right)$},
		%xlabel style = {at={(axis cs:50,0.5)},anchor=north},
		ylabel = {$f_a (Hz)$},
		ylabel style = {at={(axis cs:0.1,2.1)},anchor=west}
		]
		\addplot[color=stewartpink,smooth, thick] {2};
	\end{axis}
	\end{tikzpicture}
	\caption{Example of signal and its analytic correspondent.}
	\label{fig:analytical_example}
\end{figure} %>>>

\examplebar
\end{example}

%-------------------------------------------------
\subsection{The Cauchy Principal Value} %<<<2

	The main operational problem with the Hilbert Transform, as defined in \eqref{eq:hilb_transf_def}, is that for limited signals its integral is not defined because the denominator has a singularity at $t=\tau$. This makes computing the transform of the simplest signals like polynomials and sinusoids impossible. In order to circumvent that, the Cauchy Principal Value, or simply principal value, is used. This is mathematical tool in Theory of Singularities that allows assigning some value to certain improper integrals that would otherwise remain undefined. The definition of the principal value depends on the particular function it is applied to: let $f(x)\in\left[\mathbb{R}\to\mathbb{C}\right]$ a complex function such that $b$ is a singularity of $f$, that is, $f(b)$ is not defined. Suppose $b$ is some finite real number and consider the interval $\left[a,c\right]$ containing $b$, where

\begin{equation} \lim\limits_{\varepsilon\to 0^+} \int_{a}^{b-\varepsilon} f(x)dx = \pm \infty\end{equation}

	\noindent and

\begin{equation} \lim\limits_{\varepsilon\to 0^+} \int_{b+\varepsilon}^{c} f(x)dx = \mp \infty,\end{equation}

	\noindent which is to say that the improper integrals from $a$ to $b$ and from $b$ to $c$ diverge with opposite signs. Because of this, it is obvious that the integral

\begin{equation} \int_{a}^{c} f(x)dx \label{eq:hilbert_transform_impossible_integral} \end{equation}

	\noindent cannot be defined because of the difficult point $b$. Then the Cauchy Principal Value of $f$ on $\left[a,c\right]$ is denoted as a ``dashed integral'' and defined as

\begin{equation} \dashint_{a}^{c} f(x)dx = \lim\limits_{\varepsilon\to 0^+} \left[\int_{a}^{b-\varepsilon} f(x)dx + \int_{b+\varepsilon}^{c} f(x)dx  \right]\end{equation}

	\noindent which is a way of assigning some value to the divergent integral \eqref{eq:hilbert_transform_impossible_integral}. To integrate $f$ over the reals, then

\begin{equation} \dashint_{\mathbb{R}} f(x)dx = \dashint_{-\infty}^{\infty} f(x)dx = \lim\limits_{\varepsilon\to 0^+} \left[ \lim\limits_{a\to -\infty} \int_{a}^{b-\varepsilon} f(x)dx + \lim\limits_{c\to\infty} \int_{b+\varepsilon}^{c} f(x)dx  \right]\end{equation}

	For instance, let $f(x) = 1/\left(x-b\right)$, where $b$ is some real: famously, the integral of this function on any interval containing $b$ cannot be defined due to the divergent nature at $b$. Taking the principal value, for any real or infinite $a$, 

\begin{equation} \dashint_{-a}^{a} \dfrac{1}{x-b}dx = 0. \end{equation}

	Now suppose that the singularity of $f$ is located at infinity; then

\begin{equation} \dashint_{-\infty}^{\infty} f(x)dx = \lim\limits_{a\to\infty} \left[\int_{-a}^{a} f(x)dx \right], \end{equation}

	\noindent where

\begin{equation} \lim\limits_{a\to\infty} \int_{-a}^{0} f(x)dx = \pm \infty\end{equation}

	\noindent and

\begin{equation} \lim\limits_{a\to\infty} \int_{0}^{a} f(x)dx = \mp \infty .\end{equation}

	For instance, integrating the sine function over the reals yields an undefined improper integral because the function is underfined at infinity. Applying the principal value yields

\begin{equation} \dashint_{-\infty}^{\infty} \sin\left(x\right) dx = \lim\limits_{a\to\infty} \left[\int_{-a}^{a} \sin\left(x\right) dx \right] = 0 . \end{equation}

	Finally, in the more difficult case where $f$ has a singularity at both infinity and a finite point $b$, then the definition needs to consider the particular point $b$ and the singularity at infinity:

\begin{equation} \dashint_{\mathbb{R}} f(x)dx = \lim\limits_{a\to\infty} \left\{ \lim\limits_{\varepsilon\to 0^+} \left[\int_{b-a}^{b-\varepsilon} f(x)dx + \int_{b+\varepsilon}^{b+a} f(x)dx \right] \right\}, \end{equation}

	\noindent which is the more general definition.

\begin{example}[CPV of the Sinc function] %<<<
	Take the sinc function

\begin{equation} f\left(x\right) = \dfrac{\sin\left(x\right)}{x}\end{equation}

	\noindent for some real number $b$. Then

\begin{equation} \dashint_{\mathbb{R}} \dfrac{\sin\left(x\right)}{x}dx = \lim\limits_{a\to\infty} \left\{ \lim\limits_{\varepsilon\to 0^+} \left[\int_{-a}^{-\varepsilon} \dfrac{\sin\left(x\right)}{x}dx + \int_{\varepsilon}^{a} \dfrac{\sin\left(x\right)}{x}dx \right] \right\} \end{equation}

	Because both sine and $1/x$ are odd functions, the sinc function is even, meaning

\begin{equation} \dashint_{\mathbb{R}} \dfrac{\sin\left(x\right)}{x}dx = 2\lim\limits_{a\to\infty} \left\{ \lim\limits_{\varepsilon\to 0^+} \left[ \int_{\varepsilon}^{a} \dfrac{\sin\left(x\right)}{x}dx \right] \right\} \end{equation}

	Now, because

\begin{equation} \lim\limits_{x\to 0} \dfrac{\sin\left(x\right)}{x}dx = 1, \end{equation}

	\noindent then the limit on $\varepsilon$ can be removed:

\begin{equation} \dashint_{\mathbb{R}} \dfrac{\sin\left(x\right)}{x}dx = 2\lim\limits_{a\to\infty} \int_{0}^{a} \dfrac{\sin\left(x\right)}{x}dx  \end{equation}

	Despite this antiderivative function having no analytic expression, it is known that its improper version converges and is equal to

\begin{equation} \lim\limits_{a\to\infty} \int_{0}^{a} \dfrac{\sin\left(x\right)}{x}dx = \dfrac{\pi}{2} \end{equation}

	\noindent meaning

\begin{equation} \dashint_{\mathbb{R}} \dfrac{\sin\left(x\right)}{x}dx = \pi. \end{equation}

\examplebar
\end{example} %>>>

	Finally, the formal definition of the Cauchy Principal Value induces the definition of the Hilbert Transform.

\begin{definition}[Hilbert Transform] \label{def:hilbert_transform} %<<<

	The Hilbert Transform of a complex function $x\in\left[\mathbb{R}\to\mathbb{C}\right]$ is defined as the principal value of the convolution of $x$ with the function $1/\pi t$, called the Cauchy Kernel:

\begin{equation} \mathbf{H}\left[x\right] = \dfrac{1}{\pi}\hspace{2mm} \dashint_{\mathbb{R}} \dfrac{x\left(\tau\right)}{t - \tau} d\tau \end{equation}

	\noindent provided that this integral exists as a principal value.

\end{definition} 
%>>>

%-------------------------------------------------
\subsection{Properties and application to signals of interest} %<<<2

	The operational properties of the Hilbert Transform are immediate from its definitions. For instance, its linearity: 

\begin{equation} \mathbf{H}\left[x + \alpha y\right] = \dfrac{1}{\pi}\hspace{2mm} \dashint_{\mathbb{R}} \dfrac{x\left(\tau\right) + \alpha y\left(\tau\right)}{t - \tau} d\tau \end{equation}

	\noindent because the limits of the definition \ref{def:hilbert_transform} are linear, as wel as the integral, this means that the principal value is linear:

\begin{align}
	\mathbf{H}\left[x + \alpha y\right] &= \dfrac{1}{\pi} \left[\hspace{2mm}\dashint_{\mathbb{R}} \dfrac{x\left(\tau\right)}{t - \tau} d\tau + \dashint_{\mathbb{R}} \dfrac{\alpha y\left(\tau\right)}{t - \tau} d\tau\right] = \nonumber\\[3mm]
	&= \dfrac{1}{\pi} \left[\hspace{2mm}\dashint_{\mathbb{R}} \dfrac{x\left(\tau\right)}{t - \tau} d\tau\right]  + \dfrac{1}{\pi}\left[\hspace{2mm}\dashint_{\mathbb{R}} \dfrac{\alpha y\left(\tau\right)}{t - \tau} d\tau\right] = \nonumber\\[3mm]
	&= \dfrac{1}{\pi} \left[\hspace{2mm}\dashint_{\mathbb{R}} \dfrac{x\left(\tau\right)}{t - \tau} d\tau\right]  + \dfrac{1}{\pi}\alpha\left[\hspace{2mm}\dashint_{\mathbb{R}} \dfrac{y\left(\tau\right)}{t - \tau} d\tau\right] = \nonumber\\[3mm]
	&= \mathbf{H}\left[x\right] + \alpha\mathbf{H}\left[y\right]
\end{align}

	As for derivatives, we smartly swap the convolution variables

\begin{equation} \dfrac{d}{dt}\mathbf{H}\left[x\right] = \dfrac{d}{dt} \left[\hspace{2mm}\dashint_{\mathbb{R}} \dfrac{x\left(t-\tau\right)}{\tau} d\tau \right] \end{equation}

	\noindent and using Leibnitz' Rule for integrals, since the integrand is on $\tau$ and not on $t$,

\begin{equation} \dfrac{d}{dt}\mathbf{H}\left[x\right] = \dashint_{\mathbb{R}} \dfrac{d}{dt} \left[\dfrac{x\left(t-\tau\right)}{\tau} \right]d\tau =  \dashint_{\mathbb{R}} \dfrac{x'\left(t-\tau\right)}{\tau} d\tau = \mathbf{H}\left[\dfrac{dx}{dt}\right] \label{eq:hilvert_diff}\end{equation}

	\noindent which is to say that the differential functional and the Hilbert Transform commute: $\mathbf{D}_\mathbb{C}\circ\mathbf{H} = \mathbf{H}\circ\mathbf{D}_\mathbb{C}$.

	In the representation of phasorial quantities, the most important property of the Hilbert transform is the fact that the transform of the exponential complex function yields a quadrature signal:

\begin{equation}
	 \mathbf{H}\left[e^{j\omega t} \right] = \left\{\begin{array}{l} e^{j\left(\omega t - \frac{\pi}{2}\right)}, \text{ if } \omega > 0 \\[5mm] e^{j\left(\omega t + \frac{\pi}{2}\right) } , \text{ if } \omega < 0 \end{array}\right.
\end{equation}


	\noindent which in turn means that the transform of a sine or cosine yields a quadrature sine or cosine, that is,

\begin{gather}
	 \mathbf{H}\left[\cos\left(\omega t + \phi\right)\right] = \left\{\begin{array}{l} \cos\left(\omega t + \phi - \dfrac{\pi}{2} \right) = \sin\left(\omega t + \phi\right), \text{ if } \omega > 0 \\[5mm] \cos\left(\omega t + \phi + \dfrac{\pi}{2} \right) = -\sin\left(\omega t + \phi\right), \text{ if } \omega < 0 \end{array}\right. \\[10mm]
%
	 \mathbf{H}\left[\sin\left(\omega t + \phi\right)\right] = \left\{\begin{array}{l} \sin\left(\omega t + \phi - \dfrac{\pi}{2} \right) = -\cos\left(\omega t + \phi\right), \text{ if } \omega > 0 \\[5mm] \sin\left(\omega t + \phi + \dfrac{\pi}{2} \right) = \cos\left(\omega t + \phi\right), \text{ if } \omega < 0  \end{array}\right.
\end{gather}

	\noindent for a positive $\omega$. This property allows, in turn, the definition of a ``quadrature signal'' of periodic functions; let $f(t)$ be a function of period $T$. Taking the Fourier Series of $f$ yields

\begin{equation} f(t) = \sum_{k\in\mathbb{Z}} a_k e^{jk\omega t}, a_k = \dfrac{1}{T} \int_{T} f(x)e^{-jk\omega x}{dx}\end{equation}

	Therefore

\begin{equation}  \mathbf{H}\left[ f(t) \right] = \mathbf{H}\left[ \sum_{k\in\mathbb{Z}} a_k e^{jk\omega t}\right] = \sum_{k\in\mathbb{Z}} a_k e^{j\left(k\omega - \sign\left(k\right)\frac{\pi}{2}\right)} = -\sum_{k\in\mathbb{Z}}\sign\left(k\right) a_k j e^{jk\omega} \end{equation}

	In the specific realm of Dynamic Phasor Theory, the most explored property of the Hilbert Transform is the Bedrosian Identity, as presented in theorem \ref{theo:extended_bedrosian}.

\begin{definition}[Support of a complex function]
	Let $f\in\left[\mathbb{R}\supseteq X\to \mathbb{C}\right]$; then the \textbf{support} of $f$, denoted $\supp\left(f\right)$, is the closure of its complementary pre-image of zero, that is,

\begin{equation} \supp\left(f\right) = \overline{\left\{x\in X: f\left(x\right)\neq 0\right\}} \end{equation}

	\noindent where the overline represents the closure of a set.

\end{definition}
\begin{theorem}[Extended Bedrosian Identity \pcite{Xu2006}]\label{theo:extended_bedrosian} %<<<
	Let $a\leq 0$, $b \geq 0$ and $f,g\in L^2\left(\mathbb{R}\right)$ such that

\begin{equation} \supp\left(\mathbf{F}\left[f\right]\right)\subset \left[a,b\right] \text{ and } \supp\left(\mathbf{F}\left[g\right]\right)\subset \left(-\infty,b\right)\cup \left(a,\infty\right) \label{eq:bedrosian_condition}\end{equation}

	\noindent where $\mathbf{F}\left[\cdot\right]$ represents the Fourier Transform. Then $f$ and $g$ satisfy

\begin{equation} \mathbf{H}\left[f(t)g(t)\right] = f(t)\mathbf{H}\left[g(t)\right] \end{equation}
\end{theorem}
\hrule
\vspace{5mm}
%>>>

	The Bedrosian Identity defines that when two functions $f$ and $g$ satisfy \eqref{eq:bedrosian_condition}, then their multiplication is such that, then the ``slow'' $f(t)$ can be cast out of the Hilbert operator while the ``faster'' component $g(t)$ is kept inside, greatly simplifying the transformation process. Intuitively, \eqref{eq:bedrosian_condition} means that $f(t)$ is ``slower'' than $g(t)$ in the sense that the spectrum of $g$ ``envelopes'' that of $f$, since it is composed of higher harmonics.

	\cite{Xu2006} then show that this property is especially useful in the research and study of Power Systems because it allows representing phasorial signals of interest, for instance, phase signals where the amplitude varies or there is a rapid increase in frequency. For instance, suppose a signal $x(t) = m(t)e^{j\left(\omega t + \phi\right)}$, where $m(t)$ is slower than $\omega$, that is,

\begin{equation} \supp\left(\mathbf{F}\left[m\right]\right) \subset \left(-\omega,\omega\right) \end{equation}

	\noindent then

\begin{equation} \mathbf{H}\left[m(t)\cos\left(\omega t + \phi\right)\right] = m(t) \mathbf{H}\left[\cos\left(\omega t + \phi\right)\right] = m(t)e^{j\left(\omega t + \phi\right)} \end{equation}

	Much alike, \cite{derviskadicPhasorsModelingPower2020} shows that the Bedrosian Identity can be used to produce Dynamic Phasors for certain signals of interest in Power Systems. For instance, consider

\begin{equation} x(t) = M \cos \left(\omega t + \phi + R t^2\right), \label{eq:ramping_coefficient_signal}\end{equation}

	\noindent where $M$ is a constant amplitude and $R$ is a frequency ramping coefficient, modelling an unstable frequency growing linearly in time. Then

\begin{equation} \mathbf{H}\left[M \cos\left(\omega t + \phi + Rt^2\right)\right] = M \mathbf{H}\left[\cos\left(\omega t + \phi + Rt^2\right)\right] = Me^{j\left(\omega t + \phi + Rt^2\right)} .\end{equation}

	Also consider the signal

\begin{equation} x(t) = M_0\left[1 + k\theta(t)\right]\cos \left(\omega t + \phi\right), \end{equation}

	\noindent where $\theta(t)$ is the Heaviside step, modelling a sudden change in amplitude. Then

\begin{equation} \mathbf{H}\left[x\right] = M_0\left[1 + k\theta(t)\right] e^{j\left(\omega t + \phi\right)}. \label{eq:amplitude_delta_signal}\end{equation}

%-------------------------------------------------
\subsection{Shortcomings of the Hilbert Transform} %<<<2

%-------------------------------------------------
\subsubsection{Representation of signals in time}

	The matter of fact is that, while powerful the Hilbert Transform fails in the most basic of tasks sought, since not all signals in time can be easily represented, only those that adhere to the Bedrosian Identity. Reestated, the capacity of the HT to produce easily representable complex function relies on a very specific nature of the signals being considered, meaning only a certain class of signals can be contemplated. Further, being an integral transform, it relies on the fact that the only real signals applicable are those that have a ``nice'' (as in, analytically representable) transform.

	For instance, \eqref{eq:ramping_coefficient_signal} models a signal with a linearly rampant frequency while \eqref{eq:amplitude_delta_signal} models a signal with amplitude variation. These are clearly simplifications of certain transient phenomena which, in practicality, are much more sophisticated.

%-------------------------------------------------
\subsubsection{Differentials}

	While the SFTF is able to produce complex differential systems that are somehow simulatable and indeed present numerical benefits, such is not the case with the Hilbert Transform. For instance, given some linear system

\begin{equation} \sum_{k=0}^n \alpha_k x^{(k)} - f(t) = 0.\label{eq:hilbert_original_system}\end{equation}

	Apply the HT to this equation

\begin{equation} \mathbf{H}\left[\sum_{k=0}^n \alpha_k x^{(k)}\right] - \mathbf{H}\left[f\right] = 0 \end{equation}

	\noindent and using the HT's linearity,

\begin{equation} \sum_{k=0}^n \alpha_k \mathbf{H}\left[x^{(k)}\right] - \mathbf{H}\left[f\right] = 0 .\end{equation}

	Now using the HT differentiation property \eqref{eq:hilvert_diff}, 

\begin{equation} \sum_{k=0}^n \alpha_k \left(\mathbf{H}\left[x\right]\right)^{(k)} - \mathbf{H}\left[f\right] = 0 \label{eq:hilbert_transf_system}\end{equation}

	\noindent thus summing up both equations yields

\begin{equation} \eqref{eq:hilbert_original_system} + j\times\eqref{eq:hilbert_transf_system}: \sum_{k=0}^n \alpha_k x_a^{(k)} - f_a = 0\end{equation}

	\noindent where $x_a,f_a$ are the analytic signals of $x(t)$ and $f(t)$. This resulting equation is the exact same differential equation as the original, presenting no particular benefits on the application of the HT to linear systems. One might even argue that the resulting equation is even more difficult to solve than the original, because it has an added dimension. In short, the Hilbert Transform is unable to transform linear differential equations into complex (``phasorial'') equivalents that present modelling or numerical benefits over the original equations of the system, defeating the purpose of transforms in the first place.

%-------------------------------------------------
\subsubsection{Power signals}

	Finally, the Hilbert Transform is unable to represent power signals. \cite{derviskadicPhasorsModelingPower2020} cites that the HT is able to produce some notion of complex power, by the following construction. Let $v(t),i(t)$ the voltage across and current through a bipole, and denote their analytic signals as $\hat{v} = v(t) + j\mathbf{H}\left[v\right]$ and $\hat{i} = i(t) + j\mathbf{H}\left[i\right]$. Then consider the quantities

\begin{equation}\left\{\begin{array}{l} p_1(t) = \hat{v}\hat{i} = v(t)i(t) - \mathbf{H}\left[v\right]\mathbf{H}\left[i\right] + j\left( \mathbf{H}\left[v\right]i(t) - v(t)\mathbf{H}\left[i\right]\right) \\[3mm] p_2(t) = \hat{v}\overline{\hat{i}} = v(t)i(t) + \mathbf{H}\left[v\right]\mathbf{H}\left[i\right] + j\left( \mathbf{H}\left[v\right]i(t) - v(t)\mathbf{H}\left[i\right]\right)\end{array}\right. .\end{equation}

	Then the sum of $p_1$ and $p_2$ yields

\begin{equation} p_3(t) = p_1(t) + p_2(t) = 2v(t)i(t) + j2\mathbf{H}\left[v\right]i(t) .\end{equation}

	\noindent so that the real part of $p_3(t)$ is twice the instantaneous power $p(t)$, while the imaginary part does not have any specific meaning and is cited as a ``modelling artifact''. Thus, while the Hilbert Transform can produce \textit{some} notion of complex power, it can only reconstruct the instantaneous power but cannot produce solid notions of active and reactive power.

%-------------------------------------------------
\section{Proposed Dynamic Phasors Theory} \label{sec:proposed_dptheory} %<<<1

	It becomes now clear that the current techniques fail at some point:

\begin{itemize}
	\item The STFT does produce differential complex systems that have \textit{some} accuracy in representing signals in time (depending on how ``slow'' the signals are, as per theorem \ref{theo:fdp_quasi_static}), it requires approximations to do so, and has a particular problem when expressing power signals;
	\item The Hilbert Transform can represent \textit{some} signals of interest, but it does not produce convenient differential models and does not represent power signals in time.
\end{itemize}

	The first path to filling the gaps of these current techniques is to adopt a proper representation of the signals involved. Inspired by the ``simple'' PLL subsystem of Figure \ref{fig:pll_example} and by the IEEE Standard C37.118.1-2011 for Synchrophasor Measurements for Power Systems \pcite{IEEEStandardSynchrophasor2011}, the following representation is proposed: instead of considering signals that can be expressed by \eqref{eq:dynamic_sinusoid_example} where the time-varying frequency multiplies time, let us consider signals of the form $x(t) = m(t)\cos\left(\theta(t)\right)$ where the angle $\theta(t)$ can be written as the sum of a time-varying phase and the integral of the time-varying frequency frequency. Reestated, $x(t)$ is such that, for some time-varying frequency $\omega(t)$ chosen, there is a solution $\phi(t)$ to \eqref{eq:apparent_angle_def}, called the \textbf{apparent phase} of $x(t)$ with respect to $\omega(t)$.

\begin{equation} \theta(t) = \psi(t) + \phi(t),\ \psi(t) = \int_{t_0}^t \omega(s)ds . \label{eq:apparent_angle_def}\end{equation}

	With this representation in mind, we rewrite definition \ref{def:sinusoid} to a more precise version. Definition \ref{def:sinusoid_dynamic} describes a \textbf{generalized sinusoid}, or simply \textbf{sinusoid}, as a signal that has a ``sinewave shape'' with time-varying amplitude and frequency, such that the absolute angle can be broken down into an accumulated angle $\psi(t)$ and a time-varying phase $\phi(t)$ as per \eqref{eq:apparent_angle_def}.

\begin{definition}\label{def:sinusoid_dynamic}%<<<
	A \textbf{generalized sinusoid} or simply sinusoid is a $x(t)\in\left[\mathbb{R}\to\mathbb{R}\right]$ if there are two real signals called \textbf{amplitude} $m(t)$ and \textbf{absolute angle} $\theta(t)$ such that $x(t) = m(t)\cos\left(\theta(t)\right)$. Further, given some apparent frequency $\omega(t)$, $x(t)$ is a \textbf{generalized sinusoid at the apparent frequency $\boldsymbol{\omega}(t)$} if \eqref{eq:apparent_angle_def} has a solution $\phi(t)$ called the \textbf{apparent phase}.
\end{definition} %>>>

	In this definition, the \textbf{absolute angle} of $x(t)$, $\theta(t)$, is the ``whole angle'' as measured by the measuring device, with $\omega(t)$, called the \textbf{apparent frequency} the notion of the time-varying frequency, $\phi(t)$ the \textbf{apparent phase} the notion of time-varying phase and $\psi(t)$ is the angle accumulated by $\omega(t)$ from some initial time $t_0$, most probably $t_0 = 0$. In the case of a PLL, $\omega(t)$ is given by a feedback loop (as will be shown later); in case of the transform proposed here, $\omega(t)$ is supposed arbitrary in principle, and more requirements will be added later. We further divide \textbf{generalized sinusoids} into two categories: \textbf{static or stationary sinusoids} if $m(t)$ and $\dot{\theta}(t)$ are constant, and \textbf{nonstationary sinusoids} if either or both $m(t)$ and $\dot{\theta}(t)$ are time-varying. Notably, in a stationary sinusoidal case, $m,\omega$ and $\phi$ are constant, so $\psi(t) = \omega t$. For cleanliness of the test, we shorten the terminology and refer to generalized sinusoidas as simply ``sinusoids'', even though classically the word means only the static ones.

	The question on the nature of generalized sinusoids and the feasibility of such representation is discussed thoroughly on section \ref{sec:discussion_proposed_representation}. For now it suffices to say that when we assume a signal admits a sinusoidal representation we will say so explicitly as in "\textbf{assume $x(t)$ has a sinusoidal representation}", however weak this assumption is.

\begin{definition}[Admissibility of a sinusoidal representation] A signal $x(t)\in\left[\mathbb{R}\to\mathbb{R}\right]$ \textbf{admits a sinusoidal representation} if there exist functions $m(t),\ \theta(t)$ such that $x(t) = m(t)\cos\left(\theta(t)\right)$. Additionally, $x(t)$ admits a sinusoidal representation \textbf{at the frequency $\omega(t)$} if there exists a solution $\phi$ to $\phi(t) = \theta(t) - \psi(t),\ \psi(t) = \int_0^t \omega(s)ds$.

	Equivalently, $x(t)$ admits a sinusoidal representation if there exists a function $f(t)\in\left[\mathbb{R}\to\mathbb{C}\right]$ such that $x(t) = \Re\left[f(t)\right]$. The signal $x(t)$ then admits a representation at $\omega(t)$ if $f(t)$ is such that there exists a solution $\phi$ to $\phi(t) = \arg\left[f(t)\right] - \psi(t)$.
\end{definition}

%-------------------------------------------------
\subsection{Construction of the Dynamic Phasor Transform} %<<<2

	One of the issues with the current literature is the fact that the representation of a signal $x(t) = m(t)\cos\left(\psi(t) + \phi(t)\right)$ as a time-varying complex function $X(t) = m(t)e^{j\phi(t)}$ is assumed but the exact process by which this representation is constructed is not given. For instance, the IEEE Standard C37.118.1-2011 states that the synchrophasor \eqref{eq:synchrophasor_complex} represents \eqref{eq:synchrophasor_time}; yet this affirmation is only a representation and the exact mechanics by which one quantity is constructed from the other is not shown.

	 Such construction is proposed as follows. First we note that the core of the PLL is a two-fold process: a ``$\alpha\beta$'' transform followed by a ``dq'' transform, the latter dependent on some frequency signal $\omega(t)$ supplied by some control. If $x(t)$ is a single-phase quantity, we assume that it assumes a sinusoidal representation, as discussed thoroughly in subsection \ref{subsec:comments_def_sin}. What is generally called the $\alpha\beta$ transform is in fact the transformation of the input signal $x(t)$ into a generator function $f(t) = x_\alpha(t) + jx_\beta(t)$, that is, $x(t)$ is represented by two components $x_\alpha$ and $x_\beta$ such that $x_\alpha$ is in phase with $x(t)$ and $x_\beta$ is in quadrature:

\begin{equation} \mathbf{x}_{\alpha\beta} = \left[\begin{array}{c} x_\alpha(t) \\[3mm] x_\beta(t) \end{array}\right].\end{equation}

	If $x(t)$ admits a sinusoidal representation and the amplitude $m(t)$ and the argument $\theta(t)$ of the generator function $f(t) = m(t)e^{j\theta(t)}$ are known, then $x_\alpha$ and $x_\beta$ are intuitively obtained as

\begin{equation} \mathbf{x}_{\alpha\beta} = \left[\begin{array}{c} x_\alpha(t) \\[3mm] x_\beta(t) \end{array}\right] = m(t)\left[\begin{array}{c} \cos\left(\theta\right) \\[3mm] \sin\left(\theta\right) \end{array}\right] .\end{equation}

% PLL SUBSTYSTEM <<<
\begin{figure} 
\centering
\begin{tikzpicture}[scale=1,>={Stealth[inset=0mm,length=1.5mm,angle'=50]}]

\node [draw, minimum width=2cm, very thick, minimum height=2cm, left=0] (tab_block) {};

\draw (tab_block.south west) -- (tab_block.north east);

\node at ([shift=({{ 2cm*sqrt(2)/4},{-2cm*sqrt(2)/4}})]tab_block.north west) {\large $t$};
\node at ([shift=({{-2cm*sqrt(2)/4},{ 2cm*sqrt(2)/4}})]tab_block.south east) {\large $\alpha\beta$};

\node at ([shift=({-15mm,0})]tab_block.west) (signalinput) {$x(t)$};
\draw[->] ([shift=({4mm,0})]signalinput.center) -- ([shift=({-1mm,0})]tab_block.west);

\node [draw, minimum width=2cm, very thick, minimum height=2cm, right=2cm of tab_block] (abdq_block) {};
\draw (abdq_block.south west) -- (abdq_block.north east);
\node at ([shift=({{ 2cm*sqrt(2)/4},{-2cm*sqrt(2)/4}})]abdq_block.north west) {\large $\alpha\beta$};
\node at ([shift=({{-2cm*sqrt(2)/4},{ 2cm*sqrt(2)/4}})]abdq_block.south east) {\large $dq$};

\draw[->] ([shift=({0, 3.33mm})]tab_block.east) -- ([shift=({-1mm, 3.33mm})]abdq_block.west) node[midway, above] {$x_\alpha(t)$};
\draw[->] ([shift=({0,-3.33mm})]tab_block.east) -- ([shift=({-1mm,-3.33mm})]abdq_block.west) node[midway, below] {$x_\beta(t)$};

\draw[->] ([shift=({0, 3.33mm})]abdq_block.east) -- ([shift=({10mm, 3.33mm})]abdq_block.east) node[right] {$x_d(t)$};
\draw[->] ([shift=({0,-3.33mm})]abdq_block.east) -- ([shift=({10mm,-3.33mm})]abdq_block.east) node[right] {$x_q(t)$};

\node [draw, minimum width=1cm, very thick, minimum height=1cm, below=1cm of abdq_block] (omega_integrator) {$\int$};

\node [below=1cm of omega_integrator.south] (omega_signal) {$\omega(t)$};

\draw[->] (omega_signal.north) -- ([shift=({0,-1mm})]omega_integrator.south);
\draw[->] (omega_integrator.north) -- ([shift=({0,-1mm})]abdq_block.south) node [midway, right] {$\psi(t)$};

\end{tikzpicture}
\caption{Example PLL block for inspiration of the Differential Dynamic Phasors.}
\label{fig:pll_example}
\end{figure}
%>>>
More deeply, this process is justified because if $x(t)$ admits a sinusoidal representation it is, in essence, a two-dimensional signal: it depends on an amplitude signal and an angle signal. This is akin to the fact that the Static Phasor Operator transforms a one-dimensional static sinusoid into a complex number, which is two-dimensional. Ultimately, given $m(t)$ and $\theta(t)$, no information is gained or lost due to the $\alpha\beta$ transformation.

	Naturally, this transform is linear and invertible: given the two-dimensional vector $\left[x_\alpha(t),x_\beta(t)\right]^\transpose$, then this vector is naturally diffeomorphic to a complex number $x_\alpha + jx_\beta$, thus

\begin{equation} x(t) = x_\alpha(t) = m_x(t)\cos\left(\theta_x\right) \text{, where } m_x(t) = \left\lvert x_\alpha + jx_\beta\right\rvert \text{ and } \theta_x = \arg\left(x_\alpha + jx_\beta\right) .\end{equation}

	Therefore this transform is also invertible. Further, since complex addition is linear, then this $\alpha\beta$ transform, as well as its inverse, are also invertible. Then, rotate the vector $\mathbf{x}_{\alpha\beta}$ by a rotational transformation

\begin{equation} \mathbf{T}_\psi = \left[\begin{array}{cc} \cos\left(\psi\right) & \sin\left(\psi\right) \\[3mm] -\sin\left(\psi\right) & \cos\left(\psi\right)\end{array}\right],\ \mathbf{T}_\psi^{-1} = \left[\begin{array}{cc} \cos\left(\psi\right) & -\sin\left(\psi\right) \\[3mm] \sin\left(\psi\right) & \cos\left(\psi\right)\end{array}\right]  \end{equation}

	\noindent resulting in

\begin{equation} \mathbf{x}_{dq} = \left[\begin{array}{c} x_d(t)\\[3mm] x_q(t) \end{array}\right] = \mathbf{T}_\psi\mathbf{x}_{\alpha\beta} = m(t)\left[\begin{array}{c} \cos\left(\phi\right) \\[3mm] \sin\left(\phi\right) \end{array}\right] \end{equation}

	\noindent where $\phi(t)$ is the solution to \eqref{eq:apparent_angle_def} . The ``dq'' notation is directly inherited from the Power System literature: the ``d'' component stands for \textit{direct} (in phase) axis and the ``q'' component for \textit{quadrature}, and the terminology will be explained later. Naturally, since $\mathbf{x}_{\alpha\beta}$ can be represented as a complex number, so can $\mathbf{x}_{dq}$ be represented by the number $x_d + jx_q$. Quickly one recognizes $\mathbf{T}_\psi$ is a rotation matrix at the angle $-\psi(t)$; as such, its inverse $\mathbf{T}_\psi^{-1}$ is the rotation matrix at the angle $\psi(t)$, and this inverse always exists because the determinant of $\mathbf{T}_\psi$ is always unitary independently of $\psi(t)$. This means that the entire process is bijective and unique: given the frequency signal $\omega(t)$, $\mathbf{x}_{dq}$ reconstructs $x(t)$ and vice-versa biunvocally.

	We now want to show that the ``dq'' transformed quantity $\mathbf{x}_{dq}$ is equivalent (infinitely diffeomorphic, in fact) to a function the complex plane, which we will call the Dynamic Phasor representation. A couple hints at this fact are that it is obvious that if the aplitude $m(t)$ and the angle $\theta(t)$ of the signal \eqref{eq:apparent_angle_def} are known (therefore so are the $x_\alpha$ and $x_\beta$ components), then one can define the time-varying complex function $X_p(t)$ given by

\begin{equation} X_p(t) = x_\alpha(t) + jx_\beta(t) = m(t)e^{j\theta(t)} . \label{eq:xab_xp}\end{equation}

	Because the linear transform $\mathbf{T}_\psi$ is a rotation matrix at the angle $-\psi(t)$, applying it to  $\mathbf{x}_{\alpha\beta}$ means that the complex equivalent $x_p(t)$ is rotated by $e^{-j\psi(t)}$, yielding

\begin{equation} X(t) = m(t)e^{j\theta(t)} e^{-j\psi(t)}= m(t)e^{j\left(\theta(t) - \psi(t)\right)} = m(t)e^{j\phi(t)} \label{eq:xcomp_xdq}\end{equation}

	\noindent which is exactly $X(t) = x_d(t) + jx_q(t)$. In order to prove this line of thought, we define a \textit{complexification operator} that transforms a two-dimensional real function (that is, any $\mathbf{x}\in\left[\mathbb{R}\to\mathbb{R}^2\right]$) into a complex function. The theorem proves that this transform is not only invertible, but that itself and its inverse are linear and infinitely differentiable in the space $\left[\mathbb{R}\to\mathbb{R}^2\right]$.

\begin{theorem}[$dq$ and complex space equivalence]\label{theo:rho_diff_inf} %<<<
	Consider a function $\mathbf{x} = \left[u(t),v(t)\right]^\intercal\in\left[\mathbb{R}\to\mathbb{R}\right]^2$, and let $\rho$ denote a complex equivalence functional mapping given by

\begin{equation} \rho: \left\{\begin{array}{rcl} \left[\mathbb{R}\to\mathbb{R}\right]^2 &\to& \left[\mathbb{R}\to\mathbb{C}\right] \\[3mm] \mathbf{x} &\mapsto& \left[1,j\right]\mathbf{x} \end{array}\right. \end{equation}

	\noindent that is, $\rho$ takes a two-dimensional real function of a single real variable $\mathbf{x}(t)$ and delivers a complex function $X(t) = u(t) + jv(t)$. The inverse transform is given by

\begin{equation} \rho^{-1}: \left\{\begin{array}{rcl} \left[\mathbb{R}\to\mathbb{C}\right] &\to& \left[\mathbb{R}\to\mathbb{R}\right]^2 \\[3mm] X(t) &\mapsto& \left[\begin{array}{c} \Re\left[X\left(t\right)\right]\\[3mm] \Im\left[X\left(t\right)\right]\end{array}\right] \end{array}\right. \end{equation}

	Then $\rho$ is a canonic infinite diffeomorphism, that is: it is unique (except for some complex scaling), infinitely differentiable, bijective and the inverse is also infinitely differentiable. 
\end{theorem}

\textbf{Proof:} because $\rho$ is a function of a function — called a \textit{functional} — the derivative used is the Fréchet Derivative as defined in \eqref{eq:def_frechet}. The functional derivative of $\rho$ at $\mathbf{x}$ is defined as the bounded linear map $\mathbf{A}\left[\mathbf{x}\right]$ that satisfies

\begin{equation} \lim\limits_{\left\lVert \Delta\mathbf{x}\right\rVert\to 0} \dfrac{\left\lVert \rho\left[\mathbf{x} + \Delta\mathbf{x}\right] - \rho\left[\mathbf{x}\right] - \mathbf{A}\left[\mathbf{x}\right]\Delta\mathbf{x}\right\rVert}{\left\lVert \Delta\mathbf{x}\right\rVert} = 0\end{equation}

	\noindent where $\mathbf{A}\left[\mathbf{x}\right]\Delta\mathbf{x}$ denotes $\mathbf{A}$ calculated at $\mathbf{x}$ applied onto a $\Delta\mathbf{x}$, and $\Delta\mathbf{x}\in\left[\mathbb{R}\to\mathbb{R}\right]^2$ is small enough so the limit exists.  It is clear that $\mathbf{A}\left[\mathbf{x}\right]\Delta\mathbf{x} = \left[1,j\right]\Delta\mathbf{x}$ for any $\mathbf{x}$:

\begin{equation} \left[1,j\right]\left(\mathbf{x} + \Delta\mathbf{x}\right) - \left[1,j\right]\mathbf{x} - \left[1,j\right]\Delta\mathbf{x} = 0 . \end{equation}

	Therefore $\rho$ is a linear functional as its derivative is constant. Most importantly, however, is that the differential of $\rho$ calculated at any $\mathbf{x}$ is equal to $\rho$ itself; these fact mean that $\rho$ is infinitely differentiable and all subsequent differentials will be equal to $\rho$ — all higher-order derivatives $\delta^n \rho\left[\mathbf{x}\right]$, calculated at any $\mathbf{x}$ where they exist, will be identical to $\rho$ itself.

	As for the inverse $\rho^{-1}$, because real and imaginary parts are also infinitely differentiable \pcite{ahlfors1979complex}, infinite differentiability of $\rho^{-1}$ is easy to prove: the differential of $\rho^{-1}$ at $X$ applied on a $\Delta X$ is calculated as

\begin{equation} \dfrac{\delta\left(\rho^{-1}\right)}{\delta X} \Delta X = \left[\begin{array}{c} \Re\left[\Delta X(t)\right] \\[5mm] \Im\left[\Delta X(t)\right] \end{array}\right] .\end{equation}

	Indeed,

\begin{equation} \left[\begin{array}{c} \Re\left[X(t) + \Delta X(t)\right] \\[5mm] \Im\left[X(t) + \Delta X(t)\right] \end{array}\right] - \left[\begin{array}{c} \Re\left[X(t)\right] \\[5mm] \Im\left[X(t)\right] \end{array}\right] - \left[\begin{array}{c} \Re\left[\Delta X(t)\right] \\[5mm] \Im\left[\Delta X(t)\right] \end{array}\right] = \left[\begin{array}{c} 0 \\[5mm] 0 \end{array}\right]  \end{equation}

	And it is easy to see that the subsequent differentials $\delta^n\left(\rho^{-1}\right)\left[X\right]$ will be identical. It is now only left to prove that $\rho$ is unique apart from a scalar multiplication. Take some non-zero real $\alpha$. Then $\rho_\alpha$ can be defined as

\begin{equation} \rho_\alpha\left[\mathbf{x}\right] \mathbf{y}(t) = \alpha\left[1,j\right]\mathbf{y} \text{ and } \rho^{-1}_\alpha\left[X\right] Y= \dfrac{1}{\alpha}\left[\begin{array}{c} \Re\left(Y\right)\\[3mm] \Im\left(Y\right)\end{array}\right]. \end{equation}

	\noindent which are also infinitely diffeomorphic. Adopt the \textit{canonic} transformation as the version of $\alpha = 1$. \hfill$\blacksquare$

\vspace{5mm}
\hrule
\vspace{5mm}
%>>>

	The existence of a (quite a mouthful) canonic infinite diffeomorphism between $\mathbf{x} = \left[u(t),v(t)\right]^\transpose$ and $X(t) = u(t) + jv(t)$ means that these entities are effectively one the same, but represented in two different topological spaces; because of this, the notation $\mathbf{x} \simeq X$ will be used. This can be read as \textit{$\mathbf{x}$ and $X$ are equivalent}, or that \textit{$X$ is the complex version (or equivalent) of $\mathbf{x}$}. Also, the $\rho$ operator will thenceforth be called the \textit{complex equivalence operator} or simply \textit{complexification}.

	This functional mapping $\rho$ then justifies the bijection between the $\alpha\beta$ transform of a sinusoid and the complex number $x_\alpha(t) + jx_\beta(t)$, as per \eqref{eq:xab_xp}, and the bijection between $\mathbf{x}_{dq}$ and $X(t) = x_d(t) + jx_q(t)$ as per \eqref{eq:xcomp_xdq}. Using the tandem process comprised of the $\alpha\beta$ transform, followed by the $dq$ transform at some frequency signal $\omega(t)$ and then the complexification $\rho$, one achieves a Dynamic Phasor Transform (DPT), that is, a bijection between a sinusoid $x(t)$ and a time-varying complex function $X(t)$, as in Figure \ref{fig:complexification_process_dps}.

% COMPLEXIFICATION IN THE DP DOMAIN <<<
\begin{figure}[htb]
\centering
	\scalebox{0.75}{
	\begin{tikzpicture}[scale=1,>={Stealth[inset=0mm,length=2.5mm,angle'=50]}]
	%\node[left] (xinput) at (0,0) {$m(t)\cos\left(\psi(t) + \phi(t)\right)$};
	\node[left] (xinput) at (0,0) {$x(t)$};
	\node [draw, minimum width=15mm, very thick, minimum height=15mm, right=20mm of xinput]   (ab_block)  {$\alpha\beta$};
	\node [draw, minimum width=15mm, very thick, minimum height=15mm, right=20mm of ab_block] (dq_block)  {$\mathbf{T}_\psi$};
	\node [draw, minimum width=15mm, very thick, minimum height=15mm, right=20mm of dq_block] (rho_block) {$\rho$};
	%\node [right, right=20mm of rho_block] (output) {$X(t) = x_d + jx_q = m(t)e^{j\phi(t)}$};
	\node [right, right=20mm of rho_block] (output) {$X(t)$};

	\draw [->] (xinput.east) -- ([shift=({-1mm,0})]ab_block.west); 
	\draw [->] (ab_block.east) -- ([shift=({-1mm,0})]dq_block.west) node[midway,above] {$\mathbf{x}_{\alpha\beta}$}; 
	\draw [->] (dq_block.east) -- ([shift=({-1mm,0})]rho_block.west) node[midway,above] {$\mathbf{x}_{dq}$}; 
	\draw [->] (rho_block.east) -- (output.west); 

	\node [draw, minimum width=15mm, very thick, minimum height=15mm, above=15mm of dq_block] (int_block) {$\int$};

	\node[above] (omegat) at ([shift=({0,20mm})]int_block) {$\omega(t)$} ;
	\draw [->] (omegat) -- ([shift=({0,1mm})]int_block.north);
	\draw [->] (int_block.south) -- ([shift=({0,1mm})]dq_block.north) node[near start,right] {$\psi(t)$};

	\node [draw, rounded corners, stewartblue, very  thick, dashed, minimum width=10cm, minimum height=3cm] at (dq_block.center)  (invol) {};

	\node [draw, minimum width=15mm, very thick, stewartblue, minimum height=15mm, below=60mm of dq_block] (ps_block) {$\mathbf{P_D^\omega}$};

	%\node[left] (xinput_ps) at ([shift=({-20mm,0})]ps_block.west) {$m(t)\cos\left(\psi(t) + \phi(t)\right)$};
	\node[left] (xinput_ps) at ([shift=({-20mm,0})]ps_block.west) {$x(t)$};
	\draw [->] (xinput_ps) -- ([shift=({-1mm,0})]ps_block.west) ;
	%\node[right] (output_ps) at ([shift=({20mm,0})]ps_block.east) {$X(t) = x_d + jx_q = m(t)e^{j\phi(t)}$};
	\node[right] (output_ps) at ([shift=({20mm,0})]ps_block.east) {$X(t)$};
	\draw [->] (ps_block.east) -- ([shift=({-1mm,0})]output_ps.west) ;

	\node [draw, minimum width=15mm, very thick, minimum height=15mm, above=15mm of ps_block] (intps_block) {$\int$};
	\node[above] (omegatps) at ([shift=({0,20mm})]intps_block) {$\omega(t)$} ;
	\draw [->] (omegatps) -- ([shift=({0,1mm})]intps_block.north);
	\draw [->] (intps_block.south) -- ([shift=({0,1mm})]ps_block.north) node[midway,right] {$\psi(t)$};

	\draw [dashed, very thick, stewartblue]  (invol.south east) -- (ps_block.north east);
	\draw [dashed, very thick, stewartblue]  (invol.south west) -- (ps_block.north west);
	\end{tikzpicture}
	}
	\caption{The process of \textit{complexification} of a sinusoid into a complex Dynamic Phasor.}
	\label{fig:complexification_process_dps}
\end{figure} %>>>

	Notably, this process is invertible and diffeomorphic: from a complex function $X(t)$ one uses the inverse $\rho^{-1}$ yielding two $dq$ components; then, these are transformed to $\alpha\beta$ quantities using $\mathbf{T}^{-1}_{\psi(t)}$, and the first component of the $\alpha\beta$ vector is taken to deliver $x(t)$. Therefore, we can define the proposed Dynamic Phasor Transform.

\begin{definition}[Dynamic Phasor Transform (DPT)]\label{def:dptransform}
	Consider a frequency signal $\omega(t)$. The Dynamic Phasor Transform at $\omega$, denoted $\mathbf{P_D}^\omega\left[x\right]$ is defined as 

\begin{equation} \mathbf{P_D^{\omega}}: \left\{\begin{array}{rcl} \left[\mathbb{R}\to\mathbb{R}\right] &\to& \left[\mathbb{R}\to\mathbb{C}\right] \\[3mm] m(t)\cos\left(\psi(t) + \phi(t)\right) &\mapsto& X(t) = m(t)e^{j\phi(t)} \end{array}\right. \end{equation}

	\noindent and its inverse is defined as

\begin{equation} \mathbf{P_D^{\left(-\omega\right)}}\left[X\right]: \left\{\begin{array}{rcl} \left[\mathbb{R}\to\mathbb{C}\right] &\to& \left[\mathbb{R}\to\mathbb{R}\right] \\[3mm] X(t) &\mapsto& \Re\left[X(t)e^{j\psi(t)}\right] \end{array}\right. \end{equation}

\end{definition}

	Here one wonders if any signal $x(t)$ is ``phasorializable'', that is if an arbitrary signal $x(t)$ admits a Dynamic Phasor representation. This is equivalent to asking whether any signal $x(t)$ can be written as a generalized sinusoid $m(t)\cos\left(\omega(t) + \phi(t)\right)$, because if this is true then $X(t) = m(t)e^{j\phi(t)}$ is the Dynamic Phasor of $x(t)$ at the apparent frequency $\omega(t)$. As discussed in subsection \ref{subsec:comments_def_sin}, this representation is rather forgiving — the restrictions for the sinusoidal representation seem to be nonexistant. Therefore, it would seem that the requirements for a real signal to be phasorializable are very weak. Again in the name of mathematical rigour we will explicitly say \textbf{we assume the signal is phasorializable} when this is assumed, and this assumption is equivalent to supposing the signal admits a sinusoidal representation.

\begin{definition}[Phasorializability] A signal $x(t)$ is \textbf{phasorializable} if it admits a sinusoidal representation $m(t)\cos\left(\psi(t) + \phi(t)\right)$ at some apparent frequency $\omega(t)$. In this case, $X(t) = m(t)e^{j\phi(t)}$ is the Dynamic Phasor of $x(t)$ at $\omega(t)$. \end{definition}

%-------------------------------------------------
\subsection{Properties of the Dynamic Phasor Transform}

	Having constructed the proposed Dynamic Phasor Transform, we must assert its properties; the first and possibly cornerstone propery being that the DPT generalizes the Static Phasor Operator, in the sense that the SPO is a particular case of the DPT. Indeed, given statically sinusoidal signals at a particular frequency $\omega_0$, $\mathbf{p_S}$ is equivalent to $\mathbf{P_D^{\left(-\omega_0\right)}}$. Take a sinusoidal signal $x(t) = m\cos\left(\theta(t)\right)$ with a constant amplitude, and such that there exists a positive real $\omega_0$ for which there exists a constant solution $\phi$ for the equation $\theta = \omega t + \phi$. Then obviously 

% DYNAMIC PHASOR DIAGRAM <<<
\begin{figure}[htb]
\centering
	\begin{tikzpicture}[scale=2,>={Stealth[inset=0mm,length=1.5mm,angle'=50]}]
		\draw [fill=none,gray, thick] (0,0) circle (10 mm) node [gray] {};
		\draw [->, thick, black!30] (   -20mm,  0   ) -- (   20mm,  0   );
		\draw [->, thick, black!50] (      0, -15mm ) -- (   0   ,  17mm);

		\draw [->, thick, black] (0,0) -- (14mm,0) coordinate(realvec);
		\draw [->, thick, black] (0,0) -- (0,14mm) coordinate(imagvec);

		\node (realveclabel) at ([shift=({0,-2mm})]realvec) {$R = 1e^{j0}$};
		\node (realveclabel) at ([shift=({4mm,-2mm})] imagvec) {$I = 1e^{j\frac{\pi}{2}}$};

		\node [black!50] (reAxisLabel) at (22mm,0) {Re};
		\node [black!50] (imAxisLabel) at (0,19mm) {Im};

	%	\node [label={[label distance=0.1mm]30:$X = m(t)e^{j\phi(t)}$}] (X) at ({10mm*cos(40)},{10mm*sin(40)}) {};
	%	\draw [->,thick] (0,0) -- (X.center);
	%	\draw [->,stewartblue,thick] ({8mm*cos(0)},{8mm*sin(0)}) arc[start angle=0, end angle = 38, radius = 8mm];
		\draw [->,stewartblue,thick] ({8mm*cos(290)},{8mm*sin(290)}) arc[start angle=290, end angle = 328, radius = 8mm];

	%	\node [color=stewartblue] (philabel) at ({12mm*cos(20)},{12mm*sin(20)}) {$\phi(t)$};
		\node [color=stewartblue] (philabelrotated) at ({6mm*cos(310)},{6mm*sin(310)}) {$\phi(t)$};

		\node (rotX) at ({cos(330)},{sin(330)}) {};
		\node (XomegatLabel) at ({13mm*cos(340)},{13mm*sin(340)}) {$Xe^{j\psi(t)}$};
		\draw [->,thick] (0,0) -- (rotX.center);

		\node (rotRe) at ({13mm*cos(290)},{13mm*sin(290)}) {};
		\node (rotIm) at ({13mm*cos(20)},{13mm*sin(20)}) {};
		\node (RomegatLabel) at ({14mm*cos(290)},{14mm*sin(290)}) {$Re^{j\psi(t)}$};
		\node (IomegatLabel) at ({16mm*cos(20)},{16mm*sin(20)}) {$Ie^{j\psi(t)}$};
		\draw [->,thick] (0,0) -- (rotRe.center);
		\draw [->,thick] (0,0) -- (rotIm.center);

		\draw [->,thick,gray] (0,0) -- ({10mm*cos(240)},{10mm*sin(240)});
		\node [label={[gray,label distance=0.0mm]245:$m(t)$}] (mt) at ({9mm*cos(245)},{9mm*sin(245)}) {};

		%\draw [->,gray,thick] ({6mm*cos(25)},{6mm*sin(25)}) arc[start angle=25, end angle = 63, radius = 6mm];
		\node [green!50!black] (omegat) at ({6mm*cos(135)},{6mm*sin(135)}) {$\psi(t)$};

		\draw [-{Stealth[inset=0mm,length=3mm,angle'=50]},green!50!black, line width = 1mm] (4mm,0) arc[start angle=0, end angle = 287, radius = 4mm];
		
		% SINEWAVE PLOT AXES
		\draw [->, gray, thick]  (   -15mm, -20mm  ) -- (   15mm,  -20mm  );
		\draw [->, gray, thick]  (      0,  -20mm  ) -- (    0  ,  -50mm  );

		\node[gray] (axistlabel) at (0mm,-52mm) {$t$};

		\begin{axis}[color=stewartblue, at={(-12.5mm,-52mm)}, rotate=-90, width=35mm, height=25mm, scale only axis, yticklabel=\empty, xticklabel=\empty, axis line style={draw=none}, tick style={draw=none}, scaled ticks = false]
			\addplot[domain=0:0.033, smooth, samples=100] {0.001*(1 + 0.5*exp(-12*x)*sin(deg(500*3.14159*x)))*cos(deg(200*x*(1 - 6*exp(-150*x)*sin(deg(10*3.14159*x)))))};
			\addplot[domain=0.033:0.075, smooth, dashed, dash pattern=on 1pt off 1pt, samples=100] {0.001*(1 + 0.5*exp(-12*x)*sin(deg(500*3.14159*x)))*cos(deg(200*x*(1 - 6*exp(-150*x)*sin(deg(10*3.14159*x)))))};
		\end{axis}

		\node (rotXaxis) at ([shift=({0,-27.5mm})]rotX.center) {};
		\draw [color=stewartblue,fill] (rotXaxis) circle (0.5mm);

		\draw[dashed,color=stewartblue, thick] (rotX) -- (rotXaxis) ;

		\node [stewartblue] (xsignal) at (25mm, -40mm) {$x(t)  = m(t)\cos\left(\psi(t) + \phi(t)\right)$};

	\end{tikzpicture}
	\caption{Generalized sinusoidal signal as the real projection of a rotated dynamic phasor.}
	\label{fig:dynamic_phasor_representation}
\end{figure} %>>>

\begin{equation} \mathbf{P_D^{\omega_0}}\left[x\right] = me^{j\phi} = \mathbf{p_S}\left[x\right]. \end{equation}

	Further, given any complex number $X = me^{j\phi}$ and a constant frequency signal $\omega_0$, then

\begin{equation} \mathbf{P_D^{\left(-\omega_0\right)}}\left[me^{j\phi}\right] = \Re\left[me^{j\phi}e^{j\omega_0 t}\right] = m\cos\left(\omega_0 t + \phi\right) \end{equation}

	\noindent proving that $\mathbf{p_S}$ is a particular case of $\mathbf{P_D}$, and the same relationship holds for the inverse transforms. Due to this, an adaptation of figure \ref{fig:static_phasor_representation} is unavoidable; such adaptation is represented in figure \ref{fig:dynamic_phasor_representation}. This figure shows a snapshot in time, where the signal $x(t)$ is represented as a Dynamic Phasor $X(t)$. The real axis is represented by the number $R = 1e^{j0}$. The phasor $X(t)$ is rotated by an angle of $\psi(t)$, and the signal $x(t)$ is the real projection of the rotated phasor $Xe^{j\psi(t)}$.

	Due to this representation, the DPT allows generating a rotating axis, known as the ``DQ'' axis, whence the ``direct-quadrature'' nomenclature is born. Looking at Figure \ref{fig:dynamic_phasor_representation}, one notes that with respect to the static real-imaginary frame the number $Xe^{j\psi(t)}$ is such that

\begin{equation} Xe^{j\psi(t)} = x_\alpha + jx_\beta,\end{equation}

	\noindent meaning that $Xe^{j\psi(t)}$ is the complexification of $\mathbf{x}_{\alpha\beta}$ on the static frame. This is equivalent to saying that $Xe^{j\psi(t)}$ represents $x(t)$ in a static reference frame that is the real-imaginary axis of Figure \ref{fig:dynamic_phasor_imreaxis}. By definition, the real or horizontal projection of $x_\alpha + j x_\beta$ is the sinusoid $x(t)$, also represented in the figure.

	The static real-imaginary frame is generated at $t=0$ when the system starts counting time, so that if the initial time adopted is delayed or advanced, this frame is tilted accordingly but stays static along time. If, however, one adopts as a reference the rotating axes made by the rotating vectors $Re^{j\psi(t)}$ and $Ie^{j\psi(t)}$, then projecting the rotating phasor $Xe^{j\psi(t)}$ into this new frame one obtains $X(t) = x_d + jx_q = m(t)e^{j\phi(t)}$ itself, because the angular distance between the rotated phasor $X(t)e^{j\psi(t)}$ and the rotated real reference $R(t)e^{j\psi(t)}$ is always $\phi(t)$ and the rotation process does not alter the amplitude $m(t)$. Figure \ref{fig:dynamic_phasor_imreaxis} shows the real-imaginary rotated by $\psi(t)$, generating the DQ frame. In this gist, we can adopt not the static real and imaginary axes as references, but these rotated real and imaginary frames. These frames will be called ``DQ'' frame; here, ``D'' stands for ``direct'' because this axis is in direct phase with the rotated real axis, whereas ``Q'' stands for ``quadrature'' because the Q axis is in quadrature with the rotated real axis. As such, the projections of phasors against this frame are called their ``dq'' components — thus generating the nomenclature and notation ``dq'' for $\mathbf{x}_{dq}$.

	In other words, fundamentally what the ``dq'' transform (represented by $\mathbf{T}_{\psi}$ in the bidimensional frame or $e^{-j\psi(t)}$ in the complex domain) does is generating a new rotating frame such that the complex quantities involved, when projected against this frame, do not depend directly on the frequency $\omega(t)$ or its corresponding accumulated angle $\psi(t)$, but represent the time-varying Dynamic Phasor functions directly as opposed to their static frame versions — that is, the quantity $x_\alpha + jx_\beta$. Because the DQ frame is rotated at $\psi(t)$, this means that the frequency of rotation is $\omega(t)$, such that the vector $Xe^{j\psi(t)}$ is represented in this frame by $X(t)$. In other words, this rotating frame is the direct representation of the DPT, as the Dynamic Phasors produced by $\mathbf{P_D^{\omega}}$ are represented directly onto this frame.

% DYNAMIC PHASOR DIAGRAM OF IMAGE REAL STATIC FRAME <<<
\begin{figure}[htb!]
\centering
\scalebox{0.8}{
	\begin{tikzpicture}[scale=2,>={Stealth[inset=0mm,length=1.5mm,angle'=50]}]

		\node (origin) at (0,0) {};
		\draw [->] (   -2mm,  0   ) -- (   40mm,  0   ) node (xaxis) {};
		\draw [->] (      0, -2mm ) -- (   0   ,  40mm) node (yaxis) {};

		\node (reAxisLabel) at (42mm,0) {Re};
		\node (imAxisLabel) at (0,42mm) {Im};

		\draw [->, black!50, name path = daxis] (0,0) -- ({45mm*cos(25)}, {45mm*sin(25)});
		\draw [->, black!50, name path = qaxis] (0,0) -- ({45mm*cos(115)},{45mm*sin(115)});

		\node[right,black!50] (DAxisLabel) at ({47mm*cos(25) - 2mm} ,{47mm*sin(25)})  {$D = Re^{j\psi(t)}$};
		\node[black!50]       (QAxisLabel) at ({47mm*cos(115)},      {47mm*sin(115)}) {$Q = Ie^{j\psi(t)}$};

		\node [black!50] (omegat) at ({42mm*cos(32)},{42mm*sin(32)}) {$\omega(t)$};
		\draw [-{Stealth[inset=0mm,length=3.5mm,angle'=50]}, black!50, line width = 1mm] ({42mm*cos(20)},{42mm*sin(20)}) arc[start angle=20, end angle = 30, radius = 42mm];

		\draw [->,black!50,thick] ({38mm*cos(0)},{38mm*sin(0)}) arc[start angle=0, end angle = 23, radius = 38mm];
		\node [right,gray] (psilabel) at ({39mm*cos(11)},{39mm*sin(11)}) {$\psi(t)$};

		\node [right, stewartblue] (elabel) at ({35mm*cos(65)},{35mm*sin(65)}) {$x_\alpha + j x_\beta = Xe^{j\psi(t)}$};
		\node [right, stewartblue] (amplilabel) at ({17mm*cos(74)},{17mm*sin(74)}) {$m(t)$};
		\draw [->, stewartblue] (0,0) -- (elabel);

		\draw [gray, dashed] (elabel) -- (elabel |- xaxis) node (xtproj) {} ;
		\node [gray, below]  (xtprojlabel) at (xtproj.center) {$x(t)$} ;

		\draw [->, stewartblue] ({25mm*cos(25)},{25mm*sin(25)}) arc[start angle=25, end angle=54, radius = 25mm];
		\node [stewartblue] (phivlabel) at ({22mm*cos(39)},{22mm*sin(39)}) {$\phi(t)$};

		% Obtaining Ed and Eq projections
		\path [name path = edprojection] (elabel) -- +($(origin)-0.5*(QAxisLabel)$);
		\path [name intersections={of=edprojection and daxis, by=Ed}];
		\node [circle,fill=stewartblue,inner sep=1.5pt,label={[text=stewartblue]-60:$x_d$}] at (Ed) {};
		\draw [dashed,stewartblue] (Ed) -- (elabel);
		
		\path [name path = eqprojection] (elabel) -- +($(origin)-(DAxisLabel)$);
		\path [name intersections={of=eqprojection and qaxis, by=Eq}];
		\node [circle,fill=stewartblue,inner sep=1.5pt,label={[text=stewartblue]-120:$x_q$}] at (Eq) {};
		\draw [dashed,stewartblue] (Eq) -- (elabel);

	\end{tikzpicture}
	}
	\caption
[Phasorial schematic of the Dynamic Phasor Transform as a rotational transform.]
{Phasorial schematic of the Dynamic Phasor Transform as a rotational transform. In black the real-imaginary static frame, and in gray the rotated ``DQ'' frame that rotates at the apparent frequency $\omega(t)$. Naturally, the function $f(t) = x_\alpha + jx_\beta$ when projected onto the real frame is $x(t)$; however, when this quantity is projected onto the DQ frame one obtains the Dynamic Phasor $X(t) = x_d + jx_q$.}
	\label{fig:dynamic_phasor_imreaxis}
\end{figure} %>>>

	Notably, since the Static Phasor Operator is a particular case of the Dynamic Phasor Transform, the SPO is given as the particular case where the DQ frame that rotates at a fixed frequency $\omega_0$, such that the sinusoidal signals considered, when projected into this new rotating frame, become static complex numbers. In the literature, it is often said that the sinusoidal static signals are ``rotating vectors'' that decompose as static numbers onto the frame.

	In the Power System literature, the achievement of a DQ frame with time-varying rotating frequency is paramount because it allows the adoption of reference frames which frequency are time-varying. In general, phasorial diagrams of power systems are represented with respect to a reference or ``slack'' bus, generally rotating at the fixed synchronous frequency, even though the machine attached to this bus is subject to transient phenomena of amplitude and frequency. This generates a confusing phenomena that the ``slack'' machine is not static with respect to the ``slack'' reference; with the time-varying generalization of Figure \ref{fig:dynamic_phasor_imreaxis}, this allows adoption of a reference DQ frame which rotating frequency is time varying. Adopting this frequency as the frequency of the reference bus, then this slack bus is static with respect to the reference frame.
	Further, it follows from the definition is that the DPT is linear, as is its inverse. This can be proven in two ways: first, it stems directly from the basic fact that all the transforms involved ($\alpha\beta$,dq and $\rho$) are linear. Alternatively, one can easily rewrite theorem \ref{theo:ps_morphism} with time-varing quantities. Again, the comparison with Static Phasors is again inevitable; Figure \ref{fig:dpt_linearity} shows the linearity schematic of the DPT, in line with the linearity of static phasors as in Figure \ref{fig:spo_linearity}.

% STATIC PHASOR DIAGRAM LINEARITY <<<
\begin{figure}[htb]
\centering
	\begin{tikzpicture}[scale=1.5,>={Stealth[inset=0mm,length=1.5mm,angle'=50]}]
		\draw [->, thick, black!50] (   -30mm,  0   ) -- (   30mm,  0   );
		\draw [->, thick, black!50] (      0, -5mm ) -- (   0   ,  15mm);

		\node [gray] (imlabel) at ( 0mm, 17mm) {Im};
		\node [gray] (relabel) at (32mm,  0mm) {Re};

		\node (O) at (0,0)       {};
		\node (X) at (9mm,12mm)  {};
		%\node [stewartblue] (Xlabel) at ([shift=({0,2mm})]X)  {$X(t)e^{j\psi(t)}$};
		\node (Y) at (7mm,2mm)   {};
		\node (2Y) at (14mm,4mm) {};
		%\node [stewartgreen] (Ylabel) at ([shift=({-3mm,-3.5mm})]Y)  {$Y(t)e^{j\psi(t)}$};
		\node (2Y) at (14mm,4mm) {};
		%\node [stewartyellow] (2Ylabel) at ([shift=({3.5mm,-1mm})]2Y)  {$2Y(t)e^{j\psi(t)}$};

		\node [label={[color=stewartblue]right:$X(t)e^{j\psi(t)}$}] at           (33mm, 15mm) {};
		\node [label={[color=stewartgreen]right:$Y(t)e^{j\psi(t)}$}] at           (33mm, 12mm) {};
		\node [label={[color=stewartyellow]right:$2Y(t)e^{j\psi(t)}$}] at           (33mm, 09mm) {};
		\node [label={[color=stewartpurple]right:$\left[X(t) + 2Y(t)\right]e^{j\psi(t)}$}] at           (33mm, 06mm) {};

		\node (SUM) at (23mm,16mm) {};
		%\node [stewartpurple] (SUMlabel) at ([shift=({2mm, 2mm})]SUM)  {$\left[X(t) + 2Y(t)\right]e^{j\psi(t)}$};

		\draw [->, thick, stewartblue]  (O.center) -- (X.center);
		\draw [->, thick, stewartyellow]  (O.center) -- (2Y.center);
		\draw [->, thick, stewartgreen]  (O.center) -- (Y.center);
		\draw [thick, dashed, stewartblue!50]  (2Y.center) -- (SUM.center);
		\draw [thick, dashed, stewartgreen!50] (X.center) -- (SUM.center);

		\draw [->, thick, stewartpurple] (O.center) -- (SUM.center);

		\draw [->, gray, thick]  (   -35mm, -10mm  ) node (beginXaxis) {} -- (   35mm,  -10mm  ) node (endXaxis) {};
		\draw [->, gray, thick]  (      0,  -10mm  ) node (beginYaxis) {} -- (    0  ,  -70mm  ) node (endYaxis) {} ;

		\node [gray] (tlabel) at ( 0mm,-72mm) {$t$};

		\draw [thick,dashed,stewartblue!50]     (X) -- (beginXaxis -|   X);
		\draw [thick,dashed,stewartgreen!50]    (Y) -- (beginXaxis -|   Y);
		\draw [thick,dashed,stewartyellow!50]   (2Y) -- (beginXaxis -|  2Y);
		\draw [thick,dashed,stewartpurple!50] (SUM) -- (beginXaxis -| SUM);

		\begin{axis}[at={(-33.5mm,-71.5mm)}, rotate=-90, width=67mm, height=71mm, scale only axis, yticklabel=\empty, xticklabel=\empty, axis line style={draw=none}, tick style={draw=none}, scaled ticks = false, ymin = -1, ymax = 1.1]
			\addplot[domain=0:0.07, smooth, color = stewartblue  , samples=100] {0.25*(1 + 0.5*exp(-12*x)*sin(deg(500*3.14159*x)))*cos(deg(200*x*(1 - 6*exp(-150*x)*sin(deg(10*3.14159*x)))))};
			\addplot[domain=0:0.07, smooth, color = stewartgreen , samples=100] {0.22*(1 -1.5*exp(-50*x)*sin(deg(500*3.14159*x)))*cos(deg(200*x*(1 - 6*exp(-150*x)*sin(deg(10*3.14159*x)))))};
			\addplot[domain=0:0.07, smooth, color = stewartyellow, samples=100] {0.41*(1 -1.5*exp(-50*x)*sin(deg(500*3.14159*x)))*cos(deg(200*x*(1 - 6*exp(-150*x)*sin(deg(10*3.14159*x)))))};
			\addplot[domain=0:0.07, smooth, color = stewartpurple, samples=100] {0.25*(1 + 0.5*exp(-12*x)*sin(deg(500*3.14159*x)))*cos(deg(200*x*(1 - 6*exp(-150*x)*sin(deg(10*3.14159*x))))) + 0.44*(1 -1.5*exp(-50*x)*sin(deg(500*3.14159*x)))*cos(deg(200*x*(1 - 6*exp(-150*x)*sin(deg(10*3.14159*x)))))};
	%		\addplot[color = stewartblue, domain=0:720, smooth, samples=100] {15mm*cos(x+53.13)};
	%		\addplot[color = stewartyellow, domain=0:720, smooth, samples=100] {2*7.28mm*cos(x+15)};
	%		\addplot[color = stewartpurple, domain=0:720, smooth, samples=100] {28.02mm*cos(x+34.82)};
		\end{axis}

		\node [label={[color=stewartblue]right:$x(t)$}] at           (33mm, -40mm) {};
		\node [label={[color=stewartgreen]right:$y(t)$}] at          (33mm, -43mm) {};
		\node [label={[color=stewartyellow]right:$2y(t)$}] at        (33mm, -46mm) {};
		\node [label={[color=stewartpurple]right:$x(t) + 2y(t)$}] at (33mm, -49mm) {};

	\end{tikzpicture}
	\caption{Dynamic Phasor Transform linearity schematic.}
	\label{fig:dpt_linearity}
\end{figure} %>>>

%-------------------------------------------------
\section{The Dynamic Phasor Transform applied to linear systems} %<<<1

	We now apply the DPT and use its properties to study how it transforms linear systems. Given a nonstationarily-excited linear system differential model of the form

\begin{equation} \sum\limits_{k=0}^{n} \alpha_k x^{\left(k\right)} - m(t)\cos\left(\psi(t) + \phi(t)\right) = 0, \label{eq:linsys_sinusoidal_forcing}\end{equation}

	\noindent that is, $x(t)$ is governed by a LTI differential equation excited by some sinusoidal forcing at an apparent frequency $\omega(t)$, then how do the ``dq'' components of $x(t)$ behave at that same apparent frequency? In a geometric interpretation, let us assume that the static-frame quantity $f(t) = x_\alpha(t) + jx_\beta(t)$ of figure \ref{fig:dynamic_phasor_imreaxis} is the solution of

\begin{equation} \sum\limits_{k=0}^{n} \alpha_k f^{\left(k\right)} -  m\left(t\right)e^{j\left[\psi(t) + \phi(t)\right]} = 0, \label{eq:linsys_sinusoidal_forcing_complex}\end{equation}

	\noindent then how does the representation of $f(t)$ with respect to the DQ frame behave?

%-------------------------------------------------
\subsection{A dq-equivalent linear system} %<<<2

	To achieve an equivalent dq equation for the linear system, lemmas \ref{theo:dq_1p_diff} and \ref{lemma:1p_t_ndifftminus_product} prove operational and differential properties of the dq transform $\mathbf{T}_{\psi(t)}$ to be used in the proof of theorem \ref{theo:1p_ode_solution}, which provides the sought ``dq equivalent system'' from the original linear system . Lemma \ref{theo:dq_1p_diff} shows that $d^k\mathbf{x}_{dq}/dt^k$ can be written as compositions of the differentials of $\mathbf{T}$ and $x$, and the other way around.  This lemma allows for a matricial time derivation of dq quantities, that is, obtain $\dot{\mathbf{x}}_{dq}$ from $\dot{x}$ and vice-versa. Lemma \ref{lemma:1p_t_ndifftminus_product} generalizes the Chain Rule for a k-th order differential of a complex matrix by using then generalization of the k-th order Chain Rule for single-variable functions, known as Faà Di Bruno's formula \pcite{DiBruno1855}.
	
\begin{lemma}[n-th order time differentiation of $dq$ transformed phasor quantities]\label{theo:dq_1p_diff}%<<<
	Let $n\in\mathbb{N}^*$, $\mathbf{x}_{\alpha\beta}$ the $\alpha\beta$ transform of a sinusoid $x(t)$, $\mathbf{T}_\theta$ the $dq$ Transform operator where $\theta(t)$ is $C^n$-class, and $\mathbf{x}_{dq} = \mathbf{T}_\theta \mathbf{x}_{\alpha\beta}$. Then

\begin{equation} \dfrac{d^n \mathbf{x}_{dq}}{dt^n} = \dfrac{d^n\left(\mathbf{T}_\theta \mathbf{x}_{\alpha\beta}\right)}{dt^n} = \sum\limits_{k=0}^{n} {n\choose k} \left(\dfrac{d^{k} \mathbf{T}_\theta}{dt^k}\right) \left(\dfrac{d^{\left(n-k\right)} \mathbf{x}_{\alpha\beta}}{dt^{\left(n-k\right)}}\right), \end{equation}

	\noindent and

\begin{equation} \dfrac{d^n\mathbf{x}_{\alpha\beta}}{dt^n} = \dfrac{d^n\left(\mathbf{T}^{-1}_\theta \mathbf{x}_{dq}\right)}{dt^n} = \sum\limits_{k=0}^{n} {n\choose k} \left(\dfrac{d^{k} \mathbf{T}^{-1}_\theta}{dt^k}\right) \left(\dfrac{d^{\left(n-k\right)} \mathbf{x}_{dq}}{dt^{\left(n-k\right)}}\right) .\end{equation}

	Particularly for $n=1$,

\begin{equation} \dfrac{d\mathbf{x}_{\alpha\beta}}{dt} = \dfrac{d}{dt} \left(\mathbf{T}^{-1}_\theta \mathbf{x}_{dq}\right) = \mathbf{T}^{-1}_\theta \dfrac{d\mathbf{x}_{dq}}{dt} + \dfrac{d\mathbf{T}^{-1}_\theta}{dt} \mathbf{x}_{dq}, \end{equation}

	\noindent and

\begin{equation} \dfrac{d\mathbf{x}_{dq}}{dt} = \dfrac{d}{dt} \left(\mathbf{T}_\theta \mathbf{x}_{\alpha\beta}\right) = \mathbf{T}_\theta \dfrac{d\mathbf{x}_{\alpha\beta}}{dt} + \dfrac{d\mathbf{T}_\theta}{dt} \mathbf{x}_{\alpha\beta}, \end{equation}
 
	\noindent where

\begin{equation}
        \dfrac{d\mathbf{T}_\theta }{dt} =
\dfrac{d\theta}{dt}
\left[\begin{array}{cc}
         -\sin\left(\theta\right) &  \cos\left(\theta\right)                    \\[5mm]
         -\cos\left(\theta\right) & -\sin\left(\theta\right)
\end{array}\right]\end{equation}

	and

\begin{equation}
        \dfrac{d\mathbf{T}^{-1}_{\theta} }{dt} =
\dfrac{d\theta}{dt}
\left[\begin{array}{cc}
         -\sin\left(\theta\right) & -\cos\left(\theta\right)                    \\[5mm]
          \cos\left(\theta\right) & -\sin\left(\theta\right)
\end{array}\right]
\end{equation}

\end{lemma}
\textbf{Proof:} let $\mathbf{M}\in\left[\mathbb{R}\to\mathbb{C}^{m\times n}\right]$ and $\mathbf{G}\in\left[\mathbb{R}\to\mathbb{C}^{p\times q}\right]$ for some naturals $m,n,p,q$. $\mathbf{M}$ and $\mathbf{G}$ are defined by their elements $a_{ij}\left(t\right)\in\left[\mathbb{R}\to\mathbb{C}\right]$ such that $a_{ij}$ are $C^n$ class. Then the tangent matrix is defined as the element-wise differentiation

\begin{equation} \dfrac{d\mathbf{M}}{dt} = \left\{\dfrac{dm_{ij}}{dt}\right\} . \end{equation}

	Also suppose that $m,n,p,q$ are such that $\mathbf{MG}$ exists. Then

\begin{equation} \dfrac{d\left(\mathbf{MG}\right)}{dt} = \mathbf{M}\left(\dfrac{d\mathbf{G}}{dt}\right) + \left(\dfrac{d\mathbf{M}}{dt}\right) \mathbf{G} \end{equation}

	\noindent which comes directly from matrix (tensor) calculus \pcite{bishopTensorAnalysisManifolds1980}. Note that this equation is similar to a differentiation product rule $\left(fg\right)' = fg' + f'g$. Differentiating for the second derivative, one obtains

\begin{equation} \dfrac{d^2\left(\mathbf{MG}\right)}{dt^2} = \mathbf{M}\left(\dfrac{d^2\mathbf{G}}{dt^2}\right) + 2\left(\dfrac{d\mathbf{M}}{dt}\right)\left(\dfrac{d\mathbf{G}}{dt}\right) + \left(\dfrac{d^2\mathbf{M}}{dt^2}\right) \mathbf{G} \end{equation}

	\noindent which also looks like a second-order differentiation product rule: $\left(fg\right)'' = fg'' + 2f'g' + f''g$. And again for the third,

\small
\begin{equation}
	\dfrac{d^3\left(\mathbf{MG}\right)}{dt^3} = \mathbf{M}\left(\dfrac{d^3\mathbf{G}}{dt^3}\right) + 3\left(\dfrac{d^2\mathbf{M}}{dt^2}\right)\left(\dfrac{d\mathbf{G}}{dt}\right) + 3\left(\dfrac{d\mathbf{M}}{dt}\right)\left(\dfrac{d^2\mathbf{G}}{dt^2}\right) + \left(\dfrac{d^3\mathbf{M}}{dt^3}\right) \mathbf{G}
\end{equation}
\normalsize

	\noindent akin to the third-order differentiation product rule. These results suggest that the matrix product rule is given by an equation akin to that of the Leibnitz Rule for single-variable complex functions:

\begin{equation} \dfrac{d^n\left(\mathbf{MG}\right)}{dt^n} = \sum\limits_{k=0}^{n} {n\choose k} \left(\dfrac{d^k \mathbf{M}}{dt^k}\right)  \left(\dfrac{d^{\left(n-k\right)} \mathbf{G}}{dt^{\left(n-k\right)}}\right), \end{equation}

	where the zero-degree derivative is equal to the identity. The proof follows by induction: as shown, the results are true for $n=1,2,3$. By inductive hypothesis, suppose the result is true for some $n-1$. Then for some $n$,

\begin{align}
	 \dfrac{d^n\left(\mathbf{MG}\right)}{dt^n}
	&= \dfrac{d}{dt}\left[\dfrac{d^{\left(n-1\right)} \left(\mathbf{MG}\right)}{dt^{\left(n-1\right)}}\right] = \nonumber\\[3mm]
	&= \dfrac{d}{dt}\left[\sum\limits_{k=0}^{n-1} {n-1\choose k} \left(\dfrac{d^k \mathbf{M}}{dt^k}\right)  \left(\dfrac{d^{\left(n-1-k\right)} \mathbf{G}}{dt^{\left(n-1-k\right)}}\right)\right] = \nonumber\\[3mm]
	&= \sum\limits_{k=0}^{n-1} {n-1\choose k} \left[\left(\dfrac{d^{\left(k+1\right)} \mathbf{M}}{dt^{\left(k+1\right)}}\right)  \left(\dfrac{d^{\left(n-1-k\right)} \mathbf{G}}{dt^{\left(n-1-k\right)}}\right) + \left(\dfrac{d^k \mathbf{M}}{dt^k}\right)  \left(\dfrac{d^{\left(n-k\right)} \mathbf{G}}{dt^{\left(n-k\right)}}\right)\right]
\end{align}

	Substituting $j = k+1$,

\begin{equation}
	 \dfrac{d^n\left(\mathbf{MG}\right)}{dt^n} = \sum\limits_{j=1}^{n} {n-1\choose j-1} \left(\dfrac{d^{\left(j\right)} \mathbf{M}}{dt^{\left(j\right)}}\right)  \left(\dfrac{d^{\left(n-j\right)} \mathbf{G}}{dt^{\left(n-j\right)}}\right) + \sum\limits_{k=0}^{n-1}{n-1\choose k}\left(\dfrac{d^{k} \mathbf{M}}{dt^k}\right)  \left(\dfrac{d^{\left(n-k\right)} \mathbf{G}}{dt^{\left(n-k\right)}}\right)
\end{equation}

	It is clear that both parts of the sum are in fact the same expression but lacking the extremes:

\small
\begin{align}
	\dfrac{d^n\left(\mathbf{MG}\right)}{dt^n}
	&= \sum\limits_{j=1}^{n-1} {n-1\choose j-1} \left(\dfrac{d^{\left(j\right)} \mathbf{M}}{dt^{\left(j\right)}}\right)  \left(\dfrac{d^{\left(n-j\right)} \mathbf{G}}{dt^{\left(n-j\right)}}\right) + \sum\limits_{k=1}^{n-1}{n-1\choose k}\left(\dfrac{d^{k} \mathbf{M}}{dt^k}\right)  \left(\dfrac{d^{\left(n-k\right)} \mathbf{G}}{dt^{\left(n-k\right)}}\right) + \nonumber\\[3mm] &\hspace{30mm} + {n-1\choose n-1}\left(\dfrac{d^{\left(n\right)} \mathbf{M}}{dt^{\left(n\right)}}\right)  \left(\dfrac{d^{\left(0\right)} \mathbf{G}}{dt^{\left(0\right)}}\right) + {n\choose 0}\left(\dfrac{d^{\left(0\right)} \mathbf{M}}{dt^{\left(0\right)}}\right)  \left(\dfrac{d^{\left(n\right)} \mathbf{G}}{dt^{\left(n\right)}}\right) \nonumber\\[3mm]
%
%
%
	&= \sum\limits_{k=1}^{n-1} \overbrace{\left[{n-1\choose k} + {n-1\choose k-1}\right]}^{={n\choose k}} \left(\dfrac{d^{k} \mathbf{M}}{dt^k}\right)  \left(\dfrac{d^{\left(n-k\right)} \mathbf{G}}{dt^{\left(n-k\right)}}\right) + \left(\dfrac{d^{\left(n\right)} \mathbf{M}}{dt^{\left(n\right)}}\right)  \left(\dfrac{d^{\left(0\right)} \mathbf{G}}{dt^{\left(0\right)}}\right) + \left(\dfrac{d^{\left(0\right)} \mathbf{M}}{dt^{\left(0\right)}}\right)  \left(\dfrac{d^{\left(n\right)} \mathbf{G}}{dt^{\left(n\right)}}\right) \nonumber\\[3mm]
	&= \sum\limits_{k=1}^{n-1} {n\choose k} \left(\dfrac{d^{k} \mathbf{M}}{dt^k}\right) \left(\dfrac{d^{\left(n-k\right)} \mathbf{G}}{dt^{\left(n-k\right)}}\right) + {n\choose 0}\left(\dfrac{d^{\left(n\right)} \mathbf{M}}{dt^{\left(n\right)}}\right)  \left(\dfrac{d^{\left(0\right)} \mathbf{G}}{dt^{\left(0\right)}}\right) + {n\choose n}\left(\dfrac{d^{\left(0\right)} \mathbf{M}}{dt^{\left(0\right)}}\right)  \left(\dfrac{d^{\left(n\right)} \mathbf{G}}{dt^{\left(n\right)}}\right) \nonumber\\[3mm]
	&= \sum\limits_{k=0}^{n} {n\choose k} \left(\dfrac{d^{k} \mathbf{M}}{dt^k}\right) \left(\dfrac{d^{\left(n-k\right)} \mathbf{G}}{dt^{\left(n-k\right)}}\right) 
\end{align}
\normalsize

	\noindent and this result can be seen as Leibnitz Rule for single-variable matrix functions. The index changing is done by adopting $p = n-k$:

\begin{equation} \dfrac{d^n\left(\mathbf{MG}\right)}{dt^n} = \sum\limits_{p=0}^{n} {n\choose \left(n-p\right)} \left(\dfrac{d^{\left(n-p\right)} \mathbf{M}}{dt^{\left(n-p\right)}}\right) \left(\dfrac{d^p \mathbf{G}}{dt^p}\right) \end{equation}

	From combinatorics, $C^n_{\left(n-p\right)} = C^n_p$:

\begin{equation} \dfrac{d^n\left(\mathbf{MG}\right)}{dt^n} = \sum\limits_{p=0}^{n} {n\choose p} \left(\dfrac{d^{\left(n-p\right)} \mathbf{M}}{dt^{\left(n-p\right)}}\right) \left(\dfrac{d^p \mathbf{G}}{dt^p}\right)  \end{equation}

	Results follow immediately by adopting $\mathbf{x}_{\alpha\beta} = \mathbf{T}_\theta \mathbf{x}_{dq}$ and $\mathbf{x}_{dq} = \mathbf{T}^{-1}_\theta \mathbf{x}_{\alpha\beta}$. \hfill$\blacksquare$

\vspace{5mm}
\hrule
\vspace{5mm}
%>>>

\begin{lemma} \label{lemma:1p_t_ndifftminus_product}%<<<

	Let $n\in\mathbb{N}$ and consider the rotational transform $\mathbf{T}_\theta$ and the inverse $\mathbf{T}_{\left(-\theta\right)}$ where $\theta$ is n-th order differentiable. Then

\begin{equation} \mathbf{T}_\theta \dfrac{d^n\mathbf{T}^{-1}_\theta}{dt^n} = \sum\limits_{k=0}^n \mathbf{G}_k B_{\left(n,k\right)}\left(\dot{\theta},\ddot{\theta},...,\theta^{(n-k+1)}\right), \end{equation}

	\noindent where $B_{\left(n,k\right)}$ are the incomplete exponential Bell Polynomials and

\begin{equation}
\mathbf{G}_k = 
\left[\begin{array}{ccc}
 \cos\left( \dfrac{k\pi}{2}\right) &  -\sin\left(\dfrac{k\pi}{2}\right) \\[5mm]
 \sin\left( \dfrac{k\pi}{2}\right) & \phantom{-} \cos\left(\dfrac{k\pi}{2}\right)
\end{array}\right].
\end{equation}

	Particularly for $n=1$,

\begin{equation} \mathbf{T}_\theta\dfrac{d\mathbf{T}_\theta^{-1}}{dt} = \dfrac{d\theta}{dt} \left[\begin{array}{ccc}    0 & -1 \\[5mm] 1 & 0 \end{array}\right] \end{equation}

%	And for $n=2$,
%
%\begin{equation} \mathbf{T}_\theta\dfrac{d^2\mathbf{T}_\theta^{-1}}{dt^2} = \ddot{\theta}\left[\begin{array}{cc} 0 & -1 \\ 1 & 0\end{array}\right] + \left(\dot{\theta}\right)^2\left[\begin{array}{cc} -1 & 0 \\ 0 & -1\end{array}\right] \end{equation}

\end{lemma} 

\textbf{Proof:} for an arbitrary order $n \geq 0$, one needs to use the Faà Di Bruno's formula \pcite{DiBruno1855} for the n-th order Chain Rule. The formula states that, for two single-variable n-th order differentiable functions $f$ and $g$, the chain rule is given by

\begin{equation} \dfrac{d^n}{dx^n} f\left(g\left(x\right)\right)= \sum\limits_{k=0}^n f^{\left(k\right)}\left(g\left(x\right)\right) B_{\left(n,k\right)}\left(g'\left(x\right),g''\left(x\right),...,g^{\left(n-k+1\right)}\left(x\right)\right), \end{equation}

	\noindent where the $B_{\left(n,k\right)}$ are the incomplete exponential Bell Polynomials. Consider $t^{-1}_{\left(i,j\right)}$ as the $i,j$ element of $\mathbf{T}^{-1}$. Then	

\begin{equation} \dfrac{d^n}{dt^n} t^{-1}_{\left(i,j\right)} \left(\theta\left(t\right)\right)= \sum\limits_{k=0}^n \dfrac{d^k t^{-1}_{\left(i,j\right)}\left(\theta\right)}{d\theta^k} B_{\left(n,k\right)}\left(\dot{\theta},\ddot{\theta},...,\theta^{(n-k+1)}\right). \end{equation}

	But because the indexes $n$ and $k$ are not related to $i$ and $j$, 

\begin{gather} \dfrac{d^n\mathbf{T}^{-1}_\theta}{dt^n} = \nonumber\\[5mm]
	\left[\begin{array}{cc}
\sum\limits_{k=0}^n \dfrac{d^k t^{-1}_{\left(1,1\right)}}{d\theta^k} B_{\left(n,k\right)}\left(\dot{\theta},\ddot{\theta},...,\theta^{(n-k+1)}\right) & \sum\limits_{k=0}^n \dfrac{d^k t^{-1}_{\left(1,2\right)}}{d\theta^k} B_{\left(n,k\right)}\left(\dot{\theta},\ddot{\theta},...,\theta^{(n-k+1)}\right) \\[5mm]
\sum\limits_{k=0}^n \dfrac{d^k t^{-1}_{\left(2,1\right)}}{d\theta^k} B_{\left(n,k\right)}\left(\dot{\theta},\ddot{\theta},...,\theta^{(n-k+1)}\right) & \sum\limits_{k=0}^n \dfrac{d^k t^{-1}_{\left(2,2\right)}}{d\theta^k} B_{\left(n,k\right)}\left(\dot{\theta},\ddot{\theta},...,\theta^{(n-k+1)}\right)
\end{array}\right] = \nonumber\\[5mm]
%
	= \sum\limits_{k=0}^n  B_{\left(n,k\right)}\left(\dot{\theta},\ddot{\theta},...,\theta^{(n-k+1)}\right)\left[\begin{array}{ccc}
\dfrac{d^k t^{-1}_{\left(1,1\right)}}{d\theta^k} & \dfrac{d^k t^{-1}_{\left(1,2\right)}}{d\theta^k} \\[5mm]
\dfrac{d^k t^{-1}_{\left(2,1\right)}}{d\theta^k} & \dfrac{d^k t^{-1}_{\left(2,2\right)}}{d\theta^k}
\end{array}\right]
\end{gather}

	\noindent which in matrix form means

\begin{equation} \dfrac{d^n\mathbf{T}^{-1}_\theta}{dt^n} = \sum\limits_{k=0}^n \dfrac{d^k\mathbf{T}^{-1}_\theta}{d\theta^k} B_{\left(n,k\right)}\left(\dot{\theta},\ddot{\theta},...,\theta^{(n-k+1)}\right), \end{equation}

	But knowing that

\begin{equation}
	\left\{\begin{array}{l}
		\dfrac{d^n \cos\left(\theta\right)}{d\theta^n} = \cos\left(\theta + \dfrac{n\pi}{2}\right) \\[5mm]
		\dfrac{d^n \sin\left(\theta\right)}{d\theta^n} = \sin\left(\theta + \dfrac{n\pi}{2}\right)
	\end{array}\right. ,
\end{equation}

	\noindent then for $n\geq 1$,

\begin{equation} \dfrac{d^k \mathbf{T}^{-1}_\theta}{d\theta^k} =  \left[\begin{array}{cc} \cos\left(\theta + \dfrac{k\pi}{2}\right) & -\sin\left(\theta + \dfrac{k\pi}{2}\right) \\[5mm] \sin\left(\theta + \dfrac{k\pi}{2}\right) & \cos\left(\theta + \dfrac{k\pi}{2}\right)\end{array}\right] = \mathbf{T}^{-1}\left(\theta + \dfrac{k\pi}{2}\right).\end{equation}

	Now calculate the matrix multiplication:

\begin{align}
\mathbf{T}_\theta \dfrac{d^k\mathbf{T}_\theta^{-1}}{d\theta^k} = \mathbf{T}_\theta\mathbf{T}^{-1}\left(\theta + \dfrac{k\pi}{2}\right) = \mathbf{T}\left(-\dfrac{k\pi}{2}\right) &= \left[\begin{array}{cc} \cos\left(-\dfrac{k\pi}{2}\right) & \sin\left(-\dfrac{k\pi}{2}\right) \\[5mm] -\sin\left(-\dfrac{k\pi}{2}\right) & \phantom{-}\cos\left(- \dfrac{k\pi}{2}\right)\end{array}\right] \nonumber\\[5mm] &= \left[\begin{array}{cc} \cos\left(\dfrac{k\pi}{2}\right) & -\sin\left(\dfrac{k\pi}{2}\right) \\[5mm] \sin\left(\dfrac{k\pi}{2}\right) & \phantom{-}\cos\left(\dfrac{k\pi}{2}\right)\end{array}\right]
\end{align}

	Call this matrix $\mathbf{G_k}$ and the proof is complete. For the particular case $n=1$ one can use this result or simply compute directly:

\begin{align}
	\mathbf{T}_\theta\dfrac{d\mathbf{T}^{-1}_\theta}{dt} &= 
	\left[\begin{array}{cc}
		 \cos\left(\theta\right) & \sin\left(\theta\right) \\[5mm]
		-\sin\left(\theta\right) & \cos\left(\theta\right)
	\end{array}\right]\dfrac{d\theta}{dt}
	\left[\begin{array}{cc}
		-\sin\left(\theta\right) & -\cos\left(\theta\right) \\[5mm]
		\cos\left(\theta\right) & -\sin\left(\theta\right)
	\end{array}\right]
	%
	=\dfrac{d\theta}{dt}\left[\begin{array}{ccc}
		0 & -1 \\[5mm]
		1 &  0
	\end{array}\right]
\end{align}
\hfill$\blacksquare$

\vspace{5mm}
\hrule
\vspace{5mm}
%>>>

\begin{theorem}[Solutions to LTI ODEs with phasorial forcing] \label{theo:1p_ode_solution}%<<<

	Let $m\left(t\right),\theta\left(t\right)\in\left[\mathbb{R}\to\mathbb{R}\right]$ and consider the Hurwitz-stable linear ODE with a phasorial forcing:

\begin{equation} \sum\limits_{k=0}^{n} \alpha_k x^{\left(k\right)} - f(t) = 0, \label{eq:theo_1p_ode_solution_original_ode}\end{equation}

	\noindent with a set of initial conditions $x_0,x'_0,...,x^{(n-1)}_0$ where $f(t)$ admits a sinusoidal representation at some apparent frequency $\omega(t)$, which is supposed a $C^{\left(n-1\right)}$-class real function. Consider the ``dq equivalent'' system

\begin{equation} \sum\limits_{i=0}^n \mathbf{K}_i(t) \left(\dfrac{d^i \mathbf{z}_{dq}}{dt^i }\right) - \mathbf{f}_{dq} = 0 , \label{eq:theo_1p_ode_solution_dq_equiv}\end{equation}

	\noindent with a set of initial conditions $(\mathbf{z}_{dq})_0,(\mathbf{z}'_{dq})_0,...,(\mathbf{z}^{(n-1)}_{dq})_0$ , where $\mathbf{f}_{dq}$ is the $dq$ transform of the forcing at the frequency $\omega(t)$,

\begin{equation} \mathbf{K}_i(t) = \sum\limits_{k=i}^{n} \alpha_k{k\choose i} \left[\sum\limits_{c=0}^{k-i} \mathbf{G}_c B_{\left(k-i,c\right)}\left(\omega,\dot{\omega},\ddot{\omega},...,\omega^{(k-i-c)}\right) \right] ,\label{eq:ki_transf} \end{equation}

	\noindent are time-variant matrices where $B_{\left(i,j\right)}$ are the incomplete exponential Bell Polynomials and and $\mathbf{G}_k$ are calculated as

\begin{equation}
\mathbf{G}_k = 
\left[\begin{array}{ccc}
 \cos\left( \dfrac{k\pi}{2}\right) &  -\sin\left(\dfrac{k\pi}{2}\right) \\[5mm]
 \sin\left( \dfrac{k\pi}{2}\right) & \phantom{-} \cos\left(\dfrac{k\pi}{2}\right)
\end{array}\right]
\end{equation}

	Then there exist two positive reals $a,b$ such that the solution $x$ to the original ODE \eqref{eq:theo_1p_ode_solution_original_ode} satisfies

\begin{equation} \left\lVert \mathbf{x}_{\alpha\beta} - \mathbf{T}^{-1}_\psi\mathbf{z}_{dq}\right\rVert \leq ae^{-bt}, \label{eq:theo_1p_ode_solution_exp}\end{equation}

	\noindent with $\mathbf{x}_{dq}$ is the unique solution to the dq system \eqref{eq:theo_1p_ode_solution_dq_equiv}. Reestated, the solution $\mathbf{z}_{\alpha\beta}$ reconstructed by \eqref{eq:theo_1p_ode_solution_dq_equiv} is the globally steady-state stable solution of \eqref{eq:theo_1p_ode_solution_original_ode}.

\end{theorem}
\textbf{Proof:} consider the original single-phase LTI ODE

\begin{equation} \sum\limits_{k=0}^n \alpha_k x^{\left(k\right)} - f(t) = 0.\end{equation}

	By hypothesis this system is Hurwitz stable, that is, the solution $x(t)$ tends exponentially to a particular solution: $\left\lVert x(t) - x_p(t)\right\rVert \leq ae^{-bt}$ for some two reals $a$ and $b$. Finding a particular solution $z(t)$, let $z_0,z'_0,...,z^{(n-1)}_0$ the initial conditions of the particular solution. Using the $\alpha\beta$ transform to generate an equivalent two-dimensional ODE:

\begin{equation} \sum\limits_{k=0}^n \alpha_k \left[\begin{array}{c} z_\alpha \\ z_\beta \end{array}\right]^{\left(k\right)} - \left[\begin{array}{c} f_\alpha(t)\\ f_\beta(t) \end{array}\right] = 0\end{equation}

	And transform the equation through $\mathbf{T}_\psi$:

\begin{gather}
	\mathbf{T}_\psi\left\{\sum\limits_{k=0} \alpha_k \left[\begin{array}{c} z_\alpha \\ z_\beta  \end{array}\right]^{\left(k\right)}\right\} - \mathbf{T}_\psi\left\{\left[\begin{array}{c} f_\alpha(t) \\ f_\beta(t) \end{array}\right]\right\} = 0 \\[5mm]
%
	\sum\limits_{k=0}^n \alpha_k \mathbf{T}_\psi\left[\begin{array}{c} z_\alpha \\ z_\beta \end{array}\right]^{\left(k\right)} - \mathbf{f}_{dq} = 0
\end{gather}

	\noindent where $\mathbf{f}_{dq}$ is the $dq$ transform of the forcing at the chosen frequency $\omega(t)$. Let $\mathbf{z}_{dq} = \mathbf{T}_\psi \left[z_\alpha,z_\beta\right]$ the $dq$ transform of $\mathbf{z}_{\alpha\beta}$:

\begin{equation} \sum\limits_{k=0}^n \alpha_k \mathbf{T}_\psi\left(\mathbf{T}_\psi^{-1}\mathbf{z}_{dq}\right)^{\left(k\right)} - \mathbf{f}_{dq} = 0,\  \psi(t) = \int_0^t \omega(s)ds \end{equation}

	Apply lemma \ref{theo:dq_1p_diff}:

\begin{equation} \sum\limits_{k=0}^n \alpha_k\left\{\mathbf{T}_\psi\left[ \sum\limits_{p=0}^{k} {k\choose p} \left(\dfrac{d^{p} \mathbf{T}^{-1}_\psi}{dt^p}\right) \left(\dfrac{d^{\left(k-p\right)} \mathbf{z}_{dq}}{dt^{\left(k-p\right)}}\right) \right]\right\} - \mathbf{f}_{dq} = 0 \end{equation}

	And because both $\mathbf{T}$ and $\mathbf{T}^{-1}$ are linear,

\begin{equation} \sum\limits_{k=0}^n \alpha_k \sum\limits_{p=0}^{k} {k\choose p} \mathbf{T}_\psi\left[\left(\dfrac{d^{\left(k-p\right)} \mathbf{T}^{-1}_\psi}{dt^{\left(k-p\right)}}\right) \left(\dfrac{d^p \mathbf{z}_{dq}}{dt^p}\right) \right] - \mathbf{f}_{dq} = 0 \end{equation}

	Now apply lemma \ref{lemma:1p_t_ndifftminus_product}:

\begin{equation} \sum\limits_{k=0}^n \alpha_k \left\{\sum\limits_{p=0}^{k} {k\choose p} \left[\sum\limits_{c=0}^{k-p} \mathbf{G}_c B_{\left(k-p,c\right)}\left(\omega,\dot{\omega},\ddot{\omega},...,\omega^{(k-p-c)}\right) \right] \left(\dfrac{d^p \mathbf{z}_{dq}}{dt^p }\right)\right\} - \mathbf{f}_{dq} = 0 \end{equation}

	To isolate the derivatives of $\mathbf{z}_{dq}$, one must solve the triangular sum of this equation. The 0-th derivatives are present at all $k$ indexes; the first, for the $k$ indexes $1$ through $n$; the second for $2$ to $n$. In general, the i-th derivative is present for indexes $k$ from $i$ to $n$.

\begin{equation} \sum\limits_{i=0}^n \left\{\sum\limits_{k=i}^{n} \alpha_k{k\choose i} \left[\sum\limits_{c=0}^{k-i} \mathbf{G}_c B_{\left(k-i,c\right)}\left(\omega,\dot{\omega},\ddot{\omega},...,\omega^{(k-i-c)}\right) \right]\right\} \left(\dfrac{d^i \mathbf{z}_{dq}}{dt^i }\right) - \mathbf{f}_{dq} = 0 .\end{equation}

	Finally, we group the matrix inside the sum as

\begin{equation} \mathbf{K}_i(t) = \sum\limits_{k=i}^{n} \alpha_k{k\choose i} \left[\sum\limits_{c=0}^{k-i} \mathbf{G}_c B_{\left(k-i,c\right)}\left(\omega,\dot{\omega},\ddot{\omega},...,\omega^{(k-i-c)}\right) \right] \end{equation}

	yielding

\begin{equation} \sum\limits_{i=0}^n \mathbf{K}_i(t) \left(\dfrac{d^i \mathbf{z}_{dq}}{dt^i }\right) - \mathbf{f}_{dq} = 0 .\end{equation}

	Therefore, $z(t) = z_\alpha(t)$ where $\mathbf{z}_{\alpha\beta} = \mathbf{T}^{-1}_{\psi(t)}\mathbf{z}_{dq}$ is a particular solution to the original system, and \eqref{eq:theo_1p_ode_solution_exp} follows. \hfill$\blacksquare$

\vspace{5mm}
\hrule
\vspace{5mm}
%>>>

	In short, theorem \ref{theo:1p_ode_solution} shows how a linear system is ``converted'' into a ``dq version'', as depicted in Figure \ref{fig:dqification}. The figure shows the ``original'' linear system in time domain on top, and a ``dq version'' on the bottom. In short, theorem \ref{theo:1p_ode_solution} shows that the system in time domain, as a red block, can be converted into a dq version (blue block) by converting the input signal $\mathbf{f}(t)$ into its ``dq'' version $\mathbf{f}_{dq}(t)$, then processed in the dq frame, and which response $\mathbf{x}_{dq}(t)$ is reconstructed into its time counterpart through the inverse transformation $\mathbf{T}^{-1}_{\psi(t)}$.

	One very important note about theorem \ref{theo:1p_ode_solution} comes about initial conditions. In its presented form, the theorem supposes that the ``dq equivalent'' system \eqref{eq:theo_1p_ode_solution_dq_equiv} is such that the initial conditions of $\mathbf{z}_{dq}$ are arbitrary, that is, the initial conditions of $\mathbf{z}_{dq}$ and its derivatives up to the $(n-1)$-th of the equivalent dq system do not need to reconstruct the initial conditions of the original system $x_0,x'_0,...,x^{(n-1)}_0$. In this case, $\mathbf{z}_{dq}$ reconstructs $x(t)$ with fading exponential precision, as per \eqref{eq:theo_1p_ode_solution_exp}.

	This confusing arbitrariness in the initial conditions of the dq equivalent system is needed because it is often interesting to have this system not start exactly from the same conditions as the original system. Such necessity will become more apparent later. It is, however, immediate to note that if the initial conditions of $\mathbf{z}_{dq}$ and its derivatives reconstruct the initial conditions of $x(t)$, then  $\left\lVert \mathbf{x}_{\alpha\beta} - \mathbf{T}^{-1}_\psi\mathbf{z}_{dq}\right\rVert  = 0$ at initial time. Because in Hurwitz linear systems the only stability possible is the exponential, the difference beween the general solution $x(t)$ and the particular solution $\mathbf{T}^{-1}_\psi\mathbf{z}_{dq}$ can only decrease in time and, since this distance is null at initial time, it then remains null for all subsequent time instants. In other words, if the initial conditions of the dq system are chosen \textit{just right}, then its solution is exactly $\mathbf{x}_{dq}$, and $\mathbf{T}^{-1}_\psi\mathbf{x}_{dq}$ reconstructs $x(t)$ in time \textbf{loslessly}.

%-------------------------------------------------
\subsection{Complexification of LTI ODEs with phasorial forcing}\label{subsec:complexification} %<<<2
	
	We now want to use the complexification operator $\rho$ to escalate the results of theorem \ref{theo:1p_ode_solution} to the Dynamic Phasor $X(t)$ of the sinusoid $x(t)$. First it is shown that the $dq$ transform $\mathbf{T}_{\psi(t)}$, and its particularizations $\mathbf{G}_k$, are equivalent to rotations on the Dynamic Phasor space. This is shown by theorem \ref{theo:complex_space_operations} which proves that any countersymmetric matrix like $\mathbf{T}_\psi(t)$ is equivalent to a rotation in the complex domain.

\begin{theorem}[$dq$ and complex space operations] \label{theo:complex_space_operations}%<<<
	Consider $\mathbf{x}\in\left[\mathbb{R}\to\mathbb{R}^2\right]$, and $X \simeq \mathbf{x}$ its Dynamic Phasor representation. Take a countersymmetric matrix $\mathbf{A}\in\mathbb{R}^{2\times 2}$, that is, a matrix defined as

\begin{equation} \mathbf{A} = \left[\begin{array}{cc} a & -b \\ b & a \end{array}\right],\ a,b \in\mathbb{R} \label{eq:countersymmetric_matrix}. \end{equation}

	Then,

\begin{equation} \mathbf{A}\mathbf{x} \simeq \left(a + jb\right)X = Me^{j\phi} X \end{equation}

	\noindent where $M = \sqrt{\det\left(\mathbf{A}\right)} = \left\lvert a + jb\right\rvert = \sqrt{a^2 + b^2}$ and $\phi = \arg\left(a + jb\right)$.  Particularly, this implies the $\mathbf{T}_\psi(t)$ operator in the $\alpha\beta$ space is equivalent to a rotation by $-\psi(t)$ on the complex space, that is, a multiplication by $e^{-j\psi(t)}$:

\begin{equation} \mathbf{T}_{\psi(t)} \mathbf{x} \simeq e^{-j\psi(t)} X \end{equation}

	\noindent and also that the $\mathbf{G}_k$ operator in the $\alpha\beta$ space is equivalent to a rotation by $j^k$on the complex space:

\begin{equation} \mathbf{G}_k \mathbf{x} \simeq j^kX \end{equation}

\end{theorem}
\textbf{Proof:} calculate $\mathbf{y}$ such that

\begin{equation}
	\mathbf{y} = \mathbf{A}\mathbf{x} = M\left[\begin{array}{cc} a & -b \\ b & a \end{array}\right]\left[\begin{array}{c} x_d \\[3mm] x_q\end{array}\right] = \left[\begin{array}{c} ax_d - bx_q \\[3mm] ax_d + bx_q \end{array}\right]
\end{equation}

	At the same time, consider the complex number

\begin{equation} Y = \left(a + jb\right) X = \left(a + jb\right)\left(x_d + jx_q\right) = \left(ax_d - bx_q\right) + j\left(bx_d + ax_q \right) \end{equation}

	Meaning $Y = \left[1,j\right] \mathbf{y}$ and $\mathbf{y} = \left[\Re\left(Y\right),\Im\left(Y\right)\right]^\intercal$, therefore $Y \simeq  \mathbf{y}$. The results for $\mathbf{T}_{\psi}$ and $\mathbf{G}_k$ follows immediately adopting $\mathbf{A} = \mathbf{T}_{\psi}$ or $\mathbf{G}_k$. The polar form follows once $\mathbf{A}$ is written as

\begin{equation} \mathbf{A} = \sqrt{a^2 + b^2} \left[\begin{array}{cc} \dfrac{a}{\sqrt{a^2+b^2}} & \dfrac{b}{\sqrt{a^2+b^2}} \\[5mm] \dfrac{-b}{\sqrt{a^2+b^2}} & \dfrac{a}{\sqrt{a^2+b^2}} \end{array}\right] = M\left[\begin{array}{cc} \cos\left(\phi\right) & -\sin\left(\phi\right) \\[3mm] \sin\left(\phi\right) & \cos\left(\phi\right) \end{array}\right]  \end{equation}

\hfill$\blacksquare$

\vspace{5mm}
\hrule
\vspace{5mm}
%>>>

	Theorem \ref{theo:complex_space_operations} is a direct reflex of the fact that any countersymmetric matrix $\mathbf{A}$ as in \eqref{eq:countersymmetric_matrix} is diffeomorphic to a complex number $a + jb$. As a matter of fact the entire complex numbers can be constructed as a set of such matrices \pcite{ahlfors1979complex}. We now want to prove that the complexification operator $\rho$ maintains the differentiation operation, that is, if a signal $x(t)$ is represented by a Dynamic Phasor $X(t)$, then the derivatives of both signals are also related, that is, $x^{(k)}(t)$ is represented by $X^{(k)}(t)$ with $k\geq 1$.

\begin{theorem}[Invariancy of differentiation under the complex equivalence operator] \label{corollary:invariance_differentiation} %<<<
	The differentiation operation is invariant under the complex equivalence operator $\rho$, that is,

\begin{equation} \mathbf{D}^k_\mathbb{C}\left[\rho\left[\mathbf{x}\right]\right] = \rho\left[\mathbf{D}^k_{\mathbb{R}^2}\left[\mathbf{x}\right]\right]  \text{ for any } k\in\mathbb{N} \end{equation}

	\noindent or in shorter version,

\begin{equation} \mathbf{x} \simeq X \Leftrightarrow \dfrac{d^k\mathbf{x}}{dt^k} \simeq \dfrac{d^k X}{dt^k} \text{ for any } k\in\mathbb{N}\end{equation}
\end{theorem}
\textbf{Proof:} let $\mathbf{D}_{\mathbb{R}}$ denote the differential operator on $\mathbb{R}$, $\mathbf{D}_{\mathbb{C}}$ denote the simple derivative on the complex numbers, and $\mathbf{D}_{\mathbb{R}^2}$ the operator on $\mathbb{R}^2$, that is,

\begin{equation} \dfrac{d\mathbf{x}(t)}{dt} = \mathbf{D}_{\mathbb{R}^2}\left[\mathbf{x}\left(t\right)\right] = \left[\begin{array}{c} \mathbf{D}_{\mathbb{R}}\left[x_d(t)\right] \\[5mm] \mathbf{D}_{\mathbb{R}}\left[x_q(t)\right] \end{array}\right]. \end{equation}

	Two proofs are possible. The first uses functional analysis: by the Chain Rule on Banach Spaces, for some $\mathbf{x}\in\left[\mathbb{R}\to\mathbb{R}^2\right]$,

\begin{equation} \mathbf{D}_\mathbb{C}\left[\rho\left[\mathbf{x}\right]\right] = \mathbf{D}_\mathbb{C}\left[\rho \circ \mathbf{x}\right] = \dfrac{\delta \rho\left[\mathbf{x}\right]}{\delta \mathbf{x}}\left[\mathbf{D}_{\mathbb{R}^2}\left[\mathbf{x}\right]\right]. \end{equation}

	By theorem \ref{theo:rho_diff_inf}, the variational derivative $\delta\rho\left[\mathbf{x}\right]$ is equal to $\rho$ itself:

\begin{equation} \mathbf{D}_\mathbb{C}\left[\rho\left[\mathbf{x}\right]\right] = \rho\left[\mathbf{D}_{\mathbb{R}^2}\left[\mathbf{x}\right]\right], \end{equation}

	\noindent proving the proposition for $n=1$. For an arbitrary $n\in\mathbb{N}$ the process is the same, noting that the $n$-th derivative $\delta\rho\left[\mathbf{x}\right]$ always exists due to the infinitely diffeomorphic nature of $\rho$, is linear by definition and equal to $\rho$ itself. The second proof is done by simple inspection and induction. We can directly compute that $\rho$ applied to the derivative $d\mathbf{x}/dt$ is equivalent to a functional $\rho\left[D\left[\mathbf{x}\right]\right]$ combined:

\begin{equation} \rho\left(\dfrac{d\mathbf{x}}{dt}\right) = \rho\left[\mathbf{D}_{\mathbb{R}^2}\left[\mathbf{x}\right]\right] = \left(\rho\circ \mathbf{D}_{\mathbb{R}^2}\right)\left[\mathbf{x}\right]\end{equation}

	For a complex function $z(t) = x(t) + jy(t)\in\left[\mathbb{R}\to\mathbb{C}\right]$ where $x,y\in\left[\mathbb{R}\to\mathbb{R}\right]$, 

\begin{equation} \mathbf{D}_{\mathbb{C}}\left[z(t)\right] = \mathbf{D}_{\mathbb{R}}\left[x(t)\right] + j\mathbf{D}_{\mathbb{R}}\left[y(t)\right] .\end{equation}

	Then

\begin{equation} \dfrac{d}{dt}\left(\rho\left[\mathbf{x}\right]\right) = \mathbf{D}_{\mathbb{C}}\left[\rho\left[\mathbf{x}\right]\right] = \left(\mathbf{D}_{\mathbb{C}}\circ \rho\right)\left[\mathbf{x}\right] .\end{equation}

	We first show the proposition for $k=1$, that is, that the differentiation operator $D$ and the complexification operator $\rho$ commute, that is,

\begin{equation} \left(\rho\circ \mathbf{D}_{\mathbb{R}^2}\right)\left[\mathbf{x}\right]\equiv \left(\mathbf{D}_{\mathbb{C}}\circ \rho\right)\left[\mathbf{x}\right]\end{equation}

	And this can be done by a direct calculation:

\begin{equation} \left(\rho\circ \mathbf{D}_{\mathbb{R}^2}\right)\left[\mathbf{x}\right] = \left[1,j\right]\left[\begin{array}{c} \mathbf{D}_{\mathbb{R}}\left[x_d(t)\right] \\[5mm] \mathbf{D}_{\mathbb{R}}\left[x_q(t)\right] \end{array}\right] = \mathbf{D}_{\mathbb{R}}\left[x_d(t)\right] + j\mathbf{D}_{\mathbb{R}}\left[x_q(t)\right]\end{equation}

	\noindent but at the same time

\begin{equation} \left( \mathbf{D}_{\mathbb{C}}\circ\rho\right)\left[\mathbf{x}\right] = \mathbf{D}_{\mathbb{C}}\left[x_d(t) + jx_q(t)\right]  = \mathbf{D}_{\mathbb{R}}\left[x_d(t)\right] + j\mathbf{D}_{\mathbb{R}}\left[x_q(t)\right] \end{equation}

	\noindent proving both operators are equivalent. The next step is proving that the commutation of $D$ and $\rho$ is maintained throughout the differentiation orders, that is,

\begin{equation} \left(\rho\circ \mathbf{D}^k_{\mathbb{R}^2}\right)\left[\mathbf{x}\right]\equiv \left(\mathbf{D}^k_{\mathbb{C}}\circ \rho\right)\left[\mathbf{x}\right]\end{equation}

	\noindent for any natural $k$. This can be done by induction. The base case $k=1$ has been proven. For the inductive hypothesis, suppose the statement holds for a $k$. Then

\begin{equation} \left(\rho\circ \mathbf{D}^{\left(k+1\right)}_{\mathbb{R}^2}\right) = \left(\rho\circ \left(\mathbf{D}^k_{\mathbb{R}^2}\circ \mathbf{D}_{\mathbb{R}^2}\right)\right)  .\end{equation} 

	But since function composition is associative,

\begin{equation} \left(\rho\circ \left(\mathbf{D}^k_{\mathbb{R}^2}\circ \mathbf{D}_{\mathbb{R}^2}\right)\right) = \left(\left(\rho\circ \mathbf{D}^k_{\mathbb{R}^2}\right)\circ \mathbf{D}_{\mathbb{R}^2}\right)  \end{equation}

	\noindent and using the inductive hypothesis,

\begin{equation} \left(\left(\rho\circ \mathbf{D}^k_{\mathbb{R}^2}\right)\circ \mathbf{D}_{\mathbb{R}^2}\right) = \left(\left( \mathbf{D}^k_\mathbb{C}\circ\rho\right)\circ \mathbf{D}_{\mathbb{R}^2}\right) \end{equation}

	\noindent and again using association,

\begin{equation} \left(\left( \mathbf{D}^k_\mathbb{C}\circ\rho\right)\circ \mathbf{D}_{\mathbb{R}^2}\right) = \left( \mathbf{D}^k_\mathbb{C}\circ \left(\rho\circ \mathbf{D}_{\mathbb{R}^2}\right)\right) .\end{equation}

	Now using that the property is knowingly true for $k=1$,

\begin{equation} \left( \mathbf{D}^k_\mathbb{C}\circ \left(\rho\circ \mathbf{D}_{\mathbb{R}^2}\right)\right) = \left( \mathbf{D}^k_\mathbb{C}\circ \left(\mathbf{D}_\mathbb{C}\circ\rho\right)\right) = \left(\left( \mathbf{D}^k_\mathbb{C}\circ \mathbf{D}_\mathbb{C}\right)\circ\rho\right) = \left( \mathbf{D}^{\left(k+1\right)}_\mathbb{C}\circ\rho\right) \end{equation}

	\noindent which then proves that

\begin{equation} \left(\rho\circ \mathbf{D}^{\left(k+1\right)}_{\mathbb{R}^2}\right) = \left( \mathbf{D}^{\left(k+1\right)}_\mathbb{C}\circ\rho\right) \end{equation} 

	\noindent and the proposition is proven by induction. In shorter equivalence notation,

\begin{equation} \mathbf{x} \simeq X \Rightarrow \dfrac{d^k\mathbf{x}}{dt^k} \simeq \dfrac{d^kX}{dt^k},\ k\in\mathbb{N}. \end{equation}

	The converse implication is immediate once one notices that $X(t)$ necessarily reconstructs $x(t)$, including its initial conditions; therefore, if $d^kX/dt^k \simeq d^k\mathbf{x}/dt^k$ then one can integrate $d^kX/dt^k$ a number of $k$ times, using those initial conditions, to obtain $\mathbf{x}$ directly. Alternatively, one can repeat this theorem proof for $\rho^{-1}$, and the proof is identical. \hfill$\blacksquare$

\vspace{5mm}
\hrule
\vspace{5mm} %>>>

	Therefore, the diffeomorphic nature of the complexification operator means that it makes possible to transform the real differential equation of an electrical grid on the variable $\mathbf{x}$, the $dq$ transform of a phasorial quantity, onto a complex differential equation on its complex version $X$. Theorem \eqref{theo:1p_ode_solution} proves that the time differential equation of the electrical grid of the form \eqref{eq:theo_1p_ode_solution_original_ode} can be transformed into an ODE for the $dq$ transformed version \eqref{eq:theo_1p_ode_solution_dq_equiv}. The objective is to use the complexification operator and its properties to show that the dq-equivalent equation is also equivalent to a differential equation in the Dynamic Phasor complex space.

\begin{theorem}[Complex equivalence of phasorially excited LTI ODEs]\label{corollary:complex_equivalence_phasorialodes} %<<<

	Take the LTI ODE \eqref{eq:theo_1p_ode_solution_original_ode} of theorem \ref{theo:1p_ode_solution}, the same apparent frequency $\omega(t)$ signal, and the dq-equivalent ODE to the complex differential equation \eqref{eq:theo_1p_ode_solution_dq_equiv}. Consider the complex differential equation

\begin{equation} \sum\limits_{i=0}^n \beta_i^n(t) Z^{(i)} - F = 0,\ Z(t) = z_d(t) + jz_q(t), \label{eq:theo_1p_ode_equivalent_complex_ode} \end{equation}

	\noindent equipped with	initial conditions $Z_0,Z'_0,Z''_0,...,Z^{(n-1)}_0$ calculated from the initial conditions of the dq system as

\begin{equation} Z_0 = z_{d0} + jz_{q0},\ Z'_0 = z'_{d0} + jz'_{q0},\ ...\ ,Z^{(n-1)}_0 = z^{(n-1)}_{d0} + jz^{(n-1)}_{q0}. \end{equation}
	
	\noindent where $F = \mathbf{P_D^\omega}\left[f\right]$ is the Dynamic Phasor Transform of the forcing $f(t)$, and the $\beta_i^n(t)$ are time-varying complex coefficients given by

\begin{equation} \beta_i^n(t) = \sum\limits_{k=i}^{n} \alpha_k{k\choose i} \left[\sum\limits_{c=0}^{k-i} j^cB_{\left(k-i,c\right)}\left(\omega,\dot{\omega},\ddot{\omega},...,\omega^{(k-i-c)}\right) \right].  \end{equation}

	\noindent Then $z(t) = \mathbf{P_D^{\left(-\omega\right)}}\left[Z\right]$ and there exist $a,b\in\mathbb{R}^+$ such that

\begin{equation} \left\lVert x(t) - \mathbf{P_D^{\left(-\omega\right)}}\left[Z\right] \right\rVert \leq ae^{-bt} ,\end{equation} 

	\noindent or, in other words, the solution $z(t)$ reconstructed by $\mathbf{P_D^{\left(-\omega\right)}}\left[Z\right]$ is the globally steady state exponentially stable solution of the original LTI ODE. Particularly, if the initial conditions of $Z(t)$ reconstruct the initial conditions of $x(t)$ at initial time, that is,

\begin{equation} Z_0 = x_{d0} + jx_{q0},\ Z'_0 = x'_{d0} + jx'_{q0},\ ...\ ,Z^{(n-1)}_0 = x^{(n-1)}_{d0} + jx^{(n-1)}_{q0}. \end{equation}

	\noindent then $Z(t) = X(t)$, that is, \eqref{eq:theo_1p_ode_equivalent_complex_ode} reconstructs $x(t)$ loslessly.
\end{theorem}
\textbf{Proof:} continuing from theorem \ref{theo:1p_ode_solution}, pick the dq equivalent system
\begin{equation} \sum\limits_{i=0}^n \mathbf{K}_i(t) \left(\dfrac{d^i \mathbf{z}_{dq}}{dt^i }\right) - \mathbf{f}_{dq} = 0 , \label{theo:1p_ode_solution_2}\end{equation}

	\noindent where

\begin{equation} \mathbf{K}_i(t) = \sum\limits_{k=i}^{n} \alpha_k{k\choose i} \left[\sum\limits_{c=0}^{k-i} \mathbf{G}_c B_{\left(k-i,c\right)}\left(\omega,\dot{\omega},\ddot{\omega},...,\omega^{(k-i-c)}\right) \right] .\end{equation}

	We first note that the $\mathbf{K}_i(t)$ matrices are countersymmetric, because they are composed of compositions of the $\mathbf{G}_k$, which are countersymmetric, multiplied by the $\alpha_k$ and the $B_{\left(k-i,c\right)}$ — which are one-dimensional numbers. As such, we can use theorem \ref{theo:complex_space_operations}; applying $\rho$ to \eqref{theo:1p_ode_solution_2} and using the linearity of $\rho$,

\begin{equation} \rho\left[\sum\limits_{i=0}^n \mathbf{K}_i\mathbf{z}^{(i)}_{dq} - \mathbf{f}_{dq}\right] = 0 \Leftrightarrow \sum\limits_{i=0}^n \rho\left[\mathbf{K}_i\mathbf{z}^{(i)}_{dq}\right] - \rho\left[\mathbf{f}_{dq}\right] = 0 . \label{eq:complex_equiv_odes_4}\end{equation}

	Denote  $\rho\left[\mathbf{f}_{dq}\right] = F(t)$ and the theorem resumes to calculating $\rho\left[\mathbf{K}_i\mathbf{z}^{(i)}_{dq}\right]$. Direct computation yields

\begin{equation} \rho\left[\mathbf{K}_i\mathbf{z}^{(i)}_{dq}\right] = \rho\left[\left\{\sum\limits_{k=i}^{n} \alpha_k{k\choose i} \left[\sum\limits_{c=0}^{k-i} \mathbf{G}_c B_{\left(k-i,c\right)}\left(\omega,\dot{\omega},\ddot{\omega},...,\omega^{(k-i-c)}\right) \right]\right\}\mathbf{z}^{(i)}_{dq}\right] .\end{equation}

	Now using the linearity of matrix and scalar multiplications,

\begin{equation} \rho\left[\mathbf{K}_i\mathbf{z}^{(i)}_{dq}\right] = \rho\left[\sum\limits_{k=i}^{n} \left[ \sum\limits_{c=0}^{k-i} \alpha_k{k\choose i} B_{\left(k-i,c\right)}\left(\omega,\dot{\omega},\ddot{\omega},...,\omega^{(k-i-c)}\right) \mathbf{G}_c\mathbf{z}^{(i)}_{dq}\right] \right].\end{equation}

	Again using the linearity of $\rho$, this functional can act inside the sums and the scalar portion can be noted outside its application:

\begin{equation} \rho\left[\mathbf{K}_i\mathbf{z}^{(i)}_{dq}\right] = \sum\limits_{k=i}^{n} \left[ \sum\limits_{c=0}^{k-i} \alpha_k{k\choose i} B_{\left(k-i,c\right)}\left(\omega,\dot{\omega},\ddot{\omega},...,\omega^{(k-i-c)}\right)  \rho\left[\mathbf{G}_c\mathbf{z}^{(i)}_{dq}\right] \right]. \label{eq:complex_equiv_odes_1}\end{equation}

	Now we use theorem \ref{theo:complex_space_operations} to yield that

\begin{equation} \rho\left[\mathbf{G}_c\mathbf{z}^{(i)}_{dq}\right] = j^c \rho\left[\mathbf{z}^{(i)}_{dq}\right] \label{eq:complex_equiv_odes_5}\end{equation}

	\noindent and by theorem \ref{corollary:invariance_differentiation}, $\rho\left[\mathbf{z}^{(i)}_{dq}\right] = Z^{(i)}$ and \eqref{eq:complex_equiv_odes_5} is equal to

\begin{equation} \rho\left[\mathbf{G}_c\mathbf{z}^{(i)}_{dq}\right] = j^c \rho\left[\mathbf{z}^{(i)}_{dq}\right] = j^c Z^{(i)}.\end{equation}

	Substituting this into \eqref{eq:complex_equiv_odes_1},

\begin{equation} \rho\left[\mathbf{K}_i\mathbf{z}^{(i)}_{dq}\right] = \sum\limits_{k=i}^{n} \left[ \sum\limits_{c=0}^{k-i} \alpha_k{k\choose i} B_{\left(k-i,c\right)}\left(\omega,\dot{\omega},\ddot{\omega},...,\omega^{(k-i-c)}\right)  j^c Z^{(i)} \right], \label{eq:complex_equiv_odes_2}\end{equation}

	\noindent and now because both $\alpha_k$ and $Z^{(i)}$ are not indexed by $c$, they can transcend the inner sum:

\begin{equation} \rho\left[\mathbf{K}_i\mathbf{z}^{(i)}_{dq}\right] = \sum\limits_{k=i}^{n} \alpha_k{k\choose i}\left[ \sum\limits_{c=0}^{k-i} j^c B_{\left(k-i,c\right)}\left(\omega,\dot{\omega},\ddot{\omega},...,\omega^{(k-i-c)}\right)\right] Z^{(i)}. \label{eq:complex_equiv_odes_3}\end{equation}

	Let

\begin{equation} \beta_n^k(t) = \sum\limits_{k=i}^{n} \alpha_k{k\choose i}\left[ \sum\limits_{c=0}^{k-i} j^c B_{\left(k-i,c\right)}\left(\omega,\dot{\omega},\ddot{\omega},...,\omega^{(k-i-c)}\right)\right] \label{eq:complex_equiv_odes_6}\end{equation}

	\noindent thus 

\begin{equation} \rho\left[\mathbf{K}_i\mathbf{z}^{(i)}_{dq}\right] = \beta_i^n(t) Z^{(i)}\end{equation}

	\noindent substituting this into \eqref{eq:complex_equiv_odes_4},

\begin{equation} \sum\limits_{i=0}^n \beta_i^n(t) Z^{(i)} - F(t) = 0 \end{equation}

	\noindent and, from the main result \eqref{eq:theo_1p_ode_solution_exp} of theorem \ref{theo:1p_ode_solution}, since $z(t) = \mathbf{P_D^{\left(-\omega\right)}}\left[Z\right]$,

\begin{equation} \left\lVert x(t) - \mathbf{P_D^{\left(-\omega\right)}}\left[Z\right] \right\rVert \leq ae^{-bt} .\end{equation} 

	Finally, consider that the initial conditions of $Z(t)$ reconstruct $x(0)$, that is,

\begin{equation} Z_0 = x_{d0} + jx_{q0},\ Z'_0 = x'_{d0} + jx'_{q0},\ ...\ ,Z^{(n-1)}_0 = x^{(n-1)}_{d0} + jx^{(n-1)}_{q0}. \end{equation}

	Then, at time $t = 0$, $\lVert x^{(k)}(t) - \mathbf{P_D^{\left(-\omega\right)}}\left[Z^{(k)}\right] \rVert = 0$. But because in a Hurwitz-stable linear system the distance from the general solution $x(t)$ and the particular solution $\mathbf{P_D^{\left(-\omega\right)}}\left[Z\right] $ can only decrease, because it is exponentially assymptotic, this yields

\begin{equation} \left\lVert x(t) - \mathbf{P_D^{\left(-\omega\right)}}\left[Z\right] \right\rVert  = 0 \end{equation} 

	\noindent for all time instants $t\geq 0$; therefore, $x(t) = \mathbf{P_D^{\left(-\omega\right)}}\left[Z\right]$ at all time instants. Because $\mathbf{P_D}$ is bijective, this yields $Z(t) = X(t)$. \hfill$\blacksquare$

\vspace{3mm}
\hrule
\vspace{3mm}

%>>>

% "COMPLEXIFIED" SYSTEM <<<
\begin{figure} 
\centering
\tikzexternaldisable
\begin{tikzpicture}[scale=1,>={Stealth[inset=0mm,length=1.5mm,angle'=50]}]

\node[stewartpink] at (0,0) (signalinput) {$f(t)$};
\node [draw, stewartpink, minimum width=35mm, very thick, minimum height=2cm, below=15mm of signalinput] (system_block) {$\displaystyle\sum_{k=0}^n \alpha_k x^{(k)} - f(t) = 0$};
\draw[->, stewartpink] (signalinput.south) -- ([shift=({0,1mm})]system_block.north);
\draw[->, stewartpink] (system_block.south) -- ([shift=({0, -15mm})]system_block.south) node[below] (signaloutput) {$x(t)$};

%------------------ DQ SYSTEM
\node [draw, stewartgreen, minimum width=35mm, very thick, minimum height=2cm, right=15mm of system_block] (system_dq_block) {$\displaystyle\sum_{i=0}^n \mathbf{K}_i \mathbf{x}^{(i)}_{dq} - \mathbf{f}_{dq} = \mathbf{0}$};

\node [draw, stewartgreen, minimum width=10mm, very thick, minimum height=10mm, above=10mm of system_dq_block.north] (tpsi_input) {$\mathbf{T}_\psi$};
\draw[->, stewartgreen] ([shift=({-5mm,0})]tpsi_input.west) node[left] {$\omega(t)$} -- ([shift=({-1mm,0})]tpsi_input.west) ;

\node [draw, stewartgreen, minimum width=10mm, very thick, minimum height=10mm, above=10mm of tpsi_input.north] (ab_input) {$\alpha\beta$};

\node [above, stewartpink, minimum width=10mm, very thick, minimum height=10mm, above=10mm of ab_input.north] (signal_dq_input) {$f(t)$};

\draw[->, stewartpink] ([shift=({0,0})]signal_dq_input.south) -- ([shift=({0,1mm})]ab_input.north);
\draw[->, stewartgreen] (ab_input.south) -- ([shift=({0,1mm})]tpsi_input.north) node[midway,right] {$\mathbf{f}_{\alpha\beta}(t)$}; ;
\draw[->, stewartgreen] (tpsi_input.south) -- ([shift=({0,1mm})]system_dq_block.north) node[midway,right] {$\mathbf{f}_{dq}(t)$};

\node [draw, stewartgreen, minimum width=10mm, very thick, minimum height=10mm, below=10mm of system_dq_block.south] (tpsi_out) {$\mathbf{T}^{-1}_\psi$};
\draw[->, stewartgreen] ([shift=({-5mm,0})]tpsi_out.west) node[left] {$\omega(t)$} -- ([shift=({-1mm,0})]tpsi_out.west) ;

\node [draw, stewartgreen, minimum width=10mm, very thick, minimum height=10mm, below=10mm of tpsi_out.south] (ab_out) {$\left(\alpha\beta\right)^{-1}$};

\node [above, stewartpink, minimum width=10mm, very thick, minimum height=10mm, below=10mm of ab_out.south] (signal_dq_output) {$x(t)$};

\draw[->, stewartgreen] ([shift=({0,0})]system_dq_block.south) -- ([shift=({0,1mm})]tpsi_out.north) node[midway,right] {$\mathbf{x}_{dq}(t)$};
\draw[->, stewartgreen] (tpsi_out.south) -- ([shift=({0,1mm})]ab_out.north) node[midway,right] {$\mathbf{x}_{\alpha\beta}(t)$};
\draw[->, stewartpink] (ab_out.south) -- ([shift=({0,1mm})]signal_dq_output.north) node[midway,right] {};

%------------------ COMPLEXIFIED SYSTEM
\node [draw, stewartblue, minimum width=35mm, very thick, minimum height=2cm, right=15mm of system_dq_block] (system_complex_block) {$\displaystyle\sum_{i=0}^n \beta_i^n (t) X^{(i)} - F = 0$};

\node [draw, stewartblue, minimum width=10mm, very thick, minimum height=10mm, above=10mm of system_complex_block.north] (rho_complex_input) {$\rho$};

\node [draw,stewartgreen, minimum width=10mm, very thick, minimum height=10mm, above=10mm of rho_complex_input.north] (tpsi_complex_input) {$\mathbf{T}_\psi$};

\node [draw,stewartgreen, minimum width=10mm, very thick, minimum height=10mm, above=10mm of tpsi_complex_input.north] (ab_complex_input) {$\alpha\beta$};

\node [above, stewartpink, minimum width=10mm, very thick, minimum height=10mm, above=10mm of ab_complex_input.north] (signal_complex_input) {$f(t)$};

\draw[->, stewartpink] ([shift=({0,0})]signal_complex_input.south) -- ([shift=({0,1mm})]ab_complex_input.north);
\draw[->, stewartgreen] (ab_complex_input.south) -- ([shift=({0,1mm})]tpsi_complex_input.north) node[midway,right] {$\mathbf{f}_{\alpha\beta}(t)$}; ;
\draw[->, stewartgreen] (tpsi_complex_input.south) -- ([shift=({0,1mm})]rho_complex_input.north) node[midway,right] {$\mathbf{f}_{dq}(t)$};
\draw[->, stewartblue] (rho_complex_input.south) -- ([shift=({0,1mm})]system_complex_block.north) node[midway,right] {$F(t)$};

\draw[->,stewartgreen] ([shift=({-5mm,0})]tpsi_complex_input.west) node[left] {$\omega(t)$} -- ([shift=({-1mm,0})]tpsi_complex_input.west) ;

\node [draw, stewartblue, minimum width=10mm, very thick, minimum height=10mm, below=10mm of system_complex_block.south] (rho_complex_output) {$\rho^{-1}$};

\node [draw, stewartgreen, minimum width=10mm, very thick, minimum height=10mm, below=10mm of rho_complex_output.south] (tpsi_complex_output) {$\mathbf{T}^{-1}_\psi$};

\node [draw, stewartgreen, minimum width=10mm, very thick, minimum height=10mm, below=10mm of tpsi_complex_output.south] (ab_complex_output) {$\left(\alpha\beta\right)^{-1}$};

\node [below, stewartpink, minimum width=10mm, very thick, minimum height=10mm, below=10mm of ab_complex_output.south] (signal_complex_output) {$x(t)$};

\draw[->, stewartgreen] ([shift=({-5mm,0})]tpsi_complex_output.west) node[left] {$\omega(t)$} -- ([shift=({-1mm,0})]tpsi_complex_output.west) ;

\draw[->, stewartblue] ([shift=({0,0})]system_complex_block.south) -- ([shift=({0,1mm})]rho_complex_output.north) node[midway,right] {$X(t)$};
\draw[->, stewartgreen] (rho_complex_output.south) -- ([shift=({0,1mm})]tpsi_complex_output.north) node[midway,right] {$\mathbf{x}_{dq}(t)$};
\draw[->, stewartgreen] (tpsi_complex_output.south) -- ([shift=({0,1mm})]ab_complex_output.north) node[midway,right] {$\mathbf{x}_{\alpha\beta}(t)$};
\draw[->, stewartpink] (ab_complex_output.south) -- ([shift=({0,1mm})]signal_complex_output.north) node[midway,right] {};

%------------------- BRACES
\path[draw,stewartblue,decorate,decoration=brace] ([shift=({8mm,0})]ab_complex_input.north east) -- ([shift=({8mm,0})]rho_complex_input.south east) node[right,midway]{\hspace{1mm} $\mathbf{P_D^{\omega}}$};
\path[draw,stewartblue,decorate,decoration=brace] ([shift=({8mm,0})] rho_complex_output.north east) -- ([shift=({8mm,0})]ab_complex_output.south east -| rho_complex_output.north east) node[right,midway]{\hspace{1mm} $\mathbf{P_D^{\left(-\omega\right)}}$};

\draw[-{Stealth[inset=0mm,length=5mm,angle'=50]},stewartpink,line width=2mm] (system_block.east) -- ([shift=({-1mm,0})]system_dq_block.west) node[midway,above,yshift=2mm] {Th. \ref{theo:1p_ode_solution}};
\draw[-{Stealth[inset=0mm,length=5mm,angle'=50]},stewartgreen,line width=2mm] (system_dq_block.east) -- ([shift=({-1mm,0})]system_complex_block.west) node[midway,above,yshift=2mm] {Th. \ref{corollary:complex_equivalence_phasorialodes}};

\draw[dashed,stewartpink] (system_block.north east) -- (ab_input.north west);
\draw[dashed,stewartpink] (system_block.south east) -- (ab_out.south west);

\draw[dashed,stewartpink] (ab_input.north east) -- (ab_complex_input.north west);
\draw[dashed,stewartpink] (ab_out.south east) -- (ab_complex_output.south west);

\draw[dashed,stewartgreen] (system_dq_block.north east) -- (rho_complex_input.north west);
\draw[dashed,stewartgreen] (system_dq_block.south east) -- (rho_complex_output.south west);

\end{tikzpicture}
\tikzexternalenable

\caption
[Schematic of a linear system being transformed into a ``dq'' version and then into a Dynamic Phasor version]
{Schematic of a linear system being transformed into a ``dq'' version and then into a Dynamic Phasor version as per theorems \ref{theo:1p_ode_solution} and \ref{corollary:complex_equivalence_phasorialodes}. In {\color{stewartpink} pink} the original time-domain system which is transformed into the dq-equivalent version by theorem \ref{theo:1p_ode_solution}; in {\color{stewartgreen} green} the ``dq apparatus'' comprised of the $\alpha\beta$ and dq transforms needed. Through theorem \ref{corollary:complex_equivalence_phasorialodes} the dq system is converted into the complex ODE by means of the complexification functional $\rho$, and this process is noted in {\color{stewartblue} blue.} The tandem operations $\alpha\beta$-dq-$\rho$ are, by definition, the Dynamic Phasor Transform $\mathbf{P_D}$, and the inverse operations comprise the inverse transform. }
\label{fig:dqification}
\end{figure}
%>>>

	Figure \ref{fig:dqification} shows a schematization of theorems \ref{theo:1p_ode_solution} and \ref{corollary:complex_equivalence_phasorialodes}. In the figure, the original time-domain system in red is translated as a dq-equivalent system; this translation is offered by theorem \ref{theo:1p_ode_solution}, such that the time signal of the forcing $f(t)$ is transported to its dq version and then processed by the system, yielding the dq version of the output, which is then reversed to time domain. Following this, theorem \ref{corollary:complex_equivalence_phasorialodes} shows that this dq system is equivalent to a complexified system by further transforming $\mathbf{f}_{dq}$ into its complexification $F(t)$, processing this signal through the complex version of the system, yielding the Dynamic Phasor quantity of the output $X(t)$, which is then de-complexified to yield $x(t)$.

%-------------------------------------------------
\subsection{Discussion on theorem \ref{corollary:complex_equivalence_phasorialodes}}\label{subsec:discussion_complexification} %<<<2

	Several topics arise from theorem \ref{corollary:complex_equivalence_phasorialodes}. We start by again discuss the initial conditions of the equivalent complex system \eqref{eq:theo_1p_ode_equivalent_complex_ode} and why it is interesting to have arbitrary initial conditions. 

	In general the signals reconstructed from the phasorial equivalent ODE are not representative of the original signal $x(t)$ at initial time. This is true even for classical phasors: a revisitation of theorem \ref{theo:phasors_solutions} shows that the sinusoidal solution \eqref{eq:linear_ode_phasor_solution_1} is exponentially stable because the sinusoidal solution does not necessatily reconstruct the solution $x(t)$ of the original linear system \eqref{eq:linear_ode_phasor_solution_1}. If the sinusoidal solution and the original solution have the same initial conditions, then they are one and the same.

	In this regard, it is useful to consider that the phasorial system does not start at the same initial conditions than the original time-domain system. In Power Systems this is especially useful because, in general, the initial conditions of the electrical grid are calculated by Power Flow algorithms that calculates these initial conditions from active and reactive power balances in the grid; the initial conditions of the differential equations of the agents are calculated ``backwards'', that is, the initial conditions of the phasorial equations are do not reconstruct the initial conditions of the time-domain equations because the phasorial ones are set so as to comply with the Power Flow algorithms. If, however, the initial conditions of the phasorial system are the same than that of the original system, then the phasorial system reconstructs the original solution without any approximations or losses in time.

	Another question raised by theorem \ref{corollary:complex_equivalence_phasorialodes} is if this theorem generalizes classical phasors; at a first glance, if $\omega(t) = \omega_0$ constant then the theorem should fall back into its static counterpart. Indeed, if $\omega_0$ is constant and the forcing $F(t)$ is a constant phasor at $\omega_0$, then the complexified version \eqref{eq:theo_1p_ode_equivalent_complex_ode} becomes

\begin{equation} \sum\limits_{i=0}^n \beta_i^n(t) Z^{(i)} - F = 0,\ Z(t) = z_d(t) + jz_q(t), \beta_i^n = \sum\limits_{k=i}^{n} \alpha_k{k\choose i} \left[\sum\limits_{c=0}^{k-i} j^cB_{\left(k-i,c\right)}\left(\omega_0,0,0,\cdots,0\right) \right]  \end{equation}

	\noindent and immediately one notices that the $\beta_i^n$ are constant and not time-variant anymore. By the properties of the Bell Polynomials,

\begin{equation} B_{\left(k-i,c\right)}\left(\omega_0,0,...0\right) = \left\{\begin{array}{l} \omega_0^{k} \text{, if } k-i=c \\[2mm] 0 \text{, if otherwise} \end{array}\right. \end{equation}

	\noindent and the $\beta_i^n$ become

\begin{equation} \beta_i^n = \sum\limits_{k=i}^{n} \alpha_k{k\choose i} \left(j\omega_0\right)^{(k-i)} .\end{equation}

	Particularly,

\begin{equation} \beta_0^n = \sum\limits_{k=0}^{n} \alpha_k {k\choose 0} \left(j\omega_0\right)^{k} = \sum\limits_{k=0}^{n} \alpha_k \left(j\omega_0\right)^{k} ,\end{equation}

	\noindent and the model becomes

\begin{equation} \overbrace{\sum\limits_{i=1}^n \beta_i^n Z^{(i)}}^{\text{Transient sum}} + \overbrace{\beta_0^n Z - F}^{\text{Static behavior}} = 0 .\end{equation}

	Quickly one notices that the portion $\beta_0^n Z - F$ is the ``static'' equation that would be obtained by static phasors theory, while the summation on the left is a transient behavior pertaining to initial conditions. This transient term certainly fades exponentially over time so that the equation

\begin{equation} \beta_0^n Z_\infty - F = 0 \Leftrightarrow \lim\limits_{t\to\infty} \left\lvert Z(t) - Z_\infty\right\rvert = 0 \text{ (exp.)},\end{equation}

	\noindent where ``(exp.)'' means exponential tendency, describes the assymptotic behavior of $Z$ — that is, $Z$ tends to a constant static phasor $Z_\infty$ which naturally reconstructs a static sinusoid. Particularly, if we assume $Z$ is a static phasor, then the transient sum is identically null and $Z(t) = Z_\infty$. Thus, theorem \ref{corollary:complex_equivalence_phasorialodes} is a generalization of the classical phasors theorem \ref{theo:phasors_solutions_reproof}.

	These results also beg the question that if $\omega(t)$ is not exactly constant but ``almost constant'', then the static behavior still approximates the steady-state solution of the phasorial equivalent system. The answer is yes: chapter \ref{chapter:choice_apparent_frequency} proves in section \ref{sec:qsh_proof} that if the circuit is much quicker than $\omega_0$, then the behavior of the phasorial equivalent model is sufficiently approximated by the static approximation, that is,

\begin{equation} Z_a = \dfrac{F}{\beta_0^n} = \dfrac{F}{\sum\limits_{k=0}^{n} \alpha_k \left(j\omega_0\right)^{k}} \end{equation}

	\noindent sufficiently approximates $Z(t)$ with a precision that gets better as the circuit gets ``quicker'' and/or the frequency $\omega(t)$ is ``slower''. In this context, a ``slow'' frequency means that it is close to a constant $\omega_0$ that is sufficiently small, and a ``fast'' circuit means that the Hurwitz Polynomial of the original time-domain circuit

\begin{equation} H(x) = \sum\limits_{k=0}^n \alpha_k x^k \end{equation}

	\noindent is such that its roots have negative yet large real parts.	

\begin{example}[Application of theorem \ref{corollary:complex_equivalence_phasorialodes}] \label{example:rlc_dpt} %<<<

	Consider the RLC circuit of figure \ref{fig:complexification_example}, comprised of a RLC circuit fed by a voltage $v(t)$. Suppose that the circuit is excited by a nonstationary voltage

% MODELLING EXAMPLE: RLC CIRCUIT <<<
\begin{figure}[htb!]
\centering
        \begin{tikzpicture}[american,scale=1,transform shape,line width=0.75, cute inductors, voltage shift = 1]
	\ctikzset{/tikz/circuitikz/voltage/distance from node=10mm}
		\draw (0,0)
			to[vsource,sources/scale=1.25, v>=$v(t)$,invert] (0,4)
			to[L,l=$L$,f>^=$i_{L}$,v>=$v_{L}$,-*] (4,4) 
			to[C,l=$C$,f>^=$i_{C}$,v>=$v_{C}$,-*] (4,0) 
			to[short] (0,0); 
		\draw (4,4)
			to[short,f>^=$i_{R}$] (7,4) 
			to[R,l=$R$,v>=$v_{R}$] (7,0) 
			to[short]  (4,0);
        \end{tikzpicture}
	\caption{Second-order circuit for example application of theorem \ref{corollary:complex_equivalence_phasorialodes}.}
	\label{fig:complexification_example}
\end{figure} %>>>

\begin{equation} v(t) = m_v(t)\cos\left(\psi(t)\right) \text{, with } \psi = \int_0^a \omega(a)da \text{, where } \omega(t) = \omega_0\left[1 + Me^{-\alpha t}\sin\left(\beta t\right)\right], \label{eq:example_voltage_freq_def}\end{equation}

 	\noindent modelling a base frequency $\omega_0$ that is transiently disturbed and stabilizes after some time. The frequency $\omega(t)$ of \eqref{eq:example_voltage_freq_def} yields an angle displacement

\begin{equation} \psi(t) = \omega_0\left(t + \dfrac{M\left\{\beta - e^{-\alpha t}\left[\alpha\sin\left(\beta t\right) + \beta\cos\left(\beta t\right)\right]\right\}}{\alpha^2 + \beta^2} \right) .\end{equation}

	This frequency signal was specifically chosen because its Fourier Series is given in terms of Bessel Functions of the first kind; more precisely,

\begin{equation} \mathbf{F}\left[e^{ja\sin\left(\omega_0 t\right)}\right] = \sum_{n\in\mathbb{Z}} J_n\left(a\right)e^{jn\omega_0 t} ,\end{equation}

	\noindent where $J_n$ represents the Bessel Function of first kind, n-th order. This means that constructing a Dynamic Phasor representation for $v(t)$ using either both Hilbert Transform and STFT techniques would be inexorably difficult; due to the inherent complex nature of Bessel Functions and the infinite terms, operationalizing this specific signal using these techniques would need some sort of approximation. This example shows that the Dynamic Phasor technique proposed in this thesis can easily deal with signals complicated as these without need for such approximations.

	The numerical values adopted are $\omega_0=120\pi$ rad/s, $R = 100\Omega,C = 1mF,L = 4mH, \alpha=5s^{-1},\beta=10\pi\ \text{rad}.s^{-1},M=0.1 $. The objective is to find the time signal $v_R(t)$ of the voltage over the resistive load $R$. First, apply Kirchoff's Voltage Law to the left loop and Kirchoff's Current Law to the center top node

\begin{equation} \left\{\begin{array}{l} -v(t) + v_L(t) + v_C(t) = 0 \\[3mm] v_C(t) - v_R(t) = 0 \\[3mm] i_L(t) - i_C(t) - i_R(t) = 0 \end{array}\right. .\end{equation}

	Using the current-voltage relationships of the components,

\begin{equation} \left\{\begin{array}{l} -v(t) + L\dot{i}_L(t) + v_C(t) = 0 \\[3mm] v_C(t) - v_R(t) = 0 \\[2mm] i_L(t) - C\dot{v}_C(t) - \dfrac{1}{R}v_R(t) = 0 \end{array}\right. .\end{equation}

	Substituting the second and third equations into the first,

\begin{equation} - \dfrac{1}{LC} v(t) + \ddot{v}_R(t) + \dfrac{1}{RC}\dot{v}_R(t) + \dfrac{1}{LC} v_R(t) = 0 . \label{eq:rlc_time_diffeq}\end{equation}

	Now we apply theorem \ref{corollary:complex_equivalence_phasorialodes}. Adopt the frequency signal of \eqref{eq:example_voltage_freq_def} and \eqref{eq:rlc_time_diffeq} is equivalent to

\begin{equation} \ddot{V}_R(t) + \dot{V}_R(t)\left(\dfrac{1}{RC} + 2j\omega(t)\right) + V_R\left\{ \dfrac{1}{LC}  -\omega^2(t) + j \left[ \dot{\omega}(t) + \dfrac{1}{RC}\omega(t)\right]\right\} -\dfrac{1}{LC} V(t) = 0, \label{eq:rlc_complex_diffeq}\end{equation}

	\noindent where $V(t) = \mathbf{P_D^{\omega}}\left[v\right]$, which is naturally $m(t)e^{j0}$. This differential equation was integrated in time and the resulting complex signal $V_R(t)$ is shown in Figure \ref{fig:amp_phase_voltage_signals}. To show the capability of the DPT to reconstruct time signals, Figure \ref{fig:voltage_signals} shows in blue the time signal obtained by integrating the original time ODE \eqref{eq:rlc_time_diffeq}, and in red the signal reconstructed from the Dynamic Phasor $V_R(t)$ obtained by integrating the complex equation \eqref{eq:rlc_complex_diffeq}, such that both time signal and complex signal have the same initial conditions. It is immediate to see that both signals are identical, highlighting that the DPT from $V_R(t)$ reconstructs the time signal $v_R(t)$ without losses..

	On the other hand, Figure \ref{fig:voltage_signals_perturbed} shows the same time signal from the original ODE in blue; in red, the signal reconstructed from the complex differential equation \eqref{eq:rlc_complex_diffeq}. In this case, however, the initial conditions are perturbed and do not match, as shown by the zoomed-in version of the simulation start period. The figure shows a zoomed-in version of the simulation final period, allowing to observer that, as per the theorem statement, the signal reconstructed from $V_R(t)$ indeed approaches $v(t)$ as time grows, even though the initial conditions are not the same. The

% AMPLITUDE PHASE TIME CURVES <<<
\begin{figure}
        \begin{center}
                \begin{tikzpicture}
                        \begin{axis}[
                                width = \columnwidth,
                                height = 1/1.618*\columnwidth,
                                title={Amplitude and phase signals from time and DP simulations with matching initial conditions},
                                xlabel={Time (s)},
                                %ylabel={$\left\lvert V_R(t)\right\rvert$ (V)},
				y axis line style = {red, thick},
				every y tick label/.append style ={red},
				every y tick/.append style ={thick, red},
                                xmin=0, xmax=1,
                                ymin=0, ymax=42,
                                xtick={0,0.1,...,1},
				ylabel style = {align=center},
				axis y line*=left,
                                every axis plot/.append style={thick},
				legend pos = north east
                        ]
                                \addplot[red,smooth] table[col sep=comma,header=false,x index=0,y index=3]{data/rlc_sim/data_rlc_sim_dps.csv};
				\addlegendentry{$\left\lvert V_R(t)\right\rvert$ (V)}
                        \end{axis}
%
                        \begin{axis}[
                                width = \columnwidth,
                                height = 1/1.618*\columnwidth,
                                xmin=0, xmax=1,
		        	axis y line*=right,
                               %ylabel={$\text{arg}\left(V_R(t)\right)$ ($\times \pi$ rad)},
				y axis line style = {blue, thick},
				every y tick label/.append style ={blue},
				every y tick/.append style ={thick, blue},
				ylabel near ticks,
                                ymin=-0.25, ymax=0.25,
                                xtick={0,0.1,...,1},
                                ytick={-0.2,-0.1,...,0.2},
				axis x line=none,
		        	ylabel style = {align=center},
                                every axis plot/.append style={thick},
				legend pos = south east
                        ]
                                \addplot[blue ,smooth] table[col sep=comma,header=false,x index=0,y expr = \thisrowno{4}/3.141596]{data/rlc_sim/data_rlc_sim_dps.csv};
				\addlegendentry{$\text{arg}\left(V_R(t)\right)$ ($\times\ \pi$ rad)}
                        \end{axis}
                \end{tikzpicture}
        \caption
	[Amplitude and phase signals of the Dynamic Phasor obtained by integrating the complex differential equation.]
	{Amplitude (red) and phase (blue) signals of the Dynamic Phasor $V_R(t)$ obtained by integrating the complex differential equation \eqref{eq:rlc_complex_diffeq}.}
        \label{fig:amp_phase_voltage_signals}
        \end{center}
\end{figure}
% >>>

% VOLTAGE TIME CURVES <<<
\begin{figure}
        \begin{center}
                \begin{tikzpicture}
                        \begin{axis}[
				name = ax_main,
                                width = 0.9*\columnwidth,
                                height = 0.9*1/1.618*\columnwidth,
                                title={Voltage signals from time and DP simulations with exact initial conditions},
                                xlabel={Time (s)},
                                ylabel={$v_R(t)$ and $\mathbf{P_D^{\left(-\omega\right)}}\left[V_R\right]$ (V)},
                                xmin=0, xmax=1,
                                ymin=-42, ymax=42,
                                xtick={0,0.1,...,1},
                                ytick={-40,-30,...,40}, 
                                legend pos=south east,
                                ymajorgrids=true,
                                xmajorgrids=true,
                                %grid style=dashed,
                                colormap name=hsv2,
                                cycle list={[ colors of colormap={100,200,300,400,500,600,700,800,900,1000} ]},
                                every axis plot/.append style={thick},
                        ]
                                \addplot[blue ,smooth] table[col sep=comma,header=false,x index=0,y index=1]{data/rlc_sim/data_rlc_sim_dps.csv};
                        \coordinate (c1) at (axis cs:0,-42);
                        \coordinate (c2) at (axis cs:0.1,-42);
                        \end{axis}
%
                        \begin{axis}[
                                name = ax_zoomed,
                                at={($(ax_main.north east)-(0.9\columnwidth,1.75/1.618*\columnwidth)$)},
                                width = 0.9*1\columnwidth,
                                height = 0.9*1/1.618*\columnwidth,
                                xmin=0, xmax=0.1,
                                ymin=-42, ymax=42,
                                xtick={0,0.01,...,0.1},
				xlabel={Time (ms)},
				xticklabels={$0$,$10$,$20$,$30$,$40$,$50$,$60$,$70$,$80$,$90$,$100$},
                                ytick={-40,-30,...,40},
				tick label style={/pgf/number format/fixed},
				legend columns=2,
				legend style={/tikz/every even column/.append style={column sep=0.5cm}},
                                ymajorgrids=true,
                                xmajorgrids=true,
                                %grid style=dashed,
                                every axis plot/.append style={thick},
                                axis background/.style = {
                                        preaction = {
                                        path picture = {
                                        \draw[fill=white,line width=0mm] (axis cs:0,400) rectangle (axis cs:0.1,-40);
                                                }
                                        }
                                }
                        ]
				\addplot[blue, smooth]         table[col sep=comma,header=false,x index=0,y index=1]{data/rlc_sim/data_rlc_sim_dps.csv};
				\addlegendentry{$v_R(t)$}
				\addplot[red,  smooth, dashed, dash pattern=on 2pt off 4pt, line cap=round] table[col sep=comma,header=false,x index=0,y index=2]{data/rlc_sim/data_rlc_sim_dps.csv};
				\addlegendentry{$\mathbf{P_D^{\left(-\omega\right)}}\left[V_R\right]$}
                        \end{axis}
                        % draw dashed lines from rectangle in first axis to corners of second
                        \draw [gray,dashed] (c1) -- (ax_zoomed.north west);
                        \draw [gray,dashed] (c2) -- (ax_zoomed.north east);
                \end{tikzpicture}
        \caption
[Voltage across the resistor of the circuit of Figure \ref{fig:complexification_example} using exact initial conditions.]
{Voltage across the resistor of the circuit of Figure \ref{fig:complexification_example} using exact initial conditions. In blue, the signal $v_R(t)$ obtained by integrating the time differential equation \eqref{eq:rlc_time_diffeq}. In red, the signal obtained by the inverse transform of the Dynamic Phasor $V_R(t)$ of the solution of the complex differential equation \eqref{eq:rlc_complex_diffeq}. Top plot shows only the blue line; bottom plot shows a zoomed-in version with both lines juxtaposed for comparison.}
        \label{fig:voltage_signals}
        \end{center}
\end{figure}
% >>>

% VOLTAGE TIME CURVES (PERTURBED) <<<
\begin{figure}
        \begin{center}
                \begin{tikzpicture}
                        \begin{axis}[
				name = ax_main,
                                width = 0.9*\columnwidth,
                                height = 0.9*1/1.618*\columnwidth,
                                title={Voltage signals from time and DP simulations with perturbed initial conditions},
                                xlabel={Time (s)},
                                ylabel={$v_R(t)$ and $\mathbf{P_D^{\left(-\omega\right)}}\left[V_R\right]$ (V)},
                                xmin=0, xmax=1,
                                ymin=-42, ymax=42,
                                xtick={0,0.1,...,1},
                                ytick={-40,-30,...,40}, 
                                legend pos=south east,
                                ymajorgrids=true,
                                xmajorgrids=true,
                                %grid style=dashed,
                                colormap name=hsv2,
                                cycle list={[ colors of colormap={100,200,300,400,500,600,700,800,900,1000} ]},
                                every axis plot/.append style={thick},
				legend columns=2,
                        ]
				\addplot[blue, smooth]         table[col sep=comma,header=false,x index=0,y index=1]{data/rlc_sim/data_rlc_sim_perturbed.csv};
				\addlegendentry{$v_R(t)$}
				\addplot[red,  smooth, dashed, dash pattern=on 4pt off 2pt, line cap=round] table[col sep=comma,header=false,x index=0,y index=2]{data/rlc_sim/data_rlc_sim_perturbed.csv};
				\addlegendentry{$\mathbf{P_D^{\left(-\omega\right)}}\left[V_R\right]$}
                        \coordinate (c1) at (axis cs:0,-42);
                        \coordinate (c2) at (axis cs:0.1,-42);
%
                        \coordinate (c3) at (axis cs:0.9,-42);
                        \coordinate (c4) at (axis cs:1.0,-42);
                        \end{axis}
%
                        \begin{axis}[
                                name = ax_zoomed_start,
                                at={($(ax_main.north east)-(0.9\columnwidth,1.3/1.618*\columnwidth)$)},
                                width = 0.6*1\columnwidth,
                                height = 0.6*1/1.618*\columnwidth,
                                xmin=0, xmax=0.1,
                                ymin=-42, ymax=42,
                                xtick={0,0.01,...,0.1},
				xlabel={Time (ms)},
				xticklabels={$0$,$10$,$20$,$30$,$40$,$50$,$60$,$70$,$80$,$90$,$100$},
                                ytick={-40,-30,...,40},
				tick label style={/pgf/number format/fixed},
				legend columns=2,
				legend style={/tikz/every even column/.append style={column sep=0.5cm}},
                                ymajorgrids=true,
                                xmajorgrids=true,
                                %grid style=dashed,
                                every axis plot/.append style={thick},
                                axis background/.style = {
                                        preaction = {
                                        path picture = {
                                        \draw[fill=white,line width=0mm] (axis cs:0,400) rectangle (axis cs:0.1,-40);
                                                }
                                        }
                                }
                        ]
				\addplot[blue, smooth]         table[col sep=comma,header=false,x index=0,y index=1]{data/rlc_sim/data_rlc_sim_perturbed.csv};
				\addplot[red,  smooth, dashed, dash pattern=on 4pt off 2pt, line cap=round] table[col sep=comma,header=false,x index=0,y index=2]{data/rlc_sim/data_rlc_sim_perturbed.csv};
                        \end{axis}
                        % draw dashed lines from rectangle in first axis to corners of second
                        \draw [gray,dashed] (c1) -- (ax_zoomed_start.north west);
                        \draw [gray,dashed] (c2) -- (ax_zoomed_start.north east);
%
                        \begin{axis}[
                                name = ax_zoomed_final,
                                at={($(ax_main.north east)-(0.4\columnwidth,1.9/1.618*\columnwidth)$)},
                                width = 0.6*1\columnwidth,
                                height = 0.6*1/1.618*\columnwidth,
                                xmin=0.9, xmax=1,
                                ymin=-42, ymax=42,
                                xtick={0.90,0.91,...,1},
				xlabel={Time (ms)},
				xticklabels={$900$,$910$,$920$,$930$,$940$,$950$,$960$,$970$,$980$,$990$,$1000$},
                                ytick={-40,-30,...,40},
				tick label style={/pgf/number format/fixed},
				legend columns=2,
				legend style={/tikz/every even column/.append style={column sep=0.5cm}},
                                ymajorgrids=true,
                                xmajorgrids=true,
                                %grid style=dashed,
                                every axis plot/.append style={thick},
                                axis background/.style = {
                                        preaction = {
                                        path picture = {
                                        \draw[fill=white,line width=0mm] (axis cs:0,400) rectangle (axis cs:0.1,-40);
                                                }
                                        }
                                }
                        ]
				\addplot[blue, smooth]         table[col sep=comma,header=false,x index=0,y index=1]{data/rlc_sim/data_rlc_sim_perturbed.csv};
				\addplot[red,  smooth, dashed, dash pattern=on 4pt off 2pt, line cap=round] table[col sep=comma,header=false,x index=0,y index=2]{data/rlc_sim/data_rlc_sim_perturbed.csv};
                        \end{axis}
                        % draw dashed lines from rectangle in first axis to corners of second
                        \draw [gray,dashed] (c3) -- (ax_zoomed_final.north west);
                        \draw [gray,dashed] (c4) -- (ax_zoomed_final.north east);
                \end{tikzpicture}
        \caption
[Voltage across the resistor of the circuit of Figure \ref{fig:complexification_example} using perturbed initial conditions.]
{Voltage across the resistor of the circuit of Figure \ref{fig:complexification_example} using perturbed initial conditions. In blue, the signal $v_R(t)$ obtained by integrating the time differential equation \eqref{eq:rlc_time_diffeq}. In red, the signal obtained by the inverse transform of the Dynamic Phasor $V_R(t)$ of the solution of the complex differential equation \eqref{eq:rlc_complex_diffeq} which initial conditions are different that those of the time-domain equation.}
        \label{fig:voltage_signals_perturbed}
        \end{center}
\end{figure}
% >>>

	As a contrast, we can model the system using the Short-Time Fourier Transform. Using the differential property \eqref{sys:fdp_sys_fundamental_harmonic} onto the differential equation \eqref{eq:rlc_complex_diffeq_stft}, where the ``F'' subscript stands for ``Fourier'', highlighting that this equation uses the STFT. In this equation, as discussed in subsection \ref{subsec:infinite_complex_systems}, the capital letters denote the first harmonic of the signals they represent. Integrating this equation, and then using the inversion formula \eqref{eq:fourierSeries} yields a time signal that is plotted on figure \ref{fig:voltage_signals_stft} against the signal $\mathbf{P_D^{\left(-\omega\right)}}\left[V_R\right]$ reconstructed from the proposed DPT.

\begin{equation} \ddot{V}_{RF}(t) + \dot{V}_{RF}(t)\left(\dfrac{1}{RC} + 2j\omega(t)\right) + V_{RF}\left\{ \dfrac{1}{LC}  -\omega^2(t) + j \left[ \dot{\omega}(t) + \dfrac{1}{RC}\omega(t)\right]\right\} -\dfrac{1}{LC} V_F(t) = 0, \label{eq:rlc_complex_diffeq_stft}\end{equation}

% VOLTAGE TIME CURVES VS STFT <<<
\begin{figure}
        \begin{center}
                \begin{tikzpicture}
                        \begin{axis}[
				name = ax_main,
                                width = 0.9*\columnwidth,
                                height = 0.9*1/1.618*\columnwidth,
                                title={Voltage signals reconstructed from STFT and from $\mathbf{P_D}$},
                                xlabel={Time (s)},
                                ylabel={$\mathbf{P_D^{\left(-\omega\right)}}\left[V_R\right]$, $\mathbf{P_D^{\left(-\omega\right)}}\left[V_{RF}\right]$},
                                xmin=0, xmax=1,
                                ymin=-42, ymax=42,
                                xtick={0,0.1,...,1},
                                ytick={-40,-30,...,40}, 
                                legend pos=south east,
                                ymajorgrids=true,
                                xmajorgrids=true,
                                %grid style=dashed,
                                colormap name=hsv2,
                                cycle list={[ colors of colormap={100,200,300,400,500,600,700,800,900,1000} ]},
                                every axis plot/.append style={thick},
				legend columns=2,
                        ]
				\addplot[blue, smooth] table[col sep=comma,header=false,x index=0,y index=1]{data/rlc_sim/data_stft.csv};
				\addlegendentry{$\mathbf{P_D^{\left(-\omega\right)}}\left[V_R\right]$}
				\addplot[red, smooth, dashed, dash pattern=on 1pt off 1pt, line cap=round] table[col sep=comma,header=false,x index=0,y index=2]{data/rlc_sim/data_stft.csv};
				\addlegendentry{$\mathbf{STFT^{-1}}\left[V_{RF}\right]$}
                        \coordinate (c1) at (axis cs:0,-42);
                        \coordinate (c2) at (axis cs:0.1,-42);
%
                        \coordinate (c3) at (axis cs:0.9,-42);
                        \coordinate (c4) at (axis cs:1.0,-42);
                        \end{axis}
%
                        \begin{axis}[
                                name = ax_zoomed_start,
                                at={($(ax_main.north east)-(0.9\columnwidth,1.3/1.618*\columnwidth)$)},
                                width = 0.6*1\columnwidth,
                                height = 0.6*1/1.618*\columnwidth,
                                xmin=0, xmax=0.1,
                                ymin=-42, ymax=42,
                                xtick={0,0.01,...,0.1},
				xlabel={Time (ms)},
				xticklabels={$0$,$10$,$20$,$30$,$40$,$50$,$60$,$70$,$80$,$90$,$100$},
                                ytick={-40,-30,...,40},
				tick label style={/pgf/number format/fixed},
				legend columns=2,
				legend style={/tikz/every even column/.append style={column sep=0.5cm}},
                                ymajorgrids=true,
                                xmajorgrids=true,
                                %grid style=dashed,
                                every axis plot/.append style={thick},
                                axis background/.style = {
                                        preaction = {
                                        path picture = {
                                        \draw[fill=white,line width=0mm] (axis cs:0,400) rectangle (axis cs:0.1,-40);
                                                }
                                        }
                                }
                        ]
				\addplot[blue, smooth] table[col sep=comma,header=false,x index=0,y index=1]{data/rlc_sim/data_stft.csv};
				\addplot[red, smooth] table[col sep=comma,header=false,x index=0,y index=2]{data/rlc_sim/data_stft.csv};
                        \end{axis}
                        % draw dashed lines from rectangle in first axis to corners of second
                        \draw [gray,dashed] (c1) -- (ax_zoomed_start.north west);
                        \draw [gray,dashed] (c2) -- (ax_zoomed_start.north east);
%
                        \begin{axis}[
                                name = ax_zoomed_final,
                                at={($(ax_main.north east)-(0.4\columnwidth,1.9/1.618*\columnwidth)$)},
                                width = 0.6*1\columnwidth,
                                height = 0.6*1/1.618*\columnwidth,
                                xmin=0.9, xmax=1,
                                ymin=-42, ymax=42,
                                xtick={0.90,0.91,...,1},
				xlabel={Time (ms)},
				xticklabels={$900$,$910$,$920$,$930$,$940$,$950$,$960$,$970$,$980$,$990$,$1000$},
                                ytick={-40,-30,...,40},
				tick label style={/pgf/number format/fixed},
				legend columns=2,
				legend style={/tikz/every even column/.append style={column sep=0.5cm}},
                                ymajorgrids=true,
                                xmajorgrids=true,
                                %grid style=dashed,
                                every axis plot/.append style={thick},
                                axis background/.style = {
                                        preaction = {
                                        path picture = {
                                        \draw[fill=white,line width=0mm] (axis cs:0,400) rectangle (axis cs:0.1,-40);
                                                }
                                        }
                                }
                        ]
				\addplot[blue, smooth] table[col sep=comma,header=false,x index=0,y index=1]{data/rlc_sim/data_stft.csv};
				\addplot[red, smooth] table[col sep=comma,header=false,x index=0,y index=2]{data/rlc_sim/data_stft.csv};
                        \end{axis}
                        % draw dashed lines from rectangle in first axis to corners of second
                        \draw [gray,dashed] (c3) -- (ax_zoomed_final.north west);
                        \draw [gray,dashed] (c4) -- (ax_zoomed_final.north east);
                \end{tikzpicture}
        \caption
[Comparison of signals as reconstructed by the Dynamic Phasor Transform and as reconstructed by the STFT]
{Comparison of the voltage signals as reconstructed by the proposed Dynamic Phasor Transform (in blue) and as reconstructed by the Short-Time Fourier Transform (in red) by integrating \ref{eq:rlc_complex_diffeq_stft} and using the inversion formula \eqref{eq:fourierSeries}.}
        \label{fig:voltage_signals_stft}
        \end{center}
\end{figure}
% >>>

	Figure \ref{fig:voltage_signals_stft} then illustrates how the STFT is unable to deal with this system and the signals involved. Not only it clearly does not reconstruct the signal in the first time instants, as shown by the zoomed subplot, but it also fails to reconstruct the signal even at steady-state, showing a considerable angle difference.

\examplebar
\end{example} %>>>

%-------------------------------------------------
\section{Nonstationary Complex Power} %<<<1

	Finally, we now want to show that the proposed Dynamic Phasor Transform is able to beget the notion of complex power under nonstationary regimens. The idea is to show that the induced notion of complex power is nigh-identical to that found in static phasors, as per theorem \ref{theo:sfp_complex_apparent_power}.

\begin{theorem}[Generalized Complex Power]\label{theo:activereactivepower} %<<<
	Let $V = m_v(t)e^{j\phi_v(t)}$ and $I = m_i(t)e^{j\phi_i(t)}$ represent the dynamical phasors of the voltage across and current through a bipole and consider the quantity 

\begin{equation} S(t) = \frac{1}{2}\left<V(t),I(t)\right> = P(t) + jQ(t)\ \left\{\begin{array}{l} P(t) = \dfrac{m_v(t)m_i(t)}{2}\cos\left[\phi_v(t) - \phi_i(t)\right] \\[3mm] Q(t) = \dfrac{m_v(t)m_i(t)}{2}\sin\left[\phi_v(t) - \phi_i(t)\right] \end{array}\right. \label{eq:dynamical_complex_power_def}\end{equation}

	\noindent called \textbf{complex power}. Then $S(t)$ is such that the instantaneous power performed by the bipole can be calculated as

\begin{equation} p(t) = P(t) \left[1 + \cos\left(2\psi + 2\phi_v\right) \right] + Q(t) \sin\left(2\psi + 2\phi_v\right). \label{eq:dynamical_complex_instpower}\end{equation}

\end{theorem}
\textbf{Proof:} for \eqref{eq:dynamical_complex_instpower} and \eqref{eq:dynamical_complex_power_def}, write $p(t) = v(t)i(t)$:

\begin{equation} p(t) = m_v\cos\left(\psi(t) + \phi_v(t)\right)m_i\cos\left(\psi(t) + \phi_i(t)\right) \end{equation} 

	\noindent and denote $\Delta\phi(t) = \phi_v(t) - \phi_i(t)$. Then $\phi_v(t) + \phi_i(t) = 2\phi_v(t) - \Delta\phi(t)$; therefore using

\begin{equation} \cos(a)\cos(b) = \dfrac{1}{2}\left[\cos(a+b) + \cos(a-b)\right],\end{equation}

	\noindent one obtains

\begin{align} p(t) &= m_i(t) m_v(t) \dfrac{1}{2} \left[ \cos\left(2\psi(t) + \phi_v(t) + \phi_i(t)\right) + \cos\left(\phi_v(t) - \phi_i(t)\right)\right] \nonumber\\[3mm] &= \dfrac{m_i(t) m_v(t)}{2} \left\{\cos\left[2\left(\psi(t) + \phi_v(t)\right) - \Delta\phi(t)\right] + \cos\left[\Delta\phi(t)\right]\right\} \end{align}

	Using $\cos(a-b) = \cos(a)\cos(b) + \sin(a)\sin(b)$,

\begin{equation} p(t) = \dfrac{m_i(t) m_v(t)}{2} \left\{\raisebox{5mm}{} \cos\left(\Delta\phi(t)\right)\left\{\raisebox{3mm}{} 1 + \cos\left[2\left(\psi(t) + \phi_v(t)\right)\right]\right\} + \sin\left(\Delta\phi(t)\right)\sin\left[2\left(\psi(t) + \phi_v(t)\right)\right] \right\} . \label{eq:nonst_complex_apparent_power_eq1} \end{equation}

	Let

\begin{equation} P = \dfrac{m_i m_v}{2} \cos\left(\Delta\phi(t)\right),\ Q = \dfrac{m_i m_v}{2} \sin\left(\Delta\phi(t)\right) \end{equation}

	\noindent then

\begin{equation} p(t) = P\left\{1 + \cos\left[2\left(\psi(t) + \phi_v(t)\right) \right]\right\} + Q\sin\left[2\left(\psi(t) + \phi_v(t) \right)\right] . \label{eq:nononst_complex_apparent_power_eq2} \end{equation}
\hfill$\blacksquare$
\vspace{5mm}
\hrule
\vspace{5mm} %>>>

	Having now proven that the notions of active and reactive power induced by the Dynamic Phasor Transform are the same as the ones of static phasors, we must now prove that these new notions have the same physical meaning. We first prove that, also identically to static active power, this generalized notion of active power is such that $P(t)$ is the average power over some interval.

\begin{theorem}[Active Power as average power in some period]\label{theo:activepowerperiod} %<<<
	Let $V = m_v(t)e^{j\phi_v(t)}$ and $I = m_i(t)e^{j\phi_i(t)}$ represent the dynamical phasors of the voltage across and current through a bipole. Consider the equation on $T(t)$:

\begin{equation} \dfrac{1}{T(t)}\int_{t}^{t+T(t)} p(s)ds = P(t) \label{eq:theo_cp_power_avg_power_period_eq} \end{equation}

	Let $\left[t_0,t_f\right)$ be such that $m_v,m_i,\phi_v,\phi_i$ are defined, bounded with bounded derivatives, and there exists a positive solution $T_0 = T\left(t_0\right)$ for \eqref{eq:theo_cp_power_avg_power_period_eq}.	Then there exists a unique function $T(t)$ defined in $\left[t_0,t_f\right)$ that satisfies \eqref{eq:theo_cp_power_avg_power_period_eq} and $T\left(t_0\right) = T_0$, meaning there exists a $T(t)$ such that $P(t)$ is the average value of the instantaneous power $p(t)$ over $\left[t,t+T\left(t\right)\right)$.
\end{theorem}
\textbf{Proof:} start with the average power equation \eqref{eq:theo_cp_power_avg_power_period_eq}; differentiating it with respect to time and using Leibniz's Integral rule yields

\begin{equation}
\dot{T}\left(t\right)\left[p\left(t+T\left(t\right)\right) - P\left(t\right)\right] = T\left(t\right)\dot{P}\left(t\right) + p\left(t\right)- p\left(t+T\left(t\right)\right) \label{eq:period_dot_1}
\end{equation}

	Now let us analyze the term of \eqref{eq:period_dot_1} in brackets that multiplies $\dot{T}$; call this term $u\left(t,T(t)\right)$. Clearly, $u$ can only be zero for isolated and distinct values of $t$, not being null in a continuum of values unless $m_v,m_i,\phi_v,\phi_i,\psi$ have very particular qualities — making reasonable the genericity argument that this will most likely not happen for arbitrary signals. Then, let $t_1 < t_2 < ... < t_k$ be the roots of $u$, $t_0 < t_1$, $t_k \leq t_f$, and denote $I_j = \left(t_j,t_{j+1}\right)$; in each interval $I_j$, \eqref{eq:period_dot_1} can be written as $\dot{T}(t) = f\left(t,T(t)\right)$, thus if $f\in C^1$ for each $I_j$ and an initial solution $T(t_j)$ can be found for each $t_j$, then by the Picard-Lindelöf Existence and Uniqueness Theorem \cite[p. 188]{Perko2001} a unique solution $T_j$ exists for each $I_j$. Continuous differentiability of $f$ can be obtained by requiring $m_v,m_i,\phi_v,\phi_i$ continuously differentiable; then all that is left to prove is that an initial condition can be found for each $I_j$, achievable by using the continuation of the solutions of \eqref{eq:period_dot_1}. Suppose $T_0 = T\left(t_0\right)$ is known and exists for some $t_0$. Then the IVP has a unique solution in the entire $I_0$, and this solution exists until $t_1$ is reached. However, by \eqref{eq:theo_cp_power_avg_power_period_eq}, $T(t)$ must be continuous — by definition $P(t)$ is continuous and so is the integral — therefore define

\begin{equation} T_1 = T\left(t_1\right) = \lim\limits_{t\to t_1^-} T\left(t\right) \label{eq:period_dot_3} \end{equation}

	 and if this limit exists and is finite, adopt $t_1,T\left(t_1\right)$ as a new IVP. The solution will exist on $I_1$ until $t_2$ is reached; if the limit \eqref{eq:period_dot_3} exists for $t_2$, adopt $t_2,T_2$ as the initial value for $I_2$, and so on. This process can be continued until such time $t_f$ when either $u$ is zero, the limit \eqref{eq:period_dot_3} does not exist or diverges for a certain $t_j$, or one of the amplitude or phase signals is not defined or they break the conditions needed to make $f$ compliant with the requirements of the Picard theorem. Thus, the IVP \eqref{eq:period_dot_1} with $T\left(t_0\right) = T_0$ has a unique solution on $\left[t_0,t_f\right)$. \hfill$\blacksquare$
\vspace{5mm}
\hrule
\vspace{5mm} %>>>

	Naturally one wonders whether theorems \ref{theo:activereactivepower} and \ref{theo:activepowerperiod} are suitable to represent the common active and reactive powers in static phasors. The customary definitions of $P$ and $Q$ are immediate from the theorem; as for the period $T$, it is simple to prove that $T_0 = T(0) = 2\pi/\omega$ is a solution to \eqref{eq:theo_cp_power_avg_power_period_eq} (so is any multiple of $\pi/\omega$). Then \eqref{eq:period_dot_1} becomes

\begin{equation} \dot{T}\left(t\right)\left[p\left(t+T\left(t\right)\right) - P\left(t\right)\right] = 0 \label{eq:period_dot_4} \end{equation}

	Analyzing $u\left(t,T\left(t\right)\right)$, through simple algebra one obtains

\begin{equation} u\left(t,T\left(t\right)\right) = \dfrac{m_vm_i}{2} \cos\left(2\omega t + \phi_v + \phi_i\right) \end{equation}

	therefore one can obtain the roots of $u$:

\begin{equation} t_i = \dfrac{1}{2\omega}\left(i\pi + \dfrac{\pi}{2} - \phi_v - \phi_i\right), i\in\mathbb{Z} \end{equation}

	At a first glance, in each of the $t_i$ \eqref{eq:period_dot_4} becomes $0\times \dot{T}\left(t_i\right) = 0$ and $\dot{T}$ is undefined. For any other time instants however $\dot{T}\left(t\right) = 0$ by \eqref{eq:period_dot_4} and by the continuity of $T$ the limit \eqref{eq:period_dot_3} exists for all $t_i$ and is $T_0$, therefore $T(t) = T_0$ for all $t$.

	Theorem \ref{theo:activereactivepower} establishes the same active and reactive power quantities that static phasors enjoy, and with exactly the same interpretation: $P(t)$ is the power performed by the bipole over some interval $\left[t,t+T(t)\right]$, while $Q(t)$ vanishes over the same interval; moreover, it is simple to prove the generalized counterpart of equation \eqref{eq:active_reactive_current} and theorem \ref{corollary:direct_quad_current}, stating that the (generalized) sinusoidal current can be decomposed into one component in phase with the voltage and another in quadrature with voltage.

\begin{theorem}[Direct and quadrature components of sinusoidal currents]\label{theo:direct_quad_current_nonst} %<<<
	Let $v,i,P,Q$ as defined in theorem \ref{theo:activereactivepower}. Then $i$ can be written as

\begin{equation} i(t) = \dfrac{2P(t)}{m_v(t)}\cos\left(\psi(t) + \phi_v(t)\right) + \dfrac{2Q(t)}{m_v(t)}\sin\left(\psi(t) + \phi_v(t)\right) .\end{equation}
\end{theorem}
\textbf{Proof.} By simple algebraic manipulation:

\begin{align}
	i(t)
	&= m_i(t)\cos\left(\psi(t) + \phi_v(t)\right) \nonumber\\[3mm]
	&= m_i(t)\cos\left(\psi(t) + \phi_v(t) - \Delta\phi\right) \nonumber\\[3mm]
	&= m_i(t)\left[\cos\left(\psi(t) + \phi_v(t)\right)\cos\left(\Delta\phi(t)\right) + \sin\left(\psi(t) + \phi_v(t)\right)\sin\left(\Delta\phi(t)\right) \right] \nonumber\\[3mm]
	&= \dfrac{2P(t)}{m_v(t)}\cos\left(\psi(t) + \phi_v(t)\right) + \dfrac{2Q(t)}{m_v(t)}\sin\left(\psi(t) + \phi_v(t)\right)
\end{align} \hfill$\blacksquare$

\vspace{5mm}
\hrule
\vspace{5mm} %>>>

\begin{example}[Application of theorem \ref{theo:activereactivepower}] \label{example:rlc_dpt_power} %<<<

	Continuing from example \ref{corollary:complex_equivalence_phasorialodes}, consider the RLC circuit of figure with the same voltage excitation and the same adopted values. Figure \ref{fig:amp_phase_voltage_signals_power} shows the active $P(t)$ and reactive power $Q(t)$ supplied by the excitation source $V(t)$, as calculated through equation \eqref{eq:dynamical_complex_power_def}. At the same time, figure \ref{fig:period_signals} shows the many period signals $T(t)$ calculated through equation \eqref{eq:theo_cp_power_avg_power_period_eq}. This figure shows that in the proposed framework, much like in static phasors, equation \eqref{eq:theo_cp_power_avg_power_period_eq} may have several (or infinite) solutions for $T(t)$; in general, the period adopted is the smallest positive one. To this extent, the figure shows a zoomed-in version showing the first solution. Finally, all solutions tend to a multiple of $\pi/\omega_0$, which is expected as the excitation signal $v(t)$ ``tends'' to a static sinusoid, then $T(t)$ tends to the period of a static phasor.

% POWER CURVES %<<<
\begin{figure}
        \begin{center}
                \begin{tikzpicture}
                        \begin{axis}[
                                width = 1\columnwidth,
                                height = 1/1.618*\columnwidth,
                                title={Active and reactive power supplied by $V(t)$},
                                xlabel={Time (s)},
				ylabel style = red,
 %                              ylabel={$P(t)$ (W)},
				y axis line style = {red, thick},
				every y tick label/.append style ={red},
				every y tick/.append style ={thick, red},
                                xmin=0, xmax=1,
                                ymin=40, ymax=220,
                                xtick={0,0.2,...,1},
                                ytick={40,80,...,200},
				ylabel style = {align=center},
				axis y line*=left,
                                %ymajorgrids=true,
                                %xmajorgrids=true,
                                every axis plot/.append style={thick},
				legend pos = south east
                        ]
                                \addplot[red,smooth] table[col sep=comma,header=false,x index=0,y index=6]{data/rlc_sim/data_rlc_sim_dps.csv};
				\addlegendentry{$P$ (W)}
                        \end{axis}
%
                        \begin{axis}[
                                width = 1\columnwidth,
                                height = 1/1.618*\columnwidth,
                                xmin=0, xmax=1,
		        	axis y line*=right,
                               % ylabel={$Q(t)$ (VAR) },
				y axis line style = {blue, thick},
				every tick label/.append style ={blue},
				every y tick/.append style ={thick, blue},
				ylabel near ticks,
                                ymin=-91, ymax=0,
                                xtick={-90,-80,...,0},
				axis x line=none,
                                %ytick={-0.2,-0.1,...,1},
		        	ylabel style = {align=center},
                                %ymajorgrids=true,
                                %xmajorgrids=true,
                                every axis plot/.append style={thick},
				legend pos = north east
                        ]
                                \addplot[blue ,smooth] table[col sep=comma,header=false,x index=0,y index = 7]{data/rlc_sim/data_rlc_sim_dps.csv};
				\addlegendentry{$Q$ (VAR)}
                        \end{axis}
                \end{tikzpicture}
        \caption
[Dynamic Phasor simulation of active and reactive power output.]
{Dynamic Phasor simulation of active P (red, left axis) and reactive Q (blue, right axis) power output by the voltage source $V(t)$ of the circuit of Figure \ref{fig:complexification_example} as calculated by theorem \ref{theo:activereactivepower}.}
        \label{fig:amp_phase_voltage_signals_power}
        \end{center}
\end{figure}
% >>>

% PERIOD TIME CURVES <<<
\begin{figure}
        \begin{center}
                \begin{tikzpicture}
                        \begin{axis}[
				name = ax_main,
                                width = 0.9\columnwidth,
                                height = 0.9/1.618*\columnwidth,
                                title={Period signals from DP simulations},
                                xlabel={Time (s)},
                                ylabel={Period $\left(\times \pi/\omega_0\right)$},
                                xmin=0, xmax=1,
                                ymin=0, ymax=10.5,
                                xtick={0,0.1,...,1},
                                ytick={0,2,...,10}, 
                                legend pos=south east,
                                ymajorgrids=true,
                                xmajorgrids=true,
                                %grid style=dashed,
                                colormap name=hsv,
                                cycle list={[ colors of colormap={0,100,200,300,400,500,600,700,800,900} ]},
                                every axis plot/.append style={thick},
                        ]
                        \addplot table[col sep=comma,header=false,x index=0,y index=1] {data/rlc_sim/period.csv};
                        \addplot table[col sep=comma,header=false,x index=0,y index=2] {data/rlc_sim/period.csv};
                        \addplot table[col sep=comma,header=false,x index=0,y index=3] {data/rlc_sim/period.csv};
                        \addplot table[col sep=comma,header=false,x index=0,y index=4] {data/rlc_sim/period.csv};
                        \addplot table[col sep=comma,header=false,x index=0,y index=5] {data/rlc_sim/period.csv};
                        \addplot table[col sep=comma,header=false,x index=0,y index=6] {data/rlc_sim/period.csv};
                        \addplot table[col sep=comma,header=false,x index=0,y index=7] {data/rlc_sim/period.csv};
                        \addplot table[col sep=comma,header=false,x index=0,y index=8] {data/rlc_sim/period.csv};
                        \addplot table[col sep=comma,header=false,x index=0,y index=9] {data/rlc_sim/period.csv};
                        \addplot table[col sep=comma,header=false,x index=0,y index=10]{data/rlc_sim/period.csv};
                        \coordinate (c1) at (axis cs:0,0.92);
                        \coordinate (c2) at (axis cs:1,0.92);
                        \end{axis}
%
                        \begin{axis}[
                                name = ax_zoomed,
                                at={($(ax_main.north east)-(0.75\columnwidth,1.75/1.618*\columnwidth)$)},
                                width = 0.9\columnwidth,
                                height = 0.9/1.618*\columnwidth,
                                xmin=0, xmax=1,
                                ymin=0.92, ymax=1.055,
                                xtick={0,0.1,...,1},
                                ytick={0.925,0.95,...,1.05},
				xlabel=\empty,
				ylabel=\empty,
				tick label style={/pgf/number format/fixed},
				legend columns=2,
				legend style={/tikz/every even column/.append style={column sep=0.5cm}},
                                ymajorgrids=true,
                                xmajorgrids=true,
                                %grid style=dashed,
                                every axis plot/.append style={thick},
                                axis background/.style = {
                                        preaction = {
                                        path picture = {
                                        \draw[fill=white,line width=0mm] (axis cs:0,1.055) rectangle (axis cs:1,0.92);
                                                }
                                        }
                                }    
                        ]
				\addplot[red, smooth]         table[col sep=comma,header=false,x index=0,y index=1]{data/rlc_sim/period.csv};
                        \end{axis}
                        % draw dashed lines from rectangle in first axis to corners of second
                        \draw [gray,dashed] (c1) -- (ax_zoomed.north west);
                        \draw [gray,dashed] (c2) -- (ax_zoomed.north east);
                \end{tikzpicture}
        \caption
[Period signals as defined in theorem \ref{theo:activereactivepower} calculated for the circuit of figure \ref{fig:complexification_example}.]
{Period signals $T(t)$ as defined in theorem \ref{theo:activereactivepower} calculated for the circuit of figure \ref{fig:complexification_example}. Each curve belongs to a period signal $T(t)$ corresponding to a distinct initial period $T_0$ obtained by numerically solving \eqref{eq:theo_cp_power_avg_power_period_eq} at $t=0$ and using this value as a initial condition for integrating the differential equation \eqref{eq:period_dot_1}. The bottom plot shows a zoom-in detailing the smallest positive solution which is generally the one adopted as period.}
	\label{fig:period_signals}
        \end{center}
\end{figure}
% >>>

\examplebar
\end{example} %>>>

%-------------------------------------------------
\section{Some circuit analysis in Dynamic Phasor domain} %<<<1

	Following the developments of Dynamic Phasor Theory, we now want to prove that this theory proposed begets some network analysis results that one can use to make circuit resolution easier. We first prove the Dynamic Phasor equivalents of Kirchoff's Laws as direct consequences of the linearity of the Dynamic Phasor Transform.

\begin{theorem}[Kirchoff's Current Law in the Dynamic Phasor domain] \label{theo:kirchoff_current_1p} %<<<
Let $i_p(t)$, $p = 1,...,q$ be the nonstationary sinusoidal currents of a certain network meeting at a node, $I_p(t)$ their dynamic phasors. Then

\begin{equation} \sum\limits_{p=1}^q I_p(t) = 0 \end{equation}

\end{theorem}
\textbf{Proof.} By Kirchoff's Current Law in time domain, $\sum i_p(t) = 0$. Applying the dynamic phasor transform and using its linearity yields $\sum I_p(t) = 0$. \hfill$\blacksquare$ 
\vspace{5mm}
\hrule
\vspace{5mm} %>>>
	
\begin{theorem}[Kirchoff's Voltage Law in the Dynamic Phasor domain] \label{theo:kirchoff_voltage_1p} %<<<
Let $v_p(t)$, $p = 1,...,q$ be the nonstationary sinusoidal voltages of a certain network around a certain closed loop, $V_p(t)$ their dynamic phasors. Then

\begin{equation} \sum\limits_{p=1}^q V_p(t) = 0 \end{equation}

\end{theorem}
\textbf{Proof:} akin to theorem \ref{theo:kirchoff_current}. \hfill$\blacksquare$
\vspace{5mm}
\hrule
\vspace{5mm} %>>>

	Further, we want to know what are the voltage-current relationships of linear elements in the Dynamic Phasor Domain.

\begin{theorem}[Dynamic Phasor Capacitive Relationship]\label{theo:1p_capacitive_conductance} % <<<
Let $v(t)$ be the voltage across a capacitor like in the figure below. Denote $V =\mathbf{P_D^{\omega}} \left[v\right] = v_d\left(t\right) + jv_q\left(t\right)$ as the corresponding phasor of $v(t)$, $\omega$ as its apparent frequency and $\psi = \int_{0}^{t} \omega(x)dx$. Also let $\mathbf{T}_\psi$ be the $dq$ transform matrix where $\omega = \dot{\psi}$ exists and is continuous.

\begin{center}
        \begin{tikzpicture}[american,scale=1.2,transform shape,line width=0.75]
		\draw(0,0)   to[short,o-o]       (3,0);
		\draw(0,3)   to[short,o-o]       (3,3);
		\draw(1.5,3) to[C,l=$C$,f>^=$i$] (1.5,0);
                \draw(3,0) to[open,european,voltage/distance from node=0.5mm, voltage/bump b=1, v=$v$] (3,3);
        \end{tikzpicture}
\end{center}

	Then the current through the capacitor is such that

\begin{equation}
\left\{\begin{array}{l}
        i_d = C\dfrac{dv_d}{dt} - \omega C v_q \\[5mm]
        i_q = C\dfrac{dv_q}{dt} + \omega C v_d
\end{array}\right.
\end{equation}

	\noindent and the complex phasor $I$ obtained through the equation

\begin{equation} I = C\dfrac{dV}{dt} + j\omega C V \end{equation}

	is equal to the phasor corresponding to $i(t)$, $\mathbf{P_D^{\omega}}\left[i\right] = i_d\left(t\right) + ji_q\left(t\right)$.

\end{theorem}
\textbf{Proof:} writing the time differential equations,

\begin{equation} \mathbf{i}_{\alpha\beta} = 
\left[\begin{array}{c} i_\alpha \\ i_\beta \end{array}\right] = 
\left[\begin{array}{c} C\dfrac{dv_\alpha}{dt} \\[5mm] C\dfrac{dv_\beta}{dt} \end{array}\right] \Leftrightarrow
 \mathbf{i}_{\alpha\beta} = C\dfrac{d\mathbf{v}_{\alpha\beta}}{dt} \end{equation}

	Multiplying both sides by $\mathbf{T}_{\psi}$,

\begin{align}
	\mathbf{i}_{dq} &= \mathbf{T}_\psi \mathbf{i}_{\alpha\beta} \nonumber\\[3mm]
	=& \mathbf{T}_\psi C\dfrac{d\mathbf{v}_{\alpha\beta}}{dt} \nonumber\\[3mm] 
	\substack{\text{(Lemma \ref{theo:dq_1p_diff})} \\ =}\hspace{2mm}& \mathbf{T}_\psi C\left[ \mathbf{T}^{-1}_\psi \dfrac{d}{dt}\left(\mathbf{v}_{dq}\right) + \dfrac{d}{dt}\left(\mathbf{T}^{-1}_\psi \right) \mathbf{v}_{dq} \right] \nonumber\\[3mm] 
	=& C\left[ \mathbf{T}_\psi \mathbf{T}^{-1}_\psi \dfrac{d}{dt}\left(\mathbf{v}_{dq}\right) + \mathbf{T}_\psi \dfrac{d}{dt}\left(\mathbf{T}^{-1}_\psi \right) \mathbf{v}_{dq} \right] \nonumber\\[3mm] 
	=& C\left[ \dfrac{d}{dt}\left(\mathbf{v}_{dq}\right) + \mathbf{T}_{\psi}\dfrac{d}{dt}\left(\mathbf{T}^{-1}_\psi \right) \mathbf{v}_{dq} \right] \nonumber\\[3mm]
	\substack{\text{(Lemma \ref{lemma:1p_t_ndifftminus_product})} \\ =}\hspace{2mm}& C\left\{ \dfrac{d}{dt}\left(\mathbf{v}_{dq}\right) + \dfrac{d\psi}{dt} \left[\begin{array}{cc}    0 & -1 \\[5mm] 1 & 0 \end{array}\right] \mathbf{v}_{dq0} \right\} \nonumber\\[3mm]
%
%
&= \left[\begin{array}{l}
        C\dfrac{dv_d}{dt} - \omega C v_q \\[5mm]
        C\dfrac{dv_q}{dt} + \omega C v_d
\end{array}\right]
\end{align}

	Applying the complexification operator,

\begin{equation} I = i_d + ji_q = C\dfrac{dv_d}{dt} - \omega C v_q + j\left(C\dfrac{dv_q}{dt} + \omega C v_d\right) = C\dfrac{d}{dt}\left(v_d + jv_q\right) + j\omega C\left(v_d + jv_q\right) = C\dfrac{dV}{dt} + j\omega C V \end{equation}

\hfill$\blacksquare$
\vspace{5mm}
\hrule
\vspace{5mm}
%>>>

\begin{theorem}[Dynamic Phasor Inductive Relationship]\label{theo:1p_inductive_impedance} %<<< 
Let $i(t)$ be the current through an inductor like in the figure below. Denote $I =\mathbf{P_D^{\omega}} \left[i\right] = i_d\left(t\right) + ji_q\left(t\right)$ as the corresponding phasor of $i(t)$, $\omega$ as its apparent frequency and $\psi = \int_{0}^{t} \omega(x)dx$. Also let $\mathbf{T}_\psi$ be the $dq$ transform matrix where $\omega = \dot{\psi}$ exists and is continuous.

\begin{center}
        \begin{tikzpicture}[american,scale=1.2,transform shape,line width=0.75, cute inductors]
		\draw(0,0)	to[short,o-]		(1,0)
				to[L,l=$L$,f>^=$i$]	(4,0)
				to[short,-o]		(5,0);
                \draw(0,-0.2) to[open,european,voltage/distance from node=0.5mm, voltage/bump b=2, v<=$v$] (5,-0.2);
        \end{tikzpicture}
\end{center}

	Then the current through the capacitor is such that

\begin{align}
\left\{\begin{array}{l}
        v_d = L\dfrac{di_d}{dt} - \omega L i_q \\[5mm]
        v_q = L\dfrac{di_q}{dt} + \omega L i_d \\[5mm]
\end{array}\right.
\end{align}

	\noindent and the complex phasor $V$ obtained through the equation

\begin{equation} V = L\dfrac{dI}{dt} + j\omega L I \end{equation}

	is equal to the phasor corresponding to $v(t)$, $\mathbf{P_D^{\omega}}\left[v\right] = v_d\left(t\right) + jv_q\left(t\right)$.

\end{theorem}
\textbf{Proof:} a repetition of theorem \ref{theo:1p_capacitive_conductance}. Writing the time differential equations,

\begin{equation} \mathbf{v}_{\alpha\beta} = 
\left[\begin{array}{c} v_\alpha \\ v_\beta \end{array}\right] = 
\left[\begin{array}{c} L\dfrac{di_\alpha}{dt} \\[5mm] L\dfrac{di_\beta}{dt} \end{array}\right] \Leftrightarrow
\mathbf{v}_{\alpha\beta} = L\dfrac{d\mathbf{i}_{\alpha\beta}}{dt} \end{equation}

	Multiplying both sides by $\mathbf{T}_{\psi}$,

\begin{align}
	\mathbf{v}_{dq} &= \mathbf{T}_\psi \mathbf{v}_{\alpha\beta} \nonumber\\[3mm]
	&= \mathbf{T}_\psi L\dfrac{d\mathbf{i}_{\alpha\beta}}{dt} \nonumber\\[3mm] 
	&\substack{\text{(Lemma \ref{theo:dq_1p_diff})} \\ =}\hspace{2mm} \mathbf{T}_\psi L\left[ \mathbf{T}^{-1}_\psi \dfrac{d}{dt}\left(\mathbf{i}_{dq}\right) + \dfrac{d}{dt}\left(\mathbf{T}^{-1}_\psi \right) \mathbf{i}_{dq} \right] \nonumber\\[3mm] 
	&= L\left[ \mathbf{T}_\psi \mathbf{T}^{-1}_\psi \dfrac{d}{dt}\left(\mathbf{i}_{dq}\right) + \mathbf{T}_\psi \dfrac{d}{dt}\left(\mathbf{T}^{-1}_\psi \right) \mathbf{i}_{dq} \right] \nonumber\\[3mm] 
	&= L\left[ \dfrac{d}{dt}\left(\mathbf{i}_{dq}\right) + \mathbf{T}_{\psi}\dfrac{d}{dt}\left(\mathbf{T}^{-1}_\psi \right) \mathbf{i}_{dq} \right] \nonumber\\[3mm]
	&\substack{\text{(Lemma \ref{lemma:1p_t_ndifftminus_product})} \\ =}\hspace{2mm} L\left\{ \dfrac{d}{dt}\left(\mathbf{i}_{dq}\right) + \dfrac{d\psi}{dt} \left[\begin{array}{cc}    0 & -1 \\[5mm] 1 & 0 \end{array}\right] \mathbf{i}_{dq} \right\} \nonumber\\[3mm]
%
%
&= \left[\begin{array}{l}
        L\dfrac{di_d}{dt} - \omega L i_q \\[5mm]
        L\dfrac{di_q}{dt} + \omega L i_d
\end{array}\right]
\end{align}

	Applying the complexification operator,

\begin{equation} V = v_d + jv_q = C\dfrac{di_d}{dt} - \omega L i_q + j\left(L\dfrac{di_q}{dt} + \omega L i_d\right) = L\dfrac{d}{dt}\left(i_d + ji_q\right) + j\omega L\left(i_d + ji_q\right) = L\dfrac{dI}{dt} + j\omega L I \end{equation}

\hfill$\blacksquare$

\vspace{5mm}
\hrule
\vspace{5mm}
% >>>

	Theorems \ref{theo:1p_capacitive_conductance} and \ref{theo:1p_inductive_impedance} are essentially the application of theorems \ref{theo:1p_ode_solution} and \ref{corollary:complex_equivalence_phasorialodes} to the equations $i = C\dot{v}$ and $v = L\dot{i}$. It is not difficult to see that the current-voltage relationship of a resistor is also linear, that is, $v(t) = Ri(t) \Rightarrow V(t) = RI(t)$. Theorems \ref{theo:1p_capacitive_conductance} and \ref{theo:1p_inductive_impedance} determine that inductors and capacitors have phasorial relationships of the form, completing the relationships as in \eqref{eq:Linear_relationships_1p}.

\begin{equation} \left\{\begin{array}{l} \text{Linear inductor: } v(t) = L \dot{i}(t) \Rightarrow V(t) = L\dot{I} + j\omega L I \\[3mm] \text{Linear capacitor: } i(t) = C \dot{v}(t) \Rightarrow I(t) = C\dot{V} + j\omega C V \\[3mm] \text{Linear resistor: } v(t) = Ri(t) \Rightarrow V(t) = RI(t) \end{array}\right. \label{eq:Linear_relationships_1p} \end{equation}

	In essence, these relationships stem from the fact that

\begin{equation} y(t) = \dot{x}(t) \Rightarrow Y(t) = \dot{X} + j\omega X, \label{eq:dpt_diff_eq}\end{equation}

	\noindent which can be asserted by applying theorem \ref{corollary:complex_equivalence_phasorialodes} to the differential equation $\dot{x}(t) - y(t) = 0$. Interestingly, \eqref{eq:dpt_diff_eq} is strikingly similar to the differentiation property \eqref{eq:stft_derivative_3} of the Short Time Fourier Transform, and exactly identical to the single-harmonic-approximated equations \eqref{sys:fdp_sys_fundamental_harmonic}. This is a fortunate result because, since most of the Dynamic Phasor literature, as well as Power Systems modelling using Dynamic Phasors, is based on the STFT, the Dynamic Phasor framework proposed here preserves the modelling procedures and techniques of the current literature.

	Formally, the Dynamic Phasor Transform proposed transforms derivatives in time domain to a particular operation in the Dynamic Phasor domain, and this will be explored later in this thesis. It will be proven that the combination of the complex operations form algebraic structures, which make modelling very simple and intuitive. For now, we use theorems \ref{theo:kirchoff_current_1p} through \ref{theo:1p_inductive_impedance} to prove that circuit analysis in the Dynamic Phasor domain is very close to that of static phasors, in the sense that these theorems make it possible to undertake the entire analysis in the complex domain instead of obtaining equations from the time domain.

\begin{example}[Circuit analysis in the DP domain]\label{example:dpdomain_secondorder} %<<<

	Consider the second-order circuit of figure \ref{fig:complexification_example_network_1p} where the same second-order circuit of example \ref{example:rlc_dpt} is shown, but in the Dynamic Phasor domain.

% MODELLING EXAMPLE: RLC CIRCUIT <<<
\begin{figure}[htb!]
\centering
        \begin{tikzpicture}[american,scale=1,transform shape,line width=0.75, cute inductors, voltage shift = 1]
	\ctikzset{/tikz/circuitikz/voltage/distance from node=10mm}
		\draw (0,0)
			to[vsource,sources/scale=1.25, v>=$V(t)$,invert] (0,4)
			to[L,l=$L$,f>^=$I_{L}$,v>=$V_{L}$,-*] (4,4) 
			to[C,l=$C$,f>^=$I_{C}$,v>=$V_{C}$,-*] (4,0) 
			to[short] (0,0); 
		\draw (4,4)
			to[short,f>^=$I_{R}$] (8,4) 
			to[R,l=$R$,v>=$V_{R}$] (8,0) 
			to[short]  (4,0);

		% DRAWING VOLTAGE LOOPS
		\draw[rounded corners=10,loop, draw opacity=0.3,->, color=blue] (0.5,0.5) -- (0.5,3.5) -- (3.5,3.5) -- (3.5,0.5) -- (1,0.5) ;
		\draw[rounded corners=10,loop, draw opacity=0.3,->, color=red] (4.5,0.5) -- (4.5,3.5) -- (7.5,3.5) -- (7.5,0.5) -- (5,0.5) ;
		% DRAWING NODE LABELS
		\node[shape=circle,draw,inner sep=1pt] at (  4,4.5) {$1$};
		
		% DRAWING LOOP LABELS

		\node[color=blue] at (2,2) {$L1$} ;
		\node[color=red ] at (6,2) {$L2$} ;
        \end{tikzpicture}
	\caption{Second-order circuit for example application of circuit analysis in the DP domain.}
	\label{fig:complexification_example_network_1p}
\end{figure} %>>>

	Applying Kirchoff's Current Law in the DP domain (theorem \ref{theo:kirchoff_current_1p}) in node 1 one obtains

\begin{equation} (KCL):\  I_L - I_C - I_R = 0\end{equation}

	\noindent and using Kirchoff's Voltage Law in the DP domain (theorem \ref{theo:kirchoff_voltage_1p}) in the voltage nodes yields

\begin{equation}\left\{\begin{array}{l} (L1):\ V_C - V + V_L = 0 \\[3mm] (L2):\ V_R = V_C \end{array}\right. .\end{equation}

	Finally, using the voltage-current relationships of the elements,

\begin{equation}\left\{\begin{array}{rl} (KCL):&\ I_L - \left(C\dot{V}_C + j\omega C V_C\right) - \dfrac{V_R}{R} = 0 \\[3mm] (L1):&\ V_C - V + L\dot{I}_L + j\omega LI_L = 0 \\[3mm] (L2):&\ V_R = V_C \end{array}\right. .\end{equation}

	Applying the third equation to the other two,

\begin{equation}\left\{\begin{array}{l} I_L - \left(C\dot{V}_R + j\omega C V_R\right) - \dfrac{V_R}{R} = 0 \\[3mm] V_R - V + L\dot{I}_L + j\omega LI_L = 0 \end{array}\right. .\end{equation}

	Now, differentiating the first equation, 

\begin{equation}\left\{\begin{array}{l} \dot{I}_L - \left(C\ddot{V}_R + j\dot{\omega} C V_R + j\omega C \dot{V}_R\right) - \dfrac{\dot{V}_R}{R} = 0 \\[3mm] V_R - V + L\dot{I}_L + j\omega LI_L = 0 \end{array}\right. .\end{equation}

	\noindent and substituting $\dot{I}_L$ from the second equation into the differentiated first equation:

\begin{equation}\dfrac{-V_R + V}{L} + j\omega I_L - \left(C\ddot{V}_R + j\dot{\omega} C V_R + j\omega C \dot{V}_R\right) - \dfrac{\dot{V}_R}{R} = 0 .\end{equation}

	Finally, one can isolate $I_L$ from the (KCL) equation and

\begin{equation}\dfrac{-V_R + V}{L} + j\omega \left[\left(C\dot{V}_R + j\omega C V_R\right) + \dfrac{V_R}{R}\right] - \left(C\ddot{V}_R + j\dot{\omega} C V_R + j\omega C \dot{V}_R\right) - \dfrac{\dot{V}_R}{R} = 0 .\end{equation}

	Dividing the equation by $C$ and grouping the terms, 

\begin{equation} \ddot{V}_R(t) + \dot{V}_R(t)\left(\dfrac{1}{RC} + 2j\omega(t)\right) + V_R\left\{ \dfrac{1}{LC}  -\omega^2(t) + j \left[ \dot{\omega}(t) + \dfrac{1}{RC}\omega(t)\right]\right\} -\dfrac{1}{LC} V(t) = 0, \label{eq:rlc_complex_diffeq_dpt}\end{equation}

	\noindent which is the exact same equation as \eqref{eq:rlc_complex_diffeq} of example \ref{example:rlc_dpt}.

\examplebar
\end{example} %>>>

	In example \ref{example:rlc_dpt}, the final equation \eqref{eq:rlc_complex_diffeq} that models $V_R(t)$ is obtained by first obtaining the model in the time domain, and then using theorem \ref{corollary:complex_equivalence_phasorialodes} to transport the time-domain differential equation to the equivalent Dynamic Phasor differential equation. In contrast, example \ref{example:dpdomain_secondorder} shows that the circuit analysis can be carried entirely in the complex domain, by virtue of theorems \ref{theo:kirchoff_current_1p} through \ref{theo:1p_inductive_impedance}.

%-------------------------------------------------
\section{Three-Phase Dynamic Phasors} %<<<1

	We now want to import all the results of the Dynamic Phasor Transform to three-phase signals. It will be shown that with minimal adaptations, the Dynamic Phasor Transform can be constructed for three-phase signals with high resemblance to the DPT for single-phase signals. Further, it will be shown that a counterpart to theorems \ref{theo:1p_ode_solution} and \ref{corollary:complex_equivalence_phasorialodes} can be proven for three-phase signals, with the added challenge of dealing with an extra dimension — the zero-sequence component. Finally, it will be shown that the notions of complex, active and reactive power of theorem \ref{theo:activereactivepower} are also maintained.

%------------------------------------------------
\subsection{Synchronization basics: the $\alpha\beta\gamma$ and $dq0$ transforms} %<<<2

	We first show that in three-phase systems we can easily define counterparts to the $\alpha\beta$ and the $dq$ transform. Naturally, in three-phases a dimension is added; however, these transformations in three-phase systems are simpler because they are very known linear transformations based on particular matrices. The historical developments of these transforms can be found in  \cite{orourkeGeometricInterpretationReference2019,Park1929,Krause1965} and \cite{clarke1938problems}.

	First the definition of a three-phase signal is presented as a three-dimensional signal. The definition of a poly-phase quantity dates back to the initial developments in polyphase analysis by \cite{Fortescue1918}, who proved that any set of $N$ unbalanced phasors (therefore a polyphase quantity) can be written as the linear combination of $N$ symmetrical sets of balanced phasors; the set as a whole manifests in a single frequency. Since this thesis is based on single and three-phase systems, the definitions and theorems below focus on $N=3$.

	Then the $\alpha\beta\gamma$ or Clarke Transform is presented as the power-invariant variation of the original transform conceived by Emily Clarke in the 1930s. Clarke proposed this transform as a means to simplify the analysis of three-phase systems, in particular unbalanced systems, using the tools available at the time, which were largely based on the positive and negative sequence transform \pcite{clarke1938problems}. The innovation of Clarke's method was that three-phase quantities, when projected onto a stationary axis at a particular angle, were transformed into orthogonal quantities called $\alpha$, $\beta$ and $\gamma$ components and, if the original quantity was a balanced three-phase signal, the resulting transformed quantities would yield a null $\gamma$ component — effectively transforming three components into two orthogonal ones, rendering the analysis much easier.

	Finally, the Park Transform is presented as a linear transform equivalent to the rotation of the three-phase quantity of an arbitrary angle. This transform translates a time-varying three-phase system into a set of two axes, direct and quadrature, and a ``zero-sequence''component. The composition of the Park and Clarke transforms forms the $dq0$ transform.

	The definitions and theorems given are made to be as general as possible to avoid the natural terminology that comes with the historic fact that the transformations were built upon the analyses of polyphase systems heavily influenced by synchronous machinery; most of the jargon involved in the literature refers to elements of electrical machines such as stators, rotors and flux linkages. In recent years, there has been a push in the literature to expand these transforms and analyses to embrace modern switched systems; to this extent, \cite{orourkeGeometricInterpretationReference2019} presents a thorough development of the Clarke-Park or $dq0$ transform as a generalized geometric transform on the $\mathbb{R}^3$, offering a more formal approach that allows for the understanding of these transformations in the context of three-phase analyses not bound by a particular technology.

%--------------------------------------------------------------------------------------------------
\subsubsection{The Clarke Transform} %<<<3

\begin{definition}[Three-phase signal]\label{def:three_phase_signal} %<<<
	A three phase signal $\mathbf{x}$ is a three-dimensional quantity comprised of three sinusoidal signals, that is, there exist three positive functions $m_a$, $m_b$, $m_c$ called moduli (modulus in the singular) and three functions $\theta_a$, $\theta_b$ and $\theta_c$ called absolute angles such that

\begin{equation} \mathbf{x} = \left[\begin{array}{c} x_a\left(t\right)\\ x_b\left(t\right) \\ x_c\left(t\right) \end{array}\right] = \left[\begin{array}{c} m_a\left(t\right)\cos\theta_a(t)\\ m_b\left(t\right)\cos\theta_b(t) \\ m_c\left(t\right)\cos\theta_c(t) \end{array}\right], \end{equation}

	which is known as the $abc$ representation; the components are named phases $a$, $b$ and $c$. A three-phase quantity is called \textbf{balanced} if the three phases are:

\begin{itemize}
	\item \textbf{Symmetric}: they have the same amplitude, that is, $m_1 = m_2 = m_3$;
	\item \textbf{Direct}: they are delayed copies of one single function;
	\item \textbf{Sequential}: phases are delayed by the same quantity.
\end{itemize}

	In other words, there exist two functions: a $m\left(t\right)$, called modulus and a $\theta\left(t\right)$, called absolute angle, such that

\begin{equation} \mathbf{x} = m(t)\left[
	\begin{array}{c}
		\cos\left(\theta\right) \\[5mm]
		\cos\left(\theta - \dfrac{2\pi}{3}\right) \\[5mm]
		\cos\left(\theta + \dfrac{2\pi}{3}\right)
	\end{array}\right] \label{eq:balanced_threephase_def}
\end{equation}
	
\end{definition} %>>>

	One of the main properties of balanced three-phase signals is that, albeit being three-dimensional quantities, they only need two dimensions to be described: a modulus $m(t)$ and an angle $\theta(t)$. This already gives an idea that this quantity can be described by a complex function $X(t)$. Here we also import all the definitions of generalized sinusoids from the single-phase case: a ``balanced three-phase sinusoid'' is that which angle $\theta(t)$ can be written as the combination

\begin{equation} \theta(t) = \psi(t) + \phi(t),\ \psi(t) = \int_0^t \omega(s)ds \end{equation}

	\noindent where $\omega(t)$ is a chosen apparent frequency and $\phi(t)$ the corresponding apparent phase.

	The notation for phases $a$, $b$ and $c$ are legacy notations for the three windings $a$, $b$ and $c$ of a three-phase synchronous machine upon which the definitions were built. Because of this, the $abc$ representation is largely defined as the time-signals pertaining to the three phases of a voltage or current in a certain system under study. For this reason, the three phases of an inverter are also denominated as such. Also for the sake of clarity, a three-phase signal will be also denoted as a ``3$\phi$ signal''.

	In the context of Electric Power System, a ``three-phase quantity'' will generally be either a three-phase voltage or current; because the first analyses were made for machinery, magnetic fluxes can also be depicted as such quantities in some researches.

	In the 1920s, \cite{Fortescue1918} presented a method whereby a polyphase network was decomposed into symmetrical components, allowing a much simpler analysis of such networks especially in the context of network unbalances. \cite{clarke1938problems} greatly improved and simplified over Fortescue's results, developing a method of transforming a three-phase system into a sequence of linearly independent complex phasors; finally, in 1943, Clarke published her transform, defined below.

\begin{definition}[Clarke or $\alpha\beta\gamma$ transform]\label{def:clarke_transform} %<<<
	The Clarke Transform is the linear transformation

\begin{equation}
\mathbf{T_{\alpha\beta\gamma}} = \sqrt{\dfrac{2}{3}}
\left[\begin{array}{ccc}
1 & -\dfrac{1}{2} & -\dfrac{1}{2} \\[5mm]
0 &  \dfrac{\sqrt{3}}{2} & -\dfrac{\sqrt{3}}{2} \\[5mm]
\dfrac{1}{2} & \dfrac{1}{2} & \dfrac{1}{2}
\end{array}\right]
\end{equation}

\end{definition} %>>>

	In the original paper by Clarke, the transform was presented as scaled not by $\sqrt{2/3}$ but $2/3$. Definition \ref{def:clarke_transform} makes it power invariant: due to the fact that $\mathbf{T}_{\alpha\beta\gamma}$ has a determinant of $\sqrt{3}/2$, the three-phase power on the $\alpha\beta\gamma$ space $p_{3\phi}^{\alpha\beta\gamma} = v_\alpha i_\alpha + v_\beta i_\beta + v_\gamma i_\gamma$ was not equal to the three-phase power on the $abc$ space $p_{3\phi}^{abc} = v_ai_a + v_bi_b + v_ci_c$, but proportional to it by a factor of $3/2$. This is due to the fact that, without this scaling factor, the $\mathbf{T_{\alpha\beta\gamma}}$ transform is not unitary (its inverse is not equal to its transpose). The unitarity becomes true when the rooted scaling is used; hence, such factor was later added \pcite{CHATTOPADHYAY2008}.

	Also notably, the Clarke Transform is invertible and linear. Further, the ingenuity of Clarke's transform is that a balanced three-phase quantity as in definition \eqref{eq:balanced_threephase_def}, when transformed through the matrix $\mathbf{T_{\alpha\beta\gamma}}$, yields

\begin{equation} \mathbf{T_C} \left( m\hspace{-0.6mm}\left(t\right)\left[\begin{array}{c} \cos\theta\hspace{-0.6mm}\left(t\right) \\[5mm] \cos\left(\theta\hspace{-0.6mm}\left(t\right) - \dfrac{2\pi}{3}\right) \\[5mm] \cos\left(\theta\hspace{-0.6mm}\left(t\right) + \dfrac{2\pi}{3}\right) \end{array}\right] \right) = \sqrt{3}m(t)\hspace{-0.6mm}\left[\begin{array}{c} \cos\theta\hspace{-0.6mm}\left(t\right) \\[5mm] \sin\theta\hspace{-0.6mm}\left(t\right)\\[5mm] 0\end{array}\right], \end{equation}

	\noindent and it is obvious that this quantity is diffeomorphic to $m(t)e^{j\theta(t)}$, thus representing the complex phasor of $x(t)$ with respect to the fixed time reference, the same way that the $\alpha\beta$ transform did with single-phase quantities.

%--------------------------------------------------------------------------------------------------
\subsubsection{The Park Transform} %<<<3

	Not concurrently with Clarke's analysis, \cite{Park1929} published a generalization of the Two-Reaction Theory of Synchronous Machines by Blondel, which was later expanded in \cite{Doherty1926}. The Park Transform was used to express the flux linkages in salient-pole synchronous machines by defining two axes of rotation: axis $d$ for ``direct'' and axis $q$ for ``quadrature'', the former being directly aligned with the machine rotor and the latter aligned in quadrature with the rotor. The flux linkages are then projected onto the $abc$ magnetic axes. Finally, a zero-sequence component ``$0$'' was added. Figure \ref{fig:park_transform_rotor} shows the diagram as drawn by Park, showing the synchronous machine $abc$ phases and the rotating $d$ and $q$ frames.



% SYNCHRONOUS MACHINE SCHEMATIC <<<Add commentMore actions
\begin{figure}[ht]
\scalebox{1}{
\begin{tikzpicture}[scale=1,>={Stealth[inset=0mm,length=1.5mm,angle'=50]}, transform shape]
\ctikzset{color=stewartblue}Add commentMore actions

\node (O) at (0,0) {};

\pgfmathsetmacro{\rotorangle}{50}

\node[draw, circle, fill=gray!75!white, thick, minimum size=68mm] (graybg) at (0,0) {}; % SUM CIRCLE
\node[draw, circle, fill=white,         thick, minimum size=60mm] (whitefg) at (0,0) {}; % SUM CIRCLE

% rotor drawing <<<
\begin{scope}[rotate=\rotorangle]
	\path [name path=outercircle] (0,0) circle (28.5mm) ; % SUM CIRCLE
	\path [name path = leftline]  (-2.5mm,-30mm) -- (-2.5mm,30mm);
	\path [name path = rightline] ( 2.5mm,-30mm) -- ( 2.5mm,30mm);

	\path[name path = leftmostline]  (-10mm,10mm) -- (-10mm,30mm);
	\path[name path = rightmostline] ( 10mm,10mm) -- ( 10mm,30mm);

	\path [name intersections={of=outercircle and leftmostline, by={southwest_intersection}}];
	\path [name intersections={of=outercircle and rightmostline, by={southeast_intersection}}];

	\path[name path = leftmostline]  (-10mm, 10mm) -- (-10mm, 30mm);
	\path[name path = rightmostline] ( 10mm, 10mm) -- ( 10mm, 30mm);

	\path[name path = botsouthline] ( -12mm, 22mm) -- ( 12mm, 22mm);

	\path [name intersections={of=botsouthline and rightmostline, by={southeast_T}}];

	\path [name intersections={of=botsouthline and leftmostline, by={southwest_T}}];

	\path [name intersections={of=botsouthline and leftline, by={southwest_inner_T}}];
	\path [name intersections={of=botsouthline and rightline, by={southeast_inner_T}}];

		% Compute angles and radius using \path let
		\node (A) at (southwest_intersection) {};
		\node (B) at (southeast_intersection) {};
		\node (C) at (southwest_T) {};
		\node (D) at (southeast_T) {};
		\node (E) at (southwest_inner_T) {};
		\node (F) at (southeast_inner_T) {};

		% Get coordinates for calculations
		\path (O); \pgfgetlastxy{\Ox}{\Oy}
		\path (A); \pgfgetlastxy{\Ax}{\Ay}
		\path (B); \pgfgetlastxy{\Bx}{\By}

		% Compute angles
		\pgfmathsetmacro{\angleA}{atan2(\Ay - \Oy,\Ax - \Ox)}
		\pgfmathsetmacro{\angleB}{atan2(\By - \Oy,\Bx - \Ox)}

		% Compute radius (distance from O to A)
		\pgfmathsetmacro{\radius}{veclen(\Ax - \Ox,\Ay - \Oy)/28.45274}

		% Draw the arc centered at O from A to B
		\draw[thick] (E.center) -- (C.center) -- ([shift=(\angleA:\radius)]O.center) arc(\angleA:\angleB:\radius) -- (D.center) -- (F.center);

	% SAME PROCESS FOR BOTTOM T

	\path[name path = leftmostline]  (-10mm,-10mm) -- (-10mm,-30mm);
	\path[name path = rightmostline] ( 10mm,-10mm) -- ( 10mm,-30mm);

	\path [name intersections={of=outercircle and leftmostline, by={southwest_intersection}}];

	\path [name intersections={of=outercircle and rightmostline, by={southeast_intersection}}];

	\path[name path = leftmostline]  (-10mm,-10mm) -- (-10mm,-30mm);
	\path[name path = rightmostline] ( 10mm,-10mm) -- ( 10mm,-30mm);

	\path[name path = botsouthline] ( -12mm,-22mm) -- ( 12mm,-22mm);

	\path [name intersections={of=botsouthline and rightmostline, by={southeast_T}}];
	\path [name intersections={of=botsouthline and leftmostline, by={southwest_T}}];
	\path [name intersections={of=botsouthline and leftline, by={southwest_inner_T}}];
	\path [name intersections={of=botsouthline and rightline, by={southeast_inner_T}}];

	% Compute angles and radius using \path let
		\node (G) at (southwest_intersection) {};
		\node (H) at (southeast_intersection) {};
		\node (I) at (southwest_T) {};
		\node (J) at (southeast_T) {};
		\node (K) at (southwest_inner_T) {};
		\node (L) at (southeast_inner_T) {};

		% Get coordinates for calculations
		\path (O); \pgfgetlastxy{\Ox}{\Oy}
		\path (G); \pgfgetlastxy{\Ax}{\Ay}
		\path (H); \pgfgetlastxy{\Bx}{\By}

		% Compute angles
		\pgfmathsetmacro{\angleA}{atan2(\Ay - \Oy,\Ax - \Ox)}
		\pgfmathsetmacro{\angleB}{atan2(\By - \Oy,\Bx - \Ox)}

		% Compute radius (distance from O to A)
		\pgfmathsetmacro{\radius}{veclen(\Ax - \Ox,\Ay - \Oy)/28.45274}

		% Draw the arc centered at O from A to B
		\draw[thick] (F.center) -- (L.center) -- (J.center) -- ([shift=(\angleB:\radius)]O.center) arc(\angleB:\angleA:\radius) -- (I.center) -- (K.center) -- (E.center);

	\def\coil#1{ {4mm * cos(\t * pi r)}, {1mm * (2*#1 + \t) + 1mm*sin(\t * pi r)) - 15mm} }

	    % Draw the part of the coil behind the rectangle
	    \foreach \n in {1,...,15} { \draw[domain={0:1},stewartblue,smooth,thick, variable=\t,samples=15] plot (\coil{\n});  }

		\draw[domain={1:0.5},stewartblue,smooth,variable=\t, thick, samples=15, preaction={draw,white,line width=1pt}] plot (\coil{0}) node (coilstart) {};

	    \fill[fill=white] (-2.35mm,-20mm) rectangle (2.35mm,20mm);
		

	    % Draw the part of the coil in front of the rectangle
	    \foreach \n in {0,1,...,14} {
		\draw[domain={1:2},stewartblue,smooth,variable=\t, thick, ,samples=15, preaction={draw,white,line width=2pt}] plot (\coil{\n});
		}

		\draw[domain={1:1.4},stewartblue, thick, smooth,variable=\t,samples=15,] plot (\coil{15}) node (coilend) {}; 

		\draw[thick,stewartblue,preaction={draw,white,line width=2pt}] (coilend.center) -- ++ (20mm,0) node (battstart) {};

		\draw[thick,stewartblue] (battstart.center) to [vsourceAM, l=$v_F$, f^<=$i_F$] (battstart |- coilstart) -- ([shift=({3mm,0})]coilstart.center);

		\draw[thick,stewartpink] (-10mm,0) circle (1mm) node (dcoilright) {} ;
		\draw[thick,stewartpink] ($(dcoilright) + ({1mm*cos(45)},{1mm*sin(45)})$) -- ($(dcoilright) + ({-1mm*cos(45)},{-1mm*sin(45)})$);
		\draw[thick,stewartpink] ($(dcoilright) + ({1mm*cos(135)},{1mm*sin(135)})$) -- ($(dcoilright) + ({-1mm*cos(135)},{-1mm*sin(135)})$);

		\draw[thick,stewartpink] (10mm,0) circle (1mm) node (dcoilleft) {} ;
		\draw[thick,stewartpink, fill] (dcoilleft.center) circle (0.25mm) ;

		\draw[thick,stewartpink] (0,-25mm) circle (1mm) node (dcoilup) {} ;
		\draw[thick,stewartpink] ($(dcoilup) + ({1mm*cos(45)},{1mm*sin(45)})$) -- ($(dcoilup) + ({-1mm*cos(45)},{-1mm*sin(45)})$);
		\draw[thick,stewartpink] ($(dcoilup) + ({1mm*cos(135)},{1mm*sin(135)})$) -- ($(dcoilup) + ({-1mm*cos(135)},{-1mm*sin(135)})$);

		\draw[thick,stewartpink] (0,25mm) circle (1mm) node (dcoilsouth) {} ;
		\draw[thick,stewartpink, fill] (dcoilsouth.center) circle (0.25mm) ;
\end{scope} %>>>

	\node[stewartpink] at ([shift=({ 2.5mm, 2.5mm})]dcoilsouth) {$q$};
	\node[stewartpink] at ([shift=({-2.5mm,-2.5mm})]dcoilup) {$q'$};

	\node[stewartpink] at ([shift=({ 3.5mm, 0mm})]dcoilleft) {$d$};
	\node[stewartpink] at ([shift=({-2.5mm,-2.5mm})]dcoilright) {$d'$};

% A COIL
		\draw[thick] (32mm,0) circle (1mm) node (ain) {} ;
		\node at (37mm,0) {$a$} ;
		\draw[thick, fill] (ain.center) circle (0.25mm) ;

		\draw[thick] (-32mm,0) circle (1mm) node (aout) {} ;
		\draw[thick] ($(aout) + ({1mm*cos(45)},{1mm*sin(45)})$) --   ($(aout) + ({-1mm*cos(45)},{-1mm*sin(45)})$);
		\draw[thick] ($(aout) + ({1mm*cos(135)},{1mm*sin(135)})$) -- ($(aout) + ({-1mm*cos(135)},{-1mm*sin(135)})$);
		\node at (-37mm,0) {$a'$} ;

% B COIL
\begin{scope}[rotate=120]
		\draw[thick] (32mm,0) circle (1mm) node (ain) {} ;
		\draw[thick, fill] (ain.center) circle (0.25mm) ;

		\draw[thick] (-32mm,0) circle (1mm) node (aout) {} ;
		\draw[thick] ($(aout) + ({1mm*cos(45)},{1mm*sin(45)})$) --   ($(aout) + ({-1mm*cos(45)},{-1mm*sin(45)})$);
		\draw[thick] ($(aout) + ({1mm*cos(135)},{1mm*sin(135)})$) -- ($(aout) + ({-1mm*cos(135)},{-1mm*sin(135)})$);
\end{scope}

\node at ({37mm*cos(120)},{37mm*sin(120)}) {$b$} ;
\node at ({37mm*cos(300)},{37mm*sin(300)}) {$b'$} ;

% C COIL
\begin{scope}[rotate=-120]
		\draw[thick] (32mm,0) circle (1mm) node (ain) {} ;
		\draw[thick, fill] (ain.center) circle (0.25mm) ;

		\draw[thick] (-32mm,0) circle (1mm) node (aout) {} ;
		\draw[thick] ($(aout) + ({1mm*cos(45)},{1mm*sin(45)})$) --   ($(aout) + ({-1mm*cos(45)},{-1mm*sin(45)})$);
		\draw[thick] ($(aout) + ({1mm*cos(135)},{1mm*sin(135)})$) -- ($(aout) + ({-1mm*cos(135)},{-1mm*sin(135)})$);
\end{scope}

\node at ({37mm*cos(-120)},{37mm*sin(-120)}) {$c$} ;
\node at ({37mm*cos(60)},{37mm*sin(60)}) {$c'$} ;

\pgfmathsetmacro{\sizein}{35}
\pgfmathsetmacro{\sizeout}{50}
\pgfmathsetmacro{\sizemid}{(\sizein + \sizeout)/2}

\draw[thick] (0mm, \sizein*1mm) -- (0mm, \sizeout*1mm);

\pgfmathsetlengthmacro{\a}{\sizein*1mm*cos(\rotorangle + 90)}
\pgfmathsetlengthmacro{\b}{\sizein*1mm*sin(\rotorangle + 90)}
\pgfmathsetlengthmacro{\c}{\sizeout*1mm*cos(\rotorangle + 90)}
\pgfmathsetlengthmacro{\d}{\sizeout*1mm*sin(\rotorangle + 90)}

\draw[thick] (\a,\b) -- (\c,\d);

\draw[->,thick] (0mm,\sizemid*1mm) arc(90:{90+\rotorangle-2}:\sizemid*1mm);

\draw[->,thick] ({\sizemid*0.9mm*cos(90+\rotorangle-10)},{\sizemid*0.9mm*sin(90+\rotorangle-10)}) arc({90+\rotorangle-10}:{90+\rotorangle+10}:\sizemid*0.9mm);

\node at ({\sizemid*1.1mm*cos(90+\rotorangle/2)},{\sizemid*1.1mm*sin(90+\rotorangle/2)}) {$\theta$};

\node at ({\sizemid*1mm*cos(90+\rotorangle+10)},{\sizemid*1mm*sin(90+\rotorangle+10)}) {$\omega$};

% second rotor drawing <<<
\begin{scope}[xshift = 90mm]

\node (O) at (0,0) {};

\node[draw, circle, fill=gray!75!white, thick, minimum size=68mm] (graybg) at (0,0) {}; % SUM CIRCLE
\node[draw, circle, fill=white,         thick, minimum size=60mm] (whitefg) at (0,0) {}; % SUM CIRCLE

\draw[->, gray,thick] (0,-40mm) -- (0,40mm) ;
\draw[->, gray,thick] (-40mm,0) -- (40mm,0) ;

\pgfmathsetmacro{\sizein}{35}
\pgfmathsetmacro{\sizeout}{50}
\pgfmathsetmacro{\sizemid}{(\sizein + \sizeout)/2}

\draw[thick] (0mm, \sizein*1mm + 5mm) -- (0mm, \sizeout*1mm);

\pgfmathsetlengthmacro{\a}{(\sizein*1mm + 5mm)*cos(\rotorangle + 90)}
\pgfmathsetlengthmacro{\b}{(\sizein*1mm + 5mm)*sin(\rotorangle + 90)}
\pgfmathsetlengthmacro{\c}{\sizeout*1mm*cos(\rotorangle + 90)}
\pgfmathsetlengthmacro{\d}{\sizeout*1mm*sin(\rotorangle + 90)}

\draw[thick] (\a,\b) -- (\c,\d);

\draw[->,thick] (0mm,\sizemid*1mm) arc(90:{90+\rotorangle-2}:\sizemid*1mm);

\draw[->,thick] ({\sizemid*1.1mm*cos(90+\rotorangle-10)},{\sizemid*1.1mm*sin(90+\rotorangle-10)}) arc({90+\rotorangle-10}:{90+\rotorangle+10}:\sizemid*0.9mm);

\node at ({\sizemid*1.1mm*cos(90+\rotorangle/2)},{\sizemid*1.1mm*sin(90+\rotorangle/2)}) {$\theta$};

\node at ({\sizemid*1.1mm*cos(90+\rotorangle+10)},{\sizemid*1.1mm*sin(90+\rotorangle+10)}) {$\omega$};


% rotated DQ <<<
\begin{scope}[rotate=\rotorangle]

	\path [name path=outercircle] (0,0) circle (28.5mm) ; % SUM CIRCLE
	\path [name path = leftline]  (-2.5mm,-30mm) -- (-2.5mm,30mm);
	\path [name path = rightline] ( 2.5mm,-30mm) -- ( 2.5mm,30mm);

	\path[name path = leftmostline]  (-10mm,10mm) -- (-10mm,30mm);
	\path[name path = rightmostline] ( 10mm,10mm) -- ( 10mm,30mm);

	\path [name intersections={of=outercircle and leftmostline, by={southwest_intersection}}];
	\path [name intersections={of=outercircle and rightmostline, by={southeast_intersection}}];

	\path[name path = leftmostline]  (-10mm, 10mm) -- (-10mm, 30mm);
	\path[name path = rightmostline] ( 10mm, 10mm) -- ( 10mm, 30mm);

	\path[name path = botsouthline] ( -12mm, 22mm) -- ( 12mm, 22mm);

	\path [name intersections={of=botsouthline and rightmostline, by={southeast_T}}];

	\path [name intersections={of=botsouthline and leftmostline, by={southwest_T}}];

	\path [name intersections={of=botsouthline and leftline, by={southwest_inner_T}}];
	\path [name intersections={of=botsouthline and rightline, by={southeast_inner_T}}];

		% Compute angles and radius using \path let
		\node (A) at (southwest_intersection) {};
		\node (B) at (southeast_intersection) {};
		\node (C) at (southwest_T) {};
		\node (D) at (southeast_T) {};
		\node (E) at (southwest_inner_T) {};
		\node (F) at (southeast_inner_T) {};

		% Get coordinates for calculations
		\path (O); \pgfgetlastxy{\Ox}{\Oy}
		\path (A); \pgfgetlastxy{\Ax}{\Ay}
		\path (B); \pgfgetlastxy{\Bx}{\By}

		% Compute angles
		\pgfmathsetmacro{\angleA}{atan2(\Ay - \Oy,\Ax - \Ox)}
		\pgfmathsetmacro{\angleB}{atan2(\By - \Oy,\Bx - \Ox)}

		% Compute radius (distance from O to A)
		\pgfmathsetmacro{\radius}{veclen(\Ax - \Ox,\Ay - \Oy)/28.45274}

		% Draw the arc centered at O from A to B
		\draw[thick] (E.center) -- (C.center) -- ([shift=(\angleA:\radius)]O.center) arc(\angleA:\angleB:\radius) -- (D.center) -- (F.center);

	% SAME PROCESS FOR BOTTOM T

	\path[name path = leftmostline]  (-10mm,-10mm) -- (-10mm,-30mm);
	\path[name path = rightmostline] ( 10mm,-10mm) -- ( 10mm,-30mm);

	\path [name intersections={of=outercircle and leftmostline, by={southwest_intersection}}];

	\path [name intersections={of=outercircle and rightmostline, by={southeast_intersection}}];

	\path[name path = leftmostline]  (-10mm,-10mm) -- (-10mm,-30mm);
	\path[name path = rightmostline] ( 10mm,-10mm) -- ( 10mm,-30mm);

	\path[name path = botsouthline] ( -12mm,-22mm) -- ( 12mm,-22mm);

	\path [name intersections={of=botsouthline and rightmostline, by={southeast_T}}];
	\path [name intersections={of=botsouthline and leftmostline, by={southwest_T}}];
	\path [name intersections={of=botsouthline and leftline, by={southwest_inner_T}}];
	\path [name intersections={of=botsouthline and rightline, by={southeast_inner_T}}];

	% Compute angles and radius using \path let
		\node (G) at (southwest_intersection) {};
		\node (H) at (southeast_intersection) {};
		\node (I) at (southwest_T) {};
		\node (J) at (southeast_T) {};
		\node (K) at (southwest_inner_T) {};
		\node (L) at (southeast_inner_T) {};

		% Get coordinates for calculations
		\path (O); \pgfgetlastxy{\Ox}{\Oy}
		\path (G); \pgfgetlastxy{\Ax}{\Ay}
		\path (H); \pgfgetlastxy{\Bx}{\By}

		% Compute angles
		\pgfmathsetmacro{\angleA}{atan2(\Ay - \Oy,\Ax - \Ox)}
		\pgfmathsetmacro{\angleB}{atan2(\By - \Oy,\Bx - \Ox)}

		% Compute radius (distance from O to A)
		\pgfmathsetmacro{\radius}{veclen(\Ax - \Ox,\Ay - \Oy)/28.45274}

		% Draw the arc centered at O from A to B
		\draw[thick] (F.center) -- (L.center) -- (J.center) -- ([shift=(\angleB:\radius)]O.center) arc(\angleB:\angleA:\radius) -- (I.center) -- (K.center) -- (E.center);

	    \fill[fill=white] (-2.35mm,-20mm) rectangle (2.35mm,20mm);

\end{scope} %>>>

\pgfmathsetmacro{\axissize}{40}

\pgfmathsetlengthmacro{\a}{\axissize*1mm*cos(\rotorangle - 90)}
\pgfmathsetlengthmacro{\b}{\axissize*1mm*sin(\rotorangle - 90)}
\pgfmathsetlengthmacro{\c}{\axissize*1mm*cos(\rotorangle + 90)}
\pgfmathsetlengthmacro{\d}{\axissize*1mm*sin(\rotorangle + 90)}

\draw[->,thick,stewartpink] (\a,\b) -- (\c,\d);
\node[stewartpink] at ([shift=({4mm,0})]\c,\d) {$Q$};

\pgfmathsetlengthmacro{\a}{\axissize*1mm*cos(\rotorangle + 180)}
\pgfmathsetlengthmacro{\b}{\axissize*1mm*sin(\rotorangle + 180)}
\pgfmathsetlengthmacro{\c}{\axissize*1mm*cos(\rotorangle)}
\pgfmathsetlengthmacro{\d}{\axissize*1mm*sin(\rotorangle)}

\draw[->,thick,stewartpink] (\a,\b) -- (\c,\d);
\node[stewartpink] at ([shift=({2mm,2mm})]\c,\d) {$D$};

\end{scope} %>>>

\end{tikzpicture}
}
\caption
[Schematic of a salient-pole synchronous machine with the rotating $DQ$ frame as conceived in \cite{Park1929}.]
{Schematic of a salient-pole synchronous machine with the rotating $DQ$ frame as conceived in \cite{Park1929}. The left schematic shows the $a$, $b$ and $c$ stator wirings; in blue the rotor circuit with the field voltage $v_F$ and field current $v_D$. Park then creates two virtual coils $d$ and $q$ that translate the stator effect onto the rotor, generating the direct-quadrature $DQ$ rotating frame as in the right schematic.}
	\label{fig:park_transform_rotor}
\end{figure} %>>>

	A somewhat generalized definition of Park's transformation is shown in definition \ref{def:dq0_transform}, where the linear transformation is defined as a rotational transform as a function of an arbitrary angle $\theta$.

\begin{definition}[Park transform]\label{def:park_transform} The Park Transform takes an argument angle $\theta$ and delivers the rotating linear transformation

\begin{equation}
\mathbf{T_P}\left(\theta\right) = 
\left[\begin{array}{ccc}
\phantom{-}\cos\left(\theta\right) & \sin\left(\theta\right) & 0 \\[5mm]
-\sin\left(\theta\right) & \cos\left(\theta\right) & 0 \\[5mm]
0 & 0 & 1
\end{array}\right]
\end{equation}

\end{definition}

	Originally, Park's $dq$ axes referred to a set of orthogonal axes rotating at the rotor speed $\omega_s$ of a synchronous machine, that is, using a transformation angle of $\theta = \omega_s t$ (see figure \ref{fig:park_transform_rotor}); this eliminated the varying inductances arisen from the reluctances in synchronous machine analyses. Researchers then used Park's idea  and explored different reference frames for the $dq$ axes; \cite{Krause1965} later showed that all of the diference reference frames were particular cases of the Park Transform, using some arbitrary reference frame; this justifies defining the Park transform as a transformation of an arbitrary angle.

%-------------------------------------------------
\section{The $dq0$ Transform and the Three-Phase Dynamic Phasor} %<<<2

	It becomes now obvious that The composition of the $\alpha\beta\gamma$ and the Park transforms onto a balanced three-phase signal yields a very useful result: the quantity is transformed into a pair of continuous but not oscillating signals — one might see the literature define these as ``DC signals''. This composition was later named the ``$dq0$ Transform''.

\begin{definition}[Clarke-Park or $dq0$ transform]\label{def:dq0_transform} The Clarke-Park or $dq0$ Transform $\mathbf{T}$ is a rotating linear transformation, defined as the composition of the Clarke Transform followed by the Park Transform applied at an angle $\theta$:

\begin{equation} \mathbf{T}_\theta = \mathbf{T_C}\mathbf{T_P}\left(\theta\right) = \sqrt{\dfrac{2}{3}}
\left[\begin{array}{ccc}
\phantom{-}\cos\left(\theta\right) & \phantom{-}\cos\left(\theta - \dfrac{2\pi}{3}\right) & \phantom{-}\cos\left(\theta + \dfrac{2\pi}{3}\right) \\[5mm]
-\sin\left(\theta\right) & -\sin\left(\theta - \dfrac{2\pi}{3}\right) & -\sin\left(\theta + \dfrac{2\pi}{3}\right) \\[5mm]
\dfrac{1}{\sqrt{2}} & \dfrac{1}{\sqrt{2}} & \dfrac{1}{\sqrt{2}} 
\end{array}\right]
 \end{equation}

	\noindent and the inverse transform is given by

\begin{equation} \mathbf{T}^{-1}_\theta = \left(\mathbf{T}_\theta\right)^\transpose = 
\sqrt{\dfrac{2}{3}}\left[\begin{array}{ccc}
\cos\left(\theta\right)                   & -\sin\left(\theta\right)                   & \dfrac{1}{\sqrt{2}} \\[5mm]
\cos\left(\theta - \dfrac{2\pi}{3}\right) & -\sin\left(\theta - \dfrac{2\pi}{3}\right) & \dfrac{1}{\sqrt{2}} \\[5mm] 
\cos\left(\theta + \dfrac{2\pi}{3}\right) & -\sin\left(\theta + \dfrac{2\pi}{3}\right) & \dfrac{1}{\sqrt{2}}
\end{array}\right]
 \end{equation}
\end{definition}

	Similarly to figure \ref{fig:pll_example} one can also devise a block model of this transform In block schematics of control systems, the $dq0$ transform is generally represented as depicted in figure \ref{fig:dq0_block_modelling}. The block takes three arguments: phases $a$, $b$ and $c$ of an input quantity and an angle $\theta$, and performs the transformation as per definition \ref{def:dq0_transform} to yield the three $d$, $q$ and $0$ components. 

% PLL SUBSTYSTEM <<<
\begin{figure} 
\centering
\begin{tikzpicture}[scale=1,>={Stealth[inset=0mm,length=1.5mm,angle'=50]}]

\node [draw, minimum width=2cm, very thick, minimum height=2cm, left=0] (tab_block) {};

\draw (tab_block.south west) -- (tab_block.north east);

\node at ([shift=({{ 2cm*sqrt(2)/4},{-2cm*sqrt(2)/4}})]tab_block.north west) {\large $abc$};
\node at ([shift=({{-2cm*sqrt(2)/4},{ 2cm*sqrt(2)/4}})]tab_block.south east) {\large $\alpha\beta\gamma$};


\node at ([shift=({-15mm, 7mm})]tab_block.west) (signalinputa) {$x_a(t)$};
\node at ([shift=({-15mm, 0})]tab_block.west) (signalinputb) {$x_b(t)$};
\node at ([shift=({-15mm,-7mm})]tab_block.west) (signalinputc) {$x_c(t)$};

\draw[->] (signalinputa.east) -- ([shift=({-1mm,0})] signalinputa -| tab_block.west);
\draw[->] (signalinputb.east) -- ([shift=({-1mm,0})] signalinputb -| tab_block.west);
\draw[->] (signalinputc.east) -- ([shift=({-1mm,0})] signalinputc -| tab_block.west);

\node [draw, minimum width=2cm, very thick, minimum height=2cm, right=2cm of tab_block] (abdq_block) {};
\draw (abdq_block.south west) -- (abdq_block.north east);
\node at ([shift=({{ 2cm*sqrt(2)/4},{-2cm*sqrt(2)/4}})]abdq_block.north west) {\large $\alpha\beta\gamma$};
\node at ([shift=({{-2cm*sqrt(2)/4},{ 2cm*sqrt(2)/4}})]abdq_block.south east) {\large $dq$};

\draw[->] ([shift=({0, 7mm})]tab_block.east) -- ([shift=({-1mm, 7mm})]abdq_block.west) node[midway, above] {$x_\alpha(t)$};
\draw[->]                   (tab_block.east) -- ([shift=({-1mm, 0mm})]abdq_block.west) node[midway, above] {$x_\beta(t)$};
\draw[->] ([shift=({0,-7mm})]tab_block.east) -- ([shift=({-1mm,-7mm})]abdq_block.west) node[midway, above] {$x_\gamma(t)$};

\draw[->] ([shift=({0, 7mm})]abdq_block.east) -- ([shift=({10mm, 7mm})]abdq_block.east) node[right] {$x_d(t)$};
\draw[->] ([shift=({0, 0mm})]abdq_block.east) -- ([shift=({10mm, 0mm})]abdq_block.east) node[right] {$x_q(t)$};
\draw[->] ([shift=({0,-7mm})]abdq_block.east) -- ([shift=({10mm,-7mm})]abdq_block.east) node[right] {$x_0(t)$};

\node [draw, minimum width=1cm, very thick, minimum height=1cm, below=1cm of abdq_block] (omega_integrator) {$\int$};

\node [below=1cm of omega_integrator.south] (omega_signal) {$\omega(t)$};

\draw[->] (omega_signal.north) -- ([shift=({0,-1mm})]omega_integrator.south);
\draw[->] (omega_integrator.north) -- ([shift=({0,-1mm})]abdq_block.south) node [midway, right] {$\psi(t)$};

\end{tikzpicture}
\caption{Block model of the three-phase dq transform.}
\label{fig:dq0_block_modelling}
\end{figure}
%>>>

\begin{theorem}[$dq0$ Transform of balanced three-phase quantities]\label{theo:dq0_balanced_3p} %<<<
Let $\mathbf{x}$ be a balanced three-phase signal with modulus $m(t)$ and angle $\theta(t)$ such with apparent frequency $\omega(t)$ and apparent phase angle $\phi(t)$. Then adopt

\begin{equation} \psi(t) = \int_0^t \omega(x)dx \end{equation}

	as the angle of rotation of the $dq0$ transform $\mathbf{T}_\psi$. Then

\begin{equation} \mathbf{T}_\psi \mathbf{x} = \mathbf{T}_{\psi} \left(m\left[
	\begin{array}{c}
		\cos\left(\theta \right) \\[5mm]
		\cos\left(\theta  - \dfrac{2\pi}{3}\right) \\[5mm]
		\cos\left(\theta  + \dfrac{2\pi}{3}\right)
	\end{array}\right] \right) = 
m\sqrt{\dfrac{3}{2}}\left[
	\begin{array}{c}
		\cos\left(\phi\right) \\[5mm]
		\sin\left(\phi\right) \\[5mm]
		0
	\end{array}\right] 
\end{equation}

\end{theorem}

\textbf{Proof:} by direct calculation. First consider the Park transform of an arbitrary angle $\alpha$. Then
\begin{align}
\sqrt{\dfrac{3}{2}} x_d
	&= m\left[\raisebox{5mm}{}\cos\left(\theta\right)\cos\left(\alpha\right) + \right. \nonumber\\[3mm]
	&\hspace{1cm} +\cos\left(\theta - \dfrac{2\pi}{3}\right)\cos\left(\alpha - \dfrac{2\pi}{3}\right)  + \nonumber\\[3mm]
	&\left. \hspace{2cm} + \cos\left(\theta + \dfrac{2\pi}{3}\right)\cos\left(\alpha + \dfrac{2\pi}{3}\right) \right]
\end{align}

	From the Prostaph\ae resis Formulas, $\cos\left(a\right)\cos\left(b\right) = \frac{1}{2}\left[\cos\left(a-b\right) + \cos\left(a+b\right)\right]$:

\begin{align}
\sqrt{\dfrac{3}{2}} x_d
	=& \dfrac{m}{2}\left[\raisebox{5mm}{} \cos\left(\theta + \alpha\right) + \cos\left(\theta - \alpha\right) + \right. \nonumber\\[3mm]
	&\hspace{1cm} \cos\left(\theta + \alpha - \dfrac{4\pi}{3}\right) + \cos\left(\theta - \alpha\right) + \nonumber\\[3mm]
	&\hspace{2cm} \left. \cos\left(\theta + \alpha + \dfrac{4\pi}{3}\right) + \cos\left(\theta - \alpha\right)\right] = \nonumber\\[3mm]
	=& \dfrac{3}{2} m\cos\left(\theta - \alpha\right) \nonumber\\[3mm]
	x_d =& \sqrt{\dfrac{3}{2}}m\cos\left(\theta - \alpha\right)
\end{align}

	Much the same way, $\sin\left(a\right)\cos\left(b\right) = \frac{1}{2}\left[\sin\left(a+b\right) + \sin\left(a-b\right)\right]$:

\begin{align}
\sqrt{\dfrac{3}{2}}x_q &= \dfrac{1}{2}m \left[\raisebox{5mm}{} \cos\left(\theta\right)\sin\left(\alpha\right) \right.+ \nonumber\\[3mm]
	&\hspace{1cm} + \cos\left(\theta - \dfrac{2\pi}{3}\right)\sin\left(\alpha - \dfrac{2\pi}{3}\right) + \nonumber\\[3mm]
	&\hspace{2cm}\left. +\cos\left(\theta + \dfrac{2\pi}{3}\right)\sin\left(\alpha + \dfrac{2\pi}{3}\right)\right] =\nonumber\\[3mm]
	=& \dfrac{1}{2}m\left[\raisebox{5mm}{} \sin\left(\theta + \alpha\right) + \sin\left(\theta - \alpha\right) + \right.\nonumber\\[3mm]
	&\hspace{1cm} + \sin\left(\theta + \alpha - \dfrac{4\pi}{3}\right) + \sin\left(\theta - \alpha\right) + \nonumber\\[3mm]
	&\hspace{2cm} + \left. \sin\left(\theta + \alpha + \dfrac{4\pi}{3}\right) + \sin\left(\theta - \alpha\right) \right] = \nonumber\\[3mm]
	=& \dfrac{3}{2} m\sin\left(\theta - \alpha\right) \nonumber\\[3mm]
	x_q =& \sqrt{\dfrac{3}{2}}m\sin\left(\theta - \alpha\right)
\end{align}

	And 

\begin{equation} \sqrt{\dfrac{3}{2}}\mathbf{x}^{0} = m\left[\dfrac{1}{\sqrt{2}}\cos\left(\theta + \phi\right) + \dfrac{1}{\sqrt{2}}\cos\left(\omega t + \phi - \dfrac{2\pi}{3}\right) + \dfrac{1}{\sqrt{2}}\cos\left(\omega t + \phi + \dfrac{2\pi}{3}\right)\right] = 0 \end{equation}

	Finally, adopting the arbitrary angle $\alpha$ as  $\psi(t) = \displaystyle \int_{t_0}^t \omega(x)dx$ yields

\begin{align}
\left\{ \begin{array}{rl}
	x_d &= \sqrt{\dfrac{3}{2}}m(t)\cos\left[\theta(t) - \psi(t)\right] = \sqrt{\dfrac{3}{2}}m(t)\cos\left[\phi(t)\right] \\[5mm]
	x_d &= \sqrt{\dfrac{3}{2}}m(t)\sin\left[\theta(t) - \psi(t)\right] = \sqrt{\dfrac{3}{2}}m(t)\sin\left[\phi(t)\right] \\[5mm]
	x_0 &= 0,
\end{array}\right.
\end{align}


	which is the result wanted. \hfill $\blacksquare$

\vspace{5mm}
\hrule
\vspace{5mm}
%>>>

	Theorem \ref{theo:dq0_balanced_3p} then shows that a balanced three-phase quantity is equivalent to a two-dimensional quantity $x_d,x_q$ plus a indentically null zero-sequence component. Therefore, if we use the same complexification operator $\rho$ as theorem \ref{theo:rho_diff_inf}, we can disregard the zero-sequence component without loss of information to yield a complex number $X(t) = x_d + jx_q$.

\begin{definition}[Three-phase Dynamic Phasor Transform] \label{theo:clarke_park_phasor_transform}%<<<
	Let $\mathbf{x}$ be a three-phase signal with modulus $m(t)$, apparent frequency $\omega(t)$ and apparent phase $\phi(t)$. Then define the Three-Phase Dynamic Phasor Transform as 

\begin{equation}
	\mathbf{P_{3\phi}^{\omega}}\left[\cdot\right] \vcentcolon \left\{\begin{array}{rcl}
	\left[\mathbb{R}\to \mathbb{R}^3\right] &\to& \left[\mathbb{R}\to \mathbb{C}\right]\\[3mm]
	\mathbf{x}(t) &\mapsto& X\left(t\right)
\end{array}\right. ,
\end{equation}
\end{definition}
%>>>

	It is important to note that the definition of $\mathbf{P_{3\phi}}$ states that the domain is the set of three-dimensional real functions, and not necessarily the balanced ones. Naturally, the matrix $\mathbf{T}_{\psi}$ can be applied to any three-dimensional vectors; in the case of balanced ones, the zero-sequence component will be null, hence $\mathbf{P_{3\phi}}\left[\mathbf{x}\right] = X(t)$ completely reconstructs $\mathbf{x}(t)$. In other words, if $x_0(t)$ is ignored, $\mathbf{P_{3\phi}}$ is invertible.

	Here, ``ignored'' means ``understood''. In practice, if a certain signal $x_0(t)$ is picked, then the image of $\mathbf{P_{3\phi}}$ through the entire $\left[\mathbb{R}\to\mathbb{C}\right]$ forms an equivalence class. More specifically, the image of $\mathbf{T}_\psi$ is homeomorphic to the quotient group $\left[\mathbb{R}\to\mathbb{C}\right]/\left[\mathbb{R}\to\mathbb{R}\right]$, that is, any signal produced by the transformation can be described by a complex function (its Dynamic Phasor) and a real function (its zero-sequence component). This means that if two signals have the same zero-sequence signal, then their corresponding Dynamic Phasors are unique; hence the idea of a ``ignored'' zero-sequence signal. Most of the times, this ``understood'' signal is the null function, because the most studied subgroup of three-phase signals is perhaps the one that contains balanced signals.

	Also importantly, both the transform and its inverse are linear due to the linearity and invertibility of the operations involved; therefore, combining the Clarke Transform, the Park Transform and the complexification functional we obtain $\mathbf{P_{3\phi}}$, as shown in Figure \ref{fig:3p_dq0_block_modelling}.

% PLL SUBSTYSTEM <<<
\begin{figure} 
\centering
\scalebox{0.75}{
\begin{tikzpicture}[scale=1,>={Stealth[inset=0mm,length=1.5mm,angle'=50]}]

\node [draw, minimum width=2cm, very thick, minimum height=2cm, left=0] (tab_block) {};

\draw (tab_block.south west) -- (tab_block.north east);

\node at ([shift=({{ 2cm*sqrt(2)/4},{-2cm*sqrt(2)/4}})]tab_block.north west) {\large $abc$};
\node at ([shift=({{-2cm*sqrt(2)/4},{ 2cm*sqrt(2)/4}})]tab_block.south east) {\large $\alpha\beta\gamma$};


\node at ([shift=({-15mm, 7mm})]tab_block.west) (signalinputa) {$x_a(t)$};
\node at ([shift=({-15mm, 0})]tab_block.west) (signalinputb) {$x_b(t)$};
\node at ([shift=({-15mm,-7mm})]tab_block.west) (signalinputc) {$x_c(t)$};

\draw[->] (signalinputa.east) -- ([shift=({-1mm,0})] signalinputa -| tab_block.west);
\draw[->] (signalinputb.east) -- ([shift=({-1mm,0})] signalinputb -| tab_block.west);
\draw[->] (signalinputc.east) -- ([shift=({-1mm,0})] signalinputc -| tab_block.west);

\node [draw, minimum width=2cm, very thick, minimum height=2cm, right=2cm of tab_block] (abdq_block) {};

\node [draw, rounded corners, stewartblue, very  thick, dashed, minimum width=11cm, minimum height=3cm] at ([shift=({-1mm,0})]abdq_block.center)  (invol) {};

\draw (abdq_block.south west) -- (abdq_block.north east);
\node at ([shift=({{ 2cm*sqrt(2)/4},{-2cm*sqrt(2)/4}})]abdq_block.north west) {\large $\alpha\beta\gamma$};
\node at ([shift=({{-2cm*sqrt(2)/4},{ 2cm*sqrt(2)/4}})]abdq_block.south east) {\large $dq$};

\draw[->] ([shift=({0, 7mm})]tab_block.east) -- ([shift=({-1mm, 7mm})]abdq_block.west) node[midway, above] {$x_\alpha(t)$};
\draw[->]                   (tab_block.east) -- ([shift=({-1mm, 0mm})]abdq_block.west) node[midway, above] {$x_\beta(t)$};
\draw[->] ([shift=({0,-7mm})]tab_block.east) -- ([shift=({-1mm,-7mm})]abdq_block.west) node[midway, above] {$x_\gamma(t)$};

\node [draw, minimum width=1cm, very thick, minimum height=1cm, below=1cm of abdq_block] (omega_integrator) {$\int$};

\node [below=1cm of omega_integrator.south] (omega_signal) {$\omega(t)$};
\draw[->] (omega_signal.north) -- ([shift=({0,-1mm})]omega_integrator.south);
\draw[->] (omega_integrator.north) -- ([shift=({0,-1mm})]abdq_block.south) node [midway, right] {$\psi(t)$};

\node [draw, minimum width=15mm, very thick, minimum height=15mm] (rho_block) at ([shift=({30mm,3.5mm})]abdq_block.east) {$\rho$};

\node at ([shift=({0, 7mm})] abdq_block.east) (xdout) {};
\node at ([shift=({0, 0mm})] abdq_block.east) (xqout) {};

\draw[->] (xdout.center) -- ([shift=({-1mm, 0mm})] xdout -| rho_block.west) node[above,midway] {$x_d(t)$};
\draw[->] (xqout.center) -- ([shift=({-1mm, 0mm})] xqout -| rho_block.west) node[above,midway] {$x_q(t)$};

\draw[->] (rho_block.east) -- ([shift=({15mm,0mm})]rho_block.east) node[right] (xphasorout) {$X(t)$};

\node at ([shift=({0,-7mm})]abdq_block.east) (x0out) {};
\draw[->] (x0out.center) -- (x0out -| xphasorout.west) node[right] {$x_0(t)$};

\node [draw, stewartblue, minimum width=2cm, very thick, minimum height=2cm, below=5cm of abdq_block] (p3p_block) {$\mathbf{P_{3\phi}}$};
\node at ([shift=({-15mm, 7mm})]p3p_block.west) (signalinputa_p3p) {$x_a(t)$};
\node at ([shift=({-15mm, 0})]  p3p_block.west) (signalinputb_p3p) {$x_b(t)$};
\node at ([shift=({-15mm,-7mm})]p3p_block.west) (signalinputc_p3p) {$x_c(t)$};

\draw[->] (signalinputa_p3p.east) -- ([shift=({-1mm,0})] signalinputa_p3p -| p3p_block.west);
\draw[->] (signalinputb_p3p.east) -- ([shift=({-1mm,0})] signalinputb_p3p -| p3p_block.west);
\draw[->] (signalinputc_p3p.east) -- ([shift=({-1mm,0})] signalinputc_p3p -| p3p_block.west);

\node [draw, minimum width=1cm, very thick, minimum height=1cm, below=1cm of p3p_block] (omega_integrator_p3p) {$\int$};
\node [below=1cm of omega_integrator_p3p.south] (omega_signal_p3p) {$\omega(t)$};
\draw[->] (omega_signal_p3p.north) -- ([shift=({0,-1mm})]omega_integrator_p3p.south);
\draw[->] (omega_integrator_p3p.north) -- ([shift=({0,-1mm})]p3p_block.south) node [midway, right] {$\psi(t)$};

\draw[->] ([shift=({0, 3.33mm})]p3p_block.east) -- ([shift=({10mm, 3.33mm})]p3p_block.east) node[right] {$X(t)$};
\draw[->] ([shift=({0,-3.33mm})]p3p_block.east) -- ([shift=({10mm,-3.33mm})]p3p_block.east) node[right] {$x_0(t)$};

\draw[dashed,stewartblue] (invol.south east) -- (p3p_block.north east);
\draw[dashed,stewartblue] (invol.south west) -- (p3p_block.north west);

\end{tikzpicture}
}
\caption{Three-phase Dynamic Phasor Transform block model.}
\label{fig:3p_dq0_block_modelling}
\end{figure}
%>>>

%------------------------------------------------
\subsection{Three-Phase Dynamic Phasors as representations of solutions of ODEs} %<<<2

	Having constructed the Three-Phase Dynamic Phasor Transform, we now want to prove that this transform is able to translate linear systems in time to phasorial systems in the complex space, that is, we want to prove the three-phase converse of theorem \ref{theo:1p_ode_solution}. In order to do this, we use lemmas \ref{theo:dq_1p_diff} and \ref{lemma:1p_t_ndifftminus_product} and adapt them to a three-phase scenario. These two lemmas are then applied to theorem \ref{theo:3p_ode_solution} to yield the required result.

\begin{lemma}[n-th order time differentiation of $dq0$ transformed 3$\phi$ quantities]\label{theo:dq0_3p_diff}%<<<
Let $n\in\mathbb{N}^*$, $\mathbf{x}$ be a 3$\phi$ quantity, $\mathbf{T}_\theta$ the $dq0$ Transform operator where $\theta(t)$ is $C^n$-class, and $\mathbf{y} = \mathbf{T}_\theta\mathbf{x}$. Then

\begin{equation} \dfrac{d^n \mathbf{y}}{dt^n} = \dfrac{d^n\left(\mathbf{T}_\theta \mathbf{x}\right)}{dt^n} = \sum\limits_{k=0}^{n} {n\choose k} \left(\dfrac{d^{k} \mathbf{T}_\theta}{dt^k}\right) \left(\dfrac{d^{\left(n-k\right)} \mathbf{x}}{dt^{\left(n-k\right)}}\right), \end{equation}

	and

\begin{equation} \dfrac{d^n\mathbf{x}}{dt^n} = \dfrac{d^n\left(\mathbf{T}^{-1}_\theta \mathbf{x}\right)}{dt^n} = \sum\limits_{k=0}^{n} {n\choose k} \left(\dfrac{d^{k} \mathbf{T}^{-1}_\theta}{dt^k}\right) \left(\dfrac{d^{\left(n-k\right)} \mathbf{y}}{dt^{\left(n-k\right)}}\right) \end{equation}

	Particularly for $n=1$,

\begin{equation} \dfrac{d\mathbf{x}}{dt} = \dfrac{d}{dt} \left(\mathbf{T}^{-1}_\theta \mathbf{y}\right) = \mathbf{T}^{-1}_\theta \dfrac{d\mathbf{y}}{dt} + \dfrac{d\mathbf{T}^{-1}_\theta}{dt} \mathbf{y}, \end{equation}

and

\begin{equation} \dfrac{d\mathbf{y}}{dt} = \dfrac{d}{dt} \left(\mathbf{T}_\theta \mathbf{x}\right) = \mathbf{T}_\theta \dfrac{d\mathbf{x}}{dt} + \dfrac{d\mathbf{T}_\theta}{dt} \mathbf{x}, \end{equation}
 
	where

\begin{equation}
        \dfrac{d\mathbf{T}^{-1}_{\theta} }{dt} =
\sqrt{\dfrac{3}{2}}\dfrac{d\theta}{dt}
\left[\begin{array}{ccc}
         -\sin\left(\theta\right)                    & \cos\left(\theta\right)                    & 0 \\[5mm]
         -\sin\left(\theta - \dfrac{2\pi}{3} \right) & \cos\left(\theta - \dfrac{2\pi}{3} \right) & 0 \\[5mm]
         -\sin\left(\theta + \dfrac{2\pi}{3} \right) & \cos\left(\theta + \dfrac{2\pi}{3} \right) & 0 
\end{array}\right]
\end{equation}

	and

\begin{equation}
        \dfrac{d\mathbf{T}_\theta }{dt} =
\sqrt{\dfrac{2}{3}}\dfrac{d\theta}{dt}
\left[\begin{array}{ccc}
         -\sin\left(\theta\right)                    & -\sin\left(\theta - \dfrac{2\pi}{3} \right) & -\sin\left(\theta + \dfrac{2\pi}{3}\right) \\[5mm]
          \cos\left(\theta\right)                    &  \cos\left(\theta - \dfrac{2\pi}{3} \right) &  \cos\left(\theta + \dfrac{2\pi}{3}\right) \\[5mm]
          0                                          &  0                                          & 0
\end{array}\right]
\end{equation}
\end{lemma}
\textbf{Proof:} identical to lemma \ref{theo:dq0_3p_diff}.
\vspace{5mm}
\hrule
\vspace{5mm}
%>>>

\begin{lemma} \label{lemma:t_ndifftminus_product}%<<<

	Let $n\geq 1$ be a natural and let $\mathbf{T}_\theta$ and $\mathbf{T}_\theta$ denote the Clarke-Park Transform and its inverse of an angle $\theta$, where $\theta$ is n-th order differentiable. Then

\begin{equation} \mathbf{T}_\theta \dfrac{d^n\mathbf{T}^{-1}_\theta}{dt^n} = \sum\limits_{k=1}^n \mathbf{S}_k B_{\left(n,k\right)}\left(\dot{\theta},\ddot{\theta},...,\theta^{(n-k+1)}\right), \end{equation}

	where $B_{\left(n,k\right)}$ is the incomplete exponential Bell Polynomial and

\begin{equation}
\mathbf{S}_k = 
\left[\begin{array}{ccc}
 \cos\left( \dfrac{k\pi}{2}\right) &  -\sin\left(\dfrac{k\pi}{2}\right) & 0\\[5mm]
 \sin\left( \dfrac{k\pi}{2}\right) &  \phantom{-}\cos\left(\dfrac{k\pi}{2}\right) & 0\\[5mm]
0 & 0 & 0
\end{array}\right] \text{, for } k\geq 0 \text{ and } \mathbf{S}_0 = \mathbf{I_3}
\end{equation}

	Notably, $\mathbf{S}_k$ can be written in terms of the matrices $\mathbf{G}_k$ of lemma \ref{lemma:1p_t_ndifftminus_product} as

\begin{equation}
\mathbf{S}_k = 
\left[\begin{array}{ccc} \\[-1mm]
\multicolumn{2}{c}{\left[\mathbf{G}_k\right]} & 0\\[5mm]
0 & 0 & 0
\end{array}\right], \text{ for } k\geq 1 \text{ and }
%
\mathbf{S}_0 = 
\left[\begin{array}{ccc} \\[-1mm]
\multicolumn{2}{c}{\left[\mathbf{G}_0\right]} & 0\\[5mm]
0 & 0 & 1
\end{array}\right]. \label{eq:sk_gk_equiv}
\end{equation}

	Particularly for $n=1$,

\begin{equation} \mathbf{T}_\theta\dfrac{d\mathbf{T}_\theta^{-1}}{dt} = \dfrac{d\theta}{dt} \left[\begin{array}{ccc}    0 & -1 & 0 \\[5mm] 1 & 0  & 0 \\[5mm]  0 & 0  & 0 \end{array}\right] \end{equation}

\end{lemma}
\textbf{Proof:} the first-order case can be obtained by direct computation as

\begin{align}
\mathbf{T}_\theta\dfrac{d\mathbf{T}^{-1}_\theta}{dt} &= \dot{\theta}
\left[\begin{array}{ccc}
\phantom{-}\cos\left(\theta\right) & \phantom{-}\cos\left(\theta - \dfrac{2\pi}{3}\right) & \phantom{-}\cos\left(\theta + \dfrac{2\pi}{3}\right) \\[5mm]
-\sin\left(\theta\right) & -\sin\left(\theta - \dfrac{2\pi}{3}\right) & -\sin\left(\theta + \dfrac{2\pi}{3}\right) \\[5mm]
\dfrac{1}{\sqrt{2}} & \dfrac{1}{\sqrt{2}} & \dfrac{1}{\sqrt{2}} 
\end{array}\right]
\left[\begin{array}{ccc}
-\sin\left(\theta\right)                   & -\cos\left(\theta\right)                   & 0 \\[5mm]
-\sin\left(\theta - \dfrac{2\pi}{3}\right) & -\cos\left(\theta - \dfrac{2\pi}{3}\right) & 0 \\[5mm] 
-\sin\left(\theta + \dfrac{2\pi}{3}\right) & -\cos\left(\theta + \dfrac{2\pi}{3}\right) & 0 
\end{array}\right] \nonumber\\[5mm]
%
&=\dot{\theta}\left[\begin{array}{ccc}
0 & -1 & 0 \\[5mm]
1 &  0 & 0 \\[5mm]
0 &  0 & 0
\end{array}\right]
\end{align}

	For an arbitrary order $n \geq 1$, one needs to use the Faà Di Bruno's formula \pcite{DiBruno1855} for the n-th order Chain Rule. The formula states that, for two single-variable n-th order differentiable functions $f$ and $g$, the chain rule is given by

\begin{equation} \dfrac{d^n}{dx^n} f\left(g\left(x\right)\right)= \sum\limits_{k=0}^n f^{\left(k\right)}\left(g\left(x\right)\right) B_{\left(n,k\right)}\left(g'\left(x\right),g''\left(x\right),...,g^{\left(n-k+1\right)}\left(x\right)\right), \end{equation}

	where the $B_{\left(n,k\right)}$ are the incomplete exponential Bell Polynomials. Consider $t^{-1}_{\left(i,j\right)}$ as the $i,j$ element of $\mathbf{T}^{-1}$. Then	

\begin{equation} \dfrac{d^n}{dt^n} t^{-1}_{\left(i,j\right)} \left(\theta\left(t\right)\right)= \sum\limits_{k=0}^n \dfrac{d^k t^{-1}_{\left(i,j\right)}\left(\theta\right)}{d\theta^k} B_{\left(n,k\right)}\left(\dot{\theta},\ddot{\theta},...,\theta^{(n-k+1)}\right). \end{equation}

	But because the indexes $n$ and $k$ are not related to $i$ and $j$, 

\footnotesize
\begin{gather} \dfrac{d^n\mathbf{T}^{-1}_\theta}{dt^n} = \nonumber\\[5mm]
	\left[\begin{array}{ccc}
\sum\limits_{k=0}^n \dfrac{d^k t^{-1}_{\left(1,1\right)}}{d\theta^k} B_{\left(n,k\right)}\left(\dot{\theta},\ddot{\theta},...,\theta^{(n-k+1)}\right) & \sum\limits_{k=0}^n \dfrac{d^k t^{-1}_{\left(1,2\right)}}{d\theta^k} B_{\left(n,k\right)}\left(\dot{\theta},\ddot{\theta},...,\theta^{(n-k+1)}\right) & \sum\limits_{k=0}^n \dfrac{d^k t^{-1}_{\left(1,3\right)}}{d\theta^k} B_{\left(n,k\right)}\left(\dot{\theta},\ddot{\theta},...,\theta^{(n-k+1)}\right) \\[5mm]
\sum\limits_{k=0}^n \dfrac{d^k t^{-1}_{\left(2,1\right)}}{d\theta^k} B_{\left(n,k\right)}\left(\dot{\theta},\ddot{\theta},...,\theta^{(n-k+1)}\right) & \sum\limits_{k=0}^n \dfrac{d^k t^{-1}_{\left(2,2\right)}}{d\theta^k} B_{\left(n,k\right)}\left(\dot{\theta},\ddot{\theta},...,\theta^{(n-k+1)}\right) & \sum\limits_{k=0}^n \dfrac{d^k t^{-1}_{\left(2,3\right)}}{d\theta^k} B_{\left(n,k\right)}\left(\dot{\theta},\ddot{\theta},...,\theta^{(n-k+1)}\right) \\[5mm]
\sum\limits_{k=0}^n \dfrac{d^k t^{-1}_{\left(3,1\right)}}{d\theta^k} B_{\left(n,k\right)}\left(\dot{\theta},\ddot{\theta},...,\theta^{(n-k+1)}\right) & \sum\limits_{k=0}^n \dfrac{d^k t^{-1}_{\left(3,2\right)}}{d\theta^k} B_{\left(n,k\right)}\left(\dot{\theta},\ddot{\theta},...,\theta^{(n-k+1)}\right) & \sum\limits_{k=0}^n \dfrac{d^k t^{-1}_{\left(3,3\right)}}{d\theta^k} B_{\left(n,k\right)}\left(\dot{\theta},\ddot{\theta},...,\theta^{(n-k+1)}\right) 
\end{array}\right] = \nonumber\\[5mm]
%
	= \sum\limits_{k=0}^n  B_{\left(n,k\right)}\left(\dot{\theta},\ddot{\theta},...,\theta^{(n-k+1)}\right)\left[\begin{array}{ccc}
\dfrac{d^k t^{-1}_{\left(1,1\right)}}{d\theta^k} & \dfrac{d^k t^{-1}_{\left(1,2\right)}}{d\theta^k} & \dfrac{d^k t^{-1}_{\left(1,3\right)}}{d\theta^k} \\[5mm]
\dfrac{d^k t^{-1}_{\left(2,1\right)}}{d\theta^k} & \dfrac{d^k t^{-1}_{\left(2,2\right)}}{d\theta^k} & \dfrac{d^k t^{-1}_{\left(2,3\right)}}{d\theta^k} \\[5mm]
\dfrac{d^k t^{-1}_{\left(3,1\right)}}{d\theta^k} & \dfrac{d^k t^{-1}_{\left(3,2\right)}}{d\theta^k} & \dfrac{d^k t^{-1}_{\left(3,3\right)}}{d\theta^k} 
\end{array}\right]
\end{gather}
\normalsize

	Which in matrix form means

\begin{equation} \dfrac{d^n\mathbf{T}^{-1}_\theta}{dt^n} = \sum\limits_{k=0}^n \dfrac{d^k\mathbf{T}^{-1}_\theta}{d\theta^k} B_{\left(n,k\right)}\left(\dot{\theta},\ddot{\theta},...,\theta^{(n-k+1)}\right). \end{equation}

	But knowing that

\begin{equation}
\left\{\begin{array}{l}
\dfrac{d^n \cos\left(\theta\right)}{d\theta^n} = \cos\left(\theta + \dfrac{n\pi}{2}\right) \\[5mm]
\dfrac{d^n \sin\left(\theta\right)}{d\theta^n} = \sin\left(\theta + \dfrac{n\pi}{2}\right)
\end{array}\right. .
\end{equation}

	Here we must remove the case $k=0$ because, for $k\geq 1$, the third column of $d^k\mathbf{T}^{-1}_\theta/d\theta^k$ is zero due to the differentiated constants, but this does not happen at $k = 0$. In this case, $d^0\mathbf{T}^{-1}_\theta/d\theta^0 = \mathbf{T}^{-1}$ and $B_{(n,0)} = 1$ for $n=0$ and $B_{(n,0)} = 0$ if else. For $n\geq 1$,

\begin{equation} \dfrac{d^k \mathbf{T}^{-1}_\theta}{d\theta^k} =  \mathbf{K}_{\left(\theta + \frac{k\pi}{2}\right)}.\end{equation}

	where $\mathbf{K}$ is equal to $\mathbf{T}^{-1}$ but with a null third column because for $n\geq 1$ the third column is composed of differentiated constants. Therefore

\begin{equation} \dfrac{d^n\mathbf{T}^{-1}_\theta}{dt} = \sum\limits_{k=0}^n \mathbf{K}_{\left(\theta + \frac{k\pi}{2}\right)} B_{\left(n,k\right)}\left(\dot{\theta},\ddot{\theta},...,\theta^{(n-k+1)}\right) \end{equation}

	Now calculate the matrix multiplication:

\small
\begin{align}
\mathbf{T}\mathbf{K}_{\left(\theta + \frac{k\pi}{2}\right)} = 
\left[\begin{array}{ccc}
\cos\left(\theta\right) & \cos\left(\theta - \dfrac{2\pi}{3}\right) & \cos\left(\theta + \dfrac{2\pi}{3}\right) \\[5mm]
\sin\left(\theta\right) & \sin\left(\theta - \dfrac{2\pi}{3}\right) & \sin\left(\theta + \dfrac{2\pi}{3}\right) \\[5mm]
\dfrac{1}{\sqrt{2}} & \dfrac{1}{\sqrt{2}} & \dfrac{1}{\sqrt{2}} 
\end{array}\right]
\left[\begin{array}{ccc}
\cos\left(\theta + \dfrac{k\pi}{2}\right)                  & \sin\left(\theta + \dfrac{k\pi}{2}\right)                   & 0 \\[5mm]
\cos\left(\theta + \dfrac{k\pi}{2}- \dfrac{2\pi}{3}\right) & \sin\left(\theta + \dfrac{k\pi}{2} - \dfrac{2\pi}{3}\right) & 0 \\[5mm] 
\cos\left(\theta + \dfrac{k\pi}{2}+ \dfrac{2\pi}{3}\right) & \sin\left(\theta + \dfrac{k\pi}{2} + \dfrac{2\pi}{3}\right) & 0 
\end{array}\right] \nonumber\\[5mm]
\end{align}
\normalsize

	Computing the elements of $\mathbf{TK}$ is done through simple calculations  repeat the ones of the proof for theorem \ref{theo:dq0_balanced_3p}. For an arbitrary $\alpha$,

\begin{align}
\mathbf{T}\mathbf{K}_{\left(\alpha\right)} = 
\left[\begin{array}{ccc}
 \cos\left(\theta-\alpha\right) &  \sin\left(\theta - \alpha\right) & 0\\[5mm]
-\sin\left(\theta-\alpha\right) & -\cos\left(\theta - \alpha\right) & 0\\[5mm]
0 & 0 & 0
\end{array}\right]
\end{align}

	Therefore for $\alpha = \theta + k\pi/2$,

\begin{align}
\mathbf{T}\mathbf{K}_{\left(\theta + \frac{k\pi}{2}\right)} = 
\left[\begin{array}{ccc}
 \cos\left(-\dfrac{k\pi}{2}\right) &  \sin\left(-\dfrac{k\pi}{2}\right) & 0\\[5mm]
-\sin\left(-\dfrac{k\pi}{2}\right) & -\cos\left(-\dfrac{k\pi}{2}\right) & 0\\[5mm]
0 & 0 & 0
\end{array}\right]
= 
\left[\begin{array}{ccc}
 \cos\left( \dfrac{k\pi}{2}\right) &  -\sin\left(\dfrac{k\pi}{2}\right) & 0\\[5mm]
 \sin\left( \dfrac{k\pi}{2}\right) &   \cos\left(\dfrac{k\pi}{2}\right) & 0\\[5mm]
0 & 0 & 0
\end{array}\right] .
\end{align}

	Call this matrix $\mathbf{S_k}$ for $k\geq 1$. For the case $k = 0,\ \mathbf{K}_{\left(\theta + \frac{0\pi}{2}\right)} = \mathbf{T}^{-1}_{\theta}$, meaning $\mathbf{S}_0 = \mathbf{I}_3$ the identity matrix. \hfill$\blacksquare$
\vspace{5mm}
\hrule
\vspace{5mm}
%>>>

\begin{theorem}[Solutions to LTI ODEs with three-phase forcing] \label{theo:3p_ode_solution}%<<<

	Let $m\left(t\right),\theta\left(t\right)\in\left[\mathbb{R}\to\mathbb{R}\right]$ and consider the Hurwitz stable linear ODE with a three-phase phasorial forcing:

\begin{equation} \sum\limits_{k=0}^{n} \alpha_k \mathbf{x}^{\left(k\right)} - \mathbf{f_3}(t) = 0, \label{eq:theo_3p_ode_solution_original_ode}\end{equation}

	\noindent where $\mathbf{x},\mathbf{f_3}\in\left[\mathbb{R}\to\mathbb{R}^3\right]$ with a set of initial conditions $x_0,x'_0,...,x^{(n-1)}_0$. Let $\omega(t)$ be a $C^{\left(n-1\right)}$-class real function, and consider the set of decoupled ODEs of the ``dq equivalent'' and system with a zero-sequence

\begin{equation}
\left\{\begin{array}{l}
	\displaystyle \sum\limits_{i=0}^n \mathbf{K}_i (t) \dfrac{d^i \mathbf{z}_{dq}}{dt^i} - \mathbf{f}_{dq}  = \left[\begin{array}{c} 0 \\[3mm] 0 \end{array}\right]\\[5mm]
	\displaystyle \sum\limits_{i=0}^n \eta_i(t) \dfrac{d^i z_0}{dt^i} - f_0 = 0
\end{array}\right. , \label{eq:theo_3p_ode_solution_dq_equiv}
\end{equation}

	\noindent with a set of initial conditions $(\mathbf{z}_{dq})_0,(\mathbf{z}'_{dq})_0,...,(\mathbf{z}^{(n-1)}_{dq})_0$ , where $\mathbf{f}_{dq}$ is the $dq$ transform of the forcing at the frequency $\omega(t)$,

\begin{gather}
	\mathbf{K}_i(t) = \sum\limits_{k=i}^{n} \alpha_k{k\choose i} \left[\sum\limits_{c=0}^{k-i} \mathbf{G}_c B_{\left(k-i,c\right)}\left(\omega,\dot{\omega},\ddot{\omega},...,\omega^{(k-i-c)}\right) \right] \\[3mm]
	\eta_i(t) = \sum\limits_{k=i}^{k} \alpha_k {k\choose p} \left[\sum\limits_{c=0}^{k-i} B_{\left(k-p,c\right)}\left(\omega,\dot{\omega},\ddot{\omega},...,\omega^{(k-p-c)}\right)\right] 
\end{gather}

	\noindent where the $B_{\left(i,j\right)}$ are the incomplete exponential Bell Polynomials and and $\mathbf{G}_k$ are calculated as

\begin{equation}
\mathbf{G}_k = 
\left[\begin{array}{ccc}
 \cos\left(\dfrac{k\pi}{2}\right) &  -\sin\left(\dfrac{k\pi}{2}\right) \\[5mm]
 \sin\left(\dfrac{k\pi}{2}\right) & \phantom{-} \cos\left(\dfrac{k\pi}{2}\right)
\end{array}\right]
\end{equation}

	Then there exist two positive reals $a,b$ such that the solution $x$ to the original ODE \eqref{eq:theo_3p_ode_solution_original_ode} satisfies

\begin{equation} \left\lVert \mathbf{x} - \mathbf{T}^{-1}_\psi\mathbf{z}_{dq0}\right\rVert \leq ae^{-bt}, \label{eq:theo_3p_ode_solution_exp}\end{equation}

	\noindent with $\mathbf{z}_{dq0}$ is the unique solution to the dq system \eqref{eq:theo_3p_ode_solution_dq_equiv}. Reestated, the solution $\mathbf{z}_{\alpha\beta\gamma}$ reconstructed by \eqref{eq:theo_3p_ode_solution_dq_equiv} is the globally steady-state stable solution of \eqref{eq:theo_3p_ode_solution_original_ode}.

\end{theorem}
\textbf{Proof:} consider the original LTI ODE

\begin{equation} \sum\limits_{k=0}^n \alpha_k \mathbf{x}^{\left(k\right)} - \mathbf{f_3}(t) = 0.\end{equation}

	By hypothesis this system is Hurwitz stable, that is, the solution $x(t)$ tends exponentially to a particular solution: $\left\lVert x(t) - x_p(t)\right\rVert \leq ae^{-bt}$ for some two reals $a$ and $b$. Finding a particular solution $z(t)$, let $z_0,z'_0,...,z^{(n-1)}_0$ the initial conditions of the particular solution. Using $\mathbf{T}_\psi$ transform to generate an equivalent dq0 ODE:

\begin{equation} \sum\limits_{k=0}^n \alpha_k \mathbf{T}_\psi\left(\mathbf{T}_\psi^{-1}\mathbf{z}_{dq0}\right)^{\left(k\right)} - \mathbf{f}_{dq0} = 0 \end{equation}

	Apply lemma \ref{theo:dq0_3p_diff}:

\begin{equation} \sum\limits_{k=0}^n \alpha_k\left\{\mathbf{T}_\psi\left[ \sum\limits_{p=0}^{k} {k\choose p} \left(\dfrac{d^{p} \mathbf{T}^{-1}_\psi}{dt^p}\right) \left(\dfrac{d^{\left(k-p\right)} \mathbf{z}_{dq0}}{dt^{\left(k-p\right)}}\right) \right]\right\} - \mathbf{f}_{dq0} = 0 \end{equation}

	And because both $\mathbf{T}$ and $\mathbf{T}^{-1}$ are linear,

\begin{equation} \sum\limits_{k=0}^n \alpha_k \sum\limits_{p=0}^{k} {k\choose p} \mathbf{T}_\psi\left[\left(\dfrac{d^{\left(k-p\right)} \mathbf{T}^{-1}_\psi}{dt^{\left(k-p\right)}}\right) \left(\dfrac{d^p \mathbf{z}_{dq0}}{dt^p}\right) \right] - \mathbf{f}_{dq0} = 0 \end{equation}

	Now apply lemma \ref{lemma:t_ndifftminus_product}:

\begin{equation} \sum\limits_{k=0}^n \alpha_k \left\{\sum\limits_{p=0}^{k} {k\choose p} \left[\sum\limits_{c=0}^{k-p} \mathbf{S}_c B_{\left(k-p,c\right)}\left(\omega,\dot{\omega},\ddot{\omega},...,\omega^{(k-p-c)}\right) \right] \left(\dfrac{d^p \mathbf{z}_{dq0}}{dt^p }\right)\right\} - \mathbf{f}_{dq0} = 0 \end{equation}

	To isolate the derivatives of $\mathbf{z}_{dq}$, one must solve the triangular sum of this equation. The 0-th derivatives are present at all $k$ indexes; the first, for the $k$ indexes $1$ through $n$; the second for $2$ to $n$. In general, the i-th derivative is present for indexes $k$ from $i$ to $n$.

\begin{equation} \sum\limits_{i=0}^n \left\{\sum\limits_{k=i}^{n} \alpha_k{k\choose i} \left[\sum\limits_{c=0}^{k-i} \mathbf{S}_c B_{\left(k-i,c\right)}\left(\omega,\dot{\omega},\ddot{\omega},...,\omega^{(k-i-c)}\right) \right]\right\} \left(\dfrac{d^i \mathbf{z}_{dq0}}{dt^i }\right) - \mathbf{f}_{dq0} = 0 . \label{eq:3p_original_ode_complete}\end{equation}

	We now note that the matrices $\mathbf{S}_c$ have null third row and column, except for $c = 0$ as per \eqref{eq:sk_gk_equiv}, and can be expressed as a block composition of the $\mathbf{G}_c$. Thus we use \eqref{eq:sk_gk_equiv} and separate the case $c=0$ to yield

\small
\begin{gather}
	\sum\limits_{c=0}^{k-1} \mathbf{S}_c B_{\left(k-i,c\right)}\left(\omega,\dot{\omega},\ddot{\omega},...,\omega^{(k-p-c)}\right) = \nonumber\\[3mm]
	\left[\begin{array}{ccc} \\[-1mm] \multicolumn{2}{c}{\left[\mathbf{G}_0\right]} & 0\\[5mm] 0 & 0 & 1 \end{array}\right] B_{\left(k-i,0\right)}\left(\omega,\dot{\omega},\ddot{\omega},...,\omega^{(k-p)}\right) + \sum\limits_{c=1}^{k-i} \left[\begin{array}{ccc} \\[-1mm] \multicolumn{2}{c}{\left[\mathbf{G}_c\right]} & 0\\[5mm] 0 & 0 & 0 \end{array}\right] B_{\left(k-p,c\right)}\left(\omega,\dot{\omega},\ddot{\omega},...,\omega^{(k-p-c)}\right)
\end{gather}
\normalsize

	\noindent and one notes that the fact that $\mathbf{G}_c$ is isolated in a block and that $\mathbf{G}_0$ has the single unit element on the bottom right makes this equation equivalent to two de-coupled equations, one bi-dimensional in the dq frame and another single-dimensional in the zero-sequence:

\begin{equation}
	\sum\limits_{c=0}^{k-i} \mathbf{S}_c B_{\left(k-i,c\right)}\left(\omega,\dot{\omega},\ddot{\omega},...,\omega^{(k-p-c)}\right) = 
\left[\begin{array}{c}
\displaystyle\sum\limits_{c=0}^{k-i} \mathbf{G}_c B_{\left(k-i,c\right)}\left(\omega,\dot{\omega},\ddot{\omega},...,\omega^{(k-p-c)}\right) \\[5mm]
\displaystyle\sum\limits_{c=0}^{k-i} B_{\left(k-i,c\right)}\left(\omega,\dot{\omega},\ddot{\omega},...,\omega^{(k-i-c)}\right)
\end{array}\right] .
\end{equation}

	Thus \eqref{eq:3p_original_ode_complete} is equivalent to two de-coupled equations:

\begin{equation}
\left\{\begin{array}{l}
	\displaystyle \sum\limits_{i=0}^n \left\{\sum\limits_{k=i}^{n} \alpha_k {k\choose p} \left[\sum\limits_{c=0}^{k-i} \mathbf{G}_c B_{\left(k-p,c\right)}\left(\omega,\dot{\omega},\ddot{\omega},...,\omega^{(k-p-c)}\right) \right]\right\} \left(\dfrac{d^i \mathbf{z}_{dq}}{dt^i}\right) - \mathbf{f}_{dq}  = \left[\begin{array}{c} 0 \\[3mm] 0 \end{array}\right]\\[5mm]
	\displaystyle \sum\limits_{i=0}^n \left\{\sum\limits_{k=i}^{k} \alpha_k {k\choose p} \left[\sum\limits_{c=0}^{k-i} B_{\left(k-p,c\right)}\left(\omega,\dot{\omega},\ddot{\omega},...,\omega^{(k-p-c)}\right)\right] \right\} \dfrac{d^i z_0}{dt^i} - f_0 = 0
\end{array}\right.
\end{equation}

	Finally, we group the terms inside the sums as

\begin{gather}
	\mathbf{K}_i(t) = \sum\limits_{k=i}^{n} \alpha_k{k\choose i} \left[\sum\limits_{c=0}^{k-i} \mathbf{G}_c B_{\left(k-i,c\right)}\left(\omega,\dot{\omega},\ddot{\omega},...,\omega^{(k-i-c)}\right) \right] \\[3mm]
	\eta_i(t) = \sum\limits_{k=i}^{k} \alpha_k {k\choose p} \left[\sum\limits_{c=0}^{k-i} B_{\left(k-p,c\right)}\left(\omega,\dot{\omega},\ddot{\omega},...,\omega^{(k-p-c)}\right)\right] 
\end{gather}

	yielding

\begin{equation}
\left\{\begin{array}{l}
	\displaystyle \sum\limits_{i=0}^n \mathbf{K}_i (t) \dfrac{d^i \mathbf{z}_{dq}}{dt^i} - \mathbf{f}_{dq}  = \left[\begin{array}{c} 0 \\[3mm] 0 \end{array}\right]\\[5mm]
	\displaystyle \sum\limits_{i=0}^n \eta_i(t) \dfrac{d^i z_0}{dt^i} - f_0 = 0
\end{array}\right. .
\end{equation}

	Therefore, $z(t) = \mathbf{T}^{-1}_{\psi(t)}\mathbf{z}_{dq0}$ is a particular solution to the original system, and \eqref{eq:theo_3p_ode_solution_exp} follows. \hfill$\blacksquare$

\vspace{5mm}
\hrule
\vspace{5mm}
%>>>

	Furthermore, it is simple to see that we can apply the results of subsection \ref{subsec:complexification} to transform this theorem into a complex version:

\begin{theorem}[Complex equivalence of three-phase phasorially excited LTI ODEs]\label{corollary:3p_complex_equivalence_phasorialodes} %<<<

	Take the three-phase LTI ODE \eqref{eq:theo_3p_ode_solution_original_ode} of theorem \ref{theo:3p_ode_solution}, the same apparent frequency $\omega(t)$ signal, and the dq-equivalent ODE to the complex differential equation \eqref{eq:theo_3p_ode_solution_dq_equiv}. Consider the set of differential equations

\begin{equation}
\left\{\begin{array}{l}
	\displaystyle \sum\limits_{i=0}^n \beta_i^n(t) Z^{(i)} - F = 0 ,\\[5mm]
	\displaystyle \sum\limits_{i=0}^n \eta_i(t) \dfrac{d^i z_0}{dt^i} - f_0 = 0
\end{array}\right. , \label{eq:theo_3p_ode_solution_complex_equiv}
\end{equation}

	\noindent with $Z(t) = z_d(t) + jz_q(t)$, equipped with	initial conditions $Z_0,Z'_0,Z''_0,...,Z^{(n-1)}_0$ calculated from the initial conditions of the dq system as

\begin{equation} Z_0 = z_{d0} + jz_{q0},\ Z'_0 = z'_{d0} + jz'_{q0},\ ...\ ,Z^{(n-1)}_0 = z^{(n-1)}_{d0} + jz^{(n-1)}_{q0}. \end{equation}
	
	\noindent where $F = \rho\left[f_d + jf_q\right]$ is the Dynamic Phasor Transform of the forcing $\mathbf{f}_3(t)$, and the $\beta_i^n(t)$ are time-varying complex coefficients given by

\begin{equation} \beta_i^n(t) = \sum\limits_{k=i}^{n} \alpha_k{k\choose i} \left[\sum\limits_{c=0}^{k-i} j^cB_{\left(k-i,c\right)}\left(\omega,\dot{\omega},\ddot{\omega},...,\omega^{(k-i-c)}\right) \right].  \end{equation}

	\noindent and the $\eta(t)$ as defined in Theorem \ref{theo:3p_ode_solution}. Then $\mathbf{z}_{dq}(t) = \rho^{-1}\left[Z\right]$ is such that there exist $a,b\in\mathbb{R}^+$ such that

\begin{equation} \left\lVert \mathbf{x} - \mathbf{T}^{-1}_\psi\left[\begin{array}{c} z_d(t) \\[3mm] z_q(t) \\[3mm] z_0(t) \end{array}\right]\right\rVert \leq ae^{-bt}. \label{eq:theo_3p_ode_complex_solution_exp}\end{equation}

	Particularly, if the initial conditions of $Z(t)$ and of $z_0(t)$ reconstruct the initial conditions of $\mathbf{x}(t)$ at initial time, then $z_d,z_q,z_0$ reconstructs $\mathbf{x}(t)$ loslessly.
\end{theorem}
\textbf{Proof:} identical to theorem \ref{corollary:complex_equivalence_phasorialodes}, by using the complexification operator onto the dq portion of \eqref{eq:theo_3p_ode_solution_dq_equiv}.
\vspace{3mm}
\hrule
\vspace{3mm}
%>>>

%-------------------------------------------------
\subsection{On the zero-sequence component}\label{subsec:zeroseq_comp} %<<<2

	A discussion on theorem \ref{corollary:3p_complex_equivalence_phasorialodes} can be made regarding the zero-sequence component $z_0(t)$. Naturally, if $z_0(t) = 0$ then $\left[z_d,z_q\right] = \rho^{-1}\left[Z\right]$ reconstructs $\mathbf{x}$ with fading exponential precision in time, that is, $Z(t)$ is sufficient to describe $\mathbf{x}$ in time, which is to say that $\mathbf{x}$ becomes balanced exponentially. Particularly, if the initial conditions of $Z(t)$ reconstruct the initial conditions of $\mathbf{x}(t)$, then $Z(t)$ reconstructs $\mathbf{x}$ perfectly, meaning $\mathbf{x}(t)$ is balanced.

	Being able to reconstruct the three-phase $\mathbf{x}(t)$ with only the Dynamic Phasor $Z(t)$ is certainly an easement, but it requires that $z_0(t) = 0$ at all times which is a rather hard requirement. We now explore more general conditions on $z_0(t)$ so that the phasor $Z(t)$ can still be used almost exclusively.

\begin{corollary}[Bounds of the solutions of LTI ODEs with three-phase forcing] \label{corollary:bounds_solution_3p_ode}%<<<

	Let $\mathbf{x}(t)$ the solution of the original time-domain LTI ODE \eqref{eq:theo_3p_ode_solution_original_ode}. Let $\mathbf{z}_{dq}^B = \left[z_d(t),z_q(t),0\right]$, the subscript ``B'' for ``balanced'', where $z_d,z_q,z_0$ are the solutions to the dq0-equivalent system \eqref{eq:theo_3p_ode_solution_dq_equiv}. Then there exist $a,b\in\mathbb{R}_+$ such that

\begin{equation} \left\lVert \mathbf{x} - \mathbf{T}^{-1}_\psi\mathbf{z}_{dq}^B \right\rVert_2 \leq ae^{-bt} + \left\lvert z_0(t) \right\rvert .\end{equation}

	Particularly, if $\mathbf{z}_{dq}$ reconstructs $\mathbf{x}$ perfectly (like if they have the same initial conditions) then

\begin{equation} \left\lVert \mathbf{x} - \mathbf{x}_B \right\rVert_2 \leq \left\lvert x_0(t)\right\rvert ,\end{equation}

	\noindent where $\left\lVert\cdot\right\rVert_2$ is the Euclidean norm  and $\mathbf{x}_B = \mathbf{T}^{-1}_\psi\left[x_d(t),x_q(t),0\right]$.
\end{corollary}
\textbf{Proof:} let $\mathbf{z}_{dq} = \left[z_d,z_q,z_0\right]^\transpose,\ \mathbf{z}_{dq}^B = \left[z_d, z_q,0\right]^\transpose$. Calculating the distance between $\mathbf{z}$ and $\mathbf{z}_\infty$ yields

\begin{equation} \left\lVert \mathbf{z}_{dq} - \mathbf{z}_{dq}^B \right\rVert =  \left\lVert \left[\begin{array}{c} z_d(t) \\[3mm] z_q(t) \\[3mm] z_0(t) \end{array}\right] - \left[\begin{array}{c} z_d(t) \\[3mm] z_q(t) \\[3mm] 0 \end{array}\right]\right\rVert = \left\lvert z_0(t)\right\rvert . \label{eq:balanced_3p_ode_solution_eq2}\end{equation}

	Now note that

\begin{equation} \left\lVert \mathbf{x} - \mathbf{T}^{-1}_\psi \mathbf{z}_{dq}^B \right\rVert = \left\lVert \mathbf{x} - \mathbf{T}^{-1}_\psi \mathbf{z}_{dq} - \left(\mathbf{T}^{-1}_\psi \mathbf{z}_{dq}^B - \mathbf{T}^{-1}_\psi \mathbf{z}_{dq}\right) \right\rVert \leq  \left\lVert \mathbf{x} - \mathbf{T}^{-1}_\psi\mathbf{z}_{dq} \right\rVert + \left\lVert\mathbf{T}^{-1}_\psi \mathbf{z}_{dq} - \mathbf{T}^{-1}_\psi\mathbf{z}_{dq}^B \right\rVert . \label{eq:balanced_3p_ode_solution_eq4}\end{equation}

	Now use the result \eqref{eq:theo_3p_ode_solution_exp} of theorem \ref{theo:3p_ode_solution}, and that

\begin{equation} \left\lVert\mathbf{T}^{-1}_\psi \mathbf{z}_{dq} - \mathbf{T}^{-1}_\psi\mathbf{z}_{dq}^B \right\rVert = \left\lVert\mathbf{T}^{-1}_\psi\left(\mathbf{z}_{dq} - \mathbf{z}_{dq}^B\right) \right\rVert \leq \left\lVert\mathbf{T}^{-1}_\psi \right\rVert \left\lVert\mathbf{z}_{dq} - \mathbf{z}_{dq}^B \right\rVert \end{equation}

	\noindent yields

\begin{equation} \left\lVert \mathbf{x} - \mathbf{T}^{-1}_\psi \mathbf{z}_{dq}^B \right\rVert \leq ae^{-bt} + \left\lVert\mathbf{T}^{-1}_\psi \right\rVert\left\lvert z_0(t)\right\rvert . \label{q:balanced_3p_ode_solution_eq4}\end{equation}

	Now we estimate the norm of the operator. Using the Euclidean norm $\left\lVert \left(\cdot\right)\right\rVert_2$, by theorem \ref{theo:spectral_norm}, the Euclidean norm of a matrix $\mathbf{A}$ is given by its singular value, that is, the square root of the largest eigenvalue of the adjoint matrix $\mathbf{A}^\hermitian \mathbf{A}$. But since $\mathbf{T}^{-1}_\psi$ is orthonormal (its transpose equals its inverse), then 

\begin{equation} \left(\mathbf{T}^{-1}_\psi\right)^\hermitian \mathbf{T}^{-1}_\psi = \overline{\mathbf{T}_\psi} \mathbf{T}^{-1}_\psi , \end{equation}

	\noindent where the overline $\overline{\left(\cdot\right)}$ denotes the complex conjugate. Because both matrices are real, their conjugates are equal to themselves and

\begin{equation} \overline{\mathbf{T}_\psi} \mathbf{T}^{-1}_\psi = \mathbf{T}_\psi \mathbf{T}^{-1}_\psi = \mathbf{I}\end{equation}

	\noindent meaning the largest eigenvalue of this matrix is the singular value of the identity matrix, which is trivially unitary. Therefore

\begin{equation} \left\lVert \mathbf{T}^{-1}_\psi \right\rVert_2 = 1 \label{eq:balanced_3p_ode_solution_eq3} \end{equation}

	Finally, substituting \eqref{eq:balanced_3p_ode_solution_eq3} into \eqref{eq:balanced_3p_ode_solution_eq4} yields

\begin{equation} \left\lVert \mathbf{x} - \mathbf{T}^{-1}_\psi\mathbf{z}_{dq}^B\right\rVert_2 \leq ae^{-bt} + \left\lvert z_0(t) \right\rvert .\label{eq:balanced_3p_ode_solution_eq5}\end{equation}

	Specifically, if $\mathbf{z}(t)$ reconstructs $\mathbf{x}$ through the same initial conditions,

\begin{equation} \left\lVert \mathbf{x} - \mathbf{T}^{-1}_\psi\mathbf{x}_{dq}^B\right\rVert_2 \leq \left\lvert x_0(t) \right\rvert ,\end{equation}

	\noindent where $\mathbf{x}_{dq}^B = \left[x_q(t),x_0(t),0\right]^\transpose$. Let $\mathbf{x}_B = \mathbf{T}_\psi^{-1}\mathbf{x}_{dq}^B$ and the proof is complete.

\hfill$\blacksquare$
\vspace{5mm}
\hrule
\vspace{5mm}
%>>>

	In essence what corollary \ref{corollary:bounds_solution_3p_ode} states is that the distance between $\mathbf{x}$ , the solution of the original system, and the balanced three-phase version $\mathbf{z}^B$ of the solution of the dq-equivalent system is basically a fading exponential added to $z_0$. In the case the initial conditions of the dq0 equivalent system are the same as that of $x(t)$, the distance is only $\left\lvert x_0(t)\right\rvert$. Thus, it follows that if $z_0(t)$ vanishes assymptotically, then $\mathbf{x}$ becomes ``assymptotically balanced'', in the sense that it tends to a three-phase balanced quantity. The simplest case where $z_0(t)$ vanishes in time is, obviously, if the forcing $\mathbf{f}$ is balanced. In this case, $z_0(t) = 0$ is clearly a solution to the zero-sequence portion of \eqref{eq:theo_3p_ode_solution_complex_equiv}.

	Therefore, if $z_0(t)$ vanishes, then $\mathbf{x}^B$ is a stable steady-state solution of $\mathbf{x}$. This steady-state solution the nice property that it reconstructed by a signal $\mathbf{z}_{dq}^B$ which zero-sequence component is null, meaning that $\mathbf{x}$ tends to a balaced three-phase quantity; therefore this stable solution admits a purely phasorial representation $X(t) = x_d(t) + jx_q(t)$. Ultimately, the result of corollary \ref{corollary:bounds_solution_3p_ode} means that the stability of the steady-state solution $\mathbf{x}^B$ is the very same stability as that of $x_0(t)$, in the sense that the difference $\left\lVert \mathbf{x} - \mathbf{x}^B\right\rVert$ is bound by the same function. Therefore if $x_0$ is assymptotically stable so is $\mathbf{x}^B$; if it is exponentially stable, so is the steady-state solution. Therefore, the least needed characteristic of the three-phase forcing $\mathbf{f}_3(t)$ that causes the steady-state solution of the original ODEs \eqref{eq:theo_3p_ode_solution_original_ode} to be stable is that its zero-sequence component $f_0(t)$ define a stable ODE; this means that $\mathbf{f}_3(t)$ does not need to be actually balanced to yield a balanced solution.

	In short, $\mathbf{x}(t)$ will tend to an assymptotic quantity if the combination of the system coefficients $\left(\alpha_k\right)_{k=0}^n$, the apparent frequency $\omega(t)$ and the zero-sequence component of the forcing $f_0(t)$ are such that the equation

\begin{equation}
	\sum\limits_{i=0}^n \eta_i^n \dfrac{d^i z_0}{dt^i} - f_0 = 0,\ \eta_i(t) = \sum\limits_{k=i}^{n} \alpha_k {k\choose p} \left[\sum\limits_{c=0}^{k-i} B_{\left(k-p,c\right)}\left(\omega,\dot{\omega},\ddot{\omega},...,\omega^{(k-p-c)}\right)\right] \label{eq:3p_zeroseq_ode}
\end{equation}

	\noindent has an assymptotically vanishing solution $z_0(t)$. There is, unfortunately, no way to know preemptively if such is the case because this differential equation is linear but not time-invariant as the coefficients are time-varying; this is especially disappointing because one expects that if the forcing $\mathbf{f}_3$ is balanced, then the response $\mathbf{x}(t)$ of the system will also be balanced. Naturally, in the static case, if $\omega(t) = \omega_0$ one can prove using the same line of thought as subsection \ref{subsec:discussion_complexification} that if $f_0$ is identically null at $\omega_0$ then the system yields a Hurwitz-stable linear differential equation with constant coefficients — therefore $z_0$ tends to zero exponentially, thus $\mathbf{x}$ tends exponentially to a balanced quantity. Given the right initial conditions, $z_0$ is identically null and $\mathbf{x}$ is balanced at all times.	

	For time-varying frequencies, the time-varying nature of the coefficients pose a great challenge. While there exist many results about linear systems with time-varying coefficients (see for instance chapter 12 of \cite{beffaWeaklyNonlinearSystems2024}) guaranteeing stability (and by correlation guaranteeing that $z_0$ vanishes as time grows) is not as direct as LTI systems. For the specific case of equation \eqref{eq:3p_zeroseq_ode}, we will show in chapter \ref{chapter:choice_apparent_frequency} section \ref{sec:3p_assymp_freq} that if the circuit is very ``quick'' — the roots of the Hurwitz polynomial of the time-domain differential equation

\begin{equation} H(x) \sum\limits_{k=0}^n \alpha_k x^k \end{equation}

	\noindent have negative but large real parts — and the apparent frequency $\omega(t)$ is equivalent (in a sense that will be formally defined) to a synchronous value $\omega_0$ that is sufficiently small (the frequency is ``slow'') then the equivalent zero-sequence differential equation yields a Hurwitz stable differential equation, so that if $f_0(t)$ is bounded then $z_0(t)$ is also bounded. Particularly, if the forcing $\mathbf{f}_3$ is balanced and $f_0$ is null, then $z_0$ will assymptotically tend to zero, being zero at all times given proper initial conditions.

	In general, three-phase circuits are designed so that each phase is identical and they share the same loads. In this case, if the phases are excited by balanced excitations, then their responses (voltages and currents) will also be balanced; hence such a circuit is called a balanced circuit. The simplicity of such circuits is that because the quantities involved inevitably tend to a balanced quantity, then they can all be transformed into phasors; this allows for a single-phase representation of the balanced three-phase network, due to the fact that if the behavior of a single phase is known, then the behavior of the other two are easily drawn from the known phase.

%-------------------------------------------------
\section{Three-phase Generalized Complex Power} %<<<1

	To complete the Three-Phase Dynamic Phasors modelling, we now show that the proposed transform is able to generate a notion of complex power for three-phase circuits under generalized sinusoidal regimens.	

\begin{theorem}[Generalized Three-Phase Complex Power]\label{theo:3p_activereactivepower} %<<<
	Let $V = m_v(t)e^{j\phi_v(t)}$ and $I = m_i(t)e^{j\phi_i(t)}$ represent the three-phase dynamical phasors of the balanced voltage $\mathbf{v} = \left[v_a,v_b,v_c\right]^\transpose$ across and balanced current $\mathbf{i} = \left[i_a,i_b,i_c\right]^\transpose$ through a three-phase bipole and consider the quantity 

\begin{equation} S(t) = \left<V(t),I(t)\right> = P(t) + jQ(t)\ \left\{\begin{array}{l} P(t) = m_v(t)m_i(t)\cos\left[\phi_v(t) - \phi_i(t)\right] \\[3mm] Q(t) = m_v(t)m_i(t)\sin\left[\phi_v(t) - \phi_i(t)\right] \end{array}\right. \end{equation}

	\noindent called \textbf{complex power}. Then $S(t)$ is such that the instantaneous power performed by each phase is

\begin{equation} p_\alpha(t) = \dfrac{1}{3}P\left\{1 + \cos\left[2\left(\psi(t) + \phi_v(t) + 2\alpha\right) \right]\right\} + \dfrac{1}{3}Q\sin\left[2\left(\psi(t) + \phi_v(t) + 2\alpha \right)\right]  \end{equation}

	\noindent where $\alpha = 0$ for phase a, $-2\pi/3$ for phase b and $+2\pi/3$ for phase c. Finally, the total power performed by the three-phase bipole is

\begin{equation} p_{3\phi}(t) = \overbrace{v_a(t)i_a(t)}^{p_a(t)} + \overbrace{v_b(t)i_b(t)}^{p_b(t)} + \overbrace{v_c(t)i_c(t)}^{p_c(t)} = P(t).\end{equation}
\end{theorem}
\textbf{Proof:} basically a re-proof of theorem \ref{theo:activereactivepower}, but for three phases. First write

\begin{equation} p_{3\phi}(t) = \overbrace{v_a(t)i_a(t)}^{p_a(t)} + \overbrace{v_b(t)i_b(t)}^{p_b(t)} + \overbrace{v_c(t)i_c(t)}^{p_c(t)}.\end{equation}

	Because the voltage and current are supposed balanced,

\begin{equation} \mathbf{v} = \sqrt{\dfrac{2}{3}} m_v(t) \left[\begin{array}{c} \cos\left(\psi(t) + \phi_v(t)\right) \\[3mm] \cos\left(\psi(t) + \phi_v(t) - \dfrac{2\pi}{3}\right) \\[3mm] \cos\left(\psi(t) + \phi_v(t) + \dfrac{2\pi}{3}\right)\end{array}\right],\ \mathbf{i} = \sqrt{\dfrac{2}{3}} m_i(t) \left[\begin{array}{c} \cos\left(\psi(t) + \phi_i(t)\right) \\[3mm] \cos\left(\psi(t) + \phi_i(t) - \dfrac{2\pi}{3}\right) \\[3mm] \cos\left(\psi(t) + \phi_i(t) + \dfrac{2\pi}{3}\right)\end{array}\right] .\end{equation}

	\noindent meaning

\begin{equation} p_{3\phi}(t) = \dfrac{2}{3} m_v(t)m_i(t) \left[\begin{array}{l} \overbrace{\cos\left(\psi(t) + \phi_v(t)\right)\cos\left(\psi(t) + \phi_i(t)\right)}^{p'_a(t)} + \\[3mm] \hspace{5mm} \overbrace{\cos\left(\psi(t) + \phi_v(t) - \dfrac{2\pi}{3}\right)\cos\left(\psi(t) + \phi_i(t) - \dfrac{2\pi}{3}\right)}^{p'_b(t)} + \\[3mm] \hspace{10mm} \overbrace{\cos\left(\psi(t) + \phi_v(t) + \dfrac{2\pi}{3}\right)\cos\left(\psi(t) + \phi_i(t) + \dfrac{2\pi}{3}\right)}^{p'_c(t)} \end{array} \right] .\end{equation}

	Consider $\alpha\in\left\{-\frac{2\pi}{3},0,\frac{2\pi}{3}\right\}$ and let the expression 

\begin{equation} p_\alpha(t) = \dfrac{2}{3}m_v(t)m_i(t)\cos\left(\psi(t) + \phi_v(t) + \alpha\right)\cos\left(\psi(t) + \phi_i(t) + \alpha\right), \end{equation} 

	\noindent such that $p_a(t) = p_0(t),\ p_b(t) = p_{-\frac{2\pi}{3}},\ p_c(t) = p_{+\frac{2\pi}{3}}$,  and denote $\Delta\phi(t) = \phi_v(t) - \phi_i(t)$. Then $\phi_v(t) + \phi_i(t) = 2\phi_v(t) - \Delta\phi(t)$; therefore using

\begin{equation} \cos(a)\cos(b) = \dfrac{1}{2}\left[\cos(a+b) + \cos(a-b)\right],\end{equation}

	\noindent one obtains

\begin{align} p_\alpha(t) &= \dfrac{m_v(t)m_i(t)}{3} \left[ \cos\left(2\psi(t) + \phi_v(t) + \phi_i(t) + 2\alpha\right) + \cos\left(\phi_v(t) - \phi_i(t)\right)\right] \nonumber\\[3mm] &= \dfrac{m_v(t)m_i(t)}{3} \left\{\cos\left[2\left(\psi(t) + \phi_v(t)\right) - \Delta\phi(t) + \alpha\right] + \cos\left[\Delta\phi(t)\right]\right\} \end{align}

	Using $\cos(a-b) = \cos(a)\cos(b) + \sin(a)\sin(b)$,

\begin{equation} p_\alpha(t) = \dfrac{m_v(t)m_i(t)}{3} \left\{ \begin{array}{l} \cos\left(\Delta\phi(t)\right)\left\{\raisebox{3mm}{} 1 + \cos\left[2\left(\psi(t) + \phi_v(t) + 2\alpha\right)\right]\right\} + \\[3mm] \hspace{20mm} +\sin\left(\Delta\phi(t)\right)\sin\left[2\left(\psi(t) + \phi_v(t) + 2\alpha\right)\right] \end{array} \right\} . \label{eq:3p_nonst_complex_apparent_power_eq1} \end{equation}

	Let

\begin{equation} P = m_v(t)m_i(t) \cos\left(\Delta\phi(t)\right),\ Q = m_v(t)m_i(t) \sin\left(\Delta\phi(t)\right) \end{equation}

	\noindent then

\begin{equation} p_\alpha(t) = \dfrac{1}{3}P\left\{1 + \cos\left[2\left(\psi(t) + \phi_v(t) + 2\alpha\right) \right]\right\} + \dfrac{1}{3}Q\sin\left[2\left(\psi(t) + \phi_v(t) + 2\alpha \right)\right] . \label{eq:3p_nonst_complex_apparent_power_eq2} \end{equation}

	Now 

\begin{align}
	p_{3\phi}(t) &= p_{0}(t) + p_{\left(-\frac{2\pi}{3}\right)}(t) + p_{\left(+\frac{2\pi}{3}\right)}(t) = \nonumber\\[3mm] &= \dfrac{1}{3}P\left[ 3 + \begin{array}{l} \cos\left[2\left(\psi(t) + \phi_v(t)\right) \right] + \\[3mm] \hspace{5mm} + \cos\left[2\left(\psi(t) + \phi_v(t) + \dfrac{4\pi}{3}\right) \right] \\[3mm] \hspace{10mm} + \cos\left[2\left(\psi(t) + \phi_v(t) - \dfrac{4\pi}{3}\right) \right] \end{array}\right] + \dfrac{1}{3}Q\left[\begin{array}{l} \sin\left[2\left(\psi(t) + \phi_v(t)\right) \right] + \\[3mm] \hspace{5mm} + \sin\left[2\left(\psi(t) + \phi_v(t) + \dfrac{4\pi}{3}\right) \right] \\[3mm] \hspace{10mm} + \sin\left[2\left(\psi(t) + \phi_v(t) - \dfrac{4\pi}{3}\right) \right] \end{array}\right]
\end{align}

	\noindent and since

\begin{equation} \cos\left(x\right) + \cos\left(x + \dfrac{4\pi}{3}\right) + \cos\left(x - \dfrac{4\pi}{3}\right) = 0 \end{equation}

	\noindent for any $x$, then this means $p_{3\phi}(t) = P(t)$. \hfill$\blacksquare$
\vspace{5mm}
\hrule
\vspace{5mm} %>>>

	It is now simple to see that the expression \eqref{eq:3p_nonst_complex_apparent_power_eq1} for $p_\alpha$ can be used to draw three-phase versions of theorems \ref{theo:activepowerperiod} and \ref{theo:direct_quad_current_nonst}. More specifically, define

\begin{equation} v_\alpha(t) = m_v(t)\cos\left( \psi(t) + \phi_v(t) + \alpha\right),\ i_\alpha(t) = m_i(t)\cos\left(\psi(t) + \phi_i(t) + \alpha\right) \end{equation}

	\noindent where $\alpha = 0$ for phase a, $-2\pi/3$ for phase b and $+2\pi/3$ for phase c, it is simple to prove that there exists some $T(t)$ such that

\begin{equation} \dfrac{1}{T(t)}\int_{t}^{t+T(t)} p_\alpha(s)ds = P(t) \end{equation}

	\noindent and that

\begin{equation} i(t) = \sqrt{\dfrac{3}{2}}\dfrac{P(t)}{m_i(t)}\cos\left(\psi(t) + \phi_v(t)\right) + \sqrt{\dfrac{3}{2}}\dfrac{Q(t)}{m_v(t)}\sin\left(\psi(t) + \phi_v(t)\right) ,\end{equation}

	\noindent meaning that the three-phase active and reactive powers have the exact same physical meanings as the single-phase counterparts.

%-------------------------------------------------
\section{Some circuit analysis in three-phase domain and example simulation} %<<<1

	Finally, we want to repeat the results of theorems \ref{theo:kirchoff_current_1p} through \ref{theo:1p_inductive_impedance} for a three-phase scenario. The proof of Kirchoff's Laws is elementary and will not be re-done.

\begin{theorem}[Time-dependant 3$\phi$ capacitive impedance]\label{theo:3p_capacitive_conductance} % <<<
Let $\mathbf{v} = \left[v_a,v_b,v_c\right]$ a balanced 3$\phi$ voltage across a three-phase bank capacitors of value $C$, like in the figure below. Denote $V = \mathbf{P^{\omega}_{3\phi}}\left[\mathbf{v}\right] = v_d\left(t\right) + jv_q\left(t\right)$ as the corresponding phasor of $\mathbf{v}$, $\omega$ as its apparent frequency and $\psi = \int_{0}^{t} \omega(x)dx$. Also let $\mathbf{T}_\psi$ be the $dq0$ transform matrix at $\phi(t)$.

% THREEPHASE CAPACITOR <<<
\begin{center}
        \begin{tikzpicture}[american,scale=1,transform shape,line width=0.75, cute inductors, voltage shift = 1,>={Stealth[inset=0mm,length=1.5mm,angle'=50]}]
	\ctikzset{/tikz/circuitikz/voltage/distance from node=22mm}
		\draw (0,0)                             node (starta) {} to [short,o-o] ++(6,0) node (enda) {};
		\draw ([shift=({4,0})]starta.center) to [C,l=$C$, *-*,f>^=$i_a$,v=$v_a(t)$] ++(0,-6) node(bottoma) {};
		\draw (starta |- bottoma) to [short,o-o] (enda |- bottoma) ;
%	
		\draw[preaction={draw,white,line width=2mm}] ([shift=({2.5,-0.5})]starta.center) node (startb) {} to [short,o-o] ++(6,0) node (endb) {};
		\draw ([shift=({4,0})]startb.center) to [C,l=$C$, *-*,f>^=$i_b$,v=$v_b(t)$] ++(0,-6) node(bottomb) {};
		\draw (startb |- bottomb) to [short,o-o] (endb |- bottomb) ;
%	
		\draw[preaction={draw,white,line width=2mm}] ([shift=({2.5,-0.5})]startb.center) node (startc) {} to [short,o-o] ++(6,0) node (endc) {};
		\draw ([shift=({4,0})]startc.center) to [C,l=$C$, *-*,f>^=$i_c$,v=$v_c(t)$] ++(0,-6) node(bottomc) {};
		\draw (startc |- bottomc) to [short,o-o] (endc |- bottomc) ;
        \end{tikzpicture}
\end{center} %>>>

	Then the 3$\phi$ current through the bank of capacitors $\mathbf{i} = \left[i_a,i_b,i_c\right]$ is such that

\begin{align}
\left\{\begin{array}{l}
        i_d = C\dfrac{dv_d}{dt} - \omega C v_q \\[5mm]
        i_q = C\dfrac{dv_q}{dt} + \omega C v_d \\[5mm]
        i_0 = 0
\end{array}\right.
\end{align}

	Therefore the phasor

\begin{equation} I = C\dfrac{dV}{dt} + j\omega(t) C V \end{equation}

	\noindent is equal to the phasor corresponding to $\mathbf{i}$, $\mathbf{P^{\omega}_{3\phi}}\left[\mathbf{i}\right] = i_d\left(t\right) + ji_q\left(t\right)$.

\end{theorem}
\textbf{Proof:} writing the time differential equations,

\begin{equation} \mathbf{i} = 
\left[\begin{array}{c} i_a \\ i_b \\ i_c \end{array}\right] = 
\left[\begin{array}{c} C\dfrac{dv_a}{dt} \\[5mm] C\dfrac{dv_b}{dt} \\[5mm] C\dfrac{dv_c}{dt} \end{array}\right] \Leftrightarrow
 \mathbf{i} = C\dfrac{d\mathbf{v}}{dt} \end{equation}

	Applying the $dq0$ to both sides at the angle $\psi$,

\begin{align}
	\mathbf{i}_{dq0} &= \mathbf{T}_\psi \mathbf{i} \nonumber\\[3mm]
	&= \mathbf{T}_\psi C\dfrac{d\mathbf{v}}{dt} \nonumber\\[3mm] 
	&\substack{\text{(Lemma \ref{theo:dq0_3p_diff})} \\ =}\hspace{2mm} \mathbf{T}_\psi C\left[ \mathbf{T}^{-1}_\psi \dfrac{d}{dt}\left(\mathbf{v}_{dq0}\right) + \dfrac{d}{dt}\left(\mathbf{T}^{-1}_\psi \right) \mathbf{v}_{dq0} \right] \nonumber\\[3mm] 
	&= C\left[ \mathbf{T}_\psi \mathbf{T}^{-1}_\psi \dfrac{d}{dt}\left(\mathbf{v}_{dq0}\right) + \mathbf{T}_\psi \dfrac{d}{dt}\left(\mathbf{T}^{-1}_\psi \right) \mathbf{v}_{dq0} \right] \nonumber\\[3mm] 
	&= C\left[ \dfrac{d}{dt}\left(\mathbf{v}_{dq0}\right) + \mathbf{T_P}\left(\theta\right)\dfrac{d}{dt}\left(\mathbf{T}^{-1}_\psi \right) \mathbf{v}_{dq0} \right] \nonumber\\[3mm]
	&\substack{\text{(Lemma \ref{lemma:t_ndifftminus_product})} \\ =}\hspace{2mm} C\left\{ \dfrac{d}{dt}\left(\mathbf{v}_{dq0}\right) + \dfrac{d\psi}{dt} \left[\begin{array}{ccc}    0 & -1 & 0 \\[5mm] 1 & 0  & 0 \\[5mm]  0 & 0  & 0 \end{array}\right] \mathbf{v}_{dq0} \right\} \nonumber\\[3mm]
%
%
&= \left[\begin{array}{l}
        C\dfrac{dv_d}{dt} - \omega C v_q \\[5mm]
        C\dfrac{dv_q}{dt} + \omega C v_d \\[5mm]
        C\dfrac{dv_0}{dt}
\end{array}\right]
\end{align}

	Now, because $\mathbf{v}$ is a balanced 3$\phi$ voltage, $v_0 \equiv 0$ and $V = \mathbf{P^{\omega}_{3\phi}}\left[\mathbf{v}\right] = v_d + jv_q$ completely describes $\mathbf{v}$, and the complex equation

\begin{equation} I = C\dfrac{dV}{dt} + j\omega(t)C V \end{equation}

	\noindent is such that $I$ is the phasor representation $\mathbf{P_{3\phi}}\left[\mathbf{i}\right]$ of $\mathbf{i}$.  \hfill$\blacksquare$

\vspace{5mm}
\hrule
\vspace{5mm}
%>>>

\begin{theorem}[Time-dependant 3$\phi$ inductive impedance]\label{theo:3p_inductive_impedance} %<<< 
Let $\mathbf{i} = \left[i_a,i_b,i_c\right]$ be a balanced 3$\phi$ current across a three-phase bank of inductors of value $L$, like in the figure below. Denote $I = \mathbf{P^{\omega}_{3\phi}}\left[\mathbf{i}\right] = i_d\left(t\right) + ji_q\left(t\right)$ as the corresponding phasor of $\mathbf{i}$, $\omega$ as its apparent frequency and $\psi = \int_{0}^{t} \omega(x)dx$. Also let $\mathbf{T}_\psi$ be the $dq0$ transform matrix at $\psi(t)$.

% THREEPHASE INDUCTOR BANK <<<
\begin{center}
        \begin{tikzpicture}[american,scale=1,transform shape,line width=0.75, cute inductors, voltage shift = 1,>={Stealth[inset=0mm,length=1.5mm,angle'=50]}]
	\ctikzset{/tikz/circuitikz/voltage/distance from node=10mm}
		\draw (0,0) to [short,f>_=$i_a$,o-] ++(1,0) to [L,l=$L$, -o] ++(4,0);
		\draw (1,-0.5) to [open, v^=$v_a(t)$] ++(4,0) ;
%
		\draw (0.5,-2) to [short,f>_=$i_b$,o-] ++(1,0) to [L,l=$L$, -o] ++(4,0);
		\draw (1.5,-2.5) to [open, v^=$v_b(t)$] ++(4,0) ;
%
		\draw (1,-4) to [short,f>_=$i_c$,o-] ++(1,0) to [L,l=$L$, -o] ++(4,0);
		\draw (2,-4.5) to [open, v^=$v_c(t)$] ++(4,0) ;
        \end{tikzpicture}
\end{center} %>>>

	Then the 3$\phi$ voltage across the inductors $\mathbf{v} = \left[v_a,v_b,v_c\right]$ is such that

\begin{align}
\left\{\begin{array}{l}
        v_d = L\dfrac{di_d}{dt} - \omega L i_q \\[5mm]
        v_q = L\dfrac{di_q}{dt} + \omega L i_d \\[5mm]
        v_0 = 0
\end{array}\right.
\end{align}

	Therefore the phasor

\begin{equation} V = L\dfrac{dI}{dt} + j\omega(t)L I \end{equation}

	is equal to the phasor representation of $\mathbf{v}$, $\mathbf{P^{\omega}_{3\phi}}\left[\mathbf{v}\right] = v_d(t) + jy_q(t)$.
\end{theorem}
\textbf{Proof:} writing the time differential equations,

\begin{equation} \mathbf{i} = 
\left[\begin{array}{c} v_a \\ v_b \\ v_c \end{array}\right] = 
\left[\begin{array}{c} L\dfrac{di_a}{dt} \\[5mm] L\dfrac{di_b}{dt} \\[5mm] L\dfrac{di_c}{dt} \end{array}\right] \Leftrightarrow
 \mathbf{v} = L\dfrac{d\mathbf{i}}{dt} \end{equation}

	Applying the $dq0$ to both sides at the angle $\psi$,

\begin{align}
	\mathbf{v}_{dq0} &= \mathbf{T}_\psi \mathbf{v} \nonumber\\[3mm]
	=& \mathbf{T}_\psi L\dfrac{d\mathbf{i}}{dt} \nonumber\\[3mm] 
	\substack{\text{(Lemma \ref{theo:dq0_3p_diff})} \\ =}\hspace{2mm}& \mathbf{T}_\psi L\left[ \mathbf{T}^{-1}_\psi \dfrac{d}{dt}\left(\mathbf{i}_{dq0}\right) + \dfrac{d}{dt}\left(\mathbf{T}^{-1}_\psi \right) \mathbf{i}_{dq0} \right] \nonumber\\[3mm] 
	=& L\left[ \mathbf{T}_\psi \mathbf{T}^{-1}_\psi  \dfrac{d}{dt}\left(\mathbf{i}_{dq0}\right) + \mathbf{T}_\psi \dfrac{d}{dt}\left(\mathbf{T}^{-1}_\psi \right) \mathbf{i}_{dq0} \right] \nonumber\\[3mm] 
	=& L\left[ \dfrac{d}{dt}\left(\mathbf{i}_{dq0}\right) + \mathbf{T}_\psi \dfrac{d}{dt}\left(\mathbf{T}^{-1}_\psi  \right) \mathbf{i}_{dq0} \right] \nonumber\\[3mm] 
	\substack{\text{(Lemma \ref{lemma:t_ndifftminus_product})}\\ =}\hspace{2mm}& L\left\{ \dfrac{d}{dt}\left(\mathbf{i}_{dq0}\right) + \dfrac{d\psi}{dt} \left[\begin{array}{ccc}    0 & -1 & 0 \\[5mm] 1 & 0  & 0 \\[5mm]  0 & 0  & 0 \end{array}\right] \mathbf{i}_{dq0} \right\} \nonumber\\[3mm]
	=&
\left[\begin{array}{c}
        L\dfrac{di_d}{dt} - \omega L i_q \\[5mm]
        L\dfrac{di_q}{dt} + \omega L i_d \\[5mm]
        L\dfrac{di_0}{dt}
\end{array}\right]
\end{align}

	Now, because $\mathbf{i}$ is a balanced 3$\phi$ current, $v_0 = 0$ and $I = \mathbf{P^{\omega}_{3\phi}}\left[\mathbf{i}\right]$ completely describes $\mathbf{i}$ and the complex equation

\begin{equation} V = L\dfrac{dI}{dt} + j\omega(t)L I,  \end{equation}

	\noindent is such that $V$ is the phasor representation $\mathbf{P_{3\phi}}\left[\mathbf{v}\right]$ of $\mathbf{v}$.  \hfill$\blacksquare$

\vspace{5mm}
\hrule
\vspace{5mm}
% >>>

	And we now use theorems \ref{theo:3p_capacitive_conductance} and \ref{theo:3p_inductive_impedance} to yield an exemplary modelling of a three-phase system.

\begin{example}[Dynamic Phasor modelling of a three-phase inverter-based Power System]\label{example:3p_eps_modelling}

	Consider the circuit of figure \ref{fig:ibr_modelling_example}, comprised of an inverter device with a $LR$ current filter of inductance $L_F$ and resistance $R_F$. The inverter outputs a balanced three-phase bridge voltage $e(t)$ and a three-phase bus current $i(t)$. The system is attached to an infinite bus $V_\infty$ through a double transmission line of inductance $2L$ (resulting an inductance $L$ when both lines are operational) and resistance $R$ (resulting a resistance $R$ when both lines are operational), and the terminal voltage at the connection point is $v(t)$.

	The system is equipped with two controllers. The first controller is a synchronization block in the form of a Phase-Locked Loop, schematized in figure \ref{fig:3p_pll_curr_control}. This PLL works by estimating the frequency of $v(t)$ at the connection point, and outputs a frequency $\omega_P(t)$ that is passed to the inverter bridge, such that $e(t)$ is generated with an apparent frequency $\omega_P(t)$. The PLL works by generating a local DQ frame and rotating this frame, by adjusting $\omega_P$, so as to align the DQ frame to $V_q$. This is done by estimating $V_q$ in real time and vanishing $V_q$ through a PI controller, thus estimating the frequency of the voltage $v(t)$.

	Further, the system is controlled by the current control of figure \ref{fig:3p_curr_control}. This current control consists of two PI controllers that aim to set the phasor of the current $I(t)$ to a setpoint $I_d^* + jI_q^*$. This setpoint is supposed static for this modelling. For this example, we will prove that the PI controllers of the current control adopt high integral gains, such that the current $I(t)$ reaches the setpoint much quicker than the system reaction, meaning we can consider $I_d(t)$ and $I_q(t)$ as equal to their setpoints at all times.

% MODELLING EXAMPLE: INVERTER SYSTEM <<<
\begin{figure}[htb!]
\centering
        \begin{tikzpicture}[american,scale=1,transform shape,line width=0.75, cute inductors, voltage shift = 1,>={Stealth[inset=0mm,length=1.5mm,angle'=50]}]
	\ctikzset{/tikz/circuitikz/voltage/distance from node=10mm}
		\node [draw, minimum width=20mm, very thick, minimum height=20mm] (inv_block) at (0,0) {};
		\draw (0,-0.6) to [D] ++ (0,1.2);
		\node (econn) at (2,0) {};
		\draw (inv_block.east) -- (econn.center);
		\draw [line width=1mm] ([shift=({0,0.5})]econn.center) -- ++(0,-1);
		\node (elabel) at ([shift=({0,1})]econn.center) {$e(t)$};
		\draw (econn.center) to [short,f>=$i(t)$] ++(2,0) to [L,l=$L_F$] ++(1,0) to [R,l=$R_F$] ++(2,0) node (vconn) {};
		\draw [line width=1mm] ([shift=({0,1})]vconn.center) -- ++(0,-2);
		\node (vlabel) at ([shift=({0,1.5})]vconn) {$v(t)$};

		\node (midnode) at ($(inv_block.center)!0.5!(vconn.center)$) {};

		\draw ([shift=({0, 0.75})]vconn.center) to [short] ++(0.5,0) to [L,l=$2L$] ++(1,0) to [R,l=$2R$] ++(2,0) node (vinfup) {};
		\draw ([shift=({0,-0.75})]vconn.center) to [short] ++(0.5,0) to [L,l=$2L$] ++(1,0) to [R,l=$2R$] ++(2,0);

		\node (vinfconn) at (inv_block -| vinfup) {};
		\draw [line width=1mm] ([shift=({0,1})]vinfconn.center) -- ++(0,-2);
		\draw (vinfconn.center) to [short] ++(1,0) to [esource,l=$V_\infty$,sources/scale=1.25, name=vinfsource] ++(1,0);
		\node (vinflabel) at (vinfsource) {{\Large $\infty$}};

		\node [draw, minimum width=20mm, very thick, minimum height=20mm, below=20mm of midnode] (pll_block) {PLL};

		\draw[->] (vconn) |- ([shift=({1mm,0})]pll_block.east);
		\draw[->] (pll_block.west) -| ([shift=({0,-1mm})]inv_block.south);

		\node (omega_pll_label) at ([shift=({-10mm,3mm})]pll_block.west) {$\omega_P(t)$};
        \end{tikzpicture}
	\caption{Inverter-based circuit for example modelling of nonstationary three-phase system.}
	\label{fig:ibr_modelling_example}
\end{figure} %>>>

	Before modelling this system, one needs to get a full grasp of all references and phasorial representations involved. Naturally, the device responsible for generating the angle and time references is the synchronization device, the PLL. When the PLL is turned on and starts counting time at $t=0$, it essentially generates two frames: a static real-imaginary frame and a mobile DQ frame. The DQ frame starts exactly in phase with the real-imaginary frame at $t=0$, and rotates at the time-varying $\omega_P$ angular frequency that is the PLL estimation of the frequency of $v(t)$. What the PLL essentially does is adjust the DQ frame, through the frequency estimation, so that the DQ frame is in phase with the Dynamic Phasor $V(t)$ of $v(t)$, generated against that same DQ frame. This is done by vanishing quadrature signal $V_q(t)$ through a PI controller which output is a frequency deviation $\Delta\omega(t)$, which is then added to the synchronous frequency $\omega_0$ to generate the frequency estimation $\omega_P$. As such, if $V_q > 0$, this means that $V(t)$ is ahead of the DQ frame, and $\Delta\omega$ rises to match the DQ frame to the voltage; conversely, if $V_q < 0$ then $V(t)$ is behind the DQ frame, and $\Delta\omega$ lowers to match the frame to the voltage.

% PLL SUBSTYSTEM <<<
\begin{figure} 
\centering
\scalebox{1}{
\begin{tikzpicture}[scale=1,>={Stealth[inset=0mm,length=1.5mm,angle'=50]}]

\node[left] at (0,0) (signalinputb) {$v_b(t)$};
\node[left] at ([shift=({0, 5mm})]signalinputb.east) (signalinputa) {$v_a(t)$};
\node[left] at ([shift=({0,-5mm})]signalinputb.east) (signalinputc) {$v_c(t)$};

\node [draw, minimum width=2cm, very thick, minimum height=2cm, right=1cm of signalinput] (abdq_block) {};

\draw[->] (signalinputa.east) -- ([shift=({-1mm,0})] signalinputa -| abdq_block.west);
\draw[->] (signalinputb.east) -- ([shift=({-1mm,0})] signalinputb -| abdq_block.west);
\draw[->] (signalinputc.east) -- ([shift=({-1mm,0})] signalinputc -| abdq_block.west);

\draw (abdq_block.south west) -- (abdq_block.north east);
\node at ([shift=({{ 2cm*sqrt(2)/4},{-2cm*sqrt(2)/4}})]abdq_block.north west) {\large $abc$};
\node at ([shift=({{-2cm*sqrt(2)/4},{ 2cm*sqrt(2)/4}})]abdq_block.south east) {\large $dq0$};

\draw[->] ([shift=({0, 3.33mm})]abdq_block.east) -- ([shift=({10mm,3.33mm})]abdq_block.east) node[at end, above] {$V_d(t)$};

\node (arrowpi) at ([shift=({30mm,-3.33mm})]abdq_block.east) {};
\node at (arrowpi) [draw, minimum width=2cm, very thick, minimum height=10mm] (pi_block) {$\xi_P + \dfrac{\xi_I}{s}$};

\draw[->] ([shift=({0,-3.33mm})]abdq_block.east) -- ([shift=({-1mm,0})]pi_block.west) node[midway, below] {$V_q(t)$};

\node (pilabel) at ([shift=({0mm,2mm})]pi_block.north) {PI Controller};

\node[draw, circle, very thick, minimum size=10mm, right=2cm of pi_block] (sum) {}; % SUM CIRCLE
\node (omegazero_node) at ([shift=({0mm,15mm})]sum.north) {$\omega_0$};
\draw[->] (omegazero_node.south) -- ([shift=({0mm,1mm})]sum.north);
\node at ([shift=({3mm,2mm})]sum.north) {$+$};

\draw[->] (pi_block.east) -- ([shift=({-1mm,0})]sum.west) node[midway, below] {$\Delta\omega$};
\node at ([shift=({-2mm,3mm})]sum.west) {$+$};

\node [right=8mm of pi_block] (midway_deltaomega) {};
\node [draw, very thick, minimum width=1cm, minimum height=1cm, above=15mm of midway_deltaomega] (angle_integrator) {$\dfrac{1}{s}$};
\draw[->] (midway_deltaomega.center) -- ([shift=({0mm,-1mm})]angle_integrator.south);
\draw[->] (angle_integrator.north) -- ([shift=({0mm, 10mm})]angle_integrator.north);
\node at ([shift=({0,12mm})]angle_integrator.north) {$\Delta\psi(t)$};

\node [right=10mm of sum] (midway_omega) {};
\node [draw, very thick, minimum width=1cm, minimum height=1cm, below=10mm of midway_omega] (psi_integrator) {$\dfrac{1}{s}$};
\draw[->] (sum.center -| psi_integrator) -- ([shift=({0mm,1mm})]psi_integrator.north);

\node [below=10mm of psi_integrator] (midway_psi) {};
\draw[-] (psi_integrator.south) -- (midway_psi.center);
\draw[->] (midway_psi.center) -| ([shift=({0mm,-1mm})]abdq_block.south);

\node [right, right=10mm of midway_omega] (omeganode) {$\omega_P$};
\draw[->] (sum.east) -- ([shift=({-3mm,0mm})]omeganode.center);

\node [right=10mm of midway_psi] (psinode) {$\psi$};
\draw[->] (midway_psi.center) -- ([shift=({-2mm,0mm})]psinode.center);
\end{tikzpicture}
}
\caption{Three-phase Phase Locked Loop synchronization subsystem for the circuit of figure \ref{fig:ibr_modelling_example}.}
\label{fig:3p_pll_curr_control}
\end{figure}
%>>>

	Insofar as the DQ frame and the real-imaginary frames are locally generated, to complete the modelling one needs to consider the angle reference of the grid, which is defined by the infinite bus voltage $v_\infty$. This voltage has by definition a static frequency $\omega_0$, and a constant amplitude. This means that with respect to the static real-imaginary frame the vector $V_\infty$ has constant amplitude and rotates at a fixed frequency $\omega_0$. More importantly, this voltage has a fixed phase with respect to the synchronous reference of the grid; the problem here being that the PLL subsystem has no knowledge of the grid reference, and it must be estimated. In order to do this, a vector $R$ for ``reference'' is generated; this vector also starts in phase with the real axis and spins at the synchronous frequency $\omega_0$ and simulates the synchronous grid reference with respect to the local DQ frame, such that by definition the angle displacement between $R$ and $V_\infty$ is a fixed $\phi_0$.	The phasorial diagram is shown in figure \ref{fig:dynamic_phasor_dqaxis_ibr}.

	By definition, the DQ frame and the synchronous reference $R$ have an angle displacement that is given by 

\begin{equation} \Delta\psi(t) = \int_0^t \Delta\omega(s) ds = \int_0^t \left[\omega_P(s) - \omega_0\right]ds, \end{equation}

	\noindent thus measuring how advanced the DQ frame is with respect to the real reference vector $R$. Because $V_\infty$ has a constant angle diference $\phi_0$ with respect to $R$, then naturally it starts at $t=0$ as $V_\infty = \left\lvert V_\infty\right\rvert e^{j\phi_0}$, meaning that with respect to the static frame it is described in time as $V_\infty = \left\lvert V_\infty\right\rvert e^{j\phi_0}e^{j\omega_0 t}$. Thus it  has an angle displacement with the DQ frame of $\phi_0 + \Delta\psi(t)$. Therefore, with respect to the DQ frame, the infinite bus voltage is modelled as

\begin{equation} V_\infty = \left\lvert V_\infty\right\rvert e^{j\left(\phi_0 + \Delta\psi(t)\right)} \end{equation}

	\noindent and $\phi_0$ is calculated from the initial conditions of the system.  If the entire diagram is spun by $-j\psi(t)$, placing the DQ frame as the reference frame, one achieves the representation of all quantities with respect to the DQ frame, as shown in figure \ref{fig:dynamic_phasor_dqaxis_ibr_dqframe} which is a copy of figure \ref{fig:dynamic_phasor_dqaxis_ibr} but with all quantities rotated by $-j\psi(t)$.

% DYNAMIC PHASOR DIAGRAM OF THREE-PHASE SYSTEM <<<
\begin{figure}[htb!]
\centering
\scalebox{0.8}{
	\begin{tikzpicture}[scale=2,>={Stealth[inset=0mm,length=1.5mm,angle'=50]}]
		\draw [->] (   -2mm,  0   ) -- (   40mm,  0   ) node (xaxis) {};
		\draw [->] (      0, -2mm ) -- (   0   ,  40mm) node (yaxis) {};

		\node (reAxisLabel) at (42mm,0) {Re};
		\node (imAxisLabel) at (0,42mm) {Im};

		\draw [->, black!50] (0,0) -- ({40mm*cos(25)}, {40mm*sin(25)});
		\draw [->, black!50] (0,0) -- ({40mm*cos(115)},{40mm*sin(115)});

		\node[label={[text=stewartpink, label distance=1mm]0:$Re^{j\omega_0 t}$}] (ReAxisLabel) at ({40mm*cos(10)} ,{40mm*sin(10)})  {};
		\draw[->, stewartpink] (0,0) -- (ReAxisLabel.center);
		\draw [->, stewartpink] ({25mm*cos(10)},{25mm*sin(10)}) arc[start angle=10, end angle = 23, radius = 25mm];
		\node [stewartpink] (philabel) at ({28mm*cos(17)},{28mm*sin(17)}) {$\Delta\psi(t)$};

		\node[right,black!50] (DAxisLabel) at ({42mm*cos(25) - 2mm} ,{42mm*sin(25)})  {$D$};
		\node[black!50]       (QAxisLabel) at ({42mm*cos(115)},      {42mm*sin(115)}) {$Q$};

		\node [black!50] (omegat) at ({35mm*cos(32)},{35mm*sin(32)}) {$\omega_P(t)$};
		\draw [-{Stealth[inset=0mm,length=3.5mm,angle'=50]}, black!50, line width = 1mm] ({35mm*cos(20)},{35mm*sin(20)}) arc[start angle=20, end angle = 30, radius = 35mm];

		\node [stewartpink] (omegat) at ({35mm*cos(17)},{35mm*sin(17)}) {$\omega_0$};
		\draw [-{Stealth[inset=0mm,length=3.5mm,angle'=50]}, stewartpink, line width = 1mm] ({35mm*cos(5)},{35mm*sin(5)}) arc[start angle=5, end angle = 15, radius = 35mm];

		\draw [->,black!50,thick] ({38mm*cos(0)},{38mm*sin(0)}) arc[start angle=0, end angle = 23, radius = 38mm];
		\node [right,gray] (psilabel) at ({39mm*cos(18)},{39mm*sin(18)}) {$\psi(t)$};

		\node [right, stewartyellow] (elabel) at ({35mm*cos(65)},{35mm*sin(65)}) {$E(t)e^{j\psi(t)}$};
		\draw [->, stewartyellow] (0,0) -- (elabel);

		\node[right, stewartblue] (vlabel) at ({40mm*cos(45)},{40mm*sin(45)}) {$V(t)e^{j\psi(t)}$};
		\draw [->, stewartblue] (0,0) -- (vlabel);
		\draw [->, stewartblue] ({25mm*cos(25)},{25mm*sin(25)}) arc[start angle=25, end angle=40, radius = 25mm];
		\node [stewartblue] (phivlabel) at ({28mm*cos(32)},{28mm*sin(32)}) {$\phi_v(t)$};

		\node[label={[text=stewartgreen, label distance=1mm]0:$\left\lvert V_\infty\right\rvert e^{j\phi_0}e^{j\omega_0 t}$}] (vinflabel) at ({42mm*cos(-20)},{42mm*sin(-20)}) {};
		\draw [->,   stewartgreen] (0,0) -- (vinflabel.center);
		\draw [->,   stewartgreen] ({30mm*cos(10)},{30mm*sin(10)}) arc[start angle=10, end angle = -18, radius = 30mm];
		\node [stewartgreen] (philabel) at ({33mm*cos(-15)},{33mm*sin(-15)}) {$\phi_0$};

		\node[right, stewartpurple] (ilabel) at ({35mm*cos(-10)},{35mm*sin(-10)}) {$I(t)e^{j\psi(t)}$};
		\draw [->,   stewartpurple] (0,0) -- (ilabel);
	\end{tikzpicture}
	}
	\caption
[Phasor diagram for the system being studied in the real-imaginary static frame.]
{Phasor diagram for the system being studied in the real-imaginary static frame. Note: in this scenario, $\phi_0$ is negative for clarity of the schematic.}
	\label{fig:dynamic_phasor_dqaxis_ibr}
\end{figure} %>>>

% DYNAMIC PHASOR DIAGRAM OF THREE-PHASE SYSTEM IN THE DQ FRAME <<<
\begin{figure}[htb!]
\centering
\scalebox{0.8}{
	\begin{tikzpicture}[scale=2,>={Stealth[inset=0mm,length=1.5mm,angle'=50]}]
		\draw [->, black!50] (   -2mm,  0   ) -- (   40mm,  0   ) node (xaxis) {};
		\draw [->, black!50] (      0, -2mm ) -- (   0   ,  40mm) node (yaxis) {};

		\node[right,black!50] (DAxisLabel) at ({42mm*cos(0) - 2mm} ,{42mm*sin(0)})  {$D$};
		\node[black!50]       (QAxisLabel) at ({42mm*cos(90)},      {42mm*sin(90)}) {$Q$};

		\node [label={[text=stewartpink, label distance=1mm]0:$Re^{-j\Delta\psi(t)}$}] (ReAxisLabel) at ({42mm*cos(-15)} ,{42mm*sin(-15)})  {};
		\draw [->, stewartpink] (0,0) -- (ReAxisLabel.center);
		\draw [->, stewartpink] ({20mm*cos(-15)},{20mm*sin(-15)}) arc[start angle=-15, end angle = -2, radius = 20mm];
		\node [stewartpink] (philabel) at ({23mm*cos(-8)},{23mm*sin(-8)}) {$\Delta\psi(t)$};

		%\node [black!50] (omegat) at ({35mm*cos(7)},{35mm*sin(7)}) {$\omega(t)$};
		%\draw [-{Stealth[inset=0mm,length=3.5mm,angle'=50]}, black!50, line width = 1mm] ({35mm*cos(-5)},{35mm*sin(-5)}) arc[start angle=-5, end angle = 5, radius = 35mm];

		\node [stewartpink] (omegat) at ({38mm*cos(-8)},{38mm*sin(-8)}) {$\omega_0 - \omega_P(t)$};
		\draw [-{Stealth[inset=0mm,length=3.5mm,angle'=50]}, stewartpink, line width = 1mm] ({38mm*cos(-20)},{38mm*sin(-20)}) arc[start angle=-20, end angle = -10, radius = 38mm];

		\node[right, stewartyellow] (elabel) at ({35mm*cos(40)},{35mm*sin(40)}) {$E(t)$};
		\draw [->, stewartyellow] (0,0) -- (elabel);

		\node[right, stewartblue] (vlabel) at ({40mm*cos(20)},{40mm*sin(20)}) {$V(t)$};
		\draw [->, stewartblue] (0,0) -- (vlabel);
		\draw [->, stewartblue] ({25mm*cos(0)},{25mm*sin(0)}) arc[start angle=0, end angle = 17, radius = 25mm];
		\node [stewartblue] (phivlabel) at ({28mm*cos(8)},{28mm*sin(8)}) {$\phi_v(t)$};

		\node[label={[text=stewartgreen, label distance=1mm]0:$\left\lvert V_\infty\right\rvert e^{j\left(\phi_0 + \Delta\psi(t)\right)}$}] (vinflabel) at ({42mm*cos(-45)},{42mm*sin(-45)}) {};
		\draw [->,   stewartgreen] (0,0) -- (vinflabel.center);
		\draw [->,   stewartgreen] ({30mm*cos(-15)},{30mm*sin(-15)}) arc[start angle=-15, end angle = -43, radius = 30mm];
		\node [stewartgreen] (philabel) at ({33mm*cos(-40)},{33mm*sin(-40)}) {$\phi_0$};

		\node[right, stewartpurple] (ilabel) at ({35mm*cos(-35)},{35mm*sin(-35)}) {$I(t)$};
		\draw [->,   stewartpurple] (0,0) -- (ilabel);

	\end{tikzpicture}
	}
	\caption
[Phasor diagram for the system being studied in the mobile DQ frame.]
{Phasor diagram for the system being studied in the mobile DQ frame. Note: in this scenario, $\phi_0$ is negative for clarity of the schematic.}
	\label{fig:dynamic_phasor_dqaxis_ibr_dqframe}
\end{figure} %>>>

	We first model the circuit.  It is only natural to adopt the frequency signal $\omega_P$ for the Dynamic Phasor Transform of the modelling. Once this is set, one can start modelling the system through the Dynamic Phasor relationships of theorems \ref{theo:3p_capacitive_conductance} and \ref{theo:3p_inductive_impedance}, obtaining

\begin{equation} \left\{\begin{array}{l} E = V + R_F I + L_F\left(\dot{I} + j\omega L I\right) \\[3mm] V = \left|V\right|_\infty e^{j\left(\phi_0 + \Delta\psi\right)} + RI + L\left(\dot{I} + j\omega L I\right) \end{array}\right. \label{eq:circuit_3pmodel_complex} \end{equation}

	\noindent and separating the system into direct and quadrature components,

\begin{equation} \left\{\begin{array}{l}E_d = V_d + R_FI_d - \omega L_FI_q + L_F \dot{I}_d\\[2mm] E_q = R_FI_q + \omega L_FI_d + V_q + L_F \dot{I}_q\\[2mm] V_d = \left|V\right|_\infty\cos\left(\phi_0 + \Delta\psi\right) + RI_d - \omega L I_q + L \dot{I}_d \\[2mm] V_q = RI_q + \omega LI_d - \left|V\right|_\infty\sin\left(\phi_0 + \Delta\psi\right) +  L \dot{I}_q\end{array}\right. \label{eq:circuit_3pmodel_vq} \end{equation}

	Where $\omega$ is the apparent frequency signal adopted, which is understood as being $\omega_P$ from now on. Using these equations we can achieve a model of the PLL. The original PLL equations are

\begin{equation} \left\{\begin{array}{l} \dot{\left(\Delta\psi\right)} = \omega_P - \omega_0 \\[3mm] \dot{\omega}_P = \xi_IV_q + \xi_P\dot{V}_q \end{array}\right. \end{equation}

	\noindent and one obtains both $V_q$ and $\dot{V}_q$ from the electrical equations \eqref{eq:circuit_3pmodel_vq}:

\begin{gather}
	\hspace{-5cm} \dfrac{d\omega_P}{dt} = \xi_I\left[ RI_q + \omega_P L I_d - \left|V\right|_\infty\sin\left(\phi_0+\Delta\psi\right)\right] + \nonumber\\[3mm] \hspace{3cm} + \xi_P\left[  R \dfrac{dI_q}{dt} + L\omega_P \dfrac{dI_d}{dt} + L \dfrac{d\omega_P}{dt}I_d + \left|V\right|_\infty\dfrac{d\Delta\psi}{dt}\cos\left(\phi_0+\Delta\psi\right) \right]
\end{gather}

	\noindent now considering that $d\left(\Delta\psi\right)/dt = \omega_P$ and isolating $d\omega_P/dt$,

\begin{gather}
        \left(1-\xi_P LI_d\right)\dfrac{d\omega_P}{dt} = \xi_I\left[ RI_q + \omega_P L I_d - \left|V\right|_\infty\sin\left(\phi_0+\Delta\psi\right)\right] + \\[3mm] \hspace{25mm} +  \xi_P\left[  R \dfrac{dI_q}{dt} + L\omega_P \dfrac{dI_d}{dt}\right] + \xi_P\left[ \left|V\right|_\infty\Delta\omega \cos\left(\phi_0+\Delta\psi\right) \right] \nonumber\\[5mm]
%
        \dfrac{d\omega_P}{dt} = \xfrac{5mm}{3mm}{\xi_I\left[ RI_q + \omega L I_d - \left|V\right|_\infty\sin\left(\phi_0+\Delta\psi\right)\right] + \xi_P\left(R \dfrac{dI_q}{dt} + L\omega_P \dfrac{dI_d}{dt}\right) + \xi_P\left[\raisebox{5mm}{} \left|V\right|_\infty\Delta\omega \cos\left(\phi_0 + \Delta\psi\right) \right]}{1-\xi_PLI_d} \label{eq:diffomega_diq_did}
\end{gather}

	This equation makes the dynamic modelling of the PLL-controlled system; the equations for $I_d$ and $I_q$ and their derivatives are needed to complete the model, which we obtain from the model of the current control. The current control works as follows: a setpoint $I_d^* + jI_q^*$ is supplied to the current control, and the control supplies references $E^*_d + jE^*_q$ for the direct and quadrature components of the bridge voltage $e(t)$ so that the bridge acts to impose a three-phase voltage $\left[e_a,e_b,e_c\right]$. We suppose that the switching bridge is fast enough so that the voltage $e(t)$ is immediately set to the setpoints, that is, $E_d = E^*_d$ and $E_q = E^*_q$ at all times.

% CURRENT CONTROL SYSTEM <<<
\begin{figure} 
\centering
\scalebox{0.75}{
\begin{tikzpicture}[american,scale=1,transform shape,line width=0.75, cute inductors, voltage shift = 1,>={Stealth[inset=0mm,length=1.5mm,angle'=50]}]
\node (origin) at (0,0) {};

\node [draw, very thick, isosceles triangle, minimum height=15mm, minimum width=15mm] at (0mm, 15mm) (omegaLGainD) {$\omega L_F$};
\node [draw, circle, very thick, minimum size=10mm, right=20mm of omegaLGainD] (sumDomega) {}; % SUM CIRCLE
\node at ([shift=({-2mm,-3mm})]sumDomega.west) {$-$};
\node at ([shift=({-3mm, 2mm})]sumDomega.north) {$+$};

\node [draw, very thick, isosceles triangle, minimum height=15mm, minimum width=15mm] at (0mm,-15mm) (omegaLGainQ) {$\omega L_F$};
\node [draw, circle, very thick, minimum size=10mm, right=20mm of omegaLGainQ] (sumQomega) {}; % SUM CIRCLE
\node at ([shift=({-2mm, 3mm})]sumQomega.west) {$+$};
\node at ([shift=({-3mm,-2mm})]sumQomega.south) {$+$};

\node at (-10mm, 40mm) [draw, very thick, isosceles triangle, minimum height=15mm, minimum width=15mm] (rGainUp) {$R_F$};
\node at (-10mm,-40mm) [draw, very thick, isosceles triangle, minimum height=15mm, minimum width=15mm] (rGainDown) {$R_F$};

\node at (-40mm, 60mm) [draw, circle, very thick, minimum size=10mm] (sumDin) {}; % SUM CIRCLE
\node at ([shift=({-2mm, 3mm})]sumDin.west) {$+$};
\node at ([shift=({-3mm,-2mm})]sumDin.south) {$-$};

\node at (-40mm,-60mm) [draw, circle, very thick, minimum size=10mm] (sumQin) {}; % SUM CIRCLE
\node at ([shift=({-2mm,-3mm})]sumQin.west) {$+$};
\node at ([shift=({-3mm, 2mm})]sumQin.north) {$-$};

\node at (60mm, 60mm) [draw, circle, very thick, minimum size=10mm] (sumDoff) {}; % SUM CIRCLE
\node at ([shift=({-2mm, 3mm})]sumDoff.west) {$+$};
\node at ([shift=({-3mm,-2mm})]sumDoff.south) {$+$};

\node at (60mm,-60mm) [draw, circle, very thick, minimum size=10mm] (sumQoff) {}; % SUM CIRCLE
\node at ([shift=({-2mm,-3mm})]sumQoff.west) {$+$};
\node at ([shift=({-3mm, 2mm})]sumQoff.north) {$+$};

\path[name path=qSumInOut] (sumQin) -- (sumQoff) node [pos=0.5, midway] (piQ_block_node) {};
\path[name path=dSumInOut] (sumDin) -- (sumDoff) node [pos=0.5, midway] (piD_block_node) {};

\node at (piD_block_node) [draw, minimum width=2cm, very thick, minimum height=10mm] (piD_block) {$k_P^d + \dfrac{k_I^d}{s}$};
\node at (piQ_block_node) [draw, minimum width=2cm, very thick, minimum height=10mm] (piQ_block) {$k_P^q + \dfrac{k_I^q}{s}$};

\node (idIn) [left=60mm of omegaLGainD]    {$I_d$};
\node (iqIn) [left=60mm of omegaLGainQ]  {$I_q$};
\node (idInBreak) [right=40mm of idIn] {};
\node (iqInBreak) [right=40mm of iqIn] {};

\path[name path=northsouth] (sumDin) -- (sumQin);

\path[name path=Dhor] (idIn) -- (omegaLGainD);
\path[name path=Qhor] (iqIn) -- (omegaLGainQ);
\path[name path=DhorUp, very thick] (rGainUp.west) -- ([shift=({-50mm,0})]rGainUp.west);
\path[name path=QhorDown, very thick] (rGainDown.west) -- ([shift=({-50mm,0})]rGainDown.west);

\path [name intersections={of=northsouth and Dhor,by=idInBreak}];
\path [name intersections={of=northsouth and Qhor,by=iqInBreak}];
\path [name intersections={of=northsouth and DhorUp,by=idInBreakRGain}];
\path [name intersections={of=northsouth and QhorDown,by=iqInBreakRGain}];

\node (idRefIn) [left=20mm of sumDin] {$I_d^*$};
\node (iqRefIn) [left=20mm of sumQin] {$I_q^*$};

\draw[->] ([shift={(4mm,0mm)}]idRefIn) to ([shift=({-2mm,0})]sumDin.west);
\draw[->] ([shift={(4mm,0mm)}]iqRefIn) to ([shift=({-2mm,0})]sumQin.west);

\draw[->] (idInBreak.center) to ([shift=({-17mm,0})]omegaLGainQ.west) to ([shift=({-2mm,0})]omegaLGainQ.west);
\draw[->] (iqInBreak.center) to ([shift=({-17mm,0})]omegaLGainD.west)   to ([shift=({-2mm,0})]omegaLGainD.west)  ;

\draw[->] ([shift={(4mm,0mm)}]idIn.center) -| ([shift={(0mm,-2mm)}]sumDin.south);
\draw[->] ([shift={(4mm,0mm)}]iqIn.center) -| ([shift={(0mm, 2mm)}]sumQin.north);

\draw[->] (idInBreakRGain.center) to ([shift={(-2mm, 0mm)}]rGainUp.west);
\draw[->] (iqInBreakRGain.center) to ([shift={(-2mm, 0mm)}]rGainDown.west);

\draw[->] (sumDin.east) -- ([shift={(-2mm, 0mm)}]piD_block.west);
\draw[->] (piD_block.east) -- ([shift={(-2mm, 0mm)}]sumDoff.west) node[midway, above] {$V_d^*$};

\draw[->] (sumQin.east) -- ([shift={(-2mm, 0mm)}]piQ_block.west);
\draw[->] (piQ_block.east) -- ([shift={(-2mm, 0mm)}]sumQoff.west) node[midway, below] {$V_q^*$} ;
\draw[->] (sumDomega.east) -| ([shift={(0mm, -2mm)}]sumDoff.south);
\draw[->] (omegaLGainD.east) -- ([shift={(-2mm, 0mm)}]sumDomega.west);

\draw[->] (rGainUp.east) -| ([shift={(0mm, 2mm)}]sumDomega.north);
\draw[->] (rGainDown.east) -| ([shift={(0mm,-2mm)}]sumQomega.south);
\draw[->] (sumQomega.east) -| ([shift={(0mm,  2mm)}]sumQoff.north);
\draw[->] (omegaLGainQ.east) -- ([shift={(-2mm, 0mm)}]sumQomega.west);

\node [draw, minimum width=30mm, very thick, minimum height=150mm] (bridgeblock) at ([shift=({40mm,0})] origin -| sumQoff) {};

\draw ([shift={(0mm, -15mm)}]bridgeblock.center) to [D,/tikz/circuitikz/bipoles/length=3cm,line width=0.75mm] ([shift={(0mm, 15mm)}]bridgeblock.center);

\draw[->] (sumQoff.east) -- ([shift={(-1mm, 0mm)}]bridgeblock.west |- sumQoff) node[midway,below] {$E_d^*$};
\draw[->] (sumDoff.east) -- ([shift={(-1mm, 0mm)}]bridgeblock.west |- sumDoff) node[midway,above] {$E_q^*$};

\draw[->] ([shift={(0mm, 20mm)}]bridgeblock.east) -- ([shift={(10mm, 20mm)}]bridgeblock.east) node[right] {$e_a$};
\draw[->] ([shift={(0mm,  0mm)}]bridgeblock.east) -- ([shift={(10mm,  0mm)}]bridgeblock.east) node[right] {$e_b$};
\draw[->] ([shift={(0mm,-20mm)}]bridgeblock.east) -- ([shift={(10mm,-20mm)}]bridgeblock.east) node[right] {$e_c$};

\end{tikzpicture}
}
\caption{Current control subsystem for the inverter system of figure \ref{fig:ibr_modelling_example}.}
\label{fig:3p_curr_control}
\end{figure}
%>>>

	In order to enforce the current setpoints, the current controller aims to adjust the terminal voltage $V(t)$ by inputting the differences $I_d^* - I_d$ and $I_q^* - I_q$ into PI controllers and generating reference signals $V_d^*,V_q^*$:

\begin{equation} \left\{\begin{array}{l}
\dfrac{d V_d }{dt} = k_P^d\dfrac{d\left(I_d^* - I_d\right)}{dt} + k_I^d\left(I_d^* - I_d\right) \\[5mm]
%
\dfrac{d V_q}{dt} = k_P^q\dfrac{d\left(I_q^* - I_q\right)}{dt} + k_I^q\left(I_q^* - I_q\right) 
\end{array}\right. \label{eq:vdiff}
\end{equation}

	However, the system itself cannot adjust the terminal voltage; rather, it can actuate on the bridge voltage. To this extent, it supposes that the relationship between $E$ and $V$ is given by

\begin{equation} E = V + \left(R_F + j\omega L_F\right)I \left\{\begin{array}{l} E_d = V_d + R_FI_d - j\omega I_q \\[3mm] E_q = V_q + R_FI_q + j\omega I_d \end{array}\right. \label{eq:example_quasistatic_supposition}\end{equation}

	\noindent therefore yielding from \eqref{eq:vdiff}:

\begin{equation} \left\{\begin{array}{l}
\dfrac{d\left(R_FI_d - \omega L_FI_q + V_d\right)}{dt} = k_P^d\dfrac{d\left(I_d^* - I_d\right)}{dt} + k_I^d\left(I_d^* - I_d\right) + R_F\dfrac{dI_d}{dt} - \dfrac{d\left(\omega L_F I_q\right)}{dt} \\[5mm]
%
\dfrac{d\left(R_FI_q + \omega L_FI_d + V_q\right)}{dt} = k_P^q\dfrac{d\left(I_q^* - I_q\right)}{dt} + k_I^q\left(I_q^* - I_q\right) + R_F\dfrac{dI_q}{dt} + \dfrac{d\left(\omega L_F I_d\right)}{dt} 
\end{array}\right. \label{eq:evdiff}
\end{equation}

	\noindent and this generates the ``crossed signals'' seen on the current control schematic on figure \ref{fig:3p_curr_control}. This control clearly supposes a static phasor framework, highlighting its incompatibility with the Dynamic Phasor modelling. However, since this is a widely-used controller, it will be maintained for this modelling, and a better controller will be proposed later in this thesis. Developing the PI controller equations of the current controller. Using the circuit equations \eqref{eq:circuit_3pmodel_vq} on \eqref{eq:evdiff},

\begin{equation}
\left\{\begin{array}{l}
\dfrac{d}{dt}\left[\left|V\right|_\infty\cos\left(\phi_0+\Delta\psi\right) + RI_d - \omega LI_q\right] = k_P^d\dfrac{d\left(I_d^* - I_d\right)}{dt} + k_I^d\left(I_d^* - I_d\right) \\[5mm]
%
\dfrac{d}{dt}\left[R I_q + \omega_P L I_d - \left|V\right|_\infty\sin\left(\phi_0+\Delta\psi\right)\right] = k_P^q\dfrac{d\left(I_q^* - I_q\right)}{dt} + k_I^q\left(I_q^* - I_q\right) 
\end{array}\right.
\end{equation}

        And developing this system,

\begin{equation}
\left\{\begin{array}{l}
-\left|V\right|_\infty \Delta\omega\sin\left(\phi_0+\Delta\psi\right) + R\dfrac{dI_d}{dt} - L\left(\omega_P \dfrac{dI_q}{dt} + \dfrac{d\omega_P}{dt}I_q\right) = k_P^d\dfrac{d\left(I_d^* - I_d\right)}{dt} + k_I^d\left(I_d^* - I_d\right) \\[5mm]
%
R\dfrac{dI_q}{dt} + L\left(\omega_P \dfrac{dI_d}{dt} + \dfrac{d\omega_P}{dt}I_d\right) - \left|V\right|_\infty\Delta\omega\cos\left(\phi_0+\Delta\psi\right) = k_P^q\dfrac{d\left(I_q^* - I_q\right)}{dt} + k_I^q\left(I_q^* - I_q\right) 
\end{array}\right. \label{sys:idiq_system_1}
\end{equation}

        Substituting \eqref{eq:diffomega_diq_did} into the first equation,

\begin{gather}
\left(R - \dfrac{LI_q\xi_PL\omega}{1-\xi_PL_FI_d} + \xi_P^d\right)\dfrac{dI_d}{dt} - L_F\left(\omega_P + \dfrac{I_q\xi_PR}{1-\xi_PL_FI_d}\right) \dfrac{dI_q}{dt} = \nonumber\\[5mm] \hspace{30mm} = L_FI_q\left\{\xfrac{3mm}{2mm}{\xi_I\left[ RI_q + \omega_P L I_d - \left|V\right|_\infty\sin\left(\phi_0+\Delta\psi\right)\right] + \xi_P \left|V\right|_\infty\Delta\omega \cos\left(\phi_0+\Delta\psi\right)}{1-\xi_PL_FI_d}\right\} + \nonumber\\[3mm] \hspace{20mm} + \left|V\right|_\infty \Delta\omega\sin\left(\phi_0+\Delta\psi\right) + k_P^d\dfrac{dI_d^*}{dt} + k_I^d\left(I_d^* - I_d\right) \label{eq:d_diff_eq}
\end{gather}

        Now substitute \eqref{eq:diffomega_diq_did} into the second equation of \eqref{sys:idiq_system_1}:

\begin{gather}
\left(R + \dfrac{LI_dk_PR}{1-k_PL_FI_d} + k_P^q \right)\dfrac{dI_q}{dt} + L\left(\omega_P + \dfrac{I_dk_PL_F\omega_P}{1-k_PL_FI_d}\right)\dfrac{dI_d}{dt} = \nonumber\\[5mm]
%
\hspace{20mm}-LI_d\left\{\xfrac{2mm}{2mm}{\xi_I\left[ RI_q + \omega_P L I_d - \left|V\right|_\infty\sin\left(\phi_0+\Delta\psi\right)\right] + \xi_P\left|V\right|_\infty\Delta\omega \cos\left(\phi_0+\Delta\psi\right)}{1-\xi_PL_FI_d}\right\} + \nonumber\\[3mm] \hspace{20mm} + \left|V\right|_\infty\Delta\omega\cos\left(\phi_0+\Delta\psi\right) + k_P^q\dfrac{dI_q^*}{dt} + k_I^q\left(I_q^* - I_q\right) \label{eq:q_diff_eq}
\end{gather}

	Thus \eqref{eq:d_diff_eq} and \eqref{eq:q_diff_eq} form a system of equations from which $\dot{I}_d$ and $\dot{I}_q$ can be obtained. We now make a timescale argument on the controller equations: dividing \eqref{eq:d_diff_eq} by $k_I^d$ and \eqref{eq:q_diff_eq} by $k_I^q$, and then making these gains very high, one obtains $I_d^* - I_d \approx 0$ and $I_q^* - I_q \approx 0$. Reestated, by means of adoption of high integral gains for the current control the PI controllers become considerably fast so that we can consider that the components $I_d$ and $I_q$ are very close to their references at all times. Since we are adopting constant references, we can use $\dot{I}_d = \dot{I}_q = 0$ and simplify the PLL model \eqref{eq:diffomega_diq_did} as

\begin{equation} \dfrac{d\omega_P}{dt} = \xfrac{5mm}{3mm}{\xi_I\left[ RI_q + \omega L I_d - \left|V\right|_\infty\sin\left(\phi_0+\Delta\psi\right)\right] + \xi_P\left[\raisebox{5mm}{} \left|V\right|_\infty\Delta\omega \cos\left(\phi_0 + \Delta\psi\right) \right]}{1-\xi_PLI_d} \label{eq:diffomega_diq_did_simplified} \end{equation}

	Therefore, the circuit equations \eqref{eq:circuit_3pmodel_vq} and the simplified PLL equations \eqref{eq:diffomega_diq_did_simplified} form a differential-algebraic model of the system. Using these equations, we simulate a line-break fault. At $t=1s$, one of the transmission lines that ties the terminal connection $v(t)$ to the infinite bus $v_\infty(t)$ breaks, and is restored at $t=2s$. Note that the system modelling during fault is obtained by substituting $L$ by $2L$ and $R$ by $2R$ due to one line not being operational.

	The parameters and initial conditions adopted are shown in tables \ref{tab:3p_sim_params} and \ref{tab:3p_sim_init_conds}. The resulting plots of the simulation are shown in figure \ref{fig:freqsignal_3psim} for the PLL frequency $\omega_P$, \ref{fig:voltagesignals_3psim} for the terminal voltage phasor $V(t)$ and figure \ref{fig:powersimsignals} for the active and reactive power supplied by the inverter.

% TABLES OF PARAMETERS <<<
\renewcommand{\arraystretch}{1.2}
\begin{table}[h]
\begin{center}
\begin{tabular}{ c|c|c|c|c|c|c|c|c|c } 
\hline 
Parameter & $L$ & $R$ & $L_F$ & $R_F$ & $\left\lvert V_\infty \right\rvert$ & $\phi_0$ & $\xi_I$ & $\xi_P$ & $\omega_0$ \\
\hline
Value & 10mH & 0 $\Omega$ & 2H & 10m$\Omega$ & 100 V & $\pi/6$ & 10 & 5 & 120$\pi$ \\ 
\hline
\end{tabular}
\end{center}
\caption{Parameter values adopted for the three-phase system simulation.}
\label{tab:3p_sim_params}
\end{table} %>>>

% TABLES OF INITIAL CONDS <<<
\renewcommand{\arraystretch}{1.2}
\begin{table}[h]
\begin{center}
\begin{tabular}{ c|c|c|c|c} 
\hline 
Parameter & $I_d$ & $I_q$ & $V_d$ & $V_q$ \\
\hline
Value & 1A & 13.262647 A & 82.845892 V & 0 V \\ 
\hline
\end{tabular}
\end{center}
\caption{Initial values adopted for the three-phase system simulation.}
\label{tab:3p_sim_init_conds}
\end{table}
 %>>>

% FREQUENCY SIGNAL FROM SIMULATION <<<
\begin{figure}[t]
\centering
                \begin{tikzpicture}
                        \begin{axis}[
                                width  = \textwidth,
                                height = \textwidth / 1.618 ,
%                               title={Frequency signal $\omega$},
                                xlabel={$t (s)$},
                                ylabel={$\omega_P \left(\times 120\pi\right)$},
				ymajorgrids=true,
                                xmajorgrids=true,
                                xmin=0, xmax=7,
                                ymin=0.997, ymax=1.0024,
                                xtick={0,1,...,7},
                                ytick={0.997,0.998,...,1.002},
				yticklabel style={/pgf/number format/precision=3},
                                legend pos=north east,
                                legend cell align={left},
                                every axis plot/.append style={very thick, no marks},
                        ]
			\addplot[color=blue, thick] table [x index=0,y index=1,col sep=comma] {data/3psim/data_3psim.csv};
                        \end{axis}
                \end{tikzpicture}
        \caption{Resulting frequency signal of fault simulation.}
        \label{fig:freqsignal_3psim}
\end{figure}
% >>>

% VOLTAGE SIGNALS FROM SIMULATION <<<
\begin{figure}[h]
\centering
                \begin{tikzpicture}
                        \begin{axis}[
                                width  = \textwidth,
                                height = \textwidth / 1.618 ,
				ylabel near ticks,
                                xlabel={$t (s)$},
                                ylabel={$V_d,V_q\left(\times \left|V_\infty\right|\right)$},
				ymajorgrids=true,
                                xmajorgrids=true,
                                xmin=0, xmax=7,
                                ymin=-0.55, ymax=1,
                                ytick={-0.5,-0.25,...,1},
                                xtick={0,1,...,7},
                                legend pos=south east,
                                legend cell align={left},
                                every axis plot/.append style={very thick, no marks},
                        ]
			\addplot[color=blue, thick] table [x index=0,y index=2,col sep=comma] {data/3psim/data_3psim.csv};
			\addplot[color=red , thick] table [x index=0,y index=3,col sep=comma] {data/3psim/data_3psim.csv};
			\legend{$V_q(t)$,$V_d(t)$}
                        \end{axis}
                \end{tikzpicture}
        \caption{Resulting voltage signals of fault simulation.}
        \label{fig:voltagesignals_3psim}
\end{figure}
% >>>

% POWER SIGNALS <<<
\begin{figure}[h]
\centering
                \begin{tikzpicture}
                        \begin{axis}[
                                width  = \textwidth,
                                height = \textwidth / 1.618 ,
                                xlabel={$t (s)$},
                                ylabel={$P,Q \left(\times\left|V_\infty\right|I_d\right)$},
                                xmin=0, xmax=7,
                                ymin=-0.5, ymax=1,
				ymajorgrids=true,
                                xmajorgrids=true,
                                xtick={0,1,...,7},
                                ytick={-0.4,-0.2,...,0.8},
                                legend pos=south east,
                                legend cell align={left},
                                every axis plot/.append style={very thick, no marks},
                        ]
			\addplot[color=blue, thick] table [x index=0,y index=4,col sep=comma] {data/3psim/data_3psim.csv};
			\addplot[color=red, thick] table [x index=0,y index=5,col sep=comma]  {data/3psim/data_3psim.csv};
			\legend{$P(t)$,$Q(t)$}
                        \end{axis}
		\end{tikzpicture}
\vspace{-5mm}
        \caption{Resulting power signals of fault simulation.}
        \label{fig:powersimsignals}
\end{figure}
% >>>

\examplebar
\end{example}


\chapter{Effects of apparent frequency}\label{chapter:choice_apparent_frequency}

	While chapter \ref{chapter:dynamic_phasor_theory} discusses the proposed Dynamic Phasor theory in depth, we now want to analyze in further detail what are the effects of the choice and the characteristic of the apparent frequency signal $\omega(t)$ in the Dynamic Phasors produced by the transform proposed. The main motivators for this analysis are two: given that the apparent frequency signal $\omega(t)$ has to be preemptively chosen in order to apply the Dynamic Phasor Transform, one asks if there is a ``optimal'' signal that makes numerical simulations faster, or makes modelling procedures easier, while keeping the signals and systems modelled intact. Further, what happens if the circuit under study has different frequency and/or angle references like a Power System which many agents have local estimations of frequency?

	This chapter is separated into two parts. In the first part, we study what happens when a certain system is excited by a ``slow'' frequency signal, that is, the excitation of the differential equations is defined at an apparent frequency that changes slowly or almost constant. This part proves the Quasi-Static Modelling or Hypothesis: it is proven that, if the frequency signal is indeed ``almost constant'', the system can be approximated by its static phasor modelling with a degree of precision that depends on how ``quick'' the circuit is.

	In the second  part we analyze the effects of the specific choice of apparent frequency, that is, what are are the differences between phasorial differential equations obtained using two different apparent frequency signals to transform the same system. The short version of the contribution is that, as long as both frequency signals are minimally close (their difference is integrable), there is a diffeomorphism between the differential systems that they define; in this regard, these systems are somewhat equivalent.

%-------------------------------------------------
\section{Steady-state phasor approximation and timescales}\label{chapter:algebraic_solvability_timescales} %<<<1
%--------------------------------------------------------------------------------------------------

	As described in the introduction of this thesis, the Quasi-Static Hypothesis (QSH) in the context of linear circuits theory refers to the simplification of the dynamics of an electrical circuit by supposing that the circuit network is significantly quicker than the excitaton signals that power it. This allows assuming that the circuit transient behaviors can be neglected and considering the steady-state behavior to be good approximations of system dynamics. In practical terms, the QSH greatly simplifies dynamic models of electrical circuit networks by reducing model complexity and abating resources needed for numerical simulations and computations.

	The power system literature has been prolific in producing results and analysis of the QSH, due to the fact that power system dynamic models are generally large and comprise multiple subsystems working at distinct timescales; because of this, simulating dynamic behaviors over long time intervals is prohibitively time consuming and resource demanding \pcite{xiaozhewangIssuesQuasiSteadyState2013}. There are a wide plethora of studies lining QSH applications and its limits, as well as pertaning computational optimizations, with the main goal of long-term frequency and voltage stability analysis \pcite{wangAnalyticalStudiesQuasi2014} as well as transient stability studies \pcite{wangQuasiSteadystateModel2014}. In general, the hypothess is established from the full dynamic equations of the system, and the simplification is then applied to yield an approximate model \pcite{wangContinuationBasedQuasiSteadyStateAnalysis2006}. 

	It has long been known that certain transient phenomena can manifest in particular timescales, and there is a wide body of literature on the taxonomy of concepts in power system stability \pcite{hatziargyriouetal2021,farrokhabadi2020,powersystemstabilityieee/cigrejointtaskforceDefinitionClassificationPower2004,vancutsem1995,kundurPowerSystemStability1994} that emphasizes short, mid, and long term stability phenomena. The usual argument for justifying the QSH is that the circuit dynamics concentrate within the very short or short timescales (generally sub-second timeframes); in effect, for mid and long-term dynamic studies the circuit dynamics can be safely disregarded and the steady-state model is a good approximation of the network behavior. Owing to this, in most power system studies, the electrical network is modelled as a set of algebraic equations, facilitating modelling and computation by greatly reducing system complexity. 

	While the existing body of work has certainly made significant strides, one key aspect missing from the literature however is a mathematically sound and solid justification of the QSH, that is, a proof that in a ``faster'' circuit the steady-state solution of the circuit differential equations is indeed close to the actual transient solution of these differential equations. A particular reason for this gap is the fact that the majority of power system literature uses phasor-equivalent models, as opposed to electromagnetic transient models \pcite{favuzza2024}, for their capacity to express electrical quantities in terms of amplitudes and phases. Yet, the definition of ``sinusoids in transient behavior'' — with time-varying amplitude, phase and frequency —, as well as the representation of such sinusoids as Dynamic Phasors (DPs) with an equally solid mathematical background was amiss, preventing researchers to develop these concepts with the required rigor.

	We have presented in \cite{volpatoDynamicPhasorTransform2022} a robust and proof of the QSH using a theoretical framework for generalized sinusoids and their Dynamic Phasors, but we used the Short-Time Fourier Transform, as shown in theorem \ref{theo:fdp_quasi_static}, which was published in that paper. In this thesis, we shall use the framework of Dynamic Phasors proposed in chapter \ref{chapter:dynamic_phasor_theory} to achieve a mathematical modelling of a linear circuit excited with sinusoidal signals, and impose upon the model the fact that the circuit is ``quicker'' than the excitations, that is, it achieves sinusoidal steady-state faster the excitations change considerably in time. This is coupled with a generic modelling of the excitation frequency, which may depend on the circuit voltages and currents, to yield the result that as the circuit becomes faster the steady-state approximation becomes closer to the actual solutions of the circuit differential equations — mathematically validating the QSH for such circuits.

%-------------------------------------------------
\subsection{Revisiting nonstationary sinusoids and Dynamic Phasors: Sigma Spaces} %<<<2

	Given a linear time invariant differential equation

\begin{equation} \sum\limits_{k=0}^{n} \alpha_k x^{\left(k\right)} - f(t) = 0 \text{ (single-phase) or }\sum\limits_{k=0}^{n} \alpha_k \mathbf{x}^{\left(k\right)} - \mathbf{f_3}(t) = 0 \text{ (three-phase)}\end{equation}

	\noindent then by theorems \ref{corollary:complex_equivalence_phasorialodes} for the single-phase case and \ref{theo:3p_ode_solution} for the three-phase, once an apparent frequency $\omega$ is chosen, the linear system can be transformed into a phasorial equivalent differential equation.

	This begets many questions, for instance: suppose that for some particular frequency signal $\omega_1$ the differential equations have a solution. It is the case that a solution also exists for any apparent frequency signal? If so, what is the largest class of frequency signals that yield a solution?

	In order to start the analysis, we define sinusoids in a particular time interval, which allows us to understand the effects of frequency signals in a particular time interval. We further define Sigma Spaces as the spaces of sinusoids.

	Consider a closed interval $I = \left[t_0,t_F\right)$, where $t_0$ can be $-\infty$ and $t_F$ can be $+\infty$. We expand the definition of a sinusoid as a real signal $x(t)$ defined in $I$ such that there exist two signals $m(t),\delta(t)\in C\left(I\right)$ such that $x(t)$ can be written as

\begin{equation} x(t) = m(t)\cos\left(\delta(t)\right)\forall\ t\in I. \label{eq:nonstationarydef_def}\end{equation}

	Further, let $\omega(t)$ be called an \textbf{apparent frequency} signal. Then the \textbf{apparent phase of $x(t)$ respective to $\omega(t)$} is the angle $\phi_\omega(t)$ that satisfies

\begin{equation} \delta(t) = \psi(t) + \phi_\omega(t) \text{ with } \psi(t) = \int_{t_0}^t \omega(s)ds,\ \forall t\in I\label{eq:nonstationarydef_def_phi_interval}\end{equation}

	\noindent then $x(t)$ \textbf{can be represented or written at} $\omega$ in $I$. Further the space of all sinusoids at the apparent frequency $\omega$ defined in $I$ is denoted $\Sigma_\omega\left(I\right)$, or simply $\Sigma_{\omega}$ when $I$ is understood.

	A couple notes to this definition stand out. The first note is that, at a first glance, a sinusoid $x(t)$ that can be defined at a certain frequency $\omega_1$ might not be defined in another signal $\omega_2$, hence the need to define a specific space $\Sigma_\omega$ of sinusoids at the frequency $\omega$. This means that $x(t)\in\Sigma_\omega$ implying that $x(t)$ can be defined at the frequency $\omega$. Also, this definition allows for the notion of an apparent phase with respect to a particular signal $\omega$; in the case a signal can be defined in two different frequency signals $\omega_1$ and $\omega_2$, the apparent phase signals $\phi_{\left(\omega_1\right)}$ and $\phi_{\left(\omega_2\right)}$ obtained from each frequency oviously differ. Finally, the objective of defining all signals in an interval $I$ is done to be able to also express unstable signals that show increasing or otherwise exploding behavior in a finite interval $I\subsetneqq \mathbb{R}$, or simply to make the analysis in a localized interval and not the entirety of the reals.

	We now want to study the relationships between sigma spaces. Theorem \ref{theo:sigma_equivalence} proves that a signal $x(t)\in\Sigma_{\left(\omega_1\right)}$ can also be defined in another $\Sigma_{\left(\omega_1\right)}$ if the difference $\omega_1 - \omega_2$ is integrable.

\begin{theorem}\label{theo:sigma_equivalence} %<<<
	Consider two apparent frequency functions $\omega_1,\omega_2$ defined in some interval $I\subset \mathbb{R}$ such that $\Delta\omega = \omega_2 - \omega_1$ is integrable in $I$ (that is, $\Delta\psi (t) = \int_0^t \Delta\omega(s)ds$ exists and converges for all $t\in I$). Then every element $x_1\in\Sigma_{\left(\omega_1\right)}$ is also an element of $\Sigma_{\left(\omega_2\right)}$, and vice-versa. \end{theorem}
\textbf{Proof:} adopt $\Delta\omega(t) = \omega_2(t) - \omega_1(t)$. It is simple to see that $\psi_1$ and $\psi_2$ are related by

\begin{equation} \psi_2(t) - \psi_1(t) = \int_{0}^t \Delta\omega(s)ds \end{equation}

	and by hypothesis the integral exists and converges. Therefore let $x_1 = m_1(t)\cos\left(\psi_1(t) + \phi_1(t)\right)$. Then

\small\begin{equation} x_1(t) = m_1(t)\cos\left(\psi_2(t) + \phi_1(t) - \int_0^t \Delta\omega(s)ds\right) \end{equation}\normalsize

	which means $x_1$ can be represented as an element of $\Sigma_{\left(\omega_2\right)}$ with amplitude $m_1(t)$ and apparent phase $\phi_2$ relative to $\omega_2$

\small\begin{equation} \phi_2(t) = \phi_1(t) - \int_0^t \Delta\omega(s)ds \label{eq:angle_relationship}\end{equation}\normalsize

	Now take $x_2 = m_2(t)\cos\left(\psi_2(t) + \phi_2(t)\right)$ and

\small\begin{equation} x_2(t) = m_2(t)\cos\left(\psi_1(t) + \phi_2(t) + \int_0^t \Delta\omega(s)ds\right) \end{equation}\normalsize

	meaning $x_2$ can also be written as an element of $\Sigma_{\left(\omega_1\right)}$. \hfill$\blacksquare$ \vspace{5mm}\hrule\vspace{5mm}%>>>

	What theorem \ref{theo:sigma_equivalence} entails to is basically that any sinusoidal signal defined at a $\omega_1$ apprent frequency and in an interval $I$ can also be defined in any other frequency $\omega_2$, as long as $\Delta\psi$ can be defined in the entirety of $I$. This means that, given $\omega_1$ and $I$, any $\Sigma_{\left(\omega_2\right)}$ with an integrable $\Delta\psi(t)$ is exactly equal to $\Sigma_{\left(\omega_1\right)}$.

	Borrowing from Analysis and Measure Theory, since a function $f(x)$ is Lebesge integrable in $I$ if and only if it is absolutely integrable in $I$ (that is, it belongs to $L^1\left(I\right)$), the pool of nonstationary sinusoids respective to a certain ``reference'' frequency $\omega_0(t)$ in an interval $I$ is, in essence, a union of all $\Sigma_\omega$ where the difference $\omega(t) - \omega_0(t)$ is Lebesgue integrable in $I$, or conversely, $\Delta\psi$ is defined in $I$.

\begin{corollary} \label{theo:sigma_spaces_are_equal} Given $\omega_1$ and $\omega_2$ such that $\left(\omega_1(t)-\omega_2(t)\right)\in L^1\left(I\right)$, then $\Sigma_{\omega_1}^I = \Sigma_{\omega_2}^I$ .\end{corollary}
\textbf{Proof:} from theorem \ref{theo:sigma_equivalence}, if $\left(\omega_1 - \omega_2\right)\in L\left(I\right)$ then $x(t)\in\Sigma_1 \Leftrightarrow x(t)\in\Sigma_2$, which is the exact definition of equality between sets, that is, $\Sigma_{\omega_1}^I = \Sigma_{\omega_2}^I$. \hfill$\blacksquare$ \vspace{5mm}\hrule\vspace{5mm}

	Further, corolary \ref{theo:sigma_spaces_are_equal} shows that the implication that any $x(t)\in\Sigma_{\omega_1}$ is also in $\Sigma_{\omega_2}$, this means both spaces are essentially equal because they have the same elements. Finally, if two different frequency signals generate the same space in an interval $I$ — meaning they generate the same nonstationary sinusoidal functions — then we establish an equivalence relationship between the frequency signals, seen as they define the same sinusoidal signals.

\newcommand{\equivfreq}[1]{\ \substack{ #1 \\ \sim}\ }

\begin{definition}\label{def:equivalent_freqs} Given $\omega_1$ and $\omega_2$ such that $\left(\omega_1(t)-\omega_2(t)\right)\in L^1\left(I\right)$, then $\omega_1$ and $\omega_2$ are \textbf{equivalent in $I$}, denoted $\omega_1 \equivfreq{I} \omega_2$.\end{definition}

	The naming of this equivalence is intentional, since this relationship fulfills the requirements of a set equivalence. It is \textbf{reflexive} because $\omega_1 \equivfreq{I} \omega_1$ for any $I$ where $\omega_1$ is defined; it is \textbf{symmetric} since $\omega_1 \equivfreq{I} \omega_2$ if and only if $\omega_2\equivfreq{I}\omega_1$; and it is \textbf{transitive}: if $\omega_1\equivfreq{I}$ and $\omega_2\equivfreq{I}\omega_3$ then $\omega_1\equivfreq{I}\omega_3$. While reflexivity and symmery are trivial to prove, transitivity can be proven with some algebra:

\begin{equation} \left\lvert \omega_3 - \omega_1\right\rvert = \left\lvert \omega_3 - \omega_2 - \left(\omega_1 - \omega_2\right)\right\rvert \leq \left\lvert \omega_3 - \omega_2\right\rvert + \left\lvert \omega_2 - \omega_1\right\rvert\end{equation}

	\noindent and because $\omega_1\equivfreq{I}\omega_2$ and $\omega_2\equivfreq{I}\omega_3$, then the integrals of both terms on the right exists, therefore the integral of the term on the left also exists. Therefore, one can draw the conclusion that if $x(t)$ is defined at an apparent frequency $\omega_0$, then it admits a representation for any other equivalent $\omega$. 

%-------------------------------------------------
\subsection{Characteristics of $L^1\left(I\right)$}\label{subsec:characteristics_l1}

	It is natural to ask what is the largest pool of frequency signals $\omega$ that yields a $\Sigma^I_\omega$ space, and if there is a standard base to this pool so that we can know if a particular frequency signal is admissible (that is, it generates Nonstationary Sinusoid signals) and to draw further conclusions about the space and its constituents. In a first glance, one can think that the results so far lead to the fact that, given an interval $I$, any $\omega\in L^1\left(I\right)$ can be used as an apparent frequency signal — which would mean any nonstationary sinusoid defined in $I$ can be represented at $\omega$. Such is not the case, however, because the space $L^1\left(I\right)$ has no unconditional basis \pcite{lindenstrauss2013classical}, that is, there is no set of functions in $L^1\left(I\right)$ that can unconditionally generate the whole space. Even if the pool of frequency signals is reduced, so that the resulting subspace does admit an unconditional basis, because $L^1\left(I\right)$ is a Banach Space but not a Hilbert Space, that is, not every Cauchy sequence in $L^1$ converges to a limit, and this leads to many deep faults in this space.

	While these concepts from topology sound somewhat esoterical to a reader in their first contact, such concepts are not fancy as they seem. For instance, the set of polynomials of order $n$, denoted $P^n$, with the basis $\mathbf{B}_n = \left(1,x,x^2,...,x^n\right)$ can be interpreted as a point in that basis:

\begin{equation} P(x) = \sum_{k=0}^n \alpha_x k^k \Leftrightarrow \left[P\right]_{\mathbb{B}} = \left[\alpha_0,\alpha_1,...,\alpha_n\right] .\end{equation}

	Naturally, one can imagine that any infinitely differentiable function that has a Taylor Series at $x=0$ admits a representation in the space of power series $P^\infty$; for instance,

\begin{equation} e^x = \sum_{k\in\mathbb{N}} \dfrac{x^k}{k!} \Leftrightarrow \left[e^x\right]_{\mathbf{B}_\infty} = \left(1,\dfrac{1}{2},\dfrac{1}{3!},\dfrac{1}{4!},\cdots\right) = \left(\dfrac{1}{k}\right)_{k\in\mathbb{N}} .\end{equation}

	It is not difficult, however, to find signals that are infinitely differentiable at $x=0$ but have a non-converging Taylor Series, for instance,

\begin{equation} f(x) = \left\{\begin{array}{l} e^{-1/x^2} \text{, if } x\neq 0\\[3mm] 0 \text{ if } x= 0\end{array}\right. \label{eq:problem_eq_1}\end{equation}

	\noindent is infinitely differentiable at $x=0$ but its Taylor Series is divergent because all coefficients are null. Other pathological examples exist, for instance

\begin{equation} f(x) = \sum_{n\in\mathbb{N}} e^{-\sqrt{2^n}} \cos\left(2^n x\right) \label{eq:problem_eq_2}\end{equation}

	\noindent is infinitely differentiable everywhere but analytic nowhere, that is, its Taylor Function does not converge at any point. Therefore, the functions \eqref{eq:problem_eq_1} and \eqref{eq:problem_eq_2} cannot be expressed in any basis $\mathbf{B}_n$, for $n$ natural or even infinite. Maybe, one thinks, another basis (say $\mathbf{B}'$) can generate these functions; then the union $\mathbf{B}_\infty\cup\mathbf{B}'$ is the new candidate to a complete basis. Even then, there will still be some signal that cannot be expressed in that particular basis: no basis can generate the entirety of $\left[\mathbb{R}\to\mathbb{R}\right]$.

	Much the same way, because there is no inner product that induces a complete metric in $L^1$, this means that this space is ``wider'' than any inner product can express, culminating with the fact that there is no definable inner product that will induce a complete topology of $L^1$, lest a limitation of signals of interest is adopted. Therefore no useful decomposition in the scope of this analysis is available to give more information on the space of apparent frequency signals admissible.

	For simplicity, we can limit the roster to that of frequency signals we are interested in. In Power Systems, we are generally interested in frequency signals that are equivalent to a constant synchronous or reference frequency $\omega_0$; at the same time, in Modulation Theory, we are interested in frequency signals that are equivalent to a (constant) carrier frequency $\omega_0$. Therefore, we want the space of Nonstationary Sinusoids represented in an apparent frequency $\omega$ that is equivalent to a reference $\omega_0$ in a given interval $I$, that is, the space $\Sigma_{\left(\omega_0\right)}$.

	In short, ``how close'' does a signal $\omega(t)$ has to be to a synchronous frequency value $\omega_0$ to be a valid apparent frequency summarizes to their difference having to be integrable in the time interval being considered. Further, any additional consideration will require a particularization of the frequency signals, removing certain ones from the pool and weakening the analysis.

%-------------------------------------------------
\subsection{Consequences of characteristics of $\Sigma$ spaces on linear circuit theory} %<<<2

	We now investigate the consequences of the qualities of sigma spaces on the linear circuits transformed by the Dynamic Phasor Transform. In order to be able to have our analysis done in a matrix form, we define the Dynamic Phasor Transform of a vector of signals.

\begin{definition}[Dynamic Phasor Transform of a vector or sequence of sinusoids] \label{def:dptransform_vector}%<<<
	The \textbf{Dynamic Phasor Transform} (DPT) of a vector of sinusoids $\mathbf{x}\in\Sigma_{\omega}^n$ is a functional transform $\mathbf{P^\omega_D}\in \left[\Sigma_\omega^n  \to \left[\mathbb{R}\to\mathbb{C}\right]^n\right]$ where

\begin{equation} \mathbf{P_D^\omega}\left[x\right] = \left[\raisebox{3mm}{} \mathbf{P_D^\omega}\left[x_1\right],\mathbf{P_D^\omega}\left[x_2\right],...,\mathbf{P_D^\omega}\left[x_n\right]\right]^\intercal \end{equation}

	\noindent and equally with the inverse transform: given $X(t) = \left[X_1(t),X_2(t),...,X_n(t)\right]^\transpose\in\left[\mathbb{R}\to\mathbb{C}\right]^n$, define $\mathbf{P^{\left(-\omega_D\right)}}\in \left[\left[\mathbb{R}\to\mathbb{C}\right]^n \to \Sigma_\omega^n\right]$ where

\begin{equation} \mathbf{P_D^{\left(-\omega\right)}}\left[X\right] = \left[\raisebox{3mm}{} \mathbf{P_D^{\left(-\omega\right)}}\left[X_1\right],\mathbf{P_D^{\left(-\omega\right)}}\left[X_2\right],...,\mathbf{P_D^{\left(-\omega\right)}}\left[X_n\right]\right]^\transpose \end{equation}
\end{definition} %>>>

	Definition \ref{def:dptransform_vector} allows us to define the transformation for matrix systems of the type $\dot{\mathbf{x}} = \mathbf{Ax + Bf}$, which we now use to analyze electrical circuits in matrix form. A natural question induced by the complexification theorems \ref{corollary:complex_equivalence_phasorialodes} for the single-phase case and \ref{theo:3p_ode_solution} for the three-phase case is whether the currents and voltages in a passive linear circuit, when excited with multiple geberakuzed sinusoidal voltages or currents, are also generalized sinusoids. As proven in chapter \ref{chapter:classical_phasors}, this certainly is the case when a linear system is excited by static sinusoids. Further, what happens if the excitations have different apparent frequencies — like in power systems where each agent is equipped with a frequency control or adjustment, which are generally independent from other agents? Thence, imagine a linear circuit of $n$ nodes and excited by $p$ voltage and current sources (``forcings''), modelled by

\begin{equation} \dot{\mathbf{x}}(t) = \mathbf{Ax}(t) + \mathbf{Bf}(t), \label{eq:nonautodiffeq_sigma} \end{equation}

	\noindent where $\mathbf{x}(t) = \left[v_1,v_2,...,v_c,i_1,i_2,...,i_d\right]^\intercal\in \left[\mathbb{R}\to\mathbb{R}^n\right]$ is the vector of states; $\mathbf{f}(t)\in\left[\mathbb{R}\to\mathbb{R}^p\right]$ is composed of the $p$ forcings, and $\mathbf{A}\in\mathbb{R}^{(n\times n)}$ and $\mathbf{B}\in\mathbb{R}^{(n\times p)}$ are obtained through the combinations of resistance, capacitance and inductance parameters of the circuit. Furthermore, the capacitor voltages and inductor currents chosen as state variables ``sufficiently describe'' the circuit, as any node voltage or any branch current in the circuit can be obtained by some combination of the elements of $\mathbf{x}$ and its derivative.  

	Moreover, theorems \ref{theo:phasors_solutions} and \ref{theo:phasors_solutions_reproof} show that if the vector of excitations $\mathbf{f}(t)$ are sinusoidal sources of a fixed frequency $\omega$, and if the circuit has at least one resistance, then the solutions of \eqref{eq:nonautodiffeq_sigma} will be the sum of vanishing exponential terms plus a sinusoidal steady-state part composed of sinusoids at the exact frequency $\omega$. The question arises if such is the case under non-stationary conditions. To prove this true, theorem \ref{theo:sigma_invariancy} states that $\Sigma$ spaces are closed under linear combinations and differentiations; then, theorem \ref{theo:sigma_equivalence} proves that a signal of apparent frequency $\omega_2$ can be (diffeomorphically) written as a sinusoid of another $\omega_1$. This yields theorem \ref{theorem:sols_are_nonst} proving that if the forcing $\mathbf{f}(t)$ is composed of $p$ sinusoidal signals $f_i(t)$, each with its own apparent frequency $\omega_i$, they can all be written in a ``common'' frequency $\omega_0(t)$ by theorem \ref{theo:sigma_equivalence}, that is, $f(t)$ also belongs to $\Sigma_{\left(\omega_0\right)}$. 

\begin{theorem}\label{theo:sigma_invariancy}%<<<
The $\Sigma_\omega$ space is invariant under time differentiation and linear combinations, that is, for any $x_1,x_2\in\Sigma_\omega$,

\begin{itemize} \item $a(t),b(t)\in C\left(\mathbb{R}\right) \Rightarrow a(t)x_1 + b(t)x_2\in\Sigma_\omega$; \item $x\in\Sigma_\omega \Rightarrow\dot{x} \in \Sigma_\omega$\end{itemize}
\end{theorem}
\textbf{Proof:} for the linear combination, adopt \eqref{eq:nonstationarydef_def} and compute $a(t)x_1 + b(t)x_2$. With some algebra this yields

\begin{align}
	a(t)x_1 + b(t)x_2 &= \cos\left(\psi(t)\right)\left[a(t)m_1(t)\cos\left(\phi_1(t)\right) + b(t)m_2(t)\cos\left(\phi_2(t)\right)\right] + \nonumber\\[2mm] &\hspace{15mm} - \sin\left(\psi(t)\right)\left[a(t)m_1(t)\sin\left(\phi_1(t)\right) + b(t)m_2(t)\sin\left(\phi_2(t)\right)\right] \nonumber\\[2mm]
	&= \cos\left(\psi(t)\right)p(t) - \sin\left(\psi(t)\right)q(t)
\end{align}
	
	Let $c(t)$, $\alpha(t)$ such that

\begin{equation} c(t) = \left\lvert p(t) + jq(t)\right\rvert,\ \alpha(t) = \arg\left(p(t) + jq(t)\right) \end{equation}

	Then $a(t)x_1 + b(t)x_2 = c(t)\cos\left(\psi(t) + \alpha(t)\right)\in\Sigma_\omega$. For the differentiation, compute $\dot{x}$:

\begin{equation} \dot{x}(t) = \dot{m}(t)\cos\left(\psi(t) + \phi(t)\right) + m(t)\left[\omega(t) + \dot{\phi}(t)\right]\cos\left(\psi(t) + \phi(t) + \dfrac{\pi}{2} \right), \end{equation}

	\noindent which is a linear combination of vectors of $\Sigma_\omega$, thus a vector itself by the linear combination result before. \hfill$\blacksquare$
\vspace{5mm}
\hrule
\vspace{5mm}
%>>>

	Now, we revisit theorem \ref{theorem:ode_matrix_equiv} which states that the components $x_i(t)$ of the solution to \eqref{eq:nonautodiffeq_sigma} obey the n-th order differential equation

\begin{equation} \sum_{k=0}^n \alpha_k x_i^{(k)} - g_i(t) = 0, \label{eq:theo_nthnonst_end_freq}\end{equation}

	\noindent where $g_i$ is the i-th component of

\begin{equation} \mathbf{g} = \sum_{k=1}^n \alpha_k \left[\sum_{j=0}^{k-1} \mathbf{A}^j \mathbf{Bf}^{(k-j)}\right] \label{eq:theo_nthnonst_g_freq}\end{equation}

	\noindent and the $\alpha_i$ are the coefficients of the characteristic polynomial of $\mathbf{A}$. This yields theorem \ref{theorem:sols_are_nonst}: since the forcing $\mathbf{f}(t)$ can be written in some common frequency $\omega$, that is, $\mathbf{f}\in\Sigma_\omega^p$ for some $\omega(t)$, we combine this result with theorems \ref{theo:1p_ode_solution} for the single-phase case and \ref{theo:3p_ode_solution} for the three-phase case proves which prove that the phasorial and dq-equivalent ODEs defined by linear systems, when excited by sinusoidal signals at some frequency $\omega$, respond with signals at that same frequency. This means that each $x_i(t)\in\Sigma_\omega$, thus $x\in\Sigma_\omega^n$ as we wanted to prove; by Kirchoff's Laws, this implies that all voltages and currents of the system belong to $\Sigma_\omega$.

\begin{theorem}[Linear circuits excited at a frequency $\omega(t)$ respond at the same frequency]\label{theorem:sols_are_nonst} %<<<
	Suppose $\mathbf{f}$ is a vector of $p$ nonstationary sinusoids each defined at some apparent frequency $\omega_p$ where these frequency signals are mutually equivalent. Due to theorem \ref{theorem:ode_matrix_equiv} we can suppose $\mathbf{f}\in \Sigma_\omega^p$ for some $\omega(t)$ that is equivalent to all $\omega_p$. Thus the $g_i$ are linear combinations of the $f_i$ meaning $g_i\in\Sigma_\omega$. By theorems \ref{theo:1p_ode_solution} and \ref{theo:3p_ode_solution}, \eqref{eq:theo_nthnonst_end_freq} implies that the steady-state solution $x_{is}$ of each $x_i$ belongs to $\Sigma_\omega$, therefore the steady-state solution $x_s$ to $x$ is in $\Sigma_\omega^n$. Finally, because the state $x(t)$ completely describes the circuit, and any node voltage and any branch current is a linear combination of $\mathbf{x}(t)$ and its derivatives, any branch current and any node voltage in the circuit is in $\Sigma_\omega$.
\end{theorem} \vspace{5mm}\hrule\vspace{5mm}%>>>

	The result of theorem \ref{theorem:sols_are_nonst} is essentially that a linear circuit, when excited with sinusoids, will respond with currents and voltages with a steady-state sinusoidal behavior at the same apparent frequency than the excitations — therefore the steady-state solution $x_s(t)$ admits a DPT. If the chosen initial conditions reconstruct the solution, then $x(t) = x_s(t)$. Thus the differential equation \eqref{eq:nonautodiffeq_sigma} can be transformed to a DP differential equation.

\begin{theorem}\label{theo:dp_diffeq} %<<<
	Consider the differential equation \eqref{eq:nonautodiffeq_sigma} with $\mathbf{f}\in\Sigma_\omega^n$, $\mathbf{x}_s$ the steady-state solution to $\mathbf{x}(t)$, $X = \mathbf{P_D^\omega}\left[\mathbf{x}_s\right]$ and $F = \mathbf{P_D^\omega}\left[\mathbf{f}\right]$. Then $X(t)$ satisfies

\begin{equation} \dot{X} = \left(\mathbf{A} - j\omega(t)\mathbf{I_n}\right)X + \mathbf{B}F(t), \label{eq:theo_nonautodiffeq_def}\end{equation}

\end{theorem}
\textbf{Proof:} take the i-th line of \eqref{eq:theo_nonautodiffeq_def}, write $x_i(t) = m_i(t)\cos\left(\psi(t) + \phi_i(t)\right)$, and compute $\dot{x}_i$:

\begin{equation} \sum\limits_{k=1}^n a_{ik} x_k + \sum\limits_{k=1}^m b_{ik} f_k = \dot{m}_i \cos\left(\psi(t) + \phi_i(t)\right) - m_i\left[\omega(t) + \dot{\phi}_i(t)\right] \cos\left(\psi(t) + \phi_i(t) + \dfrac{\pi}{2}\right) \end{equation}

	Because the DPT is bijective, we can apply it to this entire equation; because it is linear, it can operate inside the sums and the $a_{ik},b_{ik}$ multiplications:

\begin{gather}
	\dot{m}_ie^{j\phi_i(t)} + m_i\left[\omega(t) + \dot{\phi}_i(t)\right]e^{j\left(\phi_i(t) + \frac{\pi}{2}\right)} = \sum\limits_{k=1}^n a_{ik} X_k + \sum\limits_{k=1}^m b_{ik} F_k \\[3mm] 
	\overbrace{\dot{m}_ie^{j\phi_i(t)} + jm_i \dot{\phi}_i(t)e^{j\phi_i(t)}}^{\dot{X}_i} + \overbrace{j\omega m_i e^{j\phi_i(t)}}^{jX_i(t)} = \sum\limits_{k=1}^n a_{ik} X_k + \sum\limits_{k=1}^m b_{ik} F_k
\end{gather}

	\noindent which is equivalent to $\dot{X} + j\omega X = \mathbf{A}X + \mathbf{B}F \Rightarrow \dot{X} = \left(\mathbf{A} - j\omega(t)\mathbf{I_n}\right)X + \mathbf{B}F(t)$. \hfill$\blacksquare$ \vspace{5mm}\hrule\vspace{5mm} %>>>

	Finally, theorem \ref{theo:dp_diffeq} proves that the matrix ODE \eqref{eq:nonautodiffeq_sigma} can be complexified into the complex matrix ODE \eqref{eq:theo_nonautodiffeq_def}, allowing to express a matrix system phasorially.

%------------------------------------------------- 
\section{Sigma Spaces in the phasor domain} %<<<1

	We now use the phasorial modelling of matrix system to show that two phasorial differential systems, generated by two different frequency signals from the same time-domain linear circuit, yield diffeomorphic models. We first prove that the phasorial transformations $X_1$ and $X_2$ generated from the same sinusoid $x(t)$ using two different frequencies are related by a bijection.

\begin{theorem}[DPTs at different frequencies are homeomorphic] \label{theo:homeomorphic_phasors} %<<<
	Let $I = \left(t_0,t_F\right)\in \mathbb{R}$, $x\in \Sigma^I_{\left(\omega_1\right)}$, $\omega_2\equivfreq{I}\omega_1$, and $X_1 = \mathbf{P^{\omega_1}_D}\left[x\right]$, $X_2 = \mathbf{P^{\omega_2}_D}\left[x\right]$. Define $\Delta\omega(t) = \omega_2(t) - \omega_1(t)$. Then $X_1$ and $X_2$ are related by the diffeomorphism

\begin{equation} X_2 = X_1e^{-j\Delta\psi(t)}, \text{ with } \Delta\psi(t) = \int_{t_0}^t \Delta\omega(s)ds\ \forall t\in I \label{eq:x1x2_homeomorphism}\end{equation}
\end{theorem}
\textbf{Proof:} take $x = m_1\cos\left(\psi_1(t) + \phi_1(t)\right)$. Then $X_1 = \mathbf{P_D^{\omega_1}}\left[x\right] = m_1e^{j\phi_1}$. At the same time, $X_2 = \mathbf{P_D^{\omega_2}}\left[x\right] = m_1e^{j\phi_2}$ for $\phi_2$ given by \eqref{eq:angle_relationship}; therefore $X_2$ can be written as

\begin{equation} X_2 = m_1e^{j\left(\phi_1 - \Delta\psi\right)} = m_1e^{j\phi_1}e^{-j\Delta\psi} = X_1e^{-j\Delta\psi}\end{equation}

	\noindent with $\Delta\psi$ defined as in \eqref{eq:x1x2_homeomorphism}. \hfill$\blacksquare$\vspace{5mm}\hrule\vspace{5mm}
%>>>

\begin{corollary} \label{corollary:images_sigma_diffeomorphic} The images of $\Sigma_{\omega_0}$ by two DPTs at different but equivalent frequencies are diffeomorphic, that is, if $\omega_1,\omega_2\equivfreq{I}\omega_0$ then for every element $X_1\in\mathbf{P_D^{\omega_1}}\left[\Sigma\right]$ there exists a biunivocally related element $X_2\in\mathbf{P_D^{\omega_2}}\left[\Sigma\right]$.
\end{corollary}

	In Topology, diffeomorphisms are equivalence relationships between topological spaces. In the case of corollary \ref{corollary:images_sigma_diffeomorphic}, the existence of such diffeomorphism means that the Dynamic Phasors generated by $\mathbf{P_D^{\omega_1}}\left[\Sigma\right]$ are equivalent to those generated by $\mathbf{P_D^{\omega_2}}\left[\Sigma\right]$. Such equivalence relationship is deep and far reaching, for instance, in the theory of Dynamical Systems.

\begin{theorem}[Topological equivalence between continuous Dynamical Systems \pcite{kuznetzovElementsAppliedBifurcation2023}] \label{theo:dynamical_systems_diffeomorphism}
	Let

\begin{equation} \left(D_1\right):\ \dot{x} = f\left(x,t\right),\ x\left(t_0\right) = x_0 \text{, and } \left(D_2\right):\ \dot{y} = g\left(y,t\right),\ y\left(t_0\right) = y_0 \end{equation}

	\noindent two continuous Dynamical Systems with $f = h\circ g$ for some diffeomorphism $h\in\left[\mathbb{R}^n\to\mathbb{R}^n\right]$ where $x,y\in\left[\mathbb{R}\to\mathbb{R}^n\right]$ that is, an invertible differentiable relation with a differentiable inverse, that is, the jacobian of $h$ with respect to $x$ exists and is invertible for all $x$ considered. Then these systems are \textbf{topologically equivalent}, that is, an orbit $x(t)$ of $D_1$ is biunivocally related to one orbit $y(t)$ of $D_2$ by $x = h\circ y$.
\end{theorem}

	What theorem \ref{theo:dynamical_systems_diffeomorphism} states is that if two dynamical systems are related by a diffeomorphism then their orbits are related by the same relationship, effectively making the dynamical systems equivalent — they are diffeomorphically equivalent. Due to this, we can use theorem \ref{theo:homeomorphic_phasors} and its corollary \ref{corollary:images_sigma_diffeomorphic} to show that the Dynamical Phasors obtained by solving a certain circuit in two different apparent frequencies are also equivalent.

\begin{theorem}[Models of the same circuit at different frequencies are diffeomorphic]\label{theo:diff_freqs} %<<<
	Suppose a passive linear circuit modelled by

\begin{equation} \dot{\mathbf{x}}(t) = \mathbf{Ax}(t) + \mathbf{Bf}\left(t\right), \label{eq:nonautodiffeq} \end{equation}

	\noindent where $\mathbf{x}\in\Sigma_{\left(\omega_1\right)}^n$, $\mathbf{f}\in\Sigma_{\left(\omega_1\right)}^p,\ \mathbf{A}\in\mathbb{C}^{\left(n\times n\right)},\ \mathbf{B}\in\mathbb{C}^{\left(n\times p\right)}$. Let $\omega_2\equivfreq{I}\omega_1$ in some interval $I$ and then imagine that this circuit is expressed in two different frequencies, yielding the differential equations

\begin{equation}\left\{\begin{array}{l} \dot{X}_1(t) = f_1\left(X_1,t\right) \\[3mm] \dot{X}_2(t) = f_2\left(X_2,t\right) \end{array}\right. \label{eq:two_sols_freq_announce}\end{equation}\normalsize

	Then these systems are diffeomorphic in $I$.
\end{theorem}
\textbf{Proof:} take the initial system \eqref{eq:nonautodiffeq} and transform it using the two frequency signals, yielding

\begin{equation}\left\{\begin{array}{l} \dot{X}_1(t) = \left(\mathbf{A} - j\omega_1(t)\mathbf{I}_n\right)X_1(t) + \mathbf{B}F_1\left(t\right) = f_1\left(X_1,t\right) \\[3mm] \dot{X}_2(t) = \left(\mathbf{A} - j\omega_2(t)\mathbf{I}_n\right)X_2(t) + \mathbf{B}F_2\left(t\right) = f_2\left(X_2,t\right) \end{array}\right. \label{eq:two_sols_freq}\end{equation}

	From theorem \ref{theo:homeomorphic_phasors}, define $\Delta\omega(t) = \omega_2(t) - \omega_1(t)$. Then $X_1$ and $X_2$ are related by

\begin{equation} X_2 = X_1e^{-j\Delta\psi(t)}, \text{ with } \Delta\psi(t) = \int_{t_0}^t \Delta\omega(s)ds,\ t\in I. \label{eq:homeomorphic_phasors_equiv}\end{equation}

	Thus

\begin{equation} \dfrac{dX_2}{dt} = \dfrac{d}{dt}\left[X_1e^{-j\Delta\psi(t)}\right] = \dfrac{dX_1}{dt}e^{-j\Delta\psi(t)} + X_1\dfrac{d}{dt}\left[e^{-j\Delta\psi(t)}\right] = \dfrac{dX_1}{dt}e^{-j\Delta\psi(t)} - X_1 \Delta\omega e^{-j\Delta\psi(t)}. \end{equation}

	By equations \eqref{eq:two_sols_freq}, this means

\begin{equation} f_2\left(X_2,t\right) = f_1\left(X_1,t\right)e^{-j\Delta\psi(t)} - \Delta\omega X_1 e^{-j\Delta\psi(t)} \end{equation}

	Again using \eqref{eq:homeomorphic_phasors_equiv}, this equation means

\begin{equation} f_2\left(X_2,t\right) = f_1\left(X_2e^{j\Delta\psi(t)},t\right)e^{-j\Delta\psi(t)} - \Delta\omega X_2 \end{equation}

	\noindent which is differentiable with respect to $f_1$ and $X_2$. Naturally this relationship is invertible as

\begin{equation} f_1\left(X_1,t\right) = f_2\left(X_1e^{-j\Delta\psi(t)},t\right)e^{j\Delta\psi(t)} + \Delta\omega X_1 \end{equation}

	\noindent thus there is a diffeomorphism between $f_1$ and $f_2$ in $I$, which means that the solutions $X_1$ and $X_2$ of \eqref{eq:two_sols_freq} are equivalent in this interval. In other words, one can integrate any of the two equations and can obtain the solution to the other. \hfill$\blacksquare$\vspace{5mm}\hrule\vspace{5mm} 
%>>>

\begin{example}[Example application of theorems \ref{theo:homeomorphic_phasors} and \ref{theo:diff_freqs}]\label{example:diff_freqs} %<<<

	Consider again the second-order circuit of figure \ref{fig:different_freqs_example_network_1p} where the same second-order circuit of example \ref{example:rlc_dpt} is shown. This circuit is excited by a voltage

\begin{equation} v(t) = m_v(t)\cos\left(\psi(t)\right) \text{, with } \psi = \int_0^a \omega(a)da \text{, where } \omega(t) = \omega_0\left[1 + Me^{-\alpha t}\sin\left(\beta t\right)\right], \label{eq:example_voltage_def}\end{equation}

 	\noindent yields an angle displacement

\begin{equation} \psi(t) = \omega_0\left(t + \dfrac{M\left\{\beta - e^{-\alpha t}\left[\alpha\sin\left(\beta t\right) + \beta\cos\left(\beta t\right)\right]\right\}}{\alpha^2 + \beta^2} \right) .\end{equation}

% MODELLING EXAMPLE OF DIFFERENT FREQUENCIES <<<	
\begin{figure}[htb!]
\centering
        \begin{tikzpicture}[american,scale=1,transform shape,line width=0.75, cute inductors, voltage shift = 1]
	\ctikzset{/tikz/circuitikz/voltage/distance from node=10mm}
		\draw (0,0)
			to[vsource,sources/scale=1.25, v>=$V(t)$,invert] (0,4)
			to[L,l=$L$,f>^=$I_{L}$,v>=$V_{L}$,-*] (4,4) 
			to[C,l=$C$,f>^=$I_{C}$,v>=$V_{C}$,-*] (4,0) 
			to[short] (0,0); 
		\draw (4,4)
			to[short,f>^=$I_{R}$] (8,4) 
			to[R,l=$R$,v>=$V_{R}$] (8,0) 
			to[short]  (4,0);
        \end{tikzpicture}
	\caption{Second-order circuit for example application of theorem \ref{theo:diff_freqs}.}
	\label{fig:different_freqs_example_network_1p}
\end{figure} %>>>

	We now model the circuit using the apparent frequency $\omega(t)$, and call the resulting phasor of the voltage across the load as $V_R(t)$:

\begin{equation}\left(\omega(t)\right): \ddot{V}_R(t) + \dot{V}_R(t)\left(\dfrac{1}{RC} + 2j\omega(t)\right) + V_R\left\{ \dfrac{1}{LC}  -\omega^2(t) + j \left[ \dot{\omega}(t) + \dfrac{1}{RC}\omega(t)\right]\right\} -\dfrac{1}{LC} m_v(t) = 0, \label{eq:rlc_complex_diffeq_1}\end{equation}

	Now, modelling the circuit using the constant frequency $\omega_0$. Denote as $V_{R0}(t)$ as the Dynamic Phasor of $v_R(t)$ modelled using $\omega_0$:

\begin{equation}\left(\omega_0\right): \ddot{V}_{R0}(t) + \dot{V}_{R0}(t)\left(\dfrac{1}{RC} + 2j\omega_0\right) + V_{R0}\left(\dfrac{1}{LC} - \omega_0^2 + j \dfrac{1}{RC}\omega_0\right) - \dfrac{1}{LC} m_v(t)e^{j\left(\psi(t) - \omega_0 t\right)} = 0, \label{eq:rlc_complex_diffeq_0}\end{equation}

	\noindent which is the exact same equation as \eqref{eq:rlc_complex_diffeq} of example \ref{example:rlc_dpt}. Notably, both equations differ fundamentally in the fact that since $\omega_0$ is constant, its derivatives vanish; also, the excitation vector $v(t)$ yields different phasors: in the time-varying frequency $V(t)$ is in phase with the DQ transform axis because the sinusoid $v(t)$ is defined exactly at $\omega(t)$, whereas for the static frequency $\omega_0$, the phasor $V(t)$ varies in time. 

	We want to validate theorem \ref{theo:homeomorphic_phasors} showing that by solving the ODE at constant frequency \eqref{eq:rlc_complex_diffeq_0}, we can obtain the solution to the ODE with time-varying frequency \ref{eq:rlc_complex_diffeq_1} not by integrating it, but through the transformation $V_R(t) = V_{R0}e^{j\left(\psi(t) - \omega_0 t\right)}$. To this extent, figures \ref{fig:amp_phase_voltage_signals_diffreqs_real} and \ref{fig:amp_phase_voltage_signals_diffreqs_imag} show the real and imaginary parts of three signals:

\begin{itemize}
	\item In blue, $V_R(t)$ as the complex voltage obtained by integrating the complex ODE \eqref{eq:rlc_complex_diffeq_1} at the time-varying frequency $\omega(t)$;
	\item In green, $V_{R0}(t)$ as the complex voltage obtained by integrating \eqref{eq:rlc_complex_diffeq_0} at the fixed frequency $\omega_0$;
	\item In dashed red, the composition $V_{R0}e^{j\left(\psi(t) - \omega_0 t\right)}$ that, by theorems \ref{theo:homeomorphic_phasors} should be equal to $V_{R}(t)$.
\end{itemize}

	The figures indeed validate theorem \ref{theo:homeomorphic_phasors}, since the dashed red line clearly overlaps with the blue lines; this means that instead of solving the ODE with time-varying frequency \eqref{eq:rlc_complex_diffeq_1}, one can solve \eqref{eq:rlc_complex_diffeq_0}, defined at the fixed frequency $\omega_0$, and then obtain the solution to \eqref{eq:rlc_complex_diffeq_1} by the transformation $V_{R0}e^{j\left(\psi(t) - \omega_0 t\right)}$.

% AMPLITUDE REAL PART TIME CURVES <<<
\begin{figure}
        \begin{center}
                \beginpgfgraphicnamed{timesim_comp_diffreqs_real}
                \begin{tikzpicture}
                        \begin{axis}[
                                width =  \columnwidth,
                                height = \columnwidth/1.618,
                                title={Real components of voltages},
                                xlabel={Time (s)},
                                ylabel={$\Re\left(V_R(t),\ V_{R0}(t)\right)$ (V)},
                                xmin=0, xmax=1,
                                ymin=-15, ymax=42,	
				ymajorgrids=true,
                                xmajorgrids=true,
                                xtick={0,0.1,...,1},
                                ytick={-15,-10,...,40},
				ylabel style = {align=center},
				axis y line*=left,
                                every axis plot/.append style={thick},
                                legend pos=south east,
				legend cell align = {left}
                        ]
                                \addplot[blue,smooth]                                      table[col sep=comma,header=false,x index=0,y index=8]{data/rlc_sim/data_rlc_sim_dps.csv};
				\addlegendentry{$\Re\left(V_R(t)\right)$}
                                \addplot[red,dashed, dash pattern=on 4pt off 4pt,smooth, line cap = round] table[col sep=comma,header=false,x index=0,y index=2]{data/rlc_sim/data_rlc_sim_omega0.csv};
				\addlegendentry{$\Re\left[V_{R0}e^{j\left(\psi(t) - \omega_0 t\right)}\right]$}
                                \addplot[stewartgreen,smooth]                             table[col sep=comma,header=false,x index=0,y index=5]{data/rlc_sim/data_rlc_sim_omega0.csv};
				\addlegendentry{$\Re\left(V_{R0}(t)\right)$}
                        \end{axis}
                \end{tikzpicture}
        \endpgfgraphicnamed
        \caption
[Real components of simulated voltages and reconstructed voltage.]
{Real components of $V_R(t)$ (blue), $V_{R0}(t)$ (green) and the reconstructed voltage $V_{R0}e^{j\left(\psi(t) - \omega_0 t\right)}$ which should be equal to $V_R(t)$.}
        \label{fig:amp_phase_voltage_signals_diffreqs_real}
        \end{center}
\end{figure}
% >>>

% AMPLITUDE IMAGINARY PART TIME CURVES <<<
\begin{figure}
        \begin{center}
                \beginpgfgraphicnamed{timesim_comp_diffreqs_imag}
                \begin{tikzpicture}
                        \begin{axis}[
                                width =  \columnwidth,
                                height = \columnwidth/1.618,
                                title={Imaginary components of voltages},
                                xlabel={Time (s)},
                                ylabel={$\Im\left(V_R(t),\ V_{R0}(t)\right)$ (V)},
                                xmin=0, xmax=1,
                                ymin=-17, ymax=32,	
				ymajorgrids=true,
                                xmajorgrids=true,
                                xtick={0,0.1,...,1},
                                ytick={-15,-10,...,30},
				ylabel style = {align=center},
				axis y line*=left,
                                every axis plot/.append style={thick},
                                legend pos=south east,
				legend cell align = {left}
                        ]
                                \addplot[blue,smooth]                                    table[col sep=comma,header=false,x index=0,y index=9]{data/rlc_sim/data_rlc_sim_dps.csv};
				\addlegendentry{$\Im\left(V_R(t)\right)$}
                                \addplot[red,dashed, dash pattern=on 4pt off 4pt,smooth, line cap = round] table[col sep=comma,header=false,x index=0,y index=3]{data/rlc_sim/data_rlc_sim_omega0.csv};
				\addlegendentry{$\Im\left[V_{R0}e^{j\left(\psi(t) - \omega_0 t\right)}\right]$}
                                \addplot[stewartgreen,smooth]                            table[col sep=comma,header=false,x index=0,y index=6]{data/rlc_sim/data_rlc_sim_omega0.csv};
				\addlegendentry{$\Im\left(V_{R0}(t)\right)$}
                        \end{axis}
                \end{tikzpicture}
        \endpgfgraphicnamed
        \caption
[Imaginary components of simulated voltages and reconstructed voltage.]
{Imaginary components of $V_R(t)$ (blue), $V_{R0}(t)$ (green) and the reconstructed voltage $V_{R0}e^{j\left(\psi(t) - \omega_0 t\right)}$ which should be equal to $V_R(t)$.}
        \label{fig:amp_phase_voltage_signals_diffreqs_imag}
        \end{center}
\end{figure}
% >>>

	Further, Figure \ref{fig:voltage_signals_diffreqs} shows that both $V_R(t)$ and $V_{R0}(t)$ reconstruct the same exact signal in time, which corroborates with the fact that both the differential models \eqref{eq:rlc_complex_diffeq_1} and \eqref{eq:rlc_complex_diffeq_0} are able to accurately reconstruct sinusoids in time even though they define different phasors.

% VOLTAGE TIME CURVES <<<
\begin{figure}
        \begin{center}
                \begin{tikzpicture}
                        \begin{axis}[
				name = ax_main,
                                width = 0.9*\columnwidth,
                                height = 0.9*1/1.618*\columnwidth,
                                title={Reconstructed voltage signals from DP simulations},
                                xlabel={Time (s)},
                                ylabel={$\mathbf{P_D^{\left(-\omega_0\right)}} \left[V_{R0}\right]$ and $\mathbf{P_D^{\left(-\omega\right)}} \left[V_R\right]$ (V)},
                                xmin=0, xmax=1,
                                ymin=-42, ymax=42,
                                xtick={0,0.1,...,1},
                                ytick={-40,-30,...,40}, 
                                legend pos=south east,
                                ymajorgrids=true,
                                xmajorgrids=true,
                                %grid style=dashed,
                                colormap name=hsv2,
                                cycle list={[ colors of colormap={100,200,300,400,500,600,700,800,900,1000} ]},
                                every axis plot/.append style={thick},
                        ]
                                \addplot[blue ,smooth] table[col sep=comma,header=false,x index=0,y index=1]{data/rlc_sim/data_rlc_sim_dps.csv};
                        \coordinate (c1) at (axis cs:0,-42);
                        \coordinate (c2) at (axis cs:0.1,-42);
                        \end{axis}
%
                        \begin{axis}[
                                name = ax_zoomed,
                                at={($(ax_main.north east)-(0.9\columnwidth,1.75/1.618*\columnwidth)$)},
                                width = 0.9*1\columnwidth,
                                height = 0.9*1/1.618*\columnwidth,
                                xmin=0, xmax=0.1,
                                ymin=-42, ymax=42,
                                xtick={0,0.01,...,0.1},
				xlabel={Time (ms)},
				xticklabels={$0$,$10$,$20$,$30$,$40$,$50$,$60$,$70$,$80$,$90$,$100$},
                                ytick={-40,-30,...,40},
				tick label style={/pgf/number format/fixed},
				legend columns=2,
				legend style={/tikz/every even column/.append style={column sep=0.5cm}},
                                ymajorgrids=true,
                                xmajorgrids=true,
                                %grid style=dashed,
                                every axis plot/.append style={thick},
                                axis background/.style = {
                                        preaction = {
                                        path picture = {
                                        \draw[fill=white,line width=0mm] (axis cs:0,400) rectangle (axis cs:0.1,-40);
                                                }
                                        }
                                }
                        ]
				\addplot[blue, smooth, line cap=round] table[col sep=comma,header=false,x index=0,y index=2]{data/rlc_sim/data_rlc_sim_dps.csv};
				\addlegendentry{$\mathbf{P_D^{\left(-\omega\right)}}\left[V_R\right]$}
                                \addplot[red,dashed, dash pattern=on 4pt off 4pt,smooth, line cap = round] table[col sep=comma,header=false,x index=0,y index=4]{data/rlc_sim/data_rlc_sim_omega0.csv};
				\addlegendentry{$\mathbf{P_D^{\left(-\omega_0\right)}}\left[V_{R0}\right]$}
                        \end{axis}
                        % draw dashed lines from rectangle in first axis to corners of second
                        \draw [gray,dashed] (c1) -- (ax_zoomed.north west);
                        \draw [gray,dashed] (c2) -- (ax_zoomed.north east);
                \end{tikzpicture}
        \caption
[Voltage across the resistor of the circuit of Figure \ref{fig:different_freqs_example_network_1p}.]
{Voltage across the resistor of the circuit of Figure \ref{fig:different_freqs_example_network_1p} as reconstructed by the solution $V_R(t)$ of the frequency-varying model \eqref{eq:rlc_complex_diffeq_1} (in blue) and the one reconstructed from the fixed-frequency model \eqref{eq:rlc_complex_diffeq_0} (in dashed red).}
        \label{fig:voltage_signals_diffreqs}
        \end{center}
\end{figure}
% >>>

	We also know that the matrix model of this circuit is given by

\begin{equation} \dfrac{d}{dt} \left[ \begin{array}{l} v_C \\[3mm] i_L \end{array}\right] = \left[\begin{array}{cc} -\dfrac{1}{RC} & \dfrac{1}{C} \\[5mm] - \dfrac{1}{L} & 0 \end{array}\right] \left[\begin{array}{c} v_C \\[5mm] i_L \end{array}\right] + \left[\begin{array}{c} 0 \\[3mm] -\dfrac{1}{L}\end{array}\right] v(t)\end{equation}

	\noindent which, adopting the time-varying frequency $\omega(t)$, yields a complex-equivalent system

\begin{equation} \dfrac{d}{dt} \left[ \begin{array}{l} V_C \\[3mm] I_L \end{array}\right] = \left(\left[\begin{array}{cc} -\dfrac{1}{RC} & \dfrac{1}{C} \\[5mm] - \dfrac{1}{L} & 0 \end{array}\right] - j\omega \mathbf{I}_2\right) \left[\begin{array}{c} V_C \\[5mm] I_L \end{array}\right] + \left[\begin{array}{c} 0 \\[3mm] -\dfrac{1}{L}\end{array}\right] V(t) . \label{eq:matrix_model_varying}\end{equation}

	Let $V_{C0},\ I_{L0},\ V_0$ the equivalent phasors at the frequency $\omega_0$. From theorem \ref{theo:homeomorphic_phasors}

\begin{equation} \left[ \begin{array}{l} V_{C0} \\[3mm] I_{L0} \\[3mm] V_0\end{array}\right] = \left[\begin{array}{l} V_C \\[3mm] I_L \\[3mm] V\end{array}\right] e^{j\left(\psi(t) - \omega_0 t\right)} ,\end{equation}

	\noindent and applying this to \eqref{eq:matrix_model_varying},

\begin{equation} \dfrac{d}{dt} \left(\left[ \begin{array}{l} V_{C0} \\[3mm] I_{L0} \end{array}\right]e^{j\left(\omega_0 t - \psi(t)\right)}\right) = \left(\left[\begin{array}{cc} -\dfrac{1}{RC} & \dfrac{1}{C} \\[5mm] - \dfrac{1}{L} & 0 \end{array}\right] - j\omega \mathbf{I}_2\right) \left[ \begin{array}{l} V_{C0} \\[3mm] I_{L0} \end{array}\right]e^{j\left(\omega_0 t - \psi(t)\right)} + \left[\begin{array}{c} 0 \\[3mm] -\dfrac{1}{L}\end{array}\right] V(t) . \label{eq:matrix_model_varying_1}\end{equation}

	Developing this equation,

\begin{align}
	 & e^{j\left(\omega_0 t - \psi(t)\right)}\dfrac{d}{dt} \left[ \begin{array}{l} V_{C0} \\[3mm] I_{L0} \end{array}\right] + \dfrac{d}{dt}\left[e^{j\left(\omega_0 t - \psi(t)\right)} \right] \left[ \begin{array}{l} V_{C0} \\[3mm] I_{L0} \end{array}\right] = \nonumber\\[3mm] &\hspace{4cm} = \left(\left[\begin{array}{cc} -\dfrac{1}{RC} & \dfrac{1}{C} \\[5mm] - \dfrac{1}{L} & 0 \end{array}\right] - j\omega \mathbf{I}_2\right) \left[ \begin{array}{l} V_{C0} \\[3mm] I_{L0} \end{array}\right]e^{j\left(\omega_0 t - \psi(t)\right)} + \left[\begin{array}{c} 0 \\[3mm] -\dfrac{1}{L}\end{array}\right] V(t) \nonumber\\[3mm]
	 & e^{j\left(\omega_0 t - \psi(t)\right)}\dfrac{d}{dt} \left[ \begin{array}{l} V_{C0} \\[3mm] I_{L0} \end{array}\right] + \left(\omega_0 - \omega(t)\right)e^{j\left(\omega_0 t - \psi(t)\right)} \left[ \begin{array}{l} V_{C0} \\[3mm] I_{L0} \end{array}\right] = \nonumber\\[3mm] &\hspace{4cm} = \left(\left[\begin{array}{cc} -\dfrac{1}{RC} & \dfrac{1}{C} \\[5mm] - \dfrac{1}{L} & 0 \end{array}\right] - j\omega \mathbf{I}_2\right) \left[ \begin{array}{l} V_{C0} \\[3mm] I_{L0} \end{array}\right]e^{j\left(\omega_0 t - \psi(t)\right)} + \left[\begin{array}{c} 0 \\[3mm] -\dfrac{1}{L}\end{array}\right] V(t) \nonumber\\[3mm]
	 & e^{j\left(\omega_0 t - \psi(t)\right)}\dfrac{d}{dt} \left[ \begin{array}{l} V_{C0} \\[3mm] I_{L0} \end{array}\right] = \left(\left[\begin{array}{cc} -\dfrac{1}{RC} & \dfrac{1}{C} \\[5mm] - \dfrac{1}{L} & 0 \end{array}\right] - j\omega \mathbf{I}_2 + j\left(\omega(t) - \omega_0\right) \mathbf{I}_2\right) \left[ \begin{array}{l} V_{C0} \\[3mm] I_{L0} \end{array}\right]e^{j\left(\omega_0 t - \psi(t)\right)} + \left[\begin{array}{c} 0 \\[3mm] -\dfrac{1}{L}\end{array}\right] V(t)  \label{eq:matrix_model_varying_2}
\end{align}

	Multiplying the entire equation by $e^{j\left(\psi(t) - \omega_0 t\right)}$, and noting that $V_0 = e^{j\left(\psi(t) - \omega_0 t\right)}V(t)$,

\begin{equation} \dfrac{d}{dt} \left[ \begin{array}{l} V_{C0} \\[3mm] I_{L0} \end{array}\right] = \left(\left[\begin{array}{cc} -\dfrac{1}{RC} & \dfrac{1}{C} \\[5mm] - \dfrac{1}{L} & 0 \end{array}\right] - j\omega_0 \mathbf{I}\right) \left[ \begin{array}{l} V_{C0} \\[3mm] I_{L0} \end{array}\right] + \left[\begin{array}{c} 0 \\[3mm] -\dfrac{1}{L}\end{array}\right] V_0(t)  \label{eq:matrix_model_varying_3} \end{equation}

	\noindent which is the exact equation that would be obtained by modelling the circuit at $\omega_0$.

\examplebar
\end{example} %>>>

%-------------------------------------------------
\section{Determining if a 3$\phi$ system yields a balanced assymptotic solution}\label{sec:3p_assymp_freq} %<<<1

	In subsection \ref{subsec:zeroseq_comp} we discussed that a three-phase system yields a phasorial equation and a zero-sequence differential equation of the form

\begin{equation} \sum\limits_{i=0}^n \eta_i^n z_0^{(i)} - f_0 = 0,\ \eta_i(t) = \sum\limits_{k=i}^{n} \alpha_k {k\choose i} \left[\sum\limits_{c=0}^{k-i} B_{\left(k-i,c\right)}\left(\omega,\dot{\omega},\ddot{\omega},...,\omega^{(k-i-c)}\right)\right] \label{eq:qsh_init_zeroseq_diffeq}\end{equation}

	\noindent and because this equation is linear but has time-varying coefficients, analysis is made much more difficult. We want to prove that if the excitation $\omega(t)$ is equivalent to a fairly slow $\omega_0$ and the circuit is ``fast'' enough, then the three-phase when excited by a balanced forcing will be such that its response will tend to a balanced response.

	Assuming $\omega(t)$ is equivalent to a $\omega_0$, we can solve \eqref{eq:qsh_init_zeroseq_diffeq} in $\omega_0$ and we can reconstruct the solution at $\omega(t)$; we thus make our analysis in $\omega_0$. Denote $h_0(t)$ as the zero-sequence component of the forcing $\mathbf{f}_3$ measured at $\omega_0$:

\begin{equation} \sum\limits_{i=0}^n \eta_i \dfrac{d^i z_0}{dt^i} - h_0 = 0,\ \eta_i(t) = \sum\limits_{k=i}^{n} \alpha_k {k\choose i} \left[\sum\limits_{c=0}^{k-i} B_{\left(k-i,c\right)}\left(\omega_0,0,0,\cdots,0\right)\right] \label{eq:3p_zeroseq_omega0} \end{equation}

	\noindent and the $\eta_i$ become time invariant. Developing their expression yields

\begin{equation}
	\eta_i = \sum\limits_{k=i}^{n} \alpha_k {k\choose i} \left[\sum\limits_{c=0}^{k-i} B_{\left(k-i,c\right)}\left(\omega_0,0,0,\cdots,0\right)\right] = \sum_{k=i}^n \alpha_k {k\choose i} \left(\sum\limits_{c=0}^{k-i} \omega_0^c \right)
\end{equation}

	\noindent therefore \eqref{eq:3p_zeroseq_omega0} is Hurwitz Stable if the polynomial

\begin{equation} H_3(x) = \sum_{k=0}^n \eta_k x^k \end{equation}

	\noindent is Hurwitz; further, the differential equation is assymptotically stable. If we denote $P_k(\omega_0)$ as the sum of the first $k$ powers of $\omega_0$ as

\begin{equation} P_k\left(\omega_0\right) = 1 + \omega_0 + \omega_0^2 + \cdots + \omega_0^k = \sum_{j=0}^k \omega_0^k \end{equation}

	\noindent then we can further develop the $\eta_k$ of $H_3$ as

\begin{equation}
\left\{\begin{array}{l}
	\eta_n = \alpha_n P_0 \\[3mm]
%
	\eta_{(n-1)} = n\alpha_nP_1 + \alpha_{(n-1)}P_0\\[3mm]
%
	\eta_{(n-2)} = n(n-1)\alpha_n P_2 + (n-1)\alpha_{(n-1)}P_1 + \alpha_{(n-2)}P_0\\[3mm]
%
	\eta_{(n-3)} = n(n-1)(n-2)\alpha_n P_3+ (n-1)(n-2)\alpha_{(n-1)}P_2 + (n-2)\alpha_{(n-2)}P_1 + \alpha_{(n-3)}P_0\\[3mm]
%
	\hspace{3cm}\vdots
\end{array}\right.
\end{equation}

	\noindent thus we can write $H_3$ as a triangular sum

\begin{equation}
H_3 = \left\{\begin{array}{l}
	P_0\left(\alpha_n x^n + \alpha_{(n-1)}x^{(n-1)} + \alpha_{(n-2)}x^{(n-2)} + \cdots + \alpha_1x + \alpha_0\right) + \\[3mm]
%
	P_1\left( n\alpha_{n}x^{(n-1)} + (n-1)\alpha_{(n-1)}x^{(n-2)} + \cdots + 2\alpha_2x + \alpha_1\right) + \\[3mm]
%
	P_2\left( n(n-1)\alpha_{n}x^{(n-2)} + (n-1)(n-2)\alpha_{(n-1)}x^{(n-2)} + \cdots + 2\alpha_2\right) + \\[3mm]
%
	\hspace{3cm}\vdots
\end{array}\right.
\end{equation}

	\noindent and one notices that the $k$-th row of this triangular sum is equal to $P_k$ times the $k$-th derivative of $H_1$:

\begin{equation} H_3(x) = \sum_{k=0}^n P_k H_1^{(k)}(x) \label{eq:h3_h1_rel}\end{equation}

	\noindent where $H_1$ is the polynomial of the original circuit ODE $H_1 = \sum_{k=0}^n \alpha_kx^k$, and we know that this polynomial is Hurwitz, therefore its roots are all in the open left half plane.

	Initially, one uses the Gauss-Lucas Theorem to show that because $H_1$ is Hurwits, all derivatives $P^{(k)}$ (up to the $(n-1)$-th derivative) are also Hurwitz stable; since $H_3$ is a linear combination of $P(z)$ and its derivatives, it should also be Hurwitz.

% GAUSS-LUCAS EXAMPLE <<<
\definecolor{gausslucas1}{HTML}{065573}
\definecolor{gausslucas2}{HTML}{4D8D7A}
\definecolor{gausslucas3}{HTML}{D9BC2C}
\definecolor{gausslucas4}{HTML}{F57D39}
\definecolor{gausslucas5}{HTML}{FF0000}

\begin{figure}[t]
\centering
\scalebox{2}{
	\begin{tikzpicture}[scale=10,>={Stealth[inset=0mm,length=1mm,angle'=50]},square/.style={regular polygon,regular polygon sides=4}]
		\draw [->, line width = 0.1mm] (   -5mm,  0   ) -- (   1mm,  0   ); \node[label={\tiny $\Re$}] at (1mm,-0.6mm) {};
		\draw [->, line width = 0.1mm] (      0, -2mm ) -- (   0   ,  3mm) node[label={\tiny $\Im$}] (yaxis) {};
%
		\foreach \x/\y in {-5/2,-1.5/1,-1/0,-3/3,-3/-2,-2/2} {\node [circle, fill=gausslucas1, inner sep=0mm, minimum size=1mm] (p) at (\x mm,\y mm) {};}
		\draw[gausslucas1, fill=none, line width = 0.15mm] (-1mm,0mm) -- (-3mm,-2mm) -- (-5mm,2mm) -- (-3mm,3mm) -- (-2mm,2mm) -- (-1.5mm,1mm) -- (-1mm,0);
%
		\foreach \x/\y in {-4.33957/1.95852 , -2.76774/-1.32603 , -2.73412/2.54262 , -1.83004/1.4764 , -1.2452/0.348488} {\node [circle, fill=gausslucas2, inner sep=0mm, minimum size=1mm] (p) at (\x mm,\y mm) {};}
		\draw[gausslucas2, fill=none, line width = 0.15mm] (-4.33957mm, 1.95852mm) -- (-2.73412mm, 2.54262mm) -- (-1.83004 mm, 1.4764 mm) -- (-1.2452 mm, 0.348488 mm) -- (-2.76774mm, -1.32603mm) -- (-4.33957mm, 1.95852mm);
%
		\foreach \x/\y in {-3.71663/1.9115 , -2.5573/-0.657236 , -2.4692/1.98348 , -1.59021/0.762254} {\node [circle, fill=gausslucas3, inner sep=0mm, minimum size=1mm] (p) at (\x mm,\y mm) {};}
		\draw[gausslucas3, fill=none, line width = 0.15mm] (-3.71663 mm,1.9115mm ) -- (-2.4692 mm,1.98348mm ) -- (-1.59021 mm,0.762254mm ) -- (-2.5573 mm,-0.657236mm ) -- (-3.71663 mm,1.9115mm );
%
		\foreach \x/\y in {-3.19306/1.81086 , -2.39361/0.0203507 , -2.16333/1.20949 } {\node [circle, fill=gausslucas4, inner sep=0mm, minimum size=1mm] (p) at (\x mm,\y mm) {};}
		\draw[gausslucas4, fill=none, line width = 0.15mm] (-3.19306 mm,1.81086 mm) -- (-2.39361 mm,0.0203507 mm) -- (-2.16333 mm, 1.20949 mm) -- (-3.19306 mm,1.81086 mm);
%
		\foreach \x/\y in {-2.78939/1.4853 , -2.37728/0.514701} {\node [circle, fill=gausslucas5, inner sep=0mm, minimum size=1mm] (p) at (\x mm,\y mm) {};}
		\draw[gausslucas5, fill=none, line width = 0.15mm] (-2.78939 mm, 1.4853 mm) -- (-2.37728 mm, 0.514701 mm);
%
		\node [circle, fill=gray, inner sep=0mm, minimum size=1mm] (p) at (-2.58333 mm,1 mm) {};
%
		\node at (0.5mm, 3mm   ) [fill = gausslucas1, square ,inner sep = 0mm, minimum size = 2.5mm] (v100) {}; \node[scale = 0.8,right,gausslucas1] at (0.6mm,3mm)   {\tiny $r\left(P\right)$};
		\node at (0.5mm, 2.5 mm) [fill = gausslucas2, square ,inner sep = 0mm, minimum size = 2.5mm] (v100) {}; \node[scale = 0.8,right,gausslucas2] at (0.6mm,2.5mm) {\tiny $r\left(P'\right)$};
		\node at (0.5mm, 2   mm) [fill = gausslucas3, square ,inner sep = 0mm, minimum size = 2.5mm] (v100) {}; \node[scale = 0.8,right,gausslucas3] at (0.6mm,2mm)   {\tiny $r\left(P''\right)$};
		\node at (0.5mm, 1.5 mm) [fill = gausslucas4, square ,inner sep = 0mm, minimum size = 2.5mm] (v100) {}; \node[scale = 0.8,right,gausslucas4] at (0.6mm,1.5mm) {\tiny $r\left(P^{(3)}\right)$};
		\node at (0.5mm, 1   mm) [fill = gausslucas5, square ,inner sep = 0mm, minimum size = 2.5mm] (v100) {}; \node[scale = 0.8,right,gausslucas5] at (0.6mm,1mm)   {\tiny $r\left(P^{(4)}\right)$};
		\node at (0.5mm, 0.5 mm) [fill = gray       , square ,inner sep = 0mm, minimum size = 2.5mm] (v100) {}; \node[scale = 0.8,right,gray       ] at (0.6mm,0.5mm) {\tiny $r\left(P^{(5)}\right)$};

	\end{tikzpicture}
}
\caption
[Gauss-Lucas application example.]
{Gauss-Lucas application example to the polynomial $P(z)$ of \eqref{eq:poly_gausslucas_example}.}
\label{fig:gausslucas_example}
\end{figure} %>>>

\begin{theorem}[Gauss-Lucas Theorem \pcite{ahlfors1979complex}]
	Given $P\in\left[\mathbb{C}\to\mathbb{C}\right]$ a nonconstant polynomial with complex coefficients, all zeros of $P'$ belong to the convex hull of the set of zeros of $P$, that is, the smallest convex polygon containing the roots of $P$.
\end{theorem}
\hrule
\vspace{3mm}

	As an example of this theorem, take the polynomial

\begin{equation} P(z) = \prod_{k=1}^6 \left(z - z_k\right)\left\{\begin{array}{l} z_1 = -5000 + j2000 \\ z_2 = -1500 + j1000 \\ z_3 = -1000 \\ z_4 = -3000 - j2000 \\ z_5 = -3000 + j3000 \\ z_6 = -2000 + j2000 \end{array}\right.  \label{eq:poly_gausslucas_example} \end{equation}

	\noindent which roots are explicit and all in the left open half plane (thus $P$ is Hurwitz); therefore the roots of the derivatives are certainly in the polygon formed by the roots of $P$. The roots of $P$ and its derivatives are shown in figure \ref{fig:gausslucas_example}.

	The example shows that if $P$ is Hurwitz stable (which it is as per definition \ref{eq:poly_gausslucas_example}) then all its derivatives are also Hurwitz stable. This unfortunately does not mean $H_3$ is Hurwitz: the sum of stable polynomials is not always stable, that is, the class of Hurwitz stable polynomials is not closed to linear combinations. For instance, $S(x) = (x+1)^3$ and $R(x) = x + 20$ are Hurwitz stable, but their sum is not.

	Hence, the proof that $H_3$ is Hurwitz needs an additional restriction: we now want to show that if the roots of $H_1$ are large enough (the circuit is ``quick enough''), the roots of $H_3$ approach the roots of $H_1$ as $\omega_0$ gets smaller and tends to zero (the excitation gets ``slower''). We first prove that the distance between $H_3$ and $H_1$ has an upper bound that gets smaller with $\omega_0$ and as the roots get larger.

\begin{lemma}[Rouché's Theorem \pcite{ahlfors1979complex}] Consider two $f,g\in\left[K\subset\mathbb{C}\to\mathbb{C}\right]$ holomorphic in $K$ with a closed contour $\partial K$. If $\left\lvert g(z)\right\rvert < \left\lvert f(z)\right\rvert$ on $\partial K$, then $f$ and $f+g$ have the same number of zeros in $K$, each zero counter as many times as its multiplicity.
\end{lemma}
\begin{theorem}[$H_3$ approaches $H_1$ under the QSH]\label{theo:h3approachesh1} %<<<
	Consider a central point $z_0\in\mathbb{C}$ and a radius $R\in\mathbb{R}^+$ such that the roots of $H_1$ are inside the disc of radius $R$ centered at $z_0$, that is,

\begin{equation} \left\lvert z_k - z_0\right\rvert \leq R\ \forall z_k\in r\left(H_1\right). \end{equation}

	Then 

\begin{equation} \left\lvert H_3(x) - H_1(x)\right\rvert \leq \varepsilon\left\lvert\sum_{k=1}^n x^k\right\rvert + \omega_0 \left\lvert \sum_{k=1}^n P_{(k-1)}H_1^{(k)}(x)\right\rvert \text{, where } \lim\limits_{\left\lvert z_0\right\rvert\to\infty} \varepsilon = 0.  \end{equation}

\end{theorem}
\textbf{Proof.} By definition, $P_k = 1 + \omega_0 P_{(k-1)}$ for $k \geq 1$; thus,

\begin{equation} H_3(x) = H_1(x) + \sum_{k=1}^n \left(1 + \omega_0 P_{(k-1)}\right) H_1^{(k)}(x) = \sum_{k=1}^n H_1^{(k)}(x) + \omega_0 \sum_{k=1}^n P_{(k-1)}H_1^{(k)}(x) \label{eq:h3_h1_rel}\end{equation}

	Thus by the triangular inequality

\begin{equation} \left\lvert H_3(x) - \sum_{k=0}^n H_1^{(k)}(x)\right\rvert = \omega_0 \left\lvert \sum_{k=1}^n P_{(k-1)}H_1^{(k)}(x)\right\rvert \label{eq:h3_triang_ineq}\end{equation}

	\noindent and now we want to show that the term $\sum_{k=0}^n H_1^{(k)}(x)$ tends to $H_1(x)$ as the roots of $H_1(x)$ get larger in absolute value. For this, let us consider a central point $z_0$ and a radius $R$ such that the roots of $H_1$ are inside the disc of radius $R$ centered at $z_0$, that is,

\begin{equation} \left\lvert z_k - z_0\right\rvert < R,\ k= 1,2,\cdots,n \Rightarrow \left\lvert z_0\right\rvert - R \leq \left\lvert z_k\right\rvert \leq \left\lvert z_0\right\rvert + R \label{eq:roots_radius_ineq}\end{equation}

	\noindent for some radius $R$ and some number $z_0$. We also know, by the Gauss-Lucas Theorem, that all roots of all derivatives of $H_1$ will also be inside this circle; thus the right part of \eqref{eq:h3_triang_ineq} is limited above in this circle. We additionally know that $H_1$ is Hurwitz, so $z_0$ is certainly in the open half left plane and $R$ is less than $\left\lvert z_0\right\rvert$. We can obtain the coefficients of $H_1(x)$ through the roots using Vieta's Formulas \pcite{ahlfors1979complex}: the $k$-th coefficient is obtained as the sum of the roots multiplied in groups of $k$ as in

\begin{equation}
	\left\{\begin{array}{l}
		r_1 + r_2 + \cdots + r_n = -\alpha_{(n-1)} \\[3mm]
		\left(r_1r_2 + r_1r_3 + \cdots + r_1r_n\right) + \left(r_2r_3 + r_2r_4 + \cdots + r_2r_n\right) + \cdots + r_{(n-1)}r_n = a_{(n-2)} \\[3mm]
		\left(r_1r_2r_3 + r_1r_2r_4 + \cdots + r_1r_{(n-1)}r_n\right) + \cdots + r_{(n-2)}r_{(n-1)}r_n = a_{(n-3)} \\[3mm]
		\hspace{2cm} \vdots \\[3mm]
		r_1r_2 \cdots r_n = \left(-1\right)^n a_0
	\end{array}\right. \label{eq:vietas_formulas}
\end{equation}

	\noindent and using \eqref{eq:roots_radius_ineq} we immediately notice that 

\begin{equation} \alpha_{(n-k)} = O\left(\left\lvert z_0\right\rvert^k\right) .\end{equation}

	It thus becomes clear that the differentiation operation causes the resulting polynomial to go down in order; if $z_0$ is big enough, the coefficients of $H_1(x)$ dominates over the coefficients of its derivatives. In formal terms, let $\beta_k$ the coefficients of $\sum_{k=0}^n H_1^{(k)}(x)$, that is,

\begin{equation} Q(x) = \sum_{k=0}^n H_1^{(k)}(x) = \sum_{k=0}^n \beta_kx^k .\end{equation}

	Thus the coefficients $\beta_k$ are the sums of the coefficients of $H_1$ and its derivatives. For the $i-th$ derivative of $H_1(x)$, the $k$-th coefficiene of the derivative is a combination of all $\alpha_{(n-k)},\ k\leq n$; summing the coefficients of all derivatives yields because the $\alpha_i$ are of a lower order of $\left\lvert z_0\right\rvert$, then

\begin{align}
	\beta_k &= \overbrace{\alpha_k}^{\text{k-th coeff. of } H_1} + \overbrace{\alpha_k O \left(\dfrac{1}{\left\lvert z_0\right\rvert}\right)}^{\text{k-th coeff. of 1st deriv.}} + \overbrace{\alpha_k O \left(\dfrac{1}{\left\lvert z_0\right\rvert^2}\right)}^{\text{k-th coeff. of 2nd deriv.}} + \cdots + \overbrace{\alpha_k O \left(\dfrac{1}{\left\lvert z_0\right\rvert^{(k-1)}}\right)}^{\text{k-th coeff. of (k-1)-th deriv.}} = \nonumber\\[3mm]
%
	&= \alpha_k + \alpha_k \sum_{i=1}^{k-1} O\left(\dfrac{1}{\left\lvert z_0\right\rvert^i}\right).
\end{align}

	Therefore

\begin{equation} \lim\limits_{\left\lvert z_0\right\rvert\to\infty} \left(\beta_k - \alpha_k\right) = 0 .\end{equation}

	Alternatively, we can write 

\begin{equation} \sum_{k=1}^n H_1^{(k)}(x) = \sum_{k=0}^n \varepsilon_kx^k \text{ such that } \lim\limits_{\left\lvert z_0\right\vert\to\infty} \varepsilon_k = 0 .\end{equation}

	Denote $\varepsilon\left(z_0\right)$ the largest among the $\varepsilon_k\left(z_0\right)$; then this equation yields

\begin{equation} \left\lvert\sum_{k=1}^n H_1^{(k)}(x)\right\rvert \leq \varepsilon \left\lvert\sum_{k=0}^n x^k\right\rvert .\end{equation}

	Applying this to \eqref{eq:h3_triang_ineq} and applying the inverse triangle inequality yields

\begin{equation} \left\lvert H_3(x) - H_1(x)\right\rvert \leq \varepsilon\left\lvert\sum_{k=1}^n x^k\right\rvert + \omega_0 \left\lvert \sum_{k=1}^n P_{(k-1)}H_1^{(k)}(x)\right\rvert \label{eq:h3_ineq_qsg}\end{equation}

\hfill$\blacksquare$ \vspace{3mm}\hrule\vspace{3mm} %>>>
\begin{corollary}[The roots of $H_3$ are close to those of $H_1$ under the QSH]\label{corollary:h3_h1_roots_qsh} %<<<
	If $H_1$ is Hurwitz stable, $H_3$ is also Hurwitz stable for $\left\lvert z_0\right\rvert$ sufficiently large and $\omega_0$ sufficiently small. Moreover, the roots of $H_3$ get closer to those of $H_1$ as $z_0$ gets larger and $\omega_0$ gets smaller.
\end{corollary}
\textbf{Proof.} Consider

\begin{equation} M\left(z_0,\omega_0\right) = \varepsilon\left\lvert\sum_{k=1}^n x^k\right\rvert + \omega_0 \left\lvert \sum_{k=1}^n P_{(k-1)}H_1^{(k)}(x)\right\rvert \end{equation}

% H3 PROOF EXAMPLE <<<
\definecolor{gradient1}{HTML}{80FF00}
\definecolor{gradient2}{HTML}{66D220}
\definecolor{gradient3}{HTML}{4DA540}
\definecolor{gradient4}{HTML}{33775F}
\definecolor{gradient5}{HTML}{1A4A7F}
\definecolor{gradient6}{HTML}{001D9F}

\begin{figure}[t]
\centering
\scalebox{2}{
	\begin{tikzpicture}[scale=10,>={Stealth[inset=0mm,length=1mm,angle'=50]},square/.style={regular polygon,regular polygon sides=4},y=1mm, x=1mm]
		\draw [->, line width = 0.1mm] (   -5mm,  0   ) -- (   1mm,  0   ); \node[label={\tiny $\Re$}] at (1mm,-0.6mm) {};
		\draw [->, line width = 0.1mm] (      0, -2mm ) -- (   0   ,  3mm) node[label={\tiny $\Im$}] (yaxis) {};

		\node at (0.8mm, 1.9mm ) [draw, black,inner sep = 0mm, minimum width = 10mm, minimum height=32mm] () {};
		\node at (0.5mm, 3.0mm ) [fill = gradient1, square ,inner sep = 0mm, minimum size = 2.5mm] () {}; \node[scale = 0.8,right,gradient1] at (0.6mm, 3.0mm)   {\tiny $260$};
		\node at (0.5mm, 2.5mm ) [fill = gradient2, square ,inner sep = 0mm, minimum size = 2.5mm] () {}; \node[scale = 0.8,right,gradient2] at (0.6mm, 2.5mm)   {\tiny $230$};
		\node at (0.5mm, 2.0mm ) [fill = gradient3, square ,inner sep = 0mm, minimum size = 2.5mm] () {}; \node[scale = 0.8,right,gradient3] at (0.6mm, 2.0mm)   {\tiny $120$};
		\node at (0.5mm, 1.5mm ) [fill = gradient4, square ,inner sep = 0mm, minimum size = 2.5mm] () {}; \node[scale = 0.8,right,gradient4] at (0.6mm, 1.5mm)   {\tiny $30$};
		\node at (0.5mm, 1.0mm ) [fill = gradient5, square ,inner sep = 0mm, minimum size = 2.5mm] () {}; \node[scale = 0.8,right,gradient5] at (0.6mm, 1.0mm)   {\tiny $20$};
		\node at (0.5mm, 0.5mm ) [fill = gradient6, square ,inner sep = 0mm, minimum size = 2.5mm] () {}; \node[scale = 0.8,right,gradient6] at (0.6mm, 0.5mm)   {\tiny $12$};
		\node[scale = 0.8,right] at (0.65mm, 3.3mm)   {\tiny $M$};
%
	\begin{scope}[cm={ 0.2645833304670139199,0.0,0.0,0.2645833304670139199,(0,-0.1363752681684345948)}]
	
	\path[draw=gradient1] %<<<
	(-15.007659,12.601877100000001) -- (-15.007659,12.601877100000001) -- (-15.821629,11.831638100000001) -- (-16.408089,11.327524100000002) -- (-16.912209,10.963684100000002) -- (-17.333619000000002,10.712577100000003) -- (-17.682449000000002,10.534994100000002) -- (-18.837799,10.000573600000003) -- (-19.009439,9.906043600000002) -- (-19.370169,9.668175600000001) -- (-19.608039,9.466136600000002) -- (-19.768879000000002,9.296644600000002) -- (-19.993159000000002,8.982629600000003) -- (-20.108729000000004,8.750686600000003) -- (-20.173729000000005,8.570162600000003) -- (-20.243129000000007,8.230388600000003) -- (-20.253329000000008,7.980448600000003) -- (-20.231929000000008,7.741647600000003) -- (-20.20382900000001,7.595329600000003) -- (-20.12902900000001,7.346050600000003) -- (-20.04792900000001,7.155442600000003) -- (-19.99312900000001,7.048939600000003) -- (-19.74958900000001,6.683506600000003) -- (-19.53533900000001,6.439972600000003) -- (-19.222889000000013,6.150386600000003) -- (-18.837769000000012,5.849768600000003) -- (-17.682409000000014,5.006493600000002) -- (-17.551239000000013,4.899496600000003) -- (-16.735819000000014,4.129258600000003) -- (-16.069379000000012,3.3590206000000027) -- (-15.432189000000012,2.5282896000000026) -- (-14.788839000000012,1.6311667000000027) -- (-13.532499000000012,-0.1934424399999971) -- (-12.817559000000012,-1.262406399999997) -- (-12.290749000000012,-2.0926523999999973) -- (-11.888259000000012,-2.8028823999999974) -- (-11.709309000000012,-3.1804753999999975) -- (-11.557919000000012,-3.610534399999998) -- (-11.520519000000013,-3.782384399999998) -- (-11.499619000000013,-3.981583399999998) -- (-11.535219000000014,-4.328722399999998) -- (-11.545819000000014,-4.368552399999999) -- (-11.696489000000014,-4.728473399999999) -- (-11.923559000000013,-5.113593399999998) -- (-12.146159000000013,-5.498711399999999) -- (-12.336239000000013,-5.9292083999999985) -- (-12.426539000000012,-6.268949399999999) -- (-12.453539000000012,-6.491392399999999) -- (-12.455539000000012,-6.654069399999999) -- (-12.443039000000013,-6.806433499999999) -- (-12.393339000000013,-7.0391873999999985) -- (-12.290669000000014,-7.2875093999999985) -- (-12.205269000000014,-7.424307399999998) -- (-12.032579000000014,-7.616866399999998) -- (-11.847719000000014,-7.751591399999998) -- (-11.712989000000015,-7.818591399999998) -- (-11.520429000000014,-7.878831399999998) -- (-11.228419000000015,-7.902521399999999) -- (-11.062539000000015,-7.882191399999999) -- (-10.750199000000014,-7.7745504) -- (-10.535449000000014,-7.6390404) -- (-10.311439000000014,-7.4242994) -- (-10.171269000000015,-7.2329864) -- (-10.064269000000015,-7.0391794) -- (-9.934098700000016,-6.6999234) -- (-9.873198700000016,-6.4615024000000005) -- (-9.836998700000017,-6.2689414) -- (-9.784498700000016,-5.8780614) -- (-9.719398700000017,-5.4216694) -- (-9.711398700000018,-5.3823094) -- (-9.613698700000018,-5.0762354) -- (-9.476048700000018,-4.8474394) -- (-9.209898700000018,-4.5932004) -- (-8.773168700000017,-4.3433513999999995) -- (-8.439658700000017,-4.2071684) -- (-8.231018700000016,-4.134708399999999) -- (-7.355358900000016,-3.887176499999999) -- (-5.358708900000016,-3.400291499999999) -- (-4.588468900000016,-3.1697594999999987) -- (-3.7667789000000154,-2.8543284999999985) -- (-3.6531589000000153,-2.8028684999999984) -- (-3.047998900000015,-2.4857784999999986) -- (-2.770078900000015,-2.3105514999999985) -- (-2.2777589000000154,-1.9386844999999986) -- (-1.9284589000000154,-1.6116934999999986) -- (-1.5075189000000155,-1.1151843999999986) -- (-1.2579089000000154,-0.7417713999999985) -- (-1.1169589000000155,-0.49215439999999855) -- (-0.9340689500000156,-0.10703543999999854) -- (-0.7372790000000156,0.4325475800000015) -- (-0.6088990000000156,0.9199355800000015) -- (-0.5817990000000156,1.0483205800000015) -- (-0.4959990000000156,1.5773036000000014) -- (-0.4465990000000156,2.1092426000000013) -- (-0.42919900000001565,2.5887965000000013) -- (-0.43919900000001566,3.180912500000001) -- (-0.46759900000001564,3.629098600000001) -- (-0.5196990000000157,4.129272500000002) -- (-0.6575490000000157,4.9795855000000016) -- (-0.8162489500000157,5.669748500000002) -- (-1.0453889000000158,6.439986500000002) -- (-1.332308900000016,7.210224500000002) -- (-1.5078689000000158,7.6161565000000016) -- (-1.6803489000000158,7.980462500000002) -- (-1.8836689000000157,8.374897500000001) -- (-2.278108900000016,9.056027600000002) -- (-2.5819689000000157,9.520938600000003) -- (-3.1578989000000157,10.291176600000002) -- (-3.845968900000016,11.061415100000001) -- (-4.683598900000016,11.831652100000001) -- (-5.506368900000016,12.4545811) -- (-6.129288900000016,12.8602601) -- (-6.450008900000015,13.0514111) -- (-7.031798900000015,13.372129099999999) -- (-7.669768900000015,13.6889791) -- (-8.526328700000015,14.0560411) -- (-8.758468700000014,14.1423711) -- (-9.210248700000013,14.2913791) -- (-9.701688700000014,14.421168100000001) -- (-9.980478700000015,14.477978100000001) -- (-10.373459000000015,14.535348100000002) -- (-10.750719000000014,14.563598100000002) -- (-11.164969000000013,14.562198100000002) -- (-11.856799000000013,14.4782281) -- (-12.291199000000013,14.3694001) -- (-13.006789000000014,14.0877491) -- (-13.233109000000013,13.9707181) -- (-13.960379000000014,13.500863099999998) -- (-14.128759000000015,13.372153099999998) -- (-14.797859000000015,12.797863099999999) -- (-15.008079000000015,12.6019151); %>>>
	
	\path[draw=gradient2] %<<<
	(-14.601489,12.5836441) -- (-14.601489,12.5836441) -- (-15.756849,11.403573100000001) -- (-16.173869,11.0294991) -- (-16.650019,10.6762821) -- (-17.060249,10.4392055) -- (-17.382409,10.2911626) -- (-17.682449,10.1755505) -- (-18.468588999999998,9.9060435) -- (-18.837798999999997,9.755200499999999) -- (-19.259458999999996,9.5209245) -- (-19.654978999999997,9.1827466) -- (-19.696378999999997,9.1358066) -- (-19.947918999999995,8.750687600000001) -- (-19.993218999999996,8.647200600000001) -- (-20.076918999999997,8.365568600000001) -- (-20.107918999999995,8.0951596) -- (-20.099918999999996,7.874118600000001) -- (-20.046318999999997,7.595330600000001) -- (-19.993518999999996,7.436424600000001) -- (-19.888498999999996,7.210211600000002) -- (-19.667268999999997,6.883963600000002) -- (-19.311608999999997,6.528298600000002) -- (-19.201248999999997,6.439978600000002) -- (-18.753828999999996,6.139190600000002) -- (-18.262588999999995,5.860196600000002) -- (-17.682808999999995,5.526491600000002) -- (-17.306098999999996,5.276209600000001) -- (-16.831698999999997,4.899502600000002) -- (-16.476338999999996,4.565491600000001) -- (-16.069528999999996,4.129264600000002) -- (-15.439978999999996,3.3590266000000013) -- (-14.866978999999995,2.5887886000000013) -- (-13.646518999999994,0.8632125800000013) -- (-12.291138999999994,-1.0342793999999986) -- (-11.520898999999995,-2.070968399999999) -- (-10.934118999999995,-2.802876399999999) -- (-10.750658999999995,-3.004436399999999) -- (-10.454318999999995,-3.276774399999999) -- (-10.365518999999994,-3.342054399999999) -- (-9.980408699999995,-3.5308373999999993) -- (-9.804008699999995,-3.573117399999999) -- (-9.595288699999996,-3.5981873999999996) -- (-9.184298699999996,-3.5989553999999995) -- (-8.439928699999996,-3.531515399999999) -- (-6.899448899999996,-3.326723399999999) -- (-6.129218899999996,-3.203469399999999) -- (-5.358978899999996,-3.052299399999999) -- (-5.156148899999996,-3.005679399999999) -- (-4.4202888999999965,-2.802855399999999) -- (-3.8184988999999963,-2.584261399999999) -- (-3.418608899999996,-2.402963399999999) -- (-2.943198899999996,-2.137682399999999) -- (-2.663148899999996,-1.949202399999999) -- (-2.278028899999996,-1.6387793999999989) -- (-2.085258899999996,-1.4551443999999987) -- (-1.892908899999996,-1.247818399999999) -- (-1.603978899999996,-0.8772603999999989) -- (-1.363888899999996,-0.49214239999999887) -- (-1.2145588999999961,-0.19890943999999888) -- (-1.0224288999999962,0.2780955800000011) -- (-0.9532288999999962,0.4938165800000011) -- (-0.8177788999999962,1.0483335800000013) -- (-0.7374788999999962,1.566551700000001) -- (-0.7096788999999961,1.846333700000001) -- (-0.6801788999999961,2.588809600000001) -- (-0.7088788999999961,3.3876496000000014) -- (-0.737478899999996,3.7224656000000014) -- (-0.784978899999996,4.129285600000002) -- (-0.9133988499999961,4.899523600000001) -- (-0.960198849999996,5.122187600000001) -- (-1.094478749999996,5.669761600000001) -- (-1.186378749999996,5.991131600000001) -- (-1.3315587499999961,6.439999600000001) -- (-1.6279387499999962,7.210237600000001) -- (-1.7208387499999962,7.423289600000001) -- (-2.086378749999996,8.172124600000002) -- (-2.413018749999996,8.750712600000002) -- (-2.657528749999996,9.141451600000002) -- (-3.048258749999996,9.704764600000003) -- (-3.511198749999996,10.291188600000002) -- (-4.0238687499999966,10.856061100000002) -- (-4.229698749999996,11.061427100000001) -- (-4.824158749999996,11.596248100000002) -- (-5.358978749999996,12.015127100000003) -- (-5.702268749999996,12.258615100000002) -- (-6.230978749999996,12.601903100000001) -- (-6.899448749999996,12.991842100000001) -- (-7.710718749999996,13.413169100000001) -- (-8.721278549999996,13.8610251) -- (-9.210168549999995,14.0383931) -- (-9.550858549999996,14.142379100000001) -- (-10.109548949999995,14.271527100000002) -- (-10.750638949999995,14.346617100000001) -- (-11.356608949999995,14.336107100000001) -- (-11.787398949999995,14.2756471) -- (-12.296978949999994,14.1423891) -- (-12.839738949999994,13.920773100000002) -- (-13.145308949999993,13.757269100000002) -- (-13.446478949999994,13.567497100000002) -- (-13.831598949999995,13.282834100000002) -- (-14.216708949999994,12.952734100000002) -- (-14.601828949999994,12.583680100000002); %>>>
	
	\path[draw=gradient2] %<<<
	(-10.365179,-5.6189164) -- (-10.365179,-5.6189164) -- (-10.447179,-5.4987144) -- (-10.600519,-5.3489694) -- (-10.750259,-5.267379399999999) -- (-10.928089,-5.229139399999999) -- (-11.019989,-5.228916399999999) -- (-11.135349000000001,-5.2443664) -- (-11.332219000000002,-5.3061164) -- (-11.520469000000002,-5.403536399999999) -- (-11.750529000000002,-5.576206399999999) -- (-11.905589000000003,-5.734373399999999) -- (-12.020729000000003,-5.883798399999999) -- (-12.122269000000003,-6.052238399999999) -- (-12.215869000000003,-6.268916399999999) -- (-12.291769000000004,-6.6550894) -- (-12.286769000000003,-6.850733399999999) -- (-12.251469000000004,-7.039154399999999) -- (-12.234469000000004,-7.095624399999999) -- (-12.151569000000004,-7.2848934) -- (-12.058569000000004,-7.424270399999999) -- (-11.905869000000004,-7.5792034) -- (-11.768939000000005,-7.672463400000001) -- (-11.701239000000005,-7.707083400000001) -- (-11.488069000000005,-7.776723400000001) -- (-11.292419000000004,-7.795533400000001) -- (-11.101859000000005,-7.775633400000001) -- (-10.864229000000005,-7.695643400000001) -- (-10.736849000000005,-7.6236434000000015) -- (-10.630499000000004,-7.544253400000001) -- (-10.507559000000004,-7.424262400000002) -- (-10.365369000000005,-7.229012400000002) -- (-10.271469000000005,-7.039143400000002) -- (-10.221969000000005,-6.895728400000002) -- (-10.171869000000004,-6.654025400000002) -- (-10.157969000000005,-6.446591400000002) -- (-10.172869000000006,-6.205201400000002) -- (-10.200469000000005,-6.0487464000000015) -- (-10.278369000000005,-5.796698400000001) -- (-10.365469000000004,-5.618869400000001); %>>>
	
	\path[draw=gradient3] %<<<
	(-11.520539,13.557181100000001) -- (-11.520539,13.557181100000001) -- (-11.677159,13.528741100000001) -- (-11.905659,13.468851100000002) -- (-12.290779,13.314110100000002) -- (-12.413979000000001,13.248920100000003) -- (-12.801169000000002,12.986997100000004) -- (-13.136979000000002,12.677845100000004) -- (-13.207279000000002,12.601875100000004) -- (-13.515379000000001,12.216755100000004) -- (-13.831229,11.726160100000003) -- (-14.326339,10.786275100000003) -- (-15.371699000000001,8.632543600000004) -- (-15.467699000000001,8.365565600000004) -- (-15.519199000000002,8.127883600000002) -- (-15.529199000000002,7.980446600000002) -- (-15.520199000000002,7.8325856000000025) -- (-15.467399000000002,7.595327600000003) -- (-15.308969000000003,7.210208600000002) -- (-15.243369000000003,7.081301600000002) -- (-14.602039000000003,5.987153600000003) -- (-13.445999000000004,4.129256600000003) -- (-12.712449000000005,3.0099016000000027) -- (-12.291329000000005,2.4040556000000026) -- (-11.521089000000005,1.3848045800000026) -- (-11.013409000000005,0.7857455800000027) -- (-10.532349000000005,0.2780665800000027) -- (-9.980608700000005,-0.2266524399999973) -- (-9.644248700000006,-0.49217039999999723) -- (-9.210378700000005,-0.7877583999999972) -- (-8.440138700000006,-1.1921243999999973) -- (-8.271628700000006,-1.2624043999999972) -- (-7.669898900000006,-1.4691633999999971) -- (-7.222098900000006,-1.584838399999997) -- (-6.899658900000006,-1.6510983999999973) -- (-6.129418900000005,-1.7583443999999973) -- (-5.600758900000005,-1.7910643999999971) -- (-5.1149189000000055,-1.7883643999999972) -- (-4.850048900000005,-1.771534399999997) -- (-4.588948900000005,-1.7426943999999969) -- (-4.238198900000006,-1.681894399999997) -- (-3.8187089000000056,-1.569359399999997) -- (-3.593678900000006,-1.4874293999999972) -- (-3.048468900000006,-1.2074113999999971) -- (-2.644538900000006,-0.896096399999997) -- (-2.521518900000006,-0.7758213999999971) -- (-2.278228900000006,-0.4872423999999972) -- (-1.991768900000006,-0.008393439999997199) -- (-1.871518900000006,0.2780725800000028) -- (-1.755488900000006,0.6631925800000028) -- (-1.681188900000006,1.048310580000003) -- (-1.639588900000006,1.433430580000003) -- (-1.6221889000000058,1.9326767000000027) -- (-1.6356889000000059,2.4611966000000027) -- (-1.6518889000000059,2.732572600000003) -- (-1.708188900000006,3.359024600000003) -- (-1.812318900000006,4.129262600000003) -- (-1.878018900000006,4.499181600000003) -- (-2.024288900000006,5.153541600000003) -- (-2.170238900000006,5.669738600000003) -- (-2.396528900000006,6.321782600000003) -- (-2.558268900000006,6.719909600000003) -- (-2.869888900000006,7.388893600000003) -- (-3.186488900000006,7.980452600000003) -- (-3.421458900000006,8.377797600000003) -- (-3.818808900000006,8.983385600000004) -- (-4.2210089000000055,9.520927600000004) -- (-4.752408900000005,10.127806600000003) -- (-4.9135989000000055,10.291165600000003) -- (-5.557628900000005,10.863058100000003) -- (-6.129518900000005,11.283870100000003) -- (-6.460988900000006,11.500173100000003) -- (-7.016538900000006,11.831642100000003) -- (-8.469088700000006,12.630736100000004) -- (-9.441498700000006,13.141093100000004) -- (-9.989318700000005,13.372118100000005) -- (-10.586919000000005,13.536147100000004) -- (-10.960539000000006,13.581707100000004) -- (-11.313329000000007,13.580007100000005) -- (-11.521189000000007,13.557217100000004); %>>>
	
	\path[draw=gradient3] %<<<
	(-19.222919,8.8528076) -- (-19.222919,8.8528076) -- (-19.349509,8.7506866) -- (-19.469579,8.6122316) -- (-19.505278999999998,8.5581316) -- (-19.608078999999996,8.3174625) -- (-19.633778999999997,8.1730115) -- (-19.628778999999998,7.960161499999998) -- (-19.586378999999997,7.787893499999998) -- (-19.492879,7.595333499999998) -- (-19.333999,7.402773499999998) -- (-19.162919,7.270690499999998) -- (-19.023958999999998,7.196450499999998) -- (-18.838278999999996,7.130070499999999) -- (-18.577078999999998,7.086300499999998) -- (-18.332358999999997,7.089400499999998) -- (-18.068049,7.136920499999998) -- (-17.736898999999998,7.264175499999998) -- (-17.682899,7.2931554999999975) -- (-17.490339,7.421348499999998) -- (-17.297779,7.610940499999998) -- (-17.168398999999997,7.8510684999999985) -- (-17.145898999999996,7.9804404999999985) -- (-17.158998999999998,8.129871499999998) -- (-17.213199,8.2810075) -- (-17.297798999999998,8.4193725) -- (-17.490358999999998,8.627715499999999) -- (-17.743928999999998,8.811695499999999) -- (-18.068039,8.9619205) -- (-18.202199,9.0016305) -- (-18.453149,9.0426305) -- (-18.806219,9.023600499999999) -- (-18.906288999999997,8.999770499999999) -- (-19.008668999999998,8.965410499999999) -- (-19.223388999999997,8.852805499999999); %>>>
	
	\path[draw=gradient3] %<<<
	(-11.135419,-6.4079344) -- (-11.135419,-6.4079344) -- (-11.308969000000001,-6.3772044) -- (-11.403369000000001,-6.3861044) -- (-11.559439000000001,-6.4420644) -- (-11.660919000000002,-6.513674399999999) -- (-11.774629000000001,-6.6541033999999994) -- (-11.815929,-6.7437933999999995) -- (-11.844929,-6.8677464) -- (-11.837929,-7.039225399999999) -- (-11.809329,-7.135305399999999) -- (-11.712829,-7.290467399999999) -- (-11.518958999999999,-7.425660399999999) -- (-11.327708999999999,-7.461560399999999) -- (-11.133378999999998,-7.422580399999999) -- (-10.916548999999998,-7.2591214) -- (-10.839248999999999,-7.1284534) -- (-10.800748999999998,-6.988492399999999) -- (-10.797748999999998,-6.852885399999999) -- (-10.824848999999999,-6.729372399999999) -- (-10.857148999999998,-6.654112399999999) -- (-10.979728999999997,-6.499166399999999) -- (-11.029328999999997,-6.461556399999999) -- (-11.134708999999997,-6.407976399999999); %>>>
	
	\path[draw=gradient4] %<<<
	(-10.352499,11.048721100000002) -- (-10.352499,11.048721100000002) -- (-10.350499,11.061401100000001) -- (-10.346499,11.465254100000001) -- (-10.365199,11.593501100000001) -- (-10.430099,11.831638100000001) -- (-10.533319,12.048608100000001) -- (-10.594919,12.1407381) -- (-10.750309000000001,12.3084851) -- (-10.942869000000002,12.4364081) -- (-10.992569000000001,12.4589781) -- (-11.219559000000002,12.5177981) -- (-11.303459000000002,12.5224981) -- (-11.520579000000001,12.4920981) -- (-11.721819000000002,12.4006481) -- (-11.921249000000001,12.2167721) -- (-11.958549000000001,12.1639121) -- (-12.049549,11.9755501) -- (-12.088349000000001,11.8120231) -- (-12.097349000000001,11.637792099999999) -- (-12.063849000000001,11.417478099999999) -- (-12.033049000000002,11.319558099999998) -- (-11.900239000000003,11.061412099999998) -- (-11.727089000000003,10.855279099999999) -- (-11.501249000000003,10.676294099999998) -- (-11.260589000000003,10.551541099999998) -- (-11.135839000000002,10.508001099999998) -- (-10.927189000000002,10.467631099999998) -- (-10.711659000000003,10.483731099999998) -- (-10.558159000000003,10.565961099999997) -- (-10.433899000000004,10.744586099999998) -- (-10.365599000000003,10.971917099999997) -- (-10.352899000000003,11.048727099999997); %>>>
	
	\path[draw=gradient4] %<<<
	(-9.7814587,9.9060436) -- (-9.7814587,9.9060436) -- (-9.9800687,9.8209736) -- (-10.134109,9.6749676) -- (-10.221309,9.5209246) -- (-10.345489,9.116088600000001) -- (-10.365189,9.016428600000001) -- (-10.404689000000001,8.7506906) -- (-10.430589000000001,8.4308726) -- (-10.430339000000002,7.9804526) -- (-10.393239000000001,7.5673996) -- (-10.333039000000001,7.2102146) -- (-10.186199000000002,6.6459386) -- (-10.081409000000003,6.3388056) -- (-9.980238700000003,6.0783346) -- (-9.802028700000003,5.6697386) -- (-9.418918700000003,4.8995006) -- (-8.990808700000002,4.1292626) -- (-8.614128700000002,3.5333916) -- (-8.439768700000002,3.2885915999999997) -- (-8.054648700000001,2.8167745999999996) -- (-7.669528900000001,2.4277635999999996) -- (-6.890208900000001,1.8185486999999996) -- (-5.9733889000000016,1.2039695799999997) -- (-5.743928900000002,1.0164955799999995) -- (-5.391428900000002,0.6631925799999996) -- (-4.973698900000002,0.22318455999999964) -- (-4.794818900000002,0.07182955999999963) -- (-4.507058900000002,-0.10704544000000038) -- (-4.206838900000002,-0.21172944000000038) -- (-4.078558900000002,-0.23193944000000033) -- (-3.9431589000000016,-0.2372394400000003) -- (-3.718338900000002,-0.2070894400000003) -- (-3.6257389000000018,-0.17894944000000035) -- (-3.433178900000002,-0.08395944000000033) -- (-3.273228900000002,0.04657855999999966) -- (-3.2192289000000023,0.10683855999999964) -- (-3.088388900000002,0.3183955799999997) -- (-3.034488900000002,0.47059657999999965) -- (-2.999488900000002,0.7117145799999997) -- (-2.999118900000002,0.7772145799999997) -- (-3.048018900000002,1.1007175799999995) -- (-3.174258900000002,1.4333905799999997) -- (-3.5710389000000022,2.2036286999999994) -- (-3.641038900000002,2.4115815999999994) -- (-3.698838900000002,2.7081225999999994) -- (-3.7269389000000017,3.3589855999999996) -- (-3.7279389000000016,3.5653105999999997) -- (-3.7584389000000016,4.1888606) -- (-3.8180389000000017,4.5885006) -- (-3.9251489000000017,5.0065706) -- (-4.076728900000002,5.4110126) -- (-4.200768900000002,5.6744756) -- (-4.334448900000002,5.9235236) -- (-5.103558900000001,7.2101746) -- (-5.658328900000001,8.2802296) -- (-6.030838900000001,8.848561600000002) -- (-6.285518900000001,9.135770600000003) -- (-6.513868900000001,9.339784600000003) -- (-6.898988900000001,9.597835600000003) -- (-7.057038900000001,9.678935600000003) -- (-7.380918900000001,9.809200600000002) -- (-7.689398900000001,9.895920600000002) -- (-8.006058900000001,9.954300600000002) -- (-8.4394687,9.994620600000001) -- (-8.918158700000001,9.999620600000002) -- (-9.2097087,9.986600600000003) -- (-9.6291087,9.940330600000003) -- (-9.7813387,9.906040600000003); %>>>
	
	\path[draw=gradient4] %<<<
	(-18.709609,8.2594855) -- (-18.709609,8.2594855) -- (-18.731909,8.282865500000002) -- (-18.837839000000002,8.338645500000002) -- (-18.895639000000003,8.344745500000002) -- (-18.988639000000003,8.3238055) -- (-19.064139000000004,8.2692855) -- (-19.117439000000005,8.1730055) -- (-19.122439000000004,8.0720515) -- (-19.084839000000002,7.9804415) -- (-19.030839000000004,7.9247915) -- (-18.935539000000002,7.8832015) -- (-18.838239,7.8849015) -- (-18.741939000000002,7.9319714999999995) -- (-18.678539,8.0132415) -- (-18.656739,8.1035215) -- (-18.675439,8.2027115) -- (-18.710039000000002,8.259451499999999); %>>>
	
	\path[draw=gradient4] %<<<
	(-11.258119,-6.9083944) -- (-11.258119,-6.9083944) -- (-11.285919,-6.8887044) -- (-11.363819,-6.8748344) -- (-11.424218999999999,-6.8995844) -- (-11.461219,-6.9429144) -- (-11.475919,-6.994684400000001) -- (-11.470918999999999,-7.0391944) -- (-11.445419,-7.0873344000000005) -- (-11.406818999999999,-7.1185644) -- (-11.323519,-7.131574400000001) -- (-11.263119,-7.103514400000001) -- (-11.223619,-7.046694400000001) -- (-11.216619,-6.9910544) -- (-11.231019,-6.9428044) -- (-11.257319,-6.9084044); %>>>
	
	\path[draw=gradient5] %<<<
	(-8.8247087,9.1734246) -- (-8.8247087,9.1734246) -- (-8.8803087,9.1358046) -- (-9.0977187,8.9432446) -- (-9.2434687,8.7506856) -- (-9.2899687,8.6705956) -- (-9.4131387,8.3655626) -- (-9.448438699999999,8.2190125) -- (-9.479838699999998,7.9804435) -- (-9.478838699999999,7.6577926) -- (-9.4556387,7.4558535) -- (-9.3640387,7.0561506) -- (-9.2099787,6.6372116) -- (-9.1201787,6.4399675) -- (-8.3339987,4.8994915) -- (-7.9767789,4.1292536) -- (-7.8702989,3.9284665999999997) -- (-7.6695189,3.5993525999999996) -- (-7.4941289,3.3590155999999998) -- (-7.2126889,3.0456085999999996) -- (-6.8992789,2.7756035999999997) -- (-6.6219988999999995,2.5887776) -- (-6.298918899999999,2.4188946) -- (-6.129038899999999,2.3467046) -- (-5.6793588999999995,2.2036557) -- (-5.2354088999999995,2.1320657) -- (-4.9736788999999995,2.1568857) -- (-4.874278899999999,2.2036557) -- (-4.732078899999999,2.3471996) -- (-4.699378899999999,2.3962196000000002) -- (-4.588508899999999,2.6039576) -- (-4.378918899999999,3.1494256000000003) -- (-4.319718899999999,3.3590166000000004) -- (-4.240818899999999,3.7816536000000003) -- (-4.222018899999999,4.1292546) -- (-4.2676188999999995,4.578747600000001) -- (-4.353418899999999,4.8994926) -- (-4.418318899999999,5.069514600000001) -- (-4.588338899999999,5.405709600000001) -- (-4.7603689,5.6697306) -- (-5.3810189,6.4624056) -- (-5.6566689,6.9121236) -- (-5.785128899999999,7.2102066) -- (-6.128818899999999,8.1755036) -- (-6.220218899999999,8.3655636) -- (-6.328278899999999,8.551224600000001) -- (-6.513938899999999,8.799356600000001) -- (-6.680868899999999,8.968873600000002) -- (-6.899058899999999,9.136878600000003) -- (-7.147058899999999,9.272914600000004) -- (-7.2841789,9.327914600000003) -- (-7.669298899999999,9.418574600000003) -- (-7.7624989,9.427674600000003) -- (-8.0530487,9.426174600000003) -- (-8.3059187,9.387264600000004) -- (-8.508358699999999,9.328334600000003) -- (-8.7322587,9.228172600000004) -- (-8.824658699999999,9.173392600000005); %>>>
	
	\path[draw=gradient5] %<<<
	(-11.135419,12.307708100000001) -- (-11.135419,12.307708100000001) -- (-11.325899000000001,12.3426881) -- (-11.399899000000001,12.3373881) -- (-11.580039000000001,12.2763481) -- (-11.713049000000002,12.1701061) -- (-11.806949000000001,12.0242331) -- (-11.852649000000001,11.8316741) -- (-11.847649,11.7161991) -- (-11.799749,11.552322100000001) -- (-11.712949,11.413809100000002) -- (-11.633049,11.333859100000002) -- (-11.508728999999999,11.253989100000002) -- (-11.373168999999999,11.208619100000002) -- (-11.271989,11.198189100000002) -- (-11.135239,11.215649100000002) -- (-10.978519,11.289829100000002) -- (-10.942719,11.318049100000001) -- (-10.807029,11.503414100000002) -- (-10.759629,11.691640100000003) -- (-10.766629,11.848063100000003) -- (-10.824729,12.024226100000003) -- (-10.868929,12.098106100000004) -- (-10.978569,12.216782100000005) -- (-11.135369,12.307732100000004); %>>>
	
	\path[draw=gradient5] %<<<
	(-4.3958389,1.6304756) -- (-4.3958389,1.6304756) -- (-4.4921389000000005,1.6042855999999999) -- (-4.588438900000001,1.5359256000000001) -- (-4.5954389,1.5279256) -- (-4.6568389,1.4334656) -- (-4.7044389,1.2990506000000002) -- (-4.7257389000000005,1.1859296000000001) -- (-4.7357389,1.0483456) -- (-4.725338900000001,0.8674296) -- (-4.6812389,0.6632276) -- (-4.6575389000000005,0.5940676) -- (-4.5883389,0.4454676) -- (-4.4706989,0.2781076) -- (-4.332058900000001,0.14926857999999998) -- (-4.2032189,0.07030857999999995) -- (-4.0752389,0.02096857999999996) -- (-3.9195389000000005,-0.005561420000000039) -- (-3.8180989000000003,-0.003961419999999993) -- (-3.6884289000000003,0.020748580000000016) -- (-3.5318789,0.09081857999999998) -- (-3.4329789,0.16638858) -- (-3.3378789,0.2781416) -- (-3.2403789,0.5129366) -- (-3.2263789000000003,0.6632616) -- (-3.2542789,0.8558205999999999) -- (-3.2602789,0.8754306000000001) -- (-3.3347789,1.0483796) -- (-3.4331788999999997,1.1906615999999999) -- (-3.5976788999999996,1.3512476) -- (-3.8182988999999994,1.49337455) -- (-4.049258899999999,1.5876545499999999) -- (-4.215248899999999,1.62605455) -- (-4.3959788999999985,1.6305545499999998); %>>>
	
	\path[draw=gradient6] %<<<
	(-8.4395887,8.6243756) -- (-8.4395887,8.6243756) -- (-8.6019087,8.3655676) -- (-8.6582087,8.1990826) -- (-8.6901087,7.9804486) -- (-8.6778087,7.7422066) -- (-8.6472087,7.5953296) -- (-8.5320787,7.302694600000001) -- (-8.4395787,7.1487126000000005) -- (-8.2470187,6.914281600000001) -- (-8.0544587,6.742661600000001) -- (-7.8060589,6.576688600000001) -- (-7.5162489,6.439972600000001) -- (-7.3384589,6.385732600000001) -- (-7.2114889,6.3672326) -- (-7.0916589,6.378412600000001) -- (-6.9778989,6.439962600000001) -- (-6.899098899999999,6.529972600000002) -- (-6.7501188999999995,6.8250806000000015) -- (-6.6556188999999994,7.108323600000001) -- (-6.592518899999999,7.402757600000001) -- (-6.569818899999999,7.661667600000001) -- (-6.592518899999999,7.980436600000001) -- (-6.656618899999999,8.222873600000002) -- (-6.718718899999999,8.3655556) -- (-6.961468899999999,8.6882356) -- (-7.035268899999999,8.750675600000001) -- (-7.284168899999999,8.894038600000002) -- (-7.469888899999999,8.950078600000001) -- (-7.669288899999999,8.970678600000001) -- (-7.870648899999999,8.952028600000002) -- (-8.0544087,8.898168600000002) -- (-8.3082687,8.750669600000002) -- (-8.439528699999999,8.624358600000003); %>>>
	
	\path[draw=gradient6] %<<<
	(-6.5139889,5.9131405) -- (-6.5139889,5.9131405) -- (-6.6516489000000005,5.9172405) -- (-6.743848900000001,5.8995905) -- (-6.817748900000001,5.8623305000000006) -- (-6.8990489,5.780850500000001) -- (-6.9646489,5.6697765) -- (-7.0119489,5.5568865) -- (-7.0881489,5.2846574) -- (-7.116548900000001,5.1170084000000005) -- (-7.1349489,4.8995385) -- (-7.1232489,4.5788454000000005) -- (-7.0721489,4.302385500000001) -- (-7.0177489,4.129300500000001) -- (-6.8990289,3.873735500000001) -- (-6.6817189,3.576372500000001) -- (-6.6096189,3.503742500000001) -- (-6.4297289,3.3590595000000008) -- (-6.1287489,3.2044765000000006) -- (-5.9184889,3.1487965000000004) -- (-5.7533589,3.1331065000000002) -- (-5.5720789,3.1454865) -- (-5.3585089,3.2052865) -- (-5.2549089,3.2554565) -- (-5.1055289,3.3590565000000003) -- (-4.973388900000001,3.4925156000000004) -- (-4.815558900000001,3.7441756) -- (-4.733158900000001,3.9844256000000002) -- (-4.708058900000001,4.1292936000000005) -- (-4.700058900000001,4.288849600000001) -- (-4.721858900000001,4.514413500000001) -- (-4.781158900000001,4.7313275) -- (-4.855658900000001,4.8995315999999995) -- (-5.033778900000001,5.1644906) -- (-5.1452489,5.2846516) -- (-5.3588189,5.4629196) -- (-5.4832389,5.545349600000001) -- (-5.7439389,5.6834696000000005) -- (-6.1290588999999995,5.827992600000001) -- (-6.3434389,5.8841426000000006) -- (-6.5141789,5.9131726); %>>>
	
	\path[draw=gradient6] %<<<
	(-4.1892489,0.8557475800000001) -- (-4.1892489,0.8557475800000001) -- (-4.2032489,0.83207758) -- (-4.2516489,0.6631855799999999) -- (-4.2496489,0.58643558) -- (-4.2031489,0.43315558) -- (-4.1400489,0.34113558) -- (-4.0705489,0.27805557999999997) -- (-3.9456389000000005,0.21542558) -- (-3.8180489000000004,0.19718558000000003) -- (-3.7410489000000005,0.20588558) -- (-3.5842189000000007,0.27801558000000004) -- (-3.4948189000000007,0.37429558) -- (-3.4453189000000006,0.48286158) -- (-3.4299189000000005,0.59774058) -- (-3.4359189000000003,0.66634058) -- (-3.4851189000000002,0.80325058) -- (-3.5701189,0.9107785800000001) -- (-3.6851689000000003,0.9882885800000001) -- (-3.7919289000000003,1.02241858) -- (-3.9141089000000004,1.02641858) -- (-4.1037789,0.9491585800000001) -- (-4.1888789,0.8556985800000001); %>>>
	
	\path[draw=gradient6] %<<<
	(-11.327979,12.180515100000001) -- (-11.327979,12.180515100000001) -- (-11.370278999999998,12.1791151) -- (-11.460178999999998,12.1563751) -- (-11.530878999999999,12.1153951) -- (-11.611379,12.024255100000001) -- (-11.642279,11.9533751) -- (-11.654879000000001,11.8316941) -- (-11.609879000000001,11.6907971) -- (-11.520679000000001,11.5919271) -- (-11.423779000000001,11.5428471) -- (-11.328079,11.5283071) -- (-11.230779,11.5438371) -- (-11.135479,11.5945871) -- (-11.051879,11.6936071) -- (-11.034678999999999,11.730877099999999) -- (-11.013079,11.8316851) -- (-11.030279,11.9568101) -- (-11.061879,12.0242401) -- (-11.151479,12.1205201) -- (-11.231679,12.1623401) -- (-11.327979,12.1805601); %>>>

	\end{scope}

	\end{tikzpicture}
}
\caption
[Level curves of $\left\lvert P(z)\right\rvert$.]
{Level curves of $\left\lvert P(z)\right\rvert = M$ for specific level values showing the neighborhoods $U\left(z_k\right)$ forming as $M$ diminishes.}
\label{fig:pz_level_curves}
\end{figure} %>>>

	\noindent and let

\begin{equation} K\left(M\right) = \left\{z\in\mathbb{C}: \left\lvert H_1(z)\right\rvert \leq  M\right\} .\end{equation}

	\noindent or, in other words, $K\left(M\right)$ is the sublevel set of $f(x,y) = \left\lvert H_1\left(x + jy\right)\right\rvert \leq M$. Because any polynomial in complex space is holomorphic, its counter-image is closed — thus $K(M)$ is always closed, and clearly contains all roots of $H_1$. It is intuitive to see that $K(M)$ becomes smaller as $M$ also gets smaller, so that if the roots of $H_1$ are all in the open left half plane, there exists a small enough $M_0$ (equivalently, a large enough $\left\lvert z_0\right\rvert$ and a small enough $\omega_0$) such that $K\left(M_0\right)$ will be enclosed in that half plane. Formally, it is known that the volume of sublevel sets of continuous functions on riemannian manifolds reduce their volumes continually as the level is reduced, and tends to zero as the level tends to zero. For instance, \cite{jubinIntrinsicVolumesSublevel2024} shows a closed formula for such volume if the function in question is thrice-differentiable. On the other hand, using the inverse triangle inequality on \eqref{eq:h3_ineq_qsg} one concludes that $0 \leq \left\lvert H_3\right\rvert \leq 2M$ in $K(M)$. Therefore $K\left(M_0\right)$ also contains all roots of $H_3$, and since it is wholly enclosed in the open left half plane, this means $H_3$ is Hurwitz stable.

	To illustrate this, figure \ref{fig:pz_level_curves} shows several level curves for the polynomial $P(z)$ of \eqref{eq:poly_gausslucas_example}. The plots clearly show that, as $M$ gets smaller, the regions defined by $\left\lvert P(z)\right\rvert = M$ become disjoint and progressively smaller, yet still closed. The figure shows that $K(260)$ is entirely in the open left half plane; therefore so will be $K(M\leq 260)$. Thus for any combination of $z_0$ and $\omega_0$ such that $M\left(z_0,\omega_0\right) \leq 260$, all roots of $H_3$ also lie in $K(M)$, and $H_3$ will be Hurwitz.

	Furthermore, pick a $z_k\in r\left(H_1\right)$. Because the roots of a polynomial are isolated, for small enough $M$, say $M_k$, $K\left(M_k\right)$ will be comprised of disconnected regions where one such region is a neighborhood of $z_k$ where no other root of $H_1$ lies. Let $U\left(z_k\right)$ be such one neighborhood around a root $z_k$, which is closed and simply connected. Because it is simply connected we can use Rouché's Theorem to conclude that $H_3 - H_1$ and $H_1$ have the same number of roots inside $U\left(z_k\right)$, thus $H_3$ has the same number of roots that $H_1$ in that region. Since $U\left(z_k\right)$ contains only one root $z_k$ of $H_1$, then there is a root of $H_3$ inside $U\left(z_k\right)$, and this root will have the same multiplicity than $z_k$.

	Further, because $M$ gets smaller as $\left\lvert z_0\right\rvert$ gets larger and $\omega_0$ gets smaller, the neighborhoods $U\left(z_k\right)$ get smaller as well, thus approximating the roots of $H_3$ to those of $H_1$. Figure \ref{fig:pz_level_curves} shows that as $M$ gets smaller, $K(M)$ becomes disconnected regions; for $M=260$, $K(M)$ is just one big region whereas for $M=230$, it becomes two regions, one of which contains only $z_2$. Thus one root of $H_3$ will also be in this neighborhood containing $z_2$. For each subsequent value of $M$ the regions become smaller and separate into neighborhoods of the roots, so that at $M=12$ $K(12)$ becomes six neighborhoods each one containing a root of $H_1$.\hfill$\blacksquare$ \vspace{3mm}\hrule\vspace{3mm} %>>>

	In short, corollary \ref{corollary:h3_h1_roots_qsh} defines that the three-phase polynomial $H_3$ will also be Hurwitz stable given that the roots of $H_1$ are ``sufficiently stable'' (have large enough negative real parts) and the frequency $\omega_0$ is sufficiently ``slow''. For instance, \eqref{eq:poly_gausslucas_example_h3} shows the roots of $H_3$ calculated for the example polynomial $P(z)$ of \eqref{eq:poly_gausslucas_example} when $\omega_0 = 200$ rad.s$^{-1}$. Figure \ref{fig:h3_p_roots} depicts the roots of $H_1$ and $H_3$ in the complex plane, showing they are indeed very close.

\begin{equation} H_3(z) = \prod_{k=1}^6 \left(z - z_k\right)\left\{\begin{array}{l} z_1 = -5158.4777 + j2003.9222 \\ z_2 = -3180.2023 - j2040.3105 \\ z_3 = -3168.9499 + j3048.5748 \\ z_4 = -2207.5821 + j2054.2120 \\ z_5 = -1741.7266 + j1006.0870 \\ z_6 = -1249.0614 - j72.485443 \end{array}\right.  \label{eq:poly_gausslucas_example_h3} \end{equation}

% H3 VS P ROOTS <<<
\begin{figure}[htb!]
\centering
\scalebox{2}{
	\begin{tikzpicture}[scale=10,>={Stealth[inset=0mm,length=1mm,angle'=50]},square/.style={regular polygon,regular polygon sides=4},y=1mm, x=1mm]
		\draw [->, line width = 0.1mm] (   -5mm,  0   ) -- (   1mm,  0   ); \node[label={\tiny $\Re$}] at (1mm,-0.6mm) {};
		\draw [->, line width = 0.1mm] (      0, -2mm ) -- (   0   ,  3mm) node[label={\tiny $\Im$}] (yaxis) {};
%
		\foreach \x/\y in {-5/2,-1.5/1,-1/0,-3/3,-3/-2,-2/2} {\node [circle, fill=gausslucas1, inner sep=0mm, minimum size=1mm] (p) at (\x mm,\y mm) {};}
%
		\node at (0.5mm, 2mm   ) [fill=black, regular polygon, regular polygon sides = 3, minimum size=1.2mm, inner sep=0mm] {}; \node[scale = 0.8, black, right] at (0.6mm, 2mm) {\tiny $r\left(H_3\right)$};
		\node at (0.5mm, 1.5mm ) [fill = gausslucas1, square ,inner sep = 0mm, minimum size = 2.5mm] (v100) {}; \node[scale = 0.8,right,gausslucas1] at (0.6mm, 1.5mm)   {\tiny $r\left(P\right)$};
	\foreach \x/\y in {-5.15847773924/2.00392222893, -3.18020230321/-2.04031048349, -3.16894990284/3.04857477312, -2.20758205951/2.05421197833, -1.74172662011/1.00608694574, -1.2490613751/-0.07248544263} { \node[fill=black, regular polygon, regular polygon sides = 3, minimum size=1.2mm, inner sep=0mm] (p) at (\x mm,\y mm) {};}
	
	\end{tikzpicture}
}
\caption
[Roots of $P(z)$ of \eqref{eq:poly_gausslucas_example} and of the three-phase polynomial $H_3$.]
{Roots of $P(z)$ of \eqref{eq:poly_gausslucas_example} and of the characteristic polynomial of the three-phase polynomial $H_3$ calculated using $\omega_0 = 200$ rad.s$^{-1}$.}
\label{fig:h3_p_roots}
\end{figure} %>>>

	Therefore, even if the apparent frequency $\omega(t)$ is time varying but equivalent to a forcing which zero-sequence component $f_0$ is null, then the zero-sequence response $z_0$ will inevitably vanish; thus the circuit three-phase response $\mathbf{x}$ will assymptotically tend to a three-phase quantity.

%-------------------------------------------------
\section{Frequency control modelling and timescales: the Quasi-static Hypothesis}\label{sec:freq_modelling_timescales} %<<<1

	From all these developments, we can conclude several things:

\begin{enumerate}
	\item If a forcing $\mathbf{f}(t)$ of sinusoids is such that each component is defined at some particular apparent frequency, but these frequencies are mutually integrable, then $\mathbf{f}(t)$ can be written in a common frequency $\omega_0(t)$;
	\item As such, if this signal $\mathbf{f}$ excites a linear system, then it will respond with a vector of sinusoids at the frequency $\omega_0$;
	\item Because of this, a linear matrix system admits a phasor-vector representation \eqref{eq:theo_nonautodiffeq_def}, where the Dynamic Phasor Transform was taken at $\omega_0$;
	\item This linear system yields to different yet equivalent models when modelled using two different frequency signals, and the solutions of the models can be reconstructed from one another;
	\item The differential equations from these two systems are diffeomorphic — ``equivalent'' in some way, and they reconstruct the same signals in time.
\end{enumerate}

	Consider equation \eqref{eq:nonautodiffeq} of a linear circuit modelling a transmission system with a vector of nonstationary sinusoidal forcings $f$ representing machine, inverter and agents voltages and currents upon the transmission grid. Each agent works at a particular local frequency $\omega_k$, like machine rotor frequency and inverter PLL frequencies, and applies a forcing $f_k$ to the grid, like machine internal voltages and stator currents and inverter bridge voltages and bus currents. In general, these quantities depend on the voltages and currents of the transmission system: for instance, induced voltages of machines depend on bus current, and the frequency of the rotor depends on electrical power given as a composition of induced voltage and currents. It is also common that the frequency $\omega_k$ depends on the forcings themselves; for instance, the machine rotor frequency depends on the internal voltage induced on the stator, which is a forcing of the transmission grid circuit, Thus we suppose that the forcings and frequencies have differential models that depend on each other and the transmission states, that is, there exist two functions $g_\omega^k$ and $g_f^k$ such that


\begin{gather}
	\boldsymbol{\Omega}_k = \left[\begin{array}{c} \omega_k \\[3mm] \dot{\omega}_k\\[3mm] \vdots \\[3mm] \omega_k^{(p)} \end{array}\right] \Rightarrow \dot{\boldsymbol{\Omega}} = g_\omega^k \left(t,\mathbf{x},\boldsymbol{\Omega}_k,\theta_k\right) \label{eq:diff_model_omega}\\
	\boldsymbol{\theta}_k = \left[\begin{array}{c} f_k \\[3mm] \dot{f}_k \\[3mm] \vdots \\[3mm] f^{(q)}_k \end{array}\right] \Rightarrow \dot{\boldsymbol{\theta}}_k = g_f^k\left(t,\mathbf{x},\boldsymbol{\Omega}_k,\boldsymbol{\theta}_k\right) \label{eq:diff_model_theta}
\end{gather}

	We suppose that the system has $m$ agents with differential models such as \eqref{eq:diff_model_omega} and \eqref{eq:diff_model_theta} and the transmission grid has a $n$-th order differential model, that is, $x$ has size $n$. Then

\begin{gather}
	\boldsymbol{\Omega} = \left[\begin{array}{c} \boldsymbol{\Omega}_1 \\[3mm] \boldsymbol{\Omega}_2 \\[3mm] \vdots \\[3mm] \boldsymbol{\Omega}_{m} \end{array}\right] \Rightarrow \dot{\boldsymbol{\Omega}} = \left[\begin{array}{c} g_\omega^1\left(t,\mathbf{x},\boldsymbol{\theta}_1,\boldsymbol{\Omega}_1\right) \\[3mm] g_\omega^2\left(t,\mathbf{x},\boldsymbol{\theta}_2,\boldsymbol{\Omega}_2\right) \\[3mm] \vdots \\[3mm] g_\omega^m\left(t,\mathbf{x},\boldsymbol{\theta}_m,\boldsymbol{\Omega}_m\right) \end{array}\right] = g_\omega\left(t,\mathbf{x},\boldsymbol{\theta},\boldsymbol{\Omega}\right) \\[5mm]
%
	\boldsymbol{\theta} = \left[\begin{array}{c} \boldsymbol{\theta}_1 \\[3mm] \boldsymbol{\theta}_2 \\[3mm] \vdots \\[3mm] \boldsymbol{\theta}_{m} \end{array}\right] \Rightarrow \dot{\boldsymbol{\theta}} = \left[\begin{array}{c} g_{\boldsymbol{\theta}}^1\left(t,\mathbf{x},\boldsymbol{\theta}_1,\boldsymbol{\Omega}_1\right) \\[3mm] g_{\boldsymbol{\theta}}^2\left(t,\mathbf{x},\boldsymbol{\theta}_2,\boldsymbol{\Omega}_2\right) \\[3mm] \vdots \\[3mm] g_{\boldsymbol{\theta}}^m\left(t,\mathbf{x},\boldsymbol{\theta}_m,\boldsymbol{\Omega}_m\right) \end{array}\right] = g_{\boldsymbol{\theta}}\left(t,\mathbf{x},\boldsymbol{\theta},\boldsymbol{\Omega}\right)
\end{gather}

	Thus we achieve a generalized Power System model

\begin{equation}
	\left\{\begin{array}{l}
		\dot{\mathbf{x}} = \mathbf{Ax + Bf},\ \mathbf{x}\left(0\right) = \mathbf{x}_0\\[2mm]
		\dot{\boldsymbol{\theta}} = g_{\boldsymbol{\theta}}\left(t,\mathbf{x},\boldsymbol{\theta},\boldsymbol{\Omega}\right) \\[2mm]
		\dot{\boldsymbol{\Omega}} = g_\omega\left(t,\mathbf{x},\boldsymbol{\theta},\boldsymbol{\Omega}\right)
	\end{array}\right. .\label{eq:lemma_time_complex}
\end{equation}

	We now transform this system into a phasorial-equivalent one. We adopt $\omega = \kappa\left(\Omega\right)$ as the frequency for the Dynamic Phasor Transform; this frequency can be for instance the grid center of frequency given by either pure averages or weighted averages of frequencies. We suppose $\kappa$ is continuous. Thus the first equation of \ref{eq:lemma_time_complex_final} can be directly transformed using theorem \ref{theo:dp_diffeq}. For the frequency and forcing dynamics, denote $\Theta = \mathbf{P_D^{\left(\omega\right)}} \left[\theta\right]$ and $X = \mathbf{P_D^{\left(\omega\right)}} \left[x\right]$ the Dynamic Phasor of the forcings and the states respectively:

\begin{equation}
	\left\{\begin{array}{l}
		\dot{\boldsymbol{\Theta}} + j\omega \mathbf{I}_m\boldsymbol{\Theta} = g_{\boldsymbol{\theta}}\left(t,\mathbf{P_D^{\left(-\omega\right)}} \left[X\right],\mathbf{P_D^{\left(-\omega\right)}} \left[\boldsymbol{\Theta}\right],\boldsymbol{\Omega}\right) \\[2mm]
		\dot{\boldsymbol{\Omega}} = g_\omega\left(t,\mathbf{P_D^{\left(-\omega\right)}} \left[X\right]\mathbf{P_D^{\left(-\omega\right)}} \left[\boldsymbol{\Theta}\right],\boldsymbol{\Omega}\right)
	\end{array}\right. \label{eq:lemma_time_complex_final}
\end{equation}

	\noindent and because $\mathbf{P_D}$ and its inverse are not only continuous but diffeomorphic in the Banach Space of Nonstationary Sinusoids \cite{volpatoDynamicPhasorTheory2025}, then this can be noted as

\begin{equation}
	\left\{\begin{array}{l}
		\dot{\boldsymbol{\Theta}} = G_{\boldsymbol{\theta}}\left(t,\mathbf{X},\boldsymbol{\Theta},\boldsymbol{\Omega}\right) \\[2mm]
		\dot{\boldsymbol{\Omega}} = G_{\boldsymbol{\omega}}\left(t,\mathbf{X},\boldsymbol{\Theta},\boldsymbol{\Omega}\right)
	\end{array}\right. 
\end{equation}

	\noindent thus achieving a generalized phasorial modelling of the Power System as

\begin{equation}
	\left\{\begin{array}{l}
		\dot{\mathbf{X}} = \left(\mathbf{A} - j\omega \mathbf{I}_n\right)\mathbf{X} + \mathbf{BF} \\[2mm]
		\dot{\boldsymbol{\Theta}} = G_{\boldsymbol{\theta}}\left(t,X,\boldsymbol{\Theta},\boldsymbol{\Omega}\right) \\[2mm]
		\dot{\boldsymbol{\Omega}} = G_{\boldsymbol{\omega}}\left(t,X,\boldsymbol{\Theta},\boldsymbol{\Omega}\right) \\[2mm]
		\omega = \kappa\left(\boldsymbol{\Omega}\right)
	\end{array}\right. ,\label{eq:lemma_time_complex_final}
\end{equation}

%-------------------------------------------------
\subsection{Exploring timescales} %<<<2

	We know turn our concern towards a particular case where the top equation of \eqref{eq:lemma_time_complex_final} — that models the electrical network dynamics — is much faster than the bottom equation that models frequency dynamics. We want to prove that if the circuit is ``fast'', then we can approximate the top equation that models the grid behavior by its steady-state behavior. Formally, we want to prove that the solution of the system

\begin{equation}
	\left\{\begin{array}{l}
		\mathbf{0} = \left(\mathbf{A} - j\omega_a \mathbf{I}_n\right)\mathbf{X}_a + \mathbf{BF}_a \\[2mm]
		\dot{\boldsymbol{\Theta}}_a = G_\theta\left(t,\mathbf{X}_a,\boldsymbol{\Theta}_a,\boldsymbol{\Omega}_a\right) \\[2mm]
		\dot{\boldsymbol{\Omega}}_a = G_\omega\left(t,\mathbf{X}_a,\boldsymbol{\Theta}_a,\boldsymbol{\Omega}_a\right) \\[2mm]
		\omega_a = \kappa\left(\boldsymbol{\Omega}_a\right)
	\end{array}\right. ,\label{eq:lemma_time_complex_final_approx}
\end{equation}

	\noindent where the subscript ``a'' denotes ``approximation or ``algebraic'', approximates the solution of the original system \eqref{eq:lemma_time_complex_final}. We first ask how we formally define a ``fast'' circuit, which albeit an intuitive concept, needs formalization, in the form of theorem \ref{theo:generic_rlc_modelling}.

	From a circuit theory perspective, this happens when the circuit RLC elements are all very low; the system supplies a high quantity of power for resistive loads while the energy storage elements cannot store big quantities of energy or, in other words, the circuit stores very little energy while quickly spending it. From a Power System perspective, this is the assumption that the frequency dynamics, are much quicker than the circuit dynamics; this is a resonable assumption if the system under scrutiny is a ``classical'' power system where the agents are electromechanical in nature, thus determining slow frequency dynamics. From a mathematics point of view, the top equation, that models the circuit, attains steady-state much quicker than the bottom equation modelling frequency dynamics, so that as the variable $\boldsymbol{\Omega}$ changes, the variable $X$ follows it in an almost-steady-state-like behavior.

	Under the assumption that the circuit is ``quick'' enough, we conclude that the grid differential equation (the first equation of \eqref{eq:lemma_time_complex_final}) attains steady-state much faster than the frequency control — the second equation — such that as $\omega(t)$ is adjusted in time $X(t)$ exhibits a composition of very small transients and the ``algebraic solution'' $X_a$ that solves $\dot{X_a} = 0$ — the grid is supposed at a permanent static sinusoidal state while frequency dynamics, much slower than that of the grid, actuates upon it. Such is the Quasi-static Hypothesis (QSH).

	We now analyze the theory of two-timescale systems to prove these statements. This theory was first proposed by Tikhonov \pcite{Khalil2002} for autonomous systems of the form

\begin{equation} \left\{\begin{array}{l} \varepsilon \dfrac{dx}{dt} = f\left(x,y\right),\ x\left(t_0\right) = x_0, \\[3mm]\phantom{\varepsilon} \dfrac{dy}{dt} = g\left(x,y\right),\ y\left(t_0\right) = y_0 \end{array}\right. \end{equation}

	\noindent where $\varepsilon$ is a small positive parameter. Such systemas are called ``singularly perturbed'' systems \pcite{albertoCaracterizacaoEstimativasArea2010} and the general interest is to analyze the behavior of the system at, or close to, $\varepsilon = 0$. 

	Tikhonov proved that, under certain conditions, the dynamics of this system can be decomposed into a ``slow dynamic'' and a ``fast dynamic'' in such a way that if the dynamics of this system can be approximated by the model obtained when $\dot{x}(t) = 0$. However, taking from the model \eqref{eq:lemma_time_complex_final}, the system under study is more complicated: it has a non-automomous system modelled by

\begin{equation}\left(\Lambda_\varepsilon\right):\ \left\{\begin{array}{l} \varepsilon \dfrac{dx}{dt} = f\left(t,x,y,\varepsilon\right),\ x\left(t_0\right) = x_0, \\[3mm]\phantom{\varepsilon} \dfrac{dy}{dt} = g\left(t,x,y,\varepsilon\right),\ y\left(t_0\right) = y_0 \end{array}\right. . \label{eq:original_single_persystem}\end{equation}

	 Here we use a generalized version of Tikhonov's Theorem for this larger class of systems as presented in \cite{Marva2012}. The state $x(t)$ is called the ``fast state'' while $y(t)$ is the ``slow state''. We denote the trajectory of this system starting from $\left(t_0,x_0,y_0\right)$ as

\begin{equation} \varphi_\varepsilon\left(t,t_0,x_0,y_0\right) = \left[x_\varepsilon\left(t,t_0,x_0,y_0\right),y_\varepsilon\left(t,t_0,x_0,y_0\right)\right]^\intercal . \end{equation}

	Considering a time interval $t_0 \leq t \leq T$, we first suppose that the states $x,y$ exist in neighborgoods of $x_0,y_0$ and that $x(t)$ and $y(t)$ stay in these neighborhoods in that time interval. This guarantees that the system does not explode or ``jerk''. Making $\varepsilon = 0$ on $\left(\Lambda_\varepsilon\right)$ one obtains the ``slow system''

\begin{equation}\left(\Lambda_s\right):\ \left\{\begin{array}{rcl} 0 &=& f\left(t,x,y,\varepsilon\right) \\[3mm]\phantom{\varepsilon} \dfrac{dy}{dt} &=& g\left(t,x,y,\varepsilon\right),\ y(0) = y_0 \end{array}\right. .\end{equation}

	\noindent yielding a set of algebraic-differential equations. This system describes the dynamics of the slower state $y(t)$ in the standard timescale $t$ supposing that $x(t)$ is ``infinitely fast'', that is, it reaches steady-state immediately and continuously. We denote the trajectory of this system as

\begin{equation} \varphi_s\left(t,t_0,x_0,y_0\right) = \left[x_s\left(t,t_0,x_0,y_0\right),y_s\left(t,t_0,x_0,y_0\right)\right]^\intercal \end{equation}

	\noindent where the subscript ``s'' stands for ``slow''. Naturally, the equation $0 = f\left(t,x,y\right)$ restricts this system to a ``slow manifold'' which contains the equilbria of the original system $\left(\Lambda_\varepsilon\right)$ at $t_0$. Also, by the implicit function theorem \pcite{Lima2017b}, if the partial derivative $f_x$ is not singular at $\left(t,\Phi\left(t,y\right),y,0\right)$ then there exists a single local solution $ x = \Phi\left(t,y\right)$ at the instant $t$, such that (locally) the system can be written in a reduced form

\begin{equation}\left(\Lambda_{r}\right):\ \left\{\begin{array}{rcl} x &=& \Phi\left(t,y\right) \\[3mm] \dfrac{dy}{dt} &=& g\left(t,\Phi\left(t,y\right),y,\varepsilon\right),\ y\left(t_0\right) = y_0 \end{array}\right. .\end{equation}

	\noindent and $\Phi$ is the candidate of the steady-state approximation for $x(t)$. We also suppose that $\Phi$ is defined in the initial neghborhoods of the initial conditions. Finally, we divide the fast variable equation of \eqref{eq:original_single_persystem} by $\varepsilon$ and denote a ``fast timescale'' $\tau = t/\varepsilon$, generating a description of that system in a fast timescale:

\begin{equation}\left\{\begin{array}{l} \dfrac{dx}{d\tau} = f\left(\tau,x,y,\varepsilon\right),\ x\left(t_0\right) = x_0, \\[3mm]\dfrac{dy}{d\tau} = \varepsilon g\left(\tau,x,y,\varepsilon\right),\ y\left(t_0\right) = y_0 \end{array}\right. \end{equation}

	\noindent and making $\varepsilon = 0$ in these equations generates a ``fast system'':

\begin{equation}\left(\Lambda_f\right): \left\{\begin{array}{l} \dfrac{dx}{d\tau} = f\left(\tau,x,y,\varepsilon\right),\ x\left(t_0\right) = x_0, \\[3mm]\dfrac{dy}{dt} = 0 \end{array}\right. \end{equation}

	\noindent which supposes that $y(t)$ is ``infinitely slow'', that is, $\left(\Lambda_f\right)$ denotes how the dynamics of the fast variable $x(t)$ vary in a fast timescale where the slow variable $y(t)$ has not has enough time to change; therefore, with respect to the dynamics of $x(t)$, $y(t)$ is constant and treated as a parameter, that is,

\begin{equation}\left(\Lambda_f\right): \dfrac{dx}{d\tau} = f\left(\tau,x,y,\varepsilon\right),\ x\left(t_0\right) = x_0 \end{equation}

	\noindent and the trajectory of this system is denoted 

\begin{equation} \varphi_f\left(t,t_0,x_0\right) = x_f\left(t,t_0,x_0\right) .\end{equation}

	Given additional requirements on $f$ and $g$, \cite{Marva2012} proves that the solution of the fast system $x_f$ vanishes quickly in time, and it also varies little in amplitude, culminating in theorem \ref{theo:qsh_approx_nonlinivps} which states that 

\begin{equation}\left\{\begin{array}{l} \lim\limits_{t\to\infty} \left\lVert x_\varepsilon(t) - \Phi\left(t,y_s\left(t\right)\right)\right\rVert = O\left(\varepsilon\right) \\[2mm] \lim\limits_{t\to\infty} \left\lVert y_\varepsilon(t) - y_s(t) \right\rVert = O\left(\varepsilon\right) \end{array}\right. \end{equation}

	\noindent that is, the behavior of the original system $\left(\Lambda_\varepsilon\right)$ can be approximated by the dynamics of the slow system. Additionally, if $\Phi$ and $y_s$ are assymptotically stable, then

\begin{equation}\left\{\begin{array}{l} \lim\limits_{t\to\infty} \left\lVert x_\varepsilon(t) - \Phi\left(t,y_s\left(t\right)\right)\right\rVert = 0 \\[2mm] \lim\limits_{t\to\infty} \left\lVert y_\varepsilon(t) - y_s(t) \right\rVert = 0 \end{array}\right. \end{equation}

	\noindent and $x(t),y(t)$ exist for all times $t\geq t_0$, meaning not only the behavior of the original system $\left(\Lambda_\varepsilon\right)$ can be approximated by the dynamics of the slow system, the trajectories converge assymptotically.

\begin{theorem}[Quasistatic-state approximation of nonlinear IVPs \pcite{Marva2012}]\label{theo:qsh_approx_nonlinivps} %<<<
	Consider the nonlinear IVP

\begin{equation}\left\{\begin{array}{l} \varepsilon \dfrac{dx}{dt} = f\left(t,x,y,\varepsilon\right),\ x\left(t_0\right) = x_0, \\[3mm]\phantom{\varepsilon} \dfrac{dy}{dt} = g\left(t,x,y,\varepsilon\right),\ y\left(t_0\right) = y_0 \end{array}\right. \label{sys:theo_quasistatic_sysdef}\end{equation}

	where $x\in\mathbb{R}^n,\ y\in\mathbb{R}^m$, $\varepsilon$ a small positive parameter. Let $S = I\times B_R\times B_{R'},\ I = \left\{t: t_0\leq t\leq T\leq \infty\right\},\ B_R = \left\{x\in\mathbb{R}^n: \left\lvert x\right\rvert \leq R\right\},\ B_{R'} = \left\{y\in\mathbb{R}^m:\left\lvert y\right\rvert \leq R'\right\}, \overline{S} = S\times\left[0,\varepsilon_0\right]$, $f,g\in C^2\left(S\right)$ and $T, \varepsilon_0$ constants. Suppose the following hypotheses H1-H4 are true:

\begin{itemize}
	\item\textbf{H1}: any solution of \eqref{sys:theo_quasistatic_sysdef} beggining in $B_R\times B_{R'}$ remains there for $t_0\leq t \leq T$;
	\item\textbf{H2}: there exists a function $\Phi\left(t,y\right)$ such that

\begin{equation} f\left(t,\Phi\left(t,y\right),y,0\right) = 0 \label{eq:theo_qsh_approx_nonlinivps_limf}\end{equation}

	for all $\left(t,y\right)\in I\times B_{R'}$. Moreover, $\Phi\in C^2\left(I\times B_{R'}\right)$ and $f_x\left(t,\Phi\left(t,y\right),y,0\right)$ is nonsingular for all $\left(t,y\right)\in I\times B_{R'}$;
	\item\textbf{H3}: the equation

\begin{equation} \dfrac{dX}{d\tau} = f\left(\alpha,X,\beta,0\right) \label{eq:theo_qsh_approx_nonlinivps_limX}\end{equation}

	has $X = \Phi\left(\alpha,\beta\right)$ as an equilibrium for each $\left(\alpha,\beta\right)\in I\times B_{R'}$ and the initial condition $x_0$ is in the domain of attraction of the equilibrium $\Phi\left(t_0,y_0\right)$;
	\item\textbf{H4}: the equation

\begin{equation} \dfrac{dz}{dt} = g\left(t,\Phi\left(t,z\right),z,0\right) \label{eq:theo_qsh_approx_nonlinivps_limz}\end{equation}

	has a solution for $t_0\leq t < \infty$, say $y^*(t)$, and $y_0$ is in the domain of attraction of $y^*(t)$.
\end{itemize}

	Then, for sufficiently small values of $\varepsilon$, $\left(x(t),y(t)\right)$ exists for $t_0 \leq t \leq T$ and

\begin{equation}\left\{\begin{array}{l} \left\lVert x(t) - \Phi\left(t,y^*\left(t\right)\right)\right\rVert = O\left(\varepsilon\right) \\[2mm] \left\lVert y(t) - y^*(t) \right\rVert = O\left(\varepsilon\right) \end{array}\right. \label{eq:qsh_approx_theo_result}\end{equation}

	Additionally, if

\begin{itemize}
	\item\textbf{H3'}: the equilibrium $X = \Phi\left(\alpha,\beta\right)$ of \textbf{H3} is assymptotically stable uniformly; and 

	\item\textbf{H4'}: the solution $y^*(t)$ of \textbf{H4} is uniformly assymptotically stable;
\end{itemize}

	then $\left(x(t),y(t)\right)$ exists for $t_0 \leq t < \infty$ and

\begin{equation}\left\{\begin{array}{l} \lim\limits_{t\to\infty} \left\lVert x(t) - \Phi\left(t,y^*\left(t\right)\right)\right\rVert = 0 \\[2mm] \lim\limits_{t\to\infty} \left\lVert y(t) - y^*(t) \right\rVert = 0 \end{array}\right. \label{eq:qsh_approx_theo_assympt}\end{equation}
\end{theorem} %>>>

%-------------------------------------------------
\subsection{Applying theorem \ref{theo:qsh_approx_nonlinivps} to the modelling } %<<<2

	We now explore theorem \ref{theo:qsh_approx_nonlinivps} by applying it to the modelling \ref{eq:lemma_time_complex_final}. We suppose that the norm $\left\lVert \mathbf{A}\right\rVert$ becomes small and acts as the perturbation $\varepsilon$. A wider discussion on what this means for the circuit is taken following the result.

\begin{theorem}[Quasi-Static Modelling of Linear Electrical Circuits]\label{theo:qsh_linear_circuits}%<<<
	Consider the Dynamic Phasor complex differential equation \eqref{eq:lemma_time_complex_final} of a PLC with nonstationary sinusoidal forcing equipped with a frequequency control, where $F\in C^2\left(\mathbb{R}\right)$, $X_0,X\in B_R\subset \mathbb{C}^n$, $\Omega_0,\Omega = \left[\omega,\dot{\omega},...,\omega^{(p)}\right]^\intercal \in B_{R'}\subset\mathbb{R}^p$, and $\Gamma \in C^2\left(B_{R'}\times B_R \times I\right)$. Suppose that $t\in I = \left[0,T\right)$ for some $T$ such that $X(I)\subset B_{R}$. Let 

\begin{equation} X_a = -\left(\mathbf{A} - j\omega_a(t)\mathbf{I}\right)^{-1}\mathbf{B}F(t) \end{equation}

	\noindent be the candidate of steady-state approximation of $X(t)$, and suppose there exist solutions $\boldsymbol{\Omega}_a,\ \boldsymbol{\Theta}_a$ to

\begin{equation} \dfrac{d}{dt}\left[\begin{array}{c} \boldsymbol{\Omega}_a \\[3mm] \boldsymbol{\Theta}_a\end{array}\right] = \Gamma \left(X_a, \boldsymbol{\Theta}_a, \boldsymbol{\Omega}_a, t\right),\ \left[\begin{array}{c} \boldsymbol{\Omega}(0) \\[3mm] \boldsymbol{\Theta}(0) \end{array}\right] = \left[\begin{array}{c} \boldsymbol{\Omega}_0 \\[3mm] \mathbf{P_D^\omega}\left[\boldsymbol{\theta}_0\right] \end{array}\right] \label{eq:theo_slowfast_omegaa_def}\end{equation}

	for $t\in\left[0,T\right)$. Then for $\left\lVert \mathbf{A}\right\rVert$ large, $X(t),\ \boldsymbol{\Omega}(t)$ and $\boldsymbol{\Theta}(t)$ exist for $\left[0,T\right)$ and

\begin{equation}\left\{\begin{array}{l} \left\lVert X(t) - X_a(t)\right\rVert = O\left(\left\lVert \mathbf{A}\right\rVert^{-1}\right) \\[2mm] \left\lVert \left[\begin{array}{c} \boldsymbol{\Omega}(t) \\[3mm] \boldsymbol{\Theta}(t) \end{array}\right] - \left[\begin{array}{c} \boldsymbol{\Omega}_a(t) \\[3mm] \boldsymbol{\Theta}_a(t) \end{array}\right]\right\rVert = O\left(\left\lVert \mathbf{A}\right\rVert^{-1}\right) \end{array}\right. \label{eq:theo_slowfast_omegaa_approxresult} \end{equation}

	Additionally, if the moduli of the components of $F(t)$ are bounded and the solution $\Omega_a$ of \eqref{eq:theo_slowfast_omegaa_def} is also bounded, then $X(t),\ \Omega(t)$ exist for $\left[0,\infty\right)$ and

\begin{equation}\left\{\begin{array}{l} \lim\limits_{t\to\infty}\left\lVert X(t) - X_a(t)\right\rVert = 0 \\[3mm] \lim\limits_{t\to\infty}\left\lVert \left[\begin{array}{c} \boldsymbol{\Omega}(t) \\[3mm] \boldsymbol{\Theta}(t) \end{array}\right] - \left[\begin{array}{c} \boldsymbol{\Omega}_a(t) \\[3mm] \boldsymbol{\Theta}_a(t) \end{array}\right]\right\rVert = 0 \end{array}\right. \label{eq:theo_slowfast_omegaa_assympt} \end{equation}

\end{theorem}
\textbf{Proof:} adopt $\varepsilon = \left(\left\lVert \mathbf{A}\right\rVert\right)^{-1}$; we want to analyze the behavior of \eqref{eq:lemma_time_complex_final} as $\varepsilon\to 0^+$. Multiply the first equation of \eqref{eq:lemma_time_complex_final} by $\varepsilon$:

\begin{equation}\left\{\begin{array}{l} \varepsilon\dot{X} = \left(\mathbf{U_A} - j\varepsilon\omega(t) \mathbf{I}\right) X + \mathbf{U_B} F\left(t\right) \\[3mm] \dfrac{d}{dt}\left[\begin{array}{c} \boldsymbol{\Omega} \\[3mm] \boldsymbol{\Theta}\end{array}\right] = \Gamma \left(X, \boldsymbol{\Theta}, \boldsymbol{\Omega}, t\right),\ \left[\begin{array}{c} \boldsymbol{\Omega}(0) \\[3mm] \boldsymbol{\Theta}(0) \end{array}\right] = \left[\begin{array}{c} \boldsymbol{\Omega}_0 \\[3mm] \mathbf{P_D^\omega}\left[\boldsymbol{\theta}_0\right] \end{array}\right] \end{array}\right. .\label{eq:modified_circuit_de4}\end{equation}

	\noindent where $\mathbf{U_A} = \varepsilon \mathbf{A}, \mathbf{U_B} = \varepsilon \mathbf{B}$. The proof follows by showing that the hypotheses H1-H4 of theorem \ref{theo:qsh_approx_nonlinivps} are satisfied.

\begin{itemize}
	\item \textbf{H1} is satisfied by ensuring $X\left(\left[0,T\right]\right)\subset B_R$;
	\item \textbf{H2} is satisfied by adopting $X_a$ as $\Phi$ and seeing that $X_a$ is a solution to $\dot{X}(t) = 0$ in \eqref{eq:modified_circuit_de4} for any $\varepsilon$;
	\item \textbf{H3} is satisfied by requiring that the circuit has at least one resistance, thus $\mathbf{A}$ will be Hurwitz stable. If this is true then

\begin{equation}\dfrac{dX}{d\tau} = \left(\mathbf{A} - j\beta \mathbf{I}\right)X\left(\tau\right) + \mathbf{B}F\left(\alpha\right)\label{eq:theo_time_complex_final_assymt}\end{equation}

	is globally assymptotically uniformly stable due to being linear with a fixed forcing and because the matrix $\mathbf{A} - j\beta \mathbf{I}$ is invertible with all eigenvalues on the left plane. Thus $\mathbf{P_D^\omega}\left[x_0\right]$ is in the domain of attraction of $X_a$.
	\item \textbf{H4} is satisfied fulfilled by requiring \eqref{eq:theo_slowfast_omegaa_def} to have a solution.
\end{itemize}

	Additionally, if $F(t)$ has all moduli bounded, then it is bounded itself as the cosines are limited to the unit. Thus the excitation $F\left(\alpha\right)$ of \eqref{eq:theo_time_complex_final_assymt} is bounded. Because the matrix $\left(A-j\omega I_n\right)$ has only eigenvalues in the left plane (because such is the case of $A$ and removing $j\omega$ from the main diagonal only changes the imaginary component of eigenvalues) then \eqref{eq:theo_time_complex_final_assymt} is assymptotically stable if $\omega$ is defined for all infinity which, combined with continuity, means that $\omega$ is bounded — which is equivalent to $\Omega_a$ also being bounded; then \textbf{H3'} and \textbf{H4'} are satisfied and $X(t),\boldsymbol{\Theta(t)},\ \boldsymbol{\Omega(t)}$ exist for $\left[0,\infty\right)$ and the assymptotic stability result \eqref{eq:theo_slowfast_omegaa_assympt} holds. \hfill$\blacksquare$ %>>>

%-------------------------------------------------
\section{Exploring theorem \ref{theo:qsh_linear_circuits} and its consequences}%<<<1

%-------------------------------------------------
\subsection{The unitary matrices $U_A$ and $U_B$} %<<<2

	While theorem \ref{theo:qsh_linear_circuits} constitutes a rigorous statement of the Quasi-Static Hypothesis, the proof presented seems nonetheless too swift and the roles of the matrices $U_A$ and $U_B$ are not clear — except for the obvious reason to transform the original system \eqref{eq:lemma_time_complex_final} into a new version \eqref{eq:modified_circuit_de4} which can leverage theorem \ref{theo:qsh_approx_nonlinivps} to obtain the desired results. We first revisit theorem \ref{theo:generic_rlc_modelling} which states that any RLC circuit can be modelled as

\begin{equation} \mathbf{E}\dot{\mathbf{x}} = \left(\mathbf{J-K}\right)\mathbf{x}(t) + \mathbf{Gf}(t) \end{equation}

	where

\begin{equation} \mathbf{E} = \left[\begin{array}{cc} \mathbf{A}_C C \mathbf{A}_C^\intercal & \mathbf{0}\\[1mm] \mathbf{0} & L \end{array}\right],\ \mathbf{G} = \left[\begin{array}{c} \mathbf{A}_i \\ \mathbf{0} \end{array}\right],\ \mathbf{J} = \left[\begin{array}{cc} \mathbf{0} & -\mathbf{A}_L \\[1mm] \mathbf{A}_L^\intercal & \mathbf{0} \end{array}\right],\ \mathbf{K} = \left[\begin{array}{cc} \mathbf{A}_R \mathbf{R}^{-1}  \mathbf{A}_R^\intercal & \mathbf{0} \\[1mm] \mathbf{0} & \mathbf{0} \end{array}\right] ,\end{equation}

	\noindent and $\mathbf{A}_i$ is the input-to-node connectivity matrix. Suppose all inductance and capacitance values of a PLC are multiplied by a certain positive value $\varepsilon$, while resistances are maintained. This will scale the matrix $\mathbf{E}$ by $\varepsilon$, making the norms of the matrices $\mathbf{A} = \mathbf{E}^{-1}\left(\mathbf{J-R}\right)$ and $\mathbf{B} = \mathbf{E}^{-1}\mathbf{G}$ of \eqref{eq:lemma_time_complex} will be divided by $\varepsilon$. Noticeably, this causes the eigenvalues of $\mathbf{A}$ to be also divided by $\varepsilon$ and its eigenvectors stay the same; by theorem \ref{def:exp_charac} (page \pageref{def:exp_charac}), this means that the exponential terms of the homogeneous response have smaller absolute values while still being stable, thus fading quicker — meaning that as the $LC$ parameters become smaller, the circuit becomes ``faster''.

	Thus, if all LC values are divided by $\left\lVert \mathbf{A}\right\rVert$, the matrix of the new circuit will be $\mathbf{U_A}$ such that $\left\lVert \mathbf{U_A}\right\rVert = 1$; $\mathbf{U_A}$ and $\mathbf{U_B}$ in essence represent a ``standard''version of the circuit where the LC parameters are scaled so that the norm of $\mathbf{U_A}$ becomes \textit{unitary}, thus the naming ``$U$''. Let the circuit represented by $\mathbf{U_A}$ and $\mathbf{U_B}$ be called the \textit{unitary version} of the original circuit of $\mathbf{A}$ and $\mathbf{B}$. As $\left\lVert \mathbf{A}\right\rVert$ is excursionated to infinity (that is, $\varepsilon$ is made smaller approximating zero), $\mathbf{U_A}$ does not change, as well as $\mathbf{U_B}$, allowing for easily applying the results of theorem \ref{theo:qsh_approx_nonlinivps}. In contrast, using the original system \eqref{eq:lemma_time_complex_final} can be problematic because as the norm of $\mathbf{A}$ is excursionated, that is, as the LC parameters are multiplied, the matrix $\mathbf{A}$ itself changes, as well as $\mathbf{B}$ (that is, the circuit itself changes), making harder the application of theorem \ref{theo:qsh_approx_nonlinivps}.

%-------------------------------------------------
\subsection{Timescale analysis} %<<<2

	Despite making the application of theorem \ref{theo:qsh_approx_nonlinivps} simpler, the usage of the unitary circuit \eqref{eq:modified_circuit_de4} comes at a cost: it is denoted as transformed by apparent frequency $\varepsilon\omega$. Let $\tau = t/\varepsilon$ be a ``fast timescale'', $t$ the original one. Then the circuit equation of the unitary system \eqref{eq:modified_circuit_de4} becomes

\begin{equation} \dfrac{dX}{d\tau} = \left(\mathbf{U_A} - j\varepsilon\omega\left(\tau\right) \mathbf{I}_n\right) X\left(\tau\right) + \mathbf{U_B} F\left(\tau\right) ,\end{equation}

	\noindent causing the Dynamic Phasor Transform in this new timescale to be performed at a scaled apparent frequency $\varepsilon\omega\left(\tau\right)$, which makes sense since the unitary circuit is ``slower'' than the original circuit. The adoption of $\mathbf{U_A}$ and $\mathbf{U_B}$ as a ``unitary reference version'' of the original circuit means that the original circuit $\mathbf{A,B}$ is translated into a new timescale $\tau$ wherein the circuit does not change with $\varepsilon$. What changes is that the DPT is taken in this new timescale at the scaled frequency $\varepsilon\omega\left(\tau\right)$, and then the circuit is translated back into the original timescale. This is to maintain the frequency timescale, which should be kept because the proof relies on the fact that as $\varepsilon$ is made smaller, the circuit is swifter but the frequency behavior is maintained. This guarantees that when $X$ is transformed back to the $\Sigma_\omega$ space through $\mathbf{P_D^{\left(-\omega\right)}}$, the equivalent $x(t)$ is the same signal used in the frequency model $\gamma$, that is, it is a solution to the original time differential equation \eqref{eq:nonautodiffeq}. In simpler words, the adoption of $\mathbf{U_A}$ and $\mathbf{U_B}$ allows to consider a ``fixed circuit'' and vary the timescale and frequency at which it is analyzed, rather than change the circuit itself (which is what using the original matrices $\mathbf{A}$ and $\mathbf{B}$ entails to) and keeping the timescales intact.

	Finally, one notices that the pertinent functions \eqref{eq:theo_qsh_approx_nonlinivps_limf}, \eqref{eq:theo_qsh_approx_nonlinivps_limX}, \eqref{eq:theo_qsh_approx_nonlinivps_limz} of theorem \ref{theo:qsh_approx_nonlinivps} are defined at the equality $\varepsilon = 0$, but $\left\lVert \mathbf{A}\right\rVert = 0$ is unattainable unless $\mathbf{A}$ is the null matrix. One might adapt the definitions, however, using limits and the results remain because $g$ and $f$ are supposed continuous.

%-------------------------------------------------
\subsection{Assymptotic stability and effects of loads} %<<<2

	In theorem \ref{theo:qsh_approx_nonlinivps}, the additional requirements of hypotheses \textbf{H3'} and \textbf{H4'} essentially make it so that the solutions $\left(x(t),y(t)\right)$ are defined to infinity rather than just some interval $\left[0,T\right)$. The need for these conditions is clear in that, if the system \eqref{sys:theo_quasistatic_sysdef} under study is unstable this means that at some time $T_\infty$ the solutions explode; therefore the balls $B_R$ and $B_{R'}$ are defined to avoid choosing $T > T_\infty$. Assymptotic stability of the equilibriums $\Phi\left(\alpha,\beta\right)$ and $y^*(t)$ assure that the system will never behave in such explosive manner, while also meaning that while $x(t)$ and $y(t)$ evolve, they are always close to $\Phi$ and $y^*$ because any deviation vanishes assyptotically. This guarantees that solutions will exist for any time $T$ chosen, ergo being defined for infinity.

	When it comes to the application of theorem \ref{theo:qsh_linear_circuits}, the principles of $B_R$, $B_{R'}$ and $T$ still stand. The purpose of the additional requirements \textbf{H3'} and \textbf{H4'} become clearer as they signify that the circuit differential equations \eqref{eq:lemma_time_complex_final} must have bounded forcings $F(t)$ and bounded frequency $\omega$. While it is obvious that an unbounded forcing can drive a circuit to instability, it is not so obvious that an unbounded frequency excitation can accomplish the same effect. This can be further evaluated through eigenvalue analysis: even if $\mathbf{A}$ has a large norm, an unbounded $\omega$ means that the number $j\omega$ can get close to an eigenvalue of $\mathbf{A}$, meaning that $\mathbf{A} - j\omega(t)\mathbf{I}_n$ can have a small eigenvalue in some interval in time; during this interval, the circuit is not much faster than the frequency and the QSM fails. This can happen if the system is not furnishing enough load power (that is, the load resistance values are not low enough to draw sufficient current) and $\omega(t)$ approximates a natural resonant frequency of the system. If the system is experiencing high loading (low load resistance values) then even if the frequency $\omega$ approaches a natural mode of the system, the high loading will expend enough energy to keep the system ``quick enough'' to keep the QSH still valid. These conclusions might explain instability effects seen in light-loaded power systems \pcite{kundurPowerSystemStability1994} as well as stability issues in some ring amplifiers \pcite{Conrad2020}.

	Figure \ref{fig:voltage_signals_highload} shows the time simulation of the ``high load'' case, comprise of ``slow'' circuit $A_S$ but the resistance $R$ was reduced to $1\Omega$, that is, the circuit load was augmented tenfold. In contrast to the ``slow'' case of figure \ref{fig:voltage_signals_slow}, where transients take long to fade, the higher load case of figure \ref{fig:voltage_signals_highload} shows that a higher loading scenario causes not only for swifter transients but also greatly reduces the distance between the solution of the phasorial differential equations and their steady-state approximation.

\begin{example}[Application of theorem \ref{theo:qsh_linear_circuits}]\label{example:rlc_timescales}

	We again consider the second-order circuit of figure \ref{fig:secondordercircuit}, modelled in \eqref{eq:examplecircuit_model}, excited by a sinusoidal voltage $v(t)$ and $R = 10\Omega$, $L = 1$mH, $C = 1$mF.

% MODELLING EXAMPLE: RLC CIRCUIT <<<
\begin{figure}[htb!]
\centering
        \begin{tikzpicture}[american,scale=1,transform shape,line width=0.75, cute inductors, voltage shift = 1]
	\ctikzset{/tikz/circuitikz/voltage/distance from node=10mm}
		\draw (0,0)
			to[vsource,sources/scale=1.25, v>=$v(t)$,invert] (0,4)
			to[L,l=$L$,f>^=$i_{L}$,v>=$v_{L}$,-*] (4,4) 
			to[C,l=$C$,f>^=$i_{C}$,v>=$v_{C}$,-*] (4,0) 
			to[short] (0,0); 
		\draw (4,4)
			to[short,f>^=$i_{R}$] (7,4) 
			to[R,l=$R$,v>=$v_{R}$] (7,0) 
			to[short]  (4,0);
        \end{tikzpicture}
	\caption{Second-order circuit.}
	\label{fig:secondordercircuit}
\end{figure} %>>>

	The circuit modelling is given by

\begin{equation}\overbrace{\dfrac{d}{dt}\left[\begin{array}{c} i_L \\ v_C \end{array}\right]}^{\dot{\mathbf{x}}} = \overbrace{\left[\begin{array}{cc} 0 & -\dfrac{1}{L} \\[3mm] \dfrac{1}{C} & -\dfrac{1}{RC} \end{array}\right]}^{\mathbf{A}} \overbrace{\left[\begin{array}{c} i_L \\ v_C \end{array}\right]}^{\mathbf{x}} + \overbrace{\left[\begin{array}{c} \dfrac{1}{L}\\[3mm] 0\end{array}\right]}^{\mathbf{B}} \overbrace{\left[\begin{array}{c} v \\ 0 \end{array}\right]}^{\mathbf{f}} \label{eq:examplecircuit_model}\end{equation}

	Using the DPT at some apparent frequency $\omega$, and using theorem \ref{theo:dp_diffeq} and yields the phasor-equivalent

\begin{equation} \dfrac{d}{dt}\left[\begin{array}{c} I_L \\ V_C \end{array}\right] = \left(\mathbf{A} - j\omega \mathbf{I} \right) \left[\begin{array}{c} I_L \\ V_C \end{array}\right] +  \mathbf{B}\left[\begin{array}{c} V \\ 0 \end{array}\right] \label{eq:example_circuit_AB_diffeq} \end{equation}

	Hence transforming \eqref{eq:example_circuit_AB_diffeq} to the unitary circuit notation with the timescale transformation yields

\small\begin{equation} \dfrac{1}{\left\lVert \mathbf{A}\right\rVert} \dfrac{d}{dt}\left[\begin{array}{c} I_L \\ V_C \end{array}\right] = \left(\mathbf{U_A} - j\dfrac{\omega}{\left\lVert \mathbf{A}\right\rVert}\mathbf{I}_2\right) \left[\begin{array}{c} I_L \\ V_C \end{array}\right] + \mathbf{U_B} \left[\begin{array}{c} V \\ 0 \end{array}\right] \end{equation}\normalsize

	To make direct calculations easier, adopt the Frobenius norm $\left\lVert \cdot\right\rVert_F$ for matrices; calculating $\left\lVert \mathbf{A}\right\rVert_F$ yields

\begin{equation} \left\lVert \mathbf{A}\right\rVert_F = \sqrt{\dfrac{1}{L^2} + \dfrac{1}{C^2} + \dfrac{1}{\left(RC\right)^2}}.\end{equation}

	We consider three situations for the circuit:

\begin{itemize}
	\item $\mathbf{A_S}$ refers to the ``slow circuit'' with parameters $R = 10\Omega$, $L = 1$mH, $C = 1$mF, thus $\left\lVert\mathbf{A_S}\right\rVert_F \approx 1417.7447$. This version is the ``standard'' or ``benchmark'' version for comparison;
	\item $\mathbf{A_F}$ refers to a ``fast'' version circuit where the $L$ and $C$ parameters are divided by 10, but the resistance is kept, meaning $R = 10\Omega$, $L = 100\mu$H, $C = 100\mu$F, thus $\left\lVert\mathbf{A_F}\right\rVert_F \approx 14177.447$. This version of the circuit serves the purpose of showing the effects of the energy elements $L$ and $C$ on system dynamics, but keeping loading $R$ intact;
	\item $\mathbf{A_L}$ refers to a ``high-load'' version circuit where the $L$ and $C$ parameters are kept, but the resistance is divided by 10, meaning $R = 1\Omega$, $L = 1$mH, $C = 1$mF, hence $\left\lVert\mathbf{A_L}\right\rVert_F\approx 17320.508$, with the purpose of showing the effects of a higher loading point on the circuit but keeping the elements $L$ and $C$ intact.
\end{itemize}

	We again consider the excitation

\begin{equation} v(t) = m_v \cos\left(\psi(t)\right),\ \psi(t) = \int_0^t \omega(s)ds \end{equation}

	\noindent with $m_v = 10$V and the apparent frequency

\begin{equation} \omega(t) = \omega_0\left[1 + Me^{-\alpha t}\sin\left(\beta t\right)\right] .\end{equation}

	\noindent where $\omega_0$ is a 1kHz base frequency $\omega_0 = 2000\pi$ and a decaying behavior $M=1$, $\alpha = 100$, $\beta = 200\pi$.
 	Figures \ref{fig:voltage_signals_slow}, \ref{fig:voltage_signals_fast} and \ref{fig:voltage_signals_highload} shows the real and imaginary portions of the Dynamic Phasor of the capacitor voltage $V_C$ for the slow, fast and high-load cases, respectively. The pictures compare the solution obtained by directly integrating the DP differental system \eqref{eq:example_circuit_AB_diffeq} (in red) to the steady-state approximation (in blue). Figures \ref{fig:voltage_signals_slow} and \ref{fig:voltage_signals_fast} are illustrative of the results of theorem \ref{theo:qsh_linear_circuits} in that it shows that the ``fast'' circuit is more well-behaved than the ``slow'' circuit, for the latter exhibits transients that linger for longer and have greater amplitude, whereas the transients of the ``fast'' circuit are quicker and smaller. This causes the steady-state approximation to be verossimile in the fast case and usable, but questionable in the ``slow'' case. Since the excitation and the frequency signals are bounded in both cases, the differential solution is assymptotically stable to the steady-state approximation in both cases, meaning that even for the slow case the steady-state approxmation is perfectly applicable after transients have worn off.

	Interestingly, in the highload case, the transients are as well-behaved as in the fast case, albeit the capacitance and inductance values being the same as in the slow case. This again corroborates the fact that the loading condition of the circuit highly contribute to its dynamics, thus reflecting on the fitment of the steady-state approximation: the approximated solution better suits the high-load case than the slow case, even though they have the same inductance and capacitance values. This means that the role of the loading on the circuit is not only to make transients faster, but also tame their effects on the final circuit behavior.

% VOLTAGE TIME CURVES (SLOW CASE) <<<
\begin{figure}
        \begin{center}
                \beginpgfgraphicnamed{timesim_slow}
                \begin{tikzpicture}
                        \begin{axis}[
				name = ax_main,
                                width = 1\columnwidth,
                                height = 0.6/1.618*\columnwidth,
                                title={Capacitor voltage $V_C$ (slow circuit)},
                                ylabel={Re$\left(V_C\right)$ (V)},
				xlabel={Time (ms)},
                                xmin=0, xmax=0.06,
                                ymin=-0.3, ymax=0.4,
                                xtick={0,0.01,...,0.06},
				xticklabels={$0$,$10$,$20$,$30$,$40$,$50$,$60$,$70$,$80$,$90$,$100$},
				scaled x ticks=false,
                                ytick={-0.3,-0.2,...,0.4}, 
                                legend pos=south east,
				legend cell align={left},
                                ymajorgrids=true,
                                xmajorgrids=true,
                                every axis plot/.append style={thick},
                        ]
                        \addplot[red,  smooth] table[col sep=comma,header=false,x index=0,y index=1]{data/quasistationary/data_slow.csv};
			\addlegendentry{Dynamic response}
                        \addplot[blue, smooth] table[col sep=comma,header=false,x index=0,y index=3]{data/quasistationary/data_slow.csv};
			\addlegendentry{Steady-state approximation}
                        \end{axis}
%
                        \begin{axis}[
                                name = ax_imaginary,
                                at={($(ax_main.south west)-(0,0.35*\columnwidth)$)},
                                width = 1\columnwidth,
                                height = 0.6/1.618*\columnwidth,
                                xmin=0, xmax=0.06,
                                ymin=-0.5, ymax=0.15,
                                xtick={0,0.01,...,0.06},
				xlabel={Time (ms)},
                                ylabel={Im$\left(V_C\right)$ (V)},
				xticklabels={$0$,$10$,$20$,$30$,$40$,$50$,$60$,$70$,$80$,$90$,$100$},
				scaled x ticks=false,
                                ytick={-0.5,-0.4,...,0.1},
				tick label style={/pgf/number format/fixed},
				legend cell align={left},
                                ymajorgrids=true,
                                xmajorgrids=true,
                                every axis plot/.append style={thick},
                        ]
			\addplot[red, smooth] table[col sep=comma,header=false,x index=0,y index=2]{data/quasistationary/data_slow.csv};
			\addlegendentry{Dynamic response}
			\addplot[blue,smooth] table[col sep=comma,header=false,x index=0,y index=4]{data/quasistationary/data_slow.csv};
			\addlegendentry{Steady-state approximation}
                        \end{axis}
%
                        \begin{axis}[
                                at={($(ax_imaginary.south west)-(0,0.65*\columnwidth)$)},
                                width = 1\columnwidth,
                                height = 1/1.618*\columnwidth,
                                title={Capacitor voltage $V_C$ (slow circuit)},
                                xlabel={$\Re\left(V_C\right)$ (V)},
                                ylabel={$\Im\left(V_C\right)$ (V)},
                                xmin=-0.3, xmax=0.4,
                                xtick={-0.3,-0.2,...,0.4}, 
                                ymin=-0.5, ymax=0.15,
                                ylabel={Im$\left(V_C\right)$ (V)},
                                ytick={-0.5,-0.4,...,0.1},
                                legend pos=south east,
				legend cell align={left},
                                ymajorgrids=true,
                                xmajorgrids=true,
                                every axis plot/.append style={thick},
                        ]
                        \addplot[red,  smooth] table[col sep=comma,header=false,x index=1,y index=2]{data/quasistationary/data_slow.csv};
			\addlegendentry{Dynamic response}
                        \addplot[blue, smooth] table[col sep=comma,header=false,x index=3,y index=4]{data/quasistationary/data_slow.csv};
			\addlegendentry{Steady-state approximation}
                        \end{axis}

                \end{tikzpicture}
        \endpgfgraphicnamed
        \caption
[Components of the voltage across the capacitor of the circuit of figure \ref{fig:secondordercircuit} for the ``slow'' case.]
{Real and imaginary components of the voltage $V_C$ across the capacitor of the circuit of figure \ref{fig:secondordercircuit} for the ``slow'' case. In red the voltage $V_C$ obtained by integrating the differential equation \eqref{eq:example_circuit_AB_diffeq}, and in blue the steady-state approximation.}
        \label{fig:voltage_signals_slow}
        \end{center}
\end{figure}
% >>>
% VOLTAGE TIME CURVES (FAST CASE) <<<
\begin{figure}
        \begin{center}
                \beginpgfgraphicnamed{timesim_fast}
                \begin{tikzpicture}
                        \begin{axis}[
				name = ax_main,
                                width = 1\columnwidth,
                                height = 0.6/1.618*\columnwidth,
                                title={Capacitor voltage $V_C$ (fast circuit)},
                                ylabel={Re$\left(V_C\right)$ (V)},
				xlabel={Time (ms)},
                                xmin=0, xmax=0.06,
                                ymin=-25.5, ymax=25,
                                xtick={0,0.01,...,0.1},
				xticklabels={$0$,$10$,$20$,$30$,$40$,$50$,$60$},
				scaled x ticks=false,
                                ytick={-20,-10,...,20}, 
                                legend pos=north east,
				legend cell align={left},
                                ymajorgrids=true,
                                xmajorgrids=true,
                                every axis plot/.append style={thick},
                        ]
                        \addplot[red,  smooth] table[col sep=comma,header=false,x index=0,y index=1]{data/quasistationary/data_fast.csv};
			\addlegendentry{Dynamic response}
                        \addplot[blue, smooth] table[col sep=comma,header=false,x index=0,y index=3]{data/quasistationary/data_fast.csv};
			\addlegendentry{Steady-state approximation}
                        \end{axis}
%
                        \begin{axis}[
                                name = ax_imaginary,
                                at={($(ax_main.south west)-(0,0.35*\columnwidth)$)},
                                width = 1\columnwidth,
                                height = 0.6/1.618*\columnwidth,
                                xmin=0, xmax=0.06,
                                ymin=-3, ymax=33,
                                xtick={0,0.01,...,0.1},
				xlabel={Time (ms)},
                                ylabel={Im$\left(V_C\right)$ (V)},
				xticklabels={$0$,$10$,$20$,$30$,$40$,$50$,$60$},
				scaled x ticks=false,
                                ytick={0,5,...,30},
                                legend pos=north east,
				tick label style={/pgf/number format/fixed},
				legend cell align={left},
                                ymajorgrids=true,
                                xmajorgrids=true,
                                every axis plot/.append style={thick},
                        ]
			\addplot[red, smooth] table[col sep=comma,header=false,x index=0,y index=2]{data/quasistationary/data_fast.csv};
			\addlegendentry{Dynamic response}
			\addplot[blue,smooth] table[col sep=comma,header=false,x index=0,y index=4]{data/quasistationary/data_fast.csv};
			\addlegendentry{Steady-state approximation}
			\end{axis}
%
			\begin{axis}[
                                at={($(ax_imaginary.south west)-(0,0.65*\columnwidth)$)},
                                width = 1\columnwidth,
                                height = 1/1.618*\columnwidth,
                                title={Capacitor voltage $V_C$ (fast circuit)},
                                xlabel={Re$\left(V_C\right)$ (V)},
                                xmin=-25.5, xmax=25,
                                xtick={-20,-10,...,20}, 
                                ymin=-3, ymax=33,
                                ylabel={Im$\left(V_C\right)$ (V)},
                                ytick={0,5,...,30},
                                legend pos=north east,
				legend cell align={left},
                                ymajorgrids=true,
                                xmajorgrids=true,
                                every axis plot/.append style={thick},
                        ]
                        \addplot[red,  smooth] table[col sep=comma,header=false,x index=1,y index=2]{data/quasistationary/data_fast.csv};
			\addlegendentry{Dynamic response}
                        \addplot[blue, smooth] table[col sep=comma,header=false,x index=3,y index=4]{data/quasistationary/data_fast.csv};
			\addlegendentry{Steady-state approximation}
                        \end{axis}
                \end{tikzpicture}
        \endpgfgraphicnamed
        \caption
[Components of the voltage across the capacitor of the circuit of figure \ref{fig:secondordercircuit} for the ``fast'' case.]
{Real and imaginary components of the voltage $V_C$ across the capacitor of the circuit of figure \ref{fig:secondordercircuit} for the ``fast'' case. In red the voltage $V_C$ obtained by integrating the differential equation \eqref{eq:example_circuit_AB_diffeq}, and in blue the steady-state approximation.}
        \label{fig:voltage_signals_fast}
        \end{center}
\end{figure}
% >>>
% VOLTAGE TIME CURVES (HIGH LOAD) <<<
\begin{figure}
        \begin{center}
                \beginpgfgraphicnamed{timesim_highload}
                \begin{tikzpicture}
                        \begin{axis}[
				name = ax_main,
                                width = 1\columnwidth,
                                height = 0.6/1.618*\columnwidth,
                                title={Capacitor voltage $V_C$ (high load)},
                                ylabel={Re$\left(V_C\right)$ (V)},
				xlabel={Time (ms)},
                                xmin=0, xmax=0.06,
                                ymin=-0.3, ymax=0.4,
                                xtick={0,0.01,...,0.1},
				xticklabels={$0$,$10$,$20$,$30$,$40$,$50$,$60$},
				scaled x ticks=false,
				scaled x ticks=false,
                                ytick={-0.3,-0.2,...,0.4}, 
                                legend pos=south east,
				legend cell align={left},
                                ymajorgrids=true,
                                xmajorgrids=true,
                                every axis plot/.append style={thick},
                        ]
                        \addplot[red,  smooth] table[col sep=comma,header=false,x index=0,y index=1]{data/quasistationary/data_highload.csv};
			\addlegendentry{Dynamic response}
                        \addplot[blue, smooth] table[col sep=comma,header=false,x index=0,y index=3]{data/quasistationary/data_highload.csv};
			\addlegendentry{Steady-state approximation}
                        \end{axis}
%
                        \begin{axis}[
                                name = ax_imaginary,
                                at={($(ax_main.south west)-(0,0.35*\columnwidth)$)},
                                width = 1\columnwidth,
                                height = 0.6/1.618*\columnwidth,
                                xmin=0, xmax=0.06,
                                ymin=-0.4, ymax=0.08,
                                xtick={0,0.01,...,0.06},
				xlabel={Time (ms)},
                                ylabel={Im$\left(V_C\right)$ (V)},
				xticklabels={$0$,$10$,$20$,$30$,$40$,$50$,$60$},
				scaled x ticks=false,
                                ytick={-0.4,-0.3,...,0},
				tick label style={/pgf/number format/fixed},
				legend cell align={left},
				legend pos = north east,
                                ymajorgrids=true,
                                xmajorgrids=true,
                                every axis plot/.append style={thick},
                        ]
			\addplot[red, smooth] table[col sep=comma,header=false,x index=0,y index=2]{data/quasistationary/data_highload.csv};
			\addlegendentry{Dynamic response}
			\addplot[blue,smooth] table[col sep=comma,header=false,x index=0,y index=4]{data/quasistationary/data_highload.csv};
			\addlegendentry{Steady-state approximation}
			\end{axis}
%
                        \begin{axis}[
                                at={($(ax_imaginary.south west)-(0,0.65*\columnwidth)$)},
                                width = 1\columnwidth,
                                height = 1/1.618*\columnwidth,
                                title={Capacitor voltage $V_C$ (high load)},
                                xlabel={Re$\left(V_C\right)$ (V)},
                                xmin=-0.3, xmax=0.4,
                                xtick={-0.3,-0.2,...,0.4}, 
                                ymin=-0.4, ymax=0.08,
                                ylabel={Im$\left(V_C\right)$ (V)},
                                ytick={-0.4,-0.3,...,0},
                                legend pos=south east,
				legend cell align={left},
                                ymajorgrids=true,
                                xmajorgrids=true,
                                every axis plot/.append style={thick},
                        ]
                        \addplot[red,  smooth] table[col sep=comma,header=false,x index=1,y index=2]{data/quasistationary/data_highload.csv};
			\addlegendentry{Dynamic response}
                        \addplot[blue, smooth] table[col sep=comma,header=false,x index=3,y index=4]{data/quasistationary/data_highload.csv};
			\addlegendentry{Steady-state approximation}
                        \end{axis}
                \end{tikzpicture}
        \endpgfgraphicnamed
        \caption
[Components of the voltage across the capacitor of the circuit of figure \ref{fig:secondordercircuit} for the ``high load'' case.]
{Real and imaginary components of the voltage $V_C$ across the capacitor of the circuit of figure \ref{fig:secondordercircuit} for the ``high load'' case. In red the voltage $V_C$ obtained by integrating the differential equation \eqref{eq:example_circuit_AB_diffeq}, and in blue the steady-state approximation.}
        \label{fig:voltage_signals_highload}
        \end{center}
\end{figure}
% >>>

\examplebar
\end{example}

%-------------------------------------------------
\section{Proving the Quasi-Static Hypothesis}\label{sec:qsh_proof} %<<<1

	In immediate practical terms, what the QSH entails is that the complexification of linear circuits of theorem \ref{corollary:complex_equivalence_phasorialodes} can be simplified greatly. Suppose a linear system 

\begin{equation} \sum_{k=0}^n \alpha_k x^{(k)} - f(t)  = 0 \end{equation} 

	\noindent that is complexified as per theorem \ref{corollary:complex_equivalence_phasorialodes}, yielding a complex differential system

\begin{equation} \sum\limits_{i=0}^n \beta^n_i(t) \dfrac{d^iX(t)}{dt^i} - F(t) = 0,\ \beta_n^k(t) = \sum\limits_{k=i}^{n} \alpha_k{k\choose i}\left[ \sum\limits_{c=0}^{k-i} j^c B_{\left(k-i,c\right)}\left(\omega,\dot{\omega},\ddot{\omega},...,\omega^{(k-i-c)}\right)\right] . \label{eq:qsh_approx_line}\end{equation}

	We now build the matrix model of this system using the line-to-matrix ODE equivalence (theorem \ref{theorem:line_to_matrixode_equiv}): let $\mathbf{Y} = \left[X,\dot{X},\ddot{X},...,X^{(n-1)}\right]$ and

\begin{equation} \boldsymbol{\beta}(t) =
\left[\begin{array}{ccccc} 0 & 1 & 0 & ... & 0 \\[3mm] 0 & 0 & 1 & ... & 0  \\[3mm] \vdots & \vdots & \vdots & \ddots & \vdots \\[3mm] 0 & 0 & 0 & ... & 1 \\[3mm] - \dfrac{\beta^n_0}{\beta^n_n} & - \dfrac{\beta^n_1(t)}{\beta^n_n(t)} & -\dfrac{\beta^n_2(t)}{\beta^n_n(t)} & ... & -\dfrac{\beta^n_{(n-1)}(t)}{\beta^n_n(t)}
\end{array}\right] ,
%
\mathbf{F} = \left[\begin{array}{c} 0 \\[3mm] 0 \\[3mm] \vdots \\[3mm] \dfrac{F(t)}{\beta_n^n} \end{array}\right]
\end{equation}

	\noindent and one can write \eqref{eq:qsh_approx_line} in matrix form $\dot{\mathbf{Y}} = \boldsymbol{\beta} + \mathbf{F}$. In accordance with the modelling of section \ref{sec:freq_modelling_timescales}, we suppose that the apparent frequency $\omega$ adopted for the Dynamic Phasor Transform and the forcing $F(t)$ are given by

\begin{equation}
	\left(\Lambda_\varepsilon\right):\ \left\{\begin{array}{l} 
		\dot{\mathbf{Y}} = \boldsymbol{\beta}\mathbf{Y} + \mathbf{F} \\[3mm]
		\dfrac{d}{dt}\left[\begin{array}{c} \boldsymbol{\Omega} \\[3mm] \boldsymbol{\Theta}\end{array}\right] = G \left(\mathbf{Y}, \boldsymbol{\Theta}, \boldsymbol{\Omega}, t\right)
\end{array} \right.
\end{equation}

	\noindent comprising the initial system being studied, where $\boldsymbol{\Omega}$ and $\boldsymbol{\Theta}$ are the differential models of the frequency and the modelling, respectively. Obtaining the slow system and the steady-state approximations $\mathbf{Y}_a,\ \boldsymbol{\Omega}_a,\ \boldsymbol{\Theta}_a$ is done by adopting

\begin{equation}
	\left(\Lambda_s\right):\ \left\{\begin{array}{l} 
		\mathbf{0} = \boldsymbol{\beta}\mathbf{Y}_a + \mathbf{F} \\[3mm]
		\dfrac{d}{dt}\left[\begin{array}{c} \boldsymbol{\Omega}_a \\[3mm] \boldsymbol{\Theta}_a \end{array}\right] = G \left(\mathbf{Y}, \boldsymbol{\Theta}_a, \boldsymbol{\Omega}_a, t\right)
\end{array} \right.
\end{equation}

	\noindent and isolating $\mathbf{Y}_a$,

\begin{equation}
	\left(\Lambda_s\right):\ \left\{\begin{array}{l} 
		\mathbf{Y}_a = \boldsymbol{\beta}^{-1} \mathbf{F} \\[3mm]
		\dfrac{d}{dt}\left[\begin{array}{c} \boldsymbol{\Omega}_a \\[3mm] \boldsymbol{\Theta}_a \end{array}\right] = G'\left(\boldsymbol{\Theta}_a, \boldsymbol{\Omega}_a, t\right)
\end{array} \right. .
\end{equation}

	But we note that if $\boldsymbol{\beta}(t)$ is invertible then

\begin{equation} \left[\boldsymbol{\beta}(t)\right]^{-1} =
\left[\begin{array}{cccccc}
- \dfrac{\beta^n_1}{\beta^n_0} & - \dfrac{\beta^n_2}{\beta^n_0} & -\dfrac{\beta^n_3}{\beta^n_0} & ... & -\dfrac{\beta^n_{(n-1)}}{\beta^n_0} & -\dfrac{\beta^n_n}{\beta^n_0} \\[3mm]
1 & 0 & 0 & ... & 0 & 0 \\[3mm]
0 & 1 & 0 & ... & 0 & 0  \\[3mm]
\vdots & \vdots & \vdots & \ddots & \vdots & \vdots \\[3mm]
0 & 0 & 0 & ... & 1 & 0
\end{array}\right] ,
\end{equation}

	\noindent and considering the steady-state approximation $X_a(t)$ of $X(t)$ is the first component of $\mathbf{Y}_a$, this yields

\begin{equation} X_a = \dfrac{\beta^n_n}{\beta^n_0}\dfrac{F(t)}{\beta^n_n} = \dfrac{F(t)}{\beta_0^n} . \label{eq:qsh_approx_slowsystemresult}\end{equation}

	By the definition of the $\beta$ coefficients,

\begin{align} \beta_0^n(t) &= \sum\limits_{k=0}^{n} \alpha_k{k\choose 0} \left[\sum\limits_{c=0}^{k} j^cB_{\left(k,c\right)}\left(\omega,\dot{\omega},\ddot{\omega},...,\omega^{(k-c)}\right) \right] = \\[3mm] &= \sum\limits_{k=0}^{n} \alpha_k \left[\sum\limits_{c=0}^{k} j^cB_{\left(k,c\right)}\left(\omega,\dot{\omega},\ddot{\omega},...,\omega^{(k-c)}\right) \right]. \label{eq:beta0_approx_tvar}\end{align}

	Naturally, if the apparent frequency is a constant $\omega_0$ then by the properties of the Bell Polynomials,

\begin{equation} B_{\left(k-i,c\right)}\left(\omega,0,...0\right) = \left\{\begin{array}{l} \omega^{k} \text{, if } k-i=c \\[2mm] 0 \text{, if otherwise} \end{array}\right. \end{equation}

	\noindent and substituting onto \eqref{eq:beta0_approx_tvar},

\begin{equation} \beta_0^n(t) = \sum\limits_{k=0}^{n} \alpha_k{k\choose 0} \left[\sum\limits_{c=0}^{k} j^cB_{\left(k,c\right)}\left(\omega,0,...,0\right)\right] = \sum\limits_{k=0}^{n} \alpha_k \left(j^k\omega_0^k\right),  \end{equation}

	\noindent and substituting this into \eqref{eq:beta0_approx_tvar},

\begin{equation} X_a = \dfrac{F(t)}{\displaystyle \sum\limits_{k=i}^{n} \alpha_k \left( j\omega_0\right)^k} .\end{equation}

	Thus showing that for a constant apparent frequency the approximated steady-state equations are the classical phasor algebraic equations. For non-constant apparent frequencies, let us suppose that $\omega(t)$ is such that its derivatives are all sufficiently small, that is,

\begin{equation} \left\lvert\dfrac{d^k\omega(t)}{dt^k}\right\rvert \leq \varepsilon(t) \text{ for some small } \varepsilon(t) \text{ and } 1\leq k \leq n. \end{equation}

	We know that polynomials are infinitely smooth with respect to the inputs, so

\begin{equation} \lim_{x_2,x_3,...,x_{\left(k-c+1\right)}\to 0} B_{\left(n,k\right)}\left(x_1,x_2,...,x_{\left(k-c+1\right)}\right) = B_{\left(n,k\right)}\left(x_1,0,...,0\right)\end{equation}

	\noindent meaning

\begin{equation} B_{\left(n,k\right)}\left(x_1,x_2,...,x_{\left(k-c+1\right)}\right) = B_{\left(k,c\right)}\left(x_1,0,...,0\right) + \sum_{i=2}^{k-c+1} O\left(x_i\right) = x_1^n + O\left(\varepsilon(t)\right)\end{equation}

	\noindent in turn meaning

\begin{equation} \beta_0^n(t) = \sum\limits_{k=0}^{n} \alpha_k \left(j\omega\right)^k + O\left(\varepsilon(t)\right) \Rightarrow X_a = \dfrac{F(t)}{\displaystyle \sum\limits_{k=i}^{n} \alpha_k \left( j\omega\left(t\right) \right)^{k}} + O\left(\varepsilon(t)\right). \label{eq:qsh_approx_composition}\end{equation}

	One can immediately notice that this result is a time-varying adaptation of the algebraic equation one would obtain if the excitation $f(t)$ were a static sinusoid with fixed frequency — hence why this solution is sometimes called ``algebraic solution''. One can also note that these results can be obtained by applying null derivatives of $X$ and $\omega$ on \eqref{eq:qsh_approx_line}, which also corroborates with the notion that the Dynamic Phasor differential equation \eqref{eq:qsh_approx_line} is approximated by a static phasor equivalent version once the Quasi-Static Modelling is applied. Particularly, if $\omega(t)$ is still time varying but slow and close to some constant $\omega_0$, that is, 

\begin{equation} \omega(t) = \omega_0 + \Delta\omega(t) \text{ where } \left\lvert\dfrac{d^k\Delta\omega(t)}{dt^k}\right\rvert \leq \varepsilon(t) \text{ for some small } \varepsilon(t) \text{ and } 1\leq k \leq n. \label{eq:close_and_slow}\end{equation}

	\noindent then

\begin{equation} X_a = \dfrac{F(t)}{\displaystyle \sum\limits_{k=i}^{n} \alpha_k \left( j\omega_0\right)^k} + O\left(\varepsilon(t)\right)\end{equation}

	\noindent and the equation becomes algebraic and the denominator becomes a static impedance quantity. This equation also means that if the frequency $\omega(t)$ assymptotically stabilizes to a certain value, that is, the limit

\begin{equation} \lim_{t\to\infty} \omega(t) = \omega_\infty \end{equation}

	\noindent exists then $\varepsilon(t)\to 0$ and at the equilibrium

	\begin{equation} X_\infty = \dfrac{F_\infty}{\displaystyle \sum\limits_{k=i}^{n} \alpha_k \left( j\omega_\infty\right)^k}\end{equation}

	\noindent where $F(t)\to F_\infty$ as $t\to\infty$, provided $F_\infty$ exists. This essentially means that the assymptotic response of the circuit is given by a classic phasor relationship, therefore allowing us to calculate the initial and final conditions of \eqref{eq:lemma_time_complex_final} using algebraic relationships.

	Thus, from a linear circuits perspective, these results in essence validade the Quasi-Static Hypothesis: as $\omega$ is supposed much slower than the circuit dynamics, $X_a$ approximates its steady-state algebraic behavior and the model becomes much close to the static phasor models using classic impedances.

	Another reason to call this solution ``algebraic'' is the fact that the impedances become algebraic equations. Applying the results to the differential equation $\dot{x} = y(t)$ one obtains $Y_a(t) = j\omega(t)X_a(t)$; thus the linear circuit bipole equations become

\begin{equation} \left\{\begin{array}{l} \text{Linear inductor: } v(t) = L \dot{i}(t) \Rightarrow V_a(t) = j\omega(t) L I_a(t) \\[3mm] \text{Linear capacitor: } i(t) = C \dot{v}(t) \Rightarrow I_a(t) = j\omega(t) C V_a(t) \\[3mm] \text{Linear resistor: } v(t) = Ri(t) \Rightarrow V(t) = RI(t) \end{array}\right. \label{eq:Linear_approximated_relationshipsp} .\end{equation}

	\noindent which are algebraic since the differential portion is dropped. Particularly interesting for Power Systems, if $\omega(t)$ is a constant synchronous frequency $\omega_0$ (or sufficiently close to it with small derivatives as in \eqref{eq:close_and_slow}) then

\begin{equation} \left\{\begin{array}{l} \text{Linear inductor: } V_a(t) = j\omega_0L I_a(t) = j x_L I_a(t) \\[3mm] \text{Linear capacitor: } I_a(t) = j\omega_0 C V_a(t) = j x_C V_a(t) \\[3mm] \text{Linear resistor: } V(t) = RI(t) \end{array}\right. \label{eq:Linear_approximated_relationships_synch} \end{equation}

	\noindent where $x_L$ and $x_C$ are the inductive and capacitive reactances measured at the synchronous frequency. This justifies many results in stability and control of Power Systems; for instance, in example \ref{example:3p_eps_modelling}, it was mentioned that the current controller of figure \ref{fig:3p_curr_control} has a flaw in that it assumes time-varying equivalent impedance equations in the form of \eqref{eq:circuit_3pmodel_complex}, leading to potentially bad controller behavior. In that particular example, by adopting small values of gains for the PI controller of the PLL synchronization subsystem, the swings in frequency $\omega$ are slow and small (as evidenced by the simulation results of $\omega(t)$ in figure \ref{fig:freqsignal_3psim}; thus, in this case, the steady-state modelling is justified.


\part{Dynamic Phasor Functionals and Control}\label{part:applications}

% ---------------------------------------------------------
\chapter{Dynamic Phasor Functionals}\label{chapter:dpos}
% ---------------------------------------------------------

	Seen as Passive Linear Circuits define linear differential systems, one of the main aspects of Linear Electrical Circuit Theory is the employment of mathematical tools to solve the Differential Equations that model electrical circuits, following a sequence that progresses in complexity as the input signals considered get more sophisticated. Initially, a circuit network is presented as excited by a sinusoidal signal, and the Classical Phasor approach was shown to sufficient to model circuit networks in steady-state regimen. Then, instead of a static sinewave, a non sinusoidal but still periodic excitation is used; the signal can be decomposed into a set of harmonics by its Fourier Series, and due to the orthogonality of each harmonic, the circuit can be separated into one individual circuit for each harmonic, and the final signal is obtained from the summation of each response of each individual circuit. Further, if the the excitation is neither sinusoidal nor periodic, but still being absolutely integrable — ``stable'' in a certain sense —, the Fourier Transform is used to decompose the input signal as a set of continuous frequency bands. Finally, if the input is non-sinusoidal, aperiodic and possibly unstable, the Laplace Transform (LT) is presented as a generalized case of the Fourier Transform where each harmonic is also decomposed into varying amplitudes, and the combination of continuous frequencies $\omega$ and continuous amplitudes $\sigma$ generates the Laplace frequency variable $s = \sigma + j\omega$. A comprehensive discussion of these tools and the escalable complexity of the excitation signals is found in \cite{scottElementsLinearCircuits1965,desoerBasicCircuitTheory1987}.

	At each step it can be shown that instead of modelling the target circuit using equations of time, leading to time ODEs — which need special procedures and techniques to be solved — the circuit can be modelled in the ``frequency'' or ``complex'' domain, leading to algebraic equations that are much simpler to solve, and the complex functions obtained as solutions of the algebrai equations are proven to be direct representations of the time functions that solve the original ODEs of the circuit. To further refine this process, each tool at each step in the escalation process is imbued with a version of the three main established circuit modelling techniques: Kirchoff's Laws (KLs), the Superposition Principle or Theorem (ST) and the Thèvenin-Norton-Theorems (TNTs).

	Chapter \ref{chapter:dynamic_phasor_theory} of this thesis takes a different approach to this sequence of tools: instead of traditionally escalating Classic or Static Phasors to integral transforms (Fourier and Laplace), Classic Phasors are expanded by using a particular transformation that was called the Dynamic Phasor Transform, which essentially consists of a particular differential operator to represent generalized sinusoidal signals as complex functions called Dynamic Phasors. While certainly powerful, the DPT is quite strenuous to work with, and its operationalization, that is, the process of using it for the specific modelling and equationing, becomes quite effortful. In this chapter, we devise a particular set of transformations in Dynamic Phasor space, which will be called the Dynamic Phasor Functionals, that aim to offer the same algebraic properties that the integral transforms enjoy, and also offer Dynamic Phasor equivalent proofs of the circuit modelling techniques mentioned.

	More specifically, we devise a sequence of functionals $\left\{\ndpo{k}\right\}_{(k\in\mathbb{Z})}$ such that $y(t) = x^{(k)}(t)\Leftrightarrow Y(t) = \ndpo{k}\left[X\right]$, that is, the k-th order differentiation in time is equivalent to the k-th order operator in Dynamic Phasor space. We further prove that these operators form very powerful algebraic structures, which allows for extensive algebraic manipulations and properties. These structures are explored to prove that not only DPFs keep very desirable modelling features like linearity, multiplication and linear combination, but also that impedances and admittances can be defined in the domain of Dynamic Phasors, and versions of Kirchoff's Laws, the Superposition Theorem and the Thèvenin-Norton Theorems are proven for this Dynamic Phasor framework. This, in turn, allows representing and modelling linear circuits under nonstationary regimens in a much clearer and intuitive way than the conventional tools like the Laplace Transform, while keeping intact the phasorial quantities of amplitudes, phases and frequencies.

%-------------------------------------------------
\section{Motivation: modelling circuit using the Laplace Transform}

	The Laplace Transform of a signal $x(t)$ is defined as

\begin{equation} X(s) = \mathbf{L}\left[x\right] = \int_{\mathbb{R}} x(t)e^{-st}dt . \label{eq:laplace_def}\end{equation}

	In the realm of linear systems, the most useful feature of this transform is the capability to algebraically represent derivatives and integrals in the complex domain, as in \eqref{sys:laplace_properties}, giving the LT a remarkable capacity to streamline the solutions of linear time ODEs.

\begin{equation}\left\{\begin{array}{l} \mathbf{L}\left[\dfrac{dx}{dt}\right] = s\mathbf{L}\left[x\right] - x(0) \stackrel{\raisebox{-2mm}{} x(0)=0}{=} s\mathbf{L}\left[x\right]\\[5mm] \mathbf{L}\left[\displaystyle \int_0^t x(a)da \right] = \dfrac{1}{s}\mathbf{L}\left[x\right]\end{array}\right. \label{sys:laplace_properties} \end{equation}
	
	Given a square integrable signal $x(t)$, then its Laplace Transform is smooth and analytic where it is defined, which can be proven using the Dominated Convergence Theorem and Morera's Theorem \pcite{ahlfors1979complex}. This is extensively explored in linear control theory \pcite{chenLinearSystemTheory2013}. Particularly for this field, the Laplace Transform is very useful because it generalizes the notion premiered by Classical Phasors of a transform that translates derivatives in time to algebraic operations in complex space:

\begin{equation} \sum_{k=0}^n \alpha_k x^{(k)} - y(t) = 0 \Leftrightarrow \sum_{k=0}^n \alpha_k s^k X(s) - Y(s) = 0 \Leftrightarrow X(s) = \dfrac{Y(s)}{\displaystyle\sum_{k=0}^n \alpha_k s^k} \end{equation}

	\noindent and, from this equivalence, many useful properties can be drawn and explored. For the Linear Circuits standpoint, the current-voltage differential equations of passive bipoles are transformed into algebraic equations, as shown in \eqref{sys:laplace_impedances}, defining the concepts of Laplace Impedances.

\begin{equation}\left\{\begin{array}{l} v(t) = L\dot{i}(t) \Leftrightarrow V(s) = sLI(s) \text{ (Linear inductor)}\\[3mm] i(t) = C\dot{v}(t) \Leftrightarrow I(s) = sCV(s) \text{ (Linear capacitor)}\\[3mm] v(t) = Ri(t) \Leftrightarrow V(s) = RI(s) \text{ (Linear resistor)}\end{array} \right. ,\label{sys:laplace_impedances}\end{equation}

	\noindent where $V(s)$ and $I(s)$ are the Laplace Transform of the time voltage signal $v(t)$ and current $i(t)$. The complex functions obtained as solutions of the algebraic complex equations are guaranteed to be direct representations of the solutions of the time differential equations of the circuit, which can be retrieved using the inverse Laplace Transform:

\begin{equation} \mathbf{L}^{-1}\left[X(s)\right] = \dfrac{1}{2\pi j} \int_{B_\alpha} X(s)e^{st}ds \label{eq:inverse_laplace_def}\end{equation}

	\noindent where $B_\alpha = \left(\alpha - j\infty,\alpha + j\infty\right)$ is a Brömwich contour, $\alpha$ is at the right of all the poles of $X(s)$. Owing to these propertes, the Laplace Trasform is seen as a be-all-end-all tool that is able to represent any signal in time and solve any linear time ODE algebraically. However, only a very limited catalog of functions have ``convenient'' or ``nice'' (that is, analytically representable) transforms, as well as inverse transforms. As a result, transforming signals to their equivalent complex frequency representations, operating the complex functions, and then going back to time signals requires the functions involved to be in this roster of ``simple'' transforms, meaning the transform is only \textit{operationalizable} in a sense, and ultimately applicable, for a limited set of functions. For an arbitrary signal $x(t)$, even if a transform $X(s)$ exists, it is often too complicated or impossible to be written as a combination of elementary functions.

	Take for instance the RLC circuit of examples \ref{example:rlc_dpt} and \ref{example:dpdomain_secondorder}. In those examples, one must go significant lengths to finally find a differential equations that models the load voltage $V_R$ with respect to the input voltage $V(t)$. In contrast, the same circuit can be modelled with the Laplace Transform, and one can simply use the voltage-current relationships \eqref{sys:laplace_impedances}. The inductor $L$ is substituted by an impedance $sL$, the capacitor by an impedance $1/sC$, and the circuit becomes that of figure \ref{fig:laplace_example}, where one notices that $V_R$ is given by an impedance $sL$ in series with an impedance that is the parallel combination of $R$ and $1/sC$; therefore $V_R(s)$ is simply obtained using an impedance divider formula:

\begin{equation} V_R(s) = V(s) \left( \dfrac{\raisebox{-8mm}{} \dfrac{1}{\raisebox{5mm}{} \dfrac{1}{R} + sC}}{\raisebox{5mm}{} sL + \dfrac{1}{\raisebox{5mm}{} \dfrac{1}{R} + sC}}\right) = V(s)\left( \dfrac{1}{\raisebox{5mm}{} s^2LC + s\dfrac{L}{R} + 1}\right) \label{eq:rlc_laplace_model}\end{equation}

	\noindent showcasing the remarkable operational properties of the Laplace Transform applied to electrical circuits; such properties are not yet available for Dynamic Phasors. At this stage, one calculates $V(s)$ as the Laplace Transform of the excitation $v(t)$, thus obtaining $V_R(s)$, and then uses the inverse LT to obtain $v_R(t)$ from $V_R(s)$. However, one immediately notices that if $v(t)$ is the generalized sinusoid \eqref{eq:example_voltage_freq_def} used in the examples, it does not have an algebraically representable $V(s)$, frustrating the process and outlining the first major advantage of the Dynamic Phasors proposed: the theory proposed allows representing signals like $v(t)$ in a simple manner.

	Further, even if $V_R(s)$ and $v_R(t)$ can be obtained (say, numerically) the resulting $V_R(s)$ loses the notions of a time-varying amplitude and phase, also outlining the fact that the proposed DP theory allows for such notions with a solid correspondence with time signals.

% MODELLING EXAMPLE: RLC CIRCUIT IN LAPLACE DOMAIN<<<
\begin{figure}[h]
\centering
        \begin{tikzpicture}[american,scale=1,transform shape,line width=0.75, cute inductors, voltage shift = 1]
	\ctikzset{/tikz/circuitikz/voltage/distance from node=10mm}
		\draw (0,0)
			to[vsource,sources/scale=1.25, v>=$V(s)$,invert] (0,4)
			to[L,l=$sL$,f>^=$I_{L}(s)$,v>=$V_{L}(s)$,-*] (4,4) 
			to[C,l=$\dfrac{1}{sC}$,f>^=$I_{C}(s)$,v>=$V_{C}(s)$,-*] (4,0) 
			to[short] (0,0); 
		\draw (4,4)
			to[short,f>^=$I_{R}(s)$] (7,4) 
			to[R,l=$R$,v>=$V_{R}(s)$] (7,0) 
			to[short]  (4,0);
        \end{tikzpicture}
	\caption{Second-order circuit for example application of the Laplace Transform.}
	\label{fig:laplace_example}
\end{figure} %>>>
	
	It becomes clear that one class of the problematic signals that do not have ``nice'' Laplace transforms is that of generalized sinusoids, as defined in this thesis: a generic sinusoid does not have an operationalizable Laplace Transform. As a matter of fact this transform is possibly nonexistant in cases where the system exhibits explosive behavior and the signals involved are unstable. This highlights an inconformity of the available toolset with respect to this class of signals and generating lack of a solid and practical theory to represent circuits and systems under nonstationary sinusoidal regimens. 

	As such, the problem at hand manifests itself as a predicament that one the one hand the Dynamic Phasor Theory proposed can translate generalized sinusoids in phasorial domain, but it lacks the operational properties of the Laplace Transform; on the other, while the Laplace Transform attains such operational features, it lacks a phasorial analytical representation for generalized sinusoids.

	Driven by this predicament, in this chapter we study the possibility of defining specific operations in the complex Dynamic Phasor space so that differentiations in time become algebraic manipulations in DP space, in doing so solving the predicament proposed: by using the DPT one can use Dynamic Phasors, and by using such proposed operations, one can model those Dynamic Phasors algebraically.

% ------------------------------------------------
\section{The Dynamic Phasor Functionals} \label{sec:operator} %<<<1

%-------------------------------------------------
\subsection{Motivation: transforming derivatives} \label{subsec:motivation_dpos} %<<<2

	Inasmuch as chapter \ref{chapter:dynamic_phasor_theory} illustrates the validity of the Dynamic Phasor approach proposed in this thesis and its usefulness compared to an established tool like the Laplace Transform, its application is strenuous because transforming the time DEs \eqref{eq:rlc_time_diffeq} into a complex equivalent DE \eqref{eq:rlc_complex_diffeq} still requires calculating the time DEs to then apply the transform. In contrast, the Laplace Transform allows directly modelling a circuit in the $s$ frequency domain by simply using simple circuit modelling techniques, as shown in the fact \eqref{eq:rlc_laplace_model} is obtained in a single line of calculations. Fundamentally, this stems from the fact that the LT transforms derivatives into algebraic operations which evolve into proving circuit techniques like the impedance divider formulas used. In the Dynamic Phasor framework, however, derivatives are transformed into a very specific transformation:

\begin{equation} y_1(t) = \dot{x}(t) \Leftrightarrow Y_1(t) = \dot{X}(t) + j\omega(t) X(t) .\label{eq:first_operator}\end{equation}

	For the second derivative, 

\begin{align}
	y_2(t) &= \ddot{x}(t) = \dot{y}_1 \Leftrightarrow Y_2(t) = \dot{Y}_1(t) + j\omega(t) Y_1(t) = \nonumber\\[3mm]
	&= \dfrac{d}{dt}\left[\dot{X}(t) + j\omega(t) X(t)\right] + j\omega(t)\left[\dot{X}(t) + j\omega(t) X(t)\right] = \nonumber\\[3mm]
	&= \ddot{X}(t) + 2j \omega(t) \dot{X}(t) + \left[ j \dot{\omega}(t) - \omega(t)^2 \right] X(t) . \label{eq:second_operator}
\end{align}

	Finally, for the third,

\begin{align}
	y_3(t) &= \dddot{x}(t) = \dot{y}_2 \Leftrightarrow Y_3(t) = \dot{Y}_2(t) + j\omega(t) Y_2(t) = \nonumber\\[3mm]
	&= \dddot{X}(t) + 3j \omega(t) \ddot{X}(t) + \left[ 3j \dot{\omega}(t) - 3 \omega(t)^2 \right] \dot{X}(t) + \left[ j \ddot{\omega}(t) - 3 \omega(t) \dot{\omega}(t) - j \omega(t)^3 \right] X(t) \label{eq:third_operator}
\end{align}

	And it becomes clear that the formulas explode in size and become quite complicated as the order $y_n$ grows; as such, using this algorithm for large-scale systems will lead to quite a painful process. Fortunately, theorem \ref{corollary:complex_equivalence_phasorialodes} gives a closed formula for the n-th order relationship.

\begin{theorem}[n-th order Dynamic Phasor Functional]\label{theo:nth_order_relationship} %<<<
	Let $x(t)$ a nonstationary sinusoidal and $n\in \mathbb{N}$. Consider $\omega(t)\in C^n$ an apparent frequency signal and let $y(t) = x^{(n)}(t)$. Then $y_n(t)$ is a nonstationary sinusoid and its Dynamic Phasor $Y(t)$ is given by

\begin{equation}\left\{\begin{array}{l} \displaystyle Y_n(t) = \sum\limits_{k=0}^n \gamma_k^n(t) X^{(k)}(t) \\ \displaystyle \gamma_k^n(t) = {n\choose k} \left[\sum\limits_{c=0}^{n-k} j^cB_{\left(n-k,c\right)}\left(\omega,\dot{\omega},\ddot{\omega},...,\omega^{(n-k-c)}\right) \right]\end{array}\right. ,\label{eq:gamma_def}\end{equation}

	\noindent with $X(t)$ and $Y_n(t)$ the dynamical phasors of $x(t)$ and $y_n(t)$.
\end{theorem}
\textbf{Proof:} apply theorem \ref{corollary:complex_equivalence_phasorialodes} and adopt $\alpha_n = 1$ and $\alpha_k = 0$ for $0 \leq k < n$. \hfill$\blacksquare$\vspace{5mm}\hrule\vspace{5mm} %>>>

	Thus, let us define a functional transformation in the complex space, such that a derivative in the time domain is represented by a first order functional map $\ndpo{1}$, that is,

\begin{equation} \ndpo{1}_\omega \left[X\right] = \left[\mathbf{D}^1 + j\omega(t)\mathbf{I}\right]\left[X\right] , \label{eq:steinmetz_1storder}\end{equation}

	\noindent with $\mathbf{I}$ the identity map. Let us call this the \textbf{first-order Dynamic Phasor Functional}. For the second order functional, from \eqref{eq:second_operator}, $\ndpo{2}$ could be defined as

\begin{equation} \ndpo{2}_\omega \left[X\right](t) = \left\{\raisebox{4mm}{} \mathbf{D}^2 + 2j\omega(t)\mathbf{D}^1 + \left[-\omega^2 + j\dot{\omega}(t)\right]\mathbf{I}\right\}\left[X\right] \label{eq:steinmetz_2ndorder}\end{equation}

	\noindent and from \eqref{eq:third_operator}, the third-order functional $\ndpo{3}$ would be defined as

\small
\begin{equation} \ndpo{3}_\omega \left[X\right](t) = \left\{\raisebox{4mm}{} \mathbf{D}^3 + 3j \omega(t) \mathbf{D}^2 + \left[ 3j \dot{\omega}(t) - 3 \omega(t)^2 \right] \mathbf{D}^1 + \left[ j \ddot{\omega}(t) - 3 \omega(t) \dot{\omega}(t) - j \omega(t)^3 \right]\mathbf{I}\right\}\left[X\right] \label{eq:steinmetz_3rdorder}\end{equation}
\normalsize

	Therefore, theorem \ref{theo:nth_order_relationship} induces a naïve definition of a n-th order Dynamic Phasor Functional $\ndpo{n}$ of the form

\begin{equation} \ndpo{n}_\omega = \sum_{k=0}^n \gamma_k^n \left(t\right)\mathbf{D}^k_\mathbb{C} \label{def:operated_dynamic_phasor}, \end{equation}

	\noindent where $\mathbf{D}_\mathbb{C}^k$ is the k-th order differential functional in the space $\left[\mathbb{R}\to\mathbb{C}\right]$ and the $\gamma_k^n$ are defined in \eqref{eq:gamma_def}. Naturally, because the 0-th order derivative is the identity, it is natural to define the 0-th order functional as the identity $\dpo^0_\omega = \mathbf{I}$. This result is also a consequence of theorem \ref{theo:nth_order_relationship}: using $n=0$ one arrives at $\gamma_0^0 = 1$.

	Formally, $\ndpo{n}_\omega$ is part of a larger class of functionals called \textbf{Differential Operators} \pcite{achiezerTheoryLinearOperators1993}, making these functionals a particular set of such differential operators in the space of complex signals.

	Naturally, the DPF depends on the apparent frequency signal $\omega(t)$ chosen, hence the subscript. Since this signal is chosen beforehand, and generally tacitly understood, this subscript will be dropped hereforth, always having such dependence in mind.

%-------------------------------------------------
\subsection{Linearity, bijectiveness and inverse operator} %<<<2

	Most importantly, we want to prove that the DPFs are linear, which is the most basic property needed from such an operator.

\begin{theorem}[DPFs are linear]\label{theo:dpf_linearity}%<<<
	The n-th order DPF $\ndpo{n}$ is linear.
\end{theorem}
\textbf{Proof:} let $X(t),Y(t)\in\left[\mathbb{R}\to\mathbb{C}\right]$ and $\alpha\in\mathbb{C}$. Then from the definition \ref{theo:nth_order_relationship}, 

\begin{align}
	\ndpo{n}\left[X + \alpha Y\right] &= \sum\limits_{k=0}^n \gamma_k^n(t) \left[X + \alpha Y\right]^{(k)}(t) = \sum\limits_{k=0}^n \gamma_k^n(t) \left[X^{(k)} + \alpha Y^{(k)}(t)\right] \nonumber\\[3mm]
	&= \sum\limits_{k=0}^n \gamma_k^n(t) X^{(k)} + \sum\limits_{k=0}^n \alpha Y^{(k)}(t) = \sum\limits_{k=0}^n \gamma_k^n(t) X^{(k)} + \alpha\sum\limits_{k=0}^n Y^{(k)}(t)  \nonumber\\[3mm]
	&= \ndpo{n}\left[X\right] + \alpha \ndpo{n}\left[Y\right]
\end{align}

\hfill$\blacksquare$
\vspace{5mm}
\hrule
\vspace{5mm}%>>>

	Further, it is natural that once a transform $\ndpo{n}$ is defined, allowing obtaining $Y(t)$ from $X(t)$, an inverse transform $\ndpo{\left(-n\right)}$ is needed in order to reconstruct $X(t)$ from $Y(t)$. Let $x(t)$ be a nonstationary signal and $y(t)$ such that $x(t) = y^{(n)}(t)$, prompting the definition $Y(t) = \ndpo{\left(-n\right)} \left[X\right]$ or

\begin{equation} X(t) = \sum_{k=0}^n \gamma_k^n (t) Y^{(k)}(t). \label{eq:inverse_steinmetz_def} \end{equation}

	However, in order for the inverse operator to be solid, it needs to be proven that the transform itself is bijective, that is, $Y(t) = \ndpo{n} \left[X\right]$ is unique to $X(t)$, and vice-versa.

\begin{theorem}[Bijectiveness of $\ndpo{n}$] \label{theo:bijection} %<<<
	Let $X(t)$ be the dynamic phasor of some given signal $x(t)$ at an apparent frequency $\omega(t)\in C^n\left(I\right)$ for $n\geq 1$ and some non-empty $I\subset\mathbb{R}$.

\begin{itemize}
	\item If $X(t)\in C^n\left(I\right)$, then $Y(t) = \ndpo{n} \left[X\right]$ exists and is unique in $I$, that is, $\ndpo{n}$ is injective; and
	\item If $X(t)\in C^1\left(I\right)$, then given an initial condition for itself and its $n-1$ derivatives, there exists a unique signal $Y(t)$ in $I$ such that $X(t) = \ndpo{n} \left[Y\right]$, that is, $\ndpo{n}$ is surjective.
\end{itemize}

\end{theorem}
\noindent\textbf{Proof:} suppose two $Y_1(t)$ and $Y_2(t)$ qualify as $\ndpo{n} \left[X\right]$. Then 

\begin{equation} Y_1(t) - Y_2(t) = \sum_{k=0}^n \gamma_k^n (t) \dfrac{d^kX(t)}{dt^k} - \sum_{k=0}^n \gamma_k^n (t) \dfrac{d^kX(t)}{dt^k} = 0\end{equation}

	\noindent meaning $Y_1 = Y_2$ and $\ndpo{n}$ is injective. For surjection, suppose $X(t) = \ndpo{n} \left[Y\right]$. Then $X(t)$ satisfies \eqref{eq:inverse_steinmetz_def}. Simple inspection yields $\gamma_n^n(t) = 1$ for any $n \geq 0$; therefore \eqref{eq:inverse_steinmetz_def} can be separated into real and imaginary parts. Let $X_R,X_I,Y_R,Y_I$ be the real and imaginary parts of $X(t)$ and $Y(t)$:

\begin{equation} \left\{\begin{array}{l} \displaystyle X_R(t) = Y_R^{(n)}(t) + \sum_{k=0}^{n-1} a_n^k\left(t\right)Y_R^{(k)} + \sum_{k=0}^{n-1} b_n^k\left(t\right)Y_I^{(k)} \\[5mm] \displaystyle X_I(t) = Y_I^{(n)}(t) + \sum_{k=0}^{n-1} c_n^k\left(t\right)Y_R^{(k)} + \sum_{k=0}^{n-1} d_n^k\left(t\right)Y_I^{(k)} \end{array}\right. \label{eq:theo_bijection_reimsys} \end{equation}

	\noindent where the $a,b,c,d$ are linear combinations of the $\gamma$ functions. Now construct

\begin{equation} \mathbf{y}(t) = \left[ Y_R,\dot{Y}_R,...,Y^{(n-1)}_R,Y_I,\dot{Y}_I,...,Y^{(n-1)}_I\right]^\intercal \label{eq:theo_bijection_ydef}\end{equation}

	Then by \eqref{eq:theo_bijection_reimsys} $Y^{(n)}_R$ and $Y^{(n)}_I$ can be written as

\begin{equation} \left\{\begin{array}{l} Y_R^{(n)}(t) = \displaystyle X_R(t) - \sum_{k=0}^{n-1} a_n^k\left(t\right)Y_R^{(k)} - \sum_{k=0}^{n-1} b_n^k\left(t\right)Y_I^{(k)} = g_R\left(t,X_R(t),\mathbf{y}(t)\right) \\[5mm] Y_I^{(n)}(t) = \displaystyle X_I(t) - \sum_{k=0}^{n-1} c_n^k\left(t\right)Y_R^{(k)} - \sum_{k=0}^{n-1} d_n^k\left(t\right)Y_I^{(k)} = g_I\left(t,X_I(t),\mathbf{y}(t)\right) \end{array}\right. . \end{equation}

	Then, from \eqref{eq:theo_bijection_ydef},

\begin{align} \mathbf{\dot{y}}(t) &= \left[ y_2(t),y_3(t),...,y_{\left(n-1\right)}(t),g_R\left(t,X_R(t),\mathbf{y}(t)\right), y_{\left(n+1\right)}(t),...,y_{(2n-2)}(t),g_I\left(t,X_I(t),\mathbf{y}(t)\right)\right]^\intercal = \nonumber\\& = f\left(t,\mathbf{y}(t)\right) \end{align}

	Because the $\gamma^n_k(t)$ are linear combinations of the $n$ derivatives of $\omega$, they are $C^1\left(I\right)$; the coefficients $a_k^n,b_k^n,c_k^n,d_k^n$ of \eqref{eq:theo_bijection_reimsys} are compositions of the $\gamma_k^n$ therefore they are also $C^1\left(I\right)$. Since the $g_R$ and $g_I$ are functions of these same coefficients and $X(t)$, these two functions are at least $C^1\left(I\right)$ (because $X$ is defined as $C^1\left(I\right)$), therefore $f$ is at least $C^1\left(I\right)$. Thus according to the Picard-Lindelöf Existence and Uniqueness Theorem \pcite{Perko2001}, given an initial condition $\mathbf{y}_0$ at $t_0$ there is a unique vector $\mathbf{y}$ in $I$ satisfying the IVP, meaning there exists unique $Y_R(t)$ and $Y_I(t)$ in $I$, therefore an unique $Y(t)$. \hfill$\blacksquare$\vspace{5mm}\hrule\vspace{5mm} %>>>

	Theorem \ref{theo:bijection} proves that $\ndpo{n}$ is bijective, therefore an inverse transform $\left(\ndpo{n}\right)^{-1}$ is possible and can be defined.

\begin{corollary}[Inverse Dynamic Phasor Functionals]\label{def:inverse} %<<<
	Given a $X(t)\in\left[\mathbb{R}\to\mathbb{C}\right]$, then $Y(t) = \left(\ndpo{n}\right)^{-1}\left[X\right]$ is defined as the signal that satisfies

\begin{equation} X(t) = \ndpo{n}\left[Y\right] = \sum_{k=0}^n \gamma_k^n (t) Y^{(k)}(t), \end{equation}

	\noindent which by theorem \ref{theo:bijection} exists and unique given the initial conditions $Y(0),Y'(0),\cdots ,Y^{(n-1)}(0)$.
\end{corollary}%>>>

 The need for initial conditions, necessary for the surjection proof, is not a foreign concept. For instance, the Laplace transform for the n-th derivative of a signal is
	
\begin{equation} \mathbf{L}\left[y^{(n)}\right] = s^n \mathbf{L}\left[y\right] - s^{(n-1)}y_0 - s^{(n-2)}y'_0 - \cdots - y^{(n-1)}_0 \label{eq:laplace_nthder_complete} \end{equation}

	\noindent thus $y_0, y'_0, y''_0,...,y^{(n-1)}_0$ must be known. Customarily however \eqref{eq:laplace_nthder_complete} is presented as $\mathbf{L}\left[y^{(n)}\right] = s^n \mathbf{L}\left[y\right]$ which assumes the system starts from a zero-energy state where the initial values of $y$ and its $n-1$ derivatives are null.

%-------------------------------------------------
\section{Algebraic structures induced by DPFs and the class $\dpS$} \label{subsec:notation_abuse} %<<<2
	
	It is immediate from the definition of $\dpo$ that the linear element equations yield

\begin{equation}\left\{\begin{array}{l} v(t) = L\dot{i}(t) \Leftrightarrow V(t) = L \ndpo{1} \left[I\right] \text{ (Linear inductor)}\\[3mm] i(t) = C\dot{v}(t) \Leftrightarrow I(t) = C \ndpo{1} \left[V\right]  \text{ (Linear capacitor)}\\[3mm] v(t) = Ri(t) \Leftrightarrow V(t) = RI(t) \text{ (Linear resistor)}\end{array} \right. \label{sys:dpo_impedances_first}\end{equation}

	\noindent and this is naturally highly resemblant of impedances in the Laplace domain, if it were not for the fact that $\dpo$ are functionals but not a complex number like the Laplace frequency $s$.

	We now dive into the operational properties of the $\ndpo{k}$ and how these properties induce algebraic structures that are able to translate differential equations in time domain to algebraic equations in the DPF space. Specifically, it will be proven that the notions of sums and products of operators is definable and that the n-th order operator $\ndpo{n}$ is an ``n-th power'' of the first-order operator $\ndpo{1}$; further, the operators can be inverted and linearly combined, so that polynomials of operators are also defineable. For the ``impedance equations'' \eqref{sys:dpo_impedances_first} these properties mean that the functional $\dpo$ behaves algebraically just like the complex frequency $s$, so that a notion of impedances in Dynamic Phasor space is well-defined as ratios of polynomials of $\dpo$ much like impedances in the Laplace domain are ratios of polynomials of $s$.

	The mathematical background for these proofs is abstract algebra; the main literature used is \cite{goncalvesIntroducaoAlgebra2021,garciaElementosAlgebra2022,hungerfordAlgebra2010,dummitAbstractAlgebra2003}). We first prove that the space of functionals $\ndpo{n}$ forms algebraic structures of being a group, a ring, a field and a vector space.

%-------------------------------------------------
\subsection{Group and ring}\label{subsec:group_ring}

	We first prove that the DPFs form a \textit{group}, through the sum of operators.

\begin{definition}[Group]\label{def:algebra_group}%<<<
	A \textbf{group} is a set $G$ equipped with a binary operation, denoted ``$+$'', such that any two elements $a,b\in G$ combined lead to $c = a + b$ where $c\in G$. Further, the operation fulfills the axioms:

\begin{itemize}
	\item \textbf{Associativity:} for any $a,b,c\in G$, $\left(a + b\right) + c = a + \left( b + c\right)$;
	\item \textbf{Neutral element:} there exists an element $e\in G$ such that $a + e = a$ for any $a\in G$;
	\item \textbf{Inverse element:} for every $a\in G$ there exists some $b\in G$ such that $a + b = b + a = e$.
\end{itemize}

	Further, $\left(G,+\right)$ is an \textbf{abelian group} if the operation is commutative, that is, $a + b = b + a$ for any two $a,b\in G$.

\end{definition} %>>>

\begin{theorem}[The DPFs form an abelian group]\label{theo:abelian} %<<<
	The set $\left\{\ndpo{k}\right\}_{(k\in\mathbb{Z})}$ equipped with the sum operation $\left(\ndpo{m} + \ndpo{n}\right)\left[X\right] = \ndpo{n}\left[X\right] + \ndpo{m}\left[X\right]$ and the neutral element $\mathbf{0}$ (the null operator) is an abelian group. \end{theorem}
\textbf{Proof.} Consider some $X(t)\in\left[\mathbb{R}\to\mathbb{C}\right]$. Let $n,m,p\in\mathbb{N}$, and without loss of generality assume $n\geq m$. Then

\begin{gather}
	\left(\ndpo{m}\left[X\right] + \ndpo{n}\left[X\right] \right) + \ndpo{p}\left[X\right] = \left(\sum_{i=0}^m \gamma_i^n (t) X^{(i)}(t) + \sum_{k=0}^n \gamma_k^n (t) X^{(k)}(t)\right) + \sum_{c=0}^p \gamma_i^c (t) X^{(c)}(t) = \nonumber\\[3mm]
	= \sum_{i=0}^m \gamma_i^n (t) X^{(i)}(t) + \left( \sum_{k=0}^n \gamma_k^n (t) X^{(k)}(t) + \sum_{c=0}^p \gamma_i^c (t) X^{(c)}(t)\right) = \ndpo{m}\left[X\right] + \left(\ndpo{n}\left[X\right] + \ndpo{p}\left[X\right]\right)
\end{gather}

	\noindent and associativity follows from the associativity of complex sums. The neutral element is defined as the null operator, such that $\mathbf{0}\left[X\right] = 0$ for any signal $X(t)$; this means that

\begin{equation} \ndpo{n}\left[X\right] + \mathbf{0}\left[X\right] = \sum_{k=0}^n \gamma_k^n (t) X^{(k)}(t) + 0 = \sum_{k=0}^n \gamma_k^n (t) X^{(k)}(t) = \ndpo{n}\left[X\right] .\end{equation}

	Also naturally, one can define the inverse element of $\ndpo{n}\left[X\right]$ from the linearity of the DPFs, by multiplying $\ndpo{n}$ by $-1$ to obtain $-\ndpo{n}\left[X\right]$. Finally, 

\begin{align} \ndpo{n}\left[X\right] + \ndpo{m}\left[X\right] &= \sum_{k=0}^n \gamma_k^n (t) X^{(k)}(t) + \sum_{i=0}^m \gamma_i^n (t) X^{(i)}(t) = \nonumber\\[3mm] &= \sum_{i=0}^m \gamma_i^n (t) X^{(i)}(t) + \sum_{k=0}^n \gamma_k^n (t) X^{(k)}(t) = \ndpo{m}\left[X\right] + \ndpo{n}\left[X\right]\end{align}

	\noindent proving commutativity.

\vspace{5mm}
\hrule
\vspace{5mm} %>>>

	In short, groups generalize the addition operation, and theorem \ref{theo:abelian} proves that the sum of operators follows immediately from the complex sum and the linearity of DPFs. Further, we prove that the DPFs form an even more special structure called a \textit{ring}.

\begin{definition}[Ring]%<<<
	A \textbf{ring} is a set $G$ equipped with two binary operations: an addition, denoted ``$+$'', and a multiplication, denoted ``$\cdot$'', such that

\begin{itemize}
	\item $G$ is an abelian group under the addition;
	\item \textbf{Multiplication is associative:} for any three $a,b,c\in G$, $\left(a\cdot b\right)\cdot c = a\cdot \left(b\cdot c\right)$;
	\item \textbf{Neutral element of multiplication:} there exists an element $1\in G$ such that $a\cdot 1 = 1\cdot a = a$ for every $a\in G$;
	\item \textbf{Multiplication is distributive with respect to addition:} for any three $a,b,c\in G$, $a\cdot\left(b + c\right) = a\cdot b + a\cdot c$ (left distributivity) and $\left(b + c\right)\cdot a = b\cdot a + c\cdot a$ (right distributivity).
\end{itemize}

	Further, $\left(G,+,\cdot\right)$ is a \textbf{commutative ring} if multiplication is commutative, that is, $a \cdot b = b \cdot a$ for any two $a,b\in G$.
\end{definition} %>>>

\begin{theorem}[The DPFs form a commutative ring]\label{theo:ring}%<<<
	The set $\left\{\ndpo{k}\right\}_{(k\in\mathbb{Z})}$ equipped with the composition operation

\begin{equation} \left(\ndpo{m}\circ \ndpo{n}\right)\left[X\right] = \left(\ndpo{n}\circ \ndpo{m}\right)\left[X\right] = \ndpo{(n+m)}\left[X\right] \label{eq:ring_composition}\end{equation}

	\noindent and the neutral element $\mathbf{I}$ (the identity or zero-order operator $\ndpo{0}$) is a commutative ring.
\end{theorem}
\textbf{Proof.} Starting from the property of differentiation, we know that for any $n,m\in\mathbb{N}$

\begin{equation} \dfrac{d^n}{dt^n}\left(\dfrac{d^m}{dt^m} x(t)\right) = \dfrac{d^m}{dt^m}\left(\dfrac{d^n}{dt^n} x(t)\right) = \dfrac{d^{(n+m)}}{dt^{(n+m)}} x(t) \end{equation}

	\noindent yielding

\begin{equation} \ndpo{n}\left[\raisebox{4mm}{} \ndpo{m}\left[X\right]\right] = \ndpo{m}\left[\raisebox{4mm}{} \ndpo{n}\left[X\right]\right] = \ndpo{(n+m)}\left[X\right] .\end{equation}

	Also from the associativity of differentiation,

\begin{equation} \dfrac{d^n}{dt^n}\left(\dfrac{d^{(m+p)}}{dt^{(m+p)}} x(t)\right) = \dfrac{d^{(n+m+p)}}{dt^{(n+m+p)}} x(t) = \dfrac{d^{((n+m)+p)}}{dt^{((n+m)+p)}} x(t) = \dfrac{d^{(n+m)}}{dt^{(n+m)}} \left(\dfrac{d^{p}}{dt^{p}} x(t)\right) \end{equation}

	\noindent yielding

\begin{equation} \ndpo{n}\left[\raisebox{4mm}{} \ndpo{(m+p)}\left[X\right]\right] = \ndpo{(n+m+p)}\left[X\right] = \ndpo{((n+m)+p)}\left[X\right] = \ndpo{(n+m)}\left[\raisebox{4mm}{} \ndpo{p}\left[X\right]\right] .\end{equation}

	For the neutral element, adopt the identity operator $\mathbf{I}\left[X\right] = X(t)$; therefore

\begin{equation} \ndpo{n}\left[\raisebox{4mm}{} \mathbf{I}\left[X\right]\right] = \ndpo{n}\left[X\right],\ \mathbf{I}\left[\raisebox{4mm}{} \ndpo{n}\left[X\right]\right] = \ndpo{n}\left[X\right] .\end{equation}

	Finally, the distributivity follows from the distributivity of derivatives:

\begin{equation} \dfrac{d^n}{dt^n}\left(\dfrac{d^m}{dt^m}x(t) + \dfrac{d^p}{dt^p}x(t)\right) = \dfrac{d^{(n+m)}}{dt^{(n+m)}}x(t) + \dfrac{d^{(n+p)}}{dt^{(n+p)}}x(t) \end{equation}

	\noindent and it follows from this that

\begin{equation} \ndpo{n}\left[\raisebox{4mm}{} \ndpo{m}\left[X\right] + \ndpo{p}\left[X\right]\right] = \ndpo{(n+m)}\left[X\right] + \ndpo{(n+p)}\left[X\right] .\end{equation}

	Commutativity also follows from the commutativity of differentials.
\hfill$\blacksquare$
\vspace{5mm}
\hrule
\vspace{5mm}%>>>

	Finally, rings generalize the operation of a multiplication; for DPFs, this means that the composition of operators behaves akin to the complex multiplication. Most importantly, we can notice that the n-th order operator $\ndpo{k}$ is in essence the composition of $n$ first order operators, that is,

\begin{equation} \ndpo{n}\left[X\right] = \overbrace{\ndpo{1}\left[\raisebox{4mm}{} \ndpo{1}\left[\raisebox{3mm}{} \dots \ndpo{1}\left[\right.\right.\right.}^{\text{n times}}  X \left. \raisebox{3mm}{} \left. \raisebox{4mm}{} \left. \right]\right]\right] \end{equation}

	\noindent such that the n-th order operator can be seen as the ``n-th power'' of the first-order operator; because of this, we can drop the ``1'' in the notation and denote $\dpo = \ndpo{1}$. Further, we can also define the inverse operations as ``negative powers'', since the composition or ``multiplication'' of the inverse operator with the operator is the identity:

\begin{equation} \left\{\begin{array}{l}
	\left(\ndpo{n}\right)^{-1}\left[\raisebox{3mm}{} \ndpo{n}\left[X\right]\right] = X(t) \Leftrightarrow \left(\ndpo{n}\right)^{-1}\circ \ndpo{n} = \mathbf{I} \\[3mm]
	\ndpo{n}\left[\raisebox{4mm}{} \left(\ndpo{n}\right)^{-1}\left[X\right]\right] = X(t) \Leftrightarrow \ndpo{n}\circ \left(\ndpo{n}\right)^{-1} = \mathbf{I}
\end{array}\right.
\end{equation}

	\noindent thus we can denote

\begin{equation} \left(\ndpo{n}\right)^{-1} = \ndpo{(-n)} = \dfrac{\mathbf{I}}{\ndpo{n}} = \left(\dfrac{\mathbf{I}}{\dpo}\right)^n .\end{equation}

	\noindent as the ``division operation''. Interestingly, in the commutative ring of Dynamic Phasor Functionals the multiplicative inverse $\ndpo{(-n)}$ exists for any nonzero element; this immediately causes the space $\left\{\ndpo{k}\right\}_{k\in\mathbb{Z}}$ to be a field, as per definition \ref{def:field}.

	Thence, we can rework the definition of DPFs to allow for negative orders and a recurrence.

\begin{definition}[Dynamic Phasor Functional $\dpo$ (DPF)]\label{def:steinmetzoperator_revisited} %<<<
	Let $x(t)$ a nonstationary sinusoid with an apparent frequency $\omega(t)$, $X(t)$ its dynamic phasor. Then the \textbf{n-th order Dynamic Phasor Functional}, denoted $\ndpo{n}$, is defined for $n>0$ as 

\begin{equation} \ndpo{n} = \sum_{k=0}^n \gamma_k^n \left(t\right)\mathbf{D^k} , \end{equation}

	where $\mathbf{D}^k$ is the k-th order differential operator:

\begin{equation} \ndpo{n} \left[X\right] = \left(\sum_{k=0}^n \gamma_k^n \left(t\right)\mathbf{D}^k\right) \left[X\right] = \sum_{k=0}^n \gamma_k^n (t) X^{(k)}(t). \end{equation}

	Equivalently, one can define the recursion

\begin{equation}\left\{\begin{array}{l} \ndpo{0}\left[X\right] = X(t) \\[3mm] \ndpo{1}\left[X\right] = \dot{X}(t) + j\omega(t)X(t) \\[3mm] \ndpo{n}\left[X\right] = \ndpo{1}\left[\raisebox{3mm}{} \ndpo{(n-1)}\left[X\right]\right] \end{array}\right. \end{equation}

	For $n < 0$, $\ndpo{n} \left[X\right]$ is the dynamic phasor $Y(t)$ that satisfies $X(t) = \dpo\left[Y\right]$ or

\begin{equation} X(t) = \sum_{k=0}^n \gamma_k^n (t) Y^{(k)}(t), \end{equation}

	\noindent together with a set of known initial conditions $Y(0),Y'(0),\cdots ,Y^{(n-1)}(0)$. Finally, for $n = 0$, $\ndpo{0}$ is the identity operator $\mathbf{I}$, that is, $ \ndpo{0} \left[X\right] = X(t)$.
\end{definition} %>>>

	Theorems \ref{theo:abelian} and \ref{theo:ring} mean in essence that DPFs can be added and multiplied to compose an entire space of operators; in practicality, these theorems formalize the fact that the DPFs proposed transform derivatives in to algebraic operations.

%-------------------------------------------------
\subsection{Field and vector space}

	We can also prove that, because any non-null DPF has a non-null inverse, the set of DPFs form a field, as per definition \ref{def:field}.

\begin{theorem}[The DPFs form a field]\label{theo:dpfs_field}%<<<
	The set $\left\{\ndpo{k}\right\}_{(k\in\mathbb{Z})}$ equipped with the sum operation of theorem \ref{theo:abelian} and the multiplication (composition) operation of theorem \ref{theo:ring} is a field.
\end{theorem}
\textbf{Proof:} almost all the properties of fields are already satsfied by the theorems that prove DPFs form an abelian group and a commutative ring; one property is missing, however, that for any non-null element there must be a non-null multiplicative inverse. Indeed pick some $\ndpo{k}$, which obligatorily transforms some non-null signal $X(t)$ into a non-null $Y(t)$. Then the inverse $\ndpo{(-k)}$, which provedly exists, is non-null.
\vspace{5mm}
\hrule
\vspace{5mm}%>>>

	Further, we can define a scaling-by-complex operation of DPFs by the following process. First, we define a family of \textit{scaling} functionals

\begin{definition}[Scaling DPFs]\label{def:scaling_dpfs} %<<<
	Consider some complex number $\alpha$ and the identity operator $\mathbf{I}$. Then the \textbf{scaling operator} $\alpha\mathbf{I}$ is such that

\begin{equation} \left(\alpha \mathbf{I}\right)\left[X\right] = \alpha X(t).\end{equation}

	And define the family of \textit{scaling} operators, denoted $\mathbb{C}\mathbf{I}\equiv \left\{\alpha \mathbf{I}\right\}_{\alpha\in\mathbb{C}}$. 
\end{definition} %>>>

	Using scaling operators, we define a multiplication-by-scalar operation in the space of Dynamic Phasor Functionals.

\begin{definition}[Complex scaling of DPFs]\label{def:scaling_dpfs_all} %<<<
	Consider some complex number $\alpha$, and consider the operator $\ndpo{k}$. Then define the multiplication-by-scalar operation as 

\begin{equation} \alpha \ndpo{k} \equiv \left(\alpha\mathbf{I}\right)\circ\ndpo{k} = \ndpo{k}\circ\left(\alpha\mathbf{I}\right) .\end{equation}
\end{definition} %>>>

	Essentially, the ``scaled'' DPF is the tandem operation of applying the DPF to a signal and then scaling the result. Due to the commutativity of composition, this also is equivalent to applying the DPF to a scaled signal. Thus we can also define a linear combination of DPFs as $\alpha\ndpo{k} + \beta\ndpo{m}$, where $\alpha\beta\in\mathbb{C}$. Ultimately, this means that DPFs also form a \textbf{vector space} over the complex numbers, as per definition \ref{def:vector_space}. The theory if abstract algebra also proves that linear combinations of DPFs are invertible.

\begin{theorem}[The DPFs form a vector space]\label{theo:vspace}%<<<
	The set $\left\{\ndpo{k}\right\}_{(k\in\mathbb{Z})}$ equipped with the multiplication-by-scalar operation of definition \ref{def:scaling_dpfs} forms a vector space over the complex numbers.
\end{theorem}
\vspace{5mm}
\hrule
\vspace{5mm}%>>>

	Moreover, we can see that adopting the field of complex numbers as scalars, then the addition in the DPF space behaves like the complex addition, and the multiplication (composition) in DPF space behaves like the multiplication in complex space. As a matter of fact, we can establish an obvious bijection between any complex number $\alpha$ and the operator $\alpha\mathbf{I}$. This yields that the Dynamic Phasor Functionals are a \textbf{infinite extension field} of the zero-order operators $\left\{\alpha\mathbf{I}\right\}_{\alpha\in\mathbb{C}}$, and also an extension of the complex numbers. 

%-------------------------------------------------
\subsection{The $\dpS$ space and polynomials of DPFs} %<<<2

	All these results mean that many algebraic entities are defineable in the space of DPFs. For instance, one can define polynomials in the space of DPFs as

\begin{equation} \mathbf{P}\left(\dpo\right) = \sum_{k=0}^n \left(\alpha_k\mathbf{I}\right)\circ \dpo = \sum_{k=0}^n \alpha_k \dpo^{k} \text{ where } \alpha_k\in\mathbb{C} \text{ and } \alpha_n\neq 0, \label{eq:def_dpo_poly}\end{equation}
	
	\noindent which is the operator such that

\begin{equation} Y(t) = \mathbf{P}\left[X\right] \Leftrightarrow Y(t) = \sum_{k=0}^n \alpha_k\ndpo{k}\left[X\right] .\end{equation}

	And we denote the set of polynomials of the form $\mathbf{P}\left(\dpL\right)$ with complex coefficients in $\dpS$ as $\mathbb{C}\left[\dpL\right]$ and, by restriction, we denote the set of polynomials of $\dpo$ with complex coefficients as $\mathbb{C}\left[\dpo\right]$. In formal terms, this means that DPFs form a \textbf{polynomial ring}. With these properties in mind, we can find the largest class of operators that fulfill the properties of being a group, a ring, a field — holding all the properties ennunciated. It is obvious that any linear combination of the DPFs, as well as any polynomial, is going to fill the properties; however, we can also consider their inverses. Therefore, we can define the complete space of Dynamic Phasor Functionals as a broader term for a myriad of combinations.

\begin{definition}[Dynamic Phasor Functional Space]\label{def:dpft_space}
	The \textbf{Dynamic Phasor Functional Space} is the set $\dpS$ built by linear combinations of polynomials of $\dpo$ and also inverse operators of those polynomials, that is, rational functions of $\dpo$:

\begin{equation} \dpS = \left\{\dpL = \dfrac{\mathbf{N}\left(\dpo\right)}{\mathbf{D}\left(\dpo\right)}: \mathbf{N,D}\in\mathbb{C}\left[\dpo\right]\right\} \end{equation}
\end{definition}

	Thus, one can come up with (literally) several infinities of functionals that belong to $\dpS$; for quick examples, the functionals

\begin{equation} \boldsymbol{\lambda}_1 = \ndpo{2} + \dfrac{4\dpo}{\left(\ndpo{3} - 2\ndpo{2} + 3\mathbf{I}\right)},\ \boldsymbol{\lambda}_2 = \ndpo{2} - 2j\sqrt{2}\ndpo{1} - 2\mathbf{I} = \left(\dpo - j\sqrt{2}\mathbf{I}\right)^2,\ \boldsymbol{\lambda}_3 = \dpo \label{eq:dps_examples}\end{equation}

	\noindent all can be inverted, multiplied, linearly combined, so on and so forth. Obviously, an inverse operation exists for any $\boldsymbol{\lambda}\in\dpS$ except for the null operator.

	Definition \ref{def:dpft_space} comes in handy when we define impedances in the Dynamic Phasor space. It is simple to see that if the functional relationships of \eqref{sys:dpo_impedances_first} are linearly combined, the resulting expressions will be polynomials of $\dpo$ and its inverses; this, $\dpS$ is a space that generalizes the notion of impedances in Dynamic Phasor space. 

	By definition, any $\boldsymbol{\lambda}_1,\boldsymbol{\lambda}_2\in\dpS$ can be multiplied, combined, inverted; as such, $\dpS$ is also an abelian group, a commutative ring, an algebraically closed field, and form a vector space over the complex numbers. Particularly, polynomials in $\dpS$ are linear combinations in $\dpS$, thus transformations in $\dpS$:

\begin{equation} \mathbf{P}\left(\cdot\right): \left\{\begin{array}{rcl} \dpS &\to& \dpS \\[3mm] \boldsymbol{\lambda} &\mapsto& \displaystyle\sum_{k=0}^n \alpha_k \boldsymbol{\lambda}^{k} \end{array}\right. \label{eq:dpfs_poly_def}\end{equation}

	Naturally, because $\dpS$ is closed to powers and linear combinations, any polynomial is a transformation in this space. We now explore the properties of such polynomials; we first start with the most basic of properties known, which is the seemingly simple property that any polynomial $\mathbf{P}$ can be written as a product of its monomials. This property defines very specific structures called algebraically closed fields.

\begin{definition}[Algebraically close fields]
	A field $F$ is said to be \textbf{algebraically closed} if the Fundamental Theorem of Algebra holds for it, that is, any polynomial in $F$ can be written as the multiplication of the monomials of its roots.
\end{definition}

	Proving a certain field is algebraically closed, however, is not immediate. Luckily, the theory of abstract algebra offers us many ways to prove this, for instance, 

\begin{theorem}[Algebraically closed fields and irreducible polynomials \pcite{goncalvesIntroducaoAlgebra2021}]
	A field $F$ is algebraically closed if only irreducible polynomials are those of degree one, that is, given any root $a\in F$, the lowest degree polynomial that has $a$ as a root is $x - a$.
\end{theorem}

	In the case of $\dpS$, this property is simple to prove.

\begin{theorem}[$\dpS$ is algebraically closed]\label{theo:dps_alg_closed} %<<<
	The set of DPFs $\dpS$ is algebraically closed.
\end{theorem}
\textbf{Proof:} let $\dpL_0\in\dpS$ and let $\mathbf{P}\left(\dpL\right)\in\mathbb{C}\left[\dpL\right]$,\ $\mathbf{P}\left(\dpL_0\right) = \mathbf{0}$. Then $\mathbf{P}\left(\dpL\right)$ is, by a polynomial division, multiple of $\left(\dpL - \dpL_0\right)$. Supposing $\mathbf{P}$ is irreducible (it cannot be factored as the multiplication of two other polynomials), and for any $\dpL_0$ the monomial $\dpL - \dpL_0$ is in $\dpS$, then for any $\dpL_0$ the polynomial $\left(\dpL - \dpL_0\right)$ exists and is the smallest one in degree that has $\dpL_0$ as solution; therefore $\mathbf{P}$ is equal to $\alpha\left(\dpL - \dpL_0\right)$ for some non-zero complex $\alpha$. \hfill$\blacksquare$
\vspace{5mm}
\hrule
\vspace{5mm} %>>>

	It follows from the Fundamental Theorem of Algebra that any polynomial $\mathbf{P}\in\dpS\left[\dpL\right]$ of degree $n$ has exactly $n$ not necessarily distinct roots $\dpL_i$ and can be written as a multiplication of the monomials of these roots:

\begin{equation} \mathbf{P}\left(\boldsymbol{\lambda}\right) = \alpha_n \prod_{i=1}^n \left( \dpL - \dpL_i\right). \label{eq:fundamental_algebra_dpfs}\end{equation}

%-------------------------------------------------
\subsection{Matrices in $\dpS$}\label{subsec:matrces_in_dpfts} %<<<2

	Following the definition of Dynamic Phasor Functionals as the larger class $\dpS$ as in definition \ref{def:dpft_space}, we now want to explore the feasibility of matrices of these operators. Such matrices become particularly useful in circuit network analysis since their manipulations allow for matrix analysis of circuits in Dynamic Phasor space, allowing us to prove the Superposition Theorem, culminating in Thèvenin and Norton's theorems, as well as the entirety of Network Analysis Theory.

	Initially, one tries to leverage the theory of linear algebra in chapter \ref{chapter:linear_systems} by prove that the space of n coordinates $\dpS^n$ is a complex space, or equivalently, that it can be written as linear combinations of a basis using complex numbers as the underlying field. This would mean that the theory developed in that chapter can be used in its integrity without modications, and the matter would be solved. So to define a matrix in this space, one starts from subsection \ref{sec:bases_matrices_operations} where a matrix is built as the representation of a linear transformation on the space considered represented against some basis. Therefore one would start by adopting the most natural basis possible:

\begin{equation} \mathbf{B} = \left\{\ndpo{k}\right\}_{k\in\mathbb{Z}} = \left\{\cdots,\ndpo{(-3)},\ndpo{(-2)},\ndpo{(-1)},\mathbf{I},\dpo,\ndpo{2},\ndpo{3},\cdots\right\},\end{equation}

	\noindent and immediately any finite Laurent Polynomial $\mathbf{P}\in\mathbb{C}\left[\dpo,\ndpo{(-1)}\right]$, that is, any finite combination

\begin{equation} \mathbf{P}\left(\dpo\right) = \sum_{k\in\mathbb{Z}_n} \alpha_k\ndpo{k} \text{ for some } n\in\mathbb{N} \label{eq:laurent_poly_xi}\end{equation}

	\noindent can be written as a finite linear combination of the elements of the basis $\mathbf{B}$, that is, any such polynomial admits a finite representation under the basis $\mathbf{B}$. However, once one considers the entirety of $\dpS$ which contains not only polynomials of the form \eqref{eq:laurent_poly_xi} but also inverses and linear combinations, this approach does not lead to happy results: we prove now that the space $\dpS$ has infinite dimension over the complex numbers, leading to infinite dimensional matrices. We use a result from Complex Analysis, the Laurent Series, to prove that any element $\dpL\in\dpS$ can be written as an infinite discrete sum of a basis.

\begin{theorem}[Laurent series of a complex function \pcite{ahlfors1979complex}]\label{theo:laurent} %<<<

	Let $f(z)\in\left[\mathbb{C}\to\mathbb{C}\right]$ analytic over some annulus around a certain point $z_0$, that is, there exist $0 \leq r < R$ such that $f(z)$ is analytic in $A = \left\{z\in\mathbb{C}:\ r \leq \left\lvert z - z_0\right\rvert \leq R\right\}$. Let $\gamma$ a continuous clockwise curve in $A$. Then for any $z\in A$,

\begin{equation} f(z) = \sum_{k\in\mathbb{Z}} a_k\left(z - z_0\right)^k \text{, where } a_k = \dfrac{1}{2\pi j} \oint_\gamma \dfrac{f(z)}{\left(z - z_0\right)^{k+1}} dz \end{equation}

\end{theorem} %>>>

	It can be shown that the specific functions that compose $\dpS$ — finite degree polynomials, their inverses and subsequent combinations — are always infinitely differentiable on the entire $\mathbb{C}$ but some finite points, called poles \pcite{ahlfors1979complex} which are simple to work around due to being removable singularities. Thus such functions are holomorphic (infinitely differentiable at some neighborhood of any complex point) and the Laurent series converges and can be calculated about any $z_0\in\mathbb{C}$. This means that this theorem can always be applied to $\dpS$, due to the fact that $\dpS$ is an extension field of $\mathbb{C}$ and a polynomial ring, and that we can adopt the basis $\mathbf{B}$. For instance, adopt

\begin{equation} \dpL = \dfrac{\ndpo{5} - \mathbf{I}}{\ndpo{3} - \dpo + 3\mathbf{I}} .\end{equation}

	The Laurent series of the converse complex polynomial calculated about $z_0 = 0$ is

\begin{equation} f(z) = \dfrac{z^5 - 1}{z^3 - z + 3} = z^2 + 1 - \dfrac{3}{z} + \dfrac{1}{z^2} - \dfrac{7}{x^3} + \dfrac{10}{z^4} - \dfrac{10}{z^5} + \cdots ,\end{equation}

	\noindent thus

\begin{equation} \dpL = \dfrac{\ndpo{5} - \mathbf{I}}{\ndpo{3} - \dpo + 3\mathbf{I}} = \ndpo{2} + \mathbf{I} - 3\ndpo{(-1)} + \ndpo{(-2)} - 7\ndpo{(-3)} + 10\ndpo{(-4)} - 10\ndpo{(-5)} + \cdots .\end{equation}

	\noindent so that we can write $\dpL$ as a representation on the basis $\mathbf{B}$ adopted

\begin{equation} \dpL = \left[\cdots , -10 ,  10 ,  -7 ,  1 ,  -3 ,  1 ,  0 ,  1 , \cdots \right]^\transpose_{\left[\mathbf{B}\right]} \end{equation}

	\noindent consequently $\dpL$ has an infinite-dimensional representation in the basis chosen. Immediately one concludes that the Laurent series of a generic element of $\dpS$ will be infinite, due to the fact that the space contains arbitrary combinations of polynomial inverse functions. Reestated, in order for a particular $\dpL$ to have a finite Laurent series it must necessarily be a finite linear combination of the elements of the basis, namely, be of the form \eqref{eq:laurent_poly_xi} — which is certainly not the case for all members of $\dpS$. Consequently, a tabular arrangement of such vectors with respect to the complex numbers will lead to infinite dimensional matrices, which are not at all useful and would void some basic results in the theory presented, like the Rank-Nullity Theorem \ref{theo:rank_nullity} which depends on finite-dimensional operators, and from which a lot of other results follow.

	To remedy this we take extra steps. First, note that the fact $\dpS$ contains the scaling operators $\left\{\alpha\mathbf{I}\right\}_{\alpha\in\mathbb{C}}$ means ultimately that the space $\dpS^n$ of functional vectors of length $n$ it is a vector space over $\dpS$ itself. To prove this, define an addition operation

\begin{equation} \left(+\right)_{\dpS^n}: \left\{\begin{array}{rcl} \dpS^n\times\dpS^n &\to& \dpS^n \\[5mm] \left(\left[\begin{array}{c} \dpL_1 \\[3mm] \dpL_2 \\[3mm] \vdots \\[3mm] \dpL_n \end{array}\right],\left[\begin{array}{c} \apL_1 \\[3mm] \apL_2 \\[3mm] \vdots \\[3mm] \apL_n \end{array}\right]\right) &\mapsto& \left[\begin{array}{c} \dpL_1 + \apL_1 \\[3mm] \dpL_2 + \apL_2 \\[3mm] \vdots \\[3mm] \dpL_n + \apL_n \end{array}\right] \end{array}\right. \end{equation}

	\noindent where the addition of two operators $(+)$ is the one of theorem \ref{theo:abelian}. Also define a multiplication-by-scalar

\begin{equation} \left(\cdot\right)_{\dpS^n}: \left\{\begin{array}{rcl} \dpS\times\dpS^n &\to& \dpS^n \\[5mm] \left(\boldsymbol{\pi},\left[\begin{array}{c} \dpL_1 \\[3mm] \dpL_2 \\[3mm] \vdots \\[3mm] \dpL_n \end{array}\right]\right) &\mapsto& \left[\begin{array}{c} \boldsymbol{\pi}\cdot\dpL_1 \\[3mm] \boldsymbol{\pi}\cdot\dpL_2 \\[3mm] \vdots \\[3mm] \boldsymbol{\pi}\cdot\dpL_n \end{array}\right] \end{array}\right. \end{equation}

	\noindent where the multiplication $(\cdot)$ of two operators is the one of theorem \ref{theo:ring}. Then these two operations wholly fulfill the definition of a vector space, as per definition \ref{def:vector_space}.

	Thus we define $\dpS$ not as a vector space over the complex numbers, but also over its own elements; we now show that this allows us to undertake the representation of linear operators over vectors in a vector space as in subsection \ref{sec:bases_matrices_operations}.

\begin{lemma} The space $\dpS^n$ of vectors of length $n$ in $\dpS$ is a field over $\dpS$ and has dimension $n$. \end{lemma}
\textbf{Proof.} Let $\mathbf{e}_k\in\dpS^n$ the vector that has an identity operator $\mathbf{I}$ on the k-th positions and the null operator everywhere else. Consider the collection $\left(\mathbf{e}_1,\mathbf{e}_2,\mathbf{e}_3, \cdots ,\mathbf{e}_n\right)$, and one can easily see that this is a basis of $\dpS^n$. For instance, adopt an arbitrary element of $\dpS^n$:

\begin{equation} \boldsymbol{\Lambda} = \left[\begin{array}{c} \dpL_{1} \\[3mm] \dpL_{2} \\[3mm] \vdots \\[3mm] \dpL_{n} \end{array}\right] \end{equation}

	\noindent and naturally $\boldsymbol{\Lambda} = \sum_{k=1}^n \dpL_k \mathbf{e}_k$, meaning that the span of the collection adopted is $\dpS^n$. Now admit that $\boldsymbol{\Lambda}$ has another set of coordinates in this collection, say $\boldsymbol{\alpha}_k$. Then

\begin{equation} \boldsymbol{\Lambda} = \sum_{k=1}^n \dpL_k \mathbf{e}_k = \boldsymbol{\Lambda} = \sum_{k=1}^n \boldsymbol{\alpha}_k \mathbf{e}_k \Leftrightarrow \sum_{k=1}^n \left(\dpL_k - \boldsymbol{\alpha}_k\right)\mathbf{e}_k = \mathbf{0}_n\end{equation}

	\noindent but since the $\mathbf{e}_k$ are defined as having $\mathbf{I}$ in the k-th position and the null operator everywhere, the only possible solution to this equation is $\dpL_k - \boldsymbol{\alpha}_k = \mathbf{0}$, meaning that the collection of the $\mathbf{e}_k$ span the entire $\dpS^n$ and are linearly independent, thus this collection is a basis of $\dpS^n$. Further, it is simple to see that by removing any of the $\mathbf{e}_k$ from the basis, the resulting collection cannot express the entire $\dpS^n$; meaning that $n$ is the least number of linearly independent vectors needed to span this set — therefore it has dimension $n$.

\begin{theorem}[Existence of DPF matrices] \label{theo:dpf_matrices_exist} %<<<
	Any linear map $\mathbf{M}\in\left[\dpS^n\to\dpS^n\right]$ admits a matrix representation $\left[\mathbf{M}\right]_{\boldsymbol{\Gamma}}\in\dpS^{(n\times n)}$ under some basis $\boldsymbol{\Gamma}$ of $\dpS^n$, and particularly under the canonical basis $\boldsymbol{\Psi}_n$ of the vectors $\left(\mathbf{e}_k\in\dpS^n\right)_k$:

\begin{equation} \boldsymbol{\Psi}_n = \left(\mathbf{e}_1,\mathbf{e}_2,\mathbf{e}_3, \cdots ,\mathbf{e}_n\right) = \left(\left[\begin{array}{c} \mathbf{I} \\[3mm] \mathbf{0} \\[3mm] \mathbf{0} \\[3mm] \vdots \\[3mm] \mathbf{0} \end{array}\right], \left[\begin{array}{c} \mathbf{0} \\[3mm] \mathbf{I} \\[3mm] \mathbf{0} \\[3mm] \vdots \\[3mm] \mathbf{0} \end{array}\right], \left[\begin{array}{c} \mathbf{0} \\[3mm] \mathbf{0} \\[3mm] \mathbf{I} \\[3mm] \vdots \\[3mm] \mathbf{0} \end{array}\right], \cdots , \left[\begin{array}{c} \mathbf{0} \\[3mm] \mathbf{0} \\[3mm] \mathbf{0} \\[3mm] \vdots \\[3mm] \mathbf{I} \end{array}\right]\right) . \label{eq:dpf_canonical_basis}\end{equation}
\end{theorem}
\noindent\textbf{Proof:} the previous lemma shows that $\dpS^n$ is a field over $\dpS$ with dimension $n$. Adopt a basis $\boldsymbol{\Gamma} = \left(\boldsymbol{\gamma}_k\right)_{k=1}^n$ as a basis of $\dpS^n$. Let $\boldsymbol{\Lambda}$ an arbitrary element of $\dpS^n$ with a set of coordinates $\left(\dpL_k\right)_{k=1}^n$ under $\Gamma$:

\begin{equation} \boldsymbol{\Lambda} = \left[\begin{array}{c} \dpL_{1} \\[3mm] \dpL_{2} \\[3mm] \vdots \\[3mm] \dpL_{n} \end{array}\right] \end{equation}

	\noindent and consider $\mathbf{M}\in\left[\dpS^n\to\dpS^n\right]$ some linear mapping. Therefore

\begin{equation} \mathbf{M}\left[\boldsymbol{\Lambda}\right] = \mathbf{M}\left[\begin{array}{c} \dpL_{1} \\[3mm] \dpL_{2} \\[3mm] \vdots \\[3mm] \dpL_{n} \end{array}\right]_{\boldsymbol{\Gamma}} = \mathbf{M}\left[\sum\limits_{k=1}^n \dpL_k \boldsymbol{\gamma}_k \right] = \sum_{k=1}^n \dpL_k \mathbf{M}\left[\boldsymbol{\gamma}_k\right] \end{equation}

	\noindent thereby allowing us to group the vectors $\mathbf{M}\left[\boldsymbol{\gamma}_k\right]$ in a tabular arrangement just like \eqref{eq:tabular_arrangement}:

\begin{equation} \left[\mathbf{M}\right]_{\boldsymbol{\Gamma}} = \left[\raisebox{15mm}{} \begin{array}{cccc} \left[\begin{array}{c} \vdots \\[3mm] \mathbf{M}\left[\boldsymbol{\gamma}_1\right] \\[3mm] \vdots \end{array}\right] & \left[\begin{array}{c} \vdots \\[3mm] \mathbf{M}\left[\boldsymbol{\gamma}_2\right] \\[3mm] \vdots \end{array}\right] & \cdots & \left[\begin{array}{c} \vdots \\[3mm] \mathbf{M}\left[\boldsymbol{\gamma}_n\right] \\[3mm] \vdots \end{array}\right]\end{array}\right]\end{equation}

	\noindent consequently achieving a matrix representation of the map $\mathbf{M}$ under the arbitrary basis $\boldsymbol{\Gamma}$, which is the very definition of a matrix as in \eqref{eq:tabular_arrangement}, and the notion of complex matrices in $\dpS$ is well-defined. Particularly, adopting the canonical basis $\boldsymbol{\Psi}_n$ as defined in \eqref{eq:dpf_canonical_basis}, we achieve a canonical matrix representation of the linear mapping $\mathbf{M}\left[\cdot\right]$.
\hfill$\blacksquare$\vspace{5mm}\hrule\vspace{5mm} %>>>

\begin{definition}[Matrices in $\dpS$]\label{def:matrices_in_dps} A matrix of DPFs $\mathbf{M}\in\dpS^{(n\times m)}$ is the tabular arrangement 

\begin{equation} \mathbf{M} = \left\{\dpL_{(i,j)}\right\}_{(i\in \mathbb{N}^*_n,j\in \mathbb{N}^*_m)} = \left[\begin{array}{ccccc} 
	\dpL_{11} & \dpL_{12} & \dpL_{13} & \dots  & \dpL_{1m} \\[3mm]
	\dpL_{21} & \dpL_{22} & \dpL_{23} & \dots  & \dpL_{2m} \\[3mm]
	\dpL_{31} & \dpL_{32} & \dpL_{33} & \dots  & \dpL_{3m} \\[3mm]
	\vdots    & \vdots    & \vdots    & \ddots & \vdots    \\[3mm]
	\dpL_{n1} & \dpL_{n2} & \dpL_{n3} & \dots  & \dpL_{nm}
\end{array}\right]_{(n\times m)}\label{eq:dpfs_matrix} \end{equation}
\end{definition}

	Due to the operational properties in $\dpS$, the linear algebra theory of chapter \ref{chapter:linear_systems} is still available for, and compatible with, this definition as we initially wanted given some minimal adaptations. One can easily define addition of matrices (trivially through the sum of the elements), scaling (trivially through scaling of its elements), matrice-by-vectors multiplications (definitions \ref{def:matrixbyvector} and \ref{def:vectorbymatrix}), multiplication of matrices by matrices of operators (definition \ref{def:matrixbymatrix}).

	Given the definition of matrix multiplication, immediately one notices that the matrix representation of $\boldsymbol{\Psi}_n$ where its vectors are its columns is the identity matrix in the space $\dpS^{(n\times n)}$, another reason to call this basis as the canonical one:

\begin{equation} \boldsymbol{\Psi}_n = \left[\begin{array}{ccccc} \mathbf{I} & \mathbf{0} & \mathbf{0} & \cdots & \mathbf{0} \\[3mm] \mathbf{0} & \mathbf{I} & \mathbf{0} & \cdots & \mathbf{0} \\[3mm] \mathbf{0} & \mathbf{0} & \mathbf{I} & \cdots & \mathbf{0} \\[3mm] \vdots & \vdots & \vdots & \ddots & \vdots \\[3mm] \mathbf{0} & \mathbf{0} & \mathbf{0} & \cdots & \mathbf{I} \end{array}\right]_{(n\times n)} .\end{equation}

	Moreover, since a neutral element of multiplication (the identity matrix) exists, then matrix invertibility (definition \ref{def:invertible_matrix}) is defineable, as well as determinants (definition \ref{def:determinant}) of such matrices. One can also continue down this path towards the eigendecomposition of these matrices and the entirety of linear algebra as presented in chapter \ref{chapter:linear_systems}.

	Finally, let us consider the vector of Dynamic Phasors $\mathbf{x} = \left[X_1(t),X_2(t),\cdots,X_n(t)\right]^\transpose$; let a collection $\left\{\dpL_{ij}\in\dpS\right\}_{i\in\mathbb{N}^*}^{j\in\mathbb{N}^*}$, and consider a linear transformation $\mathbf{M}$ in the space $\left[\mathbb{R}\to\mathbb{C}\right]^n$:

\begin{equation}
Y(t) = \mathbf{M}\left[X\right] \Leftrightarrow \left[\begin{array}{c} Y_1(t) \\[3mm] Y_2(t) \\[3mm] Y_3(t) \\[3mm] \vdots \\[3mm] Y_n(t) \end{array}\right] =
%
\left[\begin{array}{c}
	\dpL_{11}\left[X_1\right] + \dpL_{12}\left[X_2\right] + \dpL_{13}\left[X_3\right] + \cdots + \dpL_{1n}\left[X_n\right] \\[3mm]
	\dpL_{21}\left[X_1\right] + \dpL_{22}\left[X_2\right] + \dpL_{23}\left[X_3\right] + \cdots + \dpL_{2n}\left[X_n\right] \\[3mm]
	\dpL_{31}\left[X_1\right] + \dpL_{32}\left[X_2\right] + \dpL_{33}\left[X_3\right] + \cdots + \dpL_{3n}\left[X_n\right] \\[3mm]
	\vdots \\[3mm]
	\dpL_{n1}\left[X_1\right] + \dpL_{n2}\left[X_2\right] + \dpL_{n3}\left[X_3\right] + \cdots + \dpL_{nn}\left[X_n\right]
\end{array}\right] \label{eq:proposal_application}
\end{equation}

	\noindent that is, each $Y_i$ is a combination of the elements of $X(t)$

\begin{equation} Y_i(t) = \sum_{k=1}^n \dpL_{ik}\left[X_k\right] .\end{equation}

	This definition is highly resemblant of a matrix-by-vector multiplication where the columns of $\mathbf{M}$ are ``linearly combined'' as in definition \ref{def:matrixbyvector}, but the combination is given in terms of DPFs. Indeed, if we define the multiplication of a DPF and a Dynamic Phasor by their composition, as in $Y(t) = \dpL\circ X(t) = \dpL\left[X\right]$, then the transform \eqref{eq:proposal_application} can be seen as a composition too:

\begin{equation} %<<<
\mathbf{M}\left[X\right] = \mathbf{M}\circ X(t) =
%
\left[\begin{array}{ccccc}
	\dpL_{11} & \dpL_{12} & \dpL_{13} & \cdots & \dpL_{1n} \\[3mm]
	\dpL_{21} & \dpL_{22} & \dpL_{23} & \cdots & \dpL_{2n} \\[3mm]
	\dpL_{31} & \dpL_{32} & \dpL_{33} & \cdots & \dpL_{3n} \\[3mm]
	\vdots    & \vdots    & \vdots    & \ddots & \vdots    \\[3mm]
	\dpL_{n1} & \dpL_{n2} & \dpL_{n3} & \cdots & \dpL_{nn}
\end{array}\right] \circ
%
\left[\begin{array}{c} X_1(t) \\[3mm] X_2(t) \\[3mm] X_3(t) \\[3mm] \vdots \\[3mm] X_m(t) \end{array}\right]
\end{equation}%>>>

	 Therefore we can define the matrix representation $\left[\mathbf{M}\right]_{\boldsymbol{\Psi}_n}$ and the application \eqref{eq:proposal_application} as a ``multiplication'' of a DPF matrix for a signal vector:

\begin{equation} %<<<
\left(\cdot\right)_{\Xi^n}^{\left[\mathbb{R}\to\mathbb{C}\right]}:
%
\left\{\begin{array}{rcl}
	\Xi^{\left(n\times m\right)} \times \left[\mathbb{R}\to\mathbb{C}\right]^m &\to& \left[\mathbb{R}\to\mathbb{C}\right]^m \\[5mm]
\left[\begin{array}{ccccc} 
	\dpL_{11} & \dpL_{12} & \dpL_{13} & \dots  & \dpL_{1m} \\[3mm]
	\dpL_{21} & \dpL_{22} & \dpL_{23} & \dots  & \dpL_{2m} \\[3mm]
	\dpL_{31} & \dpL_{32} & \dpL_{33} & \dots  & \dpL_{3m} \\[3mm]
	\vdots    & \vdots    & \vdots    & \ddots & \vdots    \\[3mm]
	\dpL_{n1} & \dpL_{n2} & \dpL_{n3} & \dots  & \dpL_{nm}
\end{array}\right]
%
\left[\begin{array}{c} X_1(t) \\[3mm] X_2(t) \\[3mm] X_3(t) \\[3mm] \vdots \\[3mm] X_m(t) \end{array}\right] & \mapsto &
%
\left[\begin{array}{c}
\sum_{k=1}^m \dpL_{1k}\left[X_k\right] \\[3mm]
\sum_{k=1}^m \dpL_{2k}\left[X_k\right] \\[3mm]
\sum_{k=1}^m \dpL_{3k}\left[X_k\right] \\[3mm]
\vdots                                 \\[3mm]
\sum_{k=1}^m \dpL_{mk}\left[X_k\right]
\end{array}\right]
\end{array}\right. \label{eq:matrix_by_dpvec_def}
\end{equation} %>>>

	Like the matrix-by-vector has a vector-by-matrix equivalent multiplication, we can define the transpose multiplication of a vector of Dynamic Phasors by a matrix of functionals as the linear combination of the matrix rows.

\begin{equation} %<<<
\left(\cdot\right)^{\Xi^n}_{\left[\mathbb{R}\to\mathbb{C}\right]}:
%
\left\{\begin{array}{rcl}
	\left[\mathbb{R}\to\mathbb{C}\right]^n \times \Xi^{\left(n\times m\right)} &\to& \left[\mathbb{R}\to\mathbb{C}\right]^m \\[5mm]
%
\left[\begin{array}{c} X_1(t) \\[3mm] X_2(t) \\[3mm] X_3(t) \\[3mm] \vdots \\[3mm] X_n(t) \end{array}\right]^\transpose 
\left[\begin{array}{ccccc} 
	\dpL_{11} & \dpL_{12} & \dpL_{13} & \dots  & \dpL_{1m} \\[3mm]
	\dpL_{21} & \dpL_{22} & \dpL_{23} & \dots  & \dpL_{2m} \\[3mm]
	\dpL_{31} & \dpL_{32} & \dpL_{33} & \dots  & \dpL_{3m} \\[3mm]
	\vdots    & \vdots    & \vdots    & \ddots & \vdots    \\[3mm]
	\dpL_{n1} & \dpL_{n2} & \dpL_{n3} & \dots  & \dpL_{nm}
\end{array}\right]& \mapsto &
%
\left[\begin{array}{c}
\sum_{k=1}^m \dpL_{k1}\left[X_k\right] \\[3mm]
\sum_{k=1}^m \dpL_{k2}\left[X_k\right] \\[3mm]
\sum_{k=1}^m \dpL_{k3}\left[X_k\right] \\[3mm]
\vdots                                 \\[3mm]
\sum_{k=1}^m \dpL_{kn}\left[X_k\right]
\end{array}\right]^\transpose
\end{array}\right. \label{eq:dpvec_by_matrix_def}
\end{equation} %>>>

	Thus these definitions adhere to theorem \ref{theo:vector_trasnp}, that is, given a matrix $\mathbf{M}\in\Xi^{(n\times m)}$ and a vector of Dynamic Phasors $\mathbf{x}\in\left[\mathbb{R}\to\mathbb{C}\right]^n$ then $\left(\mathbf{Mx}\right)^\transpose = \mathbf{x}^\transpose\mathbf{M}^\transpose$. These definitions also adhere to the defintion of matrix multiplication \ref{def:matrixbymatrix}, so that given a matrix of Dynamic Phasors $\mathbf{X}\in\left[\mathbb{R}\to\mathbb{C}\right]^{(m\times n)}$ one can define

\begin{equation} \mathbf{XM} = \left[\raisebox{15mm}{} \begin{array}{cccc} \left[\begin{array}{c} \vdots \\[3mm] \mathbf{X}\mathbf{m}_1 \\[3mm] \vdots \end{array}\right] & \left[\begin{array}{c} \vdots \\[3mm] \mathbf{X}\mathbf{m}_2 \\[3mm] \vdots \end{array}\right] & ... & \left[\begin{array}{c} \vdots \\[3mm] \mathbf{X}\mathbf{m}_n \\[3mm] \vdots \end{array}\right]\end{array}\right] \end{equation}

	\noindent where $\mathbf{m}_k$ is the k-th column of $\mathbf{M}$ and the multiplication $\mathbf{Xm}_k$ is that of \eqref{eq:matrix_by_dpvec_def}. One can also build the converse multiplication $\mathbf{MX}$ using \eqref{eq:dpvec_by_matrix_def} and prove that $\left(\mathbf{MX}\right)^\transpose = \mathbf{X}^\transpose\mathbf{M}^\transpose$.

	With these definitions, one can now write admittance equations like $\left[V\right] = \left[\mathbf{Z}\right]\left[I\right]$:

\begin{equation}
\left[\begin{array}{c} V_1(t) \\[3mm] V_2(t) \\[3mm] V_3(t) \\[3mm] \vdots \\[3mm] V_n(t) \end{array}\right] =
%
\left[\begin{array}{ccccc}
	\mathbf{Z}_{11} & \mathbf{Z}_{12} & \mathbf{Z}_{13} & \cdots & \mathbf{Z}_{1m} \\[3mm]
	\mathbf{Z}_{21} & \mathbf{Z}_{22} & \mathbf{Z}_{23} & \cdots & \mathbf{Z}_{2m} \\[3mm]
	\mathbf{Z}_{31} & \mathbf{Z}_{32} & \mathbf{Z}_{33} & \cdots & \mathbf{Z}_{3m} \\[3mm]
	\vdots          & \vdots          & \vdots          & \ddots & \vdots          \\[3mm]
	\mathbf{Z}_{n1} & \mathbf{Z}_{n2} & \mathbf{Z}_{n3} & \cdots & \mathbf{Z}_{nm}
\end{array}\right]
%
\left[\begin{array}{c} I_1(t) \\[3mm] I_2(t) \\[3mm] I_3(t) \\[3mm] \vdots \\[3mm] I_m(t) \end{array}\right] \label{eq:admittance_dpfs}
\end{equation}

	\noindent where the $V_k$ are the Dynamic Phasors of the node voltages of a network, $I_k$ the branch currents and the matrix the equivalent impedance matrix of the network. One can also write the same equation using the admittance version $\left[I\right] = \left[\mathbf{Y}\right]\left[V\right]$; because the invertibility of matrices of DPFs exist, if $\left[\mathbf{Z}\right]$ and $\left[\mathbf{Y}\right]$ of the same circuit are invertible then $\left[\mathbf{Y}\right] = \left[\mathbf{Z}\right]^{-1}$.

	Also, because sum and multiplication of matrices of DPFs are defined, as well as sum and multiplication of matrices of Dynamic Phasors since they are complex functions, then these matrix equations can be manipulated just like complex static phasor equations.

%-------------------------------------------------
\subsection{Real and imaginary components, conjugation and the extended DPF space} \label{subsec:real_imag_dpfs}
	
	Seen as DPFs are motivated by derivatives and will escalate towards impedances in the DP domain, naturally one asks if the notion of real and imaginary components of DPFs are available so that from impedance operators one can define reactance (the imaginary part of impedance), conductance and susceptance as the real and imaginary parts of admittances. We note that by \eqref{eq:gamma_def} we can separate the numbers $\gamma$ into a real and imaginary part:

\begin{equation} \left\{\begin{array}{l} \Re\left[\gamma_k^n(t)\right] = \displaystyle {n\choose k} \left[\sum\limits_{\substack{ c=0 \\ c\in 2\mathbb{N}}}^{n-k} \left(-1\right)^{\frac{c}{2}} B_{\left(n-k,c\right)}\left(\omega,\dot{\omega},\ddot{\omega},...,\omega^{(n-k-c)}\right) \right]\\[10mm] \Im\left[\gamma_k^n(t)\right] = \displaystyle {n\choose k} \left[\sum\limits_{\substack{ c=1 \\ c\in 2\mathbb{N}+1}}^{n-k} \left(-1\right)^{\frac{c-1}{2}} B_{\left(n-k,c\right)}\left(\omega,\dot{\omega},\ddot{\omega},...,\omega^{(n-k-c)}\right) \right]\end{array}\right. ,\label{eq:gamma_def_imre}\end{equation}

	\noindent where $2\mathbb{N}$ is the set of even naturals and $2\mathbb{N}+1$ the set of odd naturals. In order to derive these equations, have in mind ${n\choose k}$ and the Bell Polynomials are real numbers. Thus

\begin{align} \ndpo{n}\left[X\right] &= \sum\limits_{k=0}^n \left\{ \Re\left[\gamma_k^n(t)\right] + j\Im\left[\gamma_k^n(t)\right]\right\} X^{(k)}(t) = \nonumber\\[3mm] &= \sum\limits_{k=0}^n \Re\left[\gamma_k^n(t)\right] X^{(k)}(t) + j\sum\limits_{k=0}^n \Im\left[\gamma_k^n(t)\right] X^{(k)}(t) .\end{align}

	Therefore, the real and imaginary components of $\ndpo{k}$ are definable as

\begin{equation} \ndpo{n} = \Re\left[\ndpo{n}\right] + j \Im\left[\ndpo{n}\right] \left\{\begin{array}{l} \Re\left[\ndpo{n}\right] = \displaystyle \sum_{k=0}^n \Re\left[\gamma_k^n(t)\right] \mathbf{D^k_\mathbb{C}} \\[10mm] \Im\left[\ndpo{n}\right] = \displaystyle \sum_{k=0}^n \Im\left[\gamma_k^n(t)\right] \mathbf{D^k_\mathbb{C}} \end{array}\right. .\label{eq:ndpo_def_imre}\end{equation}

	Naturally, given real and imaginary parts one wonders if the complex conjugation is available by negating the imaginary part. Given the relationship $Y = \dpL\left[X\right]$ between two complex signals $Y$ and $X$, one asks what is the relationship between the complex conjugate signals $\overline{Y}$ and $\overline{X}$. Borrowing from the definition \ref{def:steinmetzoperator_revisited},

\begin{equation}
	\overline{\ndpo{n}\left[X\right]} = \overline{\sum_{k=0}^n \gamma_k^n \left(t\right)\mathbf{D}^k_\mathbb{C}\left[X\right]} = \sum_{k=0}^n \overline{\gamma_k^n \left(t\right)\mathbf{D}^k_\mathbb{C}\left[X\right]} = \sum_{k=0}^n \overline{\gamma_k^n \left(t\right)} \overline{\mathbf{D}^k_\mathbb{C}\left[X\right]}
\end{equation}

	\noindent and using that $\mathbf{D}^k_\mathbb{C}$ and the complex conjucation operation commute,

\begin{equation} \overline{\ndpo{n}\left[X\right]} = \sum_{k=0}^n \overline{\gamma_k^n \left(t\right)} \mathbf{D}^k_\mathbb{C}\left[\overline{X}\right] \end{equation}

	\noindent where

\begin{align}
	\overline{\gamma_k^n(t)} &= \overline{{n\choose k} \left[\sum\limits_{c=0}^{n-k} j^c B_{\left(n-k,c\right)}\left(\omega,\dot{\omega},\ddot{\omega},...,\omega^{(n-k-c)}\right) \right]} = \nonumber\\[3mm] &= {n\choose k} \left[\sum\limits_{c=0}^{n-k} \left(-1\right)^c j^c B_{\left(n-k,c\right)}\left(\omega,\dot{\omega},\ddot{\omega},...,\omega^{(n-k-c)}\right) \right] 
\end{align}

	\noindent again having in mind that ${n\choose k}$ and the Bell Polynomials are real numbers. Thus we can define a complex conjugation operator as follows: $\overline{\ndpo{n}}$, for $n\in\mathbb{N}$, is the operator

\begin{equation}\left\{\begin{array}{l} \overline{\ndpo{n}}\left[X\right] = \displaystyle\sum\limits_{k=0}^n \overline{\gamma_k^n(t)} X^{(k)}(t) \\[5mm] \displaystyle \overline{\gamma_k^n(t)} = {n\choose k} \left[\sum\limits_{c=0}^{n-k} \left(-1\right)^c j^cB_{\left(n-k,c\right)}\left(\omega,\dot{\omega},\ddot{\omega},...,\omega^{(n-k-c)}\right) \right]\end{array}\right. ,\label{eq:gamma_def_conj}\end{equation}

	\noindent and one can repeat all theorems up until here to prove that the conjugate operators $\overline{\ndpo{n}}$ are also bijective and form a group, a ring, a vector space, such that they can be summed, multiplied and combined in the same fashion as the $\ndpo{n}$. By this definition and these results, the conjugation operation becomes distributive:

\begin{equation} \overline{\ndpo{n}\left[X\right]} = \left(\overline{\ndpo{n}}\right)\left[\overline{X}\right] \label{eq:ndpo_conj}\end{equation}

	\noindent for any $X\in\left[\mathbb{R}\to\mathbb{C}\right]$.

\begin{example}[Conjugate, real and imaginary parts of $\dpo,\ \ndpo{2}$ and $\ndpo{3}$] %<<<

	From \eqref{eq:gamma_def_conj}, we note that the signal of $\left(-1\right)^c$ on the definition of $\overline{\gamma}$ is inverted if $c$ is odd and maintained if $c$ is even, allowing for easily obtaining the conjugates of $\ndpo{k}$ once the form of this operator is known. For instance, from the definition \eqref{eq:steinmetz_1storder} of $\dpo$,

\begin{equation}
	\dpo = \mathbf{D}^1 + j\omega(t)\mathbf{I} \left\{\begin{array}{l} \overline{\dpo} = \mathbf{D}^1 - j\omega(t)\mathbf{I} \\[3mm] \Re\left[\dpo\right] = \mathbf{D^1} \\[3mm] \Im\left[\dpo\right] = \omega(t)\mathbf{I} \end{array}\right. 
\end{equation}

	\noindent and from the definition \eqref{eq:steinmetz_2ndorder} of $\ndpo{2}$

\begin{equation}
	\ndpo{2} = \mathbf{D}^2 + 2j\omega(t)\mathbf{D}^1 + \left[-\omega^2 + j\dot{\omega}(t)\right]\mathbf{I} \left\{\begin{array}{l} \overline{\ndpo{2}} = \mathbf{D}^2 - 2j\omega(t)\mathbf{D}^1 + \left[-\omega^2 - j\dot{\omega}(t)\right]\mathbf{I} \\[3mm] \Re\left[\ndpo{2}\right] = \mathbf{D}^2  - \omega^2\mathbf{I} \\[3mm] \Im\left[\ndpo{2}\right] = 2\omega(t)\mathbf{D}^1 + \dot{\omega}(t) \mathbf{I}\end{array}\right. .
\end{equation}

	Finally, from the definition \eqref{eq:steinmetz_3rdorder} of $\ndpo{3}$,

\begin{gather}
	\ndpo{3} = \mathbf{D}^3 + 3j \omega(t) \mathbf{D}^2 + \left[ 3j \dot{\omega}(t) - 3 \omega(t)^2 \right] \mathbf{D}^1 + \left[ j \ddot{\omega}(t) - 3 \omega(t) \dot{\omega}(t) - j \omega(t)^3 \right]\mathbf{I} \\[15mm]
%
	\left\{\begin{array}{l}
		\overline{\ndpo{3}} = \mathbf{D}^3 - 3j \omega(t) \mathbf{D}^2 + \left[ -3j \dot{\omega}(t) - 3 \omega(t)^2 \right] \mathbf{D}^1 + \left[ -j \ddot{\omega}(t) - 3 \omega(t) \dot{\omega}(t) + j \omega(t)^3 \right]\mathbf{I} \\[3mm]
%
		\Re\left[\ndpo{3}\right] = \mathbf{D}^3 - 3 \omega(t)^2 \mathbf{D}^1 - 3 \omega(t) \dot{\omega}(t) \mathbf{I} \\[3mm]
%
		\Im\left[\ndpo{3}\right] = 3\omega(t) \mathbf{D}^2 + 3 \dot{\omega}(t) \mathbf{D}^1 + \left[ \ddot{\omega}(t) - \omega(t)^3 \right]\mathbf{I}
	\end{array}\right.
\end{gather}
\examplebar
\end{example} %>>>

	Notably, like the real and imaginary parts of a complex number are real numbers themselves, the real and imaginary parts of an operator $\ndpo{n}$ can be restricted as transformations of real functions, that is,

\begin{equation} \Re\left(\ndpo{n}\right),\Im\left(\ndpo{n}\right)\in\left[\left[\mathbb{R}\to\mathbb{R}\right]\to\left[\mathbb{R}\to\mathbb{R}\right]\right] . \end{equation}

	This means that much like a complex number is isomorphic to $\mathbb{R}^2$ — meaning any two real numbers $a,b$ define a complex number $a + jb$ — Dynamic Phasor Functionals are isomorphic to the space $\left[\mathbb{R}\to\mathbb{R}\right]^2$, that is, given two transformations of real functions $\mathbf{A}\left[f\right], \mathbf{B}\left[f\right]$, these functions define a transformation in complex signals as

\begin{equation} \mathbf{T}\left[f\right] = \mathbf{A}\left[f\right] + j\mathbf{B}\left[f\right] \end{equation}

	\noindent and if $\mathbf{A}$ and $\mathbf{B}$ have the forms of \eqref{eq:ndpo_def_imre} then $\mathbf{T}$ is equal to $\ndpo{n}$. Furthermore, using \eqref{eq:ndpo_conj}, we can propose the conjugation operator for any $\dpL\in\dpS$: because any such operator is a ratio of polynomials of $\dpo$,

\begin{equation} \dpL = \dfrac{N\left(\dpo\right)}{D\left(\dpo\right)} = \dfrac{\displaystyle \sum_{k=0}^n \alpha_k \ndpo{k}}{\displaystyle\sum_{k=0}^d \beta_k \ndpo{k}} \end{equation}

	\noindent then $\dpL\left[X\right]$ is the signal that satisfies

\begin{equation} \sum_{k=0}^n \alpha_k \ndpo{k}\left[X\right] = \sum_{k=0}^d \beta_k \ndpo{k}\left[\dpL\left[X\right]\right] .\end{equation}

	Conjugating this entire equation,

\begin{gather}
	\overline{\sum_{k=0}^n \alpha_k \ndpo{k}\left[X\right]} = \overline{ \sum_{k=0}^d \beta_k \ndpo{k}\left[\dpL\left[X\right]\right]}  \nonumber\\[3mm]
	\sum_{k=0}^n \overline{\alpha_k} \overline{\ndpo{k}\left[X\right]} = \sum_{k=0}^d \overline{\beta_k} \overline{\ndpo{k}\left[\dpL\left[X\right]\right]} 
\end{gather}

	\noindent and by the definition \eqref{eq:gamma_def_conj} this yields

\begin{gather}
	\sum_{k=0}^n \overline{\alpha_k} \overline{\ndpo{k}\left[X\right]} = \sum_{k=0}^d \overline{\beta_k} \overline{\ndpo{k}\left[\dpL\left[X\right]\right]} \nonumber\\[3mm]
	\sum_{k=0}^n \overline{\alpha_k} \overline{\ndpo{k}}\left[\overline{X}\right] = \sum_{k=0}^d \overline{\beta_k} \overline{\ndpo{k}}\left[\overline{\dpL\left[X\right]}\right] \nonumber\\[3mm]
\end{gather}

	\noindent so the signal $\overline{\dpL\left[X\right]}$ satisfies

\begin{equation} \overline{\dpL\left[X\right]} = \left[\dfrac{\displaystyle \sum_{k=0}^n \overline{\alpha_k}\ \overline{\ndpo{k}}}{\displaystyle\sum_{k=0}^d \overline{\beta_k}\ \overline{\ndpo{k}}}\right] \left[\overline{X}\right] .\label{eq:dpl_conj_1}\end{equation}

	Borrowing from complex polynomials, the conjugate of a polynomial is often denoted as

\begin{equation} P(z) = \sum_{k=0}^n \alpha_k z^k \Leftrightarrow \overline{P}(z) = \sum_{k=0}^n \overline{\alpha_k} z^k \end{equation}

	\noindent so that we can use the same notation for polynomials of $\dpo$ and \eqref{eq:dpl_conj_1} can be written as

\begin{equation} \dpL\left[X\right] = \left[\dfrac{N\left[\dpo\right]}{D\left[\dpo\right]}\right]\left[X\right] \Leftrightarrow \overline{\dpL\left[X\right]} = \left[\dfrac{\overline{N}\left[\overline{\dpo}\right]}{\overline{D}\left[\overline{\dpo}\right]}\right]\left[X\right] ,\label{eq:dpl_conj_2}\end{equation}

	\noindent which induces a definition of $\overline{\dpL}$ as

\begin{equation} \dpL = \dfrac{N\left[\dpo\right]}{D\left[\dpo\right]} \Leftrightarrow \overline{\dpL} = \dfrac{\overline{N}\left[\dpo\right]}{\overline{D}\left[\dpo\right]} ,\label{eq:dpl_conj_3}\end{equation}

	\noindent so that this conjugation definition is also commutative, that is, $\overline{\dpL\left[X\right]} = \overline{\dpL}\left[\overline{X}\right]$. Therefore, we can also define the space of conjugate functionals 

\begin{equation} \overline{\dpS} \vcentcolon = \left\{\overline{\dpL}: \dpL\in\dpS\right\} \end{equation}

	\noindent and one can easily prove $\overline{\dpS}$ is endowed with all the properties of $\dpS$; indeed, if this chapter started by defining $\gamma_k^n$ as its conjugate, not much would change as the resulting functionals would still be invertible (through a very small adaptation of theorem \ref{theo:bijection}) and would form an abelian group, a commutative polynomial ring, and a matrix space (by repeating all theorems from theorem \ref{theo:abelian} through theorem \ref{theo:dpf_matrices_exist}). Therefore one can extend the definition \ref{def:dpft_space} of $\dpS$ to an extended space $\dpS_\mathbb{C}$ so that this space is invariant under conjugation.

\begin{definition}[Extended Dynamic Phasor Functional Space]\label{def:extended_dpft_space}
	The \textbf{Extended Dynamic Phasor Functional Space} is the set $\dpS_\mathbb{C}\ = \dpS \cup \overline{\dpS}$, that is, the set of all linear combinations of polynomials of $\dpo$, inverse operators of those polynomials, and all their conjugate operators.
\end{definition}

	Thence, the Extended DPF Space $\dpS_\mathbb{C}$ is also an abelian group, a polynomial commutative ring, a field and a vector space over itself — so that the definitions of linear combinations, polynomials as in \eqref{eq:dpfs_poly_def} are kept in this space. Further, matrices in $\dpS_\mathbb{C}$ are also well defined as are their multiplications by signals (as in \eqref{eq:matrix_by_dpvec_def} and \eqref{eq:dpvec_by_matrix_def}) and the multiplication by scalars and operators and matrices of operators. Thus one can finally define real and imaginary parts in $\dpS_\mathbb{C}$ as

\begin{equation} \Re\left(\dpL\right) = \dfrac{\dpL + \overline{\dpL}}{2},\ \Im\left(\dpL\right) = \dfrac{\dpL - \overline{\dpL}}{2j} \end{equation}

	\noindent and it is trivial to see that not only this definition is compatible with the real and imaginary parts of $\ndpo{k}$ as in \eqref{eq:ndpo_def_imre}, but also that the real and imaginary operations are closed in $\dpS_\mathbb{C}$.

%------------------------------------------------
\subsection{A topology of Dynamic Phasor Functionals}\label{subsec:topology_dpfs} %<<<2

	Seen as the space of functionals $\dpS$ generalizes the idea of operators in Dynamic Phasor space, as well as impedances for voltage and current signals, we want to define idealized versions of impedance models where the impedance tends to a short-circuit (the norm of the associated operator tends to zero) or to an open circuit (the norm tends to infinity). For instance, idealized transistor and operational amplifier models use small (ideally zero) impedances and very high (ideally infinite) gains.

	In the static phasor context, the limits associated with infinity and zero are well-defined and well-behaved, since complex analysis defines limits of complex functions. This stems from the fact that the norm of complex numbers — the absolute value — is defined and complete in its space. However, to define limits of norms of operators in $\dpS_\mathbb{C}$ one must first define a topology in this space, that is, define a norm of functionals in $\dpS_\mathbb{C}$ which induces a notion of distances.

	As shown in the chapter \ref{chapter:linear_systems} on the theory of linear systems, specifically definition \ref{def:mapping_norm}, the norm of a map is induced by the ratio of the norms of the output and the input space. This means that in order to define a norm of $\dpS$, we must first define a norm for $\left[\mathbb{R}\to\mathbb{C}\right]$. As discussed before in subsection \ref{subsec:characteristics_l1}, there does not exist a total inner product in this space, thus a norm for the entire space is unfeasible; however, Functional Analysis does offer norms for specific subspaces. For instance, for $\dpo\left[X\right]$ to exist for some signal $X(t)$ the signal must be at least differentiable, that is, belong to $C^1$. If this is the case, the output belongs to $C^0$. Gladly there exist a usual norm of $C^n$ as defined in \cite{rudin1991functional} and shown in \eqref{eq:usual_norm_cn}.

\begin{equation} \left\{\begin{array}{l} \left\lVert f\right\rVert_{C^0} = \sup\limits_{t\in\mathbb{R}} \left\lvert f(t)\right\rvert \\[3mm] \left\lVert f\right\rVert_{C^1} = \sup\limits_{t\in\mathbb{R}} \left\lvert f'(t)\right\rvert + \sup\limits_{t\in\mathbb{R}} \left\lvert f(t)\right\rvert \\[3mm] \left\lVert f\right\rVert_{C^2} = \sup\limits_{t\in\mathbb{R}} \left\lvert f''(t)\right\rvert + \sup\limits_{t\in\mathbb{R}} \left\lvert f'(t)\right\rvert + \sup\limits_{t\in\mathbb{R}} \left\lvert f(t)\right\rvert  \\[3mm] \hspace{2cm} \vdots \\[3mm] \displaystyle\left\lVert f\right\rVert_{C^n} = \sum_{k=0}^n \sup\limits_{t\in\mathbb{R}} \left\lvert f^{(k)}(t)\right\rvert \end{array}\right. . \label{eq:usual_norm_cn}\end{equation}

	For the space $\left[\mathbb{R}\to\mathbb{C}\right]$, let us adopt a similar but adjusted norm which we call the ``Dynamic Phasor Norm'' where the supremums are multiplied by the norms of the $\gamma_k^n$.

\begin{definition}[Dynamic Phasor Norm]\label{def:dpnorm} %<<<

	Consider $n\in\mathbb{N},\ \omega(t)\in C^n\left(\left[\mathbb{R}\to\mathbb{R}\right]\right)$ an apparent frequency signal, and $X$ a class $n$ smooth Dynamic Phasor signal, that is, $X\in C^n\left(\left[\mathbb{R}\to\mathbb{C}\right]\right)$. Then the \textbf{Dynamic Phasor Norm (or simply DP norm)} of $C^n$, denoted $\left\lVert \cdot \right\rVert_{D^n}$, is defined as

\begin{equation} \left\lVert \cdot\right\rVert_{{D}^n}:\left\{\begin{array}{rcl} \left[\mathbb{R}\to\mathbb{C}\right] &\to& \mathbb{R}^+ \\[5mm] X(t) &\mapsto& \displaystyle\sum\limits_{k=0}^n \tau_k^n \sup\limits_{t\in\mathbb{R}} \left\lvert X^{(k)}(t)\right\rvert \end{array}\right. \label{eq:def_dpnorm}\end{equation}

	\noindent where $\tau_k^n$ is defined as the norm of $\gamma_k^n$, these being the coefficients of the Dynamic Phasor Transform as defined in \eqref{eq:gamma_def}:

\begin{equation} \mathbb{R}^+ \ni \tau_k^n = \left\lVert \gamma_k^n\right\rVert_{C^0} = \sup\limits_{t\in\mathbb{R}} \left\lvert\gamma_k^n(t) \right\rvert = \sup\limits_{t\in\mathbb{R}} \left\lvert {n\choose k} \left[\sum\limits_{c=0}^{n-k} j^cB_{\left(n-k,c\right)}\left(\omega,\dot{\omega},\ddot{\omega},...,\omega^{(n-k-c)}\right) \right] \right\rvert \end{equation}

	\noindent that is,

\begin{equation}
	\left\{\begin{array}{l}
		\left\lVert X\right\rVert_{D^0} = \overbrace{1}^{\left\lVert \gamma_0^0\right\rVert_{C^0}}\sup\limits_{t\in\mathbb{R}} \left\lvert X(t)\right\rvert \\[5mm]
		\left\lVert X\right\rVert_{D^1} = \overbrace{1}^{\left\lVert \gamma_1^1\right\rVert_{C^0}}\sup\limits_{t\in\mathbb{R}} \left\lvert \dot{X}(t)\right\rvert + \overbrace{\sup\limits_{t\in\mathbb{R}} \left\lvert\omega\right\rvert}^{\left\lVert \gamma_0^1\right\rVert_{C^0}} \sup\limits_{t\in\mathbb{R}} \left\lvert X(t)\right\rvert \\[5mm]
		\left\lVert X\right\rVert_{D^2} = \overbrace{1}^{\left\lVert \gamma_2^2\right\rVert_{C^0}} \sup\limits_{t\in\mathbb{R}}\left\lvert \ddot{X}(t)\right\rvert + \overbrace{2\sup\limits_{t\in\mathbb{R}}\left\lvert \omega \right\rvert}^{\left\lVert \gamma_1^2\right\rVert_{C^0}} \sup\limits_{t\in\mathbb{R}}\left\lvert \dot{X}(t)\right\rvert + \overbrace{\sup\limits_{t\in\mathbb{R}} \left\lvert -\omega^2 + j\dot{\omega}\right\rvert}^{\left\lVert \gamma_0^2\right\rVert_{C^0}} \sup\limits_{t\in\mathbb{R}}\left\lvert X(t)\right\rvert \\[5mm]
		\hspace{2cm} \vdots
	\end{array}\right.
\end{equation}

\end{definition}%>>>
\begin{definitionremark} The $\tau_k^n$ are positive reals and bounded because the $B_{(n,k)}$ are polynomials and $\omega(t)$ together with its $n-1$ derivatives are all bounded since $\omega(t)$ is supposed $C^n$. \end{definitionremark}

	Proving that the DP norm of \eqref{eq:def_dpnorm} indeed satisfies the requisites of a norm (see definition \ref{def:norm_vecspaece} for these requisites) is easy to prove: since any positive definite linear combination of norms is itself a norm, and we prove that the DP norm of class $n$ is basically a linear combination of the norms of $C^0,C^1,\dots,C^n$ — and this fact is proven by scalonating the formulas:

\begin{equation}
	\left\{\begin{array}{l}
		\left\lVert X\right\rVert_{D^0} = \sup\limits_{t\in\mathbb{R}} \left\lvert X(t)\right\rvert = \left\lVert X\right\rVert_{C^0}\\[5mm]
		\left\lVert X\right\rVert_{D^1} = \left\lVert X\right\rVert_{C^1} + \left(\sup\limits_{t\in\mathbb{R}} \left\lvert \omega(t)\right\rVert  - 1\right)\left\lVert X\right\rVert_{C^0}\\[5mm]
		\left\lVert X\right\rVert_{D^2} = \left\lVert X\right\rVert_{C^2} + \left(2\sup\limits_{t\in\mathbb{R}} \left\lvert \omega(t)\right\rVert  - 1\right)\left\lVert X\right\rVert_{C^1} + \left(\sup\limits_{t\in\mathbb{R}} \left\lvert -\omega^2 + j\dot{\omega}\right\rvert - 2\sup\limits_{t\in\mathbb{R}} \left\lvert \omega(t)\right\rVert  - 1\right)\left\lVert X\right\rVert_{C^0}\\[5mm]
		\hspace{2cm} \vdots
	\end{array}\right.
\end{equation}

	\noindent and a general formula is

\begin{equation}\left\{\begin{array}{l} \left\lVert X\right\rVert_{D^0} = \left\lVert X \right\rVert_{C^0} \\[5mm] \displaystyle\left\lVert X\right\rVert_{D^n} = \left\lVert X \right\rVert_{C^n} + \sum\limits_{k=0}^{n-1}\left(\tau_k^n - \sum\limits_{i=k+1}^{n} \tau_i^n\right) \left\lVert X \right\rVert_{C^k} \end{array}\right. .\end{equation}

	Having defined a norm for Dynamic Phasors, a norm for the DPFs is induced as per definition \ref{def:mapping_norm}. We first start with the first-order functional $\ndpo{1}$, and we use the map definition \eqref{eq:norm_map_def_2}:

\begin{equation} \left\lVert \dpo\right\rVert = \sup\left\{ \dfrac{\left\lVert \dpo\left[X\right]\right\rVert_{C^0}}{\left\lVert X\right\rVert_{C^1}}:\ X\in C^1\left(\left[\mathbb{R}\to\mathbb{C}\right]\right)\right\} .\end{equation}

	 We use the properties of the supremum to compute this number, namely, that for any two $f,g$ defined in some space $D$, the supremum of the image of $f+g$ is smaller than the sum of the supremums of the individual images, that is, $\sup\left( (f+g)(D)\right) \leq \sup(f(D)) + \sup(g(D))$:

\begin{equation}
	\dfrac{\left\lVert \dpo\left[X\right]\right\rVert_{C^0}}{\left\lVert X\right\rVert_{C^1}} = \dfrac{\sup\limits_{t\in\mathbb{R}}\left\lvert \dot{X} + j\omega X\right\rvert}{\sup\limits_{t\in\mathbb{R}}\left\lvert \dot{X} \right\rvert + \sup\limits_{t\in\mathbb{R}}\left\lvert \omega\right\rvert\sup\limits_{t\in\mathbb{R}}\left\lvert X\right\rvert} \leq \dfrac{\sup\limits_{t\in\mathbb{R}}\left\lvert \dot{X}\right\rvert + \sup\limits_{t\in\mathbb{R}} \left\lvert j\omega X\right\rvert}{\sup\limits_{t\in\mathbb{R}}\left\lvert \dot{X} \right\rvert + \sup\limits_{t\in\mathbb{R}}\left\lvert \omega\right\rvert\sup\limits_{t\in\mathbb{R}}\left\lvert X\right\rvert}. \label{eq:dpo_norm_ratio}
\end{equation}

	Now we use that the supremum of the product set $AB = \left\{ab: a\in A,\ b\in B\right\}$ is the product of the supremums:

\begin{equation}
	\dfrac{\left\lVert \dpo\left[X\right]\right\rVert_{C^0}}{\left\lVert X\right\rVert_{C^1}} \leq \dfrac{\sup\limits_{t\in\mathbb{R}}\left\lvert \dot{X}\right\rvert + \sup\limits_{t\in\mathbb{R}} \left\lvert \omega \right\rvert\sup\limits_{t\in\mathbb{R}}\left\lvert X\right\rvert}{\sup\limits_{t\in\mathbb{R}}\left\lvert \dot{X} \right\rvert + \sup\limits_{t\in\mathbb{R}}\left\lvert \omega\right\rvert\sup\left\lvert X\right\rvert} = 1 \Rightarrow \left\lVert \dpo\right\rVert \leq 1
\end{equation}

	However, it is simple to see that the last ratio of \eqref{eq:dpo_norm_ratio} attains unity for any constant non-zero $X(t)$; therefore, $\left\lVert \dpo\right\rVert = 1$. For $\ndpo{2}$, 

\begin{equation} \left\lVert \ndpo{2}\right\rVert = \sup\left\{ \dfrac{\left\lVert \ndpo{2}\left[X\right]\right\rVert_{C^0}}{\left\lVert X\right\rVert_{C^2}}:\ X\in C^2\left(\left[\mathbb{R}\to\mathbb{C}\right]\right)\right\} .\end{equation}

	But

\begin{equation} \dfrac{\left\lVert \ndpo{2}\left[X\right]\right\rVert_{C^0}}{\left\lVert X\right\rVert_{C^2}} = \xfrac{5mm}{3mm}{\sup\limits_{t\in\mathbb{R}}\left\lvert \ddot{X} + 2j\omega \dot{X} + \left[-\omega^2 + j\dot{\omega}\right]X \right\rvert}{ \sup\limits_{t\in\mathbb{R}}\left\lvert \ddot{X} \right\rvert + 2\sup\limits_{t\in\mathbb{R}}\left\lvert \omega \right\rvert \sup\limits_{t\in\mathbb{R}}\left\lvert \dot{X} \right\rvert + \sup\limits_{t\in\mathbb{R}} \left\lvert -\omega^2 + j\dot{\omega}\right\rvert \sup\limits_{t\in\mathbb{R}}\left\lvert X\right\rvert } \end{equation}

	\noindent and clearly one can see that using the same infimum properties one arrives at the same conclusion that this ratio is at most one, and that it achieves unity if $X(t)$ is nonzero and constant, thus $\left\lVert\ndpo{2}\right\rVert = 1$. Therefore we can prove that $\left\lVert\ndpo{n}\right\rVert = 1$ for any order $n$.

\begin{theorem}[Dynamic Phasor Functionals have single norm]\label{theo:dpfs_unitary} %<<<
	The functional $\ndpo{n}$ has a unitary norm under the DP norm of definition \ref{def:dpnorm}, that is,

\begin{equation} \left\lVert \ndpo{n}\right\rVert = \sup\left\{ \dfrac{\left\lVert \ndpo{n}\left[X\right]\right\rVert_{C^0}}{\left\lVert X\right\rVert_{C^n}}:\ X\in C^n\left(\left[\mathbb{R}\to\mathbb{C}\right]\right)\right\} = 1 \end{equation}
\end{theorem}
\textbf{Proof:} computing the ratio,

\begin{equation} \dfrac{\left\lVert \ndpo{n}\left[X\right]\right\rVert_{D^0}}{\left\lVert X\right\rVert_{D^n}} = \xfrac{5mm}{3mm}{\displaystyle \sup\limits_{t\in\mathbb{R}}\left\lvert\sum\limits_{k=0}^n \gamma_k^n (t)X^{(k)}(t)\right\rvert }{\displaystyle \sum\limits_{k=0}^n \tau_k^n \sup\limits_{t\in\mathbb{R}}\left\lvert X^{(k)}(t)\right\rvert } .\end{equation}

	By the summation property of the supremum,

\begin{equation} \xfrac{5mm}{3mm}{\displaystyle \sup\limits_{t\in\mathbb{R}}\left\lvert\sum\limits_{k=0}^n \gamma_k^n (t) X^{(k)}(t)\right\rvert }{\displaystyle \sum\limits_{k=0}^n \tau_k^n \sup\limits_{t\in\mathbb{R}}\left\lvert X^{(k)}(t)\right\rvert } \leq \xfrac{5mm}{3mm}{\displaystyle \sum\limits_{k=0}^n \sup\limits_{t\in\mathbb{R}}\left\lvert\gamma_k^n (t) X^{(k)}(t)\right\rvert }{\displaystyle \sum\limits_{k=0}^n \tau_k^n \sup\limits_{t\in\mathbb{R}}\left\lvert X^{(k)}(t)\right\rvert } \end{equation}

	\noindent and by the multiplication property,

\begin{equation} \xfrac{5mm}{3mm}{\displaystyle \sum\limits_{k=0}^n \sup\limits_{t\in\mathbb{R}}\left\lvert\gamma_k^n(t) X^{(k)}(t)\right\rvert }{\displaystyle \sum\limits_{k=0}^n \tau_k^n \sup\limits_{t\in\mathbb{R}}\left\lvert X^{(k)}(t)\right\rvert } = \xfrac{5mm}{3mm}{\displaystyle \sum\limits_{k=0}^n \sup\limits_{t\in\mathbb{R}}\left\lvert\gamma_k^n(t) \right\rvert \sup\limits_{t\in\mathbb{R}}\left\lvert X^{(k)}\right\rvert }{\displaystyle \sum\limits_{k=0}^n \tau_k^n \sup\limits_{t\in\mathbb{R}}\left\lvert X^{(k)}(t)\right\rvert } = 1. \end{equation}

	\noindent therefore

\begin{equation} \dfrac{\left\lVert \ndpo{n}\left[X\right]\right\rVert_{D^0}}{\left\lVert X\right\rVert_{D^n}} \leq 1 .\end{equation}

	But if $X(t)$ is a constant non-null signal, then this ratio achieves the unity:

\begin{equation} \dfrac{\left\lVert \ndpo{n}\left[X\right]\right\rVert_{D^0}}{\left\lVert X\right\rVert_{D^n}} = \xfrac{5mm}{3mm}{\displaystyle \sup\limits_{t\in\mathbb{R}}\left\lvert\sum\limits_{k=0}^n \gamma_k^n (t)X^{(k)}(t)\right\rvert }{\displaystyle \sum\limits_{k=0}^n \tau_k^n \sup\limits_{t\in\mathbb{R}}\left\lvert X^{(k)}(t)\right\rvert } = \xfrac{5mm}{3mm}{\displaystyle \sup\limits_{t\in\mathbb{R}}\left\lvert \gamma_0^n (t)X(t)\right\rvert }{\displaystyle \tau_0^n \sup\limits_{t\in\mathbb{R}}\left\lvert X(t)\right\rvert } = \xfrac{5mm}{3mm}{\displaystyle \sup\limits_{t\in\mathbb{R}}\left\lvert \gamma_0^n (t)\right\rvert \sup\limits_{t\in\mathbb{R}}\left\lvert X(t)\right\rvert }{\displaystyle \tau_0^n \sup\limits_{t\in\mathbb{R}}\left\lvert X(t)\right\rvert } = 1\end{equation}

	\noindent thus $\left\lVert \ndpo{n}\right\rVert = 1$. \hfill$\blacksquare$
\vspace{3mm}
\hrule
\vspace{3mm}
%>>>

	It is simple to see that the direct computation of the norms of polynomials of $\dpo$ becomes difficult; however, due to the triangle inequality and the absolute homogeneity of norms (see definition \ref{def:norm_vecspaece}), for any $\mathbf{P}\left(\dpo\right) \in \mathbb{C}\left[\dpo\right]$ one has

\begin{equation} \left\lVert \mathbf{P}\right\rVert = \left\lVert \sum_{k=0}^n \alpha_k\ndpo{k}\right\rVert \leq \sum_{k=0}^n \left\lVert \alpha_k\ndpo{k}\right\rVert = \sum_{k=0}^n \left\lvert \alpha_k\right\rvert \left\lVert \ndpo{k}\right\rVert = \sum_{k=0}^n \left\lvert \alpha_k\right\rvert \end{equation}

	\noindent meaning any polynomial of $\dpo$ is a bounded linear transform. For the inverse operators, due to the sub-multiplicativity of norms,

\begin{equation} 1 = \left\lVert \dpo \ndpo{-1}\right\rVert \leq \left\lVert \dpo \right\rVert \left\lVert\ndpo{-1}\right\rVert \Leftrightarrow \left\lVert\ndpo{-1}\right\rVert \geq 1 .\end{equation}

	Analogously, the inverse of a polynomial $\mathbf{P}$ is such that

\begin{equation} \left\lVert \dfrac{\mathbf{I}}{\mathbf{P}}\right\rVert = \raisebox{-3mm}{$ \left\lVert \raisebox{3mm}{$ \dfrac{\mathbf{I}}{\displaystyle\sum_{k=0}^n \alpha_k\ndpo{k}} $}\right\rVert $} \geq \dfrac{1}{\displaystyle\sum_{k=0}^n \left\lvert \alpha_k\right\rvert} . \label{eq:inverse_poly_norm} \end{equation}

	Therefore, for some arbitrary $\lambda\in\dpS$, the norm is well-defined, and these proofs define a topology for the entire $\dpS$. This allows us to define interesting institutions in this space, for instance, limits: if $\lambda = \mathbf{N}\left(\dpo\right)/\mathbf{D}\left(\dpo\right)$, then $\left\lVert \dpL\right\rVert$ tends to zero if the coefficients of the numerator polynomial are arbitraryly small or those of the denominator are arbitrarily large; conversely, the norm tends to infinity if the coefficients of the denominator are arbitrarily small or those of the numerator are arbitrarily large. This allows us to have idealized models of impedances that are very low (almost short-circuits) or very high (almost open circuits).

	It will be shown later (see theorem \ref{theo:bibo_mutfs} at page \pageref{theo:bibo_mutfs}) that an arbitrary $\lambda = \mathbf{N}\left(\dpo\right)/\mathbf{D}\left(\dpo\right)$ does indeed have a finite norm (thus being a bounded operator) if and only if it is proper (the degree of the denominator is equal or higher than the degree of the numerator) and the roots of the denominator are in the open left half plane.

	It is also simple to see that all the proofs shown are maintained for the conjugate operators $\overline{\ndpo{k}}$, thus all results also extend to the conjugate space $\overline{\dpS}$; therefore, the Dynamic Phasor norm also induces a topology on the Extended Dynamic Phasor Functional space $\dpS_\mathbb{C}$.

%------------------------------------------------
\section{Circuit modelling techniques using Dynamic Phasors and the DPF} %<<<1

	Having stated and proven that the DPFs $\dpS$ form very powerful algebraic structures that allow us to operate them in convenient and familiar manners, we can explore these structures to prove that the customary circuit modelling techniques find counterparts in the Dynamic Phasor domain by means of operating DPFs. We start with Kirchoff's Laws.

\begin{theorem}[Kirchoff's Current Law in the Dynamic Phasor domain] \label{theo:kirchoff_current}
Let $i_p(t)$, $p = 1,...,q$ be the generalized sinusoidal currents of a certain network meeting at a node, $I_p(t)$ their dynamic phasors. Then

\begin{equation} \sum\limits_{p=1}^q I_p(t) = 0 \end{equation}

\end{theorem}
\noindent \textbf{Proof.} By Kirchoff's Current Law in time domain, $\sum i_p(t) = 0$. Applying the dynamic phasor transform and using its linearity yields $\sum I_p(t) = 0$. \hfill$\blacksquare$
	
\begin{theorem}[Kirchoff's Voltage Law in the Dynamic Phasor domain] \label{theo:kirchoff_voltage}
Let $v_p(t)$, $p = 1,...,q$ be the generalized sinusoidal voltages of a certain network around a certain closed loop, $V_p(t)$ their dynamic phasors. Then

\begin{equation} \sum\limits_{p=1}^q V_p(t) = 0 \end{equation}

\end{theorem}
\noindent \textbf{Proof:} akin to theorem \ref{theo:kirchoff_current}. \hfill$\blacksquare$

	While theorems \ref{theo:kirchoff_current} and \ref{theo:kirchoff_voltage} seem immediate, they have some depth to them. First, we note that the blatant statement of \textit{generalized sinusoids} serves to differ these theorems from their more restrict static sinusoid couterparts \ref{theo:kirchoff_current_phasor} and \ref{theo:kirchoff_voltage_phasor}.

	Further, it may seem like theorems \ref{theo:kirchoff_current} and \ref{theo:kirchoff_voltage} are mere rewritings of the previously proven \ref{theo:kirchoff_current_1p} and \ref{theo:kirchoff_voltage_1p}, when they are more general for while \ref{theo:kirchoff_current_1p} and \ref{theo:kirchoff_voltage_1p} consider that the currents and voltages considered all are defined at the same apparent frequency, after the developments of chapter \ref{chapter:choice_apparent_frequency} the new versions \ref{theo:kirchoff_current} and \ref{theo:kirchoff_voltage} are able to deal with generalized sinusoids defined at different frequencies.

	Indeed, the new versions do not weave considerations about the apparent frequencies upon which the currents meeting at a node (or voltages around a certain loop) are defined. Here, we are supposing that even if each individual current (voltage) is defined at its own apparent frequency, these frequency signals are all mutually equivalent (absolutely integrable as per definition \ref{def:equivalent_freqs}), hence by theorem \ref{theorem:sols_are_nonst} all currents (voltages) can be modelled in a common $\omega(t)$ that is equivalent to all frequency signals, and this common apparent frequency is adopted for the DPT that transforms the time signals into phasors.

%-------------------------------------------------
\subsection{Dynamic Impedances} %<<<2

	It is immediate from the definition of $\dpo$ that the linear element impedances can be written as

\begin{equation}\left\{\begin{array}{l} v(t) = L\dot{i}(t) \Leftrightarrow V(t) = L \dpo \left[I\right] \text{ (Linear inductor)}\\[3mm] i(t) = C\dot{v}(t) \Leftrightarrow I(t) = C \dpo \left[V\right]  \text{ (Linear capacitor)}\\[3mm] v(t) = Ri(t) \Leftrightarrow V(t) = RI(t) \text{ (Linear resistor)}\end{array} \right. \label{sys:dpo_impedances}\end{equation}

	\noindent which already looks a lot like the Laplace relationships \eqref{sys:laplace_impedances}. Moreover, immediately one notices that these equations are of the form $V(t) = \mathbf{Z}\left[I\right]$ where $\mathbf{Z}$ is an operator that relates the Dynamic Phasor of voltage $V(t)$ and the Dynamic Phasor of the current $I(t)$, a relationship highly suggestive of the idea of impedance:

\begin{equation}\left\{\begin{array}{l} \mathbf{Z}_L = L \dpo \text{ (Linear inductor)}\\[3mm] \mathbf{Z}_C = \dfrac{\mathbf{I}}{\dpo C} \text{ (Linear capacitor)}\\[3mm] \mathbf{Z}_R = R\mathbf{I} \text{ (Linear resistor)}\end{array} \right. \label{sys:dpo_impedances_formula}\end{equation}

	It also becomes clear that as these impedances are combined, inverted and operated in more complex circuits they become ratios of polynomials of $\dpo$, thus elements of $\Xi$. In a general context, in a linear bipole, the time-domain relationship between the voltage and current is given by

\begin{equation} \sum_{k=0}^n a_k v^{(k)}(t) = \sum_{k=0}^d b_k i^{(k)}(t) \label{eq:impedance_time_general}\end{equation}

	\noindent where the $a_k$ and $b_k$ are compositions of the resistances, capacitances and inductances of the bipole. Apply the Dynamic Phasor Functionals to \eqref{eq:impedance_time_general} and obtain

\begin{equation} \sum_{k=0}^{n} a_k \dpo^k\left[V\right] = \sum_{k=0}^{d} b_k \dpo^k\left[I\right] .\end{equation}

        In polynomial notation,

\begin{equation} \left(\sum_{k=0}^{n} a_k \dpo^k\right)\left[V\right] = \left(\sum_{k=0}^{d} b_k \dpo^k\right)\left[I\right] ,\end{equation}

        \noindent and because the DPFs are invertible and the inversion is akin to division, this becomes

\begin{equation} V(t) = \mathbf{Z} \left[I\right],\ \mathbf{Z} = \left(\dfrac{\sum_{k=0}^{n} a_k \dpo^k}{\sum_{k=0}^{d} b_k \dpo^k}\right) \label{eq:def_impedance}\end{equation}

        \noindent meaning that the operator $\mathbf{Z}$ is the notion of impedance in the Dynamic Phasor domain for it relates the Dynamic Phasors of the voltage and that of the current through the bipole; we therefore call this a \textbf{Dynamic Impedance}. It is only natural from the definition to conclude that such Dynamic Impedances span the entire class $\dpS$ because they are defined as ratios of polynomials of $\dpo$.

\begin{definition}[Dynamic Impedances]\label{def:steinmetz_impedance} A \textbf{Dynamic Impedance} is an operator $\mathbf{Z}\in\dpS$ relating the Dynamic Phasors of the voltage and that of the current through a bipole, that is,

\begin{equation} V(t) = \mathbf{Z} \left[I\right] = \left(\dfrac{\displaystyle\sum_{k=0}^{n} a_k \dpo^k}{\displaystyle\sum_{k=0}^{d} b_k \dpo^k}\right) \left[I\right] \label{eq:def_impedance}\end{equation}

	\noindent where the $a_k,b_k$ are complex scalars with $a_n,b_d\neq 0$.

\end{definition}

	Example \ref{example:dynimp_rlc} shows how this process is done when applied to a second-order circuit.

\begin{example}[Dynamic Impedance of a second-order circuit] \label{example:dynimp_rlc} %<<<

	Consider the RLC circuit of figure \ref{fig:dpimp_example} where the inductance, capacitance and resistance are substituted by their impedances as per \eqref{sys:dpo_impedances_formula}. We want to find the differential equation of $V_R$ as a function of $V(t)$, and the operator $\mathbf{Z}$ that is seen by the input voltage $V(t)$, that is, the operator that relates $V(t)$ and $I_L(t)$.

% MODELLING EXAMPLE: RLC CIRCUIT IN DYNAMIC PHASOR DOMAIN<<<
\begin{figure}[h]
\centering
        \begin{tikzpicture}[american,scale=1,transform shape,line width=0.75, cute inductors, voltage shift = 1]
	\ctikzset{/tikz/circuitikz/voltage/distance from node=10mm}
		\draw (0,0)
			to[vsource,sources/scale=1.25, v>=$V(t)$,invert] (0,4)
			to[L,l=$\dpo L$,f>^=$I_{L}(t)$,v>=$V_{L}(t)$,-*] (4,4) 
			to[C,l=$\dfrac{1}{\dpo C}$,f>^=$I_{C}(t)$,v>=$V_{C}(t)$,-*] (4,0) 
			to[short] (0,0); 
		\draw (4,4)
			to[short,f>^=$I_{R}(t)$] (7,4) 
			to[R,l=$R$,v>=$V_{R}(t)$] (7,0) 
			to[short]  (4,0);
        \end{tikzpicture}
	\caption{Second-order circuit for example application of the single-element Dynamic Phasor impedances.}
	\label{fig:dpimp_example}
\end{figure} %>>>

	Applying Kirchoff's Current Law in the DP domain (theorem \ref{theo:kirchoff_current_1p}) in node 1 one obtains

\begin{equation} (KCL):\  I_L - I_C - I_R = 0\end{equation}

	\noindent and using Kirchoff's Voltage Law in the DP domain (theorem \ref{theo:kirchoff_voltage_1p}) in the voltage nodes yields

\begin{equation}\left\{\begin{array}{l} (L1):\ V_C - V + V_L = 0 \\[3mm] (L2):\ V_R = V_C \end{array}\right. .\end{equation}

	Finally, using the voltage-current relationships \eqref{sys:dpo_impedances} of the elements,

\begin{equation}\left\{\begin{array}{rl} (KCL):&\ I_L - C\dpo\left[V_C\right] - \dfrac{V_R}{R} = 0 \\[3mm] (L1):&\ V_C - V + L\dpo\left[I_L\right] = 0 \\[3mm] (L2):&\ V_R = V_C \end{array}\right. .\end{equation}

	Applying the third equation to the other two,

\begin{equation}\left\{\begin{array}{l} I_L - C\dpo\left[V_R\right] - \dfrac{V_R}{R} = 0 \\[3mm] V_R - V + L\dpo\left[I_L\right] = 0 \end{array}\right. . \label{eq:example_dpos_1}\end{equation}

	Now, apply $\dpo$ to the entire first equation which is possible because the $\dpo$ is bijective, 

\begin{equation}\left\{\begin{array}{l} \dpo\left[I_L\right] - C\ndpo{2}\left[V_R\right] - \dfrac{1}{R} \dpo\left[V_R\right] = 0 \\[3mm] V_R - V + L\dpo\left[I_L\right] = 0 \end{array}\right. .\end{equation}

	\noindent and substituting $\dpo\left[I_L\right]$ from the second equation into the operated first equation:

\begin{equation} \dfrac{V - V_R}{L} - C\ndpo{2}\left[V_R\right] - \dfrac{1}{R} \dpo\left[V_R\right] = 0 \Leftrightarrow \ndpo{2}\left[V_R\right] + \dfrac{1}{RC} \dpo\left[V_R\right] + \dfrac{1}{LC} V_R - \dfrac{1}{LC} V(t) = 0. \label{eq:dpo_complex_model_rlc}\end{equation}

	Substituting the definition of $\dpo$ \eqref{eq:steinmetz_1storder} and the definition \eqref{eq:steinmetz_2ndorder} of $\ndpo{2}$,

\begin{gather}
	{\color{stewartblue}\ndpo{2}\left[V_R\right]} + {\color{stewartgreen}\dfrac{1}{RC} \dpo\left[V_R\right]} +{\color{stewartpink} \dfrac{1}{LC} V_R} -{\color{stewartyellow} \dfrac{1}{LC} V(t)} = 0  \nonumber\\[3mm] 
%
	{\color{stewartblue} \left\{\raisebox{4mm}{} \ddot{V}_R + 2j\omega(t)\dot{V}_R + \left[-\omega^2 + j\dot{\omega}(t)\right]V_R(t)\right\}} + {\color{stewartgreen} \dfrac{1}{RC} \dot{V}_R(t) + j\dfrac{1}{RC}\omega(t) V_R(t) } + {\color{stewartpink} V_R \dfrac{1}{LC} } + {\color{stewartyellow} - \dfrac{1}{LC} V(t)} = 0,
\end{gather}

	\noindent and grouping the terms,

\begin{equation} {\color{stewartblue} 1} \ddot{V}_R(t) + \dot{V}_R(t)\left({\color{stewartgreen} \dfrac{1}{RC}} + {\color{stewartblue} 2j\omega(t)}\right) + V_R\left\{ {\color{stewartpink} \dfrac{1}{LC}} {\color{stewartblue} - \omega^2(t)} + j \left[ {\color{stewartblue} \dot{\omega}(t)} + {\color{stewartgreen} \dfrac{1}{RC}\omega(t)}\right]\right\} {\color{stewartyellow} -\dfrac{1}{LC} V(t)} = 0, \end{equation}

	\noindent which is the same equation \eqref{eq:rlc_complex_diffeq} and \eqref{eq:rlc_complex_diffeq_dpt} from examples \ref{example:rlc_dpt} and \ref{example:dpdomain_secondorder}. Further, \eqref{eq:dpo_complex_model_rlc} is also able to yield the model in time domain: since $\dpo$ in the phasor domain is equivalent to $\mathbf{D}_\mathbb{R}$ in the time domain, 

\begin{equation} \ddot{v}_R + \dfrac{1}{RC} \dot{v}_R + \dfrac{1}{LC} v_R - \dfrac{1}{LC} v(t) = 0\end{equation}

	\noindent yielding the exact time domain model \eqref{eq:rlc_time_diffeq}. Finally, to obtain $I_L$ as a function of $V$, isolate $V_R$ from the second equation of \eqref{eq:example_dpos_1} and substitute on the first equation:

\begin{equation}
	I_L - C\dpo\left[V - L\dpo\left[I_L\right]\right] - \dfrac{V - L\dpo\left[I_L\right]}{R} = 0 .
\end{equation}

	Now using the linearity of the $\dpo$,

\begin{gather}
	I_L - C\dpo\left[V\right] + LC\ndpo{2}\left[I_L\right] - \dfrac{V}{R} + \dfrac{L}{R} \dpo\left[I_L\right] = 0 \\[3mm]
	LC\ndpo{2}\left[I_L\right] + \dfrac{L}{R} \dpo\left[I_L\right] + \mathbf{I}\left[I_L\right] =  C\dpo\left[V\right] + \dfrac{1}{R}\mathbf{I}\left[V\right]
\end{gather}

	Now we can write equivalent operators for each side based on the linear combination of operators:

\begin{equation} \left(LC\ndpo{2} + \dfrac{L}{R} \dpo + \mathbf{I}\right)\left[I_L\right] = \left( C\dpo  + \dfrac{1}{R}\mathbf{I}\right)\left[V\right] \end{equation}

	\noindent and using the division (inversion),

\begin{equation} \left(\xfrac{3mm}{3mm} {LC\ndpo{2} + \dfrac{L}{R} \dpo + \mathbf{I}}{ C\dpo  + \dfrac{1}{R}\mathbf{I} }\right)\left[I_L\right] = V(t) \Leftrightarrow \mathbf{Z} = \dfrac{LC}{R} \xfrac{3mm}{3mm} {\ndpo{2} + \dfrac{1}{RC} \dpo + \dfrac{1}{LC}\mathbf{I}}{ \dpo  + \dfrac{1}{RC}\mathbf{I} } \label{eq:example_analysis_z}\end{equation}

	\noindent thus showing that the operator $\mathbf{Z}$ sought is indeed a ratio of polynomials of $\dpo$.

\examplebar
\end{example} %>>>

	Conversely to Dynamic Impedances, we can define \textbf{Dynamic Admittances} as the operator $\mathbf{Y}$ that relates $I(t) = \mathbf{Y} \left[V\right]$, allowing for the definition a short circuit and an open circuit by means of the null operator.

\begin{definition}\label{def:short_opencircuit} In the context of Dynamic Phasors and Impedances, a \textbf{short-circuit} is a bipole which related impedance is the null operator. Conversely, an \textbf{open-circuit} is a bipole which related admittance is the null operator.\end{definition}

	Since $\mathbf{Z}\in\dpS^*$ and $\dpS$ is invariant to inversion, then $\mathbf{Y}\in\dpS^*$. Naturally, for the same bipole, if it is not a short or an open-circuit then $\mathbf{Z}$ and $\mathbf{Y}$ are inverse operators, that is, $\left(\mathbf{Z}\circ \mathbf{Y}\right)\left[V\right] = V(t)$ or $\mathbf{Z}\circ\mathbf{Y} = \mathbf{I}$. In the same way, $\left(\mathbf{Y}\circ \mathbf{Z}\right)\left[I\right] = I(t)$, or $\mathbf{Y}\circ\mathbf{Z} = \mathbf{I}$.

	Furthermore, one can define the resistance $\mathbf{R}$, reactance $\mathbf{X}$, conductance $\mathbf{G}$ and susceptance $\mathbf{B}$ analogously to static impedance counterparts by using the real and imaginary part operations as in subsection \ref{subsec:real_imag_dpfs}:

\begin{equation} \mathbf{Z} = \mathbf{R} + j\mathbf{X}\ \left\{\begin{array}{l} \mathbf{R} = \Re\left[\mathbf{Z}\right] \\ \mathbf{X} = \Im\left[\mathbf{Z}\right]\end{array}\right.  \text{, and } \mathbf{Y} = \mathbf{G} + j\mathbf{B}\left\{\begin{array}{l} \mathbf{G} = \Re\left[\mathbf{Y}\right] \\ \mathbf{B} = \Im\left[\mathbf{Y}\right]\end{array}\right. . \end{equation}

	One immediately notices that if the apparent frequency of the DPT is some constant $\omega_0$, then Dynamic Impedances and admittances become ratios of powers of $j\omega_0$. Additionally, in a static sinusoidal situation (constant amplitudes and phases), the Dynamic Impedance and Admittance becomes a multiplication by exactly the impedances in static phasor domain. Therefore, Dynamic Admittances generalize the concept of impedances.

	It is also immediate to notice that the existence of such impedance operators rely on the fact that polynomials of $\dpo$ exist and are also invertible, as proven in \ref{subsec:notation_abuse}. A natural question that arises from the definition is if, given both voltage and current signals, the impedance operator is unique.

\begin{theorem}[Uniqueness of Dynamic Impedance operators] \label{theo:impedance_uniqueness} %<<<
	 Given a bipole, $V(t)$ the DP of the voltage across it and $I(t)$ the DP of the current through it, if the bipole is not a short or an open-circuit, then $\mathbf{Z}$ and $\mathbf{Y}$ are unique up to scaling and coprimality of the numerator and denominator polynomials. \end{theorem}
\noindent\textbf{Proof:} we first consider the fringe cases. If the considered bipole is a short circuit then $\mathbf{Y}$ does not exist and $\mathbf{Z}$ is not unique because any current $I(t)$ yields zero voltage. The converse situation is true for an open-circuit, so let us assume that the bipole in question is neither of those.

	Let $\mathbf{Z}$ as in \eqref{eq:def_impedance}, denoted

\begin{equation} \mathbf{Z} = \dfrac{\mathbf{N}\left(\dpo\right)}{\mathbf{D}\left(\dpo\right)},\ \mathbf{N,D}\in\mathbb{C}\left[\dpo\right] . \end{equation}

	Because the field of DPFs adheres to the Fundamental Theorem of Algebra (see theorem \ref{theo:dps_alg_closed}), then $\mathbf{N}$ and $\mathbf{D}$ can be written as products of their monomials as in \eqref{eq:fundamental_algebra_dpfs}. If $\mathbf{N}$ and $\mathbf{D}$ are not coprime they share a root $\dpL_0$ and their factorization has a common $\left(\dpL - \dpL_0\right)$ term. Thus their factorizations can be simplified until they are coprime. Also, because the polynomials can also be multipled together by any complex factor, define a ``canonical'' version of $\mathbf{Z}$ where numerator and denominator are coprime and monic:

\begin{equation} \mathbf{Z}^* = \dfrac{a_n}{b_d} \left(\dfrac{\dpo^p + \sum_{k=0}^{p-1} \alpha_k \dpo^k}{\dpo^q + \sum_{k=0}^{q-1} \beta_k \dpo^k}\right)\end{equation}

	Because $\mathbf{Z}$ is not a short, we can suppose $a_n,b_d\neq 0$, so let $k_z = a_n/b_d$ and $K(t)$ such that

\begin{equation} \left\{ \begin{array}{l} K(t) = \left[k_z \left(\dpo^p + \sum_{k=0}^{p-1} \alpha_k\dpo^k\right)\right]\left[I\right] \\[3mm] K(t) = \left[\left(\dpo^q + \sum_{k=0}^{q-1} \beta_k \dpo^k\right)\right]\left[V\right] \end{array}\right. \end{equation}

	Because any non-trivial linear combination of powers of $\dpo$ is bijective, as proven in \ref{subsec:notation_abuse}, the first equation dictates, for any combination of coefficients $\alpha$, that $K(t)$ and $I(t)$ are uniquely related; by the second equation, so are $K(t)$ and $V(t)$ for any combination of $\beta$ coefficients. Therefore, by the transitivity of bijection, $V(t)$ and $I(t)$ are bijectively related. \hfill$\blacksquare$ \vspace{5mm}\hrule\vspace{5mm} %>>>

	With all these theorems in our arsenal, we can now prove nice circuit modelling tools like the series-parallel combination of impedances and admittances, as per theorems \ref{theo:series_z} and \ref{theo:parallel_y}.

\begin{theorem}[Series combination of Dynamic Impedances]\label{theo:series_z} %<<<
	Consider a series combination of $\left(\mathbf{Z}_k\right)_{k\in\mathbb{N}_n^*}$ Dynamic Impedance operators, the Dynamic Phasor voltage $V(t)$ being the voltage across the entire combination as per figure \ref{fig:series_combination}. Then the equivalent impedance $\mathbf{Z}_E$ is $\mathbf{Z}_E = \mathbf{Z}_1 + \mathbf{Z}_2 + ... + \mathbf{Z}_n$, that is, the voltage across the combination and the current through it are related by $V(t) = \mathbf{Z}_E \left[I\right]$. At the same time, the voltage across the each impedance is given by the impedance divider formula

\begin{equation} V_i(t) = \left(\dfrac{\mathbf{Z}_i}{\mathbf{Z}_E}\right) \left[V\right] .\label{eq:impedance_div}\end{equation}

% SERIES COMBINATION OF IMPEDANCES <<<
\begin{figure}[h]
\centering
\scalebox{0.75}{
\begin{tikzpicture}[american,scale=1,transform shape,line width=0.75, cute inductors, voltage shift = 0.2]
\ctikzset{/tikz/circuitikz/voltage/distance from node=1mm, bipoles/thickness=1.5, bipole label style/.style={font=\Large}, bipole voltage style/.style={font=\Large}, bipole current style/.style={font=\Large}}
\draw (0, 0) to [short, o-, f>_={\Large $I(t)$}] ++(0,-2) 
	to[generic, l=$\mathbf{Z}_1$] ++(2,0)
	to[generic, l=$\mathbf{Z}_2$] ++(2,0) node (stop1) {}
	to[open] ++(2,0) node (stop2) {}
	to[generic, l=$\mathbf{Z}_i$] ++(2,0) node (stop3) {}
	to[open] ++(2,0) node (stop4) {}
	to[generic, l=$\mathbf{Z}_n$] ++(2,0)
	to[short, -o] ++(0,2) node (endpoint) {};

	\node at ($(stop1)!0.35 !(stop2)$) [draw, circle, fill, minimum size = 0.5mm, radius=5mm, inner sep=0mm, outer sep=1mm] (tleft) {} ;
	\node at ($(stop1)!0.5  !(stop2)$) [draw, circle, fill, minimum size = 0.5mm, radius=5mm, inner sep=0mm, outer sep=1mm] (tmiddle) {} ;
	\node at ($(stop1)!0.65 !(stop2)$) [draw, circle, fill, minimum size = 0.5mm, radius=5mm, inner sep=0mm, outer sep=1mm] (tright) {} ;
	
	\node at ($(stop3)!0.35 !(stop4)$) [draw, circle, fill, minimum size = 0.5mm, radius=5mm, inner sep=0mm, outer sep=1mm] (tleft) {} ;
	\node at ($(stop3)!0.5  !(stop4) $) [draw, circle, fill, minimum size = 0.5mm, radius=5mm, inner sep=0mm, outer sep=1mm] (tmiddle) {} ;
	\node at ($(stop3)!0.65 !(stop4)$) [draw, circle, fill, minimum size = 0.5mm, radius=5mm, inner sep=0mm, outer sep=1mm] (tright) {} ;
	
	\draw ([shift=({0,-2mm})]$(stop1)!0.5 !(stop2)$) to[open,european,voltage/distance from node=0.5mm, voltage/bump b=3, v_<=$V_i(t)$] ([shift=({0,-2mm})]$(stop3)!0.5 !(stop4)$);
	
	\draw (endpoint) to[open,european, voltage/bump b=3, v_>=$V(t)$] (0,0);

\end{tikzpicture}
}
\caption{Series combination schematic for theorem \ref{theo:vsi_equiv}.}
\label{fig:series_combination}
\end{figure}
%>>>

\end{theorem}
\hfill$\blacksquare$\vspace{5mm}\hrule\vspace{5mm} %>>>

\begin{theorem}[Parallel combination of Dynamic Admittances]\label{theo:parallel_y} %<<<
	Consider a parallel combination of $\left(\mathbf{Y}_k\right)_{k\in\mathbb{N}_n^*}$ Dynamic Impedance operators, the Dynamic Phasor current $I(t)$ being the current injected into the combination as per figure \ref{fig:parallel_combination}. Then the equivalent admittance $\mathbf{Y}_E$ is $\mathbf{Y}_E = \mathbf{Y}_1 + \mathbf{Y}_2 + ... + \mathbf{Y}_n$, that is, the current through the combination and the voltage across it are related by $I(t) = \mathbf{Y}_E \left[V\right]$. At the same time, the current through each admittance is given by the admittance divider formula

\begin{equation} I_i(t) = \left(\dfrac{\mathbf{Y}_i}{\mathbf{Y}_E}\right) \left[I\right] .\label{eq:admittance_div}\end{equation}

% SERIES COMBINATION OF IMPEDANCES <<<
\begin{figure}[h]
\centering
\scalebox{0.75}{
\begin{tikzpicture}[american,scale=1,transform shape,line width=0.75, cute inductors, voltage shift = 0.2]
\ctikzset{/tikz/circuitikz/voltage/distance from node=1mm, bipoles/thickness=1.5, bipole label style/.style={font=\Large}, bipole voltage style/.style={font=\Large}, bipole current style/.style={font=\Large}}
\draw (0, 0)
	to [short, o-, f>_={\Large $I(t)$}] ++(2,0) node(conn1) {}
	to[generic, l=$\mathbf{Y}_1$] ++(0,-3) node(conn2) {}
	to[short,-o] ++(-2,0) node(conn15) {};
\draw (conn1.center)
	to[short] ++(2,0) node(conn3) {}
	to[generic, l=$\mathbf{Y}_2$] ++(0,-3) node(conn4) {}
	to[short] ++(-2,0);
\draw (conn3.center)
	to[open] ++(2,0) node (conn11) {}
	to[short] ++(1,0) node(conn5) {}
	to[generic, l=$\mathbf{Y}_i$,f>={\Large $I_i(t)$}] ++(0,-3) node(conn6) {}
	to[short] ++(-1,0) node(conn13) {};
\draw (conn5.center) to[short] ++(1,0) node(conn7) {};
\draw (conn6.center) to[short] ++(1,0) node(conn8) {};
\draw (conn7.center)
	to[open] ++(2,0) node(conn12) {}
	to[short] ++(1,0) node(conn9) {}
	to[generic, l=$\mathbf{Y}_n$] ++(0,-3) node(conn10) {}
	to[short] ++(-1,0) node(conn14) {};

\draw (conn15) to[open,european, voltage/bump b=1, v^>=$V(t)$] (0,0);
	
\node at ($(conn3)!0.35 !(conn11)$) [draw, circle, fill, minimum size = 0.5mm, radius=5mm, inner sep=0mm, outer sep=1mm] (tleft) {} ;
\node at ($(conn3)!0.5  !(conn11)$) [draw, circle, fill, minimum size = 0.5mm, radius=5mm, inner sep=0mm, outer sep=1mm] (tmiddle) {} ;
\node at ($(conn3)!0.65 !(conn11)$) [draw, circle, fill, minimum size = 0.5mm, radius=5mm, inner sep=0mm, outer sep=1mm] (tright) {} ;

\node at ($(conn7)!0.35 !(conn12)$) [draw, circle, fill, minimum size = 0.5mm, radius=5mm, inner sep=0mm, outer sep=1mm] (tleft) {} ;
\node at ($(conn7)!0.5  !(conn12)$) [draw, circle, fill, minimum size = 0.5mm, radius=5mm, inner sep=0mm, outer sep=1mm] (tmiddle) {} ;	
\node at ($(conn7)!0.65 !(conn12)$) [draw, circle, fill, minimum size = 0.5mm, radius=5mm, inner sep=0mm, outer sep=1mm] (tright) {} ; 

\node at ($(conn4)!0.35 !(conn13)$) [draw, circle, fill, minimum size = 0.5mm, radius=5mm, inner sep=0mm, outer sep=1mm] (tleft) {} ;
\node at ($(conn4)!0.5  !(conn13)$) [draw, circle, fill, minimum size = 0.5mm, radius=5mm, inner sep=0mm, outer sep=1mm] (tmiddle) {} ;	
\node at ($(conn4)!0.65 !(conn13)$) [draw, circle, fill, minimum size = 0.5mm, radius=5mm, inner sep=0mm, outer sep=1mm] (tright) {} ; 

\node at ($(conn8)!0.35 !(conn14)$) [draw, circle, fill, minimum size = 0.5mm, radius=5mm, inner sep=0mm, outer sep=1mm] (tleft) {} ;
\node at ($(conn8)!0.5  !(conn14)$) [draw, circle, fill, minimum size = 0.5mm, radius=5mm, inner sep=0mm, outer sep=1mm] (tmiddle) {} ;	
\node at ($(conn8)!0.65 !(conn14)$) [draw, circle, fill, minimum size = 0.5mm, radius=5mm, inner sep=0mm, outer sep=1mm] (tright) {} ;  \end{tikzpicture}
}
\caption{Parallel combination schematic for theorem \ref{theo:parallel_y}.}
\label{fig:parallel_combination}
\end{figure}
%>>>

\end{theorem}
\hfill$\blacksquare$\vspace{5mm}\hrule\vspace{5mm} %>>>

	These formulas allow, for instance, to quickly model circuits in the Dynamic Phasor domain, like done for the Laplace impedance \eqref{eq:rlc_laplace_model} of circuit \ref{fig:laplace_example}.

\begin{example}[Dynamic Impedance of a second-order circuit (again)] \label{example:dynimp_rlc} %<<<

	Consider the same circuit of figure \ref{fig:dpimp_example} in example \ref{example:dynimp_rlc}. Using theorems \ref{theo:series_z} and \ref{theo:parallel_y}, we can see that the load voltage $V_R$ is the individual voltage of an admittance combination that is a series combination of an impedance $\dpo L$ with a parallel combination of the impedances $\left(\dpo C\right)^{-1}$ and $R$, so that

\begin{equation} V_R = \left(\xfrac{8mm}{5mm}{\dfrac{\mathbf{I}}{\dfrac{1}{R} + \dfrac{\mathbf{I}}{\dpo C}}}{\dpo L + \dfrac{\mathbf{I}}{\dfrac{1}{R} + \dfrac{\mathbf{I}}{\dpo C}}}\right)\left[V\right] \end{equation}

	\noindent and ``multiplying'' both numerator and denominator by $R + \frac{\mathbf{I}}{RC}$,

\begin{equation} V_R(t) = \left( \dfrac{\raisebox{-8mm}{} \dfrac{\mathbf{I}}{\raisebox{5mm}{} \dfrac{\mathbf{I}}{R} + \dpo C}}{\raisebox{5mm}{} \dpo L + \dfrac{\mathbf{I}}{\raisebox{5mm}{} \dfrac{\mathbf{I}}{R} + \dpo C}}\right)\left[V\right] = \left( \dfrac{\mathbf{I}}{\raisebox{5mm}{} \ndpo{2}LC + \dpo\dfrac{L}{R} + \mathbf{I}}\right)\left[V\right] = \dfrac{1}{LC} \left( \dfrac{\mathbf{I}}{\raisebox{5mm}{} \ndpo{2} + \dpo\dfrac{1}{RC} + \dfrac{1}{LC}\mathbf{I}}\right)\left[V\right].\label{eq:rlc_dp_model}\end{equation}

	\noindent which is a direct Dynamic Phasor counterpart to \eqref{eq:rlc_laplace_model} and immediately delivers \eqref{eq:dpo_complex_model_rlc} in a single line of calculations. Finally, for the impedance $\mathbf{Z}$ relating $V$ and $I$, we again lay hold of the fact that the total impedance seen by the source $V(t)$ is series combination of an impedance $\dpo L$ with a parallel combination of the impedances $\left(\dpo C\right)^{-1}$ and $R$, yielding

\begin{equation} \mathbf{Z} = \dpo L + \dfrac{\mathbf{I}}{\dfrac{\mathbf{I}}{R} + \dpo C} \end{equation}

	\noindent and using the operational properties of $\dpS$,

\begin{equation} \mathbf{Z} = \dpo L + \dfrac{\mathbf{I}}{\dfrac{\mathbf{I}}{R} + \dpo C} = \xfrac{3mm}{3mm}{ \dpo L \left(\dfrac{\mathbf{I}}{R} + \dpo C\right)}{\dfrac{\mathbf{I}}{R} + \dpo C} + \xfrac{3mm}{3mm}{\mathbf{I}}{\dfrac{\mathbf{I}}{R} + \dpo C} = \xfrac{3mm}{3mm}{ \dpo \dfrac{L}{R} + \ndpo{2} LC + \mathbf{I} }{\dfrac{\mathbf{I}}{R} + \dpo C} = \dfrac{LC}{R} \xfrac{3mm}{3mm}{ \ndpo{2} + \dfrac{1}{RC} \dpo + \dfrac{1}{LC} \mathbf{I} }{\dpo + \dfrac{\mathbf{I}}{RC}}\end{equation}

	\noindent and this equation is exactly as the one obtained before \eqref{eq:example_analysis_z}.

\examplebar
\end{example} %>>>

%-------------------------------------------------
\subsection{Superposition, Thèvenin and Norton} %<<<2

	Now using the notion of Dynamic Impedances, theorem \ref{theo:vsi_equiv} proves the duality between a voltage source and a current source in the Dynamic Phasor context. Using this duality, the Superposition Theorem is proven next.

\begin{theorem}[Voltage and current source equivalence for Dynamic Phasors]\label{theo:vsi_equiv} %<<<
	Consider the series combination of a nonstationary sinusoidal voltage source $V_O(t)$ with an impedance operator $\mathbf{Z}$ in the left part of figure \ref{fig:dual_sources}. Then this circuit is equivalent to a nonstationary sinusoidal current source $I_S(t) = \mathbf{Z}^{-1} \left[V_O\right]$, which is the short-circuit current of the voltage-impedance combination, in parallel with an admittance operator $\mathbf{Y} = \mathbf{Z}^{-1}$, like the right part of figure \ref{fig:dual_sources}.
\end{theorem} %>>>
\textbf{Proof:} take the two-port circuits of figure \ref{fig:dual_sources}, and let us start with the circuit on the left. Use Kirchoff's Voltage Law (theorem \ref{theo:kirchoff_voltage}) to yield the equation that describes this circuit:

% VOLTAGE CURRENT DUALITY FIGURE <<<
\begin{figure}[h]
\centering
\scalebox{0.75}{
\begin{tikzpicture}[american,scale=1,transform shape,line width=0.75, cute inductors, voltage shift = 0.2]
\ctikzset{/tikz/circuitikz/voltage/distance from node=10mm,amplifiers/scale = 1.5, bipoles/thickness=1.5, bipole label style/.style={font=\Large}, bipole voltage style/.style={font=\Large}, bipole current style/.style={font=\Large}}
\draw (0, 0) to [short, o-, f>={\Large $I(t)$}] ++(2,0) 
	to[generic, l=$\mathbf{Z}$] ++(0,-2)
	to[vsource,sources/scale=1.25, v>=$V_O(t)$] ++(0,-2)
	to[short] ++(0,-0.3)
	to[short, -o] ++(-2,0) node (endpoint) {};
%
\draw(0,0) to[open,european,voltage/distance from node=0.5mm, voltage/bump b=1, v<=$V(t)$] (endpoint);
%
\draw
	(6, 0) node (istart3) {} to [short, o-, f>={\Large $I(t)$}] ++(2,0) node (istart) {}
	to[generic, l_=$\mathbf{Z}$, *-*] (istart |- endpoint) node (conn) {}
	to[short, -o] ++(-2,0) node (iend) {};
%
\draw
	(istart.center) to[short] ++(2,0) node (istart2) {}
	to[isource,invert,sources/scale=1.25, l=$I_S(t)$] (istart2 |- endpoint)
	to[short] (conn.center);
%
\draw (istart3) to[open,european,voltage/distance from node=0.5mm, voltage/bump b=1, v<=$V(t)$] (iend);
\end{tikzpicture}
}
\caption{Dual sources for the proof of the source duality theorem \ref{theo:vsi_equiv}.}
\label{fig:dual_sources}
\end{figure}
%>>>

\begin{equation} V(t) = V_O(t) + \mathbf{Z}\left[I\right] . \label{eq:kvl_equiv}\end{equation}

	At the same time, use Kirchoff's Current Law (theorem \ref{theo:kirchoff_current}) on the circuit on the right:

\begin{equation} I_S(t) + I(t) = \mathbf{Y}\left[V\right]. \label{eq:kcl_equiv}\end{equation}

	Use $V(t) = 0$ on \eqref{eq:kvl_equiv} to conclude that $I_S(t) = - \mathbf{Y}\left[V_O\right]$ is the short-circuit current of the circuit on the left. Applying $\mathbf{Y}$ to \eqref{eq:kvl_equiv} yields $\mathbf{Y} \left[V_O\right] - I(t) = I_O(t)$, which is exactly \eqref{eq:kcl_equiv} — the equation that describes the circuit on the right — meaning that the circuit on the right describes the one on the left. Similarly, using $I(t) = 0$ on \eqref{eq:kcl_equiv} yields $V_O(t) = \mathbf{Z} \left[I_S\right]$ is the open-circuit voltage of the circuit on the right, and apply $\mathbf{Z}$ on \eqref{eq:kcl_equiv} to yield \eqref{eq:kvl_equiv}, that is, the circuit on the left also describes the one on the right. \hfill$\blacksquare$

\begin{theorem}[Superposition Principle for Dynamic Phasors] \label{theo:superposition}%<<<
Consider a circuit network composed of resistors, capacitors and inductors, $n_v$ generalized sinusoidal voltage sources (or just ``voltage sources'') listed as $v^S_p(t)$ and $n_i$ generalized sinusoidal current sources (``current sources'') listed as $i^S_q(t)$, where the frequencies at which each source is defined are mutually equivalent. Dependent sources must be linear, that is, the output voltage or current is a linear operator of node voltages and branch currents. Then the Dynamic Phasor of the voltage across any two nodes $V(t)$ can be written as a sum of the DPs of voltages and currents of each source:

\begin{equation} V(t) = \sum_{p=1}^{n_v} \mathbf{A}_p \left[V^S_p\right] + \sum_{q=1}^{n_i} \mathbf{Z}^E_q \left[I^S_q\right] .\label{theo:super_generic_vol}\end{equation}

	where:

\begin{itemize}
	\item Each $\mathbf{A}_p$ is a dimensionless operator obtained by setting all independent voltage sources but the p-th one as shorts and all current sources as open circuits; and
	\item the $\mathbf{Z}^E_q$, the ``$E$'' superscript for \textit{equivalent}, are impedance operators obtained by setting all independent current sources as shorts and all current sources but the q-th as open circuits.
\end{itemize}

	Accordingly, pick a branch and denote $I(t)$ the current through it. Then

\begin{equation} I(t) = \sum_{p=1}^{n_v} \mathbf{Y}_p^E \left[V^S_p\right] + \sum_{q=1}^{n_i} \mathbf{B}_q \left[I^S_q\right] .\label{theo:super_generic_curr}\end{equation}

	where:

\begin{itemize}
	\item Each $\mathbf{Y}_p^E$ is an admittance operator obtained by setting all independent voltage sources but the p-th one as shorts and all independent current sources as open circuits; and
	\item the $\mathbf{B}_q$ are dimensionless operators obtained by setting all independent voltage sources as shorts and all independent current sources but the q-th as open circuits.
\end{itemize}

\end{theorem}
\textbf{Proof:} if each current and voltage source is defined at a particular apparent frequency, we suppose all these frequency signals are mutually equivalent as per definition \ref{def:equivalent_freqs}. Thus by theorem \ref{theorem:sols_are_nonst}, there is a signal $\omega(t)$ such that all voltage and current sources can be written at $\omega(t)$; adopt this signal for the DPT, and the voltage sources are transformed into their phasorial versions as $\mathbf{P_D^{\left(\omega\right)}}\left[v^S_p\right] = V_S^p(t)$ and $\mathbf{P_D^{\left(\omega\right)}}\left[i^S_q\right] = I_S^q(t)$. Further, adopt the frequency $\omega(t)$ for the DPFs; thus, substitute inductors by their DPF equivalents $\dpo L$, capacitances by $\left(\dpo C\right)^{-1}$ and resistances by $R$, where $\dpo$ is calculated at $\omega(t)$.

	Suppose the circuit has $n$ nodes; use lemma \ref{theo:vsi_equiv} to convert all voltage sources to current sources. Then the circuit will have $n_s = n_v + n_i$ current sources; arrange them such that the first $n_i$ are the original current sources and the following $n_v$ ones are the ``converted'' current sources. Pick two nodes; for convenience, one of these nodes will be numbered node 1 and the other the voltage reference against which all node voltages are measured. Then, for each node use Kirchoff's Current Law to yield

\begin{equation}
	\begin{array}{ccc} + \mathbf{Y}_{11} \left[V_1\right] - \mathbf{Y}_{12} \left[V_2\right] - ... - \mathbf{Y}_{1n} \left[V_n\right] &=& \left(\sum I\right)_1 \\[1mm] - \mathbf{Y}_{21} \left[V_1\right] + \mathbf{Y}_{22} \left[V_2\right] - ... - \mathbf{Y}_{2n} \left[V_n\right] &=& \left(\sum I\right)_2 \\ \vdots \\ - \mathbf{Y}_{n1} \left[V_1\right] - \mathbf{Y}_{2n} \left[V_2\right] - ... + \mathbf{Y}_{nn} \left[V_n\right] &=& \left(\sum I\right)_n \end{array} \label{eq:superptheo_nodes}
\end{equation}

	\noindent where:

\begin{itemize}
	\item $\left(\sum I\right)_j$ is the sum of currents delivered by current sources to node $j$, that is, the sum of the currents of the sources connected to node $j$ where currents injected into the node are positive and currents coming out of the node are negative;
	\item $V_j$ the voltage at node $j$ with respect to the voltage reference;
	\item $\mathbf{Y}_{ij}$ the admittance operator between nodes $i$ and $j$ constructed by building the equivalent admittance between both nodes with all voltage sources substituted by short circuits and all current sources by open circuits;
	\item and $\mathbf{Y}_{ii}$ the total admittance measured at the node $i$, consisting of sums of the opposites of $\mathbf{Y}_{ij}$ for $1\leq j \leq n$.
\end{itemize}

	This modelling holds even for fringe cases. If there is a voltage source directly conencted to node $k$, then in \eqref{eq:superptheo_nodes} the k-th row is eliminated and in each row $j$ the term $\mathbf{Y}_{jk}V_k$ is transferred to the right side, that is, $V_k$ becomes an equivalent current injection in each node. Also, if node $k$ is connected to a voltage dependent source then $V_k$ can be written as a linear operator in the voltages of other nodes and currents of branches — in the same fashion as in the first case, the k-th row is eliminated and for each j-th line, $\mathbf{Y}_{jk}V_k$ is incorporated into the other variables. The same process happens if a branch current is a dependent current source.

	In all cases, the resulting equations maintain the same form as \eqref{eq:superptheo_nodes}. Then by theorem \ref{theo:dpf_matrices_exist}, we use the matrix representation of DPFs (definition \ref{def:matrices_in_dps}) and the definitions of matrix-by-vector multiplication (\eqref{eq:matrix_by_dpvec_def} and \eqref{eq:dpvec_by_matrix_def}) to yield a matrix representation of \eqref{eq:superptheo_nodes}:

\begin{equation}
	\left[\begin{array}{cccc} \mathbf{Y}_{11} & - \mathbf{Y}_{12} & ...  & - \mathbf{Y}_{1n} \\[3mm] - \mathbf{Y}_{21} & + \mathbf{Y}_{22} & ...  & - \mathbf{Y}_{2n} \\[3mm] \vdots & \vdots & \ddots & \vdots \\[3mm] - \mathbf{Y}_{11} & - \mathbf{Y}_{12} - & ...  & + \mathbf{Y}_{1n} \end{array}\right]\left[\begin{array}{c} V_1 \\[3mm] V_2 \\[3mm] \vdots \\[3mm] V_n\end{array}\right] = \left[\begin{array}{c} \displaystyle\left(\sum I\right)_1 \\[3mm]\displaystyle \left(\sum I\right)_2 \\[3mm] \vdots \\[3mm]\displaystyle \left(\sum I\right)_n \end{array}\right] \Leftrightarrow \left[\mathbf{Y}\right]\left[V\right] = \left[I\right] .\label{eq:superptheo_nodematrix}
\end{equation}

	Because by the conclusion of subsection \ref{subsec:matrces_in_dpfts}, the matricial operations for matrices od DPFs are maintained very closely to those in complex matrices; as such, we can use Kramer's Rule in the matrix representation \eqref{eq:superptheo_nodematrix} to obtain

\begin{equation} V_1 = \dfrac{\raisebox{-14mm}{} 
\det\left(\left[\begin{array}{cccc} \left(\sum I\right)_1 & - \mathbf{Y}_{12} & ... & -\mathbf{Y}_{1n} \\[3mm] \left(\sum I\right)_2 & \phantom{-}\mathbf{Y}_{22} &...& -\mathbf{Y}_{2n} \\[3mm] \vdots & \vdots & \ddots & \vdots \\[3mm] \left(\sum I\right)_n & -\mathbf{Y}_{2n} & ... & \phantom{-}\mathbf{Y}_{nn}\end{array}\right]\right)
}{\raisebox{12mm}{}
\det\left(\left[ \begin{array}{cccc} \phantom{-}\mathbf{Y}_{11} & - \mathbf{Y}_{12} & ... & -\mathbf{Y}_{1n} \\[3mm] -\mathbf{Y}_{12} & \phantom{-}\mathbf{Y}_{22} &...& -\mathbf{Y}_{2n} \\[3mm] \vdots & \vdots & \ddots & \vdots \\[3mm] - \mathbf{Y}_{n1} & -\mathbf{Y}_{n2} & ... & \phantom{-}\mathbf{Y}_{nn}\end{array}\right]\right)
} =
%
\dfrac{\Delta_V}{\Delta}
\label{eq:superptheo_nodematrix_2}
\end{equation}

	We now argue that the determinant $\Delta$ is not null. If this were the case, then the matrix representation \eqref{eq:superptheo_nodematrix} would mean that the voltage values of the nodes obtained from the current sources in equation \eqref{eq:superptheo_nodes} are not unique, meaning that at least one voltage is not determined by that equation. This is only true in two situations: that particular node is detached from the circuit (so all its coeficcients are null), which cannot be the case because the circuit is defined as a connected one; or, some node is a short-circuit, thus two nodes have the same voltage, which also cannot be true by the definitions of nodes. Thus the matrix $\mathbf{Y}$ is not singular.

	A cofactor expansion of \eqref{eq:superptheo_nodematrix_2} by the first column yields

\begin{equation} V_1 = \sum_{k=1}^n \left(-1\right)^k \dfrac{\Delta_{k}}{\Delta}\left [\left(\sum I\right)_k\right], \end{equation} 

	Since the first $n_i$ current sources are the original curent sources and the following $n_v$ ones are the converted voltage sources, this sum can be broken down into

\begin{align}
	V_1 &= \left(\Delta\right)^{-1} \sum_{k=1}^n \left(-1\right)^k \left(\sum_{p=1}^{n_i} \Delta_{p} \left[I^S_{p}\right]\right) + \sum_{q=n_i+1}^{n_i + n_v} \Delta_{q} \left[I^S_q \right] \nonumber\\[3mm]
	    &= \left(\Delta\right)^{-1} \sum_{k=1}^n \left(-1\right)^k \left(\sum_{p=1}^{n_i} \Delta_{p} \left[I^S_{p}\right]\right) + \sum_{q=1}^{n_v}           \left(\Delta_{q} \mathbf{Y}_{q}\right)\left[ V^S_q\right] \nonumber\\[3mm]
	    &= \sum_{p=1}^{n_i}\left[ \raisebox{-3mm}{ $\overbrace{\left(\sum_{k=1}^n \left(-1\right)^k \dfrac{\Delta_p}{\Delta} \right)}^{\mathbf{A}_p} I^S_{p} $ }\right] + \sum_{q=1}^{n_v}\left[ \raisebox{-3mm}{ $\overbrace{\left( \sum_{k=1}^n \left(-1\right)^k \dfrac{\Delta_q}{\Delta}\mathbf{Z}^{-1}_q \right)}^{\mathbf{Z}_q^E} \left[V^S_{p}\right] $}\right] \label{eq:superptheo_grouping}
\end{align}

	and \eqref{eq:superptheo_grouping} yields \eqref{theo:super_generic_vol}. The proof of \eqref{theo:super_generic_curr} is similar. \hfill$\blacksquare$
%>>>

	Thence, finally, Thèvenin's and Norton's Theorems are proven in the Dynamic Phasor context as direct consequences of the Superposition Principle of theorem \ref{theo:superposition}.

\begin{theorem}[Thèvenin's Theorem for Dynamic Phasors] \label{theo:thevenin} %<<<
Consider a two-port circuit network composed of resistors, capacitors and inductors, nonstationary voltage sources and nonstationary current sources. Then the voltage across the two ports can be written as $V(t) = V_0(t) - \mathbf{Z}\left[I\right]$, $V_0$ the open-circuit voltage of the network, $I(t)$ the current drawn from the ports, $\mathbf{Z}$ the equivalent impedance operator obtained by substituting all voltage sources by short circuits and all current sources by open circuits. Further, $\mathbf{Z}$ is such that $V_0(t) = \mathbf{Z}\left[I_S\right]$, $I_S$ the short-circuit current of the network.
\end{theorem}
\textbf{Proof:} suppose the network has $n_v$ voltage sources and $n_i$ current sources. Place a test current source $I(t)$ on the terminals of the network, closing the circuit. Then by the Superposition Principle of theorem \ref{theo:superposition},

\begin{equation} V(t) = \sum_{i=1}^{n_v} \mathbf{A}_i \left[V^S_i\right] + \sum_{i=1}^{n_i} \mathbf{Z}_A \left[I^S_i\right] - \mathbf{Z} \left[I\right] .\label{theo:thevenin_generic}\end{equation}

	If the two ports are placed in open circuit condition, that is, $I(t) = 0$, then the resulting voltage is the open circuit voltage:

\begin{equation} V_0(t) = \sum_{i=1}^{n_v} \mathbf{A}_i \left[V^S_i\right] + \sum_{i=1}^{n_i} \mathbf{Z}_A \left[I^S_i\right].\end{equation}

	\noindent which substituted onto \eqref{theo:thevenin_generic} yields

\begin{equation} V(t) = V_0(t) - \mathbf{Z} \left[I\right] ,\end{equation}

	which proves the first proposition. Then, substituting $V(t) = 0$ for a short-circuit on the terminal ports, one obtains

\begin{equation} V_0(t) = \mathbf{Z} \left[ I_S\right] \tag*{\llap{$\blacksquare$}} \end{equation} %>>>

% SAMPLE CIRCUIT <<<
\begin{figure}[t]
\centering
\scalebox{0.75}{
\begin{tikzpicture}[american,scale=1,transform shape,line width=0.75, cute inductors, voltage shift = 0.2]
\ctikzset{/tikz/circuitikz/voltage/distance from node=10mm,amplifiers/scale = 1.5, bipoles/thickness=1.5, bipole label style/.style={font=\Large}, bipole voltage style/.style={font=\Large}, bipole current style/.style={font=\Large}}
\draw (0, 0) node[op amp, fill=stewartblue!50] (opamp) {}
	(opamp.-) to[short,i<=$i_N$] ++(-1, 0) coordinate(A)
	(opamp.out) to[short,-o] ++(2, 0) coordinate(C)
	(A) to[short] ++(-2,0) coordinate(in1) to[short] ++(0,-0.5) to[R,l_=$R_1$] ++(0, -1.5) to[vsource,sources/scale=1.25,i<^=$i_{1}$, v>=$v_1$, fill=stewartpink!50] ++(0,-2.5) to[short] ++(0,-0.5) coordinate(n1)
	(in1) to[short] ++(-3,0) coordinate(in2) to[short] ++(0,-0.5) to[R,l_=$R_2$] ++(0, -1.5) to[vsource,sources/scale=1.25,i<^=$i_{2}$, v>=$v_2$, fill=stewartpink!50] coordinate(v2) ++(0,-2.5) to[short] ++(0,-0.5) coordinate(n2)
	(in2) to[short] ++(-4,0) coordinate(in3) to[short] ++(0,-0.5) to[R,l_=$R_n$] ++(0, -1.5) to[vsource,sources/scale=1.25,i<^=$i_{n}$, v>=$v_n$, fill=stewartpink!50]  coordinate(vn) ++(0,-2.5) to[short] ++(0,-0.5) coordinate(Gn)
	(A) to[short,i=$i_F$] ++(0,1.5) coordinate(F1) to[short] ++(2,0) to[R=$R_F$] ++(1,0) -| ([shift={(0.5,0)}]opamp.out) 
	(F1) to[short] ++(0,1.5) to[short] ++(1.5,0) to[C=$C_F$] ++(2,0) -|  ([shift={(0.5,0)}]opamp.out) 
	(opamp.+) to[short,i<=$i_P$] ++(-1,0) coordinate(posbreak) to[short] ++(0,-0.5) to[R=$R_G$] (posbreak |- Gn) to[short,-o] (C |- Gn) coordinate(VOGND)
	(VOGND) to [open,v<=$V_o$] (C)
	(Gn) to[short] (posbreak |- Gn)
;
	\draw (posbreak |- Gn) to[short, *-] ++(0,-0.5) node[tlground] {};

\node at ([shift=({0,0.8})]$(v2)!0.45!(vn) $) [draw, circle, fill, minimum size = 0.5mm, radius=5mm, inner sep=0mm, outer sep=1mm] (tmiddle) {} ;
\node at ([shift=({0,0.8})]$(v2)!0.4!(vn)$) [draw, circle, fill, minimum size = 0.5mm, radius=5mm, inner sep=0mm, outer sep=1mm] (tleft) {} ;
\node at ([shift=({0,0.8})]$(v2)!0.5!(vn)$) [draw, circle, fill, minimum size = 0.5mm, radius=5mm, inner sep=0mm, outer sep=1mm] (tright) {} ;

\end{tikzpicture}
}
\caption{Target operational amplified filter-mixer circuit.}
\label{fig:amplifier_example}
\end{figure}
%>>>

\begin{theorem}[Norton's Theorem for Dynamic Phasors]\label{theo:norton} Consider a two-port circuit network composed of resistors, capacitors and inductors, nonstationary sinusoidal voltage sources and nonstationary current sources. Then the current through the two ports can be written as $I(t) = I_S(t) - \mathbf{Y}\left[V\right]$, $I_S$ the short-circuit current of the network, $V(t)$ the voltage across its terminals, $\mathbf{Y}$ the equivalent admittance operator obtained by substituting all voltage sources by short circuits and all current sources by open circuits. Further, $\mathbf{Y}$ is such that $I_S(t) = \mathbf{Y}\left[V_0\right]$, $V_0$ the open-circuit voltage of the network.
\end{theorem}
\textbf{Proof:} using the current portion \eqref{theo:super_generic_curr} of the superposition principle, and applying the same technique as the proof of the Thèvenin Theorem, or using the voltage and current source equivalence (lemma \ref{theo:vsi_equiv}) on the results of Thèvenin's Theorem.

%-------------------------------------------------
\section{Example application}\label{sec:example_application} %<<<1

%-------------------------------------------------
\subsection{Target circuit} %<<<2
	
	Figure \ref{fig:amplifier_example} shows a first-order filter-mixer circuit based on a commonplace operational amplifier topology with multiple inputs $v_1(t),v_2(t),...,v_n(t)$. We imagine that the many inputs have equivalent frequencies, so a common frequency $\omega(t)$ can be adoptead and each input can be written as

\begin{equation} v_k(t) = m_k(t)\cos\left(\psi(t) + \phi_k(t)\right), \label{eq:input_vis}\end{equation}

	\noindent with $m_k(t),\phi_k(t)$ known and $\psi(t)= \int_0^t \omega(s)ds$. Immediately one draws the Dynamic Phasor representation of the inputs as

\begin{equation} V_k(t) = m_k(t)e^{j\phi_k(t)}. \label{eq:phasor_input_vis}\end{equation}

	Let us assume the operational amplifier has a linear open-loop dynamic model of $v_o$ as a function of the difference $v_p(t) - v_n(t)$ — there is some linear operator $\mathbf{A}$ that is a composition of $\dpo$ and can be expressed phasorially as $V_o = \mathbf{A}\left[V_p - V_n\right]$. For instance, classical first-order delay models such as $g_1\dot{v}_o + g_0 v_o = v_p(t) - v_n(t)$ for two real numbers $g_1,g_0$ are common; such models yield

\begin{equation} V_0 = \left(\dfrac{\mathbf{I}}{g_1 \dpo + g_0 \mathbf{I}}\right)\left[V_p - V_n\right] .\label{eq:first_order_openloop}\end{equation}

	Let us also consider that the input ports are related by an input impedance operator $\mathbf{Z}_A$, completing the Dynamic Phasor equivalent model of the operational amplifier as shown in figure \ref{fig:amplifier_model}.

% OPAMP EQUIVALENT MODEL <<<
\begin{figure}[t]
\centering
\scalebox{0.75}{
\begin{tikzpicture}[american,scale=1,transform shape,line width=0.75, cute inductors, voltage shift = 0.2]
\ctikzset{/tikz/circuitikz/voltage/distance from node=10mm,amplifiers/scale = 1.5, bipoles/thickness=1.5, bipole label style/.style={font=\Large}, bipole voltage style/.style={font=\Large}, bipole current style/.style={font=\Large}}
\draw (0, 0) node[plain amp,scale = 2] (A) {};
\draw (A.bin up) -- ++(1,0) coordinate (tmp) to [generic, scale=1, l=$\mathbf{Z_A}$] (tmp |- A.bin down) -- (A.bin down);
\draw (A.bout) -- ++(-1,0) to [cvsourceAM,csources/scale=1.25,fill=stewartblue!50] ++ (-1.25,0) to [short] ++(-0.5,0) to[short] ++(0,-2) node[ground,scale=2] {} ;
\node[ocirc] at (A.+) {};
\node[below] at ($(A.+)!0.5!(A.bin down)$) {\Large $+$};
\node[above] at ($(A.-)!0.5!(A.bin up)$)   {\Large $-$};
\node[left] at (A.+) {\Large $V_p$};
\node[left] at (A.-) {\Large $V_n$};
\node[ocirc] at (A.-) {};
\node[ocirc] at (A.out) {};

\node[right] at (A.out) {\Large $V_o = \mathbf{A}\left[V_p - V_n\right]$};
\end{tikzpicture}
}
\caption{Op-amp Dynamic Phasor equivalent model.}
\label{fig:amplifier_model}
\end{figure} %>>>

	We also assume that the input impedance $\mathbf{Z_A}$ is a Dynamic Phasor Impedance

\begin{equation} \mathbf{Z_A} = \dfrac{\displaystyle\sum_{k=0}^n q_k\ndpo{k}}{\displaystyle\sum_{k=0}^d p_k\ndpo{k}} . \label{eq:impedance_za}\end{equation}

	In an ideal op-amp, $\left\lVert \mathbf{A}\right\rVert \to \infty$; intuitively, a higher $\left\lVert \mathbf{A}\right\rVert$ means that $\left\lVert V_o\right\rVert$ gets higher when $\left\lVert V_p - V_n\right\rVert$ is constant. By equation \eqref{eq:inverse_poly_norm}, this can be achieved with $g_1,g_0 \to 0$ in the case of \eqref{eq:first_order_openloop}.

	Also in the ideal model, $\left\lVert \mathbf{Z}_A\right\rVert \to \infty$, thus ``tending to an open circuit''. According to subsection \eqref{subsec:topology_dpfs}, this can be achieved and is a well-defined concept. Intuitively, this means that the input currents of the inverting and noninverting ports get smaller in size as $\left\lVert V_p - V_n\right\rVert$ is kept. As discussed on subsection \ref{subsec:topology_dpfs}, having an impedance operator with arbitrarily large (thus infinitely growing) norm entails to having a numerator operator with arbitrarily large coefficients or a denominator operator with arbitrarily small coefficients. In the case of the assumed model \eqref{eq:impedance_za}, this means

\begin{equation} \left\lVert \mathbf{Z_A}\right\rVert \to \infty \Leftrightarrow \left\lvert q_k\right\rvert \to \infty \text{ for all } 0 \leq k \leq n \text{ and } \left\lvert p_k\right\rvert \to 0 \text{ for all } 0 \leq k \leq d \end{equation}

	\noindent or equivalently maintaining $\left\lvert q_k\right\rvert$ constant and tending the $\left\lvert p_k\right\rvert$ to zero, or conversely tending the $\left\lvert q_k\right\rvert$ to infinity and maintaining the $\left\lvert p_k\right\rvert$ constant.

	The circuit has $n$ inputs $v_1$ through $v_n$, which are supposed to be nonstationary sinusoids, and a frequency signal $\omega(t)$ is picked. The objectives are:

\begin{itemize}
	\item Obtain the expression of the gain operator $\mathbf{G}_k$ that relates the contribution of an input $V_k$ to the output voltage $V_o$, first as a generalized expression and then as a particular ideal scenario;
	\item Find the input impedance seen by each input voltage source, that is, the operator $\mathbf{Z}_k$ that relates an input $V_k$ to the current $I_k$, first as a generalized expression and then as a particular ideal scenario;
	\item Retrieve the time-domain differential equations of $v_o^k(t)$ and $i_k(t)$ with respect to $v(t)$;
	%\item Simulate the system in both phasorial and time domain settings and compare the results.
\end{itemize}

%-------------------------------------------------
\subsection{Dynamic Phasor domain modelling} %<<<2

	By the Superposition Principle of theorem \ref{theo:superposition}, to obtain the output voltage contribution $v_o^k$ for a particular input source $v_k$ first one substitutes all other outputs as short-circuits, yielding the ``individual'' i-th equivalent circuit of figure \ref{fig:amplifier_example_individual}. The quantities $V_n^k$, $V_p^k$, $V_o^k$, $I_F^k$, $I_G^k$ are the inverting, non-inverting input and output voltages, current through the feedback net and current through $R_G$ due to the i-th input. $R_p^k$ is the equivalent resistance obtained by the parallel equivalent of all $R_1,R_2,...,R_n$ excluding $R_k$.

% EQUIVALENT "INDIVIDUAL" CIRCUIT <<<
\begin{figure} 
\centering
\scalebox{0.75}{
\begin{tikzpicture}[american,scale=1,transform shape,line width=0.75, cute inductors, voltage shift = 0.2]
\ctikzset{/tikz/circuitikz/voltage/distance from node=10mm,amplifiers/scale = 1.5, bipoles/thickness=1.5, bipole label style/.style={font=\Large}, bipole voltage style/.style={font=\Large}, bipole current style/.style={font=\Large}}
\draw
	(0, -3) coordinate(n1) to[R,l=$R_P^k$] (0,3) coordinate(n2)
	(n2) to[short] ++(2,0) coordinate (n3) to[R,l=$R_k$] ++(0,-2.5) to[vsource,sources/scale=1.25,i<_=$I_{i}$, v>=$V_k$, fill=stewartpink!50]  ++(0,-3) |- (n1)
	(n3.center) to[short,i=$I_A^k$] ++(3,0) node (n4) {} to[short,i=$I_G^k$,-*] ++(0,-1.5) node[left=2mm] (VP) {\Large$V_P^k$} to[generic,l=$\mathbf{Z_A}$,-*] ++(0,-2) node[left=2mm] (VN) {\Large$V_N^k$} to[short] ++(0,-0.3) to[R,l=$R_G$] ++(0,-1.5) |- (n3 |- n1)
	(n4.center) to[short,i=$I_F^k$] ++(2,0) coordinate(n5) to[short] ++(1,0) to[R,l=$R_F$] ++(1,0) to[short] ++(1,0) coordinate(n6) to[short] ++(1,0) coordinate(n7) to[cvsourceAM,csources/scale=1.25,v^=\mbox{$\mathbf{A}\left[V_P^k - V_n^k\right]$},fill=stewartblue!50] (n7 |- n1) coordinate(gndin) to[short] (n4 |- n1)
	(n5.center) |- ++(0.5,-1.5) to[C,l_=$\dfrac{\mathbf{I}}{\dpo C_F}$] ++(2,0) -| (n6.center)
	(n7.center) to[short,-o] ++(5,0) coordinate(vout)
	(n7 |- n1) to[short,-o]  (n1 -| vout) coordinate(voutgnd)
	(voutgnd) to [open,v<=$V_o^k$] (vout);
;
	\draw (gndin) to[short, *-] ++(0,-0.5) node[tlground] {};
\end{tikzpicture}
}
\caption{``Individual'' version of the operational amplifier circuit of figure \ref{fig:amplifier_example}.}
\label{fig:amplifier_example_individual}
\end{figure}
%>>>

	Using Kirchoff's Laws (theorems \ref{theo:kirchoff_voltage} and \ref{theo:kirchoff_current}), the Dynamic Impedance relationships, as well as the series-parallel combination formulas \eqref{eq:impedance_div} and \eqref{eq:admittance_div}, one arrives at \eqref{sys:example_initial_modelling_1} - \eqref{sys:example_initial_modelling_5}.

\begin{align}
	I_A^k &= I_F^k + I_G^k \label{sys:example_initial_modelling_1} \\[3mm]
	I_i   &= I_A^k + \left(R_P^k\mathbf{I}\right)^{-1}\left[ V_n^k\right] \label{sys:example_initial_modelling_2}\\[5mm]
	V_n^k &= V_k - \left(R_k\mathbf{I}\right) \left[I_i\right] \label{sys:example_initial_modelling_3}\\[3mm]
	V_n^k - V_o^k &= \left(\dfrac{R_F\mathbf{I}}{R_FC_F \dpo + \mathbf{I}}\right) \left[I_F^k\right] \label{sys:example_initial_modelling_4}\\[5mm]
	V_n^k &= \left(R_G\mathbf{I} + \mathbf{Z}_A\right) \left[I_G^k\right]  \label{sys:example_initial_modelling_5}
\end{align}

	One can isolate $I_A^k$ in terms of $V_k$ and $V_n^k$ from \eqref{sys:example_initial_modelling_2} and \eqref{sys:example_initial_modelling_3}. $I_F^k$ and $I_G^k$ can be isolated from the three bottom equation and substitute on the top one:

\begin{gather}
	\dfrac{V_k - V_n^k}{R_k} - V_n^k \dfrac{1}{R_P^k} = \left(\dfrac{R_FC_F \dpo + \mathbf{I}}{C_F\dpo}\right) \left[V_n^k - V_o^k\right] + \left( \dfrac{\mathbf{I}}{R_G + \mathbf{Z}_A}\right) \left[V_n^k\right] \\[3mm]
	V_k = \left(\dfrac{R_k}{R_P}\mathbf{I} + R_k\dfrac{R_FC_F \dpo + \mathbf{I}}{R_F} + \dfrac{R_k\mathbf{I}}{R_G\mathbf{I} + \mathbf{Z}_A}\right)\left[V_n^k \right] - R_k\left(\dfrac{R_FC_F \dpo + \mathbf{I}}{R_F}\right) \left[V_o^k \right], \label{eq:amplifier_general_kirchoff}
\end{gather}

	\noindent with $R_P$ the parallel combination of all $R_1,...,R_n$ including $R_k$. Now using the impedance divider equations \eqref{eq:impedance_div}:

\begin{equation} V_p^k = \left(\dfrac{R_G\mathbf{I}}{R_G\mathbf{I} + \mathbf{Z}_A}\right) \left[V_n^k\right] \end{equation}

	Substituting the inverse relation onto the output model of the op-amp, one obtains

\begin{equation} V_o^k = -\left(\dfrac{\mathbf{A}\mathbf{Z}_A}{R_G\mathbf{I} + \mathbf{Z}_A}\right) \left[V_n^k\right] \label{eq:vpi_vi} \end{equation}

	Substitute this equation into \eqref{eq:amplifier_general_kirchoff} to obtain an equation $V_o^k = \mathbf{G}_k \left[V_k\right]$ for the gain operator sought:

\begin{gather}
	V_k = -\left(\dfrac{R_k}{R_P}\mathbf{I} + R_k\dfrac{R_FC_F \dpo + \mathbf{I}}{R_F} + \dfrac{R_k\mathbf{I}}{R_G\mathbf{I} + \mathbf{Z}_A}\right)\left(\dfrac{\mathbf{A}\mathbf{Z}_A}{R_G\mathbf{I} + \mathbf{Z}_A}\right)^{-1} \left[V_o^k \right] - R_k\left(\dfrac{R_FC_F \dpo + \mathbf{I}}{R_F}\right) \left[V_o^k \right] \nonumber\\[3mm]
	V_k = \underbrace{-\left[\left(\dfrac{R_k}{R_P}\mathbf{I} + R_k\dfrac{R_FC_F \dpo + \mathbf{I}}{R_F} + \dfrac{R_k\mathbf{I}}{R_G\mathbf{I} + \mathbf{Z}_A}\right)\left(\dfrac{\mathbf{A}\mathbf{Z}_A}{R_G\mathbf{I} + \mathbf{Z}_A}\right)^{-1} + R_k\left(\dfrac{R_FC_F \dpo + \mathbf{I}}{R_F}\right)\right] }_{\mathbf{G}_k} \left[V_o^k \right]  \label{eq:total_gain_gi}
\end{gather} 

	Now we apply idealized the model $\left\lVert\mathbf{A}\right\rVert,\left\lVert\mathbf{Z_A}\right\rVert\to\infty $. In order to do this, we use the concepts of limits in normed functional spaces which, despite the complexity, become simple to use in this case due to the elementary functions involved:

\begin{align}
	  & \lim\limits_{\left\lVert\mathbf{A}\right\rVert,\left\lVert\mathbf{Z_A}\right\rVert\to\infty } \mathbf{G}_k = \nonumber\\[3mm]
	= & \lim\limits_{\left\lVert\mathbf{A}\right\rVert,\left\lVert\mathbf{Z_A}\right\rVert\to\infty } -\left[\left(\dfrac{R_k}{R_P}\mathbf{I} + R_k\dfrac{R_FC_F \dpo + \mathbf{I}}{R_F} + \dfrac{R_k\mathbf{I}}{R_G\mathbf{I} + \mathbf{Z}_A}\right)\left(\dfrac{\mathbf{A}\mathbf{Z}_A}{R_G\mathbf{I} + \mathbf{Z}_A}\right)^{-1} + R_k\left(\dfrac{R_FC_F \dpo + \mathbf{I}}{R_F}\right)\right] \nonumber\\[3mm]
	= & \lim\limits_{\left\lVert\mathbf{A}\right\rVert,\left\lVert\mathbf{Z_A}\right\rVert\to\infty } -\left[\left(\dfrac{R_k}{R_P}\mathbf{I} + R_k\dfrac{R_FC_F \dpo + \mathbf{I}}{R_F} + \dfrac{R_k\mathbf{I}}{R_G\mathbf{I} + \mathbf{Z}_A}\right)\left(\dfrac{R_G\mathbf{I} + \mathbf{Z}_A}{\mathbf{A}\mathbf{Z}_A}\right) + R_k\left(\dfrac{R_FC_F \dpo + \mathbf{I}}{R_F}\right)\right] \nonumber\\[3mm]
	= & \lim\limits_{\left\lVert\mathbf{A}\right\rVert,\left\lVert\mathbf{Z_A}\right\rVert\to\infty }  - \raisebox{-4mm}{$ \left[\raisebox{4mm}{$ \underbrace{\left(\dfrac{R_k}{R_P}\mathbf{I} + R_k\dfrac{R_FC_F \dpo + \mathbf{I}}{R_F} + \dfrac{R_k\mathbf{I}}{R_G\mathbf{I} + \mathbf{Z}_A}\right)}_{L_1} \underbrace{\left(\dfrac{R_G\mathbf{Z}_A^{-1} + \mathbf{I}}{\mathbf{A}}\right)}_{L_2} + \underbrace{R_k\left(\dfrac{R_FC_F \dpo + \mathbf{I}}{R_F}\right)}_{L_3} $}\right] $} \label{eq:original_limit}
\end{align}

	Now breaking down this limit: we use the properties of limits in normed vector spaces that the multiplication of limits is the limit of the multiplication, as well as the linearity of the limits. This expression then breaks down into the limit of three expressions; let's call them $L_1$, $L_2$, $L_3$. Then

\begin{equation} L_1 = \lim\limits_{\left\lVert\mathbf{A}\right\rVert,\left\lVert\mathbf{Z_A}\right\rVert\to\infty } \dfrac{R_k}{R_P}\mathbf{I} + R_k\dfrac{R_FC_F \dpo + \mathbf{I}}{R_F} + \dfrac{R_k\mathbf{I}}{R_G\mathbf{I} + \mathbf{Z}_A} \end{equation}

	The last term of this limit tends to zero because the denominator has a norm that tends to infinity:

\begin{equation} L_1 = \lim\limits_{\left\lVert\mathbf{A}\right\rVert,\left\lVert\mathbf{Z_A}\right\rVert\to\infty } \dfrac{R_k}{R_P}\mathbf{I} + R_k\dfrac{R_FC_F \dpo + \mathbf{I}}{R_F} + \cancelto{0}{\dfrac{R_k\mathbf{I}}{R_G\mathbf{I} + \mathbf{Z}_A}} = \dfrac{R_k}{R_P}\mathbf{I} + R_k\dfrac{R_FC_F \dpo + \mathbf{I}}{R_F} .\end{equation}

	For the second limit of \eqref{eq:original_limit},

\begin{equation} L_2 = \lim\limits_{\left\lVert\mathbf{A}\right\rVert,\left\lVert\mathbf{Z_A}\right\rVert\to\infty } \dfrac{R_G\mathbf{Z}_A^{-1} + \mathbf{I}}{\mathbf{A}}  \end{equation}

	\noindent which is zero because the numerator tends to $\mathbf{I}$ while the norm of the denominator tends to $\infty$. Finally for the third limit of \eqref{eq:original_limit},

\begin{equation}
	L_3 = \lim\limits_{\left\lVert\mathbf{A}\right\rVert,\left\lVert\mathbf{Z_A}\right\rVert\to\infty } R_k\left(\dfrac{R_FC_F \dpo + \mathbf{I}}{R_F}\right) = R_k\left(\dfrac{R_FC_F \dpo + \mathbf{I}}{R_F}\right) ,
\end{equation}

	\noindent an obvious result because the expression inside the limit does not depend on $\mathbf{A}$ nor on $\mathbf{Z}_A$. Thus

\begin{align}
	\lim\limits_{\left\lVert\mathbf{A}\right\rVert,\left\lVert\mathbf{Z_A}\right\rVert\to\infty } \mathbf{G}_k &= -\left(L_1\times L_2 + L_3\right) = -\left(\dfrac{R_k}{R_P}\mathbf{I} + R_k\dfrac{R_FC_F \dpo + \mathbf{I}}{R_F}\right) \times \mathbf{0} - R_k\left(\dfrac{R_FC_F \dpo + \mathbf{I}}{R_F}\right) = \nonumber\\[3mm]
	&= -R_k\left(\dfrac{R_FC_F \dpo + \mathbf{I}}{R_F}\right),
\end{align}

	\noindent which is shorthand to

\begin{equation} -V_k = R_k\left( C_F \dpo + \dfrac{1}{R_F} \mathbf{I} \right) \left[ V_o^k\right] \label{eq:vo_vi_gi_diff}\end{equation}

	Using this idealized version and definition of $\dpo$ one arrives at the complex phasorial DE for $V_o^k$

\begin{equation} -V_k = R_kC_F\left(\dot{V}_o^k + j\omega V_o^k\right) +\dfrac{R_k}{R_F}V_o^k = R_kC_F\dot{V}_o^k + V_o^k\left(j\omega R_kC_F + \dfrac{R_k}{R_F}\right) .\label{eq:diffeq_gi_voi}\end{equation}

	TThis modelling holds even for fringe cases. herefore, given the $n$ signals $v_k$ and $\omega$, the DP signals $V_k$ can each be obtained by integrating its equation \eqref{eq:diffeq_gi_voi}. The DP of the total output voltage $V_o$ is obtained by the Superposition Theorem as

\begin{equation} V_o = \sum_{i=1}^n V_o^k = \sum_{i=1}^n \mathbf{G}_k \left[V_k\right] .\label{eq:vo_sum} \end{equation}

	Now, we calculate the input impedance $\mathbf{Z}$. Take equation \eqref{eq:total_gain_gi} and substitute $V_o^k$ from \eqref{eq:vpi_vi}

\begin{align}
	V_k &= -\left[\left(\dfrac{R_k}{R_P}\mathbf{I} + R_k\dfrac{R_FC_F \dpo + \mathbf{I}}{R_F} + \dfrac{R_k\mathbf{I}}{R_G\mathbf{I} + \mathbf{Z}_A}\right)\left(\dfrac{\mathbf{A}\mathbf{Z}_A}{R_G\mathbf{I} + \mathbf{Z}_A}\right)^{-1} + R_k\left(\dfrac{R_FC_F \dpo + \mathbf{I}}{R_F}\right)\right]\left(\dfrac{\mathbf{A}\mathbf{Z}_A}{R_G\mathbf{I} + \mathbf{Z}_A}\right) \left[V_n^k\right] \nonumber\\[3mm]
	&= -\left[\left(\dfrac{R_k}{R_P}\mathbf{I} + R_k\dfrac{R_FC_F \dpo + \mathbf{I}}{R_F} + \dfrac{R_k\mathbf{I}}{R_G\mathbf{I} + \mathbf{Z}_A}\right) + R_k\left(\dfrac{R_FC_F \dpo + \mathbf{I}}{R_F}\right)\left(\dfrac{\mathbf{A}\mathbf{Z}_A}{R_G\mathbf{I} + \mathbf{Z}_A}\right)\right] \left[V_n^k\right]
\end{align}

	\noindent and substituting this into \eqref{sys:example_initial_modelling_3},

\begin{gather}
	-\left[\left(\dfrac{R_k}{R_P}\mathbf{I} + R_k\dfrac{R_FC_F \dpo + \mathbf{I}}{R_F} + \dfrac{R_k\mathbf{I}}{R_G\mathbf{I} + \mathbf{Z}_A}\right) + R_k\left(\dfrac{R_FC_F \dpo + \mathbf{I}}{R_F}\right)\left(\dfrac{\mathbf{A}\mathbf{Z}_A}{R_G\mathbf{I} + \mathbf{Z}_A}\right)\right]^{-1} \left[V_k\right] = V_k - \left(R_k\mathbf{I}\right) \left[I_i\right] \nonumber\\[3mm]
%
	\left\{\mathbf{I} + \left[\left(\dfrac{R_k}{R_P}\mathbf{I} + R_k\dfrac{R_FC_F \dpo + \mathbf{I}}{R_F} + \dfrac{R_k\mathbf{I}}{R_G\mathbf{I} + \mathbf{Z}_A}\right) + R_k\left(\dfrac{R_FC_F \dpo + \mathbf{I}}{R_F}\right)\left(\dfrac{\mathbf{A}\mathbf{Z}_A}{R_G\mathbf{I} + \mathbf{Z}_A}\right)\right]^{-1}\right\} \left[V_n^k\right] = \left(R_k\mathbf{I}\right) \left[I_i\right] \nonumber\\[3mm]
%
	\mathbf{Z}_k = \dfrac{R_k\mathbf{I}}{\left\{\mathbf{I} + \left[\left(\dfrac{R_k}{R_P}\mathbf{I} + R_k\dfrac{R_FC_F \dpo + \mathbf{I}}{R_F} + \dfrac{R_k\mathbf{I}}{R_G\mathbf{I} + \mathbf{Z}_A}\right) + R_k\left(\dfrac{R_FC_F \dpo + \mathbf{I}}{R_F}\right)\left(\dfrac{\mathbf{A}\mathbf{Z}_A}{R_G\mathbf{I} + \mathbf{Z}_A}\right)\right]^{-1}\right\}}
\end{gather}

	Alternatively, one can obtain $\mathbf{Z}_k$ by using Thèvenin's Theorem for Dynamic Phasors (theorem \ref{theo:thevenin}): on figure \ref{fig:amplifier_example_individual}, substitute $V_k$ by a short-circuit and calculate $I_i^S$ as the short-circuit current; then substitute $V_k$ by an open circuit and calculate the open circuit voltage $V_k^O$ on the terminals

	Now we apply idealized the model $\left\lVert\mathbf{A}\right\rVert,\left\lVert\mathbf{Z_A}\right\rVert\to\infty$, but this time we take a more practical approach to calculating the limit of this expression. We separate the denominator in three colored pieces:

\begin{equation}
	\mathbf{Z}_k = \dfrac{R_k\mathbf{I}}{\left\{\mathbf{I} + \left[{\color{stewartblue} \left(\dfrac{R_k}{R_P}\mathbf{I} + R_k\dfrac{R_FC \dpo + \mathbf{I}}{R_F} + \dfrac{R_k\mathbf{I}}{R_G\mathbf{I} + \mathbf{Z}_A}\right)} + {\color{stewartgreen} R_k\left(\dfrac{R_FC \dpo + \mathbf{I}}{R_F}\right)}{\color{stewartpink} \left(\dfrac{\mathbf{A}\mathbf{Z}_A}{R_G\mathbf{I} + \mathbf{Z}_A}\right)}\right]^{-1}\right\}}
\end{equation}

% EQUIVALENT THEVENIN CIRCUIT <<<
\begin{figure}[t]
\centering
\scalebox{0.75}{
\begin{tikzpicture}[american,scale=1,transform shape,line width=0.75, cute inductors, voltage shift = 0.2]
\ctikzset{/tikz/circuitikz/voltage/distance from node=10mm,amplifiers/scale = 1.5, bipoles/thickness=1.5, bipole label style/.style={font=\Large}, bipole voltage style/.style={font=\Large}, bipole current style/.style={font=\Large}}
\draw (0,0) coordinate(origin) to[vsource,sources/scale=1.25,i<_=$I_{i}$, v_=$V_1$, fill=stewartpink!50]  ++(0,-3) to[short] ++(4,0) to [generic, l_=$\mathbf{Z}_1$] ++(0,3) to[short] (origin);
\draw (0,-4) coordinate(origin) to[vsource,sources/scale=1.25,i<_=$I_{2}$, v_=$V_2$, fill=stewartpink!50]  ++(0,-3) coordinate(dotbegin) to[short] ++(4,0) to [generic, l_=$\mathbf{Z}_2$] ++(0,3) to[short] (origin);
\draw (0,-9) coordinate(origin) to[vsource,sources/scale=1.25,i<_=$I_{n}$, v_=$V_n$, fill=stewartpink!50]  ++(0,-3) to[short] ++(4,0) to [generic, l_=$\mathbf{Z}_n$] ++(0,3) to[short] (origin);
\node at ([shift=({0,0.8})]$(dotbegin.center)!0.75 !(origin.center) $) [draw, circle, fill, minimum size = 0.5mm, radius=5mm, inner sep=0mm, outer sep=1mm] (tmiddle) {} ;
\node at ([shift=({0,0.8})]$(dotbegin.center)!0.85 !(origin.center)$) [draw, circle, fill, minimum size = 0.5mm, radius=5mm, inner sep=0mm, outer sep=1mm] (tleft) {} ;
\node at ([shift=({0,0.8})]$(dotbegin.center)!0.95 !(origin.center)$) [draw, circle, fill, minimum size = 0.5mm, radius=5mm, inner sep=0mm, outer sep=1mm] (tright) {} ;

\draw (12,-12) coordinate(vostart) to [short,o-] ++(-5,0) to[cvsourceAM,csources/scale=1.25,v_=\mbox{$\mathbf{G}_n\left[V_n\right]$},fill=stewartblue!50, invert] ++(0,3) coordinate(vstart);
\draw (7,-7) coordinate(vend) to[cvsourceAM,csources/scale=1.25,v_=\mbox{$\mathbf{G}_2\left[V_2\right]$},fill=stewartblue!50, invert] ++(0,3) to [short] ++(0,1) to[cvsourceAM,csources/scale=1.25,v_=\mbox{$\mathbf{G}_1\left[V_1\right]$},fill=stewartblue!50, invert] ++(0,3) to[short,-o] ++(5,0) coordinate (voend);

\node at ([shift=({0,0.8})]$(vstart)!0.00!(vend)$) [draw, circle, fill, minimum size = 0.5mm, radius=5mm, inner sep=0mm, outer sep=1mm] (tmiddle) {} ;
\node at ([shift=({0,0.8})]$(vstart)!0.10!(vend)$) [draw, circle, fill, minimum size = 0.5mm, radius=5mm, inner sep=0mm, outer sep=1mm] (tleft) {} ;
\node at ([shift=({0,0.8})]$(vstart)!0.20!(vend)$) [draw, circle, fill, minimum size = 0.5mm, radius=5mm, inner sep=0mm, outer sep=1mm] (tright) {} ;

\draw (voend) to[open,v=$V_o$] (vostart);

\draw [draw=black,ultra thick] (2,-13) rectangle (11,1);
\end{tikzpicture}
}
\caption{Thèvenin equivalent circuit of the operational amplifier circuit of figure \ref{fig:amplifier_example} as a ``black box''.}
\label{fig:amplifier_thevenin}
\end{figure}
%>>>

	The norm of the pink piece clearly tends to infinity, while the green piece stays unchanged due to not depending on $\mathbf{A}$ nor $\mathbf{Z}_A$. The blue piece is such that the first two summands remain constant, while the third tends to zero norm. Thus the entire expression in brackets being inverted tends to infinite norm and its inverse tends to the null operator, yielding

\begin{equation} \lim\limits_{\left\lVert\mathbf{A}\right\rVert,\left\lVert\mathbf{Z_A}\right\rVert\to\infty } \mathbf{Z}_k = R_k\mathbf{I} \end{equation}

	\noindent and this equation means that each input voltage $V_k$ contributes a current $I_k = V_k/R_k$; in the phasorial domain. But because $\left\lVert \mathbf{Z}_k\right\rVert \to \infty$ implies $I_n\to 0$, then the current through the feedback branch $I_F$ is the sum of the $I_k$:

\begin{equation} I_F = \sum\limits_{k=0}^n I_k = \sum\limits_{k=0}^n \dfrac{1}{R_k} V_k \label{eq:if_ideal} \end{equation}

%-------------------------------------------------
\subsection{Correlation with the time domain} %<<<2

	To obtain the correspondent time-domain model, one uses the equivalence between $\dpo$ and derivatives on \eqref{eq:vo_vi_gi_diff} to yield

\begin{equation} -v_k = R_k C_F \dot{v}_o^k + \dfrac{R_k}{R_F}v_o^k \label{eq:vo_vi_gi_diff_time}\end{equation}

	\noindent and because the input impedance as seen by the $k$-th source is $\mathbf{Z}_k = R_k$, then

\begin{equation} i_k = \dfrac{v_k(t)}{R} \label{eq:ik_diff_time}\end{equation}

	\noindent meaning one can solve \eqref{eq:vo_vi_gi_diff_time} for $v_k$ and obtain $i_k = v_k/R_k$.

	One naturally asks if these time domain equations \eqref{eq:vo_vi_gi_diff_time} and \eqref{eq:ik_diff_time} indeed are the same equations that would be obtained if the circuit were modelled in the time domain. It is left to the reader to prove that a time-domain modelling achieves exactly these equations, that is, the time-domain model can be obtained without losses from the Dynamic Phasor model. This is done by modelling the target circuit of figure \ref{fig:amplifier_example} in the time domain using the equivalent differential equations of the operational amplifier gain and impedance

\begin{gather}
	g_1 \dot{v}_o + g_0 v_o = v_p(t) - v_n(t) \\[5mm]
	\sum_{k=0}^n q_k\dfrac{d^k}{dt^k}\left[v_p(t) - v_n(t)\right] = \sum_{k=0}^d p_k\dfrac{d^k}{dt^k} i_n(t) \\[5mm]
	i_n(t) = i_p(t)
\end{gather}

	\noindent which Dynamic Phasor Functional versions are exactly \eqref{eq:first_order_openloop} and \eqref{eq:impedance_za} previously used. Then, applying the idealized conditions

\begin{equation} \left\{\begin{array}{l} \left\lvert q_k\right\rvert \to \infty \text{ for all } 0 \leq k \leq n \text{ and } \left\lvert p_k\right\rvert \to 0 \text{ for all } 0 \leq k \leq d \\[5mm] \left\lvert g_1\right\rvert,\left\lvert g_0\right\rvert \to 0\end{array}\right.\end{equation}

	\noindent one achieves the exact same ideal model of equations \eqref{eq:vo_vi_gi_diff_time} and \eqref{eq:ik_diff_time} achieved using DPFs.

%-------------------------------------------------
\subsection{Simulation} %<<<2

	We now simulate the system. We suppose the system has two inputs, $v_1(t)$ and $v_2(t)$. The first input is modulated in amplitude but of fixed frequency:

\begin{equation} v_1 = m_1(t)\cos\left(\theta_1(t)\right) \left\{\begin{array}{l} m_1(t) = m\left[1 + M_1e^{-\alpha_1 t}\sin\left(\beta_1 t\right)\right]  \\ \theta_1(t) = \omega_0 t \end{array}\right.\end{equation}

	\noindent where $m$ is a base amplitude, $\omega_0$ a base frequency. The second input $v_2(t)$, on the other hand, is modulated in frequency, such that it has constant amplitude but a varying frequency

\begin{equation}\hspace{-2mm} v_2(t) = m_2(t)\cos\left(\theta_2(t)\right)\left\{\begin{array}{l} m_2(t) = m \\[5mm] \theta_2 = \displaystyle\int_0^a \omega_2(a)da \text{, where } \omega_2(t) = \omega_0\left[1 + M_2e^{-\alpha_2 t}\sin\left(\beta_2 t\right)\right]\end{array}\right. . \end{equation}

	The numerical values adopted are $\omega_0=120\pi$ rad/s, $\alpha_1=5s^{-1},\beta_1=10\pi\ \text{rad}.s^{-1},M_1=0.2$ for the first input and $\alpha_2=10s^{-1},\beta_2=20\pi\ \text{rad}.s^{-1},M_2=1$ for the second input. The input resistors are $R_1 = 1k\Omega,\ R_2 = 2k\Omega$ and the filter components are $R_F = 1k\Omega$ and $C_F = 10\mu F$.

	To build the Dynamic Phasor model we use the base frequency $\omega_0$ for the DPT arriving at

\begin{equation}\left\{\begin{array}{l} V_1 = m_1e^{j\phi_1} = m\left[1 + M_1e^{-\alpha_1 t}\sin\left(\beta_1 t\right)\right]e^{j0} \\[5mm] V_2 = m_2e^{j\phi_2} \text{ where } \phi_2 = \dfrac{\omega_0 M_2\left\{\beta_2 - e^{-\alpha_2 t}\left[\alpha_2\sin\left(\beta_2 t\right) + \beta_2\cos\left(\beta_2 t\right)\right]\right\}}{\alpha_2^2 + \beta_2^2} \end{array}\right. \end{equation}

	\noindent yielding the differential equations for the contributions

\begin{equation} -m_ke^{j\phi_k} = R_k\left( C_F \dpo + \dfrac{1}{R_F} \mathbf{I} \right) \left[ V_o^k\right] = R_k\left[C_F\dot{V}_o^k + V_o^k\left(j\omega_0C_F + \dfrac{1}{R_F}\right)\right],\ k=1,2 \label{eq:dpo_sim_dpmodel} \end{equation}

	\noindent and the time differential equations are obtained by separating these equations into real and imaginary parts. The Dynamic Phasor of the output is given by $V_o = V_o^1 + V_o^2$, and the reconstructed output is given by $v_o^{\text{DP}} = \Re\left(V_oe^{j\omega_0 t}\right)$, the superscript ``DP'' to highlight this is the output voltage reconstructed from Dynamic Phasors. At the same time, the time model is given by

\begin{equation} -m_k\cos\left(\theta_k(t)\right) = R_k\left(C_F\dot{v}_o^k + \dfrac{1}{R_F}\dot{v}_o^k\right),\ k=1,2 \label{eq:dpo_sim_timemodel} \end{equation}

	\noindent and the time-domain-obtained output is $v_o^{\text{T}} = v_o^1 + v_o^2$, the superscript ``T'' to denote this was obtained from the time-domain model.

	We now calculate the initial conditions. We assume that the system departs from permanent sinusoidal regimens — that is, $\dot{V}_o^1 = \dot{V}_o^2 = 0$ — yielding

\begin{equation} \left(V_o^1\right)_{t=0} = \dfrac{-m}{R_1\left(j\omega_0C_F + \dfrac{1}{R_F}\right)},\ \left(V_o^2\right)_{t=0} = \dfrac{-m}{R_2\left(j\omega_0C_F + \dfrac{1}{R_F}\right)} \end{equation}

	\noindent and we match the time-domain initial conditions

\begin{equation} v_1(0) = \left\lvert\left(V_o^1\right)_{t=0}\right\rvert \cos\left\{\arg\left[\left(V_o^1\right)_{t=0}\right]\right\},\ v_2(0) = \left\lvert\left(V_o^2\right)_{t=0}\right\rvert \cos\left\{\arg\left[\left(V_o^2\right)_{t=0}\right]\right\} .\end{equation}

	Figures \ref{fig:dpo_sim_dpcurves} and \ref{fig:dpo_sim_timecurves} show the simulation results. Figure \ref{fig:dpo_sim_dpcurves} shows the direct and quadrature components of the Dynamic Phasor voltage contributions $V_o^{(1,2)}$. Figure \ref{fig:dpo_sim_timecurves} shows the time signals obtained by the reconstruction of the DP simulation, $v_o^{\text{DP}}$, and obtained by directly integrating the time differential equations, $v_o^{\text{T}}$. The figure shows that these signals are identical, showing that the Dynamic Phasor model indeed reconstructs the time domain model losslessly.

% DIRECT AND QUADRADTURE COMPONENTS OF VOLTAGE CONTRIBUTIONS <<<
\begin{figure}
        \begin{center}
                \beginpgfgraphicnamed{timesim_slow}
                \begin{tikzpicture}
                        \begin{axis}[
				name = ax_main,
                                width = 1\columnwidth,
                                height = 1/1.618*\columnwidth,
                                title={Direct component of voltage contributions},
                                ylabel={$V_{od}^{(1,2)}$ (V)},
				xlabel={Time (ms)},
                                xmin=0, xmax=1,
                                ymin=-2.2, ymax=2.2,
                                xtick={0,0.1,...,1},
                                ytick={-2,-1,...,2}, 
                                legend pos=north east,
				legend cell align={left},
                                ymajorgrids=true,
                                xmajorgrids=true,
                                every axis plot/.append style={thick},
                        ]
                        \addplot[blue,  smooth] table[col sep=comma,header=false,x index=0,y index=1]{data/dpos/data_dpo_sim.csv};
			\addlegendentry{$V_{od}^1$}
                        \addplot[red, smooth] table[col sep=comma,header=false,x index=0,y index=3]{data/dpos/data_dpo_sim.csv};
			\addlegendentry{$V_{od}^2$}
                        \end{axis}
%
                        \begin{axis}[
                                name = ax_zoomed,
                                at={($(ax_main.south west)-(0,0.7*\columnwidth)$)},
                                title={Quadrature component of voltage contributions},
                                width = 1\columnwidth,
                                height = 1/1.618*\columnwidth,
                                xmin=0, xmax=1,
                                ymin=-2, ymax=6.6,
                                xtick={0,0.1,...,1},
				xlabel={Time (ms)},
                                ylabel={$V_{oq}^{(1,2)}$ (V)},
                                ytick={-2,-1,...,6},
				tick label style={/pgf/number format/fixed},
				legend cell align={left},
                                legend pos=north east,
                                ymajorgrids=true,
                                xmajorgrids=true,
                                every axis plot/.append style={thick},
                        ]
                        \addplot[blue,  smooth] table[col sep=comma,header=false,x index=0,y index=2]{data/dpos/data_dpo_sim.csv};
			\addlegendentry{$V_{oq}^1$}
                        \addplot[red, smooth] table[col sep=comma,header=false,x index=0,y index=4]{data/dpos/data_dpo_sim.csv};
			\addlegendentry{$V_{oq}^2$}
                        \end{axis}
                \end{tikzpicture}
        \endpgfgraphicnamed
        \caption
{Direct and quadrature components of the voltage contributions $V_o^{1}$ and $V_o^{2}$ as calculated using \eqref{eq:dpo_sim_dpmodel}.}
        \label{fig:dpo_sim_dpcurves}
        \end{center}
\end{figure}
% >>>

% TIME VOLTAGE CURVES <<<
\begin{figure}
        \begin{center}
                \begin{tikzpicture}
                        \begin{axis}[
				name = ax_main,
                                width = 0.9*\columnwidth,
                                height = 0.9*1/1.618*\columnwidth,
                                title={Time voltage signals},
                                xlabel={Time (s)},
                                ylabel={$v_o^{\text{DP}}(t)$ and $v_o^{\text{T}}(t)$ (V)},
                                xmin=0, xmax=1,
                                ymin=-7.2, ymax=7.5,
                                xtick={0,0.1,...,1},
                                ytick={-6,-4,...,6}, 
                                legend pos=south east,
                                ymajorgrids=true,
                                xmajorgrids=true,
                                %grid style=dashed,
                                every axis plot/.append style={thick},
				legend columns=2,
                        ]
				\addplot[blue, smooth]         table[col sep=comma,header=false,x index=0,y index=5]{data/dpos/data_dpo_sim.csv};
                        \coordinate (c1) at (axis cs:0  ,-7.2);
                        \coordinate (c2) at (axis cs:0.1,-7.2);
%
                        \coordinate (c3) at (axis cs:0.9,-7.2);
                        \coordinate (c4) at (axis cs:1.0,-7.2);
                        \end{axis}
%
                        \begin{axis}[
                                name = ax_zoomed_start,
                                at={($(ax_main.north east)-(0.9\columnwidth,1.35/1.618*\columnwidth)$)},
                                width = 0.6*1\columnwidth,
                                height = 0.6*1/1.618*\columnwidth,
                                xmin=0, xmax=0.2,
                                ymin=-7.2, ymax=7.5,
                                xtick={0,0.02,...,0.2},
				xlabel={Time (ms)},
				xticklabels={$0$,$20$,$40$,$60$,$80$,$100$,$120$,$140$,$160$,$180$,$200$},
                                ytick={-6,-4,...,6},
				tick label style={/pgf/number format/fixed},
				legend columns=2,
				legend style={/tikz/every even column/.append style={column sep=0.5cm}},
                                ymajorgrids=true,
                                xmajorgrids=true,
                                %grid style=dashed,
                                every axis plot/.append style={thick},
                        ]
				\addplot[blue, smooth]         table[col sep=comma,header=false,x index=0,y index=5]{data/dpos/data_dpo_sim.csv};
				\addlegendentry{$v_o^{\text{DP}}$}
				\addplot[red,  smooth, dashed, dash pattern=on 4pt off 4pt, line cap=round] table[col sep=comma,header=false,x index=0,y index=6]{data/dpos/data_dpo_sim.csv};
				\addlegendentry{$v_o^{\text{T}}$}
                        \end{axis}
                        % draw dashed lines from rectangle in first axis to corners of second
                        \draw [gray,dashed] (c1) -- (ax_zoomed_start.north west);
                        \draw [gray,dashed] (c2) -- (ax_zoomed_start.north east);
%
                        \begin{axis}[
                                name = ax_zoomed_final,
                                at={($(ax_main.north east)-(0.4\columnwidth,1.9/1.618*\columnwidth)$)},
                                width = 0.6*1\columnwidth,
                                height = 0.6*1/1.618*\columnwidth,
                                xmin=0.9, xmax=1,
                                ymin=-4, ymax=4,
                                xtick={0.90,0.91,...,1},
				xlabel={Time (ms)},
				xticklabels={$900$,$910$,$920$,$930$,$940$,$950$,$960$,$970$,$980$,$990$,$1000$},
                                ytick={-4,-3,...,4},
				tick label style={/pgf/number format/fixed},
                                ymajorgrids=true,
                                xmajorgrids=true,
                                %grid style=dashed,
                                every axis plot/.append style={thick},
                        ]
				\addplot[blue, smooth]         table[col sep=comma,header=false,x index=0,y index=5]{data/dpos/data_dpo_sim.csv};
				\addplot[red,  smooth, dashed, dash pattern=on 4pt off 4pt, line cap=round] table[col sep=comma,header=false,x index=0,y index=6]{data/dpos/data_dpo_sim.csv};
                        \end{axis}
                        % draw dashed lines from rectangle in first axis to corners of second
                        \draw [gray,dashed] (c3) -- (ax_zoomed_final.north west);
                        \draw [gray,dashed] (c4) -- (ax_zoomed_final.north east);
                \end{tikzpicture}
        \caption
[Output voltage signals as reconstructed from the DP simulations and obtained from the time domain model.]
{Output voltage signals as reconstructed from the DP simulations \eqref{eq:dpo_sim_dpmodel} and obtained from the time domain model \eqref{eq:dpo_sim_timemodel}. In blue the voltage reconstructed by adding the time signals reconstructed from the Dynamic Phasors $V_o^{(1,2)}$ depicted in figure \ref{fig:dpo_sim_dpcurves}. In dashed red the output voltage signal obtained from directly integrating the time-domain model. }
        \label{fig:dpo_sim_timecurves}
        \end{center}
\end{figure}
% >>>


\chapter{Elementary Control Theory in Dynamic Phasor Space}\label{chapter:control_theory}
% ---------------------------------------------------------

	From a circuits and modelling perspective, DPFs present a great tool because, in a very short description, they transform derivatives in the time domain into algebraic structures with a broad spectrum of properties. From the perspective of Differential Equations, this means DPFs present a more convenient way to solve ODEs in the time domain, especially those excited by generalized sinusoids. In this chapter we explore the DPFs from the perspective of a linear systems and control design.

	Let us take a closer look at example \ref{example:3p_eps_modelling}. The target system of figure \ref{fig:ibr_modelling_example} is controlled by two subsystems that are eminently phasorial: first, the PLL of figure \ref{fig:3p_pll_curr_control} and the current control system of figure \ref{fig:3p_curr_control}. 

	This current controller is a widely used controller in literature, with the specific intent of adjusting the bus current $I$ to a setpoint $I_d^* + jI_q^*$. The first discomfort one finds with this controller is that the control is is done by decoupling the $d$ and $q$ frames and generating two PI controllers, one for each component, such that $V_d$ and $V_q$ are adjusted to vanish the current error; naturally one asks whether it is simply not possible to regulate the complex Dynamic Phasor $I(t)$ to a reference $I^*$.

	Further, as discussed in the example, the current control aims to regulate terminal voltage, but the inverter is unable to control the terminal voltage directly — rather, it can adjust the bridge output voltage $E$. It is assumed that $V$ and $E$ are related by 

\begin{equation} E = V + \left(r + j\omega L\right)I ,\label{eq:pi_adjust_e}\end{equation}

	\noindent resulting in the ``crossed current signals'' multiplied by $\omega L$. This equation shows that it is assumed that the $V_d,V_q$ quantities obtained by this process obey the ``classic phasor'' relationship that if the Dynamic Phasor of a signal $x(t)$ is $X(t)$, its derivative is represented by $j\omega X(t)$, originating equations like \eqref{eq:pi_adjust_e}. This is obviously not true, as proven by the Dynamic Phasor Theory shown in this thesis. 
	Hence, in this chapter we show that the capacity of DPFs to express ODEs is not only a powerful tool to model electrical circuits but also to express control systems, particularly ones operating on nonstationary sinusoidal regimens like most controllers in Power Systems. We show that an integral transform called the $\mu$-Transform can be defined, allowing the notions of $\mu$ Transfer Functions ($\mu$TFs) like those based on Laplace Transforms for rational systems; unlike the Laplace Transform, however, using Dynamic Phasor Operators leads to much more intuitive and simple notations for power control systems, allowing to obtain the involved Dynamic Phasors in time, their amplitudes and phases, unlike current techniques. For instance, it is shown that the current controller of figure \ref{fig:3p_curr_control} can be instead represented by an equivalent PI controller in the DPFT domain, which not only is much more intuitive and useful but guaranteedly reconstructs the current and voltage signals in time.

	More importantly, it is clear that that no clear stability analysis is possible from the controller as it is. Analyzing the effects of the PI controller gains $k_P^d,k_I^d,k_P^q,k_I^q$ is not possible preemptively, and the simplest way to do it is through simulations. However, modelling that controler using $\mu$TFs allows for obtaining clear stability results and dynamical characteristics of the control system based on the poles and zeros of the DPFT.

	The new Transfer Functions also allow drawing important results about the control systems they represent, unwaivering to the input signal used; for instance, like a Laplace Transfer Function is represented in time by a impulse response, the DPTFs allow characterizing a system using its impulse response in time, or other time response to reference signals in complex domain. Further, like proper rational and Hurwitz-stable Transform functions define input-output stable linear systems (also called BIBO stability), this result is also proven for $\mu$TFs.

%-------------------------------------------------
\section{Decomposition of complex signals} %<<<1

	In order to accomplish a control theory, we dive into more fundamental characteristics of DPFs. Studying the essential structure of any linear operators inevitably starts with analyzing the core structures of these operators, like the kernel and the eigenspace, and how the vector spaces are decomposed onto such structures. In order to do this for the DPFs $\ndpo{k}$, we dive a little bit further into abstract algebra.

%-------------------------------------------------
\subsection{The Fundamental Theorem of Homomorphisms} \label{subsec:first_homo}%<<<2

	Following the definition \ref{def:algebra_group} of a group, we suppose two groups $V$ and $W$, equipped with the operations $\left(+\right)_V$ and $(+)_W$ respectively, as well as the identity or neutral elements $0_V$ and $0_W$. Suppose there is a surjective mapping $\phi\in\left[V\to W\right]$ that preserves the algebraic structure, that is,

\begin{equation}\left\{\begin{array}{l} \phi\left(v_1 \left(+\right)_V v_2\right) = \phi\left(v_1\right)\left(+\right)_W\phi\left(v_2\right) \forall v_1,v_1\in V \\[5mm] \phi\left(0_V\right) = 0_W\end{array}\right. . \label{eq:homomorphism_def}\end{equation}

	Then $\phi$ is called a \textbf{homomorphism} (``same form'' or ``shape'') because it presevers the algebraic structure of the sets. We define the kernel of this mapping as $\Ker\left(\phi\right)$ as the counter-image of the zero element:

\begin{equation} \Ker\left(\phi\right) = \left\{k\in V: \phi\left(k\right) = 0_W\right\} .\end{equation}

	It is simple to see that this kernel with the $(+)_V$ operation is a group itself; thus it is a \textbf{subgroup} of $V$. It is also simple to see, from the definition \eqref{eq:homomorphism_def} of homomorphisms, that the image of $\phi$ through $G$, denoted $\phi(G)$, is a subgroup of $W$.

	For cleanliness, we henceforth denote $(+)_V$ and $(+)_W$ by just $+$, still having in mind they are the specific operations of their particular groups. Naturally, for any $v\in V$ and any $k\in\Ker\left(\phi\right)$, $\phi\left(v+ k\right) = \phi(v)$. This is shown in figure \ref{fig:group_homo}.

% GROUP HOMOMORPHISM SCHEMATIC <<<
\begin{figure}[h]
\centering
\begin{tikzpicture}[>={Stealth[inset=0mm,length=1.5mm,angle'=50]}]
\clip(-2,-4) rectangle(8,5); % This is needed because for some reason the figure gets too tall with a blank space?
\draw[color=stewartgreen, fill=stewartgreen, fill opacity = 0.2] (0,0) ellipse[x radius=2, y radius=4] +(-1.5,4); \node[above,black] at (0,4.25) (vlabel) {$V$};
\draw[color=stewartblue, fill=stewartblue, fill opacity = 0.2] (6,0) ellipse[x radius=2, y radius=4] +(-1.5,4); \node[above,black] at (6,4.25) (wlabel) {$W$};

\draw[->] (vlabel) -- (wlabel) node[midway,above] {$\phi$};

\draw[thick,draw=none, fill=white] (-0.25,-2) ellipse[x radius=1, y radius=1.5] +(-1.5,4);
\draw[thick,draw=none, fill=stewartyellow, fill opacity = 0.3] (-0.25,-2) ellipse[x radius=1, y radius=1.5] +(-1.5,3);
\draw[thick,dashed, line cap = round, color=stewartyellow] (-0.25,-2) ellipse[x radius=1, y radius=1.5] +(-1.5,4); \node[above,black] at (-0.25,-0.5) (klabel) {$\Ker\left(\phi\right)$};

\draw[thick,draw=none, fill=white] (0.25,2) ellipse[x radius=1, y radius=1.5] +(-1.5,4);
\draw[thick,draw=none, fill=stewartpink, fill opacity = 0.3] (0.25,2) ellipse[x radius=1, y radius=1.5] +(-1.5,3);
\draw[thick,dashed, line cap = round, color=stewartpink] (0.25,2) ellipse[x radius=1, y radius=1.5] +(-1.5,4); \node[above,black] at (-1.25,1.5) (vlabel) {$[v]$};

\draw[fill] (0.3,2.75) circle (0.05) node (velement) {};
\node at ([shift=({-0.25,0})]velement) {$v$};

\draw[fill] (0.75,1.5) circle (0.05) node (vpkelement) {};
\node at ([shift=({-0.75,0})]vpkelement) {$v + k$};

\draw[fill] (-0.2,-1.3) circle (0.05) node (kelement) {};
\node at ([shift=({-0.25,0})]kelement) {$k$};

\draw[fill] (0,-2.5) circle (0.05) node (0velement) {};
\node at ([shift=({-0.4,0})]0velement) {$0_V$};

\draw[fill] (5.5,2) circle (0.05) node (welement) {};
\node at ([shift=({ 1,0})]welement) {$w = \phi\left(v\right)$};

\draw[fill] (6,-1) circle (0.05) node (0welement) {};
\node at ([shift=({ 0.4,0})]0welement) {$0_W$};

\draw[->] (velement) -- ([shift=({-0.3,0})]welement.north) ;
\draw[->] (vpkelement) -- ([shift=({-0.3,0})]welement.west) ;
\draw[->] (kelement) -- ([shift=({-0.1,0.2})]0welement) ;
\draw[->] (0velement) -- ([shift=({-0.3,-0.1})]0welement.west) ;

\draw[dashed,gray, line cap = round] (-0.25,-0.5) -- (0welement);
\draw[dashed,gray, line cap = round] (-0.25,-3.5) -- (0welement);

\draw[dashed,gray, line cap = round] (0.25,0.5) -- (welement);
\draw[dashed,gray, line cap = round] (0.25,3.5) -- (welement);

\end{tikzpicture}
\caption
[{Schematic of a homomorphism $\phi$ showing an element $[v]$ of $V/\Ker(\phi)$.}]
{{Schematic of a homomorphism $\phi$ showing an element $[v]$ of $V/\Ker(\phi)$ correspondent to a particular element $v$. Note that by definition any $v + k$ maps into $\phi(v)$, so that $[v]$ is formed by adding $v$ and all elements of $\Ker(\phi)$; this is denoted as $[v] = v\oplus\Ker(\phi)$.}}
\label{fig:group_homo}
\end{figure} %>>>

	Naturally, it would be simpler if $\phi$ were bijective, called an \textbf{isomorphism}; this would only happen if the kernel of $\phi$ were composed of only the neutral element, that is, $\Ker\left(\phi\right) = \left\{0_V\right\}$. When this is not the case, reconstructing any element of $V$ by its image is inherently problematic, because if we take an element $w\in W$, there are multiple elements that fulfill $v = \phi^{-1}\left(w\right)$, thus this inverse is not a function. Alternatively, we can say $\phi$ is not injective.

	However, we can construct some restriction of $\phi$, such that this restriction is injective, and bijective from the definition of a homomorphism. Pick a particular $w\in W$ and let $v$ an element of the pre-image of $w$. Then define

\begin{equation} \left[v\right] = v \oplus \Ker\left(\phi\right) \end{equation}

	\noindent where the direct sum is defined as $v \oplus K = \left\{v + k: k\in K\right\}$. Naturally, if $v'\in V$ and $\phi\left(v'\right) = w$, then $v'\in\left[v\right]$. Thus, $\left[v\right] = \phi^{-1}\left(w\right)$. Therefore, each $w$ in the image of $\phi$ through $V$, denoted $\phi\left(V\right)$, defines a subset in $V$. It can be easily proven that this subset is also a subgroup of $V$. Let us define the set of all subgroups constructed in such a way, that is, the set of all left cosets of $\Ker\left(\phi\right)$ in $V$, defined as the \textbf{quotient group}:

\begin{equation} V/\Ker\left(\phi\right) = \left\{\raisebox{4mm}{}  \left[v\right] = v \oplus \Ker\left(\phi\right): v\in V\right\} .\end{equation}

	The intuition here is that any group belonging to this quotient is such that it is ``compressed'' into a single element and no other element of the quotient group maps into $w$, that is, for any $w\in W$ there exists a single $Z \in V/\Ker\left(\phi\right)$ that is singularly mapped into $w$, or formally,

\begin{equation} \left(\forall w\in W\right)\left(\raisebox{4mm}{} \exists! Z \in V/\Ker\left(\phi\right): \phi\left(Z\right) = \left\{w\right\}\right) .\end{equation}

	The problem now lies in the fact that we went from groups to sets, causing a loss of structure since sets are a weaker concept (they have no standard summation nor specific properties). We would like to define a group structure for this quotient, so the algebraic entity of a quotient group is still itself a group; this allows us, for instance, to define the sum of two groups $V_1$ and $V_2$ in the quotient space as the set $\left(v_1 + v_2\right)\oplus \Ker\left(\phi\right)$.

	Thus pick two $a,b\in V$ and we want to define an addition operation $(+)_q$ (the subscript ``q'' for ``quotient'') in $V/\Ker\left(\phi\right)$ that makes it a group, that is, $\left[a\right](+)_q\left[b\right]$ fulfills the definition of a group addition. We naturaly require that the map $V\mapsto V/\Ker\left(\phi\right)$ be a homomorphism to keep the algebraic structures intact. But this means that

\begin{equation} \left[a+ b\right] = \left(a + b\right) \oplus \Ker\left(\phi\right) = \left[a\right] (+)_q\left[b\right] = \left(a\oplus \Ker\left(\phi\right)\right) (+)_q \left(b \oplus\Ker\left(\phi\right)\right) \end{equation}

	\noindent and this definition only makes sense if for any $k\in\Ker\left(\phi\right)$, $k + a = a + k$ for any $a\in V$ which is immediately true from the definition of the kernel; thus, for any $k\in\Ker\left(\phi\right)$, the \textbf{conjugation operation} of an element $a\in V$ by $k$, denoted $a + k + \left(-a\right)$ is exactly $k$, that is,

\begin{equation} k = a + k + \left(-a\right)\ \forall a\in V. \end{equation}

	This means that the kernel is not only a subgroup of $V$, but it is special in that it invariant under the conjugation operation which causes it to be invertible through the group quotient operation, and the quotients built in such a way keep the group properties intact. Thus the kernel is known as a \textbf{normal subgroup}, and this whole process is ennunciated in theorem \ref{theo:homo_fundamental}.

	The idea of a normal subgroup is important because the kernal having such property thus each $w\in W$ defines such a unique set $V'$ that belongs to $V/\Ker\left(\phi\right)$, that is, there is a bijection between $\phi\left(V\right)$ and $V/\Ker\left(\phi\right)$. This is shown in figure \ref{fig:first_homo}.

\begin{theorem}[Fundamental Theorem of Homomorphisms \pcite{garciaElementosAlgebra2022}]\label{theo:homo_fundamental}
	Let $V,W$ two groups and a homomorphism $\phi$ between them. Then $\Ker\left(\phi\right)$ is a normal subgroup of $G$, $\phi\left(G\right)$ is a subgroup of $H$ and $G/\Ker\left(\phi\right)$ is isomorphic to $\phi\left(G\right)$.
\end{theorem}\vspace{3mm}\hrule\vspace{3mm}

% FIRST HOMOMORPHISM THEOREM SCHEMATIC <<<
\begin{figure}[h]
\centering
\begin{tikzpicture}[>={Stealth[inset=0mm,length=1.5mm,angle'=50]}]
\clip(-2,-4) rectangle(8,5); % This is needed because for some reason the figure gets too tall with a blank space?

\draw[color=stewartgreen, fill=stewartgreen, fill opacity = 0.2] (0,0) ellipse[x radius=2, y radius=4] +(-1.5,4); \node[above,black] at (0,4.25) (vlabel) {$V$};
\draw[color=stewartblue, fill=stewartblue, fill opacity = 0.2] (6,0) ellipse[x radius=2, y radius=4] +(-1.5,4); \node[above,black] at (6,4.25) (wlabel) {$W$};

\draw[->] (vlabel) -- (wlabel) node[midway,above] {$\phi$};

\draw[thick, draw=none,  fill=white] (-0.5,-2.5) ellipse[x radius=0.5, y radius=1] +(-1.5,4);
\draw[thick, draw=none, fill=stewartyellow, fill opacity = 0.3] (-0.5,-2.5) ellipse[x radius=0.5, y radius=1] +(-1.5,4);
\draw[thick, dashed, line cap = round, color=stewartyellow] (-0.5,-2.5) ellipse[x radius=0.5, y radius=1] +(-1.5,4); \node[above,black] at (-0.5,-1.5) (klabel) {$\Ker\left(\phi\right)$};

\draw[thick, draw=none, fill=white] (0.5,0) ellipse[x radius=0.5, y radius=1] +(-1.5,4);
\draw[thick, draw=none, fill=stewartpink, fill opacity=0.3] (0.5,0) ellipse[x radius=0.5, y radius=1] +(-1.5,4);
\draw[thick, dashed, line cap = round, color=stewartpink] (0.5,0) ellipse[x radius=0.5, y radius=1] +(-1.5,4); \node[above,black] at (-0.5,0) (V1label) {$V_{(w_1)}$};

\draw[thick, draw=none, fill=white] (-0.3,2.5) ellipse[x radius=0.5, y radius=1] +(-1.5,4);
\draw[thick, draw=none, fill=stewartpurple, fill opacity=0.3] (-0.3,2.5) ellipse[x radius=0.5, y radius=1] +(-1.5,4);
\draw[thick, dashed, line cap = round, color=stewartpurple] (-0.3,2.5) ellipse[x radius=0.5, y radius=1] +(-1.5,4); \node[above,black] at (-1.2,1.5) (V1label) {$V_{(w_2)}$};

\draw[fill] (0.7,0) circle (0.05) node (v1element) {};
\node at ([shift=({-0.35,0})]v1element) {$\alpha_1$};

\draw[fill] (-0.1,2.5) circle (0.05) node (v2element) {};
\node at ([shift=({-0.35,0})]v2element) {$\alpha_2$};

\draw[fill] (-0.2,-2.5) circle (0.05) node (kelement) {};
\node at ([shift=({-0.25,0})]kelement) {$k$};

\draw[fill] (6,-0.1) circle (0.05) node (w1element) {};
\node at ([shift=({1.1,0})]w1element) {$w_1 = \phi\left(v_1\right)$};

\draw[fill] (5.2,2.2) circle (0.05) node (w2element) {};
\node at ([shift=({1.1,0})]w2element) {$w_2 = \phi\left(v_2\right)$};

\draw[fill] (6,-2) circle (0.05) node (0welement) {};
\node at ([shift=({ 0.4,0})]0welement) {$0_W$};

\draw[->] (v1element) -- (w1element.west) ;
\draw[->] (v2element) -- (w2element.west) ;
\draw[->] (kelement) -- ([shift=({-0.1,0.2})]0welement) ;

\draw[dashed,gray, line cap = round] (-0.25,-1.5) -- (0welement);
\draw[dashed,gray, line cap = round] (-0.25,-3.5) -- (0welement);

\draw[dashed,gray, line cap = round] (0.5, 1) -- (w1element);
\draw[dashed,gray, line cap = round] (0.5,-1) -- (w1element);

\draw[dashed,gray, line cap = round] (-0.3,1.5) -- (w2element);
\draw[dashed,gray, line cap = round] (-0.3,3.5) -- (w2element);

\end{tikzpicture}
\caption
[Schematic of the Fundamental Theorem of Homomorphisms.]
{{Schematic of the Fundamental Theorem of Homomorphisms showing two derived subgroups $V_{(w_1)}$ and $V_{(w_2)}$ generated by two images $w_1$ and $w_2$. These subgroups are represented by two elements $\alpha_1$ and $\alpha_2$ respectively so that $V_{(w_1)} = \alpha_1\oplus\Ker(\phi)$ and identically wth $V_{(w_2)} = \alpha_2\oplus\Ker(\phi)$.}}
\label{fig:first_homo}
\end{figure} %>>>

	The Fundamental Theorem of Homomorphisms has many consequences on the fundamental theory of abstract algebra, and through this, on a great portion of mathematics. For the purposes of this analysis, the main property we are looking for is that this theorem allows us to ``remove the kernel'' of our transformation from the analysis because we can, in a simple manner, construct subgroups of $V$ such that $\phi$ is bijective on $V$ and $V$ is reconstructed from $V'$ by ``adding the kernel back''.

	Indeed, since to each $w\in\phi\left(V\right)$ corresponds a set $V_{(w)} \in V/\Ker\left(\phi\right)$, which is the set of all elements that map into $w$, we can choose one $\alpha_{(w)}\in V_{(w)}$, and we let $V'$ be the set of all such $\alpha_{(w)}$. Further, we can reconstruct $V$ as $V = V' \oplus \Ker\left(\phi\right)$, that is, any element of $V$ can be obtained as the sum of an element of $V'$ and an element of the kernel, and this is guaranteed by the fact that the kernel is a normal subgroup. In simpler terms, using $V'$ we get the best of both worlds: $\phi$ is bijective on $V'$ and the other elements of $V$ are easily accessible from $V'$. Figure \ref{fig:first_homo} shows two such sets pertaining to two chosen $\alpha_1$ and $\alpha_2$, pertaining to $w_1$ and $w_2$ respectively, generating two sets $V_1$ and $V_2$.

%-------------------------------------------------
\subsection{Consequences on DPFs} %<<<2

	It is simple to see that that each $\ndpo{k}$ is a homomorphism from the set $\left[\mathbb{R}\to\mathbb{C}\right]$ to itself (thus a \textit{self-homomorphism}). By theorem \ref{theo:bijection}, any $\ndpo{k}$ is bijective in the entire set, requiring a set of initial conditions which will construct the kernel of the transform, as we will see later. Therefore, given this set of initial conditions, any $\ndpo{k}$ is a bijective homomorphism (thus an isomorphism) of $\left[\mathbb{R}\to\mathbb{C}\right]$ unto itself, called an \textit{automorphism}. This is to say that the image of $\ndpo{k}$ is the entire space $\left[\mathbb{R}\to\mathbb{C}\right]$.

	 By the Fundamental Theorem on Homomorphisms, the image of $\ndpo{k}$ through $\left[\mathbb{R}\to\mathbb{C}\right]$ is isomorphic to the quotient group $\left[\mathbb{R}\to\mathbb{C}\right]/\Ker\left(\ndpo{k}\right)$; this means that any vector $Y$ in the image of $\ndpo{k}$ can be built by choosing some $X_\eta\in\Ker\left(\ndpo{k}\right)$; then for every $Y(t)$ there corresponds a single $X_\varepsilon(t)$ such that

\begin{equation} Y(t) = \ndpo{n}\left[X\right] \Leftrightarrow X(t) = X_\varepsilon(t) + X_\eta(t) ,\end{equation}

	\noindent and $X_\varepsilon$ is not in the kernel except in the trivial case. By fixing $X_\eta$ the relationship $X(t)\leftrightarrow Y(t)$ is bijective. Effectively, this establishes an equivalence relationship between $X$ and $X_\varepsilon$; reestated, every $X_\eta\in\Ker\left(\ndpo{k}\right)$ defines a equivalence relationship that we will call \textbf{null-equivalence} such that two elements $X_1$ and $X_2$ are null-equivalent with respect to $X_\eta$ if they belong to the subspace $X_\eta \bigoplus \left[\mathbb{R}\to\mathbb{C}\right]$.

%-------------------------------------------------
\subsection{Nullspace of the DPO} %<<<2

	We now investigate what exactly is the kernel $\Ker\left(\ndpo{k}\right)$; the objective is to find a basis of this space. We first consider the negative index functionals, for instance,

\begin{equation} \ndpo{(-1)}\left[X\right] = 0 \Leftrightarrow X = \dot{0} + j\omega 0 = 0, \end{equation}

	\noindent and quickly one notes that the kernel of any $\ndpo{(-n)}$ is trivial as it only contains the null signals. For the zero-th functional, it is obvious that the kernel of $\ndpo{0} = \mathbf{I}$ is the null signal.

	Then we consider positive order functionals. For the first-order functional, we want to find $V$ such that $\dpo\left[V\right] = 0$:

\begin{equation}  \dpo\left[V\right] = 0 \Leftrightarrow \dot{V} + j\omega V = 0, \label{eq:1st_order_null_ode}\end{equation}

	\noindent and we use theorem \ref{theo:first_order_general} to solve this equation.

\begin{theorem}[General solution to first-order complex ODE]\label{theo:first_order_general} %<<<
	Let $p(t),q(t)\in\left[\mathbb{R}\to\mathbb{C}\right]$ and consider the differential equation

\begin{equation} y'(t) + p(t)y(t) = q(t).\end{equation}

	Then the general solution to this ODE is

\begin{equation} y(t) = e^{-P(t)}\int e^{P(t)}q(t)dt,\ P(t) = \int p(t)dt. \end{equation}
\end{theorem}
\textbf{Proof.} Consider some function $I(t)$ and multiply the original ODE by this function: 

\begin{equation} I(t)y'(t) + I(t)p(t)y(t) = I(t)q(t). \label{eq:first_order_eq1}\end{equation}

	Now note that if $I(t)$ satisfies $I'(t) = I(t)p(t)$, then the left side of \eqref{eq:first_order_eq1} becomes $I(t)y'(t) + I'(t)y(t)$, which by the multiplication rule is $\left(I(t)y(t)\right)'$. Finding such a function is simple: adopt $\Ln$ as some branch of the complex logarithm and

\begin{equation} I'(t) = I(t)p(t) \Leftrightarrow \dfrac{I'(t)}{I(t)} = p(t) \Leftrightarrow \dfrac{d}{dt}\Ln\left(I(t)\right) = p(t) \Leftrightarrow I(t) = e^{\int p(t)dt} = e^{P(t)}. \label{eq:first_order_eq2}\end{equation}

	Therefore \eqref{eq:first_order_eq1} becomes

\begin{equation} \dfrac{d}{dt}\left[e^{P(t)}y(t)\right] = e^{P(t)}q(t) \Leftrightarrow e^{P(t)}y(t) = \int e^{P(t)}q(t)dt \Leftrightarrow y(t) = e^{-P(t)}\int e^{P(t)}q(t)dt. \label{eq:first_order_eq3}\end{equation}

\hfill$\blacksquare$
\vspace{5mm}
\hrule
\vspace{5mm} %>>>

	Thus the solution to \eqref{eq:1st_order_null_ode} is $kR_0(t)$ where $k$ is a complex number and

\begin{equation} R_0\left(t\right) = e^{-j\psi(t)},\ \psi(t) = \int_{0}^{t} \omega(a)da\end{equation}

	\noindent so the kernel of $\dpo$ is the span of $R_0$, that is, $R_0$ alone generates the kernel $\Ker\left(\dpo^1\right)$. For the kernel of the second-order operator $\Ker\left(\dpo^2\right)$,

\begin{equation} \dpo^2\left[V\right] = 0 \left\{\begin{array}{l} V = ke^{-j\psi(t)},\ k\in\mathbb{C} \text{ or } \\[3mm] \dpo\left(V\right) = ke^{-j\psi(t)},\ k\in\mathbb{C} \end{array}\right. \end{equation}

	Therefore let $R_1$ the solution to $\dpo\left(R_1\right) = R_0$; then $R_0$ and $R_1$ generate $\Ker\left(\dpo^2\right)$. But

\begin{equation} \dpo\left[R_1\right] = R_1 \Leftrightarrow \dot{R}_1 + j\omega R_2 = R_0 \end{equation}

	\noindent and by theorem \ref{theo:first_order_general} the general solution to this ODE is

\begin{equation} R_1 = C_3R_0(t) \left[C_1 + \int \left(\dfrac{C_2}{R_0(s)}\right) R_0(s)ds\right] \end{equation}

	\noindent and without loss of generality we can choose $C_3 = C_2 = 1$ because we want basis vectors and $C_1 = 0$ because otherwise $R_1$ will have a term of $R_0$ which is already contemplated in the basis by $R_0$ itself. Therefore, $R_1 = tR_0(t)$ is an element of the base of the kernel, and $\left\{R_0(t),R_1(t)\right\}$ generates $\Ker\left(\ndpo{2}\right)$. These results suggest that

\begin{equation} R_i = \dfrac{t^{i}}{i!} R_0(t),\ 0\leq i \leq k - 1 \label{eq:ri_kerdef}\end{equation}

	\noindent is a base for $\Ker\left(\ndpo{k}\right)$. This is forthproven by a double induction. First, it needs to be established that the $R_i$ satisfying the recursion

\begin{equation}\left\{\begin{array}{l} R_0 = e^{j\psi(t)} \\[3mm] \dpo\left[R_i\right] = R_{(i-1)}\end{array}\right. \label{sys:ri_def_ind}\end{equation}

	\noindent for $1\leq i \leq k - 1$ form a basis of $\Ker\left(\ndpo{k}\right)$. Suppose this true for $k-1$. Then

\begin{equation} \ndpo{k}\left[X\right] = 0 \Leftrightarrow  \dpo\left(\ndpo{(k-1)}\left(X\right)\right) = 0 \end{equation}

	which is true if and only if either $X\in\Ker\left(\ndpo{\left(k-1\right)}\right)$ or $X$ satisfies $\ndpo{(k-1)}\left(X\right) = R_0$, that is, $\Ker\left(\ndpo{k}\right) = \Ker\left(\ndpo{(k-1)}\right)\cup \left\{R_{k}\right\}$ where $R_{k}$ satisfies $\dpo\left(R_{k}\right) = R_{(k-1)}$, and the proof is complete. The second induction proves that the $R_i$ as defined in \eqref{eq:ri_kerdef} satisfy \eqref{sys:ri_def_ind}. Suppose this true for $k-1$; then 

\begin{equation} \dpo\left(R_{k}\right) = R_{(k-1)} \Leftrightarrow \dot{R}_{k} + j\omega R_{k} = R_{(k-1)} \end{equation}

	\noindent and by theorem \ref{theo:first_order_general} the general solution to this ODE is

\begin{equation} R_{k} = C_3R_0(t) \left[C_1 + \int \left(\dfrac{C_2}{R_0(s)}\right) R_{(k-1)}(s)ds\right] .\end{equation}

	As exposed before, without loss of generality we can assume $C_3 = 1$, $C_2 = \left[\left(k-1\right)!\right]^{-1}$ because we are looking for basis vectors and $C_1 = 0$ otherwise $R_k$ will have terms of $R_0(t)$ that are already considered by $R_{0}$ itself. Therefore,

\begin{equation} R_{k} = R_0(t)\left(C_4 + \dfrac{t^{k}}{k!}\right) \substack{(C_4 = 0)\\ =} \dfrac{t^k}{k!} R_0(t) . \label{eq:rk_formula}\end{equation}

	Hence, the kernel of $\ndpo{n}$ is generated by the basis $\left\{R_k\right\}_{k\in\mathbb{N}}$ where $R_k(t)$ is given by \eqref{eq:rk_formula}. Finally, having obtained the basis for the kernel of $\ndpo{k}$ and simplified the analysis of null equivalence, we condlude that any element in the kernel — like the null decomposition $X_\eta$ of a signal $X(t)$ — entails to decomposing it into the basis of the space, which is simple seen as the kernel has a Schauder Basis (infinite but countable basis), meaning that any $X_\eta\in\Ker\left(\ndpo{k}\right)$ can be written as:

\begin{equation} X_\eta = \sum_{k\in\mathbb{N}} \eta_k^{\left[X\right]} R_k(t) \end{equation}

	\noindent for a sequence of complex scalars $\eta_k^{\left[X\right]}$; 

%-------------------------------------------------
\subsection{The nature of the null component} %<<<2

	Having characterized the kernel of $\ndpo{k}$, we now ask ourselves what truly is the nature of the kernel and the null component $X_\eta$, and for this we retake the discussion on the Fundamental Theorem of Homomorphisms. Consider two real NS signals $x(t)$ and $y(t)$ such that

\begin{equation} x^{(n)}(t) = y^{(n)}(t) \text{ for some natural } n.\end{equation}

	Sequentially integrating this equation one concludes $x$ and $y$ differ by some polynomial of order $n$:

\begin{equation} x = y + \sum_{k=0}^{n} \dfrac{\left[x_{(0)}^{(k)} - y_{(0)}^{(k)}\right]}{k!} t^k \label{eq:time_initial_conds}\end{equation}

	\noindent which is to say $x(t)$ and $y(t)$ differ by their initial conditions and, as they get successively integrated, these initial conditions become polynomials. Using the Dynamic Phasor Transform on \eqref{eq:time_initial_conds} yields

\begin{equation} X = Y + \sum_{k=0}^{n} \dfrac{\left[x_{(0)}^{(k)} - y_{(0)}^{(k)}\right]}{k!} t^k R_0(t) = Y +\sum_{k=0}^{n} \left(x_{(0)}^{(k)} - y_{(0)}^{(k)}\right)R_k(t) . \label{eq:dp_initial_conds}\end{equation}

	Immediately one notices that the difference between $X$ and $Y$ is an element of the kernel; reestated, $X$ and $Y$ are null-equivalent, that is, $\ndpo{n}\left[X\right] = \ndpo{n}\left[Y\right]$. A careful examing of the Fundamental Theorem of Homomorphisms of section \ref{subsec:first_homo} alludes to the fact that $X$ and $Y$ belong to the same kernel equivalence class, seen as they have the same image while being different signals. This means that $X$ and $Y$ belong to the same $V_{(w)}$ set of figure \ref{fig:first_homo}.

	Thus, we can pick and choose the elements of $\left[\mathbb{R}\to\mathbb{C}\right]$ to represent the entire space, that is, we can choose the equivalence class we want (the set $V'$) so that $\ndpo{k}$ becomes a bijective homomorphism thus an isomorphism. Naturally, we choose the class of Zero-Energy Signals, that is, signals with null initial conditions:

\begin{definition}[Smoothness index] Let $x(t)$ a complex signal. Then the smoothness index of $x(t)$ is the operator denoted $\mathbf{c}\left[x\right]$ that gives the maximum natural $k$ such that the k-th derivative $x^{(k)}$ exists. \end{definition}

\begin{definition}[Zero-energy start signal] A complex signal $x(t)$ is said to have Zero Energy Start or ZES if $x(0) = x'(0) = x''(0) = \cdots = x^{\left(\mathbf{c}\left[x\right]\right)}(0) = 0$. \end{definition}

	Just like the set $V$ can be obtained by adding the chosen $V'$ to the kernel of the mapping, the direct consequence of this definition is that any signal $x(t)$ is equivalent to a ZES signal through

\begin{equation} \tilde{x}(t) = x(t) - \sum_{k=0}^{\mathbf{c}\left[x\right]} x^{(k)}_{(0)}\ \dfrac{t^k}{k!} \end{equation}

	\noindent with $x^{(k)}_{(0)}$ the k-th derivative at $t=0$; applying the DPT to this equation yields

\begin{equation} \tilde{X}(t) = X(t) - \sum_{k=0}^{\mathbf{c}\left[x\right]} X^{(k)}_{(0)}\ \dfrac{t^k}{k!}R_0(t) = X(t) - \sum_{k=0}^{\mathbf{c}\left[x\right]} X^{(k)}_{(0)} R_k(t) \label{eq:zfs_reconst}\end{equation}

	\noindent where $X^{(k)}_{(0)}$ represents the k-th derivative at $t=0$, also has smoothness index $\mathbf{c}\left[x\right]$ and is a ZES signal, but $\tilde{X}(t)$ and $X(t)$ are biunivcally related as in, one can be reconstructed from the other. Adopting the null-decomposition as

\begin{equation} \eta_k^{\left[X\right]} = \left\{\begin{array}{l} x^{(k)}_{(0)}, 0\leq k \leq \mathbf{c}\left[x\right] \\[3mm] 0,\ k > \mathbf{c}\left[x\right] \end{array}\right.\end{equation}

	\noindent for any signal $x(t)$, then yields that $\tilde{X}(t)$ is null-equivalent to the null function. In other words, we choose the ZES signals $\tilde{X}$ as the equivalence class to represent the entire $\left[\mathbb{R}\to\mathbb{C}\right]$, and while $\ndpo{k}$ is bijective with respect to ZES signals (isomorphic in their space) any other signal $X(t)$ can be reconstructed from its ZES equivalent: all we have to do is use \eqref{eq:zfs_reconst}. Therefore, much like the Fundamental Theorem of Homomorphisms allows us to ``remove the kernel'' from analysis, this reflects on Dynamic Phasors as the benefit that we do not have to worry about initial conditions, ``removing'' them from our analysis. Again, we get the best of both worlds: $\ndpo{k}$ becomes bijective and the entire space of functions can be obtained from the class of ZES signals and the kernel of $\ndpo{k}$.

	The fact that the $\ndpo{k}$ are bijective in the space of ZES signals has big consequences for differential equations, and especially Laplace Transforms. Notably, this shows that the essence of the null component in the DPO space is that of taking account for initial conditions, like the same phenomenon happens in the Laplace Transform: so much so that the Transform of the derivative of a signal is

\begin{equation} \mathbf{L}\left[x^{(n)}\right] = s^n\mathbf{L}\left[X\right] - \sum_{k=0}^{n-1} s^{(n-k+1)}x_{(0)}^{(n-k)} .\label{eq:laplace_initial_conds}\end{equation}

	\noindent essentially accomodating the initial conditions. For a ZES signal, however, 

\begin{equation} \mathbf{L}\left[x^{(n)}\right] = s^n\mathbf{L}\left[X\right] \end{equation}

	\noindent which is the simpler, more used formula. We hereforth suppose, unless specifically stated, that all signals are ZES, so as to make analysis simpler. This can be done without loss of generality because, as shown, any signal can be represented by a ZES version and vice-versa. 

%-------------------------------------------------
\subsection{Eigenanalysis of the DPO} %<<<2

	We now ask if we could obtain a basis of functions that can generate the largest possible pool of ZES signals $X$. This would mean that we could reconstruct any signal from a pool (basis) of fixed signals and represent $X$ as coordinates in this basis. Functional analysis gives us a way to do this by means of the eigendecomposition of $\ndpo{k}$. Let $V$ be an eigenvector of $\dpo$ for some eigenvalue $\mu$; then

\begin{equation} \dpo\left[V\right] = \mu V \Leftrightarrow \dot{V} + j\omega V = \mu V \end{equation}

	\noindent and theorem \ref{theo:first_order_general} gives us the solution

\begin{equation} V = ke^{\mu t} R_0(t),\ k\in\mathbb{C} \label{eq:dpo_eigen}\end{equation}

	\noindent meaning $e^{\mu t}R_0(t)$ is an eigenvector of $\dpo$ with eignvalue $\mu$ for any complex $\mu$. For $\ndpo{2}$, this implies

\begin{equation} \dpo\left[\raisebox{4mm}{} \dpo\left[e^{\mu t}R_0\right]\right] = \dpo\left[\mu e^{\mu t}R_0\right] = \mu\dpo\left[e^{\mu t}R_0\right] = \mu^2 e^{\mu t}R_0, \end{equation}

	\noindent therefore $e^{\mu t}R_0$ is an eigenvector of $\dpo^2$ with eigenvalue $\mu^2$; by induction, $e^{\mu t}R_0$ is an eigenvalue of $\ndpo{k}$ with eigenvalue $\mu^k$. This means that the eigenspace of $\ndpo{k}$, denoted $\Eig\left(\ndpo{k}\right)$, is generated by $e^{\mu t}R_0$, with eigenvalues $\mu^k$.

	The decomposition on the eigenspace however is more difficult as compared to that on the kernel because the eigenspace is an uncountably infinite space, seen as any complex $\mu$ generates an eigenvector. If we are to write the coordinates of a particular $X_\varepsilon$ with respect to a basis of the eigenvectors \eqref{eq:dpo_eigen}, we need a way to extract the coordinates of that particular function with respect to the basis.

	Fortunately, as discussed in subsection \ref{subsec:inner_prod_norms}, there is a rather simple way to do this. If we can define an internal product $\left<\cdot\right>$ in the space $\left[\mathbb{R}\to\mathbb{C}\right]$, then as a direct consequence of the definition of an internal product as per equation \eqref{eq:coordinate_extraction}, the coordinate of $X$ with respect to the eigenvector $e^{\mu t}R_0(t)$ would be given by

\begin{equation} X\left(\mu\right) = \left<X, e^{\mu t} R_0(t)\right> .\end{equation}

	An issue arises, however: as discussed in subsection \ref{subsec:characteristics_l1}, there is no basis that can generate $\left[\mathbb{R}\to\mathbb{C}\right]$ unconditionally, that is, generate any vector in that space. Reestated, once a basis is admitted, we are in essence limiting our analysis to those signals which can be built using the basis and the inner product adopted. Again borrowing from Functional Analysis, we adopt the internal product

\begin{equation} \left< f(t),g(t)\right> = \int_{-\infty}^{\infty} f(t)\overline{g(t)} dt \label{eq:internal_prod} \end{equation}

	\noindent and we leave to the reader the proof that this operation indeed satisfies all properties of an inner product as outlined in definition \ref{def:complex_inner_prod}. Notably, this inner product induces a norm, as per definition \ref{def:norm_complex}:

\begin{equation} \left\lVert f(t)\right\rVert = \sqrt{\left< f(t),f(t)\right>} = \sqrt{\int_{-\infty}^{\infty} f(t)\overline{f(t)} dt} = \sqrt{\int_{-\infty}^{\infty} \left\lvert f(t)\right\rvert^2 dt}\end{equation}

	\noindent thus the inner product adopted induces notions of distances within the space being considered, therefore making possible the notions of sequences. It is generally said of this fact that \textit{the inner product adopted induces a topology for the space}. Furthermore, we can clearly see that the price we pay by adopting the inner product \eqref{eq:internal_prod} is that we confine ourselves to the functions that \textit{conform to this topology}, that is,  which norm is not infinite as induced by the inner product adopted. Specifically in this case, the set of such functions is the set $L^2$, called the set of \textit{square-integrable functions}:

\begin{equation} L^2\left(\mathbb{R}\right) = \left\{f\in\left[\mathbb{R}\to\mathbb{C}\right]: \int_{-\infty}^{\infty} \left\lvert f(t)\right\rvert^2 dt < \infty\right\}. \end{equation}

	It can be proven that the space $L^2$ is not only a Banach Space (a space with a notion of distance between vectors) but it is also a Hilbert Space, that is, the norm induced is complete in that every Cauchy sequence converges to a limit. Thus, in the space $L^2$ the notions of differentials and integrals are well-defined, like using the Frechèt Derivative of \eqref{eq:def_frechet}.

	Another perk of a Hilbert Space is that the inner product adopted can give a notion of the decomposition of a vector $X$ with respect to a basis, like that of theorem \ref{theo:orthobasis_decomp}, by calculating by the internal product of $X$ and the constituents of that basis. In this case we are using the basis of eigenvectors, yielding an integral functional transform $\mathbf{T} \left[X\right]$:

\begin{equation} \mathbf{T} \left[X\right]\left(\mu\right) = \left< X(t), e^{\mu t} R_0(t)\right> = \int_{-\infty}^{\infty} X(t) e^{\overline{\mu} t} \overline{R_0(t)}dt ,\label{eq:tmu_def}\end{equation}

	It must be noted that this equation denotes some form of decomposition but it does not give a \textit{complete decomposition} in the same sense as the one of theorem \ref{theo:orthobasis_decomp} because the basis adopted is uncountably infinite and the elements of the basis are not orthonormal. Indeed, if one atempts to find the inner product of two eigenvectors $e^{\mu t}R_0(t)$ they will find that the resulting integral simply does not converge; even worse, the norm of an eigenvector is infinite for any $\mu$.

	Naturally one asks whether an orthonormal basis of $L^2$ can be found because if so the decomposition is given and certain. Unfortunately, the answer is simply no: in the case of uncountably infinite sets such as $L^2$, there is no proof such a basis exists. Even for the ``simpler'' case of transfinite (countably infinite) dimensional spaces, it can be shown \pcite{halmosNaiveSetTheory1974} that finding such basis is possible but the process is quite contrived and requires supposition of certain logical axioms, and this discussion gravely overextends the intent of this thesis as it depends on a much larger (and honestly out of my mathematical capabilities) discussion on logic, the ZFC axiomatic theory and the quite divisive Axiom of Choice.

	That being the case, we stop the discussion on the characteristics of $L^2$ because from this point forward there lie dragons. It suffices for the purposes of this text that the decomposition onto the eigenbasis in the form of the integral transform \eqref{eq:tmu_def} is possible, exists for the specific class $L^2$ and, as will be shown later, if $\mathbf{T} \left[X\right]\left(\mu\right)$ is known then the component of $X(t)$ that belongs to the eigenspace can be reconstructed from it using an inverse transform. Hence, in a short description, it makes us fairly happy to know that the projection of a complex signal $X$ onto the kernel of $\ndpo{k}$ yields its null component $X_\eta$, and the projection of $X$ onto the eigenspace yields a ZES signal $\tilde{X}$ that represents it in the correspondent equivalence class of ZES signals, such that $X(t)$ can be completely reconstructed from the sum of $\tilde{X} + X_\eta$.

%-------------------------------------------------
\section{Connection with the Laplace Transform} \label{sec:laplace_connection}%<<<1

	Having defined the transform $\mathbf{T}\left[X\right]$ as the projection of a particular element onto the eigenbasis of $\ndpo{k}$, we can adjust its definition to yield a new transfom

\begin{equation} \mathbf{M}\left[X\right]\left(\mu\right) = \mathbf{T}\left[X\right]\left(-\overline{\mu}\right) = \int_{-\infty}^{\infty} X(t) e^{-\mu t} \overline{R_0(t)}dt = \mathbf{L}\left[X(t) \overline{R_0(t)}\right]\left(\mu\right) ,\label{eq:mtransf_mu_def}\end{equation}

	\noindent that is, $\mathbf{M}\left[X\right]$ is in essence a Laplace Transform of $X$ rotated by $\psi(t)$. Interestingly, this number is exactly the projection of $X$ onto the real axis of its stationary frame, as per figure \ref{fig:dynamic_phasor_imreaxis} — wherefore one concludes that the transform $\mathbf{M}\left[X\right]$ is basically rotating the Dynamic Phasor back to the static frame and applying the Laplace Transform at the projected quantity. However, because the originary frame is stationary, this means that \textbf{this transform does not depend on the apparent frequency signal chosen}.

	This can be seen by the fact that if a certain generalized sinusoid $x(t) = m(t)\cos\left(\theta(t)\right)$ is given, one can obtain the transform of its Dynamic Phasor without actually calculating the Dynamic Phasor itself:

\begin{equation} \mathbf{M}\left[X\right] = \mathbf{L} \left[m(t)e^{j\theta(t)}\right], \label{eq:mutransf_from_time}\end{equation}

	\noindent and the absolute angle $\theta(t)$ does not depend on $\omega(t)$. Thus, if $x(t)$ has two Dynamic Phasor representations, $X(t)$ and $\tilde{X}$, each measured at its own apparent frequencies but these frequencies are equivalent, they have the exact same $\mu$ Transform, which is the conclusion of theorem \ref{theo:muT_indep_freq}.

\begin{theorem}[The $\mu$T is invariant to the apparent frequency signal chosen]\label{theo:muT_indep_freq} Let $X(t)\in\left[\mathbb{R}\to\mathbb{C}\right]$ produced at an apparent frequency signal $\omega(t)$, and suppose $X(t)$ admits a $\mu$ Transform. Let $\tilde{X}$ produced at the apparent frequency $\tilde{\omega}$ where $\tilde{\omega}$ is equivalent to $\omega(t)$ (see the definition \ref{def:equivalent_freqs} of equivalence between apparent frequency signals). Then the $\mu$ Transform of $\tilde{X}$ at $\tilde{\omega}$ is equal to the $\mu$ Transform of $X(t)$ at $\omega(t)$.
\end{theorem}\vspace{3mm}\hrule\vspace{3mm}

	Another curious consequence of equation \eqref{eq:mtransf_mu_def} is that if $\omega(t) = 0$ then $\psi = \int_0^t \omega(s)ds = 0$, so

\begin{equation} \mathbf{M}\left[X\right]\left(\mu\right) = \mathbf{L}\left[X(t) e^{j0}\right] = \mathbf{L}\left[X\right], \end{equation}

	\noindent therefore $\mathbf{M}$ coincides with $\mathbf{L}$. In a certain sense, this means that the Laplace Transform is a particular case of the $\mu$ Transform.

	Thus let us  henceforth call $\mathbf{M}$ as the $\mu$-Transform or $\mu$T for short. Because of this connection with the Laplace Transform the properties of $\mathbf{L}$ apply here, mainly that $\mathbf{M}\left[X\right]\left(\mu\right)$ has a Region of Convergence denoted by

\begin{equation}\text{ROC}\left(\mathbf{M}\left[X\right]\right) = \left\{\mu\in\mathbb{C}: \int_{-\infty}^{\infty} \left\lvert X e^{-\text{Re}\left(\mu\right) t}\right\rvert dt < \infty \right\} \end{equation}

	\noindent such that a signal $X(t)$ admits a $\mu$T if the ROC is not empty; also, alike the Laplace Transform, $\mathbf{M}\left[X\right]\left(\mu\right)$ is analytic in the ROC. Also, if $X(t)$ is a causal signal then the ROC is of the form $\text{Re}\left(\mu\right) > a$ for some real $a$, possibly containing some points of the line $\text{Re}\left(\mu\right) = a$.

	Curiously, substituting \eqref{eq:mutransf_from_time} into the condition for the ROC, and using the fact that $\left\lvert e^{jx}\right\rvert = 1$ for any real $x$,

\begin{equation} \int_{-\infty}^{\infty} \left\lvert m(t)e^{j\theta(t)} e^{-\text{Re}\left(\mu\right) t}\right\rvert dt < \infty \Leftrightarrow \int_{-\infty}^{\infty} \left\lvert m(t) e^{-\text{Re}\left(\mu\right) t}\right\rvert dt < \infty \end{equation}

	\noindent yielding the conclusion that a Dynamic Phasor has a $\mu$-Transform if and only if its amplitude $m(t)$ has a Laplace Transform, that is, if $X(t)$ has a Laplace Transform itself.

	Therefore, the admissibility of a $\mu$ Transform is closely related to the admissibility of a Laplace Transform; furtermore, the absolute angle $\theta(t)$ of a sinusoid, nor the argument of its Dynamic Phasor, play a part in such admissibilities. The list of such conclusions proves theorem \ref{theo:admissible}.

\begin{theorem}[Admissibility of a $\mu$ Transform]\label{theo:admissible} Consider a sinusoid $x(t)$, $X(t)\in\left[\mathbb{R}\to\mathbb{C}\right]$ its correspondent Dynamic Phasor at some apparent frequency, Then the following sentences are equivalent:

\begin{itemize}
	\item The Dynamic Phasor $X(t)$ admits a $\mu$ Transform;
	\item The Dynamic Phasor $X(t)$ admits a Laplace Transform;
	\item The sinusoid $x(t)$ admits a Laplace Transform;
	\item The amplitude $m(t)$ of $x(t)$ and $X(t)$ admits a Laplace Transform;
	\item There exists $\alpha\in\mathbb{R}$ such that $\left\lvert m(t)\right\rvert e^{\alpha t} = \left\lvert X(t)\right\rvert e^{\alpha t}$ is square-integrable.
\end{itemize}

\end{theorem}\hrule\vspace{3mm}

	Moreover, naturally we ask if the argument $X(t)$ can be retrieved from $\mathbf{M}\left[X\right]$ through some inversion formula, that is, if given $\mathbf{M}\left[X\right] = F\left(\mu\right)$ there is some inverse transform onto $F$ that yields $X(t)$. The connection with the Laplace Transform naturally yields a candidate to inverse transform

\begin{align}
	\mathbf{M}^{-1}\left[F\right](t) &= \mathbf{L}^{-1}\left[F\left(\mu\right)R_0(t)\right] = \dfrac{1}{2\pi j}\lim_{\beta\to\infty} \int_{\alpha- j\beta}^{\alpha + j\beta} F\left(\mu\right) R_0(t) e^{\mu  t} d\mu = \nonumber\\[3mm] &= \dfrac{R_0(t)}{2\pi j}\lim_{\beta\to\infty} \int_{\alpha- j\beta}^{\alpha + j\beta} F\left(\mu\right) e^{\mu  t} d\mu = R_0(t)\mathbf{L}^{-1}\left[F\right].\label{eq:xmu_inverse_prop}
\end{align}

	One concludes that this prospective inverse transform makes a lot of sense: since $\mathbf{M}$ is essentially a rotation of its argument onto the stationary frame and the subsequent Laplace Transform of the projected signal, the inverse transform is the inverse Laplace Transform and then a rotation back to the DQ frame generated by the Dynamic Phasor Transform. Again we notice that if the apparent frequency $\omega(t) = 0$ then $\mathbf{M}^{-1}$ coincides with $\mathbf{L}^{-1}$, again showcasing that the $\mu$ Transform is some generalization of the Laplace Transform.

	We now prove that the formula \ref{eq:xmu_inverse_prop} indeed reconstructs the time signal intended in theorem \ref{theo:xmu_reconst}. 

\begin{theorem}[Inverse $\mu$ Transform]\label{theo:xmu_reconst} %<<<
	Let $X(t)\in\left[\mathbb{R}\to\mathbb{C}\right]$ a ZES signal and suppose there is some real $\alpha$ such that $Xe^{\alpha t}$ and $X'(t)e^{\alpha t}$ are Lebesgue integrable (in $L^1\left(\mathbb{R}\right)$). Take $\mathbf{T}\left[X\right]$ as in \eqref{eq:tmu_def}. Then 

	\begin{equation} X(t) = \dfrac{R_0(t)}{2\pi j}\lim_{\beta\to\infty} \int_{\alpha- j\beta}^{\alpha + j\beta} \mathbf{M} \left[X\right]\left(\mu\right) e^{\mu  t} d\mu \label{eq:xmu_inverse}\end{equation}

	\noindent or succintly
        
	\begin{equation} X(t) = \dfrac{R_0(t)}{2\pi j}\int_{B_\alpha} \mathbf{M}_{\left[X\right]}\left(\mu\right) e^{\mu  t} R_0\left(t\right) d\mu \label{eq:xmu_inverse_brom}\end{equation}
        
	\noindent where $B_\alpha = \left(\alpha - j\infty,\alpha + j\infty\right)$ is a Brömwich contour.
\end{theorem}
\textbf{Proof:} define

\begin{equation} D(t) = \dfrac{1}{2} + \dfrac{1}{\pi}\int_{-\infty}^t \dfrac{\sin\left(s\right)}{s}ds \end{equation}

	Notably, $D(t)$ is bounded and by the Dirichlet Integral

\begin{equation} \lim_{\alpha\to\infty} D\left(\alpha t\right) = H(t) = \left\{\begin{array}{l} 1 \text{, if } t > 0 \\[3mm] \dfrac{1}{2} \text{, if } t = 0 \\[3mm] 0 \text{, if } t < 0 \end{array}\right. \end{equation}

	\noindent with $H(t)$ an adapted version of the Heaviside step function. Take $X(t)$ as defined in the \textit{caput}. Then

\begin{align}
	I(t) &= \dfrac{R_0(t)}{2\pi} \int_{-\beta}^{\beta} \mathbf{M}\left[X\right]\left(\alpha + j\gamma\right) e^{\left(\alpha + j\gamma\right)  t} d\gamma \nonumber\\[3mm]
	     &= \dfrac{R_0(t)}{2\pi} \int_{-\beta}^{\beta} \left(\int_{-\infty}^{\infty} X(s) e^{\left(\alpha + j\gamma\right) s} \overline{R_0(s)}ds\right) e^{\left(\alpha + j\gamma\right)  t} \gamma\nonumber\\[3mm]
	     &= \dfrac{R_0(t)}{2\pi} \int_{-\beta}^{\beta} \left(\int_{-\infty}^{\infty} X(s) e^{\left(\alpha - j\gamma\right) s} \overline{R_0(s)}ds\right) e^{\left(-\alpha + j\gamma\right)  t} d\gamma
\end{align}

	By Fubini's Theorem, the integration order can be changed:

\begin{align}
	I(t) &= \dfrac{R_0(t)}{2\pi} \int_{-\infty}^{\infty} X(s) e^{\left(\alpha - j\gamma\right) s} \overline{R_0(s)} \left(\int_{-\beta}^{\beta} e^{\left(-\alpha + j\gamma\right)  t}  d\gamma\right)ds \nonumber\\[3mm]
	     &= \dfrac{R_0(t)}{2\pi} \int_{-\infty}^{\infty} X(s) e^{\alpha\left(s - t\right)} \overline{R_0(s)} \left(\int_{-\beta}^{\beta} e^{j\gamma\left(t - s\right)} d\gamma\right)ds \nonumber\\[3mm]
	     &\hspace{-3mm}= R_0(t) \left[\int_{-\infty}^{\infty} X(s) e^{\alpha\left(s - t\right)} \overline{R_0(s)} \left\{\dfrac{\sin\left[\beta\left(t-s\right)\right]}{\pi\left(t-s\right)} \right\} ds \right]
\end{align}

	By the assumption, $X(s)e^{\alpha s}$ and $\left(Xe^{\alpha s}\right)' = \left(X'(s) + \alpha X(s)\right)e^{\alpha s}$ are $L^1$. This can only be possible if both $X$ and its derivative converge exponentially to zero as the infinites, that is, $X(s)e^{\alpha s}\to 0$ and $X'(s)e^{\alpha s},\to 0$ as $s\to \pm\infty$. Because $\left\lvert R_0(s)\right\rvert$ = 1, this also yields $X(s)e^{\alpha s}\overline{R_0(s)}\to 0$ and $X'(s)e^{\alpha s}\overline{R_0(s)}\to 0$ as $s$ goes to positive or negative infinity and integration by parts yields

\begin{align}
	I(t) &= R_0(t) \left\{ \begin{array}{l} \left[X(s) e^{\alpha\left(s - t\right)} D\left[\beta\left(s-t\right)\right]\overline{R_0(s)}\right]_{s\to -\infty}^{s\to\infty} + \\[3mm] \hspace{1cm}+\displaystyle\int_{-\infty}^{\infty} \left[X(s)e^{\left(s-t\right)\alpha}\overline{R_0(s)}\right]'D\left[\beta\left(s-t\right)\right] ds \end{array}\right\}
\end{align}

	\noindent and because $H$ is bounded, the first term vanishes, yielding

\begin{equation} I(t) = R_0(t)\left[\int_{-\infty}^{\infty} \left[X(s)e^{\left(s-t\right)\alpha}\overline{R_0(s)}\right]'D\left[\beta\left(s-t\right)\right] ds\right] \end{equation}

	Taking the limit $\beta\to\infty$, because $H$ tends to $u(t)$ which is integrable, the Dominated Convergence Theorem guarantees that the limit can operate inside the integral and

\begin{align}
	I(t) &= R_0(t)\int_{-\infty}^{\infty} \left[X(s)e^{\left(s-t\right)\alpha}\overline{R_0(s)}\right]' H\left(s-t\right) ds \nonumber\\[3mm] &= -R_0(t)\left[X(s)e^{\left(s-t\right)\alpha}\overline{R_0(s)}\right]_{s=t}^{s\to\infty} \nonumber\\[3mm] &= R_0(t) \overline{R_0(t)} X(t) = \left\lvert R_0(t)\right\rvert^2 X(t) = X(t)
\end{align}
\hfill$\blacksquare$\vspace{5mm}\hrule\vspace{5mm} %>>>

	Thus, for $F\in\left[\mathbb{C}\to\mathbb{C}\right]$ we define the \textbf{Inverse $\boldsymbol{\mu}$ Transform} $\mathbf{M}^{-1}$ or simply $\mu$T$^{-1}$ as

	\begin{equation} \mathbf{M}^{-1}\left[F\right] = \dfrac{R_0(t)}{2\pi j}\int_{B_\alpha} F\left(\mu\right) e^{\mu  t} d\mu = R_0(t)\mathbf{L}^{-1}\left[F\right] \label{eq:inv_muT_def}\end{equation}

	Notably, given $X(t) = \mathbf{M}^{-1}\left[F\right]$, the generalized sinusoid that $X$ reconstructs is given by

\begin{equation} x(t) = \Re\left(X(t)e^{j\psi(t)}\right) = \Re\left(X \overline{R_0}(t)\right) = \Re\left(\mathbf{L}^{-1}\left[F\right]R_0(t)\overline{R_0}(t)\right) = \Re\left(\mathbf{L}^{-1}\left[F\right]\right)\end{equation}

	\noindent again hinting at the fact that for some real signal $x(t)$ the $\mu$T of its Dynamic Phasor is independent of the frequency signal $\omega(t)$ chosen. Furthermore, the connection of the $\mu$T with $\mathbf{L}$ also allows for calculating the inverse through the Residue Theorem, in the same pattern as the Laplace Inversion Theorem. Theorem \ref{theo:xmu_reconst_residue} shows that given a $F(z)\in\left[\mathbb{C}\to\mathbb{C}\right]$, one can obtain the Dynamic Phasor reconstructed by this function through the use of two seminal theorems from Complex Analysis: Cauchy's Residue Theorem (theorem \ref{theo:residue_theorem}) and Jordan's Lemma (theorem \ref{theo:jordans_lemma}).

\begin{theorem}[Cauchy's Residue Theorem \pcite{ahlfors1979complex}]\label{theo:residue_theorem} %<<<
	Let $U$ a simply connected open subset of $\mathbb{C}$, and a list of points $\left(a_1,a_2,\cdots,a_n\right)$. Let $U_0 = U\setminus\left\{a_1,a_2,\cdots,a_n\right\}$ and consider a function $f(z)$ is holomorphic on $U_0$. Let $\gamma$ a closed rectifyiable curve in  $U_0$, and denote the residue of $f$ around a point $c$ as

\begin{equation} \Res\left(f,c\right) = \dfrac{1}{2\pi j}\oint_{\gamma_c} f(z)dz ,\end{equation}

	\noindent where $\gamma_c$ is a clockwise circular path around $c$ of radius small so as not to enclose any other singularities but $c$. Also denote the winding number of $\gamma$ around a point $c$ as

\begin{equation} I\left(\gamma,c\right) = \dfrac{1}{2\pi j}\oint_\gamma \dfrac{1}{z} dz\end{equation}.

	Then

\begin{equation} \dfrac{1}{2\pi j}\oint_\gamma f(z)dz = \sum_{k=1}^n I\left(\gamma,a_k\right)\Res\left(f,a_k\right) .\end{equation}

	Particularly, if $\gamma$ is positively oriented and simple, all its winding numbers are $1$ and

\begin{equation} \dfrac{1}{2\pi j}\oint_\gamma f(z)dz = \sum_{k=1}^n \Res\left(f,a_k\right) .\end{equation}

\end{theorem}\vspace{3mm}\hrule\vspace{3mm}%>>>

\begin{theorem}[Jordan's Lemma]\label{theo:jordans_lemma} %<<<
	Consider the semicircular contour $C_R = \left\{Re^{j\theta}: 0\leq \theta\leq \pi\right\}$ with $R$ a positive radius, consisting of a semicircle on the upper-half plane center at the origin. If $f(z)$ is of the form $f(z) = e^{jaz}g(z)$ in $C_R$, with a positive parameter $a$, then

\begin{equation} \left\lvert \int_{C_R} f(z)dz\right\rvert \leq \dfrac{\pi M}{a},\ \text{where } M = \max\limits_{0\leq\theta\leq\pi} \left\lvert g\left(Re^{j\theta}\right)\right\rvert .\end{equation}

	Particularly, if $f$ is continuous on $C_R$ for all large $R$ and 

\begin{equation} \lim_{R\to\infty} M = 0 \end{equation}

	\noindent then

\begin{equation} \lim_{R\to\infty} \int_{C_R} f(z)dz = 0 .\end{equation}

\end{theorem}\vspace{3mm}\hrule\vspace{3mm} %>>>

\begin{theorem}[Calculating the $\mu$T$^{-1}$ through complex poles]\label{theo:xmu_reconst_residue} %<<<
	Let $X(t)\in\left[\mathbb{R}\to\mathbb{C}\right]$ ZES, and choose $\alpha$ such that the line $\Re\left(z\right) < \alpha$ contains all poles of $\mathbf{M}\left[X\right]\left(\mu\right)$, that is, all poles of $\mathbf{M}$ are on the left of $\alpha$. Then

	\begin{equation} X(t) = R_0(t)\int_{\alpha - j\infty}^{\alpha + j\infty} \mathbf{M}\left[X\right]\left(\mu\right) e^{\mu t} d\mu = 2\pi jR_0(t) \sum \Res\left(\mathbf{M}\left[X\right]\left(\mu\right) e^{\mu t},\mu_p\right) \end{equation}

	\noindent where $\mu_p$ are the poles of $\mathbf{M}\left[X\right]\left(\mu\right)$.
\end{theorem}

% SEMICIRCLE CONTOUR DRAWING <<<
\begin{figure}[htb!]
\centering
\scalebox{1}{
	\begin{tikzpicture}[scale=2,>={Stealth[inset=0mm,length=1.5mm,angle'=50]}]
		\draw [->] (   -10mm,  0   ) -- (   20mm,  0   ) node[right] (xaxis) {$\Re$};
		\draw [->] (      0, -15mm ) -- (   0   ,  15mm) node[above] (yaxis) {$\Im$};
		\draw[->,gray,dashed,line cap = round] (12mm,0) -- ({12mm - 14.5mm*cos(45)},{0mm - 14.5mm*sin(45)}) node[midway,label={[label distance = -1mm, rotate=45]above:$R$}] {};
		\node at (13mm,-1mm) {$\alpha$};
		\node[stewartpink,right] at (13mm,7.5mm) {$L_1$};
		\draw [->, stewartpink](12mm,-15mm) -- (12mm,-5mm);
		\draw [->, stewartpink](12mm,-5mm) -- (12mm,+10mm);
		\draw [    stewartpink](12mm,10mm) -- (12mm,+15mm);
		\draw [->,stewartblue] (12mm,15mm) arc[start angle=90, end angle = 160, radius = 15mm] node (start1) {};
		\node[stewartblue] at ({12mm + 18mm*cos(160)},{0mm + 18mm*sin(160)}) {$C_1$};
		\draw [   stewartblue] (start1) arc[start angle=160, end angle = 270, radius = 15mm] {};
		\draw[fill] (12mm,0) circle (0.25mm);
	\end{tikzpicture}
	}
	\caption{Integration contours for theorem \ref{theo:xmu_reconst}.}
	\label{fig:semicircle_complex_1}
\end{figure} %>>>
\textbf{Proof:} consider the semicircle path ${\color{stewartblue} C_1}$ and the horizontal path ${\color{stewartpink} L}$, and let $\gamma_1$ their union. Denote $\mathbf{M}\left[X\right]\left(\mu\right) = F\left(\mu\right)$. Then

\begin{align}
	R_0\left(t\right)\oint_{\gamma_1} F\left(\mu\right)e^{\mu t}d\mu &= R_0\left(t\right)\int_{C_1} Fe^{\mu t}\left(\mu\right)d\mu + R_0\left(t\right)\int_{L_1} Fe^{\mu t}\left(\mu\right)d\mu = \nonumber\\[3mm]
%
	&= R_0\left(t\right)\int_{C_1} Fe^{\mu t}\left(\mu\right)d\mu + R_0\left(t\right)\int_{\alpha - jR}^{\alpha + jR} Fe^{\mu t}\left(\mu\right)d\mu .
\end{align}

	But by Cauchy's Residue Theorem, since $\gamma$ is positively oriented and simple,

\begin{equation} R_0\left(t\right)\oint_{\gamma_1} Fe^{\mu t}\left(\mu\right)d\mu = 2\pi j R_0\left(t\right)\sum_{c\in\Gamma} \Res\left(Fe^{\mu t},c\right) ,\end{equation}

	\noindent where $\Gamma$ is the are enclosed by the curve $\gamma$, that is, the $c$ are the poles of $Fe^{\mu t}$ enclosed by $\gamma_1$, that is, the poles at the left of $\alpha$; thus

\begin{equation} R_0\left(t\right)\int_{C_1} Fe^{\mu t}\left(\mu\right)d\mu + R_0\left(t\right)\int_{\alpha - jR}^{\alpha + jR} Fe^{\mu t}\left(\mu\right)d\mu = 2\pi j R_0\left(t\right)\sum_{c\in\Gamma} \Res\left(Fe^{\mu t},c\right) .\end{equation}

	Naturally, the integral over $\left[\alpha - jR,\alpha + jR\right]$ becomes the integral we require as $R\to \infty$; it is obvious that as $R\to\infty$, $\Gamma$ becomes the half semiplane left of $\alpha$; thus the poles enclosed by $\gamma_1$ become the poles such that $\Re\left(c_k\right) < \alpha$, that is,

\begin{equation} R_0\left(t\right)\lim_{R\to\infty} \int_{C_1} Fe^{\mu t}\left(\mu\right)d\mu + R_0\left(t\right)\int_{\alpha - j\infty}^{\alpha + j\infty} Fe^{\mu t}\left(\mu\right)d\mu = 2\pi j R_0\left(t\right)\sum_{\Re\left(c\right) < \alpha} \Res\left(Fe^{\mu t},c\right) .\end{equation}

	We want to use Jordan's Lemma to prove that the integral over $C_1$ vanishes at $R\to \infty$. Writing $\mu = jz + \alpha$ we translate and rotate $\mu$ so that the integration curve is now the upper semicircle center at the origin:

\begin{equation} F(\mu)e^{\mu t} = \mathbf{M}\left[X\right]\left(jz + \alpha\right) e^{jz + \alpha}. \end{equation}

	Using Jordan's Lemma, we note that this integral is of the form

\begin{equation} F(\mu)e^{\mu t} = \left[\raisebox{4mm}{} \mathbf{M}\left[X\right]\left(jz + \alpha\right) e^{\alpha t}\right]e^{jtz}. \end{equation}

	If $\alpha$ is inside the ROC of $\mathbf{M}$ and $R$ is large enough so $z$ does not touch singularities, then the absolute value of the term in brackets inevitably goes to zero as $R\to\infty$; thus, by Jordan's Lemma,

\begin{equation} \lim_{R\to\infty} \int_{C_R} F(\mu)e^{\mu t}d\mu = 0 \end{equation}

	\noindent yielding

\begin{equation} R_0\left(t\right)\int_{\alpha - j\infty}^{\alpha + j\infty} Fe^{\mu t}\left(\mu\right)d\mu = 2\pi j R_0\left(t\right)\sum \Res\left(Fe^{\mu t},c\right) \end{equation}

	\noindent where the $c$ are all the residues of $Fe^{\mu t}$. Now note that $e^{\mu t}$ is differentiable everywhere (holomorphic) and has no poles, thus the poles of $Fe^{\mu t}$ are the same poles as $\mathbf{M}$; therefore,

\begin{equation} R_0\left(t\right)\int_{\alpha - j\infty}^{\alpha + j\infty} Fe^{\mu t}\left(\mu\right)d\mu = 2\pi j R_0\left(t\right)\sum \Res\left(Fe^{\mu t},\mu_p\right) \end{equation}

	\noindent where the $\mu_p$ are the poles of $\mathbf{M}\left[X\right]\left(\mu\right)$. \hfill$\blacksquare$\vspace{5mm}\hrule\vspace{5mm} %>>>

	Thus, given some complex function $M\left(\mu\right)$, theorem \ref{theo:xmu_reconst_residue} gives a simple way to reconstruct the time signal that the transform defines.

	We now give an example of the application of theorem \ref{theo:xmu_reconst_residue}. As discussed in chapter \ref{chapter:dpos}, the Laplace Transform of a generic signal (and particularly Nonstationary Sinusoids) is not analyticaly representable; this means that any example that we try to build will be in some way innocuous since most probably the signal reconstructed by some simple $\mu$T is probably not applicable to any examples. Such is indeed the case in example \ref{example:muT_reconst}, a sample function is adopted and the signal built from it is reconstructed.

\begin{example}[Reconstruction of a $\mu$T through Residue Theorem]\label{example:muT_reconst} %<<<

	Consider $M\in\left[\mathbb{C}\to\mathbb{C}\right]$:

\begin{equation} M\left(\mu\right) = \dfrac{3\mu - 22}{\left(\mu - 2j\right)\left(\mu + 5\right)^2} \end{equation}.

	Then $M$ has poles at $\mu = 2j$ and $\mu = -5$, the latter being a double pole; thus it reconstructs the signal

\begin{equation} X(t) = 2\pi j R_0\left(t\right)\left[ \Res\left(M\left(\mu\right) e^{\mu t},2j\right) + \Res\left(M\left(\mu\right) e^{\mu t},-5\right)\right] .\end{equation}

	To calculate the residues, we use Laurent's Series (theorem \ref{theo:laurent}): the residue of a pole of order $n$ is calculated as

\begin{equation} \Res\left(f,z_0\right) = \dfrac{1}{\left(n-1\right)!} \lim_{z\to z_0} \dfrac{d^{(n-1)}}{dz^{(n-1)}} \left[\left(z - z_0\right)^n f(z)\right] \end{equation}

	Starting with the simple pole at $\mu = 2j$,

\begin{align}
	\Res\left(M\left(\mu\right) R_0\left(t\right)e^{\mu t},2j\right) &= \lim_{\mu\to 2j}\left(\mu-2j\right)M\left(\mu\right) R_0\left(t\right)e^{\mu t} = R_0\left(t\right) \lim_{\mu\to 2j} \dfrac{3\mu - 22}{\left(\mu + 5\right)^2} e^{\mu t} = \nonumber\\[3mm] &= R_0\left(t\right) \dfrac{6j - 22}{\left(2j + 5\right)^2} e^{2jt} =  \dfrac{-342 + j566}{841} e^{j\left(2t + \psi(t)\right)}
\end{align}

	And for the double pole at $\mu = -5$,

\begin{align}
	R_0(t) \Res\left(M\left(\mu\right)e^{\mu t},2j\right) &= \lim_{\mu\to 2j}\left(\mu-2j\right)M\left(\mu\right) R_0\left(t\right)e^{\mu t} = R_0\left(t\right) \lim_{\mu\to 2j} \dfrac{3\mu - 22}{\left(\mu + 5\right)^2} e^{\mu t} = \nonumber\\[3mm] &= R_0\left(t\right) \dfrac{6j - 22}{\left(2j + 5\right)^2} e^{2jt} = \dfrac{6j - 22}{\left(2j + 5\right)^2} e^{j2t - j\psi(t)}
\end{align}

	For the second pole,

\begin{align}
	R_0(t) \Res\left(M\left(\mu\right)e^{\mu t},2j\right) &= R_0(t) \dfrac{1}{1!} \lim_{\mu\to -5} \dfrac{d}{d\mu}\left[\left(\mu+5\right)^2 M\left(\mu\right) e^{\mu t}\right] = \nonumber\\[3mm]
%
	&= R_0\left(t\right) \lim_{\mu\to -5} \dfrac{d}{d\mu}\left[\dfrac{\left(3\mu - 22\right)e^{\mu t}}{\left(\mu - 2j\right)}\right] \nonumber\\[3mm]
%
	&= R_0\left(t\right) \lim_{\mu\to -5} \dfrac{\left(\mu - 2j\right)\left[3e^{\mu t} + t\left(3\mu - 22\right)e^{\mu t}\right] - \left(3\mu - 22\right)e^{\mu t}}{\left(\mu - 2j\right)^2} = \nonumber\\[3mm]
%
	&= R_0\left(t\right) \left[\dfrac{\left(-5 - 2j\right)\left[3e^{-5 t} + t\left(-15 - 22\right)e^{-5 t}\right] - \left(-15 - 22\right)e^{-5 t}}{\left(-5 - 2j\right)^2}\right] = \nonumber\\[3mm]
%
	&= R_0\left(t\right)e^{-5t} \left[\dfrac{-\left(5 + 2j\right)\left(3 - 37t\right) + 37}{\left(5 + 2j\right)^2}\right] = \nonumber\\[3mm]
%
	&= \left[\dfrac{-\left(5 + 2j\right)\left(3 - 37t\right) + 37}{\left(5 + 2j\right)^2}\right]e^{-5t - j\psi(t)}
\end{align}

	Thus the reconstructed signal is

\begin{equation} X(t) = \mathbf{M}^{-1}\left[F\right] = \dfrac{1}{2\pi j}\dfrac{6j - 22}{\left(2j + 5\right)^2} e^{j2t - j\psi(t)} + \dfrac{1}{2\pi j}\left[\dfrac{-\left(5 + 2j\right)\left(3 - 37t\right) + 37}{\left(5 + 2j\right)^2}\right]e^{-5t - j\psi(t)} \end{equation}

	The real signal reconstructed by this Dynamic Phasor is

\begin{equation}  x(t) = \Re\left(\mathbf{M}^{-1}\left[F\right]\overline{R_0}(t)\right) = \dfrac{2\sqrt{130}}{58\pi}\cos\left(2t + \arctan\left(\dfrac{171}{283}\right)\right) + \dfrac{\left(4t - 6\right)e^{-5t}}{58\pi} \end{equation}

	\noindent which is indeed a simple function, again illustrating that only particular functions (static sinusoids, exponentials, polynomials and so on) will have ``nice'' Laplace transforms and nice $\mu$ transforms.

\examplebar
\end{example} %>>>

%-------------------------------------------------
\subsection{The Final Value Theorems}

	One of the also glaring advantages of the connection between the $\mu$T and the Laplace Transform is the fact that we can leverage the final value theorem available for the Laplace Transform to prove a version of this theorem for the $\mu$ Transform.

\begin{theorem}[Final Value Theorem for the Laplace Transform \pcite{chenFinalValueTheorem2007}]\label{theo:laplace_fvt}
	Suppose $F(s)\in\left[\mathbb{C}\to\mathbb{C}\right]$ with poles in either the open left half place or the origin, and that $F(s)$ has at most a single pole at the origin. Then $f(t) = \mathbf{L}^{-1}\left[F\right]$ exists, $sF(s)\to L\in\mathbb{R}$ as $s\to 0$ and

\begin{equation} \lim\limits_{s\to\ 0} sF(s) = \lim\limits_{t\to\infty} f(t) \end{equation}
\end{theorem}
\hrule
\vspace{3mm}

	While more powerful versions of the Final Value Theorem exist, as shown in \cite{chenFinalValueTheorem2007}, that of theorem \ref{theo:laplace_fvt} is the ``standard'' one by which everyone knows the theorem.

	For the $\mu$ Transform, we can use \eqref{eq:mtransf_mu_def} which directly relates $\mathbf{M}\left[\cdot\right]$ to $\mathbf{L}\left[\cdot\right]$. Thus, taken at face-value, theorem \ref{theo:laplace_fvt} would mean that

\begin{equation} \lim\limits_{s\to\ 0} \mu \mathbf{M}\left[X(t) R_0(t)\right]\left(\mu\right) = \lim\limits_{t\to\infty} X(t) \end{equation}

	\noindent but this would not be really of use because we want to have a limit in terms of $\mathbf{M}\left[X(t)\right]$ but not $\mathbf{M}\left[X(t) R_0(t)\right]$. We instead present an adaptation of this theorem: we suppose that the $\mu$ Transform of $X(t)$ is equivalent to the Laplace transform of $X(t)$ composed with a function $h$ that is,

\begin{equation} \mathbf{M}\left[X\right](\mu) = \mathbf{L}\left[X(t)R_0(t)\right](\mu) = \mathbf{L}\left[X(t)\right]\left(h\left(s\right)\right). \end{equation}

	\noindent where $h\in\left[U_0\to\mathbb{C}\right]$ is defined in some neighborhood of the origin $U_0$. In other words, for the specific signal $X(t)$, the $\mu$T is equivalent to the LT and some ``distortion'' $h$ in some vicinity of the origin. If this $h$ is ``nice'' around the origin (continuously differentiable at the origin) then theorem \ref{theo:laplace_fvt} is rather easy to adapt to the $\mu$ Transform.

\begin{theorem}[Final Value Theorem for $\mu$ Transforms]\label{theo:muT_fvt} %<<<
	Consider a continuous function $M\in\left[A\subset\mathbb{C}\to\mathbb{C}\right]$ and suppose there exists a continuous $h\in\left[S\subset\mathbb{C}\to\mathbb{C}\right]$ with $h(S) \subset A$ such that $h$ is continuously differentiable at the origin and the composition $ F\left(\mu\right) = M\left(h\left(\mu\right)\right)$ has poles either at the origin or at the open left half plane, with at most one pole at the origin. Then $M$ admits an inverse $\mu$ Transform $X(t)$ which is also the inverse Laplace Transform of $F$ and 

\begin{equation} \lim\limits_{\mu\to 0} \mu M\left(\mu_0 + h'_0\mu\right) = \lim\limits_{t\to\infty} X(t). \label{eq:fvt_uT_4}\end{equation}

	\noindent where $h\left(U_0\right) \ni \mu_0 = h(0)$ and $h'(0) = h'_0$.
\end{theorem}
\textbf{Proof:} clearly, $M$ is the candidate to the $\mu$T of some signal $X(t)$; let

\begin{equation} L\left(\mu\right) = M\left(h\left(\mu\right)\right) \end{equation}

	\noindent the candidate to the LT of $X(t)$. Multiplying both sides by $\mu$,

\begin{equation} \mu L\left(\mu \right) = \mu M\left(h\left(\mu\right)\right) \label{eq:fvt_uT_def}\end{equation}

	\noindent and by the Final Value Theorem for the LT the left limit exists at $\mu\to 0$ and $L$ admits an inverse transform $X(t)$. Because $h$ is continuously differentiable at the origin, it is analytic and continuously differentiable in some neighborhood of the origin, say $U_0$. Also because $M$ is continuous, so is $L$ and the limits of compositions of continuous functions can be used to yield

\begin{equation} \lim\limits_{s\to 0} \mu L\left(\mu\right) = \lim\limits_{\mu\to h(0)} \mu M\left(h\left(\mu\right)\right) \end{equation}

	\noindent therefore the limit on the right surely exists. Further, because $h$ is analytic in $U_0$ (it has a converging Taylor expansion), we denote $h(0) = \mu_0$ and $h'(0) = h'_0$ and Taylor expansion yields

\begin{equation} h(\mu) = \mu_0 + h'_0\mu + O\left(\mu\right)^2,\ \mu \in U_0. \end{equation}

	\noindent thus by \eqref{eq:fvt_uT_def}

\begin{equation}
	\mu L\left(\mu\right) = \mu M\left(h\left(\mu\right)\right) = \mu M\left[\mu_0 + h'_0\mu + O\left(\mu\right)^2\right].
\end{equation}

	Now because $M$ is supposed continuous we again use that the limit of a composition of continuous functions is the composition of the limits together with the linearity of limits and

\begin{equation} \lim\limits_{\mu\to 0} \mu M\left[\mu_0 + h'_0\mu + O\left(\mu\right)^2\right] = \lim\limits_{\mu\to 0} \mu M\left(\mu_0 + h'_0\mu\right) \end{equation}

	Therefore,

\begin{equation} \lim\limits_{\mu\to 0} \mu M\left(\mu_0 + h'_0\mu\right) = \lim\limits_{t\to\infty} X(t). \label{eq:fvt_uT_2}\end{equation}

	Multiplying this entire equation by $h'_0$ and using the linearity of limits with $\tau = \mu_0 + h'_0\mu$ yields

\begin{equation} \lim\limits_{\tau\to \mu_0} \left(\tau - \mu_0\right) M\left(\tau\right) = h'_0\lim\limits_{t\to\infty} X(t). \label{eq:fvt_uT_3}\end{equation}
\hfill$\blacksquare$
\begin{remark} If a single function $h$ exists with non-null $h'_0$, then one can take the scaled function $g = h(\mu)/h'_0$ so that $g'_0 = 1$ and the theorem yields

\begin{equation} \lim\limits_{\mu\to \mu_0} \left(\mu - \mu_0\right)M\left(\mu\right) = \lim\limits_{\mu \to 0} \mu M\left(\mu + \mu_0\right) = \lim\limits_{t\to\infty} X(t),\ \mu_0 = g(0).\end{equation}

	On the other hand, if $h'(0) = 0$ then the result becomes trivial

\begin{equation} \lim\limits_{\mu \to 0} \mu M\left(\mu_0\right) = \lim\limits_{t\to\infty} X(t) = 0.\end{equation}
\end{remark}
\begin{remark}\label{remark:theo_muT_fvt_null_freq} If the apparent frequency $\omega(t)$ is identically null, then $R_0(t) = 1$, and the $\mu$ Transform is equivalent to the Laplace Transform. Thus $h(\mu) = \mu$ is used, yielding $\mu_0 = 0,\ h'(0) = 1$ and this Final Value Theorem for $\mu$Ts simplifies into the theorem for the Laplace Transform (theorem \ref{theo:laplace_fvt}).
\end{remark}
\hrule
\vspace{3mm} %>>>

	It becomes obvious that the function $h$ and its existence depends on the signal $X(t)$ and the reference signal $R_0(t)$, which is defined by the apparent frequency signal $\omega(t)$ chosen. Thus, these results are still somewhat underwhelming because in a control system where many signals are time-varying, this makes the application of theorem \ref{theo:muT_fvt} streunous because it depends on finding such an $h$ for each signal involved. If we assume that the apparent frequency signal $\omega(t)$ involved in the Dynamic Phasor Transform is equivalent (see definition \ref{def:equivalent_freqs}) to a constant frequency $\omega_0$, then we can analyze the $\mu$-Transform taken at $\omega_0$ because, by theorem \ref{theo:homeomorphic_phasors}, the signals obtained at $\omega(t)$ are diffeomorphic to those obtained using $\omega_0$ and the signals have the same $\mu$ Transform as their counterparts measured at their own apparent frequencies, as per theorem \ref{theo:muT_indep_freq}.

	The benefit of this is that, by adopting the constant apparent frequency $\omega_0$, we have

\begin{equation} M_X\left(\mu\right) = \mathbf{L}\left[Xe^{j\omega_0 t}\right]\left(\mu\right) \end{equation}

	\noindent and by the frequency shift property of the Laplace Transform,

\begin{equation} M_X\left(\mu\right) = \mathbf{L}\left[Xe^{j\omega_0 t}\right]\left(\mu\right) = L_X\left(\mu - j\omega_0\right)\end{equation}

	\noindent thus the function $h$ given by $h(\mu) = \mu + j\omega_0$ is the transformation candidate \textit{for all signals involved}. But this function is infinitely differentiable (holomorphic) in the entire complex plane and its derivative at the origin is unitary; this yields corollary \ref{corollary:fvt_const_freq}.

\begin{corollary}\label{corollary:fvt_const_freq} Consider an arbitrary sinusoid $x(t)$ with a Laplace transform and an absolute angle $\theta(t)$ that admits an apparent frequency $\omega_0$, that is, the equation $\phi(t) = \theta(t) - \omega_0 t$ has a solution. Then for any Dynamic Phasor of $X(t)$ measured at any other apparent frequency $\omega(t)$ equivalent to $\omega_0$,

\begin{equation} \lim\limits_{\mu\to j\omega_0} \left(\mu - j\omega_0\right) M_X\left(\mu\right) = \lim\limits_{\mu\to 0} \mu M_X\left(\mu + j\omega_0\right) = \lim\limits_{t\to\infty} X\left(t\right) .\end{equation}

\end{corollary}

%-------------------------------------------------
\section{The $\mu$ Transform on DPFs and consequences on linear systems} \label{subsec:mutransf_and_dpos} %<<<1

	We now explore the operational properties of DPFs. Like the Laplace Transform translates a n-th order differential operator in time as a complex multiplication by $s^n$ in the frequency domain, $M$ translates a k-th order differentiation in time (equivalent to the $\ndpo{k}$ in the Dynamic Phasor domain) to a multiplication by $\mu^k$.

\begin{theorem}[$\mu$ Transform and DPFs]\label{theo:mu_transf_and_dpfs} %<<<
	Suppose a $C^k$-class $x(t)$ for some natural $k$ that is a ZES sinusoid at the apparent frequency $\omega(t)$ that admits a Laplace Transform. Consider the signal $y(t) = x^{(k)}(t)$, such that $Y(t) = \ndpo{k}\left[X\right]$. Then

\begin{equation} \mathbf{M}\left[Y\right] = \mu^k \mathbf{M}\left[X\right] \text{ and } Y(t) = \mathbf{M}^{-1}\left[\mu^k \mathbf{M}\left[X\right]\right] .\end{equation}
\end{theorem}
\textbf{Proof.} Computing $M_Y$ for $k=1$:

\begin{equation} M_Y = \int_{\mathbb{R}} \dpo \left[X(t)\right]  e^{-\mu  t} \overline{R_0\left(t\right)} dt = \int_{\mathbb{R}} \left[\dot{X} + j\omega X\right]  e^{-\mu  t} \overline{R_0\left(t\right)} dt = \int_{\mathbb{R}} \dfrac{d}{dt}\left[X \overline{R_0\left(t\right)}\right] e^{\mu t} dt \end{equation}

	\noindent and applying integration by parts,

\begin{equation}  M_Y = \left[X e^{-\mu t} \overline{R_0\left(t\right)}\right]_{-\infty}^{\infty} + \mu \int_{\mathbb{R}} X(t) e^{-\mu  t} \overline{R_0\left(t\right)} dt .\end{equation}

	Here, because $x(t)$ admits a Laplace Transform, then its amplitude is of some exponential order, thus $X(t)e^{-\mu t}$ vanishes to zero at both extrema; thus

\begin{equation}  M_Y = \mu \int_{\mathbb{R}} X(t) e^{-\mu  t} \overline{R_0\left(t\right)} dt = \mu M_X.\end{equation}

	By induction, $M_Y = \mu^k M_X$ for $k\in \mathbb{N}$ and, by inversion, this formula also extends to negative $k$, thus it is also valid for any integer $k$.

	We can also show that $\mu^k M_X(\mu)$ reconstructs $Y(t)$:

\begin{equation} Y(t) = \ndpo{k}\left[X\right] = \ndpo{k} \left[\dfrac{1}{2\pi j}  \int_{B_\alpha} M_X  e^{\mu  t} R_0\left(t\right) d\mu \right]  \end{equation}

	\noindent because the limit is not on $t$, $\ndpo{k}$ can operate inside the limit, and because it is linear it can operate inside the integral. Because the integral limits are not functions of $t$, by Leibnitz' Integral Rule,

\begin{equation} Y(t) = \dfrac{1}{2\pi j}\int_{B_\alpha} \ndpo{k} \left[ M_X  e^{\mu  t} R_0\left(t\right) \right] d\mu  \end{equation}

	\noindent because $M_X$ is not a function of time,

\begin{equation} Y(t) = \dfrac{1}{2\pi j}\int_{B_\alpha} M_X \ndpo{k} \left[e^{\mu  t} R_0\left(t\right) \right] d\mu  \end{equation}

	\noindent and because $e^{\mu t}R_0(t)$ is an eigenvector with eigenvalue $\mu^k$,

\begin{equation} Y(t) = \dfrac{1}{2\pi j} \int_{B_\alpha} M_X \mu^k e^{\mu  t} R_0\left(t\right) d\mu  \end{equation}

	\noindent meaning $Y(t) = \ndpo{k}\left[X\right] = \mathbf{M}^{-1}\left[\mu^k M_X\right]$.
\hfill$\blacksquare$
\vspace{3mm}
\hrule
\vspace{3mm} %>>>
\begin{corollary}[$\mu$ Transforms and DPFs for non ZES signals] %<<<
	If $X$ is not ZES, then to apply the theorem one must remove the initial conditions of $X$ to obtain its ZES equivalent using \eqref{eq:zfs_reconst}. Thus, if the initial conditions of $X(t)$ are

\begin{equation} X_0,X'_0,X''_0,\cdots,X^{(\mathbf{c}\left[X\right])}_0\end{equation}

	\noindent where $\mathbf{c}\left[X\right]$ can be infinite, then $\mathbf{c}\left[Y\right] = \mathbf{c}\left[X\right] - k$ and the initial conditions of $Y(t)$ are such that $Y^{i}_0 = X^{(i+k)}_0$. Then we use the theorem on the ZES equation $\mathbf{M}\left[\tilde{Y}\right] = \mu^k\mathbf{M}\left[\tilde{X}\right]$,\ where

\begin{equation} \tilde{Z}(t) = Z(t) - \sum_{i=0}^{\mathbf{c}\left[Z\right]} Z^{(i)}_{(0)}R_i(t)u(t) \end{equation}

	\noindent with $u(t)$ the Heaviside step distribution to obtain

\begin{gather}
	\mathbf{M}\left[ Y(t) - \sum_{j=0}^{\mathbf{c}\left[Y\right]} Y^{(j)}_{(0)}R_j(t) u(t)\right] = \mu^k \mathbf{M}\left[ X(t) - \sum_{i=0}^{\mathbf{c}\left[X\right]} X^{(i)}_{(0)}R_i(t)u(t)\right] \nonumber\\[3mm]
%
	 \mathbf{M}\left[ Y(t) - \sum_{i=0}^{\mathbf{c}\left[X\right] - k} X^{(j+k)}_{(0)}R_j(t)u(t)\right]  = \mu^k \mathbf{M}\left[ X(t) - \sum_{i=0}^{\mathbf{c}\left[X\right]} X^{(i)}_{(0)}R_i(t)u(t)\right] \nonumber\\[3mm]
%
	 \mathbf{M}\left[ Y(t) \right] - \sum_{i=0}^{\mathbf{c}\left[X\right] - k} X^{(j+k)}_{(0)} \mathbf{M}\left[R_j(t)u(t)\right]  = \mu^k \left\{\mathbf{M}\left[X(t)\right] - \sum_{i=0}^{\mathbf{c}\left[X\right]} X^{(i)}_{(0)} \mathbf{M}\left[R_i(t)u(t)\right] \right\} \nonumber\\[3mm]
%
	 M_Y - \sum_{i=0}^{\mathbf{c}\left[X\right] - k} X^{(j+k)}_{(0)} \mathbf{M}\left[R_j(t)u(t)\right] = \mu^k M_X - \mu^k\sum_{i=0}^{\mathbf{c}\left[X\right]} X^{(i)}_{(0)} \mathbf{M}\left[R_i(t)u(t)\right].
\end{gather}

	But

\begin{equation} \mathbf{M}\left[R_ku(t)\right] = \mathbf{M}\left[\dfrac{t^k}{k!} R_0(t)u(t)\right] = \mathbf{L}\left[\dfrac{t^k}{k!}u(t)\right] = \mu \end{equation}

	\noindent resulting

\begin{equation}  M_Y = \mu^k M_X - \sum_{i=0}^{k-1} \mu^{(k-i-1)}X^{(i)}_{(0)}. \label{eq:final_dpo_muT}\end{equation}
\end{corollary}
\hrule
\vspace{3mm} %>>>
\begin{remark} \eqref{eq:final_dpo_muT} is directly equivalent to the Laplace Transform of n-th derivative formula
\begin{equation}  \mathbf{L}\left[x^{(k)}\right] = s^k \mathbf{L}\left[x\right] - \sum_{i=0}^{k-1} s^{(k-i-1)}x^{(i)}_{(0)},\ k\in \mathbb{N}^* .\end{equation}
\end{remark}

%-------------------------------------------------
\subsection{Rational systems and $\mu$TFs} %<<<2

	Consider a system in time with input $x(t)$ and output $y(t)$ given by the ordinary linear differential equation

\begin{equation} \sum_{k=0}^{n} \alpha_kx^{(k)} = \sum_{k=0}^{d} \beta_k y^{(k)} . \label{eq:odesystem_def}\end{equation}

	Applying the Laplace Transform to both sides yields a transfer function

\begin{equation} \dfrac{Y(s)}{X(s)} = G(s) = \dfrac{\displaystyle\sum_{k=0}^{d} \beta_k s^k}{\displaystyle\sum_{k=0}^{n} \alpha_k s^k} = \dfrac{N(s)}{D(s)},\end{equation}

	\noindent with $N(s)$ and $D(s)$ polynomials. This is equivalent to stating that a linear time invariant differential system yields a \textbf{rational transfer function}; if the degree of the denominator is higher than that of the numerator, this is also called \textbf{proper}. Because most control systems are of such characteristic, rational and proper transfer functions are studied at length in control theory. This definition can be extended to DPTFs: apply the $\dpo$ to \eqref{eq:odesystem_def}:

\begin{equation} \sum_{k=0}^{n} \alpha_k \ndpo{k}\left[X\right]  = \sum_{k=0}^{d} \beta_k \ndpo{k}\left[Y\right] \label{eq:differental_dpo_system}\end{equation}

	\noindent which defines a linear complex operator in $\dpS$:

\begin{equation} Y(t)  = \mathbf{G}\left[X\right],\ \mathbf{G} = \dfrac{\displaystyle\sum_{k=0}^{n} \alpha_k \ndpo{k}}{\displaystyle\sum_{k=0}^{d} \beta_k \ndpo{k}} = \dfrac{\mathbf{N}\left(\dpo\right)}{\mathbf{D}\left(\dpo\right)}.\end{equation}

	\noindent with $N,D\in\mathbb{C}\left[\dpo\right]$. The representation of $\mathbf{G}$ as a ratio of polynomials is highly resemblant of a Transfer Function, if it not were a functional of operators. Supposing $X(t)$ and $Y(t)$ are ZES, using the $\mu$ Transform on \eqref{eq:differental_dpo_system} yields

\begin{equation} \sum_{k=0}^{n} \alpha_k \mu^k \mathbf{M} \left[X\right] = \sum_{k=0}^{d} \beta_k \mu^k \mathbf{M}\left[Y\right] \Leftrightarrow \dfrac{\mathbf{M}\left[Y\right]}{\mathbf{M}\left[X\right]} = \dfrac{N\left(\mu\right)}{D\left(\mu\right)} \label{eq:intuition_mutfs}\end{equation}

	\noindent meaning that one can indeed define a Transfer Function in the $\mu$ context, or a \textbf{$\boldsymbol{\mu}$T Transfer Function ($\boldsymbol{\mu}$TF)} that is a direct representation DP operator $\mathbf{G}$:

\begin{equation} G\left(\mu\right) = \dfrac{N\left(\mu\right)}{D\left(\mu\right)} \end{equation}

	\noindent and, as such, the definition of such entities is available.

\begin{definition}[Mu Transform Transfer functions ($\mu$TFs)]\label{def:muT_TFs}
	Given a continuous-time linear time-invariant system, the $\mu$TF is the relationship relating the $\mu$T of the input to that of the output:

\begin{equation} G\left(\mu\right) = \dfrac{\mathbf{M}\left[Y\right]}{\mathbf{M}\left[X\right]} \end{equation}
\end{definition}

	One of the main aspects of the Laplace Transform is that to every Transfer Function $G(s)$ there is an equivalent impulse response $g(t)$ such that the output $y(t)$ and input $x(t)$ are related by the convolution

\begin{equation} G(s) = \dfrac{Y(s)}{X(s)} \Leftrightarrow y(t) = x(t)\ast g(t) = \int_{-\infty}^{\infty} x\left(\tau\right) g\left(t - \tau\right)d\tau . \label{eq:ltf_convo}\end{equation}

	This means that if $g(t)$ is known, then the output can be easily calculated for a generic input $x(t)$ using the convolution; in this sense, the impulse signal $\delta(t)$ acts as a ``reference'' signal that characterizes the system $G(s)$ through its impulse response $g(t)$. 

	Naturally one asks whether $\mu$TFs have the same properties. We first define what is a convolution in the Dynamic Phasor space.

\begin{definition}[Convolution in DP space]\label{def:mut_convo} The Dynamic Phasor Convolution is a binary operation in complex signal space

\begin{equation}
	\left(\cdot\right)\ast\left(\cdot\right): \left\{\begin{array}{ccc}
		\left[\mathbb{R}\to\mathbb{C}\right]^2 &\to& \left[\mathbb{R}\to\mathbb{C}\right] \\[3mm] \left(X(t),Y(t)\right) &\mapsto& \displaystyle \int_{-\infty}^{\infty} X\left(\tau\right) \overline{R_0\left(\tau\right)}Y\left(t - \tau\right)\overline{R_0\left(t - \tau\right)} R_0\left(t\right)  d\tau \end{array}\right.
\end{equation}
\end{definition}

	This definition allows us many similar properties, like the $\mu$T of the convolution is the product of the transforms.

\begin{theorem}[$\mu$T of a convolution is the product of transforms] \label{theo:muT_conv_prod}%<<<
	Consider $X,Y\in\left[\mathbb{R}\to\mathbb{C}\right]$; then

\begin{equation} \mathbf{M}\left[X \ast Y\right] =  \mathbf{M}\left[X\right]\mathbf{M}\left[Y\right] .\end{equation}
\end{theorem}
\textbf{Proof:} by direct computation:

\begin{align}
	\mathbf{M}\left[X \ast Y\right] &= \int_{-\infty}^{\infty} \left[\int_{-\infty}^{\infty}X\left(\tau\right) \overline{R_0\left(\tau\right)}Y\left(t - \tau\right)\overline{R_0\left(t - \tau\right)} R_0\left(t\right) d\tau\right]\overline{R_0\left(t\right)} e^{\mu t}dt = \nonumber\\[3mm]
%
	&= \int_{-\infty}^{\infty} \left[\int_{-\infty}^{\infty}X\left(\tau\right) \overline{R_0\left(\tau\right)}Y\left(t - \tau\right)\overline{R_0\left(t - \tau\right)} \overbrace{R_0\left(t\right)\overline{R_0\left(t\right)}}^{=\left\lvert R_0\right\rvert^2 = 1}  d\tau\right] e^{\mu t}dt = \nonumber\\[3mm]
	&= \int_{-\infty}^{\infty} \left[\int_{-\infty}^{\infty}X\left(\tau\right) \overline{R_0\left(\tau\right)}Y\left(t - \tau\right)\overline{R_0\left(t - \tau\right)}  d\tau\right] e^{\mu t}dt \nonumber\\[3mm]
\end{align}

	Change the order of integration applying Fubini's Theorem:

\begin{align}
	\mathbf{M}\left[X \ast Y\right]	&= \int_{-\infty}^{\infty} \int_{-\infty}^{\infty}X\left(\tau\right) \overline{R_0\left(\tau\right)}Y\left(t - \tau\right)\overline{R_0\left(t - \tau\right)} e^{\mu t}dt d\tau = \nonumber\\[3mm]
%
	&= \int_{-\infty}^{\infty} X\left(\tau\right) \overline{R_0\left(\tau\right)} \int_{-\infty}^{\infty}Y\left(t - \tau\right)\overline{R_0\left(t - \tau\right)} e^{\mu t}dt d\tau = \nonumber\\[3mm]
%
	&= \int_{-\infty}^{\infty} X\left(\tau\right) \overline{R_0\left(\tau\right)}e^{\mu \tau}e^{-\mu \tau} \int_{-\infty}^{\infty}Y\left(t - \tau\right)\overline{R_0\left(t - \tau\right)} e^{\mu t}dt d\tau = \nonumber\\[3mm]
%
	&= \int_{-\infty}^{\infty} X\left(\tau\right) \overline{R_0\left(\tau\right)}e^{\mu \tau} \int_{-\infty}^{\infty}Y\left(t - \tau\right)\overline{R_0\left(t - \tau\right)} e^{\mu \left(t-\tau\right)}dt d\tau = \nonumber\\[3mm]
%
	(s = t - \tau) &= \int_{-\infty}^{\infty} X\left(\tau\right) \overline{R_0\left(\tau\right)}e^{\mu \tau} \int_{-\infty}^{\infty}Y\left(s\right)\overline{R_0\left(s\right)} e^{\mu s} ds d\tau = \nonumber\\[3mm]
	&= \left[\int_{-\infty}^{\infty} X\left(\tau\right) \overline{R_0\left(\tau\right)} e^{\mu \tau}d\tau\right]\left[ \int_{-\infty}^{\infty}Y\left(s\right)\overline{R_0\left(s\right)} e^{\mu s} ds \right] = \mathbf{M}\left[X\right]\mathbf{M}\left[Y\right]
\end{align}
\hfill$\blacksquare$\vspace{5mm}\hrule\vspace{5mm}%>>>

	Finally, another similar property of the new convolution is that the Dirac Delta distribution $\delta(t)$ is its neutral element.

\begin{theorem}[The Dirac Delta is the neutral element of DP Convolution]\label{theo:delta_neutral} %<<<
	For any $Y\in\left[\mathbb{R}\to\mathbb{C}\right]$,

\begin{equation} \delta(t) \ast Y(t) = Y(t) .\end{equation}
\end{theorem}
\textbf{Proof.} Also by direct computation:

\begin{equation}
	\delta(t) \ast Y(t) = \int_{-\infty}^{\infty} \delta\left(\tau\right) \overline{R_0\left(\tau\right)}Y\left(t - \tau\right)\overline{R_0\left(t - \tau\right)} R_0\left(t\right)  d\tau \label{eq:conv_int_1}
\end{equation}

	Using the Dirac Function property that

\begin{equation}
	\int_{-\infty}^{\infty} f(x)\delta(x)dx = f(0)
\end{equation}

	Then \eqref{eq:conv_int_1} becomes

\begin{equation}
	\delta(t) \ast Y(t) = \overline{R_0\left(0\right)}Y\left(t\right)\overline{R_0\left(t\right)} R_0\left(t\right) = Y(t)
\end{equation}
\hfill$\blacksquare$\vspace{5mm}\hrule\vspace{5mm}%>>>

	We now extract the Dynamic Phasor reconstructed by $G\left(\mu\right)$; by metonym, and taking care with notation, let such phasor be $G(t)$, that is,

\begin{equation} G(t) = \dfrac{R_0(t)}{2\pi j}\int_{B_\alpha} G\left(\mu\right) e^{\mu  t} R_0\left(t\right) d\mu .\end{equation}

	Notably,

\begin{equation} \mathbf{M}\left[\delta\right]\left(\mu\right) = \int_{-\infty}^{\infty} \delta(t) e^{-\mu t} \overline{R_0(t)}dt = e^{0 t} \overline{R_0(0)} = 1 \end{equation}

	\noindent so that $G(t)$ is the response of the system to the impulse distribution by theorem \ref{theo:muT_conv_prod}:

\begin{equation} \mathbf{M}\left[\delta(t) \ast G(t)\right] = \mathbf{M}\left[G(t)\right] = G\left(\mu\right)\end{equation}

	\noindent making yet another parallel between $\mu$TFs and Laplace TFs: it can be proven elementary by theorem \ref{theo:delta_neutral} that the output $Y(t)$ of a system and the input $X(t)$ are related by the convolution with the impulse response $G(t)$:

\begin{equation} Y(t) = X(t)\ast G(t)\label{eq:impulse_response_convo}\end{equation}

	\noindent so that, indeed, one can obtain any response $Y(t)$ by convolving the impulse response $G(t)$ and the corresponding input $X(t)$, directly related to the same property \eqref{eq:ltf_convo} of Laplace Transforms taking into account the adapted convolution for DPs in definition \ref{def:mut_convo}.

	One can also ask if there exists some similar characterization of a system in the DP domain instead of the time domain. Let $\mathbf{G}$ the $\mu$TF of a particular system. If the input $X(t)$ s not ZES, then $\mathbf{G}\left[X\right] = \mathbf{G}\left[X_\varepsilon\right] + \mathbf{G}\left[X_\eta\right]$; starting with the former,

\begin{equation} \mathbf{G}\left[X_\varepsilon\right] = \mathbf{G}\left[\int_{B_\alpha} M_X\left(\mu\right) e^{\mu  t} R_0 (t) d\mu \right]\end{equation}

	Using the same arguments as subsection \ref{subsec:mutransf_and_dpos},

\begin{equation} \mathbf{G}\left[X_\varepsilon\right] = \int_{B_\alpha} M_X\left(\mu\right) \mathbf{G}\left[e^{\mu  t} R_0 (t) \right] d\mu \end{equation}

	\noindent therefore, denote $G_\mu(t) = \mathbf{G}\left[e^{\mu  t} R_0\right]$ and

\begin{equation} \mathbf{G}\left[X_\varepsilon\right] = \int_{B_\alpha} M_X\left(\mu\right) G_\mu(t) d\mu \end{equation}

	At the same time,

\begin{equation} \mathbf{G}\left[X_\eta\right] = \mathbf{G}\left[\sum_{i=1}^{\infty} \eta_k^{\left[X\right]} R_k(t)\right] = \sum_{i=1}^{\infty} \eta_k^{\left[X\right]} \mathbf{G}\left[R_k\right] \end{equation}

	\noindent and denote $G_k(t) = \mathbf{G}\left[R_k\right]$, yielding

\begin{equation} \mathbf{G}\left[X\right] = \int_{B_\alpha} M_X\left(\mu\right) G_\mu (t) d\mu + \sum_{i=1}^{\infty} \eta_k^{\left[X\right]} G_k (t)\label{eq:linear_g_munull}\end{equation}

	Therefore, to obtain $\mathbf{G}\left[X\right]$ one can use $M_{\left[X\right]}$ and obtain $G_\mu(t)$ and $G_k(t)$ to use \eqref{eq:linear_g_munull}. In this sense, the signals $e^{\mu  t} R_0$ and $R_k,\ k\in\mathbb{N}^*$, act as reference signals that characterize the operator $\mathbf{G}$ through its responses $G_k(t)$ and $G_\mu(t)$ to the reference signals $R_k(t)$ and $e^{\mu t}R_0(t)$. Particularly, if $x(t)$ is of zero-energy start,

\begin{equation} \mathbf{G}\left[X\right]  = \int_{B_\alpha} M_X\left(\mu\right) G_\mu (t) d\mu \end{equation}

	\noindent meaning only $G_\mu(t)$ are needed.

%-------------------------------------------------
\section{BIBO stability}

%-------------------------------------------------
\subsection{For general linear systems} %<<<2

	In the realm of linear control theory, BIBO (Bounded-Input-Bounded-Output) stability, or simply input-output stability, is crucial for three main reasons. First, a BIBO-stable system is predictable even when the input signals vary; this guarantees reliability to the control system designed. Second, unstable systems have by definition ever-growing output signals, which at some point will inevitably damage components and cause malfunctions. Finally, BIBO stability guarantees that the output signal is whole, in the sense that the information conveyed by the signal will not be significantly distorted or scrambled with noise, ensuring accurate signal processing.	The classical way to ensure a control system is BIBO stable is to design its impulse response to be absolutely integrable, that is, a system with Transfer Function $G(s)$ is BIBO stable if and only if the impulse response $g(t)$ of $G(s)$ is absolutely integrable in $\mathbb{R}$.

	However, most linear control systems are not designed through their time response; instead, they are designed using transfer functions. Using the Dominated Convergence Theorem one proves that the ROC of any causal $G(s)$ is given by the right semiplane $\text{Re}(z) > a$ for some real $a$; this is equivalent to saying that a causal system is BIBO stable if and only if its Region of Convergence of $G(s)$ contains the imaginary axis. Further particularization for rational functions yields that a BIBO stable linear system is in fact one which poles of the transfer function are all on the left semiplane, that is, the denominator polynomial is Hurwitz Stable.

	By analyzing BIBO stability trough the transfer function, the designer is alleviated from the need to consider the input and output signals — a major advantage because the Laplace Transforms of most practical signals are not analytically representable. In other works, the fact BIBO stability is a characteristic of the control system means relieves the designer of the need to consider the input signal, instead ensuring that the system is reliable unwaivering to the input signal considered. The fact that one needs only to obtain the roots of the denominator means that such stability analysis is very simple and feasible with simple root finding algorithms.

%%-------------------------------------------------
\subsection{BIBO stability of the $\mu$TFs} %<<<2

	We now want to assert if the transfer functions $\mu$TFs can also be BIBO stable, and under what conditions. Theorem \ref{theo:bibo_mutfs} proves that a $\mu$TF is BIBO stable under the very same condition as Laplace Transfer Functions: through the characteristics of its poles.

\begin{theorem}[Rational $\mu$TFs are BIBO stable if propper and Hurwitz Stable] \label{theo:bibo_mutfs}
	Let $x(t)$ be a nostationary sinusoid as input to a system, $X(t)$ its Dynamic Phasors at some apparent frequency $\omega(t)$ and $\ndpo{n}$ the n-th order Dynamic Phasor Functional at $\omega$. Consider that the output of the system is given by a linear operator $\mathbf{G}$ such that the Dynamic Phasor of the output is given by $Y(t) = \mathbf{G}\left[X\right]$, and suppose $\mathbf{G}$ is a rational function of $\dpo$, that is, $\mathbf{G} = N\left(\dpo\right)/D\left(\dpo\right)$ for some $N$ and $D$ coprime polynomials. Then the system is BIBO stable if and only if $\deg\left(N\right) \leq \deg\left(D\right)$ (that is, $\mathbf{G}$ is ``proper'') and the roots of $D\left(z\right),\ z\in\mathbb{C}$ all lie in the open left semiplane, that is, have strictly negative real part.
\end{theorem}
\textbf{Proof:} let us adopt the infinity norm for complex signals:

\begin{equation} \left\lVert X \right\rVert = \sup_{t\in\mathbb{R}} \left\lvert X(t)\right\rvert .\end{equation}

	As shown in definition \ref{def:mapping_norm}, the norm of a linear operator is defined as the minimum positive value $\lambda$ such that the norm of the output is smaller than the norm of the input escalated by $\lambda$, that is,

\begin{equation} \left\lVert\mathbf{G}\right\rVert = \inf\left\{\lambda\in\mathbb{R}_+^*\cup\left\{\infty\right\}:\ \left\lVert \mathbf{G}v\right\rVert\leq \lambda\left\lVert v\right\rVert\forall v\in \left[\mathbb{R}\to\mathbb{C} \right] \right\}\end{equation}

	Notably, the norm may be infinite; it comes from the definition that the system $Y(t) = \mathbf{G}\left[X\right]$ is BIBO if and only if $\left\lVert G\right\rVert < \infty$. It befalls this proof to ensure that this is the case for the particular class of systems where $\mathbf{G} = N\left(\dpo\right)/D\left(\dpo\right)$ for two coprime polynomials $N$ and $D$ with $\deg(N) \leq \deg(D)$. By partial fractions,

\begin{equation} \mathbf{G} = P\left(\dpo\right) + \sum\limits_{\beta_i\in r\left(D\right)} \sum\limits_{j=1}^{\mu\left(\beta_i\right)} \dfrac{\alpha_{ij}}{\left(\dpo - \beta_i\mathbf{I}\right)^j}. \label{eq:partialfrac_decomp_dpo}\end{equation}

	\noindent where $\alpha_{ij}$ and $\beta_i$ are complex numbers and $P(s)$ is a polynomial. If $G(s)$ is not proper and $\deg\left(N\right) > \deg\left(D\right)$, $P$ will be nonzero, making the system unstable. Therefore, for the system to be stable, $\mathbf{G}$ needs to be proper, making $P\equiv 0$. Further, if $N$ and $D$ share roots, their quotient can be simplified until the resulting polynomials are coprime themselves and equation \eqref{eq:partialfrac_decomp_dpo} can be applied to the resulting expression. Therefore $N$ and $D$ can be supposed coprime. 
	
	First suppose all roots of $D$ are simple: in this case, $Y(t)$ can be written as a sum of first-order $Y_i(t)$ outputs:

\begin{equation} Y(t) = \sum\limits_{i=1}^{r\left(D(s)\right)} Y_i(t) = \sum\limits_{\beta_i\in r\left(D\right)} \left[\left(\dfrac{\alpha_{i}}{\dpo - \beta_i\mathbf{I}}\right) X(t)\right]. \label{eq:partialfrac_decomp_dpo}\end{equation}

	By the definition of the DPF, each $Y_i$ will be defined by a complex ODE

\begin{equation} \dot{Y}_i - \left(j\omega + \beta_i\right)Y_i = \alpha_i X(t)\end{equation}

	\noindent which general solution is

\begin{equation} Y_i(t) = J_i(t) \left[X_0 + \int_0^s J_i(s)^{-1} \alpha_i X(s)ds \right], \end{equation}

	\noindent where

\begin{equation} J_i(t) = e^{\left[\displaystyle\int_0^t \left(j\omega(s) + \beta_i\right)ds\right]} . \end{equation}

	Now consider $\beta_i = p_i + jq_i$. Then

\begin{equation} J_i(t) = e^{p_i t} e^{j\left[\displaystyle\int_0^t \left(\omega(s) + q_i\right)ds\right]} . \end{equation}

	Now let $\left\lVert X(t)\right\rVert = M < \infty$. Considering that $\left\lvert e^{jx}\right\rvert = 1$ for any real $x$, then the norm of the complex integral is one for any $\omega(t)$ and $q_i$; therefore

\begin{equation} \left\lVert J_i(t) \right\rVert \leq \left\lVert e^{p_i t} \right\rVert . \end{equation}

	Thence,

\begin{equation} \left\lVert Y_i(t) \right\lVert \leq \left\lVert e^{p_it} \right\rVert \left( M + M\left\lvert \alpha_i \right\rvert \left\lVert \int_0^s e^{-p_it}ds\right\rVert \right) \end{equation}

	Here, $p_i = 0$ causes the norm of the integral to be infinite, therefore $\left\lVert Y_i\right\rVert$ to also be infinite. Thus consider $p_i\neq 0$:

\begin{align} \left\lVert Y_i(t) \right\lVert 
	&= \left\lVert e^{p_it} \right\rVert M\left(1 + \left\lvert \alpha_i \right\rvert \dfrac{1}{\left\lvert p_i\right\rvert} \left\lVert 1 - e^{-p_it} \right\rVert \right) \nonumber\\[3mm]
	&\leq M \left(\left\lVert e^{p_it} \right\rVert + \dfrac{\left\lvert \alpha_i \right\rvert}{\left\lvert p_i\right\rvert} \left\lVert e^{p_it} -1 \right\rVert \right)
\end{align}

	If $p_i > 0$, then $e^{p_it}$ grows infinitely and $\left\lVert e^{p_it} \right\rVert = \infty$. If $p_i < 0$, then $\left\lVert e^{p_it} \right\rVert = 1$. At the same time, if $p_i > 0$, then $e^{p_it} - 1$ also explodes, thus $\left\lVert e^{-p_it} - 1 \right\rVert = \infty$. But if $p_i < 0$, then $\left\lVert e^{p_it} - 1\right\rVert = 1$. Therefore, if $p_i \geq 0$, $\left\lVert Y_i\right\rVert\to\infty$; but if $p_i < 0$, then $Y_i$ is bounded:

\begin{equation} \left\lVert Y_i(t) \right\lVert \leq M \left( 1 + \left\lvert \dfrac{\alpha_i}{p_i}\right\rvert \right) \label{eq:partialfrac_decomp_yim}\end{equation}

	\noindent and it is clear that $\left\lVert Y_i\right\rVert$ is limited if and only if $\beta_i$ has a strictly negative real part. By the definition \eqref{eq:partialfrac_decomp_dpo} of $\left\lVert \mathbf{G}\right\rVert$, \eqref{eq:partialfrac_decomp_yim} implies

\begin{equation} \left\lVert\mathbf{G}\right\rVert \leq \sum\limits_{\beta_i\in r\left(D\right)}\left( 1 + \left\lvert \dfrac{\alpha_i}{p_i}\right\rvert \right)\end{equation}

	However, if one of the $\beta_i$ has a positive real part, then the corresponding $Y_i(t)$ explodes — therefore $Y(t)$ also explodes, therefore $\left\lVert \mathbf{G}\right\rVert = \infty$. Thus all $\beta_i$ need to be on the left semiplane for the system to be BIBO stable.

	We can generalize this method to a case where $\beta_i$ has multiplicity $\mu$  greater than 1. Then $Y_i$ will be composed of a sum of terms of $Y_{i1},Y_{i2},\cdots,Y_{i\mu}$ of the form

\begin{equation} Y_{im}(t) = \left[\left(\dfrac{\alpha_{im}}{\left(\dpo - \beta_i\mathbf{I}\right)^m}\right) X(t)\right],\ 1\leq m \leq \mu\left(\beta_i\right). \label{eq:partialfrac_decomp_dpo_mth}\end{equation}
	
	Pick a particular index $m$. Create the intermediary signals $Z_j^i$ defined by subsequent operationals

\begin{equation} \left\{\begin{array}{l}
	Z^i_1(t) = \left[\left(\dfrac{\alpha_{i}}{\left(\dpo - \beta_i\mathbf{I}\right)^1}\right) X(t)\right] \\[5mm] 
	Z^i_{(k)}(t) = \left[\left(\dfrac{\mathbf{I}}{\left(\dpo - \beta_i\mathbf{I}\right)^1}\right) Z^i_{(k-1)}(t)\right],\ 2\leq k \leq m
\end{array}\right.
\end{equation}

	\noindent which define the ODEs

\begin{equation} \left\{\begin{array}{l}
	\dot{Z}_1^i - \left(j\omega + \beta_i\right)Z_1^i = \alpha_i X(t) \\[5mm]
	\dot{Z}_{(k)}^i - \left(j\omega + \beta_i\right)Z_{(k-1)}^i = Z_1^i(t),\ 2\leq k \leq m
\end{array}\right.
\end{equation}

	\noindent and the single-roots case yields that $Z_j^i$ is BIBO stable with respect to $Z_{(j-1)}^i$ and $Z_1^i$ is BIBO stable with respect to $X(t)$, such that

\begin{equation} \left\{\begin{array}{l}
	\left\lVert Z^i_1 \right\rVert \leq M \left( 1 + \left\lvert \dfrac{\alpha_i}{p_i}\right\rvert \right) \\[5mm] 
	\left\lVert Z^i_{(k)} \right\rVert \leq \left\lVert Z^i_{(k-1)} \right\rVert \left( 1 + \left\lvert \dfrac{1}{p_i}\right\rvert \right)\text{ for } 2 \leq k \leq m
\end{array}\right.
\end{equation}

	Therefore if $X(t)$ is bounded, so is $Z_1^i$, therefore so is $Z_2^i$ and so on, and all $Z_k^i$ are bounded. Therefore all $Z_k^i$ are BIBO stable with respect to $X(t)$ and, by induction,

\begin{equation} \left\lVert Z^i_k \right\rVert \leq M \left( 1 + \left\lvert \dfrac{\alpha_i}{p_i}\right\rvert \right)\left( 1 + \left\lvert \dfrac{1}{p_i}\right\rvert \right)^{(k-1)} \end{equation}

	for $1\leq k \leq m$. Therefore
	
\begin{equation} \left\lVert \mathbf{G} \right\rVert \leq \sum\limits_{\beta_i\in r\left(D\right)} \sum\limits_{k=1}^{\mu\left(\beta_i\right)} \left( 1 + \left\lvert \dfrac{\alpha_{ik}}{p_i}\right\rvert \right)\left( 1 + \left\lvert \dfrac{1}{p_i}\right\rvert \right)^{(k-1)} .\end{equation}
\hfill$\blacksquare$

\begin{corollary} A rational operator $\mathbf{G} = N\left(\dpo\right)/D\left(\dpo\right)$ is represented by the $\mu$TF $G\left(\mu\right) = N\left(\mu\right)/D\left(\mu\right)$ and is bounded if and only it is proper and the roots of $D$ are all in the open half left plane. \end{corollary}

%-------------------------------------------------
\section{Discussion and application: a new current controller proposed using DPOs}\label{subsec:new_controller} %<<<1

	Seen as the $\mu$ Transform is eminently algebraic, one can easily envision that a circuit network theory in the $\mu$ space is very simple to prove. Immediately, Kirchoff's Laws in the $\mu$ domain can be proven simply by the transform's linearity. Further, one can define an impedance in the $\mu$ domain as the ratio between the $\mu$T of the voltage and the $\mu$T of the current

\begin{equation} Z\left(\mu\right) = \dfrac{V\left(\mu\right)}{I\left(\mu\right)} \end{equation}

	\noindent so that the simple components would yield 

\begin{equation}\left\{\begin{array}{l} Z_L\left(\mu\right) = \mu L \text{ (Linear inductor)}\\[3mm] Z_C\left(\mu\right) = \dfrac{1}{\mu C} \text{ (Linear capacitor)}\\[5mm] Z_R = R \text{ (Linear resistor)}\end{array} \right. .\label{sys:muT_impedances_formula}\end{equation}

	One can also see that the Superposition Principle, Thèvenin and the Norton Theorems can be proven using proofs very similar to theorems \ref{theo:superposition}, \ref{theo:thevenin} and \ref{theo:norton}. One can also simply prove matrix relationships of impedances, admittances, vectors of Dynamic Phasors, and the entirety of network analysis is available in $\mu$ domain.

	These facts allow modelling control systems in generalized sinusoidal regimens with relative ease and simplicity due to its close relationship with the Laplace Transfer Functions.

	For instance, we revisit example \ref{example:3p_eps_modelling}. One of the issues raised in the example was that the current filter model of figure \ref{fig:ibr_modelling_example} supposed a quasi-static phasor relationship \eqref{eq:example_quasistatic_supposition}. We now rewrite that equation in the $\mu$ domain: starting from the time domain equation,

\begin{equation} e\left(t\right) - v_\infty\left(t\right) = \left(R + R_F\right)i(t) + \left(L + L_F\right) \dfrac{di(t)}{dt} \label{eq:dpo_filter_time} \end{equation}

	\noindent and applying the $\mu$ Transform,

\begin{equation} E\left(\mu\right) - V_\infty\left(\mu\right) = \left[R + R_F + \mu\left(L + L_F\right)\right] I\left(\mu\right) \label{eq:dpo_filter} \end{equation}

	\noindent where $X\left(\mu\right)$ is a shorthand for $\mathbf{M}\left[X\right]$, that is, the $\mu$ Transform of the Dynamic Phasor $X(t)$. Although strikingly simple, this equation is much better suited to represent systems based on Dynamic Phasors because \eqref{eq:dpo_filter} is equivalent to

\begin{equation} E\left(t\right) - V_\infty\left(t\right) = \left[R\mathbf{I} + R_F + \dpo\left(L + L_F\right)\right]\left[I\left(t\right)\right]  = \left(R + R_F\right)I + \left(L + L_F\right)\dpo\left[I(t)\right] \label{eq:dpo_filter_dps} \end{equation}

	\noindent and the Dynamic Phasors that solve \eqref{eq:dpo_filter_dps} are proven to be biunivocally representatives of the time signal solutions of the original time differential equation \eqref{eq:dpo_filter_time}.

% NEW PROPOSED PI CONTROLLER <<<
\begin{figure}[h]
\centering
\scalebox{0.8}{
\begin{tikzpicture}[american,scale=1,transform shape,line width=0.75, cute inductors, voltage shift = 1,>={Stealth[inset=0mm,length=1.5mm,angle'=50]}]
\node (origin) at (0,0) {};

\node at (55mm,15mm) [draw, rounded corners, stewartblue, fill=stewartblue, fill opacity=0.2, very thick, dashed, line cap = round, minimum width=90mm, minimum height=50mm] (pirounded) {};
\node [above=1mm of pirounded, stewartblue] {\large $\mu$T PI Controller};

%\node [draw, very thick, isosceles triangle, minimum height=15mm, minimum width=15mm] at (0mm, 15mm) (omegaLGainD) {$\omega L_F$};
\node [draw,very thick, shape=circle,name=inputsum, minimum width=10mm] at (0,0) {};
\draw [<-] ([shift=({-1mm,0})]inputsum.west) -- ++(-10mm,0) node[left] {$I^*$};
\draw [<-] ([shift=({0,-1mm})]inputsum.south) -- ++(0,-10mm) node[below] {$I$};
\node [name=kpgain, draw, shape=isosceles triangle,very thick, minimum height=15mm, minimum width=15mm,right=40mm of inputsum.east] {$k_P$};
\draw[->] (inputsum.east) -- ([shift=({-1mm,0})]kpgain.west);
\node [above=20mm of kpgain] (above) {};
\node [name=kigain, draw, shape=isosceles triangle,very thick, minimum height=15mm, minimum width=15mm,left=0mm of above] {$k_I$};
\node [draw, minimum width=10mm, very thick, minimum height=15mm, right=10mm of above] (integrator) {$\dfrac{1}{\mu - \mu_0}$};

\node[left=10mm of kigain.west] (temp1) {};
\draw[->] (temp1 |- inputsum) |- ([shift=({-1mm,0})]kigain.west);
\draw[->] (kigain.east) |- ([shift=({-1mm,0})]integrator.west);

\node [draw,very thick, shape=circle,name=pisum, minimum width=10mm,right=20mm of kpgain.east] {};

\draw[->] (integrator.east) -| ([shift=({0mm,1mm})]pisum.north);
\draw[->] (kpgain.east) -- ([shift=({-1mm,0})]pisum.west);

\draw [->] (pisum.east) -- ++(10mm,0) node[right] {$V$};
\end{tikzpicture}
}
\caption
{Proposed $\mu$TF-based PI controller for the current control subsystem for the inverter system of figure \ref{fig:ibr_modelling_example}.}
\label{fig:pi_utf_blockmodel}
\end{figure}
%>>>

%-------------------------------------------------
\subsection{Proposition and construction}\label{subsec:new_controller_prop_const} %<<<2

	Looking at the current controller of figure \eqref{fig:3p_curr_control} for example \ref{example:3p_eps_modelling}, one can clearly see that any  attempt at tuning the integral and proportional gains of the $D$ and $Q$ loops will be considerable worksome; stability analysis is probably only possible by simulations. Using the DPFT theory proposed, however, we can propose a better controller that is intuitive and guaranteedly BIBO stable.

	One can rewrite the equations of the power system of figure \ref{fig:ibr_modelling_example} and the controller of figure \ref{fig:3p_curr_control} to a more intuitive and theory-solid version. By denoting a time integration as $\mu^{-1}$, we propose an equivalent PI controller in the $\mu$ domain:

\begin{equation} V\left(\mu\right) = \left[k_P + k_I \left(\dfrac{1}{\mu - \mu_0}\right)\right]\left(I^*\left(\mu\right) - I\left(\mu\right)\right) \label{eq:dpft_current_control}\end{equation}

	\noindent and this generates the ``PI $\mu$TF controller'' in figure \ref{fig:pi_utf_blockmodel}. Notably, the PI controller proposed is a ratio of $\mu$ but shifted by a $\mu_0$ quantity, where one would expect the integral controller to just be defined as $\mu^{-1}$ like in the Laplace domain. This quantity exists because, due to the Final Value Theorem for $\mu$Ts (theorem \ref{theo:muT_fvt}), the final value happens as $\mu\to\mu_0$ where $\mu_0$ is the origin value of some continuous transformation $h$ which depends on the ratio of $V(\mu)$ and $I^*(\mu) - I(\mu)$. Indeed, manipulating \eqref{eq:dpft_current_control} one yields

\begin{equation} \left(\mu - \mu_0\right)V\left(\mu\right) = \left[k_P\left(\mu - \mu_0\right) + k_I \right]\left(I^*\left(\mu\right) - I\left(\mu\right)\right) \Leftrightarrow \lim_{\mu\to\mu_0} \left[I^*\left(\mu\right) - I\left(\mu\right)\right] = 0\end{equation}

	\noindent but according to the Final Value Theorem for $\mu$Ts (theorem \ref{theo:muT_fvt}), this implies

\begin{equation} \lim_{t\to\infty} \left[I^*\left(t\right) - I\left(t\right)\right] = 0\end{equation}

	\noindent showing that the PI controller proposed vanishes the steady-state error, as intended; this would not happen if the proposed integral controller were defined as $\mu^{-1}$. Notably, however, if the apparent frequency $\omega(t)$ is identically null then by remark T\ref{remark:theo_muT_fvt_null_freq} $\mu_0 = 0$ and the integral block becomes the ``Laplace integral controller'' $\mu^{-1}$.

	Now note that

\begin{equation} E(\mu) - V(\mu) = \left(R_F + \mu I_F\right)I(\mu)\end{equation}

	\noindent and incorporating this equation into the PI controller of \ref{fig:pi_utf_blockmodel} generates the current controller model for the IBR as in figure \ref{fig:partial_blockmodel}.

% NEW CURRENT CONTROL SYSTEM <<<
\begin{figure}[t]
\centering
\scalebox{0.8}{
\begin{tikzpicture}[american,scale=1,transform shape,line width=0.75, cute inductors, voltage shift = 1,>={Stealth[inset=0mm,length=1.5mm,angle'=50]}]
\node (origin) at (0,0) {};

%\node [draw, very thick, isosceles triangle, minimum height=15mm, minimum width=15mm] at (0mm, 15mm) (omegaLGainD) {$\omega L_F$};
\node [draw,very thick, shape=circle,name=inputsum, minimum width=10mm] at (0,0) {};
\draw [<-] ([shift=({-1mm,0})]inputsum.west) -- ++(-10mm,0) node[left] {$I^*$};
\draw [<-] ([shift=({0,-1mm})]inputsum.south) -- ++(0,-40mm) node[below] {$I$};
\node [name=kpgain, draw, shape=isosceles triangle,very thick, minimum height=15mm, minimum width=15mm,right=40mm of inputsum.east] {$k_P$};
\draw[->] (inputsum.east) -- ([shift=({-1mm,0})]kpgain.west);
\node [above=20mm of kpgain] (above) {};
\node [name=kigain, draw, shape=isosceles triangle,very thick, minimum height=15mm, minimum width=15mm,left=0mm of above] {$k_I$};
\node [draw, minimum width=10mm, very thick, minimum height=15mm, right=15mm of above] (integrator) {$\dfrac{1}{\mu - \mu_0}$};

\node[left=10mm of kigain.west] (temp1) {};
\draw[->] (temp1 |- inputsum) |- ([shift=({-1mm,0})]kigain.west);
\draw[->] (kigain.east) |- ([shift=({-1mm,0})]integrator.west);

\node [draw,very thick, shape=circle,name=pisum, minimum width=10mm,right=20mm of kpgain.east] {};

\draw[->] (integrator.east) -| ([shift=({0mm,1mm})]pisum.north);
\draw[->] (kpgain.east) -- ([shift=({-1mm,0})]pisum.west);

\node [draw,very thick, shape=circle,name=vsum, minimum width=10mm,right=20mm of pisum.east] {};


\draw [->] (pisum.east) -- ([shift=({-1mm,0})]vsum.west) node[midway,above] {$V$};
\draw [->] (vsum.east) -- ++(20mm,0) node[right] {$E$};

\node [name=filtergain, draw, very thick, minimum height=10mm, minimum width=15mm,below=20mm of kpgain] {$R_F + \mu L_F$};
\draw[->] (filtergain -| inputsum) -- ([shift=({-1mm,0})]filtergain.west);
\draw[->] (filtergain.east) -| ([shift=({0,-1mm})]vsum.south);
\node[below=1mm of filtergain.south] {Current filter};

\end{tikzpicture}
}
\caption
[Improved $\mu$TF-based current control subsystem for the inverter system of figure \ref{fig:ibr_modelling_example}.]
{Improved $\mu$TF-based current control subsystem for the inverter system of figure \ref{fig:ibr_modelling_example} considering current filter dynamics.}
\label{fig:partial_blockmodel}
\end{figure}
%>>>

	Finally, together with the grid equation \eqref{eq:dpo_filter}, the system can be denoted as the block model depicted in figure \ref{fig:complete_blockmodel} — a better alternative to the traditional representation using \eqref{eq:pi_adjust_e} because, by using \eqref{eq:dpo_filter} instead, the controller is designed taking into accout the frequency swings whereas the traditional control did not. The ``DPFT-based'' PI controller is shown as per \eqref{eq:dpft_current_control} with the grid equation \eqref{eq:dpo_filter} yielding the current $I$, which is then fed into the current controller, closing the loop.

%-------------------------------------------------
\subsection{Analyzing BIBO stability}\label{subsec:analyzing_bibo} %<<<2

	To obtain the equation for $I$, substituting \eqref{eq:dpft_current_control} into \eqref{eq:dpo_filter} yields

\begin{gather}
	\left(R_F + L_F\mu\right)I(\mu) + \left[k_P + k_I\left(\dfrac{1}{\mu - \mu_0}\right)\right]\left[I^*(\mu) - I(\mu)\right] - V_\infty(\mu) = \left(R + R_F + \mu\left(L + L_F\right)\right)I(\mu) \nonumber\\[3mm]
	\left[k_P + k_I\left(\dfrac{1}{\mu - \mu_0}\right)\right]I^*(\mu) + \left[- k_P - k_I\left(\dfrac{1}{\mu - \mu_0}\right) - R - L\mu \right]I(\mu) = V_\infty(\mu) \nonumber\\[3mm]
	\left[k_P\left(\mu - \mu_0\right) + k_I\right]I^*(\mu) - \left[k_P\left(\mu - \mu_0\right) + k_I  + R\left(\mu - \mu_0\right) + L\mu\left(\mu - \mu_0\right)\right]I(\mu) = \left(\mu - \mu_0\right) V_\infty(\mu)   \nonumber\\[3mm]
	I(\mu) = \dfrac{\left[k_P\left(\mu - \mu_0\right) + k_I\right]I^*(\mu) - \left(\mu - \mu_0\right) V_\infty(\mu)}{k_P\left(\mu - \mu_0\right) + k_I  + R\left(\mu - \mu_0\right) + L\mu\left(\mu - \mu_0\right)} \nonumber\\[3mm]
	I(\mu) = \dfrac{\left[k_P\left(\mu - \mu_0\right) + k_I\right]I^*(\mu) - \left(\mu - \mu_0\right) V_\infty(\mu)}{ k_I - j \left(k_P + R\right) \omega_0 + \mu\left(k_P + R - \mu_0 L\right) + L\mu^2 } .\label{eq:dpo_i_mimo}
\end{gather}

	Notably this $\mu$FT has two inputs thus it is comprised of two transfer functions:

\begin{equation} I(\mu) = G_I(\mu) I^*(\mu) + G_V(\mu) V_\infty(\mu) \end{equation}

	\noindent but both the transfer functions share poles, meaning their BIBO stability is equivalent. Calculating these poles, however, depends on knowing the parameter $\mu_0$ which is probably difficult to obtain. To ameliorate the calculations, we assume that the system is working at an apparent frequency that is equivalent to the synchronous frequency $\omega_0$ and using corollary \ref{corollary:fvt_const_freq} we obtain $\mu_0 = j\omega_0$. With this assumption, the poles are given by

\begin{align}
	\mu_{(\pm)} &= \dfrac{-\left(k_P + R - j\omega_0 L\right) \pm \sqrt{\raisebox{3.5mm}{} \left(k_P + R - j\omega_0 L\right)^2 - 4L\left[\raisebox{3mm}{} k_I - j \left(k_P + R\right) \omega_0 \right]}}{2L} = \nonumber\\[3mm]
%
	            &= \dfrac{-\left(k_P + R - j\omega_0 L\right) \pm \sqrt{\raisebox{3.5mm}{} \left(k_P + R + j\omega_0 L\right)^2 - 4Lk_I}}{2L} = \nonumber\\[3mm]
%
	            &= \dfrac{-\left(k_P + R - j\omega_0 L\right) \pm \sqrt{\raisebox{3.5mm}{} \left(k_P + R\right)^2 - 4Lk_I - \omega_0^2L^2 + j2\omega_0 L \left(k_P + R\right)}}{2L} \label{eq:discriminant}
\end{align}

	\noindent allowing us to analyze what combination of $k_P$ and $k_I$ yields a BIBO stable system, which entails to choosing the gains so that Re$\left(\mu_{(\pm)}\right) < 0$. While $k_P$ and $k_I$ can be complex, for a simplified analysis let us assume real positive gains. We know that the real part maintains complex sum and is linear with respect to real scalars, so that we can obtain the poles by using the complex square root formula

\begin{equation} \sqrt{z = a + jb} = \pm\left(\sqrt{\dfrac{\left\lvert z\right\rvert + a}{2}} + j\dfrac{b}{\left\lvert b\right\rvert}\sqrt{\dfrac{\left\lvert z\right\rvert - a}{2}}\right) \end{equation}

	\noindent and considering $2\omega_0 L \left(k_P + R\right) \geq 0$ because all parameters are positive this yields

\footnotesize
\begin{equation}
	\mu_{(\pm)} = \dfrac{1}{2L} \left[
	\begin{array}{c}
		-\left(k_P + R - j\omega_0 L\right) \pm \sqrt{\dfrac{ \sqrt{\left[\left(k_P + R\right)^2 - 4Lk_I - \omega_0^2L^2\right]^2 + 4\omega_0^2 L^2 \left(k_P + R\right)^2} + \left(k_P + R\right)^2 - 4Lk_I - \omega_0^2L^2}{2}} + \\[5mm]
%
		\pm j\sqrt{\dfrac{ \sqrt{\left[\left(k_P + R\right)^2 - 4Lk_I - \omega_0^2L^2\right]^2 + 4\omega_0^2 L^2 \left(k_P + R\right)^2} - \left(k_P + R\right)^2 + 4Lk_I + \omega_0^2L^2}{2}}
	\end{array}
	\right]
\end{equation}
\normalsize

% COMPLETE SYSTEM <<<
\begin{figure}[t]
\centering
\scalebox{0.8}{
\begin{tikzpicture}[american,scale=1,transform shape,line width=0.75, cute inductors, voltage shift = 1,>={Stealth[inset=0mm,length=1.5mm,angle'=50]}]
\node (origin) at (0,0) {};

%\node [draw, very thick, isosceles triangle, minimum height=15mm, minimum width=15mm] at (0mm, 15mm) (omegaLGainD) {$\omega L_F$};
\node [draw,very thick, shape=circle,name=inputsum, minimum width=10mm] at (0,0) {};
\draw [<-] ([shift=({-1mm,0})]inputsum.west) -- ++(-10mm,0) node[left] {$I^*$};
\draw [<-] ([shift=({0,-1mm})]inputsum.south) -- ++(0,-40mm) node[left] (buscurr) {$I$};
\node [name=kpgain, draw, shape=isosceles triangle,very thick, minimum height=15mm, minimum width=15mm,right=40mm of inputsum.east] {$k_P$};
\draw[->] (inputsum.east) -- ([shift=({-1mm,0})]kpgain.west);
\node [above=20mm of kpgain] (above) {};
\node [name=kigain, draw, shape=isosceles triangle,very thick, minimum height=15mm, minimum width=15mm,left=0mm of above] {$k_I$};
\node [draw, minimum width=10mm, very thick, minimum height=15mm, right=15mm of above] (integrator) {$\dfrac{1}{\mu - \mu_0}$};

\node[left=10mm of kigain.west] (temp1) {};
\draw[->] (temp1 |- inputsum) |- ([shift=({-1mm,0})]kigain.west);
\draw[->] (kigain.east) |- ([shift=({-1mm,0})]integrator.west);

\node [draw,very thick, shape=circle,name=pisum, minimum width=10mm,right=20mm of kpgain.east] {};

\draw[->] (integrator.east) -| ([shift=({0mm,1mm})]pisum.north);
\draw[->] (kpgain.east) -- ([shift=({-1mm,0})]pisum.west);

\node [draw,very thick, shape=circle,name=vsum, minimum width=10mm,right=20mm of pisum.east] {};

\draw [->] (pisum.east) -- ([shift=({-1mm,0})]vsum.west) node[midway,above] {$V$};

\node [name=filtergain, draw, very thick, minimum height=10mm, minimum width=15mm,below=20mm of kpgain] {$R_F + \mu L_F$};
\draw[->] (filtergain -| inputsum) -- ([shift=({-1mm,0})]filtergain.west);
\draw[->] (filtergain.east) -| ([shift=({0,-1mm})]vsum.south);
\node[below=1mm of filtergain.south] {Current filter};

\node [draw,very thick, shape=circle,name=vinfsum, minimum width=10mm, right=15mm of vsum.east] {};

\draw [<-] ([shift=({0mm,1mm})]vinfsum.north) -- ++(0,10mm) node[above] {$V_\infty$};

\draw [->] (vsum.east) -- ([shift=({-1mm,0mm})]vinfsum.west) node[midway,above] {$E$};

\node [draw, minimum width=10mm, very thick, minimum height=15mm, below=50mm of pisum.south] (gridblock) {$\dfrac{1}{R + R_F + \mu\left(L + L_F\right)}$};
\node[below=1mm of gridblock.south] {Grid equation};

\draw [->] (vinfsum.south) |- ([shift=({1mm,0mm})]gridblock.east);
\draw [->] (gridblock.west) -| ([shift=({0mm,-1mm})]inputsum.south);

\end{tikzpicture}
}
\caption
[Closed-loop model of the system of figure \ref{fig:ibr_modelling_example} in the $\mu$ domain.]
{Closed-loop model of the system of figure \ref{fig:ibr_modelling_example} in the $\mu$ domain using the improved current control of figure \ref{fig:partial_blockmodel} and incorporating transmission grid and current filter dynamics.}
\label{fig:complete_blockmodel}
\end{figure}
%>>>

	Here we notice that $\Re\left(\mu_{(-)}\right)$ is always negative or zero, thus the stability of the $\mu$TF is left to $\mu_{(+)}$:

\footnotesize
\begin{equation}
	\hspace{-4mm} \Re\left(\mu_{(+)}\right) = \dfrac{1}{2L} \left[ -\left(k_P + R\right) + \sqrt{\dfrac{ \sqrt{\left[\left(k_P + R\right)^2 - 4Lk_I - \omega_0^2L^2\right]^2 + 4\omega_0^2 L^2 \left(k_P + R\right)^2} + \left(k_P + R\right)^2 - 4Lk_I - \omega_0^2L^2}{2}}\right]
\end{equation}
\normalsize

	Therefore we want to find combinations $k_P,k_I$ so that this quantity is negative. We first calculate the geometric space of $k_P,k_I$ where this quantity is zero:

\begin{gather}
	\sqrt{2}\left(k_P + R\right) = \sqrt{\sqrt{\left[\left(k_P + R\right)^2 - 4Lk_I - \omega_0^2L^2\right]^2 + 4\omega_0^2 L^2 \left(k_P + R\right)^2} + \left(k_P + R\right)^2 - 4Lk_I - \omega_0^2L^2} \nonumber\\[5mm]
%
	2\left(k_P + R\right)^2 = \sqrt{\left[\left(k_P + R\right)^2 - 4Lk_I - \omega_0^2L^2\right]^2 + 4\omega_0^2 L^2 \left(k_P + R\right)^2} + \left(k_P + R\right)^2 - 4Lk_I - \omega_0^2L^2 \nonumber\\[5mm]
%
	\left(k_P + R\right)^2 + 4Lk_I + \omega_0^2L^2 = \sqrt{\left[\left(k_P + R\right)^2 - 4Lk_I - \omega_0^2L^2\right]^2 + 4\omega_0^2 L^2 \left(k_P + R\right)^2} \nonumber\\[5mm]
%
	\left[\left(k_P + R\right)^2 + 4Lk_I + \omega_0^2L^2\right] = \left[\left(k_P + R\right)^2 - 4Lk_I - \omega_0^2L^2\right]^2 + 4\omega_0^2 L^2 \left(k_P + R\right)^2 \nonumber\\[5mm]
%
	\overbrace{\left[\left(k_P + R\right)^2 + 4Lk_I + \omega_0^2L^2\right] - \left[\left(k_P + R\right)^2 - 4Lk_I - \omega_0^2L^2\right]^2}^{a^2 - b^2 = (a-b)(a+b)} = 4\omega_0^2 L^2 \left(k_P + R\right)^2 \nonumber\\[5mm]
%
	%\left\{\left[\left(k_P + R\right)^2 + 4Lk_I + \omega_0^2L^2 - \left(k_P + R\right)^2 + 4Lk_I + \omega_0^2L^2\right]\left[\left(k_P + R\right)^2 + 4Lk_I + \omega_0^2L^2 + \left(k_P + R\right)^2 - 4Lk_I - \omega_0^2L^2\right]\right\} = 4\omega_0^2 L^2 \left(k_P + R\right)^2 \nonumber\\[5mm]
%
	\left\{\left[8Lk_I + 2\omega_0^2L^2\right]\left[2\left(k_P + R\right)^2\right]\right\} = 4\omega_0^2 L^2 \left(k_P + R\right)^2 \nonumber\\[5mm]
%
	4Lk_I + \omega_0^2L^2 = \omega_0^2 L^2 \nonumber\\[5mm]
%
	k_I = 0
\end{gather}

	\noindent and one immediately concludes that $\Re\left(\mu_{(+)}\right)$ has the inverse signal as $k_I$, that is, it is negative if $k_I$ is positive, positive if $k_I$ is negative, and zero if $k_I$ is negative. Thus, we only have to choose $k_I > 0$ and the system will be stable. Choosing $k_I$ and $k_P$ then comes down to a choice of dynamic performance.

%-------------------------------------------------
\subsection{Simulation}\label{subsec:newcontroller_sim} %<<<2

	To simulate the system, one can translate \eqref{eq:dpo_i_mimo} into the time domain, yielding

\begin{equation} L\ndpo{2}\left[I(t)\right] + \left(k_P + R\right)\dpo\left[I(t)\right] + k_I I(t) = k_P\dpo\left[I^*(t)\right] + k_I I^*(t) - \dpo\left[V_\infty(t)\right], \end{equation}

	\noindent then knowing the expressions of $I^*(t)$ and $V_\infty(t)$ — since they are inputs — use the definitions \eqref{eq:steinmetz_1storder} of $\dpo$ and \eqref{eq:steinmetz_2ndorder} of $\ndpo{2}$ to obtain a differential equation in the complex domain, separating into real and imaginary components would yield a differential system on $I_d$ and $I_q$ that could be solved.

	However, the entire point and objective (at least in applied sciences) of integral transforms is that the time signals can be reconstructed from their transforms, and solving the time-domain differential equation rather defeats this purpose. Due to the properties of the $\mu$T, we can also obtain the expression of $I$ through inverse $\mu$ Transform using the theory developed in this chapter.

	We remember that we supposed that the apparent frequency used for Dynamic Phasor transformations is equivalent to the constant synchronous frequency $\omega_0$. In this frequency, $V_\infty(t)$ is a constant. We also consider that the Dynamic Phasor of the input reference for current $I^*(t)$ is also constant at the constant synchronous frequency.

	Thus we simulate the system response to steps in the current reference (simulating a control decision change) and a then a step in the infinite bus voltage (simulating a transient disturbance on the larger grid); that is,

\begin{equation} I^*\left(t\right) = I_0^* + u(t)\Delta I \text{ and } V_\infty\left(t\right) =  V_0 + u(t)\Delta V \end{equation}

	\noindent where $u(t)$ is the step distribution, $I^*_0$ and $V_0$ are the initial values and $\Delta I,\ \Delta V$ are the disturbance amplitudes. We now calculate $I^*(\mu)$ and $V_\infty(\mu)$: first, we take their zero-energy counterparts $I^*(t) - I_0$ and $V_\infty(t) - V_0$, so we do not have to take initial conditions into account, and these signals are perfect steps; by definition \eqref{eq:mtransf_mu_def}, the $\mu$T of a constant $k\in\mathbb{C}$ at the apparent frequency $\omega_0$ is

\begin{equation} \mathbf{M}\left[ku(t)\right] = \mathbf{L}\left[ku(t)e^{j\omega_0t}\right] = \dfrac{k}{\mu - j\omega_0}\end{equation}

	\noindent and substituting onto the expression of $I(\mu)$,

\begin{equation}
	I(\mu) = \dfrac{\left[k_P\left(\mu - j\omega_0\right) + k_I\right]\Delta I - \left(\mu - j\omega_0\right) \Delta V}{\left(\mu - j\omega_0\right)\left[\raisebox{4mm}{} k_I - j \left(k_P + R\right) \omega_0 + \mu\left(k_P + R - j\omega_0 L\right) + L\mu^2\right] } .\label{eq:dpo_i_mimo_const}
\end{equation}

	\noindent where, by definition, $I(\mu)$ is the $\mu$T of the ZES equivalent $I(t) - I_0$. Naturally we assume that the initial value $I_0$ of $I(t)$ is the initial value of the reference $I^*$ due to the PI controller, supposing that the system was at equilibrium before the disturbances. Notably, if we assume that $k_P$ and $k_I$ are chosen such that $\Re\left(\mu_{(\pm)}\right) < 0$, then using the Final Value Theorem \ref{theo:muT_fvt},

\begin{align}
	\lim\limits_{t\to\infty} I(t) &= I_0 + \lim\limits_{\mu\to j\omega_0} \left(\mu - j\omega_0\right) I\left(\mu\right) = \nonumber\\[3mm]
%
	&= I_0 + \lim\limits_{\mu\to j\omega_0} \dfrac{\left[k_P\left(\mu - j\omega_0\right) + k_I\right]\Delta I - \left(\mu - j\omega_0\right) \Delta V}{k_P\left(\mu - j\omega_0\right) + k_I  + R\left(\mu - j\omega_0\right) + L\mu\left(\mu - j\omega_0\right)} = \nonumber\\[3mm]
%
	&= I_0 + \lim\limits_{\mu\to j\omega_0} \dfrac{\left[k_P\cancelto{0}{\left(\mu - j\omega_0\right)} + k_I\right]\Delta I - \cancelto{0}{\left(\mu - j\omega_0\right)} \Delta V}{k_P\cancelto{0}{\left(\mu - j\omega_0\right)} + k_I + \cancelto{0}{R\left(\mu - j\omega_0\right)} + \cancelto{0}{L\mu\left(\mu - j\omega_0\right)}} = I_0 + \dfrac{k_I \Delta I}{k_I} = \nonumber\\[3mm]
%
	&= I_0 + \Delta I = I^* \label{eq:dpo_i_mimo_const}
\end{align}

	\noindent once again showing that the PI controller proposed in \eqref{eq:dpft_current_control} indeed vanishes the steady-state error. Furthermore, we can obtain the Dynamic Phasor in time domain associated with this function using \eqref{eq:inv_muT_def}, that is, by using the inverse Laplace Transform. We first separate \eqref{eq:dpo_i_mimo_const} through partial fractions, denoting the poles of the quadratic portion of the denominator as $\mu_{(\pm)}$:

\begin{equation} I(\mu) = \dfrac{\left[k_P\left(\mu - j\omega_0\right) + k_I\right]I^* - \left(\mu - j\omega_0\right) V_\infty}{\left(\mu - j\omega_0\right)L\left(\mu - \mu_{(+)}\right)\left(\mu - \mu_{(-)}\right)} = \dfrac{A}{\mu - j\omega_0} + \dfrac{B}{\mu - \mu_{(+)}} + \dfrac{C}{\mu - \mu_{(-)}} .\label{eq:dpo_i_mimo_const_part_fracs} \end{equation}

	\noindent and we use the Heaviside cover-up method \pcite{zillDifferentialEquationsBoundaryvalue2013} to quickly obtain

\begin{equation}
	\left\{\begin{array}{l}
		A = \Delta I\\[5mm]
		B = \dfrac{\left[k_P\left(\mu_{(+)} - j\omega_0\right) + k_I\right]\Delta I - \left(\mu_{(+)} - j\omega_0\right) \Delta V}{\left(\mu_{(+)} - j\omega_0\right)L\left(\mu_{(+)} - \mu_{(-)}\right)} \\[10mm]
		C = \dfrac{\left[k_P\left(\mu_{(-)} - j\omega_0\right) + k_I\right]\Delta I - \left(\mu_{(-)} - j\omega_0\right) \Delta V}{\left(\mu_{(-)} - j\omega_0\right)L\left(\mu_{(-)} - \mu_{(+)}\right)} 
	\end{array}\right.
\end{equation}

	Now since $\mathbf{L}\left[e^{-zt}\right] = \left(s - z\right)^{-1}$ for any complex $z$ and $\Re(s) > \Re(z)$ we use \eqref{eq:inv_muT_def} and

\begin{align}
	I(t) = I_0 + \mathbf{M}^{-1}\left[I(\mu)\right] &= I_0^* + e^{-j\omega_0 t} \mathbf{L}^{-1}\left[\dfrac{\Delta I}{\mu - j\omega_0} + \dfrac{B}{\mu - \mu_{(+)}} + \dfrac{C}{\mu - \mu_{(-)}}\right] = \nonumber\\[5mm]
%
	&= I_0^* + e^{-j\omega_0 t}\left(\Delta Ie^{j\omega_0 t} + Be^{\mu_{(+)}t} + Ce^{j\mu_{(-)}t}\right) = \nonumber\\[5mm]
%
	&= I_0^* + \Delta I + Be^{\left(\mu_{(+)} - j\omega_0\right)t} + Ce^{j\left(\mu_{(-)} - j\omega_0\right)t}
\end{align}

	\noindent and note that $I_0^* + \Delta I$ is the current reference $I^*$ after the disturbance; thus,

\begin{equation} I(t) = I^* + Be^{\left(\mu_{(+)} - j\omega_0\right)t} + Ce^{j\left(\mu_{(-)} - j\omega_0\right)t}. \end{equation}

	We again note that because $k_P$ and $k_I$ are chosen so as to make $\mu_{(\pm)}$ stable (i.e. with negative real part), their exponentials vanish and $I(t)$ approaches $I^*$ at infinity. By definition, the time signal that this Dynamic Phasor reconstructs is

\begin{equation} i(t) = \Re\left(I(t)e^{j\omega_0 t}\right) = \Re\left[\Delta Ie^{j\omega_0 t} + Be^{\mu_{(+)}t} + Ce^{\mu_{(-)}t}\right] \end{equation}

	\noindent and because the real part maintains complex sum, 

\begin{equation} i(t) = \Re\left[ I^*e^{j\omega_0 t}\right] + \Re\left[Be^{\mu_{(+)}t} + Ce^{\mu_{(-)}t}\right]. \end{equation}

	\noindent and note that $\Re\left[ I^*e^{j\omega_0 t}\right]$ is equal to $\mathbf{P_D^{(-\omega)}}\left[I^*\right]$, that is, the static sinusoid $i_\infty(t)$ reconstructed by $I^*$ after the disturbance. Denoting $I^* = \left\lvert I^*\right\rvert e^{j\phi_I}$, then $i^*(t) = \left\lvert I^* \right\rvert \cos\left(\omega_0t + \phi_I\right)$. Again considering that $\mu_{(\pm)}$ are stable, the second portion of the sum vanishes exponentially; therefore,

\begin{equation} \lim\limits_{t\to\infty} \left[ i(t) - i^*(t)\right] = 0 \label{eq:final_current_time}\end{equation}

	\noindent or equivalently, $i^*$ is the assymptotic solution of $i(t)$. This again shows that the PI controller proposed indeed forces $i(t)$ to some reference signal $i^*(t)$.

	Although seemingly simple, equation \eqref{eq:final_current_time} denotes that indeed by controlling the phasor $I$ and forcing it to a reference $I^*$, then its time counterpart $i(t)$ is also controlled and forced to the signal $i^*(t)$ represented by $I^*$, meaning that the controller in Dynamic Phasor space is equivalent to a control in the time domain.

	We adopt the values $R = 100m\Omega$, $L = 1mF$, and we imagine that at the initial state the infinite bus is in phase to the angle reference, that is, $\left\lvert V_\infty\right\rvert = 100V, \phi_\infty = 0$. We suppose that the system departs from an equilibrium and outputs a complex power $S_0 = P_0 + Q_0 = 1kW + j100VAR$ measured at the terminal bus, yielding a system where the initial condition for the current can be calculated as

\begin{equation}
	\left\{\begin{array}{l}
		P_0 = R\left(I_d^2 + I_q^2\right) + \left\lvert V_\infty\right\rvert I_d \\[5mm]
		Q_0 = \omega_0 L\left(I_d^2 + I_q^2\right) - \left\lvert V_\infty\right\rvert I_q
	\end{array}\right.
\end{equation}

	\noindent and for the adopted values this yields a solution $I_0 = 9.9899813 - j0.62230395A$; we adopt these values as the curent setpoint. We simulate the system under two scenarios: in scenario 1 the infinite bus is maintained constant but the current setpoint is augmented by 20\% at $t = 0$, that is, $\Delta I = 0.2 I_0$. In scenario 2 the current setpoint is maintained but the infinite bus suffers a 5\% increase in magnitude at $t = 0$, that is, $\Delta V = 0.05 \left\lvert V_\infty\right\rvert$.

	We use the values $k_P = 0.1,\ k_I = 10$ for the $\mu$-PI controller, and this yields a pair of poles

\begin{equation}\left\{\begin{array}{l} \mu_{(-)} = -103.926446502 - j23.3991623121\\ \mu_{(+)} = -6.07355349812 + j400.390280742 \end{array}\right.\end{equation}

	\noindent and for each scenario the values of $A$ and $B$ calculated are

\begin{gather} \text{First scenario: }\left\{\begin{array}{l} A_1 = \hphantom{-}1.99799626023 - j0.124460790227 \\ B_1 = -1.82087180780 - j0.284338508168 \\  C_1 = -1.82087180780 - j0.284338508168 \end{array}\right.\\[5mm] \text{Second scenario: }\left\{\begin{array}{l} A_2 = 0 \\ B_2 = -2.58633785371 + j11.2011269665 \\ C _2= -2.58633785371 + j11.2011269665\end{array}\right. .\end{gather}

	Figures \ref{fig:dpftsim_scen1} and \ref{fig:dpftsim_scen2} show the time evolution of the bus current component as a function of time for scenario 1 and scenario 2, respectively. Top and middle plots show direct and quadrature components as functions of time with setpoints in dashed line, bottom plots show the bus current evolving in the complex domain. Figure \ref{fig:dpftsim_time_signals} shows the time signals reconstructed from these simulations. In all figures, the color evolution represents velocity — the absolute value of $\dot{I}(t)$, calculated as

\begin{equation} \left\lvert\dfrac{d}{dt} I(t)\right\rvert = \left\lvert \dot{I}_d(t) + j\dot{I}_q(t)\right\rvert = \sqrt{\left[\dot{I}_d(t)\right]^2 + \left[\dot{I}_q(t)\right]^2} \end{equation}

	\noindent and the color gradient is represented by a blue-to-red hue where blue (``cold'') denotes a slow variation and red (``hot'') denotes fast variation.

% CURRENT DP SIGNALS FOR SCENARIO 1 <<<
\begin{figure}
        \begin{center}
                \begin{tikzpicture}
                        \begin{axis}[
				name = ax_main,
                                width = 1\columnwidth,
                                height = 0.6/1.618*\columnwidth,
                                title={Dynamic Phasor of bus current $I(t)$ for scenario 1 (current setpoint disturbance)},
                                ylabel={$I_d$ (A)},
				xlabel={Time (s)},
                                xmin=-0.025, xmax=1,
                                ymin=9.8, ymax=13,
                                xtick={-0.1,0,...,1},
                                ytick={10,11,...,13}, 
                                legend pos=south east,
				legend cell align={left},
                                ymajorgrids=true,
                                xmajorgrids=true,
                                every axis plot/.append style={thick},
                        ]
                        \addplot[mesh, colormap={redblue}{color(0cm)=(blue);color(1cm)=(red);},point meta=explicit] table[col sep=comma,header=false,x index=0,y index=1, meta index=4]{data/dpft_sim/data_dpft_sim_scenario1.csv};
			\addlegendentry{$I_d(t)$}
		        \addplot[blue,  smooth, forget plot] coordinates {(-0.1,9.9899813) (0,9.9899813)};
			\addplot[black, dashed] coordinates {(-0.1,9.9899813) (0,9.9899813) (0,11.98797756) (1,11.98797756)};
			\addlegendentry{$I^*_d(t)$}
                        \end{axis}
%
                        \begin{axis}[
				name = ax_imaginary,
                                at={($(ax_main.south west)-(0,0.35*\columnwidth)$)},
                                width = 1\columnwidth,
                                height = 0.6/1.618*\columnwidth,
                                xmin=-0.025, xmax=1,
                                ymin=-2.1, ymax=-0.1,
                                xtick={-0.1,0,...,1},
                                ytick={-2,-1.5,...,-0.5},
				xlabel={Time (s)},
                                ylabel={$I_q(t)$ (A)},
				tick label style={/pgf/number format/fixed},
				legend cell align={left},
                                legend pos=south east,
                                ymajorgrids=true,
                                xmajorgrids=true,
                                every axis plot/.append style={thick},
                        ]
                        \addplot[mesh, colormap={redblue}{color(0cm)=(blue);color(1cm)=(red);},point meta=explicit] table[col sep=comma,header=false,x index=0,y index=2, meta index=4]{data/dpft_sim/data_dpft_sim_scenario1.csv};
			\addlegendentry{$I_q(t)$}
			\addplot[blue,  smooth, forget plot] coordinates {(-0.1,-0.62230395) (0,-0.62230395)};
			\addplot[black, dashed] coordinates {(-0.1,-0.62230395) (0,-0.62230395) (0,-0.74676474) (1,-0.74676474)};
			\addlegendentry{$I^*_q(t)$}
                        \end{axis}
%
                        \begin{axis}[
				name = ax_main,
                                at={($(ax_imaginary.south west)-(0,0.65*\columnwidth)$)},
                                width = 1\columnwidth,
                                height = 1/1.618*\columnwidth,
                                title={Dynamic Phasor of bus current $I(t)$ for scenario 1},
                                xlabel={$I_d$ (A)},
                                ylabel={$I_q$ (A)},
                                xmin=9.9, xmax=13,
                                ymin=-2.1, ymax=0,
                                xtick={10,10.5,...,13},
                                ytick={-2,-1.5,...,-0.5,0},
                                legend pos=south east,
				legend cell align={left},
                                ymajorgrids=true,
                                xmajorgrids=true,
                                every axis plot/.append style={thick}
                        ]
                        \addplot[mesh, colormap={redblue}{color(0cm)=(blue);color(1cm)=(red);},point meta=explicit] table[col sep=comma,header=false,x index=1,y index=2, meta index=4]{data/dpft_sim/data_dpft_sim_scenario1.csv};
                        \end{axis}
                \end{tikzpicture}
        \endpgfgraphicnamed
        \caption
[Bus current signal results of the DPFT simulation, scenario 1.]
{Bus current signal results of the DPFT simulation, scenario 1 (disturbance on current setpoint). Top plot shows direct component as a function of time, middle plot shows quadrature component as a function of time, bottom plot shows $I(t)$ evolving in the complex plane. Color gradient means rate of growth. Dashed line represents current setpoint.}
        \label{fig:dpftsim_scen1}
        \end{center}
\end{figure}
% >>>

% CURRENT DP SIGNALS FOR SCENARIO 2 <<<
\begin{figure}
        \begin{center}
                \begin{tikzpicture}
                        \begin{axis}[
				name = ax_main,
                                width = 1\columnwidth,
                                height = 0.6/1.618*\columnwidth,
                                title={Dynamic Phasor of bus current $I(t)$ for scenario 2 (infinite bus voltage disturbance)},
                                ylabel={$I_d$ (A)},
				xlabel={Time (s)},
                                xmin=-0.025, xmax=1,
                                ymin=-2, ymax=14,
                                xtick={-0.1,0,...,1},
                                ytick={0,3,...,12}, 
                                legend pos=south east,
				legend cell align={left},
                                ymajorgrids=true,
                                xmajorgrids=true,
                                every axis plot/.append style={thick},
                        ]
                        \addplot[mesh, colormap={redblue}{color(0cm)=(blue);color(1cm)=(red);},point meta=explicit] table[col sep=comma,header=false,x index=0,y index=1, meta index=4]{data/dpft_sim/data_dpft_sim_scenario2.csv};
			\addlegendentry{$I_d(t)$}
		        \addplot[blue,  smooth, forget plot] coordinates {(-0.1,9.9899813) (0,9.9899813)};
			\addplot[black, dashed] coordinates {(-0.1,9.9899813) (1,9.9899813)};
			\addlegendentry{$I^*_d(t)$}
                        \end{axis}
%
                        \begin{axis}[
				name = ax_imaginary,
                                at={($(ax_main.south west)-(0,0.35*\columnwidth)$)},
                                width = 1\columnwidth,
                                height = 0.6/1.618*\columnwidth,
                                xmin=-0.025, xmax=1,
                                ymin=-7, ymax=15,
                                xtick={-0.1,0,...,1},
                                ytick={-5,0,...,15},
				xlabel={Time (s)},
                                ylabel={$I_q(t)$ (A)},
				tick label style={/pgf/number format/fixed},
				legend cell align={left},
                                ymajorgrids=true,
                                xmajorgrids=true,
                                every axis plot/.append style={thick},
                        ]
                        \addplot[mesh, colormap={redblue}{color(0cm)=(blue);color(1cm)=(red);},point meta=explicit] table[col sep=comma,header=false,x index=0,y index=2, meta index=4]{data/dpft_sim/data_dpft_sim_scenario2.csv};
			\addlegendentry{$I_q(t)$}
			\addplot[blue,  smooth, forget plot] coordinates {(-0.1,-0.62230395) (0,-0.62230395)};
			\addplot[black, dashed] coordinates {(-0.1,-0.62230395) (1,-0.62230395)};
			\addlegendentry{$I^*_q(t)$}
                        \end{axis}
%
                        \begin{axis}[
                                at={($(ax_imaginary.south west)-(0,0.65*\columnwidth)$)},
                                width = 1\columnwidth,
                                title={Dynamic Phasor of bus current $I(t)$ for scenario 2},
                                height = 1/1.618*\columnwidth,
                                xmin=-1.3, xmax=14,
                                ymin=-7, ymax=15,
                                xtick={0,2,...,14},
                                ytick={-6,-3,...,15},
                                xlabel={$I_d(t)$ (A)},
                                ylabel={$I_q(t)$ (A)},
				tick label style={/pgf/number format/fixed},
				legend cell align={left},
                                ymajorgrids=true,
                                xmajorgrids=true,
                                every axis plot/.append style={thick},
                        ]
                        \addplot[mesh, colormap={redblue}{color(0cm)=(blue);color(1cm)=(red);},point meta=explicit] table[col sep=comma,header=false,x index=1,y index=2, meta index=4]{data/dpft_sim/data_dpft_sim_scenario2.csv};
                        \end{axis}
                \end{tikzpicture}
        \endpgfgraphicnamed
        \caption
[Bus current signal results of the DPFT simulation, scenario 2.]
{Bus current signal results of the DPFT simulation, scenario 2 (disturbance on infinite bus voltage). Top plot shows direct component as a function of time, middle plot shows quadrature component as a function of time, bottom plot shows $I(t)$ evolving in the complex plane. Color gradient means rate of growth. Dashed line represents current setpoint.}
        \label{fig:dpftsim_scen2}
        \end{center}
\end{figure}
% >>>

% TIME CURRENT SIGNALS FOR BOTH SCENARIOS <<<
\begin{figure}
        \begin{center}
                \beginpgfgraphicnamed{timesim_slow}
                \begin{tikzpicture}
                        \begin{axis}[
				name = ax_main,
                                width = 1\columnwidth,
                                height = 1/1.618*\columnwidth,
                                title={Bus current time domain signal for scenario 1 (current setpoint disturbance)},
                                ylabel={$i(t)$ (A)},
				xlabel={Time (s)},
                                xmin=0, xmax=1,
                                ymin=-15, ymax=15,
                                xtick={0,0.1,...,1},
                                ytick={-15,-10,...,12.5},
                                legend pos=south east,
				legend cell align={left},
                                ymajorgrids=true,
                                xmajorgrids=true,
                                every axis plot/.append style={thick},
                        ]
                        \addplot[mesh, colormap={redblue}{color(0cm)=(blue);color(1cm)=(red);},point meta=explicit] table[col sep=comma,header=false,x index=0,y index=3, meta index=4]{data/dpft_sim/data_dpft_sim_scenario1.csv};
                        \end{axis}
%
                        \begin{axis}[
                                at={($(ax_main.south west)-(0,0.7*\columnwidth)$)},
                                width = 1\columnwidth,
                                height = 1/1.618*\columnwidth,
                                title={Bus current time domain signal for scenario 2 (infinite bus voltage disturbance)},
                                xmin=0, xmax=1,
                                ymin=-15, ymax=15,
                                xtick={0,0.1,...,1},
                                ytick={-15,-10,...,15},
				xlabel={Time (s)},
                                ylabel={$i(t)$ (A)},
				tick label style={/pgf/number format/fixed},
				legend cell align={left},
                                ymajorgrids=true,
                                xmajorgrids=true,
                                every axis plot/.append style={thick},
                        ]
                        \addplot[mesh, colormap={redblue}{color(0cm)=(blue);color(1cm)=(red);},point meta=explicit] table[col sep=comma,header=false,x index=0,y index=3, meta index=4]{data/dpft_sim/data_dpft_sim_scenario2.csv};
                        \end{axis}
                \end{tikzpicture}
        \endpgfgraphicnamed
        \caption
[Time signals of the bus current for both DPFT simulation scenarios.]
{Time signals of the bus current for both DPFT simulation scenarios. On the top, scenario 1 (perturbation on the current setpoint) and on the bottom, scenario 2 (perturbation on the infinite bus voltage).}
        \label{fig:dpftsim_time_signals}
        \end{center}
\end{figure}
% >>>


\part{Applications and conclusion}\label{part:ending}

% ---------------------------------------------------------
\chapter{Applications to Power System and Electronic Circuits}\label{chapter:applications}
% ---------------------------------------------------------

	In this chapter we show three applications of the theory developed in this thesis, specifically three applications that were used as main motivators in the introduction of the thesis.

	The first motivation, of section \ref{sec:omib_dynphasor_sim}, is that of modelling Power Systems using the Quasi-Static Hypothesis. In that example we model a simple One-Machine Infinite Bus system using Dynamic Phasor Functionals, considering all the transient phenomena of transmission lines; further, we use the Quasi-Static Modelling technique to yield an approximated model where these transient characteristics are not considered, and we show that this approximated model does indeed produce some simulation of the system by disregarding fast transients. This example is used to illustrate the Quasi-Static Hypothesis and how it discards fast electromagnetic phenomena in transmission systems.

	In the second motivation, of section \ref{sec:admittance_modelling_dp}, we show that the Dynamic Phasor Functionals are capable of producing models for transmission systems such that the voltages $V(t)$ of the nodes of the system and the branch currents $I(t)$ are related by an admittance operator $I(t) = \mathbf{Y}\left[V\right]$. We further show that this admittance operator is highly resemblant of the admittance matrix calculated for multimachine Power Systems, and the procedure for obtaining the admittance matrix operator is largely the same as that of commonplace techniques.

	Finally, section \ref{sec:bjt_ampli_modelling} shows the modelling of a Bipolar Junction Transistor common-emitter amplifier to show that the Dynamic Phasor Theory proposed is also applicable to nonlinear systems when they are linearized. As such, the theory hereby presented is able to produce linearized models of electronic circuits that highly resemble the models currently used, but in a generalized sinusoidal manner. Therefore, the models produced are generalizations of the linearized models adopted, allowing for an expansion of such models in generalized sinusoidal conditions.

%-------------------------------------------------
\section{Simulation of a simple Power System}\label{sec:omib_dynphasor_sim} %<<<1

	Consider the one machine versus infinite bus model of figure \ref{fig:example_omib_simulation} where the machine is connected to an infinite bus through a pure inductive line of reactance $X$ measured at the synchronous frequency $\omega_0$, corresponding to an inductance $L = X\omega_0^{-1}$. We will model a short-circuit contingency where the terminal bus of the machine is shorted to ground at $t = 0$ through $t = t_o$, the subscript ``o'' for ``opening'' because it is known as a ``fault opening time''. The objective is to model the transmission line using a Dynamic Phasors approach and compare the results with a static-behavior approximation.

% EXAMPLE: OMIB <<<
\begin{figure}[h]
\centering
\scalebox{0.8}{
        \begin{tikzpicture}[american,scale=1.2,transform shape,line width=0.75, cute inductors,>={Stealth[inset=0mm,length=1.5mm,angle'=50]}]
		\ctikzset{sources/scale=1.2}
		\node (origin) at (0,0) {};
		\node [shape=vsourcesinshape, rotate=-90] (gen1) at (-2,0) {} ;
		\draw[->] ([shift=({-1.1,0})]gen1.south) -- +(1,0) node [midway,below] {$P_{m}$};

		\draw (gen1.north) to[short] ++(1,0) coordinate(inducedbar_up) to[short] ++(0,0.5) to [short] ++(0.3,0) to[L,l=$x'$] ++(1.5,0) to [R,l=$r$] ++(1,0) to[short] ++(0.5,0) to[short] ++(0,-0.5) coordinate(terminalbar);
		\draw ([shift=({0,0.5})]terminalbar) to[L,l=$X$,-] ++(4,0)  coordinate(vinfbar_up);

		\node (vinfbar) at (vinfbar_up |- origin) {};
		\node (inducedbar) at (inducedbar_up |- origin) {};

		\draw[line width=1mm] ([shift=({0,1})]terminalbar) -- ++(0,-2); % V voltage
		\node (vtlabel) at ([shift=({0,1.5})]terminalbar) {$V(t)$};

		\draw[line width=1mm] ([shift=({0,1})]inducedbar.center) -- ++(0,-2); % V voltage
		\node (elabel) at ([shift=({0,1.5})]inducedbar) {$E(t)$};

		\draw[line width=1mm] ([shift=({0,1})]vinfbar.center) -- ++(0,-2); % V voltage

		\node (vinfsource) [shape=vsourcesinshape, rotate=-90] at ([shift=({1,0})]vinfbar) {};
		\node[right] () at ([shift=({0.5,0})]vinfsource) {$V_\infty {=} \left\lvert V_\infty\right\rvert e^{j\phi_\infty}$};

		\draw (vinfbar.center) to (vinfsource.south);

		\draw ([shift=({0,-0.5})]terminalbar.center) to[opening switch, l_=$t_o$] ++(2,0) to[short] ++(0,-1) node [tlground] {}; % V voltage
		\draw ([shift=({0,-0.5})]terminalbar.center) to[short] ++(-1,0) to[R, l_=$R_L$] ++(0,-2) node [tlground] {}; % V voltage

	\end{tikzpicture}
}
	\caption{One-Machine-Infinite-Bus System with resistive load for example modelling and simulation.}
	\label{fig:example_omib_simulation}
\end{figure} %>>>

	We adopt the classic model \eqref{eq:machine_2a_model_classical} for the machine, according to which the machine is simplified as an internal induced voltage source $E$ behind a transient reactance $x'$, measured at synchronous frequency and corresponding with an inductance $L' = x'\omega_0^{-1}$, where $E$ is supposed constant throughout the simulation. We will also assume that the mechanical power $P_m$ supplied to the machine shaft is adjusted through a simple Droop controller, that is, linearly with frequency deviations: $P_m = P_m^* - k_P\omega$, with $k_P$ some linear gain and $P_m^*$ a reference value.

	Figure \ref{fig:dynamic_phasor_dqaxis_omib} shows a phasorial diagram of the system. The machine DQ frame rotates at an angle $\omega_m$ (the subscript ``m'' for machine) that is governed by the swing equation, generating an angular distance $\delta_m$ with respect to the grid synchronous reference $R$ which rotates at $\omega_0$. The infinite bus voltage has by definition a constant amplitude and phase with respect to $R$. It is important to note that in the classicl model, the variable $\omega$ is the per-unit deviation of the rotor frequency from the synchronous frequency, that is, $\omega_m = \omega_0\omega$, causing an equivalent relationship for the angle $\delta_m = \omega_0\delta$.

	To simplify analysis, we admit that the machine is modelled by the classical model \eqref{eq:machine_2a_model_classical} — even though this model is based on the static phasor approximation, building a new model is not the scope of this text.

%------------------------------------------------- 
\subsection{System model without short} %<<<2

	When the terminal bus of the system is not shorted ($t < 0$ or $t > t_o$) the Dynamic Phasor model of the transmission line using the machine frequency $\omega_m$ as apparent frequency for the DPT is given by

% DYNAMIC PHASOR DIAGRAM OF OMIB SYSTEM <<<
\begin{figure}[htb!]
\centering
\scalebox{1}{
	\begin{tikzpicture}[scale=2,>={Stealth[inset=0mm,length=1.5mm,angle'=50]}]
		\node (origin) at (0,0) {};
		\draw [->, black!50, name path = dmaxis] (0,0) -- ({40mm*cos(30)}, {40mm*sin(30)});
		\draw [->, black!50, name path = qmaxis] (0,0) -- ({40mm*cos(120)},{40mm*sin(120)});

		\node [label={[text=stewartpink, label distance=1mm]0:$D_s$}] (ReAxisLabel) at ({40mm*cos(0)} ,{40mm*sin(0)})  {};
		\draw [->, stewartpink] (0,0) -- (ReAxisLabel.center);
		\node [label={[text=stewartpink, label distance=1mm]0:$Q_s$}] (ImAxisLabel) at ({40mm*cos(90)} ,{40mm*sin(90)})  {};
		\draw [->, stewartpink] (0,0) -- (ImAxisLabel.center);
		\draw [->, stewartpink] ({18mm*cos(0)},{18mm*sin(0)}) arc[start angle=0, end angle = 28, radius = 18mm];
		\node [stewartpink] (philabel) at ({21mm*cos(6)},{21mm*sin(6)}) {$\delta_m(t)$};

		\node [right,black!50] (DAxisLabel) at ({42mm*cos(30) - 2mm} ,{42mm*sin(30)})  {$D_m$};
		\node [black!50]       (QAxisLabel) at ({42mm*cos(120)},      {42mm*sin(120)}) {$Q_m$};

		\node [black!50] (omegat) at ({35mm*cos(37)},{35mm*sin(37)}) {$\omega_m(t)$};
		\draw [-{Stealth[inset=0mm,length=3.5mm,angle'=50]}, black!50, line width = 1mm] ({35mm*cos(25)},{35mm*sin(25)}) arc[start angle=25, end angle = 35, radius = 35mm];

		\node [stewartpink] (omegat) at ({37mm*cos(7)},{37mm*sin(7)}) {$\omega_0$};
		\draw [-{Stealth[inset=0mm,length=3.5mm,angle'=50]}, stewartpink, line width = 1mm] ({37mm*cos(-5)},{37mm*sin(-5)}) arc[start angle=-5, end angle = 5, radius = 37mm];

		\node [right, stewartyellow] (elabel) at ({35mm*cos(70)},{35mm*sin(70)}) {$E(t)$};
		\draw [->, stewartyellow] (0,0) -- (elabel);

		% Obtaining Ed and Eq projections
		\path [name path = edprojection] (elabel) -- +($(origin)-(QAxisLabel)$);
		\path [name intersections={of=edprojection and dmaxis, by=Ed}];
		\node [circle,fill=stewartyellow,inner sep=1.5pt,label={[text=stewartyellow]-60:$E_d$}] at (Ed) {};
		\draw [dashed,stewartyellow] (Ed) -- (elabel);

		\path [name path = eqprojection] (elabel) -- +($(origin)-(DAxisLabel)$);
		\path [name intersections={of=eqprojection and qmaxis, by=Eq}];
		\node [circle,fill=stewartyellow,inner sep=1.5pt,label={[text=stewartyellow]-120:$E_q$}] at (Eq) {};
		\draw [dashed,stewartyellow] (Eq) -- (elabel);

		% Drawing V
		\node [right, stewartblue] (vlabel) at ({40mm*cos(55)},{40mm*sin(55)}) {$V(t)$};
		\draw [->, stewartblue] (0,0) -- (vlabel);

		\node [label={[text=stewartgreen, label distance=1mm]0:$\left\lvert V_\infty\right\rvert e^{j\phi_\infty}$}] (vinflabel) at ({42mm*cos(15)},{42mm*sin(15)}) {};
		\draw [->,   stewartgreen] (0,0) -- (vinflabel.center);
		\draw [->,   stewartgreen] ({30mm*cos(0)},{30mm*sin(0)}) arc[start angle=0, end angle = 13, radius = 30mm];
		\node [stewartgreen] (philabel) at ({32mm*cos(7)},{32mm*sin(7)}) {$\phi_\infty$};

		\node [right, stewartpurple] (ilabel) at ({37mm*cos(45)},{37mm*sin(45)}) {$I(t)$};
		\draw [->,   stewartpurple] (0,0) -- (ilabel);
	\end{tikzpicture}
	}
	\caption
[Phasor diagram for the OMIB system being simulated.]
{Phasor diagram for the OMIB system being simulated showing the machine phase frame ``$D_m/Q_m$'' frame, the subscript ``m'' for ``machine'', and the synchronous grid angular frame ``$D_s/Q_s$'' frame.}
	\label{fig:dynamic_phasor_dqaxis_omib}
\end{figure} %>>>

\begin{gather}
	\left(\dfrac{\mathbf{I}}{r\mathbf{I} + \dpo L'}\right)\left[E - V\right] = \dfrac{1}{R_L} V + \left(\dfrac{\mathbf{I}}{L\dpo}\right)\left[V - V_\infty\right] \nonumber\\[5mm]
%
	L\dpo \left[E - V\right] = \left[\dfrac{1}{R_L}\left(r\mathbf{I} + \dpo L'\right)\dpo L\right] V + \left(r\mathbf{I} + \dpo L'\right)\left[V - V_\infty\right] \nonumber\\[5mm]
%           
	L\dpo \left[E\right] + \left(r\mathbf{I} + \dpo L'\right)\left[V_\infty\right] = \left[\left(\dfrac{r\mathbf{I} + \dpo L'}{R_L}\right)\dpo L + \dpo L + \left(r\mathbf{I} + \dpo L'\right)\right] V \nonumber\\[5mm]
%            
	L\dpo \left[E\right] + \left(r\mathbf{I} + \dpo L'\right)\left[V_\infty\right] = \left[r\mathbf{I} + \left(\dfrac{rL}{R_L} + L + L'\right)\dpo + \dfrac{LL'}{R_L}\ndpo{2}\right] V \label{eq:i_omib_diffeq}
\end{gather}

	\noindent and this achieves a second-order differential equation for $V$. Obtaining the bus current from $V$ can be done from the same law:

\begin{gather}
	I = \dfrac{1}{R_L} V + \left(\dfrac{\mathbf{I}}{L\dpo}\right)\left[V - V_\infty\right] \nonumber\\[5mm]
%
	L\dpo\left[I\right] = \dfrac{1}{R_L} L\dpo\left[V\right] + V - V_\infty = \left(\dfrac{L}{R_L}\dpo + \mathbf{I}\right)\left[V\right] - V_\infty \label{eq:v_omib_diffeq}
\end{gather}

	\noindent and substituting \eqref{eq:v_omib_diffeq} into \eqref{eq:i_omib_diffeq}. First multiply \eqref{eq:i_omib_diffeq} by $\left(\frac{L}{R_L}\dpo + \mathbf{I}\right)$:
\begin{equation}
	\hspace{-3mm} \left(\dfrac{L}{R_L}\dpo + \mathbf{I}\right)\left\{\raisebox{4mm}{} L\dpo \left[E\right] + \left(r\mathbf{I} + \dpo L'\right)\left[V_\infty\right]\right\} = \left\{\left(\dfrac{L}{R_L}\dpo + \mathbf{I}\right)\left[r\mathbf{I} + \left(\dfrac{rL}{R_L} + L + L'\right)\dpo + \dfrac{LL'}{R_L}\ndpo{2}\right]\right\}\left[ V\right]
\end{equation}

	\noindent and because operators are commutative,

\begin{equation}
	\hspace{-3mm} \left(\dfrac{L}{R_L}\dpo + \mathbf{I}\right)\left\{\raisebox{4mm}{} L\dpo \left[E\right] + \left(r\mathbf{I} + \dpo L'\right)\left[V_\infty\right]\right\} = \left[r\mathbf{I} + \left(\dfrac{rL}{R_L} + L + L'\right)\dpo + \dfrac{LL'}{R_L}\ndpo{2}\right]\left[ \left(\dfrac{L}{R_L}\dpo + \mathbf{I}\right)\left[V\right]\right]
\end{equation}

	\noindent now use \eqref{eq:v_omib_diffeq} to yield

\begin{equation}
	\left(\dfrac{L}{R_L}\dpo + \mathbf{I}\right)\left\{\raisebox{4mm}{} L\dpo \left[E\right] + \left(r\mathbf{I} + \dpo L'\right)\left[V_\infty\right]\right\} = \left[r\mathbf{I} + \left(\dfrac{rL}{R_L} + L + L'\right)\dpo + \dfrac{LL'}{R_L}\ndpo{2}\right]\left\{\raisebox{4mm}{} L\dpo\left[I\right] + V_\infty\right\}
\end{equation}

	\noindent and isolating the current $I$,

\begin{align}
	\left(\dfrac{L^2}{R_L}\ndpo{2} + L\dpo\right)\left[E\right] - \left\{\left[r\mathbf{I} + \left(\dfrac{rL}{R_L} + L + L'\right)\dpo + \dfrac{LL'}{R_L}\ndpo{2}\right] - \left(\dfrac{L}{R_L}\dpo + \mathbf{I}\right)\left(r\mathbf{I} + \dpo L'\right)\right\}\left[V_\infty\right] &= \nonumber\\[5mm] & \hspace{-9cm} = \left[r\mathbf{I} + \left(\dfrac{rL}{R_L} + L + L'\right)\dpo + \dfrac{LL'}{R_L}\ndpo{2}\right]\left\{\raisebox{4mm}{} L\dpo\left[I\right] \right\} \\[5mm]
%
	\left(\dfrac{L^2}{R_L}\ndpo{2} + L\dpo\right)\left[E\right] - \left\{\left[r\mathbf{I} + \left(\dfrac{rL}{R_L} + L + L'\right)\dpo + \dfrac{LL'}{R_L}\ndpo{2}\right] - \left[\dfrac{LL'}{R_L}\ndpo{2} + \left(\dfrac{Lr}{R_L} + L'\right)\dpo + r\mathbf{I}\right]\right\}\left[V_\infty\right] &= \nonumber\\[5mm] & \hspace{-9cm} = \left[r\mathbf{I} + \left(\dfrac{rL}{R_L} + L + L'\right)\dpo + \dfrac{LL'}{R_L}\ndpo{2}\right]\left\{\raisebox{4mm}{} L\dpo\left[I\right] \right\} \\[5mm]
%
	&\hspace{-14cm} \left(\dfrac{L^2}{R_L}\ndpo{2} + L\dpo\right)\left[E\right] - L\dpo\left[V_\infty\right] = \left[r\mathbf{I} + \left(\dfrac{rL}{R_L} + L + L'\right)\dpo + \dfrac{LL'}{R_L}\ndpo{2}\right]\left\{\raisebox{4mm}{} L\dpo\left[I\right] \right\}
\end{align}

	\noindent and dividing the entire equation by $L\dpo$:

\begin{equation}
	\left(\dfrac{L}{R_L}\dpo + \mathbf{I}\right)\left[E\right] - V_\infty  = \left[r\mathbf{I} + \left(\dfrac{rL}{R_L} + L + L'\right)\dpo + \dfrac{LL'}{R_L}\ndpo{2}\right]\left[I\right]
\end{equation}

	Now adopt $\omega(t)$ as the apparent frequency for the Dynamic Phasor Transform and expanding this equation yields

\begin{align}
& \dfrac{L L'}{R_L}\ddot{I}(t) + \left[L\left(1 + \frac{r}{R_L}\right) + L' + j\left(\frac{2 L L'}{R_L}\omega(t)\right)\right]\dot{I}(t) + \nonumber\\[5mm]
%
& \hspace{1cm} \left( r - \dfrac{L L'}{R_L}\omega^2(t) + j\left\{\dfrac{L L'}{R_L}\dot{\omega}(t) + \left[L\left(1 + \dfrac{r}{R_L}\right) + L'\right] \omega(t)\right\}\right) I(t) \nonumber\\[5mm]
%
& \hspace{2cm} = \dfrac{L}{R_L}\dot{E} + \left(1 + j\dfrac{\omega L}{R_L}\right)E - V_\infty
\end{align}

	\noindent and multiplying the entire equation by $R_L/LL'$,

\begin{align}
& \ddot{I}(t) + \left[\dfrac{R_L + r}{L'}+ \dfrac{R_L}{L} +  2j\omega(t)\right]\dot{I}(t) + \left\{ \dfrac{rR_L}{LL'} - \omega^2(t) + j\left[\dot{\omega}(t) + \left(\dfrac{R_L + r}{L'}+ \dfrac{R_L}{L}\right) \omega(t)\right]\right\} I(t) \nonumber\\[5mm]
%
& \hspace{2cm} = \dfrac{1}{L'}\dot{E} + \left(\dfrac{R_L}{LL'} + j\dfrac{\omega}{L'}\right)E - \dfrac{R_L}{LL'}V_\infty . \label{eq:omib_sim_final_curr_model}
\end{align}

	The modelling will be done in the DQ frame of the transmission grid at the synchronous frequency (the pink frame on figure \ref{fig:dynamic_phasor_dqaxis_omib}), using the synchronous frequency $\omega_0$ for the Dynamic Phasor Transform. In that frame, $V_\infty$ is a constant number and the internal induced voltage $E$ is equal to

\begin{align}
	E = \left(E_d + jE_q\right)e^{j\delta_m} \Rightarrow \dot{E} &= \left(\dot{E}_d + j\dot{E}_q\right)e^{j\delta_m} + \left(E_d + jE_q\right)\omega_m e^{j\delta_m} = \nonumber\\[5mm] &= \left[\left(\dot{E}_d + \omega_m E_d\right) + j\left(\dot{E}_q + \omega_m E_q\right)\right]e^{j\delta_m}
\end{align}

	\noindent where $E_d$ and $E_q$ are given by the differential model of the machine. Using these facts on \eqref{eq:omib_sim_final_curr_model} yields a current model

\begin{align}
& \ddot{I}(t) + \left(\dfrac{R_L + r}{L'}+ \dfrac{R_L}{L} +  2j\omega_0\right)\dot{I}(t) + \left\{ \dfrac{rR_L}{LL'} - \omega_0 + j\left(\dfrac{R_L + r}{L'}+ \dfrac{R_L}{L}\right) \omega_0 \right\} I(t) = \nonumber\\[5mm]
%
& \hspace{2cm} = \dfrac{1}{L'}\dot{E} + \left(\dfrac{R_L}{LL'} + j\dfrac{\omega_0}{L'}\right)E - \dfrac{R_L}{LL'}V_\infty
\end{align}

	\noindent dividing this entire equation by $\omega_0^2$ to achieve a per-unit-compatible model:

\begin{align}
& \dfrac{1}{\omega_0^2} \ddot{I}(t) + \left(\dfrac{R_L + r}{x'}+ \dfrac{R_L}{X} +  2j\right)\dfrac{1}{\omega_0}\dot{I}(t) + \left[ \dfrac{rR_L}{x'X} - 1 + j\left(\dfrac{R_L + r}{x'}+ \dfrac{R_L}{X}\right) \right] I(t) = \nonumber\\[5mm]
%
& \hspace{2cm} = \dfrac{1}{\omega_0 x'} \dot{E} + \left(\dfrac{R_L}{x'X} + j\dfrac{1}{x'}\right)E - \dfrac{R_L}{x'X}V_\infty. \label{eq:omib_current_final_pu_model}
\end{align}

	For this modelling we use the classical model

% MACHINE CLASSICAL MODEL <<<
\begin{equation}
	\left\{\begin{array}{l}
		\dot{\omega} = \dfrac{P_m - P_e}{2H} \\[5mm]
		\dot{\delta} = \omega \\[5mm]
		P_e = E_dI_d + E_qI_q \\[5mm]
		P_m = P_m^* - k_P\omega
	\end{array}\right. \label{eq:machine_2a_model_classical_modelling}
\end{equation} %>>>

	\noindent and in this model $E_d$ and $E_q$ are constant, and $\omega$ is given in a per-unit unit system such that the machine electrical frequency deviation is given by $\omega_m = \omega_0\left(\omega + \right)$. Similarly, the phase deviation is also given in a per-unit system such that $\delta_m = \omega_0\delta$. Coupling the model \eqref{eq:machine_2a_model_classical_modelling} to the grid equations \eqref{eq:omib_current_final_pu_model} achieves the model of the system:

% OMIB SYSTEM TOTAL MODEL <<<
\begin{equation}
	\left\{\begin{array}{l}
		\dot{\omega} = \dfrac{P_m - P_e}{2H} \\[5mm]
		\dot{\delta} = \omega \\[5mm]
		\dfrac{1}{\omega_0^2} \ddot{I} + \left(\dfrac{R_L + r}{x'} + \dfrac{R_L}{X} +  2j\right)\dfrac{1}{\omega_0}\dot{I} + \left[ \dfrac{rR_L}{x'X} - 1 + j\left(\dfrac{R_L + r}{x'}+ \dfrac{R_L}{X}\right) \right] I = \\[5mm]
%
		\hspace{2cm} = \left(\dfrac{\omega}{x'} + \dfrac{R_L}{x'X} + j\dfrac{2}{x'}\right)\left(E_d + jE_q\right)e^{j\omega_0\delta} - \dfrac{R_L}{x'X}V_\infty \\[5mm]
		P_e = E_dI_d + E_qI_q \\[5mm]
		P_m = P_m^* - k_P\omega
	\end{array}\right. \label{eq:omib_classical_total_model}
\end{equation} %>>>

	Naturally, the quasi-static approximation of this grid model is obtained by applying all current derivatives to zero:

% OMIB SYSTEM QUASISTATIC MODEL <<<
\begin{equation}
	\left\{\begin{array}{l}
		\dot{\omega} = \dfrac{P_m - P_e}{2H} \\[5mm]
		\dot{\delta} = \omega \\[5mm]
		I = \dfrac{\raisebox{-5mm}{} \left(\dfrac{\omega}{x'} + \dfrac{R_L}{x'X} + j\dfrac{2}{x'}\right)\left(E_d + jE_q\right)e^{j\omega_0\delta} - \dfrac{R_L}{x'X}V_\infty}{\raisebox{6mm}{} \dfrac{rR_L}{x'X} - 1 + j\left(\dfrac{R_L + r}{x'}+ \dfrac{R_L}{X}\right)} \\[12mm]
		P_e = E_dI_d + E_qI_q \\[5mm]
		P_m = P_m^* - k_P\omega
	\end{array}\right. \label{eq:omib_classical_quasi_model}
\end{equation} %>>>

%-------------------------------------------------
\subsection{System model while shorted} %<<<2

	Thus \eqref{eq:omib_classical_total_model} and \eqref{eq:omib_classical_quasi_model} achieve the complete and approximated models of the system when the terminar bus is not shorted. When the bus is shorted,

\begin{equation} E = \left(r\mathbf{I} + \dpo L' \right)\left[I\right] \Leftrightarrow E = L'\dot{I} + \left(r + j\omega(t)L'\right)I \end{equation} \noindent applying the modelling at the synchronous frequency $\omega_0$,

\begin{equation} E = \dfrac{x'}{\omega_0}\dot{I} + \left(r + jx'\right)I \end{equation}

	\noindent achieving a model of the system at the synchronous reference

% OMIB SYSTEM QUASISTATIC MODEL <<<
\begin{equation}
	\left\{\begin{array}{l}
		\dot{\omega} = \dfrac{P_m - P_e}{2H} \\[5mm]
		\dot{\delta} = \omega \\[5mm]
		\dot{I} = \dfrac{\omega_0}{x'}\left[\left(E_d + jE_q\right)e^{j\omega_0\delta} - \left(r + jx'\right)I \right] \\[5mm]
		P_e = E_dI_d + E_qI_q \\[5mm]
		P_m = P_m^* - k_P\omega
	\end{array}\right. \label{eq:omib_classical_short}
\end{equation} %>>>

	\noindent which generates a quasi-static model

% OMIB SYSTEM QUASISTATIC MODEL <<<
\begin{equation}
	\left\{\begin{array}{l}
		\dot{\omega} = \dfrac{P_m - P_e}{2H} \\[5mm]
		\dot{\delta} = \omega \\[5mm]
		I = \dfrac{\left(E_d + jE_q\right)e^{j\omega_0\delta}}{\left(r + jx'\right)} \\[5mm]
		P_e = E_dI_d + E_qI_q \\[5mm]
		P_m = P_m^* - k_P\omega
	\end{array}\right. \label{eq:omib_classical_short_quasistatic}
\end{equation} %>>>

%-------------------------------------------------
\subsection{Simulation} %<<<2

	The initial conditions for the simulation are calculated using power flow equations. We assume that the power angle $\delta$ is null at initial time and that the machine is supplying an initial power of $S_0 = 1 + j0.1$ and the terminal voltage and bus current are calculated by the equations

\begin{equation}
	\left\{\begin{array}{l}
		\left[\dfrac{rR_L}{x'X} - 1 + j\left(\dfrac{R_L + r}{x'} + \dfrac{R_L}{X}\right)\right]I - \left(\dfrac{R_L}{x'X} + j\dfrac{2}{x'}\right)E + \dfrac{R_L}{x'X}V_\infty = 0 \\[5mm]
		\left[E - \left(r + jx'\right)I\right]\overline{I} - S_0 = 0
	\end{array}\right.
\end{equation}

	\noindent where the first equation is the grid equation and the second equation is the power flow equation. From this system one obtains the initial values of table \ref{tab:initialcond_params} and calculates the initial mechanical power at equilibrium $P_m = P_e$, and we adopt the mechanical power setpoint $P_m^*$ as the initial mechanical power. As for parameters, we use the parameters of table \ref{tab:synchmachine_params}.

% TABLE OF INITIAL CONDITIONS <<<
\renewcommand{\arraystretch}{1.2}
\begin{table}[t]
\begin{center}
\begin{tabular}{ c|c|c|c|c } 
\hline 
\raisebox{-2mm}{} $E_d$ & $E_q$ & $I_d$ & $I_q$ & $P_m$ \\
\hline
$1.1566432$ pu & $0.50040052$ pu & $0.89092649$ pu & $-0.045016622$ pu & $1.0079578$ pu \\
\hline
\end{tabular}
\end{center}
\caption{Initial conditions of the synchronous machine for the simulation of the OMIB system of figure \ref{fig:example_omib_simulation}.}
\label{tab:initialcond_params}
\end{table} %>>>

% TABLE OF PARAMETERS <<<
\renewcommand{\arraystretch}{1.2}
\begin{table}[t]
\begin{center}
\begin{tabular}{ c|c|c|c|c|c|c|c|c|c } 
\hline 
\raisebox{-3mm}{} $\omega_0$ & $H$ & $\left\lvert V_\infty\right\rvert$ & $\phi_\infty$ & $r$ & $x'$ & $X$ & $k_P$ & $t_o$ & $R_L$\\
\hline
$120\pi$ rad.s$^{-1}$ & $1$ s & $1.1$ pu & $3^\circ$ & 0.01 pu & 0.5 pu& 0.1 pu & 10 & 0.1 s & 2.5 pu \\
\hline
\end{tabular}
\end{center}
\caption{Parameter values of the OMIB system of figure \ref{fig:example_omib_simulation} for simulation.}
\label{tab:synchmachine_params}
\end{table} %>>>

% FREQUENCY CURVES <<<
\begin{figure}
        \begin{center}
                \begin{tikzpicture}
                        \begin{axis}[
				name = ax_main,
                                width = 0.9*\columnwidth,
                                height = 0.9*1/1.618*\columnwidth,
                                title={OMIB simulation frequency signals},
                                xlabel={Time (s)},
                                ylabel={$\omega$},
                                xmin=0, xmax=2,
                                ymin=-0.043, ymax=0.04,
                                xtick={0,0.25,...,2},
                                ytick={-0.04,-0.03,...,0.04},
				ticklabel style={
					/pgf/number format/fixed,
					/pgf/number format/precision=5
				},
				scaled y ticks=false,
                                legend pos=south east,
                                ymajorgrids=true,
                                xmajorgrids=true,
                                every axis plot/.append style={thick},
				legend columns=2,
                        ]
				\addplot[blue, smooth]         table[col sep=comma,header=false,x index=0,y index=1]{data/omib_sim/data_omib_sim_short.csv};
				\addlegendentry{$\omega(t)$}
				\addplot[blue, smooth, forget plot]         table[col sep=comma,header=false,x index=0,y index=1, forget plot]{data/omib_sim/data_omib_sim_noshort.csv};
				\addplot[red,  smooth] table[col sep=comma,header=false,x index=0,y index=1]{data/omib_sim/data_omib_sim_shortquasi.csv};
				\addlegendentry{$\omega(t)$ (quasi.)}
				\addplot[red,  smooth, forget plot] table[col sep=comma,header=false,x index=0,y index=1]{data/omib_sim/data_omib_sim_noshortquasi.csv};
                        \coordinate (c1) at (axis cs:0  ,-0.043);
                        \coordinate (c2) at (axis cs:0.5,-0.043);
                        \end{axis}
%
                        \begin{axis}[
                                name = ax_zoomed_start,
                                at={($(ax_main.north east)-(0.9\columnwidth,1.75/1.618*\columnwidth)$)},
                                width = 0.9*1\columnwidth,
                                height = 0.9*1/1.618*\columnwidth,
                                xmin=0, xmax=0.5,
                                ymin=-0.043, ymax=0.04,
                                xtick={0,0.05,...,0.5},
        			xlabel={Time (s)},
                                ytick={-0.04,-0.03,...,0.04},
				ticklabel style={
					/pgf/number format/fixed,
					/pgf/number format/precision=5
				},
				scaled y ticks=false,
                                ymajorgrids=true,
                                xmajorgrids=true,
                                every axis plot/.append style={thick},
                        ]
				\addplot[blue, smooth]         table[col sep=comma,header=false,x index=0,y index=1]{data/omib_sim/data_omib_sim_short.csv};
				\addplot[blue, smooth]         table[col sep=comma,header=false,x index=0,y index=1]{data/omib_sim/data_omib_sim_noshort.csv};
				\addplot[red,  smooth] table[col sep=comma,header=false,x index=0,y index=1]{data/omib_sim/data_omib_sim_noshortquasi.csv};
				\addplot[red,  smooth] table[col sep=comma,header=false,x index=0,y index=1]{data/omib_sim/data_omib_sim_shortquasi.csv};
                        \end{axis}
                        % draw dashed lines from rectangle in first axis to corners of second
                        \draw [gray,dashed] (c1) -- (ax_zoomed_start.north west);
                        \draw [gray,dashed] (c2) -- (ax_zoomed_start.north east);
                \end{tikzpicture}
        \caption
[Frequency signals from OMIB system fault simulation.]
{Frequency signals from OMIB system fault simulation. In blue, the result of simulation using the ``complete models'' \eqref{eq:omib_classical_total_model} and \eqref{eq:omib_classical_quasi_model} and in red the result of the quasi-static models \eqref{eq:omib_classical_short} and \eqref{eq:omib_classical_short_quasistatic}.}
        \label{fig:omibsim_freq}
        \end{center}
\end{figure}
% >>>

% ID CURVES <<<
\begin{figure}
        \begin{center}
                \begin{tikzpicture}
                        \begin{axis}[
				name = ax_main,
                                width = 0.9*\columnwidth,
                                height = 0.9*1/1.618*\columnwidth,
                                title={OMIB simulation direct current component $I_d$ signal},
                                xlabel={Time (s)},
                                ylabel={$I_d$ (pu)},
                                xmin=0, xmax=2,
                                ymin=-1.1, ymax=3.5,
                                xtick={0,0.25,...,2},
                                ytick={-1,0,...,3},
                                legend pos=south east,
                                ymajorgrids=true,
                                xmajorgrids=true,
                                every axis plot/.append style={thick},
				legend columns=2,
                        ]
				\addplot[blue, smooth] table[col sep=comma,header=false,x index=0,y index=3, forget plot]{data/omib_sim/data_omib_sim_short.csv};
				\addplot[blue, smooth] table[col sep=comma,header=false,x index=0,y index=3]{data/omib_sim/data_omib_sim_noshort.csv};
				\addlegendentry{$I_d(t)$}
				\addplot[red,  smooth] table[col sep=comma,header=false,x index=0,y index=3, forget plot]{data/omib_sim/data_omib_sim_shortquasi.csv};
				\addplot[red,  smooth] table[col sep=comma,header=false,x index=0,y index=3]{data/omib_sim/data_omib_sim_noshortquasi.csv};
				\addlegendentry{$I_d(t)$ (quasi.)}
                        \coordinate (c1) at (axis cs:0  ,-1.1);
                        \coordinate (c2) at (axis cs:0.6,-1.1);
                        \end{axis}
%
                        \begin{axis}[
                                name = ax_zoomed_start,
                                at={($(ax_main.north east)-(0.9\columnwidth,1.75/1.618*\columnwidth)$)},
                                width = 0.9*1\columnwidth,
                                height = 0.9*1/1.618*\columnwidth,
                                xmin=0, xmax=0.6,
                                ymin=-1.1, ymax=3.5,
                                xtick={0,0.06,...,0.6},
                                ytick={-1,0,...,3},
				ticklabel style={
					/pgf/number format/fixed,
					/pgf/number format/precision=5
				},
				scaled y ticks=false,
        			xlabel={Time (s)},
                                ymajorgrids=true,
                                xmajorgrids=true,
                                every axis plot/.append style={thick},
                        ]
				\addplot[blue, smooth] table[col sep=comma,header=false,x index=0,y index=3, forget plot]{data/omib_sim/data_omib_sim_short.csv};
				\addplot[blue, smooth] table[col sep=comma,header=false,x index=0,y index=3]{data/omib_sim/data_omib_sim_noshort.csv};
				\addplot[red,  smooth] table[col sep=comma,header=false,x index=0,y index=3, forget plot]{data/omib_sim/data_omib_sim_shortquasi.csv};
				\addplot[red,  smooth] table[col sep=comma,header=false,x index=0,y index=3]{data/omib_sim/data_omib_sim_noshortquasi.csv};
                        \end{axis}
                        % draw dashed lines from rectangle in first axis to corners of second
                        \draw [gray,dashed] (c1) -- (ax_zoomed_start.north west);
                        \draw [gray,dashed] (c2) -- (ax_zoomed_start.north east);
                \end{tikzpicture}
        \caption
[Direct component of bus current signals from OMIB system fault simulation.]
{Direct component of bus current signals from OMIB system fault simulation. In blue, the result of simulation using the ``complete models'' \eqref{eq:omib_classical_total_model} and \eqref{eq:omib_classical_quasi_model} and in red the result of the quasi-static models \eqref{eq:omib_classical_short} and \eqref{eq:omib_classical_short_quasistatic}.}
        \label{fig:omibsim_id}
        \end{center}
\end{figure}
% >>>

% IQ CURVES <<<
\begin{figure}
        \begin{center}
                \begin{tikzpicture}
                        \begin{axis}[
				name = ax_main,
                                width = 0.9*\columnwidth,
                                height = 0.9*1/1.618*\columnwidth,
                                title={OMIB simulation quadrature current component $I_q$ signal},
                                xlabel={Time (s)},
                                ylabel={$I_q$ (pu)},
                                xmin=0, xmax=2,
                                ymin=-4.5, ymax=2.8,
                                xtick={0,0.25,...,2},
                                ytick={-4,-3,...,2},
				scaled y ticks=false,
                                legend pos=south east,
                                ymajorgrids=true,
                                xmajorgrids=true,
                                every axis plot/.append style={thick},
				legend columns=2,
                        ]
				\addplot[blue, smooth] table[col sep=comma,header=false,x index=0,y index=4, forget plot]{data/omib_sim/data_omib_sim_short.csv};
				\addplot[blue, smooth] table[col sep=comma,header=false,x index=0,y index=4]{data/omib_sim/data_omib_sim_noshort.csv};
				\addlegendentry{$I_q(t)$}
				\addplot[red,  smooth] table[col sep=comma,header=false,x index=0,y index=4, forget plot]{data/omib_sim/data_omib_sim_shortquasi.csv};
				\addplot[red,  smooth] table[col sep=comma,header=false,x index=0,y index=4]{data/omib_sim/data_omib_sim_noshortquasi.csv};
				\addlegendentry{$I_q(t)$ (quasi.)}
                        \coordinate (c1) at (axis cs:0  ,-4.5);
                        \coordinate (c2) at (axis cs:0.6,-4.5);
                        \end{axis}
%
                        \begin{axis}[
                                name = ax_zoomed_start,
                                at={($(ax_main.north east)-(0.9\columnwidth,1.75/1.618*\columnwidth)$)},
                                width = 0.9*1\columnwidth,
                                height = 0.9*1/1.618*\columnwidth,
                                xmin=0, xmax=0.6,
                                ymin=-4.5, ymax=2.8,
                                xtick={0,0.06,...,0.6},
                                ytick={-4,-3,...,2},
				ticklabel style={
					/pgf/number format/fixed,
					/pgf/number format/precision=5
				},
				scaled y ticks=false,
        			xlabel={Time (ms)},
                                ymajorgrids=true,
                                xmajorgrids=true,
                                every axis plot/.append style={thick},
                          ]
				\addplot[blue, smooth] table[col sep=comma,header=false,x index=0,y index=4, forget plot]{data/omib_sim/data_omib_sim_short.csv};
				\addplot[blue, smooth] table[col sep=comma,header=false,x index=0,y index=4]{data/omib_sim/data_omib_sim_noshort.csv};
				\addplot[red,  smooth] table[col sep=comma,header=false,x index=0,y index=4, forget plot]{data/omib_sim/data_omib_sim_shortquasi.csv};
				\addplot[red,  smooth] table[col sep=comma,header=false,x index=0,y index=4]{data/omib_sim/data_omib_sim_noshortquasi.csv};
                        \end{axis}
                        % draw dashed lines from rectangle in first axis to corners of second
                        \draw [gray,dashed] (c1) -- (ax_zoomed_start.north west);
                        \draw [gray,dashed] (c2) -- (ax_zoomed_start.north east);
                \end{tikzpicture}
        \caption
[Quadrature component of bus current signals from OMIB system fault simulation.]
{Quadrature component of bus current signals from OMIB system fault simulation. In blue, the result of simulation using the ``complete models'' \eqref{eq:omib_classical_total_model} and \eqref{eq:omib_classical_quasi_model} and in red the result of the quasi-static models \eqref{eq:omib_classical_short} and \eqref{eq:omib_classical_short_quasistatic}.}
        \label{fig:omibsim_iq}
        \end{center}
\end{figure}
% >>>

% VD CURVES <<<
\begin{figure}
        \begin{center}
                \begin{tikzpicture}
                        \begin{axis}[
				name = ax_main,
                                width = 0.9*\columnwidth,
                                height = 0.9*1/1.618*\columnwidth,
                                title={OMIB simulation direct current component $I_d$ signal},
                                xlabel={Time (s)},
                                ylabel={$V_d$ (pu)},
                                xmin=0, xmax=2,
                                ymin=-0.08, ymax=1.15,
                                xtick={0,0.25,...,2},
                                ytick={0,0.2,...,1},
				scaled y ticks=false,
                                legend pos=south east,
                                ymajorgrids=true,
                                xmajorgrids=true,
                                every axis plot/.append style={thick},
				legend columns=2,
                        ]	
				\addplot[blue, smooth] table[col sep=comma,header=false,x index=0,y index=5, forget plot]{data/omib_sim/data_omib_sim_noshort.csv};
				\addplot[blue, smooth] table[col sep=comma,header=false,x index=0,y index=5]{data/omib_sim/data_omib_sim_short.csv};
				\addlegendentry{$V_d(t)$}
				\addplot[red, smooth] table[col sep=comma,header=false,x index=0,y index=5, forget plot]{data/omib_sim/data_omib_sim_shortquasi.csv};
				\addplot[red, smooth] table[col sep=comma,header=false,x index=0,y index=5]{data/omib_sim/data_omib_sim_noshortquasi.csv};
				\addlegendentry{$V_d(t)$ (quasi.)}
                        \end{axis}
%
                        \begin{axis}[
                                name = ax_zoomed_start,
                                at={($(ax_main.north west)-(0,1.7/1.618*\columnwidth)$)},
                                width = 0.9*1\columnwidth,
                                height = 0.9*1/1.618*\columnwidth,
                                xmin=0, xmax=2,
                                ymin=-0.08, ymax=0.23,
                                xtick={0,0.25,...,2},
                                ytick={0.05,0,...,0.25},
				scaled y ticks=false,
        			xlabel={Time (s)},
                                ylabel={$V_q$ (pu)},
                                ymajorgrids=true,
                                xmajorgrids=true,
                                every axis plot/.append style={thick},
				legend columns=2,
                                legend pos=south east,
				ticklabel style={
					/pgf/number format/fixed,
					/pgf/number format/precision=5
				},
                          ]
				\addplot[blue, smooth] table[col sep=comma,header=false,x index=0,y index=6, forget plot]{data/omib_sim/data_omib_sim_noshort.csv};
				\addplot[blue, smooth] table[col sep=comma,header=false,x index=0,y index=6]{data/omib_sim/data_omib_sim_short.csv};
				\addlegendentry{$V_q(t)$}
				\addplot[red, smooth] table[col sep=comma,header=false,x index=0,y index=6, forget plot]{data/omib_sim/data_omib_sim_shortquasi.csv};
				\addplot[red, smooth] table[col sep=comma,header=false,x index=0,y index=6]{data/omib_sim/data_omib_sim_noshortquasi.csv};
				\addlegendentry{$V_q(t)$ (quasi.)}
                        \end{axis}
                
		\end{tikzpicture}
        \caption
[Direct and quadrature components of terminal voltage signals from OMIB system fault simulation.]
{Direct and quadrature components of terminal voltage signals from OMIB system fault simulation. In blue, the result of simulation using the ``complete models'' \eqref{eq:omib_classical_total_model} and \eqref{eq:omib_classical_quasi_model} and in red the result of the quasi-static models \eqref{eq:omib_classical_short} and \eqref{eq:omib_classical_short_quasistatic}.}
        \label{fig:omibsim_v}
        \end{center}
\end{figure}
% >>>

% P CURVES <<<
\begin{figure}
        \begin{center}
                \begin{tikzpicture}
                        \begin{axis}[
				name = ax_main,
                                width = 0.9*\columnwidth,
                                height = 0.9*1/1.618*\columnwidth,
                                title={Active power exported by the machine at the terminal bus},
                                xlabel={Time (s)},
                                ylabel={$P$ (pu)},
                                xmin=0, xmax=2,
                                ymin=-0.7, ymax=3.3,
                                xtick={0,0.25,...,2},
                                ytick={-0.5,0,...,3},
				scaled y ticks=false,
                                legend pos=south east,
                                ymajorgrids=true,
                                xmajorgrids=true,
                                every axis plot/.append style={thick},
				legend columns=2,
                        ]
				\addplot[blue, smooth] table[col sep=comma,header=false,x index=0,y index=7, forget plot]{data/omib_sim/data_omib_sim_noshort.csv};
				\addplot[blue, smooth] table[col sep=comma,header=false,x index=0,y index=7]{data/omib_sim/data_omib_sim_short.csv};
				\addlegendentry{$P(t)$}
				\addplot[red, smooth] table[col sep=comma,header=false,x index=0,y index=7, forget plot]{data/omib_sim/data_omib_sim_shortquasi.csv};
				\addplot[red, smooth] table[col sep=comma,header=false,x index=0,y index=7]{data/omib_sim/data_omib_sim_noshortquasi.csv};
				\addlegendentry{$P(t)$ (quasi.)}
                        \coordinate (c1) at (axis cs:0  ,-0.7);
                        \coordinate (c2) at (axis cs:0.6,-0.7);
                        \end{axis}
%
                        \begin{axis}[
                                name = ax_zoomed_start,
                                at={($(ax_main.north east)-(0.9\columnwidth,1.75/1.618*\columnwidth)$)},
                                width = 0.9*1\columnwidth,
                                height = 0.9*1/1.618*\columnwidth,
                                xmin=0, xmax=0.6,
                                ymin=-0.7, ymax=3.3,
                                xtick={0,0.06,...,0.6},
                                ytick={-0.5,0,...,3},
				scaled y ticks=false,
        			xlabel={Time (ms)},
                                ymajorgrids=true,
                                xmajorgrids=true,
                                every axis plot/.append style={thick},
				ticklabel style={
					/pgf/number format/fixed,
					/pgf/number format/precision=5
				},
                          ]
				\addplot[blue, smooth] table[col sep=comma,header=false,x index=0,y index=7, forget plot]{data/omib_sim/data_omib_sim_noshort.csv};
				\addplot[blue, smooth] table[col sep=comma,header=false,x index=0,y index=7]{data/omib_sim/data_omib_sim_short.csv};
				\addplot[red, smooth] table[col sep=comma,header=false,x index=0,y index=7, forget plot]{data/omib_sim/data_omib_sim_shortquasi.csv};
				\addplot[red, smooth] table[col sep=comma,header=false,x index=0,y index=7]{data/omib_sim/data_omib_sim_noshortquasi.csv};
                        \end{axis}
                        % draw dashed lines from rectangle in first axis to corners of second
                        \draw [gray,dashed] (c1) -- (ax_zoomed_start.north west);
                        \draw [gray,dashed] (c2) -- (ax_zoomed_start.north east);
                \end{tikzpicture}
        \caption
[Active power signals from OMIB system fault simulation.]
{Active power signals from OMIB system fault simulation. In blue, the result of simulation using the ``complete models'' \eqref{eq:omib_classical_total_model} and \eqref{eq:omib_classical_quasi_model} and in red the result of the quasi-static models \eqref{eq:omib_classical_short} and \eqref{eq:omib_classical_short_quasistatic}.}
        \label{fig:omibsim_p}
        \end{center}
\end{figure}
% >>>

% Q CURVES <<<
\begin{figure}
        \begin{center}
                \begin{tikzpicture}
                        \begin{axis}[
				name = ax_main,
                                width = 0.9*\columnwidth,
                                height = 0.9*1/1.618*\columnwidth,
                                title={Reactive power exported by the machine at the terminal bus},
                                xlabel={Time (s)},
                                ylabel={$Q$ (pu)},
                                xmin=0, xmax=2,
                                ymin=-2.1, ymax=0.8,
                                xtick={0,0.25,...,2},
                                ytick={-2,-1.5,...,0.5},
				scaled y ticks=false,
                                legend pos=south east,
                                ymajorgrids=true,
                                xmajorgrids=true,
                                every axis plot/.append style={thick},
				legend columns=2,
                        ]
				\addplot[blue, smooth] table[col sep=comma,header=false,x index=0,y index=8, forget plot]{data/omib_sim/data_omib_sim_noshort.csv};
				\addplot[blue, smooth] table[col sep=comma,header=false,x index=0,y index=8]{data/omib_sim/data_omib_sim_short.csv};
				\addlegendentry{$Q(t)$}
				\addplot[red, smooth] table[col sep=comma,header=false,x index=0,y index=8, forget plot]{data/omib_sim/data_omib_sim_shortquasi.csv};
				\addplot[red, smooth] table[col sep=comma,header=false,x index=0,y index=8]{data/omib_sim/data_omib_sim_noshortquasi.csv};
				\addlegendentry{$Q(t)$ (quasi.)}
                        \coordinate (c1) at (axis cs:0  ,-2.1);
			\coordinate (c2) at (axis cs:0.6,-2.1);
                        \end{axis}
%
                        \begin{axis}[
                                name = ax_zoomed_start,
                                at={($(ax_main.north east)-(0.9\columnwidth,1.75/1.618*\columnwidth)$)},
                                width = 0.9*1\columnwidth,
                                height = 0.9*1/1.618*\columnwidth,
                                xmin=0, xmax=0.6,
                                ymin=-2.1, ymax=0.8,
                                xtick={0,0.06,...,0.6},
                                ytick={-2,-1.5,...,0.5},
				scaled y ticks=false,
        			xlabel={Time (ms)},
                                ymajorgrids=true,
                                xmajorgrids=true,
                                every axis plot/.append style={thick},
				ticklabel style={
					/pgf/number format/fixed,
					/pgf/number format/precision=5
				},
                          ]
				\addplot[blue, smooth] table[col sep=comma,header=false,x index=0,y index=8, forget plot]{data/omib_sim/data_omib_sim_noshort.csv};
				\addplot[blue, smooth] table[col sep=comma,header=false,x index=0,y index=8]{data/omib_sim/data_omib_sim_short.csv};
				\addplot[red, smooth] table[col sep=comma,header=false,x index=0,y index=8, forget plot]{data/omib_sim/data_omib_sim_shortquasi.csv};
				\addplot[red, smooth] table[col sep=comma,header=false,x index=0,y index=8]{data/omib_sim/data_omib_sim_noshortquasi.csv};
                        \end{axis}
                        % draw dashed lines from rectangle in first axis to corners of second
                        \draw [gray,dashed] (c1) -- (ax_zoomed_start.north west);
                        \draw [gray,dashed] (c2) -- (ax_zoomed_start.north east);
                \end{tikzpicture}
        \caption
[Reactive power signal from OMIB system fault simulation.]
{Reactive power signal from OMIB system fault simulation. In blue, the result of simulation using the ``complete models'' \eqref{eq:omib_classical_total_model} and \eqref{eq:omib_classical_quasi_model} and in red the result of the quasi-static models \eqref{eq:omib_classical_short} and \eqref{eq:omib_classical_short_quasistatic}.}
        \label{fig:omibsim_q}
        \end{center}
\end{figure}
% >>>

% CURRENT TIME CURVES <<<
\begin{figure}
        \begin{center}
                \begin{tikzpicture}
                        \begin{axis}[
                                name = ax_main,
                                width = 0.9*\columnwidth,
                                height = 0.9*1/1.618*\columnwidth,
                                title={Time-domain current signals},
                                xlabel={Time (s)},
                                ylabel={$i$ (pu)},
                                xmin=0, xmax=2,
                                ymin=-3, ymax=3,
                                xtick={0,0.25,...,2},
                                ytick={-3,-2,...,3},
                                scaled y ticks=false,
                                legend pos=south east,
                                ymajorgrids=true,
                                xmajorgrids=true,
                                every axis plot/.append style={thick},
                                legend columns=2,
                        ]
                                \addplot[blue, smooth] table[col sep=comma,header=false,x index=0,y index=9, forget plot]{data/omib_sim/data_omib_sim_noshort.csv};
                                \addplot[blue, smooth] table[col sep=comma,header=false,x index=0,y index=9]{data/omib_sim/data_omib_sim_short.csv};
                                \addlegendentry{$i(t)$}
                                \addplot[red, smooth] table[col sep=comma,header=false,x index=0,y index=9, forget plot]{data/omib_sim/data_omib_sim_shortquasi.csv};
                                \addplot[red, smooth] table[col sep=comma,header=false,x index=0,y index=9]{data/omib_sim/data_omib_sim_noshortquasi.csv};
                                \addlegendentry{$i(t)$ (quasi.)}
                        \coordinate (c1) at (axis cs:0.25,-0.7);
                        \coordinate (c2) at (axis cs:0.4 ,-0.7);
			\draw[draw=gray] (axis cs:0.25,-0.7) rectangle (axis cs: 0.4,0.7);
                        \end{axis}
%
                        \begin{axis}[
                                name = ax_zoomed_start,
                                at={($(ax_main.north east)-(0.9\columnwidth,1.75/1.618*\columnwidth)$)},
                                width = 0.9*1\columnwidth,
                                height = 0.9*1/1.618*\columnwidth,
                                xmin=0.25, xmax=0.4,
                                ymin=-0.7, ymax=0.7,
                                xtick={0.25,0.275,...,0.4},
                                ytick={-0.6,-0.4,...,0.6},
                                scaled y ticks=false,
                                xlabel={Time (s)},
                                ymajorgrids=true,
                                xmajorgrids=true,
                                every axis plot/.append style={thick},
                                ticklabel style={
                                        /pgf/number format/fixed,
                                        /pgf/number format/precision=5
                                },
                          ]
                                \addplot[blue, smooth] table[col sep=comma,header=false,x index=0,y index=9, forget plot]{data/omib_sim/data_omib_sim_noshort.csv};
                                \addplot[blue, smooth] table[col sep=comma,header=false,x index=0,y index=9]{data/omib_sim/data_omib_sim_short.csv};
                                \addplot[red, smooth] table[col sep=comma,header=false,x index=0,y index=9, forget plot]{data/omib_sim/data_omib_sim_shortquasi.csv};
                                \addplot[red, smooth] table[col sep=comma,header=false,x index=0,y index=9]{data/omib_sim/data_omib_sim_noshortquasi.csv};
                        \end{axis}
                        % draw dashed lines from rectangle in first axis to corners of second
                        \draw [gray,dashed] (c1) -- (ax_zoomed_start.north west);
                        \draw [gray,dashed] (c2) -- (ax_zoomed_start.north east);
                \end{tikzpicture}
        \caption
[Bus current time domain signal reconstructed from the Dynamic Phasor $I_d + jI_q$ of figures \ref{fig:omibsim_id} and \ref{fig:omibsim_iq}.]
{Bus current time domain signal reconstructed from the Dynamic Phasor $I_d + jI_q$ of figures \ref{fig:omibsim_id} and \ref{fig:omibsim_iq}. In blue, the result of simulation using the ``complete models'' \eqref{eq:omib_classical_total_model} and \eqref{eq:omib_classical_quasi_model} and in red the result of the quasi-static models \eqref{eq:omib_classical_short} and \eqref{eq:omib_classical_short_quasistatic}.}
        \label{fig:omibsim_curr}
        \end{center}
\end{figure}
% >>>

%-------------------------------------------------
\section{A transient Power System modelling framework using Dynamic Phasor theory} \label{sec:admittance_modelling_dp}%<<<1

	It is standard in Power System stability studies that the electrical grid to which the generators and agents are coupled is modelled as a constant impedance nodal matrix, where each bus represents a node and the vertixes of the graph represent the transmission lines. The most common approach to the modelling problem is the structure-preserving model, where an electrical grid is represented by a admittance matrix $\mathbf{Y}$ such that the current injection in the buses is related to bus voltages by the equation

\begin{equation} \mathbf{I} = \mathbf{Y}\mathbf{E}, \label{eq:gridConductance} \end{equation}

	\noindent where $\mathbf{I}\in \mathbb{C}^{n}$ is a vector, $n$ being the number of buses in the system, which k-th component $I_k$ is the current injection in the k-th bus, $\mathbf{E}\in \mathbb{C}^{n}$ is the bus voltages vector and $\mathbf{Y}\in\mathbb{C}^{n\times n}$ is the admittance matrix. This matrix is seldomly divided into its imaginary and real parts, yielding $\mathbf{G}$ and $\mathbf{B}$, both in $\mathbb{R}^{n\times n}$, such that $\mathbf{Y} = \mathbf{G} + j\mathbf{B}$.

%-------------------------------------------------
\subsection{The Unified Nodal Model for transmission systems}\label{subsec:unified_model} %<<<2

	Here we propose a Dynamic Phasor expansion of this model. We want to prove that the relationship \eqref{eq:gridConductance} is also possible in a Dynamic Phasor framework using Dynamic Phasor functionals, that is,

\begin{equation} \mathbf{I} = \mathbf{Y}\left[\mathbf{E}\right], \label{eq:gridConductance_dynamic}\end{equation}

	\noindent where $\mathbf{I,E}\in\left[\mathbb{R}\to\mathbb{C}^n\right]$ are the Dynamic Phasors of current injections on buses and voltages of the buses and $\mathbf{Y}\in\dpS^{(n\times n)}$ is the matrix of DPFs associates with the grid. To do this, we expand the Unified Nodal Model developed in \cite{Monticelli1999}, called so because it encompasses line impedance, shunt reactances and transformer effects onto the transmission line. Using the Dynamic Phasor Theory of chapter \ref{chapter:dynamic_phasor_theory}, we adopt the synchronous frequency $\omega_0$ as the apparent frequency for the Dynamic Phasor Transform. 

% UNIFIED NODAL MODEL <<<
\begin{figure}[h]
\centering
        \begin{tikzpicture}[american,scale=1.2, transform shape,line width=0.75, cute inductors, >={Stealth[inset=0mm,length=1.5mm,angle'=50]}]
		\node (busk) at (0,0) {};
		\node (busklabel) at ([shift=({0,1})]busk) {$E_k$};
		\draw [line width=1mm] ([shift=({0,0.75})]busk.center) -- ++(0,-1.5);
		\node [shape=circle,draw,inner sep=1pt] at (-0.5,0) {$k$};
		\draw (busk.center) to[short, f=$I_{km}$] ++(1.5,0) to[voosource, sources/scale=1.25, name=ktransf] ++(1.25,0) to [short, f=$I_{km}'$] ++(1.5,0) coordinate (kmconn) to[generic, l=$\mathbf{y}_{km}^{sh}$] ++(0,-2) node[tlground] {} ;
%
		\node (tkmlabel) at ([shift=({0,0.2})]ktransf.north) {$1:\mathbf{t}_{km}$};
%
		\node [circle, fill=black, inner sep=0mm, minimum size=1.5mm] at (kmconn) {};
%
		\node [above=2mm of kmconn] (kmconnlabel) {$E_k'$};
		\draw (kmconn) to[generic, l=$\mathbf{y}_{km}$] ++(3,0) coordinate (mkconn);
		\draw (mkconn) to[generic, l=$\mathbf{y}_{mk}^{sh}$] ++(0,-2) node[tlground] {};
%
		\node [circle, fill=black, inner sep=0mm, minimum size=1.5mm] at (mkconn) {};
		\node [above=2mm of mkconn] (mkconnlabel) {$E_m'$};
%
		\draw (mkconn) to[voosource, sources/scale=1.25, name=mtransf] ++(2.5,0) coordinate (busm);
%
		\node (tmklabel) at ([shift=({0,0.2})]mtransf.north) {$\mathbf{t}_{mk}:1$};
%
		\draw [line width=1mm] ([shift=({0,0.75})]busm.center) -- ++(0,-1.5);
		\node (busmlabel) at ([shift=({0,1})]busm) {$E_m$};
		\node [shape=circle,draw,inner sep=1pt] at ([shift=({0.5,0})]busm) {$m$};
        \end{tikzpicture}
	\caption
[Unified transmission line ``pi model'' as devised by \cite{Monticelli1999}.]
{Unified transmission line ``pi model'' as devised by \cite{Monticelli1999}. The figure shows the transmission line between the k-th and the m-th bus of the system, equationing $I_{km}$, that is, the contribution of the m-th bus to the total current draw from the k-th bus of the system considering effects of transformers, line serial and shunt impedances. The figure does not show bus shunt impedances, which will be dealt with later in the equationing.}
	\label{fig:monticelli_transmission_line}
\end{figure} %>>>

	Figure \ref{fig:monticelli_transmission_line} shows the schematic diagram of a transmission line between the k-th and m-th buses in a hypothetical grid. In the figure, $E_k = V_ke^{j\theta_k}$ and $E_m = V_me^{j\theta_m}$ are the dynamic phasors of the voltages of the buses, $V_k$ and $V_m$ being their absolute value and $\theta_k$ and $\theta_m$ their complex angles, and $I_{km}$ is the dynamic phasor of the current that flows from bus $k$ to the transmission line (which is not the same as $I_{mk}$). $\mathbf{y}_{km} = \mathbf{g}_{km} + j\mathbf{b}_{km}$ is the admittance functional of the transmission line, $\mathbf{y}^{sh}_{km}$ and $\mathbf{y}^{sh}_{mk}$ are the line shunt admittance functionals; these are comprised of a shunt conductance $\mathbf{g}^{sh}_{km}$ which accounts for current leakages and a susceptance $j\mathbf{b}^{sh}_{km}$ being the line charge capacitance susceptance. In most Power System studies, the conductance is neglected. It is crucial not to mistake these shunt admittances for the shunt admittances attached to buses; these will be dealt with in the equationing later.

	The transformers are modelled by an operator $\mathbf{t}$ such that the voltage of the primary coil $V_1$ and the voltage of the secondary coil $V_2$ are related by $V_2 = \mathbf{t}\left[V_1\right]$. The most widely adopted model is that the transformer operator is given by $\mathbf{t} \left[V_1\right] = a e^{j\varphi}V_1$, where $a$ is the turns ratio (positive real) while $\varphi$ is voltage angle deviation caused by the transformer. It is also supposes that the transformers are lossless, that is, the apparent complex power injected on the primary coil is equal to the apparent complex power output by the secondary coil.

	Therefore, $\mathbf{t}_{km} = a_{km}e^{j\varphi_{km}}$ is the voltage ratio operator of the line transformer from the k-th bus to the m-th, and analogously with the transformer from the m-th to the k-th, yielding

\begin{equation} E'_k = \mathbf{t}_{km}\left[E_k\right] = a_{km}e^{j\varphi_{km}} E_k,\ I'_{km} = \dfrac{1}{a_{km}}e^{j\varphi_{km}} I_{km} \end{equation} 

	\noindent and analogously with $E_m,E'_m,I_{mk}$ and $I_{mk}$. Applying Kirchoff's Current Law for Dynamic Phasors on node $E_k'$ yields

\begin{equation} I_{km}' = \mathbf{y}_{km}^{sh}\left[E'_k\right] + \mathbf{y}_{km}\left[E'_k - E'_m\right] \end{equation}

	\noindent and applying the transformer models to this equation

\begin{equation} \dfrac{1}{a_{km}}e^{j\varphi_{km}} I_{km} = \mathbf{y}_{km}^{sh}\left[ a_{km}e^{j\varphi_{km}}E_k\right] + \mathbf{y}_{km}\left[a_{km}e^{j\varphi_{km}}E_k - a_{mk}e^{j\varphi_{mk}}E_m\right] .\end{equation}

	Using the linearity of Dynamic Phasor Functionals,

\begin{align}
	I_{km} &= a_{km}e^{-j\varphi_{km}} \left\{ \mathbf{y}_{km}^{sh}\left[ a_{km}e^{j\varphi_{km}}E_k\right] + \mathbf{y}_{km}\left[a_{km}e^{j\varphi_{km}}E_k - a_{mk}e^{j\varphi_{mk}}E_m\right]\right\} \nonumber\\[3mm]
	&= \mathbf{y}_{km}^{sh}\left[ a_{km}e^{-j\varphi_{km}} a_{km}e^{j\varphi_{km}}E_k\right] + \mathbf{y}_{km}\left[a_{km}e^{-j\varphi_{km}} a_{km}e^{j\varphi_{km}}E_k - a_{km}e^{-j\varphi_{km}} a_{mk}e^{j\varphi_{mk}}E_m\right] \nonumber\\[3mm]
	&= \mathbf{y}_{km}^{sh}\left[ a_{km}^2 E_k\right] + \mathbf{y}_{km}\left[a_{km}^2 E_k - a_{km}a_{mk}e^{j\left(\varphi_{mk}-\varphi_{km}\right)}E_m\right]
\end{align}

	\noindent and using linear combinations of DPFs,

\begin{equation}
	I_{km} = \left[a_{km}^2\left(\mathbf{y}_{km}^{sh} + \mathbf{y}_{km}\right)\right] \left[ E_k\right] - a_{km}a_{mk}e^{j\left(\varphi_{mk}-\varphi_{km}\right)} \mathbf{y}_{km}\left[E_m\right]
\end{equation}
	
	We can also use this equation to calculate the current injection in bus k. Suppose that this bus has a shunt load admittance $\mathbf{y}^{sh}_k$ as in figure \ref{fig:busSum}. Denote $\Omega_k$ as the set of buses adjacent to bus $k$. By the Kirchoff Current Law,

% BUS CURRENT SUM <<<
\begin{figure}
\centering
        \begin{tikzpicture}[american,scale=1.2,transform shape,line width=0.75, cute inductors, >={Stealth[inset=0mm,length=2mm,angle'=50]}]
		\ctikzset{quadpoles/transformer core/width=0.5, quadpoles/transformer core/inner=1, quadpoles/transformer core/height=1}
		\node (busk) at (0,0) {};
		\draw [<-] ([shift=({0.3,0})]busk) -- ++(-1.5,0) node[left] {$I_k$};
		\node (busklabel) at ([shift=({0,1.5})]busk) {$E_k$};
		\draw [line width=1mm] ([shift=({0,1})]busk.center) -- ++(0,-2);
		\node [shape=circle,draw,inner sep=1pt] at (0,-1.5) {$k$};
		\draw ([shift=({0,-0.75})]busk.center) to[short] ++(1,0) to [short, f<=$I_k^{sh}$] ++(0,-1) to[generic, l=$\mathbf{y}_k^{sh}$] ++(0,-2) node [tlground] {} ;
		\draw ([shift=({0,-0.25})]busk.center) to[short] ++(3,0);
		\draw ([shift=({0, 0.25})]busk.center) to[short] ++(2,0) to[short, f=$I_{km}$] ++(1,0.5);
		\draw ([shift=({0, 0.75})]busk.center) to[short] ++(1,0) to[short] ++(0,1);
	\end{tikzpicture}
		\caption{Schematic diagram of the current draw components and current input on a generic k-th bus of the system, considering the bus shunt conductance.}
		\label{fig:busSum}
\end{figure}%>>>

\begin{gather}
	I_k + I^{sh}_k = \sum\limits_{m\in\Omega_k} I_{km} \nonumber\\[3mm]
%
	I_k - \mathbf{y}_{k}^{sh}\left[E_k\right] = \left[\sum\limits_{m\in\Omega_k} a_{km}^2\left( \mathbf{y}_{km} + \mathbf{y}^{sh}_{km}\right)\right] \left[E_k\right] + \sum\limits_{m\in\Omega_k} \left(- a_{km}a_{mk}e^{j\left(\varphi_{mk}-\varphi_{km}\right)}\mathbf{y}_{km}\right)\left[E_m\right]
\end{gather}

	Arranging this equation in matricial form yields the sought grid model \eqref{eq:gridConductance_dynamic}, where:

\begin{equation}
\left\{\begin{array}{l}
	\mathbf{Y}_{kk} = \mathbf{y}_{k}^{sh} + \sum\limits_{m\in\Omega_k} a_{km}^2\left(\mathbf{y}_{km} + \mathbf{y}^{sh}_{km}\right) \\[5mm]
%
	\mathbf{Y}_{km} = - a_{km}a_{mk}e^{j\left(\varphi_{mk}-\varphi_{km}\right)}\mathbf{y}_{km}
\end{array}\right.\label{eq:multimachine_admittance_dynamic}
\end{equation}

	Notably, if the system is at phasorial equilibrium (constant amplitude, frequency and phases) then these equations fall back to the customary admittance equations where the operators $\mathbf{y}_{km}$ and $\mathbf{y}_{km}^{sh}$ become a multiplication by complex numbers.

%-------------------------------------------------
\subsection{Modelling bus loads} %<<<2

	To consider the effects of bus loads, a commonplace technique in Power System studies is to reduce the bus loads to equivalent admittances, in a linear model. Let $S^k_L$ denote the complex power of the load attached to the k-th bus calculated at equilibrium using Power Flow equations; then the equivalent admittance of this load can be calculated as

\begin{equation} Y_L^k = \dfrac{\overline{S_L^k}}{V_k^2} \label{eq:load_eq_impedance} \end{equation}

	\noindent whre the $S_L$ are calculated using Power Flow equations. From this number we can calculate the equivalent DPF of the bus load; for instance, if $Y_L^k$ is of the form $a + jb$, with positive $b$, then the corresponding impedance is

\begin{equation} Z_L^k = \dfrac{1}{Y_L^k} = \dfrac{a}{\left(a^2 + b^2\right)} - j\dfrac{b}{\left(a^2 + b^2\right)} \end{equation}

	\noindent thus equivalent to a resistance of $R_L^k = a/(a^2 + b^2)$ ohms in series with an inductance of $L_L^k = \left\lvert b\right\rvert/[\omega_0(a^2 + b^2)]$ henrys if $b$ is negative or a capacitance of $C_L^k = (a^2 + b^2)/[b\omega_0]$ farads if $b$ is positive; therefore the equivalent Dynamic Impedances are

\begin{equation} \mathbf{Z}_L^k = \left\{\begin{array}{l} L_L^k\dpo + R_L^k\mathbf{I} \text{ if } b < 0; \\[5mm] \dfrac{\mathbf{I}}{C_L^k\dpo} + R_L^k\mathbf{I} \text{ if } b > 0 \end{array}\right.\end{equation}

	\noindent and obviously the admittance operator $\mathbf{Y}_L^k$ is the inverse of the impedance operator. Denote the diagonal matrix of the corresponding DPFs of the loads

\begin{equation} \mathbf{Y_L} = \text{diag}\left(\left\{\mathbf{Y}_L^k\right\}_{k=1}^n\right)\end{equation}

	\noindent and define an ``equivalent'' system where there are no bus loads, but to each bus is added its equivalent load admittance. It can be proven that both cases (the one where loads are modelled as constant power and the simplified one where loads are modelled as constant admittances) have the same power flow solutions. Hence, the dynamic model admittance matrix is taken as $\mathbf{Y_d = Y + Y_L}$, where $\mathbf{Y}$ is the original system admittance matrix and $\mathbf{Y_d}$ is the admittance matrix of the equivalent system where loads were converted to constant shunt admittances.

	Another common technique is called \textit{matrix reduction}: instead of writing the grid model in terms of the $n$ buses, we write it only in term of the buses that have agents. Taking a closer look at the equivalent admittance matrix $\mathbf{Y_d}$, one can, with no loss of generality, admit that the first $1, 2, ..., p$ buses of the total number of buses $n$ have agents attached and the last $m = n - p$ buses have no agents attached; this means that the matrix $\mathbf{Y_d}$ can be divided as in \eqref{sys:ydDivision}.

\begin{gather} %<<<
	\hspace{-6mm}\begin{array}{C{1cm}C{1cm}} p & m \end{array} \nonumber\\[1mm]
	\mathbf{Y_d} = \left[\begin{array}{C{1cm}|C{1cm}}
		\mathbf{Y_1} & \mathbf{Y_2} \\[0.2cm]\hline\\[-3mm]
		\raisebox{3mm}{} \mathbf{Y_3} & \mathbf{Y_4}
	\end{array}\right]
	\begin{array}{C{1cm}}
		p \\[5mm]
		m \end{array} \nonumber\\
	\label{sys:ydDivision}
\end{gather} %>>>

	Where $\mathbf{Y_1}\in\dpS^{(p\times p)}$, $\mathbf{Y_2}\in\dpS^{(p\times m)}$, $\mathbf{Y_3}\in\dpS^{(m\times p)}$, $\mathbf{Y_4}\in\dpS^{(p\times p)}$; denote $\mathbf{I_A}$ as the currents injected into the agent buses (the first $p$ buses), $\mathbf{E_A}$ the complex voltages of these buses and $\mathbf{E_N}$ as the complex voltages of the non-agent buses; the trick is to understand that while the agent buses have current injected on them, the non-agent buses do not and then we can write

\begin{equation}%<<<
	\left[\begin{array}{C{1cm}}
		\raisebox{5mm}{} \mathbf{I_A}\\[0.5cm]
		\raisebox{5mm}{} \mathbf{0}
	\end{array}\right] =
	%
	\left[\begin{array}{C{1cm}C{1cm}}
		\raisebox{5mm}{} \mathbf{Y_1} & \mathbf{Y_2} \\[0.5cm]
		\raisebox{5mm}{} \mathbf{Y_3} & \mathbf{Y_4}
	\end{array}\right]
	%
	\left[\raisebox{15mm}{} \left[\begin{array}{C{1cm}}
		\raisebox{5mm}{} \mathbf{E_A} \\[0.5cm]
		\raisebox{5mm}{} \mathbf{E_N}
	\end{array}\right]\right]
\end{equation}%>>>

	Expanding these matrix equations,

\begin{equation}
\left\{\begin{array}{l}
	\mathbf{I_A} = \mathbf{Y_1}\left[\mathbf{E_A}\right] + \mathbf{Y_2}\left[\mathbf{E_N}\right] \\[5mm]
%
	\mathbf{0} = \mathbf{Y_3}\left[\mathbf{E_A}\right] + \mathbf{Y_4}\left[\mathbf{E_N}\right]
\end{array}\right. .
\end{equation}

	Isolating the last equation yields $\mathbf{E_N} = -\left(\mathbf{Y_4}^{-1}\mathbf{Y_3}\right)\left[\mathbf{E_A}\right]$, proving that indeed the voltages of non-agent buses can be expressed algebraically through agent buses voltages. However, not only this, but these equations also allow for a reduction of the number of equations in the overall power system model: substituting this algebraic equation into the first equation yields

\begin{equation} \mathbf{I_A} = \left(\mathbf{Y_1} - \mathbf{Y_2}\mathbf{Y_4}^{-1}\mathbf{Y_3}\right)\left[\mathbf{E_A}\right]. \end{equation}

	One could, naturally, raise the question if $\mathbf{Y_4}$ is singular or not, and under which circumstances. It can be also proven, by graph theory, that this matrix will always have an inverse as long as no agent buses as islanded, that is, all agent buses are connected to at least one bus in the system through a finite admittance -- which is true by construction because the system is supposed entire. Denote the \textit{reduced matrix} $\mathbf{Y_r}$ as

\begin{equation} \mathbf{Y_r} = \mathbf{Y_1} - \mathbf{Y_2}\mathbf{Y_4}^{-1}\mathbf{Y_3} \Rightarrow \mathbf{I_A} = \mathbf{Y_r}\left[\mathbf{E_A}\right] \end{equation}

	\noindent and the voltages across the buses with no agents attached can at all times be calculated by

\begin{equation} \mathbf{E_N}(t) = -\left(\mathbf{Y_4}^{-1}\mathbf{Y_3}\right)\left[\mathbf{E_A}\right]. \end{equation}

%-------------------------------------------------
\subsection{Multi-machine Power System modelling example: the Kundur two-area system} %<<<2

	Figure \ref{fig:kundur_2} shows the Kundur ``two-area system'' as first described in \cite{kleinFundamentalStudyInterarea1991}.

% KUNDUR TWO AREA CIRCUIT <<<
\begin{figure}[htb!]
\centering
\scalebox{0.8}{
        \begin{tikzpicture}[american,scale=1,transform shape,line width=0.75, cute inductors, voltage shift = 1,>={Stealth[inset=0mm,length=1.5mm,angle'=50]}]
	\ctikzset{/tikz/circuitikz/voltage/distance from node=10mm}
% BUS 1
		\node [shape=vsourcesinshape, rotate=-90] (gen1) at (0,0) {} ;
		\node (g1label) at ([shift=({0,8mm})]gen1.center) {$G_1$};
		\draw (gen1.north) to [short] ++(1,0) node (gen1_terminal) {} ;
		\draw [line width=1mm] ([shift=({0,5mm})]gen1_terminal.center) -- ++(0,-10mm);
		\node [shape=circle,draw,inner sep=1pt] (v1label) at ([shift=({0,8mm})]gen1_terminal.center) {$1$};
% BUS 5
		\draw (gen1_terminal.center) to[voosource] ++(2,0) node (bus5) {};
		\draw [line width=1mm] ([shift=({0,5mm})]bus5.center) -- ++(0,-10mm);
		\node [shape=circle,draw,inner sep=1pt] (v5label) at ([shift=({0,8mm})]bus5.center) {$5$};
% BUS 6
		\draw (bus5.center) to[short] ++(2,0) node (bus6) {};
		\node (bus56line) at ([shift=({0,3mm})] $(bus5)!0.5!(bus6)$) {25km};
		\draw [line width=1mm] ([shift=({0,5mm})]bus6.center) -- ++(0,-10mm);
		\node [shape=circle,draw,inner sep=1pt] (v6label) at ([shift=({0,8mm})]bus6.center) {$6$};
% BUS 2
		\draw ([shift=({0,-3mm})]bus6.center) to [short] ++(-1,0) to[voosource] ++(0,-2) node (bus2) {};
		\draw [line width=1mm] ([shift=({-5mm,0mm})]bus2.center) -- ++(10mm,0);
		\node [shape=circle,draw,inner sep=1pt] (v2label) at ([shift=({8mm,0})]bus2.center) {$2$};
		\node [shape=vsourcesinshape, rotate=-90, transform shape] at ([shift=({0,-1})]bus2) (gen2) {} ;
		\draw (bus2.center) to[short] (gen2.west);
		\node (g2label) at ([shift=({8mm,0})]gen2.center) {$G_2$};
% BUS 7
		\draw  (bus6.center) to[short] ++(2,0) node (bus7) {};
		\draw [line width=1mm] ([shift=({0mm,10mm})]bus7.center) -- ++(0,-20mm);
		\node (bus67line) at ([shift=({0,3mm})] $(bus6)!0.5!(bus7)$) {10km};
		\node [shape=circle,draw,inner sep=1pt] (v7label) at ([shift=({0,13mm})]bus7.center) {$7$};
		\draw[->] ([shift=({0,-8mm})]bus7.center) to [short] ++(-0.5,0) to[short] ++(0,-1) node (load7) {};
		\node (load7label) at ([shift=({0,-3mm})]load7.center) {$L_7$};
		\draw ([shift=({0,-8mm})]bus7.center) to [short] ++(0.5,0) to[cC,l=$C_7$] ++(0,-1.5) node (cap7) {} to [short] ++(0,-0.1) node[tlground] {};
% BUS 8
		\node [right=2 of bus7.center] (bus8) {};
		\draw [line width=1mm] ([shift=({0mm,10mm})]bus8.center) -- ++(0,-20mm);
		\node (bus87line) at ([shift=({0,6mm})] $(bus8)!0.5!(bus7)$) {110km};
		\node [shape=circle,draw,inner sep=1pt] (v8label) at ([shift=({0,13mm})]bus8.center) {$8$};
		\draw ([shift=({0,-3mm})]bus7.center) -| (bus8.center);
		\draw ([shift=({0, 3mm})]bus7.center) -| (bus8.center);
% BUS 9
		\node[right=2 of bus8.center] (bus9) {};
		\draw [line width=1mm] ([shift=({0mm,10mm})]bus9.center) -- ++(0,-20mm);
		\node (bus89line) at ([shift=({0,6mm})] $(bus8)!0.5!(bus9)$) {110km};
		\node[shape=circle,draw,inner sep=1pt] (v8label) at ([shift=({0,13mm})]bus9.center) {$9$};
		\draw ([shift=({0,-3mm})]bus8.center) -| (bus9.center);
		\draw ([shift=({0, 3mm})]bus8.center) -| (bus9.center);
		\draw[->] ([shift=({0,-8mm})]bus9.center) to [short] ++(0.5,0) to[short] ++(0,-1) node (load7) {};
		\node (load7label) at ([shift=({0,-3mm})]load7.center) {$L_9$};
		\draw ([shift=({0,-8mm})]bus9.center) to [short] ++(-0.5,0) to[cC,l_=$C_9$] ++(0,-1.5) node (cap7) {} to [short] ++(0,-0.1) node[tlground] {};
% BUS 10
		\draw (bus9.center) to[short] ++(2,0) node (bus10) {};
		\node (bus910line) at ([shift=({0,3mm})] $(bus9)!0.5!(bus10)$) {10km};
		\draw [line width=1mm] ([shift=({0,5mm})]bus10.center) -- ++(0,-10mm);
		\node[shape=circle,draw,inner sep=1pt] (v9label) at ([shift=({0,9mm})]bus10.center) {$10$};
% BUS 4
		\draw ([shift=({0,-3mm})]bus10.center) to [short] ++(1,0) to[voosource] ++(0,-2) node (bus4) {};
		\draw [line width=1mm] ([shift=({-5mm,0mm})]bus4.center) -- ++(10mm,0);
		\node [shape=circle,draw,inner sep=1pt] (v4label) at ([shift=({-8mm,0})]bus4.center) {$4$};
		\node [shape=vsourcesinshape, rotate=-90, transform shape] at ([shift=({0,-1})]bus4) (gen4) {} ;
		\draw (bus4.center) to[short] (gen4.west);
		\node (g4label) at ([shift=({8mm,0})]gen4.center) {$G_4$};
% BUS 11
		\draw (bus10.center) to[short] ++(2,0) node (bus11) {};
		\node (bus1011line) at ([shift=({0,3mm})] $(bus10)!0.5!(bus11)$) {25km};
		\draw [line width=1mm] ([shift=({0,5mm})]bus11.center) -- ++(0,-10mm);
		\node [shape=circle,draw,inner sep=1pt] (v11label) at ([shift=({0,9mm})]bus11.center) {$11$};
% BUS 3
		\draw (bus11.center) to[voosource] ++(2,0) node (bus3) {};
		\draw [line width=1mm] ([shift=({0,5mm})]bus3.center) -- ++(0,-10mm);
		\node [shape=circle,draw,inner sep=1pt] (v3label) at ([shift=({0,8mm})]bus3.center) {$3$};
		\node [shape=vsourcesinshape, rotate=-90] (gen3) at ([shift=({1,0})]bus3) {} ;
		\node (g3label) at ([shift=({0,8mm})]gen3.center) {$G_3$};
		\draw (gen3.south) to [short] (bus3.center);
        \end{tikzpicture}
}
	\caption{Kundur two-area system for quasi-static modelling example.}
	\label{fig:kundur_2}
\end{figure} %>>>

	The parameters used are from the base case where the base units are $230kV$ and $100MVA$. The transmission lines have an impedance $z_L = 0.0001 + j0.001\ pu$ per kilometer (line lengths are indicated in the picture), and a shunt susceptance of $0.00175\ pu$ per kilometer. The transformers have unitary transformation ratios and no dephasing ($a = 1,\ \varphi = 0$) and a series impedance of $0 + j0.15pu$.

	In the base case, the capacitors $C_7$ and $C_9$ send $S_{C_7} = 0 + j2pu$ and $S_{C_9} = 0 + j3.5pu$ to their respective buses. The loads are $S_{L_7} = 9.67 + j1pu$ and $S_{L_9} = 17.67 + j1pu$. Using these quantities and the parameters one arrives at the power flow voltages of table \ref{tab:kundur_powerflow}.

% TABLE OF POWER FLOW VOLTAGES <<<
\renewcommand{\arraystretch}{1.2}
\begin{table}[t]
\begin{center}
\scalebox{0.9}{
\begin{tabular}{ c|c|c|c|c|c|c|c|c|c|c|c } 
\hline 
\raisebox{0mm}{} Bus & 1 & 2 & 3 & 4 & 5 & 6 & 7 & 8 & 9 & 10 & 11 \\
\hline
$\left\vert V\right\rvert$ & 1.03 & 1.01 & 1.03 & 1.01 & 1.006 & 0.978 & 0.961 & 0.949 & 0.971 & 0.983 & 1.008\\
$\theta$ (deg.) & 20.2 & 10.433 & 6.885 & -17.074 & 13.737 & 3.651 & -4.759 & 18.633 & 32.234 & 23.819 & -13.511\\
\hline
\end{tabular}
}
\end{center}
\caption{Power Flow results for Kundur two-area system of figure \ref{fig:kundur_2}.}
\label{tab:kundur_powerflow}
\end{table} %>>>

	We first calculate the equivalent impedances of the loads through \eqref{eq:load_eq_impedance}, yielding

\begin{gather}
	Y_{L}^7 = \dfrac{\overline{9.67 + j1}}{0.961^2} = 10.470796 - j1.0828124 \\[5mm]
	Y_{L}^9 = \dfrac{\overline{17.67 + j1}}{0.971^2} = 18.741230 - j1.0606242
\end{gather}

	\noindent and inverting these values yields

\begin{gather}
	Z_{L}^7 = \left(10.470796 - j1.0828124\right)^{-1} = 0.094493197 + j0.0097717887 \\[5mm]
	Z_{L}^9 = \left(18.741230 - j1.0606242\right)^{-1} = 0.053187942 + j0.0030100703.
\end{gather}

	Therefore $Z_{L}^7$ corresponds to a resistance of $R_L^7 = 0.094493197pu$ in series with an inductance of

\begin{equation} L^7 = \dfrac{0.094493197}{120\pi} = 25.920475\times 10^{-6} pu \end{equation}

	\noindent  where the per-unit quantity of impedance is $(230kV)^2/(100MVA) = 529\Omega$ and the per-unit quantity of inductance is also $529H$. Analogously, $Z_{L}^9$ corresponds to a resistance of $R_L^9 = 0.053187942pu$ in series with an inductance

\begin{equation} L^9 = \dfrac{0.0030100703}{120\pi} = 7.9844594\times 10^{-6} pu .\end{equation}

	Therefore the loads are modelled by the Dynamic Phasor Functionals

\footnotesize
\begin{gather}
	\mathbf{Z}_{L}^7 = 25.920475\times 10^{-6}\dpo + 0.094493197\mathbf{I} \Leftrightarrow \mathbf{Y}_{L}^7 = \dfrac{\mathbf{I}}{25.920475\times 10^{-6}\dpo + 0.094493197\mathbf{I}}\\[5mm]
	\mathbf{Z}_{L}^9 = 7.9844594\times 10^{-6}\dpo + 0.053187942\mathbf{I} \Leftrightarrow \mathbf{Y}_{L}^9 = \dfrac{\mathbf{I}}{7.9844594\times 10^{-6}\dpo + 0.053187942\mathbf{I}}.
\end{gather}
\normalsize

	We also calculate the capacitances $C_7$ and $C_9$ attached to buses 7 and 9. From the power flow,

\begin{gather}
	Y_{C_7} = \dfrac{\overline{-j2}}{0.961^2} = j2.1656248 \Rightarrow C_7 = \dfrac{2.1656248}{120\pi} = 5.7444982\times 10^{-3} pu \\
	Y_{C_9} = \dfrac{\overline{-j3.5}}{0.971^2} = j3.7121848 \Rightarrow C_9 = \dfrac{3.7121848}{120\pi} = 9.8468760\times 10^{-3} pu
\end{gather}

	\noindent where the per-unit value of capacitance is $1/529 = 1.8903592$ mF; these values correspond to the Dynamic Admittance operators

\begin{equation} \mathbf{Y}_{C_7} = \dfrac{\mathbf{I}}{5.7444982\times 10^{-3}\dpo},\ \mathbf{Y}_{C_9} = \dfrac{\mathbf{I}}{9.8468760\times 10^{-3}\dpo} .\end{equation}

	Therefore the total shunt admittance operators at buses 7 and 9 are given by

\begin{align}
	\mathbf{y}_{sh}^7 &= \dfrac{\mathbf{I}}{25.920475\times 10^{-6}\dpo + 0.094493197\mathbf{I}} + \dfrac{\mathbf{I}}{5.7444982\times 10^{-3}\dpo} = \nonumber\\[5mm]
		&= \dfrac{5.7704187\dpo + 94.493197\mathbf{I}}{148.90012\times 10^{-6}\ndpo{2} + 542.81600\times 10^{-3}\dpo} \\[5mm]
	\mathbf{y}_{sh}^9 &= \dfrac{\mathbf{I}}{7.9844594\times 10^{-6}\dpo + 0.053187942\mathbf{I}} + \dfrac{\mathbf{I}}{9.8468760\times 10^{-3}\dpo} = \nonumber\\[5mm]
		&= \dfrac{9.8548605\dpo + 53.187942\mathbf{I}}{78.621982\times 10^{-6}\ndpo{2} + 523.73507\times 10^{-3}\dpo}
\end{align}

	Finally we calculate the admittances of the transmission lines. The system has three types of lines: 110km, 25km and 10km. Then again, the line admittances and shunt admittances are calculated as functions of the line lengths. The line impedances are $z_L = 0.0001 + j0.001\ pu$ per km, that is, a resistance of $R_L = 0.0001pu$ in series with an inductance of $0.001/120\pi\ pu$ per km, leading to the admittance operator

\begin{equation} \mathbf{y} = \dfrac{\mathbf{I}}{a\times \left( 0.0001\mathbf{I} + 2.6525824\times 10^{-6}\dpo \right)}\ pu \end{equation}

	\noindent with $a$ the line length in kilometers. The capacitive shunt admittance is given by $0.00175pu$ per kilometer, which entails to a capacitance

\begin{equation} C_{sh} = \dfrac{1}{a\times 0.65973446}\ pu \end{equation}

	\noindent defining an admittance operator

\begin{equation} \mathbf{y}_{sh} = \dfrac{1}{a\times 0.65973446}\dpo .\end{equation}

	Using these values one can build the admittance matrix operator $\mathbf{Y}$. Equations \eqref{eq:example_y1} through \eqref{eq:example_y4} show the resulting pieces $\mathbf{Y}_1$ through $\mathbf{Y}_4$ of $\mathbf{Y}$. Note: due to page space constraints the equation for $\mathbf{Y}_4$ is shown broken in half.

\begin{align}
\mathbf{Y}_1 &= \left[\begin{matrix}0.00039788736 \dpo & 0 & 0 & 0\\0 & 0.00039788736 \dpo & 0 & 0\\0 & 0 & 0 & 0\\0 & 0 & 0 & 0.00039788736 \dpo\end{matrix}\right] \label{eq:example_y1} \\[5mm]
\mathbf{Y}_2 &= \left[\begin{matrix}0 & 0 & 0 & 0 & 0 & 0\\- 0.00039788736 \dpo & 0 & 0 & 0 & 0 & 0\\0 & 0 & 0 & 0 & 0 & - 0.00039788736 \dpo\\0 & 0 & 0 & 0 & - 0.00039788736 \dpo & 0\end{matrix}\right] \label{eq:example_y2} \\[5mm]
\mathbf{Y}_3 &= \left[\begin{matrix}0 & 0 & 0 & 0\\0 & 0 & 0 & 0\\0 & 0 & 0 & 0\\0 & 0 & 0 & 0\\0 & 0 & 0 & 0\\0 & 0 & - 0.00039788736 \dpo & 0\end{matrix}\right] \label{eq:example_y3}
\end{align}

% HUGE Y4 EQUATION <<<
\newpage
\footnotesize
\begin{sideways}
\parbox{\textheight}{%
\begin{gather}
	\mathbf{Y}_4 =
	\left[\hspace{5mm} \begin{matrix}
		\frac{2.6385725 \times 10^{-8} \dpo^3 + 9.947184 \times 10^{-7}\dpo^2 + 1.00000402 \dpo + 1.5157614\times 10^{-4}\mathbf{I}}{6.631456 \times 10^{-5} \dpo^2 + 2.5\times 10^{-3}\dpo} &
		- \frac{\mathbf{I}}{6.63145600 \times 10^{-5} \dpo + 2.5\times 10^{-3}\mathbf{I}} &
		0\\[5mm]
%
		- \frac{\mathbf{I}}{6.631456 \times 10^{-5} \dpo + 2.5\times 10^{-3}\mathbf{I}} &
		\frac{7.0000281 \dpo + 1.0610330\times 10^{-3}\mathbf{I}}{1.3262912\times 10^{-4} \dpo + 5\times 10^{-3}\mathbf{I}} &
		- \frac{\mathbf{I}}{2.65258240 \times 10^{-5} \dpo + 10^{-3}\mathbf{I}}\\[5mm]
%
		0 &
		- \frac{\mathbf{I}}{2.6525824 \times 10^{-5} \dpo + 10^{-3}\mathbf{I}} &
		\frac{3.6194256 \times 10^{-3} \dpo^{2} + 7.1476829 \dpo + 1.0404948\mathbf{I}}{4.3446682 \times 10^{-8} \dpo^{3} + 1.600230\times 10^{-4} \dpo^2 + 5.970976\times 10^{-3}\dpo}\\[5mm]
%
		0 &
		0 &
		- \frac{\mathbf{I}}{1.45892032\times 10^{-4} \dpo + 5.5\times 10^{-3}\mathbf{I}}\\[5mm]
%
		0 &
		0 &
		0\\[5mm]
%
		0 &
		0 &
		0\\[5mm]
%
		0 &
		0 &
		0
\end{matrix}\right. \nonumber\\[8mm]
%
		\hspace{1cm}\left.\begin{matrix}
		0 &
		0 &
		0 &
		0\\[5mm]
%
		0 &
		0 &
		0 &
		0\\[5mm]
%
		- \frac{\mathbf{I}}{1.45892032\times 10^{-4} \dpo + 5.5\times 10^{-3}\mathbf{I}} &
		0 &
		0 &
		0\\[5mm]
%
		\frac{2.0000080 \dpo + 3.0315227\times 10^{-4}}{1.45892032\times 10^{-4} \dpo^2 + 5.5\times 10^{-3}\dpo} &
		- \frac{\mathbf{I}}{1.45892032\times 10^{-4} \dpo + 5.5\times 10^{-3}\mathbf{I}} &
		0 &
		0\\[5mm]
%
		- \frac{\mathbf{I}}{1.45892032\times 10^{-4} \dpo + 5.5\times 10^{-3}\mathbf{I}} &
		\frac{3.8975811\times 10^{-3}\dpo^{2} + 6.9325063 \dpo + 0.58609938\mathbf{I}}{2.2940641 \times 10^{-8} \dpo^{3} + 1.5368239\times 10^{-4} \dpo^2 + 5.76108577\times 10^{-3}\dpo} &
		- \frac{\mathbf{I}}{2.6525824 \times 10^{-5} \dpo + 10^{-3}\mathbf{I}} &
		0\\[5mm]
%
		0 &
		- \frac{\mathbf{I}}{2.6525824 \times 10^{-5} \dpo + 10^{-3}\mathbf{I}} &
		\frac{1.0000040 \dpo + 1.5157614\mathbf{I}}{2.6525824 \times 10^{-5} \dpo^2 + 10^{-3}\dpo} &
		- \frac{\mathbf{I}}{6.631456 \times 10^{-5} \dpo + 2.5\times 10^{-3}\mathbf{I}}\\[5mm]
%
		0 &
		0 &
		- \frac{\mathbf{I}}{6.631456 \times 10^{-5} \dpo + 2.5\times 10^{-3}\mathbf{I}} &
		0
\end{matrix}\hspace{5mm}\right]
\nonumber\\[8mm] \label{eq:example_y4}
\end{gather}
}
\end{sideways}
\normalsize
\newpage
%>>>

	Naturally, in a static phasors condition where $\omega$ is constant and all phasors are constant (thus of null derivatives), equations \eqref{eq:example_y1},\eqref{eq:example_y2},\eqref{eq:example_y3} and \eqref{eq:example_y4} for $\mathbf{Y}_1$ through $\mathbf{Y}_4$ become their steady-state equivalent used in Quasi-Static approximations.

%-------------------------------------------------
\section{Nonlinear systems: modelling of an electronic amplifier} \label{sec:bjt_ampli_modelling}%<<<1

	Consider the amplifier circuit of figure \ref{fig:common_emitter}, known as a common emitter amplifier using a bipolar junction transistor (BJT). The input voltage $v_i(t)$ is a nonstationary sinusoid defined at some frequency $\omega(t)$, $V_{CC}$ and $V_{EE}$ constant voltages, $v_o(t)$ the output voltage.

% MODELLING EXAMPLE: COMMON EMITTER AMPLIFIER <<<
\begin{figure}[h]
\centering
        \begin{tikzpicture}[american,scale=1,transform shape,line width=0.75, cute inductors, voltage shift = 1,>={Stealth[inset=0mm,length=1.5mm,angle'=50]}]
	\ctikzset{/tikz/circuitikz/voltage/distance from node=10mm}
		\node [npn] (q1) at (0,0) {$Q_1$};
		\draw (q1.C) to[short] ++(0,0.5) node (cout) {} to[R,l=$R_C$] ++(0,2) node (vcc_C) {};
		\draw (cout.center) to [C,l=$C_L$, *-] ++(2,0) node (out) {} to [R,l=$R_L$] ++(0,-2) node[tlground] {};
		\draw [->] (out.center) to[short,*-] ++(1,0) node [right] (outlabel) {$v_o(t)$};
		\draw (q1.E) to[R,l=$R_E$] ++(0,-2) node (vee_E) {};
		\draw (q1.B) to[short, -*] ++(-1,0) node (base_conn) {};
		\draw (base_conn.center) to [R, l=$R_{B1}$] (base_conn |- vcc_C) node (vcc_B) {};
		\draw (base_conn.center) to [R, l=$R_{B2}$] (base_conn |- vee_E) node (vee_B) {};
		\draw (base_conn.center) to [C, l=$C_B$] ++(-2,0) to [sV, l_=$v_i(t)$] ++(0,-3) node [tlground] {} ;
% SUPPLY BARS
		\draw [line width=1mm] ([shift=({-12.5mm,0})]$(vcc_B)!0.5!(vcc_C)$) node (mid_vcc) {} -- ++(25mm,0);
		\draw [line width=1mm] ([shift=({-12.5mm,0})]$(vee_B)!0.5!(vee_E)$) node (mid_vee) {} -- ++(25mm,0);
		\node at ([shift=({0mm,-3mm})]$(vee_B)!0.5!(vee_E)$) {$V_{EE}$};
		\node at ([shift=({0mm, 3mm})]$(vcc_B)!0.5!(vcc_C)$) {$V_{CC}$};
        \end{tikzpicture}
	\caption{Common emitter bipolar transistor amplifier circuit.}
	\label{fig:common_emitter}
\end{figure} %>>>

	For the transistor model, \eqref{eq:complete_ebers_moll} shows a commonly used model for simulating bipolar transistor circuits known as te Ebers-Moll model \pcite{ebersLargeSignalBehaviorJunction1954,grayAnalysisDesignAnalog2009} depicte din figure \ref{fig:ebers_moll}. In this model, the base-collector junction is modelled by the diode $D_R$ (the subscript ``R'' for ``reverse'') and the base-emitter junction by $D_F$ (the subscript ``F'' for ``forward''). The current sources $\alpha_F i_F$ and $\alpha_R i_R$ correspond to saturation currents on the forward bias (collector and emitter working as collector and emitter, respectively) and the reverse bias (collector working as emitter, emitter working as collector). $C_{BC}$ and $C_{BE}$ are parasitic capacitances of the junctions, as $r_B, r_C, r_E$ are parasitic resistances.

% EBERS MOLL MODEL WITH EARLY EFFECT <<<
\begin{figure}[h!]
\centering
        \begin{tikzpicture}[american,scale=1,transform shape,line width=0.75, cute inductors, voltage shift = 1,>={Stealth[inset=0mm,length=1.5mm,angle'=50]}]
	\ctikzset{/tikz/circuitikz/voltage/distance from node=10mm}
		\node [npn] (q1) at (-2,0) {};
		\draw (q1.C) to [short, -o] ++(0,0.5) node[above] (qcoll) {$C$};
		\draw (q1.E) to [short, -o] ++(0,-0.5) node[below] (qemitt) {$E$};
		\draw (q1.B) to [short, -o] ++(-0.5,0) node[left] (qbase) {$B$};

		\draw (0,0) node[left] (base) {$B$} to[short,i=$i_B$,o-] ++(1,0) to [R,l=$r_B$, -*] ++(2,0)  node (baseconn) {} to [D, color=stewartpink, stewartpink, l_=$D_R$, i=$i_R$] ++(0,2) to [short] ++(2,0) node (collconn) {} to [R,l=$r_C$,*-] ++(0,2) to [short, -o, i<=$i_C$] ++(0,1) node[above] (collector) {$C$};
		\draw (collconn.center) to [cisourceAM,/tikz/circuitikz/bipoles/length=1cm, color=stewartblue, stewartblue, l=$\alpha_F i_F$] (collconn |- baseconn) node (midconn) {} to [short] (baseconn.center);
		\draw (collconn.center) to [short] ++(2,0) node (cbcstart) {} to [C,l=$C_{BC}$] (baseconn -| cbcstart) to [short,*-*] (midconn.center);
		\draw (baseconn.center) to [D, color=stewartblue, stewartblue, l_=$D_F$, i=$i_F$] ++(0,-2) to [short] ++(2,0) node (emitterconn) {} to [R,l=$r_E$, *-] ++(0,-2) to [short, i>=$i_E$, -o] ++(0,-1) node[below] (emitter) {$E$};
		\draw (emitterconn.center) to [cisourceAM,/tikz/circuitikz/bipoles/length=1cm, color=stewartpink, stewartpink, l_=$\alpha_R i_R$] (midconn.center) to [short] (baseconn.center);
		\draw (baseconn -| cbcstart) to [C,l=$C_{BE}$] (emitterconn -| cbcstart) to [short] (emitterconn.center);
		\node at ([shift=({3mm,3mm})]baseconn) {$B'$};
		\node at ([shift=({3mm,3mm})]collconn) {$C'$};
		\node at ([shift=({3mm,3mm})]emitterconn) {$E'$};

        \end{tikzpicture}
	\caption{Large-signal Ebers Moll model for the NPN bipolar junction transistor.}
	\label{fig:ebers_moll}
\end{figure} %>>>

	The equations of this model are given by \eqref{eq:complete_ebers_moll}. In those equations, the two first equations are the exponential equations of the forward diode $D_F$ and reverse diode $D_R$; the third and fourth equations are the Early Effect correction equations. The three following equations are the parasitic resistance equations of $r_B, r_C$ and $r_E$, 

\begin{equation}
	\left\{\begin{array}{l}
		i_F = I_{ES}\left[\exp\left(\dfrac{v_{B'} - v_{E'}}{n_CV_T}\right) - 1\right] \\[5mm]
		i_R = I_{CS}\left[\exp\left(\dfrac{v_{B'} - v_{C'}}{n_FV_T}\right) - 1\right] \\[5mm]
		I_{ES} = I_{ES}^0 \left(1 + \dfrac{v_{CE}}{V_A}\right) \\[5mm]
		I_{CS} = I_{CS}^0 \left(1 + \dfrac{v_{BC}}{V_A}\right)  \\[5mm] v_{B'} = v_B - r_Bi_B \\[3mm]
		v_{C'} = v_C - r_Ci_B \\[3mm]
		v_{E'} = v_E + r_Ei_B  \\[3mm]
		i_C + i_R - \alpha_F i_F - C_{BC}\dot{v}_{C'} - C_{BC}\dot{v}_{B'} = 0 \\[3mm]
		-i_E + i_F - \alpha_R i_R - C_{BE}\dot{v}_{E'} - C_{BE}\dot{v}_{B'} = 0 
	\end{array}\right. \label{eq:complete_ebers_moll}
\end{equation}

	For the purposes of analytical analysis, we simplify the model of figure \ref{fig:ebers_moll_simplified}. We first disregard the parasitic effects of junction resistances and capacitances. We also suppose that the device is well within forward bias; thus we can ignore the reverse bias components since their current contributions are negligble. With these considerations the model becomes that of figure \ref{fig:ebers_moll_simplified}.

% SIMPLIFIED EBERS MOLL MODEL <<<
\begin{figure}[h]
\centering
        \begin{tikzpicture}[american,scale=1,transform shape,line width=0.75, cute inductors, voltage shift = 1,>={Stealth[inset=0mm,length=1.5mm,angle'=50]}]
	\ctikzset{/tikz/circuitikz/voltage/distance from node=10mm}
		\draw (0,0) node[below] (base) {$B$} to[short,i=$i_B$,o-] ++(0,1) node (baseconn) {} to [cisourceAM,/tikz/circuitikz/bipoles/length=1cm, l_=$\alpha_F i_E$, invert] ++(-2,0) to [short, -o, i<=$i_C$] ++(-1,0) node[above] (collector) {$C$};
		\draw (baseconn.center) to[D, l=$D_F$, i=$i_E$, *-o] ++(3,0) node[right] (emitter) {$E$};
		\node[right=10mm of emitter] {
		$
	\left\{\begin{array}{l}
		i_E = I_{ES}\left[\exp\left(\dfrac{v_{BE}}{n_CV_T}\right) - 1\right] \\[5mm]
		I_{ES} = I_{ES}^0 \left(1 + \dfrac{v_{CE}}{V_A}\right) \\[5mm]
		i_B + \alpha_F i_E - i_E = 0 \\[5mm]
		i_C = \alpha_F i_E
	\end{array}\right. $
		};
        \end{tikzpicture}
	\caption{Simplified large-signal Ebers Moll model for the NPN bipolar junction transistor in the forward bias region.}
	\label{fig:ebers_moll_simplified}
\end{figure} %>>>

	Developing the model equations of figure \ref{fig:ebers_moll_simplified} yields

\begin{equation}
	\left\{\begin{array}{l}
		i_C = \alpha_F I_{ES}\left[\exp\left(\dfrac{v_{BE}}{n_CV_T}\right) - 1\right] \\[3mm]
		I_{ES} = I_{ES}^0 \left(1 + \dfrac{v_{CE}}{V_A}\right) \\[5mm]
		i_C = \dfrac{\alpha_F}{1 - \alpha_F} i_B
	\end{array}\right. \label{eq:simple_ebers_moll_developed}
\end{equation}

	Now naming the tandem parameters

\begin{equation}
	\left\{\begin{array}{l}
		I_S^0 = \alpha_F I_{ES}^0 \\[3mm]
		\beta_F = \dfrac{\alpha_F}{1 - \alpha_F}
	\end{array}\right. \label{eq:simple_ebers_moll_params}
\end{equation}

	\noindent one arrives at the more known equations

\begin{equation}
	\left\{\begin{array}{l}
		i_C = I_{S}\left[\exp\left(\dfrac{v_{BE}}{n_CV_T}\right) - 1\right] \\[5mm]
		I_{S} = I_{S}^0 \left(1 + \dfrac{v_{CE}}{V_A}\right) \\[5mm]
		i_C = \beta_F i_B
	\end{array}\right. \label{eq:simple_ebers_moll_new_params}
\end{equation}

	\noindent and this achieves a simplified large-signal model \eqref{eq:dcbias_largesignal} of the amplifier circuit of figure \ref{fig:common_emitter} where the currents and voltages are depicted in figure \ref{fig:common_emitter_dcbias}. Using this model and forcing steady-state (all derivatives equal zero), the algebraic operating point equations (also called ``DC'' or ``bias'' equations) of the amplifier of figure \ref{fig:common_emitter} is achieved.

\begin{equation}
	\left\{\begin{array}{l}
		i_C = I_{S}^0 \left(1 + \dfrac{v_{C} - v_{E}}{V_A}\right)\left[\exp\left(\dfrac{v_{B} - v_{E}}{n_CV_T}\right) - 1\right] \\[5mm]
		i_C = \beta_F i_B \\[5mm]
		C_B\dfrac{d}{dt}\left(v_i - v_B\right) + \dfrac{V_{CC} - v_B}{R_{B1}} - \dfrac{v_B - V_{EE}}{R_{B2}} - i_B = 0  \\[5mm]
		-i_C + \dfrac{V_{CC} - v_{C}}{R_C} - i_L = 0 \\[5mm]
		C_L\dfrac{d}{dt}\left(v_C - v_o\right) - \dfrac{v_o}{R_L} = 0 \\[5mm]
		R_Li_L = v_o \\[5mm]
		v_E - v_{EE} = R_Ei_E \\[5mm]
		i_E = i_B + i_C
	\end{array}\right. \label{eq:dcbias_largesignal}
\end{equation}

% DC BIAS EQUIVALENT CIRCUIT OF COMMON EMITTER AMPLIFIER <<<
\begin{figure}[h]
\centering
        \begin{tikzpicture}[american,scale=1,transform shape,line width=0.75, cute inductors, voltage shift = 1,>={Stealth[inset=0mm,length=1.5mm,angle'=50]}]
	\ctikzset{/tikz/circuitikz/voltage/distance from node=10mm}
		\node [npn] (q1) at (0,0) {$Q_1$};
		\draw (q1.C) to [short, f<_=$i_C$] ++(0,1) coordinate (collconn) to [R,l_=$R_C$] ++(0,2) node (vcc_C) {};
		\draw (collconn) to [C, l=$C_L$, f_=$i_L$, *-] ++(3,0) to [R, l=$R_L$] ++(0,-2) node [tlground] {};
		\draw (q1.E) to[R,l=$R_E$, f>_=$i_E$] ++(0,-3) node (vee_E) {};
		\draw (q1.B) to[short, -*, f<_=$i_B$] ++(-1,0) node (base_conn) {};
		
		\draw (base_conn.center) to [short] ++(0,1) to [R, l=$R_{B1}$, f<^=$i_1$] (base_conn |- vcc_C) node (vcc_B) {};
		\draw (base_conn.center) to [short] ++(0,-1) to [R, l=$R_{B2}$, f>_=$i_2$] (base_conn |- vee_E) node (vee_B) {};

		\draw (base_conn.center) to [C, l=$C_B$] ++(-2,0) to [sV, l_=$v_i(t)$, f<^=$i_i$] ++(0,-3) node [tlground] {} ;
% SUPPLY BARS
		\draw [line width=1mm] ([shift=({-12.5mm,0})]$(vcc_B)!0.5!(vcc_C)$) node (mid_vcc) {} -- ++(25mm,0);
		\draw [line width=1mm] ([shift=({-12.5mm,0})]$(vee_B)!0.5!(vee_E)$) node (mid_vee) {} -- ++(25mm,0);
		\node at ([shift=({0mm,-3mm})]$(vee_B)!0.5!(vee_E)$) {$V_{EE}$};
		\node at ([shift=({0mm, 3mm})]$(vcc_B)!0.5!(vcc_C)$) {$V_{CC}$};
        \end{tikzpicture}
	\caption{``DC'' or ``bias'' equivalent circuit of common emitter BJT amplifier circuit.}
	\label{fig:common_emitter_dcbias}
\end{figure} %>>>

	Using these equations one can arrive at a linearized model:

\begin{equation}
	\left\{\begin{array}{l}
		\dfrac{\partial i_C}{\partial v_{BE}} = I_{S}^0 \left(1 + \dfrac{v_{CE}^o}{V_A}\right)\left[\dfrac{1}{n_CV_T} \exp\left(\dfrac{v_{BE}^o}{n_CV_T}\right) \right]\\[5mm]
		\dfrac{\partial i_C}{\partial v_{CE}} = i_E^o \dfrac{1}{V_A} \left[\exp\left(\dfrac{v_{BE}^o}{n_CV_T}\right) - 1\right] \\[5mm]
		\dfrac{\partial i_B}{\partial v_{BE}} = \dfrac{I_{S}^0}{\beta_F} \left(1 + \dfrac{v_{CE}^o}{V_A}\right)\left[\dfrac{1}{n_CV_T} \exp\left(\dfrac{v_{BE}^o}{n_CV_T}\right) \right] \\[5mm]
		\dfrac{\partial i_B}{\partial v_{CE}} = \dfrac{I_{S}^0}{\beta_F} \left(1 + \dfrac{1}{V_A}\right)\left[\exp\left(\dfrac{v_{BE}^o}{n_CV_T}\right) - 1\right]
	\end{array}\right.
\end{equation}

	\noindent where the superscript ``$o$'' denotes an operating point, that is, these small-signal quantities are calculated at an operating point $v_{CE}^o, v_{BE}^o$. These quantities are generally denoted in the more familiar notations

\begin{equation}
	\left\{\begin{array}{l}
		\dfrac{\partial i_C}{\partial v_{BE}} = g_m\\[5mm]
		\dfrac{\partial i_C}{\partial v_{CE}} = \dfrac{1}{r_o}\\[5mm]
		\dfrac{\partial i_B}{\partial v_{BE}} = \dfrac{1}{r_\pi} = \dfrac{1}{\beta_F r_o} \\[5mm]
		\dfrac{\partial i_B}{\partial v_{CE}} = g_\mu = \dfrac{g_m}{\beta_F}
	\end{array}\right.
\end{equation}

	Figure \ref{fig:ebers_moll_small_signal} shows the small-signal model originated by these equations and quantities.

% SMALL_SIGNAL EBERS MOLL MODEL <<<
\begin{figure}[h]
\centering
        \begin{tikzpicture}[american,scale=1,transform shape,line width=0.75, cute inductors, voltage shift = 1,>={Stealth[inset=0mm,length=1.5mm,angle'=50]}]
	\ctikzset{/tikz/circuitikz/voltage/distance from node=10mm}
		\draw (0,0) node[left] (base) {$B$} to[short,i=$i_B$,o-] ++(1,0) coordinate(baseconn) to [cisourceAM,/tikz/circuitikz/bipoles/length=1cm, l_=$g_{\mu} v_{CE}$] ++(0,-2) to [short] ++(1,0) coordinate (emitterconn);
		\draw (baseconn.center) to[short] (base -| emitterconn) to [R,l=$r_{\pi}$] (emitterconn.center);
		\draw (emitterconn.center) to[short] ++(1,0) coordinate (emittermid) to[short, -o, i=$i_E$] ++(0,-1) node[below] (emitter) {$E$};
		\draw (emittermid.center) to[short] ++(1,0) coordinate (collconn) to[R,l=$r_o$] (collconn |- base) to [short] ++(1,0) coordinate (collmid) to [cisourceAM,/tikz/circuitikz/bipoles/length=1cm, l^=$g_m v_{BE}$] (emittermid -| collmid) to [short] (collconn);
		\draw (collmid.center) to[short, -o, i<=$i_C$] ++(1,0) node[right] (collector) {$C$};
        \end{tikzpicture}
	\caption{Small-signal model for the NPN bipolar junction transistor using the simplified Ebers Moll model of figure \ref{fig:ebers_moll_simplified}.}
	\label{fig:ebers_moll_small_signal}
\end{figure} %>>>

	Further approximations are made: from \eqref{eq:simple_ebers_moll_new_params}, we now that in the forward bias the $v_{BE}^o$ is in the decimals of volts (clasically around $0.7V$) and the $V_T$ is small (around $25mV$). Hence the exponential function of their quotient is large and the expression for $i_C^o$ can be approximated

\begin{equation} i_C^o = I_{S}\left[\exp\left(\dfrac{v_{BE}^o}{n_CV_T}\right) - 1\right] \approx I_{S}\exp\left(\dfrac{v_{BE}^o}{n_CV_T}\right) .\end{equation}

	One also considers that $v_{CE}^o$ in the forward bias is generally of hundreds of volts (generally $100$ to $200mV$) and the Early voltage $V_A$ is quite high (tens or hundreds of volts). Then one can approximate the small signal quantities as

\begin{equation}
	\left\{\begin{array}{l}
		g_m \approx  \dfrac{i_C^o}{n_CV_T} \\[5mm]
		\dfrac{1}{r_o} =  i_C^o \dfrac{1}{V_A\left(1 + \dfrac{v_{CE}^o}{V_A}\right)} = \dfrac{i_C^o}{\left(V_A + v_{CE}^o\right)} \approx \dfrac{i_C^o}{V_A}
	\end{array}\right. .
\end{equation}

	Further, because the current gain $\beta_F$ is quite high (hundreds to thousands) and the voltage $v_{CE}$ is much smaller than $v_{BE}$, the current contribution $g_\mu v_{CE}$ is neglected for being much smaller than $g_m v_{BE}$. Thus the amplifier circuit of figure \ref{fig:common_emitter} becomes the small-signal version of \ref{fig:common_emitter}.

% SMALL SIGNAL VERSION OF THE COMMON EMITTER AMPLIFIER <<<
\begin{figure}[h]
\centering
\scalebox{0.95}{
        \begin{tikzpicture}[american,scale=1,transform shape,line width=0.75, cute inductors, voltage shift = 1,>={Stealth[inset=0mm,length=1.5mm,angle'=50]}]
	\ctikzset{/tikz/circuitikz/voltage/distance from node=10mm}
		%\node [npn] (q1) at (0,0) {$Q_1$};
		\draw (0,0) to [R,l=$r_\pi$, v=$v_{BE}$] ++(3,0) coordinate (emitterconn) to [R,l=$r_o$] ++(0,3) to[short] ++(2.5,0) coordinate (collector) to [cisourceAM, l_=$g_mv_{BE}$, *-] (collector |- emitterconn) to [short, -*] (emitterconn.center);
		\draw (collector.center) to [short] ++(2,0) coordinate (rc_conn) to[R,l=$R_C$] ++(0,- 2) node[tlground] {} ;
		\draw (rc_conn.center) to[C,l=$C_L$, *-] ++(2,0) node (out) {} to [R,l=$R_L$] ++(0,-2) node [tlground] (vcc_C)  {};
		\draw [->] (out.center) to[short,*-] ++(1,0) node [right] (outlabel) {$v_o(t)$};
		\draw (emitterconn.center) to[R,l=$R_E$] ++(0,-2) node[tlground] (vee_E) {};
		\draw (0,0) to[short, -*] ++(-0,0) node (base_conn) {};
		\draw (base_conn.center) to [R, l=$R_{B1}$] ++(0,-2) node[tlground] (vcc_B) {};
		\draw (base_conn.center) to [short] ++(-2,0) coordinate (input) to [R, l=$R_{B2}$, *-] ++(0,-2) node[tlground] (vee_B) {};
		\draw (input.center) to [C, l=$C_B$] ++(-2,0) to [sV, l_=$v_i(t)$] ++(0,-2) node [tlground] {} ;
% SUPPLY BARS
        \end{tikzpicture}
}
	\caption{Small-signal version of the common emitter bipolar transistor amplifier circuit of figure \ref{fig:common_emitter}.}
	\label{fig:common_emitter_small}
\end{figure} %>>>

	Transport the small-signal model of \ref{fig:common_emitter_small} to the Dynamic Phasor domain by substituting capacitances by their Dynamic Impedances to obtain the schematic of figure \ref{fig:common_emitter_small_dp}.

% SMALL SIGNAL VERSION OF THE COMMON EMITTER AMPLIFIER <<<
\begin{figure}[h]
\centering
\scalebox{0.95}{
        \begin{tikzpicture}[american,scale=1,transform shape,line width=0.75, cute inductors, voltage shift = 1,>={Stealth[inset=0mm,length=1.5mm,angle'=50]}]
	\ctikzset{/tikz/circuitikz/voltage/distance from node=10mm}
		%\node [npn] (q1) at (0,0) {$Q_1$};
		\draw (0,0) to [R,l=$r_\pi$, v=$V_{BE}$] ++(3,0) coordinate (emitterconn) to [R,l=$r_o$] ++(0,3) to[short] ++(2.5,0) coordinate (collector) to [cisourceAM, l_=$g_mV_{BE}$, *-] (collector |- emitterconn) to [short, -*] (emitterconn.center);
		\draw (collector.center) to [short] ++(2,0) coordinate (rc_conn) to[R,l_=$R_C$] ++(0,- 2) node[tlground] {} ;
		\draw (rc_conn.center) to[C,l=$\dfrac{\mathbf{I}}{\dpo C_L}$, label distance = 4mm, *-] ++(2,0) node (out) {} to [R,l=$R_L$] ++(0,-2) node [tlground] (vcc_C)  {};
		\draw [->] (out.center) to[short,*-] ++(1,0) node [right] (outlabel) {$V_o(t)$};
		\draw (emitterconn.center) to[R,l=$R_E$] ++(0,-2) node[tlground] (vee_E) {};
		\draw (0,0) to[short, -*] ++(-0,0) node (base_conn) {};
		\draw (base_conn.center) to [R, l=$R_{B1}$] ++(0,-2) node[tlground] (vcc_B) {};
		\draw (base_conn.center) to [short] ++(-2,0) coordinate (input) to [R, l=$R_{B2}$, *-] ++(0,-2) node[tlground] (vee_B) {};
		\draw (input.center) to [C, l_=$\dfrac{\mathbf{I}}{\dpo C_B}$, label distance = 4mm] ++(-2,0) to [sV, l_=$V_i(t)$] ++(0,-2) node [tlground] {} ;
% NODE LABELS
		\node [shape=circle,draw,inner sep=1pt] at (-1,0.5)  {$1$};
		\node [shape=circle,draw,inner sep=1pt] at (3.5,0.5) {$2$};
		\node [shape=circle,draw,inner sep=1pt] at (5.5,3.5) {$3$};
		\node [shape=circle,draw,inner sep=1pt] at (9.5,3.5) {$4$};
        \end{tikzpicture}
}
	\caption{Dynamic Phasor small-signal version of the common emitter bipolar transistor amplifier circuit of figure \ref{fig:common_emitter}.}
	\label{fig:common_emitter_small_dp}
\end{figure} %>>>

	Thus we use the Kirchoff's Laws in Dynamic Phasor Domain and the bipoles current-voltage relationships on the nodes:

\begin{equation}
\left\{\begin{array}{l}
	(1):\ \left(V_i - V_1\right) \dpo C_B - \dfrac{V_1}{R_{B1}} - \dfrac{V_1}{R_{B2}} - \dfrac{V_1 - V_2}{r_\pi} = 0 \\[5mm]
	(2):\ \dfrac{V_1 - V_2}{r_\pi} + g_m \left(V_1 - V_2\right) - \dfrac{V_2}{R_E} + \dfrac{V_3 - V_2}{r_o} = 0 \\[5mm]
	(3):\ -g_m \left(V_1 - V_2\right) - \dfrac{V_3 - V_2}{r_o} - \dfrac{V_3}{R_C} - \left(V_3 - V_o\right)\dpo C_L = 0 \\[5mm]
	(4):\ \left(V_3 - V_o\right)\dpo C_L - \dfrac{V_o}{R_L} = 0
\end{array}\right. \label{eq:transistor_node_eqs}
\end{equation}

	Hence leading to a 4-equation-by-4-unknowns system ($V_1,V_2,V_3,V_o$). Writing this system in matrix form yields

\begin{equation} \mathbf{A}\left[\begin{array}{c} V_1 \\[3mm] V_2 \\[3mm] V_3 \\[3mm] V_o \end{array}\right] = \left[\begin{array}{c} -\dpo C_B \\[3mm] 0 \\[3mm] 0 \\[3mm] 0 \end{array}\right]V_i, \end{equation}

	\noindent where

\begin{equation}
\hspace{-2mm}\mathbf{A} = \left[\begin{array}{cccc}
	 - \dpo C_B - \dfrac{1}{R_{B1}} - \dfrac{1}{R_{B2}} - \dfrac{1}{r_\pi} & \dfrac{1}{r_\pi} & 0 & 0  \\[5mm]
	\dfrac{1}{r_\pi} + g_m & -\dfrac{1}{r_\pi} - g_m - \dfrac{1}{R_E} - \dfrac{1}{r_o} & \dfrac{1}{r_o} & 0 \\[5mm]
	-g_m & g_m + \dfrac{1}{r_o} & - \dfrac{1}{r_o} - \dfrac{1}{R_C} - \dpo C_L & \dpo C_L \\[5mm]
	0 & 0 & \dpo C_L & -\dpo C_L - \dfrac{1}{R_L}
\end{array}\right],
\end{equation}

	\noindent and one can find the gain operator $\mathbf{G}$ such that $V_o = G\left[V_i\right]$ by inverting this matrix. This yields

\begin{equation} V_o = \mathbf{G}\left[V_i\right],\ \mathbf{G} = \dfrac{ N_2\ndpo{2} + N_1\dpo}{D_2\ndpo{2} + D_1\dpo + D_0} \end{equation}

	\noindent where

\begin{equation}\hspace{-2mm} D_2 =  C_BC_L\left[R_{C}R_ER_L +  r_o r_{\pi} \left[\left(g_m  + \dfrac{1}{r_o} +  \dfrac{1}{r_{\pi}}\right)R_E\left(R_C + R_L\right) + R_C\left(1 + \dfrac{R_L}{r_o}\right) + R_L\right]\right]  \end{equation}

\begin{align}
	D_1 &= \left\{
		\begin{array}{l}
			C_B \left[R_E\left(R_{C} + g_m r_o r_{\pi} + r_o + r_{\pi}\right) + r_{\pi}\left(r_o + R_{C}\right) \right] + \\%
			C_L \left[ R_{C}\left( R_E + R_L + r_o\right) + R_L\left(R_E + r_o\right)\right]
		\end{array}\right\} + \nonumber\\[5mm]
%
	&+ C_L \dfrac{R_CR_LR_E}{R_B}\left[ 1 + \dfrac{r_{\pi}}{R_E} + r_o r_{\pi}\left(\dfrac{1}{R_L} + \dfrac{1}{R_C}\right) \left( g_m + \dfrac{1}{r_o} + \dfrac{1}{r_{\pi}} + \dfrac{1}{R_E}\right)\right]
\end{align}

\begin{equation} D_0 = \left(R_{C} + R_E + r_o\right) + \left(R_{B1} + R_{B2}\right) r_o r_{\pi}\left[R_E\left(g_m  + \dfrac{1}{r_o}\right) + \left(1 + \dfrac{R_E}{r_{\pi}}\right)\left(1 + \dfrac{R_C}{r_o}\right)\right] \end{equation}

\small
\begin{equation} N_2 = C_LC_B r_or_\pi \left\{ R_{C} \left[ R_E \left( g_m + \dfrac{1}{r_\pi} \right) + \left(1 + \dfrac{R_E}{r_\pi}\right)\left(1 + \dfrac{R_L}{r_o}\right)\right] + R_L \left[R_E \left( g_m + \dfrac{1}{r_o} + \dfrac{1}{r_{\pi}}\right) + 1\right] \right\}\end{equation}
\normalsize

\begin{equation} N_1 = C_B \left\{R_{C}\left( R_E + r_{\pi}\right) + r_\pi r_o \left[ R_E \left(g_m + \dfrac{1}{r_o} + \dfrac{1}{r_{\pi}} \right) + 1\right] \right\} \end{equation}

	\noindent with $R_B = R_{B1}//R_{B2}$, that is, $R_B^{-1} = R_{B1}^{-1} + R_{B2}^{-1}$. To obtain an analytical expression, we further simplify this expression by adopting $r_\pi,\ r_o\to\infty$ and

\begin{equation}
\hspace{-2mm}\mathbf{A} = \left[\begin{array}{cccc}
	 - \dpo C_B - \dfrac{1}{R_{B1}} - \dfrac{1}{R_{B2}} & 0 & 0 & 0  \\[5mm]
	g_m & - g_m - \dfrac{1}{R_E} & 0 & 0 \\[5mm]
	-g_m & g_m & - \dfrac{1}{R_C} - \dpo C_L & \dpo C_L \\[5mm]
	0 & 0 & \dpo C_L & -\dpo C_L - \dfrac{1}{R_L}
\end{array}\right].
\end{equation}

	If we further assume no load ($R_L\to\infty$ and $C_L\to 0$) then the last equation \eqref{eq:transistor_node_eqs} is lost because node 4 is islanded and the approximate equations become

\begin{equation}
\left\{\begin{array}{l}
	(1):\ \left(V_i - V_1\right) \dpo C_B - \dfrac{V_1}{R_{B1}} - \dfrac{V_1}{R_{B2}} - \dfrac{V_1 - V_2}{r_\pi} = 0 \\[5mm]
	(2):\ \dfrac{V_1 - V_2}{r_\pi} + g_m \left(V_1 - V_2\right) - \dfrac{V_2}{R_E} + \dfrac{V_3 - V_2}{r_o} = 0 \\[5mm]
	(3):\ -g_m \left(V_1 - V_2\right) - \dfrac{V_3 - V_2}{r_o} - \dfrac{V_3}{R_C} = 0
\end{array}\right. \label{eq:transistor_node_eqs_noload}
\end{equation}

	\noindent and in matrix form

\begin{equation}
\left[\begin{array}{ccc}
	 - \dpo C_B - \dfrac{1}{R_{B1}} - \dfrac{1}{R_{B2}} & 0 & 0  \\[5mm]
	g_m & - g_m - \dfrac{1}{R_E} & 0 \\[5mm]
	-g_m & g_m & - \dfrac{1}{R_C}
\end{array}\right]
\left[\begin{array}{c} V_1 \\[3mm] V_2 \\[3mm] V_3 \end{array}\right] = \left[\begin{array}{c} -\dpo C_B \\[3mm] 0 \\[3mm] 0 \end{array}\right]V_i, \end{equation}

	\noindent and we can solve directly for $V_1$:

\begin{equation} V_1 = \left( \dfrac{\dpo C_B}{\dpo C_B + \dfrac{1}{R_{B1}} + \dfrac{1}{R_{B2}}}\right)\left[V_i\right] \end{equation}

	\noindent and retro-substituting on the second equation

\begin{equation} g_m v_1 - \left( g_m + \dfrac{1}{R_E}\right)V_2 = 0 \Rightarrow V_2 = \left(\dfrac{g_m}{g_m + \dfrac{1}{R_E}}\right) \left( \dfrac{\dpo C_B}{\dpo C_B + \dfrac{1}{R_{B1}} + \dfrac{1}{R_{B2}}}\right)\left[V_i\right] \end{equation}

	\noindent and from the third equation

\begin{align}
	V_3 &= R_Cg_m\left(V_2 - V_1\right) \nonumber\\[5mm]
%
	&= R_Cg_m\left(\dfrac{g_m}{g_m + \dfrac{1}{R_E}} - 1\right) \left( \dfrac{\dpo C_B}{\dpo C_B + \dfrac{1}{R_{B1}} + \dfrac{1}{R_{B2}}}\right)\left[V_i\right] = -\left(\dfrac{\dfrac{R_Cg_m}{R_E}}{g_m + \dfrac{1}{R_E}}\right) \left( \dfrac{\dpo C_B}{\dpo C_B + \dfrac{1}{R_{B1}} + \dfrac{1}{R_{B2}}}\right)\left[V_i\right] \nonumber\\[5mm]
%
	&= -\left(\dfrac{R_Cg_m}{g_mR_E + 1}\right) \left( \dfrac{\dpo C_B}{\dpo C_B + \dfrac{1}{R_{B1}} + \dfrac{1}{R_{B2}}}\right)\left[V_i\right] = -\left(\dfrac{R_C}{R_E + \dfrac{1}{g_m}}\right) \left( \dfrac{\dpo C_B}{\dpo C_B + \dfrac{1}{R_{B1}} + \dfrac{1}{R_{B2}}}\right)\left[V_i\right].
\end{align}

	Finally, we suppose $g_m^{-1} \ll R_E$ and a direct input $C_B\to\infty$, yielding

\begin{equation} V_1 = V_i,\ V_2 = V_i,\ V_3 = -\dfrac{R_C}{R_E} V_i \end{equation}


% ---------------------------------------------------------
\chapter{Discussion and conclusion}\label{chapter:discussion_conclusion}
% ---------------------------------------------------------

	The discussions presented in this chapter were mostly raised by the commitee members at the defense of this thesis. Particualr emphasis was given to the inception of Dynamic Phasors themselves and the relationship between Nonstationary Sinusoids and their representation in complex domain.

	No great discussions were made from chapter \ref{chapter:choice_apparent_frequency} onwards, since this second part of the thesis is new in the literature. Thus, regarding these chapters, this discussion will concentrate on asserting the contributions and novelties of these chapters.

%-------------------------------------------------
\section{On the proposed Dynamic Phasor representation}\label{sec:discussion_proposed_representation} %<<<1

	As discussed in the introduction, the idea of a phasorial transformation that does not rely on integral transformations is not new in the literature; to this author's best knowledge, there are two widely accepted theories: Venkatasubramanian's ``low-pass phasors'', as defined in \cite{Venkatasubramanian1994} and the Shifted Frequency Analysis of Zhang, Martí and Dommel as presented in \cite{zhangSynchronousMachineModeling2007}. The purpose of this discussion is to assert the nature of phasorial transforms and show that the theory presented in this thesis unifies and generalizes these two strategies.

%-------------------------------------------------
\subsection{Comments on definition \ref{def:sinusoid_dynamic} of sinusoids}\label{subsec:comments_def_sin} %<<<2

	Once definition \ref{def:sinusoid_dynamic} is shown, naturally one wonders what are the restrictions that need to be imposed upon a signal $x(t)$ so that it can be written in the form $m(t)\cos\left(\theta(t)\right)$ and that there exists one $\omega(t)$ such that \ref{eq:apparent_angle_def} has a solution $\phi(t)$ — in other words, how to exactly classify the class of generalized sinusoids? Indeed, it looks like the feasibility of a generalized sinusoidal representation hinges on requiring the conformity of the considered signal into a certain structure — that it looks like a ``bent'' sinewave with time-varying parameters — which would reduce the application of this definition.

	Formally, given a signal $x(t)\in\left[\mathbb{R}\to\mathbb{R}\right]$ then it is a generalized sinusoid if there is a complex signal $f(t)\in\left[\mathbb{R}\to\mathbb{C}\right]$ such that $x(t) = \Re\left[f\right]$. In other words, $f$ must be of the form

\begin{equation} f(t) = x(t) + jy(t) \label{eq:complex_extension}\end{equation}

	\noindent for some $y\in\left[\mathbb{R}\to\mathbb{R}\right]$. We call $f(t)$ a \textbf{complex generator function} of $x(t)$; if such a function exists, then we can adopt $m(t),\theta(t)\in\left[\mathbb{R}\to\mathbb{C}\right]$ as the amplitude and argument of $f$, as in $f(t) = m(t)e^{j\theta(t)}$. Furthermore, $x(t)$ admits a representation at some apparent frequency $\omega(t)$ if there exists a solution to 

\begin{equation} \phi(t) = \theta(t) - \psi(t),\ \psi(t) = \int_0^t \omega(s)ds. \end{equation}

	If this is true, then the Dynamic Phasor of $x(t)$ at the apparent frequency $\omega(t)$ is $X(t) = f(t)e^{-j\psi(t)}$ so that $x(t) = \Re\left[X(t)e^{j\psi(t)}\right]$. Immediately one notices that these simple relationships directly yield the Dynamic Phasor Transform: given $x(t)$ and its complex generator $f(t)$, one can define

\begin{equation} X(t) = \mathbf{P_D^{\omega}}\left[x\right] = f(t)e^{-j\psi(t)},\ x(t) = \mathbf{P_D^{-(\omega)}}\left[X\right] = \Re\left[f\right] = \Re\left[X(t)e^{j\psi(t)}\right] .\end{equation}

	In other words, the Dynamic Phasor Transform is in essence a rotation of the stationary complex space by an angle $\psi(t)$. The Dynamic Phasor of a quantity is the projection of said quantity onto the rotated space. This rotation is naturally only possible in the complex space, explaining the need to transform a real signal $x(t)$ into a complex generator $f(t)$.

	The question then becomes: \textbf{what are the restrictions needed on $x(t)$ so that such a $f(t)$ exists?} At a first glance, from the definition, I could not find such restriction: however contrived an example of a signal I came up with, I could always find a sinusoidal representation even though the resulting frequency $\omega(t)$ became equally obscure and exotic as the original signal. For an arbitrary signal $m(t)$, one can (rather lazily) adopt $\omega(t) = 0$ and write $x(t) = m(t)\cos\left(0\right)$; the entirety of the theory presented is possible at a null frequency signal. Alternatively, one can admit an arbitrary frequency signal $\omega(t)$ so that $\phi(t)$ will equal $-\psi(t)$. Therefore, apparently, any continuous signal admits such a representation.

	Naturally this conclusion stems from the fact that I had theoretical signals which can be manipulated or ``mathematically compelled'' to admit the form of a generalized sinusoid, a process which cannot be undertaken in a real-time signal processing scenario, or for an arbitrary real signal $x(t)$ even if its form is known. One then asks what would be a possible generalized sinusoid representation of a signal $x(t)$ given as a time series of some sampled or measured signal (here we assume $x(t)$ is continuously measured and not discretely, for this would imply the usage of discrete versions of the transforms involved which is not in the scope of this thesis). Also, one asks whether if given an expression of $x(t)$ one can find a procedural algorithm to create the complex generator $f(t)$. One could for instance, adopt the analytical signal by means of a Hilbert Transform, that is,

\begin{equation} f(t) = x_a(t) = x(t) + j\mathbf{H}\left[x\right]\label{eq:hilbert_complex_gen}\end{equation}

	\noindent and notably this function conforms to \eqref{eq:complex_extension}. Also notably, by adopting the complex generator function of \eqref{eq:hilbert_complex_gen} we arrive at the Shifted Frequency Analysis of \cite{zhangSynchronousMachineModeling2007}, who define a \textit{Shifted Frequency representation} of a low-pass signal $s(t)$ as

\begin{equation} \left<s\right>(t) = \left\{s(t) + j\mathbf{H}\left[s\right]\right\}e^{-j\omega_0 t} \end{equation}

	\noindent where $\omega_0$ is the carrier frequency in signal analysis or synchronous frequency in Power Systems. We therefore conclude that the generalized sinusoidal representation and definition presented in this thesis generalizes the SFA representation because the representation hereby proposed not only allows for time-varying frequency signals but also does not require any particular restrictions on the spectrum of the signal being transformed, that is, \eqref{eq:hilbert_complex_gen} yields the Dynamic Phasor

\begin{equation} X(t) = \left\{x(t) + j\mathbf{H}\left[x\right]\right\}e^{-j\psi(t)},\ \psi(t) = \int_0^t \omega(s)ds \label{eq:hilbert_complex_gen_phasor}\end{equation}

	\noindent and one notes that this transformation satisfies all the characteristics of the Dynamic Phasor Transform. Naturally, the adoption of the complex generator \eqref{eq:hilbert_complex_gen} and the incurring Dynamic Phasor \eqref{eq:hilbert_complex_gen_phasor} depend on if $x(t)$ admits a Hilbert Transform; there are several results that make us inclined towards the conclusion that this process is feasible for for most practical time series and expressions of interest: for instance, it is well-known that any function with compact support has a well-defined Hilbert Transform. If $x(t)$ does not have compact support, it is also a known fact in analysis \cite[p.~320]{grafakosClassicalFourierAnalysis2014} that any $p$-measurable function admits such a transform and that $\mathbf{H}\left[\cdot\right]$ is a bounded operator in $L^p\left(\mathbb{R}\right)$. In simpler terms, if there exists some $p\in\left(1,\infty\right)$ such that the $p$-norm of $x$ is finite, that is,

\begin{equation} \left\lVert x\right\rVert_p = \left[\int_{\mathbb{R}}\left\lvert x(t)\right\rvert^p dt\right]^{\frac{1}{p}} < \infty ,\end{equation}

	\noindent then there exists a constant $C_p$ such that

\begin{equation} \left\lVert \mathbf{H}\left[x\right]\right\rVert_p \leq C_p \left\lVert x\right\rVert_p \label{eq:hilbert_boundedness}\end{equation}

	\noindent where $C_p \leq 2p$ for $2\leq p< \infty$ and $C_p \leq 2p/(p-1)$ for $1 < p \leq 2$. Thus the Hilbert Transform is a linear bounded operator in the $L^p\left(\mathbb{R}\right)$ space, and this means it ``\textbf{exists almost everywhere}'' — in simpler terms, the property of having a Hilbert Transform defines a very large class of functions which probably contains most practical signals, in turn meaning that most practical signals probably admit a generalized sinusoidal representation. For $p=1$, \cite{grafakosClassicalFourierAnalysis2014} shows that \eqref{eq:hilbert_boundedness} is not true but there is one version admitting a weaker space, by proving that if $x\in L^1\left(\mathbb{R}\right)$ then there exists $C_1$ such that

\begin{equation} \left\lVert \mathbf{H}\left[x\right]\right\rVert_{\left(1,\infty\right)} \leq C_1 \left\lVert x\right\rVert_1 \label{eq:hilbert_boundedness_p1}\end{equation}

	\noindent where $\left\lVert\cdot\right\rVert_{\left(1,\infty\right)}$ is the norm of the weak Lebesgue space $L^{1,\infty}$.

	We conclude that (at least outwardly) the admission of a sinusoidal representation for an arbitrary signal is a quite permissive property because it is possible for a large class of signals, thus requiring loose restrictions. For the intents of applications and modelling, some \textit{conceptual} restrictions might be applied. Particularly for Electrical Power Systems, we are of course assuming that $\omega(t)$ is ``close to'' or does not ``deviate much from'' a certain synchronous frequency $\omega_0$. In formal terms, we are supposing that the deviation $\Delta\omega = \omega(t) - \omega_0$ is bounded and is kept reasonably small throughout the entire timespan under consideration. We can also intuitively assume $m(t)$ and $\omega(t)$ are defined positive, to avoid the counterintuitive notions of a ``backwards spin'' (negative frequency) or a ``negative size'' (negative amplitude).

	At the end of day, it looks like most signals of interest conform to the class of generalized sinusoids for the requirements to such representation are rather weak. In simpler terms, most signals of interest (both in real-time processing and in modelling and simulation) admit a sinusoidal representation. Notwithstanding this fact, this does not categorically mean such restrictions do not exist. In the name of mathematical cautiousness, when we assume a signal admits a sinusoidal representation we will say so explicitly as in "\textbf{assume $x(t)$ has a sinusoidal representation}", however weak this assumption is.

\begin{definition}[Admissibility of a sinusoidal representation] A signal $x(t)\in\left[\mathbb{R}\to\mathbb{R}\right]$ \textbf{admits a sinusoidal representation} if there exist functions $m(t),\ \theta(t)$ such that $x(t) = m(t)\cos\left(\theta(t)\right)$. Additionally, $x(t)$ admits a sinusoidal representation \textbf{at the frequency $\omega(t)$} if there exists a solution $\phi$ to $\phi(t) = \theta(t) - \psi(t),\ \psi(t) = \int_0^t \omega(s)ds$.

	Equivalently, $x(t)$ admits a sinusoidal representation if there exists a complex generator function $f(t)$ of $x(t)$, that is, there exits a $f(t)\in\left[\mathbb{R}\to\mathbb{C}\right]$ such that $x(t) = \Re\left[f(t)\right]$. The signal $x(t)$ then admits a representation at $\omega(t)$ if $f(t)$ is such that there exists a solution $\phi$ to $\phi(t) = \arg\left[f(t)\right] - \psi(t)$.
\end{definition}

%-------------------------------------------------
\subsection{The 3$\phi$ DPT and the single-phase variant} %<<<2

	We immediately notice that the three-phase variant of the Dynamic Phasor transform does not need this entire sophistication of complex expansion to exist; the three-phase transform feels, in some way, much more \textit{natural} than its single-phase counterpart. For instance, we do not need to consider whether the three-phase signal $\left[x_a(t),x_b(t),x_c(t)\right]^\transpose$ admits a phasorial representation, or if there exists some complex generating function; the transformations involved can be applied for an arbitrary three-phase quantity, balanced or not, phasorializable or not.

	The matter of fact is that the 3$\phi$ DPT is, in essence, two linear matrix transformations in tandem (the Clarke transform followed by the Park transform), as per definition \ref{def:dq0_transform}. As such, these transformations can be applied to any three-phase signal, at any apparent frequency chosen. Because the transformations are invertible and linear, the 3$\phi$DPT is also naturally invertible and linear. These characteristics are extensively explored in \cite{orourkeGeometricInterpretationReference2019}, where the Clarke-Park Transform is explored as a geometric perspective transformation in the three-dimensional space of functions.

	The naturality of the three-phase transform stems from the simple fact that by definition the three-phase signals have dimension three, and the 3$\phi$DPT is a literal transformation, that is, it has a three-dimensional input and a three-dimensional output. What is more, again, this transformation is linear and always invertible, making it very simple to deploy. No information is generated or lost; the benefit, however, is that for specific signals of interest, the particular quantities generated are naturally represented by a phasorial two-dimensional quantity and a zero-sequence component that is very conveniently null for balanced signals and excitations that comprise most analyses.

	On the single-phase case, however, in order to produce a complex phasorial quantity, we must create two dimensions from the single-dimensional input, which means information is somehow created. It is a consequence of this process that the justification and mathematical background for this creation must be precise and solid, because the creation of the extra dimension must be done carefully to maintain the properties and qualities that we want; to this extent, subsections \ref{subsec:comments_def_sin} and \ref{subsec:justifying} go to great lengths to show that this process can be done with mathematical rigour and aligns with the intuition and conception of Dynamic Phasors — complex, time varying functions represented with respect to a rotating frame on the complex plane.

	This explains why, historically, the three-phase transformations were developed earlier than the single-phase ones; as a matter of fact, the Clarke-Park transforms were quickly adapted for higher numbers of phases. For instance, six-phase DQ transformations were used to model and control six-phase and dual-three-phase machines; modernly they are used in space-vector control of PWM drives \pcite{gloseContinuousSpaceVector2016}. An expansion of these transformations for an arbitrary number of phases also exists, as developed in \cite{janaszekExtendedClarkeTransformation2016}.

	As a matter of fact, the theorems of this thesis were all first developed for the three-phase case and then for the single-case one; in the text, however, the latter is shown first because from a study and development perspective it is only natural that a transformation is first developed in single dimension and then expanded to higher dimensions.

%-------------------------------------------------
\subsection{Justifying the complex generator function and the frequency arbitrariety}\label{subsec:justifying} %<<<2

	From the dicussion of last subsection it is natural to ask whether the sophistication of obtaining some complex projector function $f(t)$ makes sense, as it apparently makes analyses harder. Reestated, the fact that in order to have Dynamic Phasors we need to ``create an additional dimension'', as the real signal $x(t)$ needs to be expanded to a complex $f(t)$. This is a natural and justified question because the subsection clearly shows that the existence of such a function for an arbitrary signal gets quite complicated, requiring quite wordly tools such as the Hilbert Transform.

	The intent of chapter \ref{chapter:dynamic_phasor_theory} on the inception of the proposed Dynamic Phasor Theory was to show that this process, while admittedly startling at first, does pay off. For instance, because we can represent the initial signal $x(t)$ as some expanded complex representative $f(t)$, we can apply modifications to the complex space onto $f(t)$ so that the equivalent quantities in the modified space are better applicable to our problems of interest; specifically for the intents of this thesis, we want to represent generalized sinusoids as Dynamic Phasors which are a convenient way to define amplitude and phase of such signals.

	The representation of signals in the complex domain naturally opens the way for a set of transformations only available to complex numbers; for instance, we can rotate the complex space in \textit{just the right way} (i.e., by the specific angle $\psi(t)$) so that the solution of certain differential equations is ameliorated. Furthermore, the inception of a complex representative allows us the complex notions of amplitude and phase in a wider reach of complex domain rather than the simpler equivalent notions for real signals.

	In particular, it was shown that a linear and time invariant differential equation in the time domain has an equivalent linear equation in the complex domain — equivalently, given a complex system in the time domain we obtain an equivalent system in phasorial domain. This, in turn, allows us to model circuits and systems in the phasorial domain; in practice, this means we can produce models, simulations and (further along the thesis) control blocks in phasor domain rather than the time domain. To this wise, chapter \ref{chapter:dynamic_phasor_theory} shows several examples of modelling. 

	Therefore, the conception of a complex generating function is in essence the core of the Dynamic Phasor Transform, and it is exacly this tool that allows the development of the theory proposed in this thesis. Further, this function allows the rotation of the complex space and the generation of a ``dq-equivalent'' differential model from a linear system, allowing one to model a circuit in phasorial domain given the time-domain model. Also importantly, we have shown that this entire process is a generalization of the classical phasor transform, defined as the Static Phasor Operator in this thesis, in such way that the intuitive models of rotating vectors and linear transformations are maintained, albeit in a mode sophisticated mathematical environment.

	Furthermore, it was shown that the particular rotation that generates the phasorial domain maintains the ideas of active and reactive power of static phasors, thus making possible the power analyses not only in a static phasor framework (that is, in phasorial equilibrium) but also in a Dynamic Phasor context, which we used model power transfer in a circuit in transient regimens. Here, the resemblance of complex power in the classical framework and in dynamic models cannot be understated. Even the formulas are exactly the same: in classical phasors the instantaneous power is given by theorem \ref{theo:sfp_complex_apparent_power} as

\begin{equation} p(t) = P \left[1 + \cos\left(2\omega t + 2\phi_v\right) \right] + Q\sin\left(2\omega t + 2\phi_v\right)\end{equation}

	\noindent where the complex power is defined as

\begin{equation} S = \frac{1}{2}\left<V,I\right> = P + jQ\ \left\{\begin{array}{l} P = \dfrac{m_vm_i}{2}\cos\left[\phi_v - \phi_i\right] \\[3mm] Q = \dfrac{m_vm_i}{2}\sin\left[\phi_v - \phi_i\right] \end{array}\right. \end{equation}

	\noindent where in the Dynamic Phasors domain these formulas are just the time-varying counterparts, as per theorem \ref{theo:activereactivepower}:

\begin{gather}
	p(t) = P(t) \left[1 + \cos\left(2\psi + 2\phi_v\right) \right] + Q(t) \sin\left(2\psi + 2\phi_v\right) \\[3mm]
	S(t) = \frac{1}{2}\left<V(t),I(t)\right> = P(t) + jQ(t)\ \left\{\begin{array}{l} P(t) = \dfrac{m_v(t)m_i(t)}{2}\cos\left[\phi_v(t) - \phi_i(t)\right] \\[3mm] Q(t) = \dfrac{m_v(t)m_i(t)}{2}\sin\left[\phi_v(t) - \phi_i(t)\right] \end{array}\right. .
\end{gather}

	Finally, the physical meanings and interpretations of the active and reactive power components are also maintained. By theorem \ref{corollary:sfp_active_average_power}, the active power $P$ in a static phasor environment is the average power over a period $T$:

\begin{equation} \dfrac{1}{T} \int_{t}^{t + T} v(x)i(x)dx = P \end{equation}

	\noindent and in a DP framework the same formula holds — albeit naturally now $T$ is time-variable: theorem \ref{theo:activepowerperiod} defines that the active power $P(t)$ is the average over $T(t)$:

\begin{equation} \dfrac{1}{T(t)}\int_{t}^{t+T(t)} p(s)ds = P(t) .\end{equation}

	Finally, much like theorem \ref{corollary:direct_quad_current} shows that in static phasor the active power accounts for the portion of current in  phase with voltage and the reactive power accounts for the portion in quadrature with voltage, as in

\begin{equation} i(t) = \dfrac{2P}{m_v}\cos\left(\omega t + \phi_v\right) + \dfrac{2Q}{m_v}\sin\left(\omega t + \phi_v\right) .\end{equation}

	\noindent the same exact phenomenon happens in Dynamic Phasors, as per theorem \ref{theo:direct_quad_current_nonst}:

\begin{equation} i(t) = \dfrac{2P(t)}{m_v(t)}\cos\left(\psi(t) + \phi_v(t)\right) + \dfrac{2Q(t)}{m_v(t)}\sin\left(\psi(t) + \phi_v(t)\right) .\end{equation}

%-------------------------------------------------
\subsection{Generalization of ``phasor calculus''} %<<<2

	Due to the extensive results on circuit analysis and complex power, we also note that the theory defined in this thesis also generalizes the ``low-pass phasor calculus'' as presented by Venkatasubramanian:

\begin{quotation}
	``\textit{\textbf{Property 4} (1) (Capacitor): The current $i_C(t)$ flowing through a capacitor $C$ with the terminal voltage $v_C(t)$ can be represented by}

\begin{equation} \hat{i}_C(t) = C\dfrac{d}{dt}\hat{e}_C(t) + j\omega_c C\hat{e}_C(t) \end{equation}

	\noindent \textit{in the phasor domain (...). (2) (Inductor): The voltage $e_L$(t) across an inductor $L$ when the $i_L(t)$ is flowing through, can be represented by}

\begin{equation} \hat{e}_L(t) = L\dfrac{d}{dt}\hat{i}_L(t) + j\omega_c L\hat{i}_L(t) \end{equation}

 	\noindent \textit{in the phasor domain (...).} ''\hfill\cite{Venkatasubramanian1994}
\end{quotation}

	Notably, these formulas are the same formulas that stem from the theory of this thesis, as per theorems \ref{theo:1p_capacitive_conductance} and \ref{theo:1p_inductive_impedance} where the capacitive and inductive relationships of the Dynamic Phasors $V$ of voltage and $I$ of current of the capacitive and inductive elements are given by

\begin{gather}
	I = C\dfrac{dV}{dt} + j\omega C V \\[3mm]
	V = L\dfrac{dI}{dt} + j\omega L I
\end{gather}

	\noindent but, in the case of these formulas, $\omega$ can be time-varying and the relationships do not require the time-domain signals to have limited spectrum. Similar relationships are found in Shifted-Frequency Analysis of \cite{zhangSynchronousMachineModeling2007}:

\begin{quotation}
	``\textit{To obtain the SFA equivalent circuit [of the inductor], we transform (5) using (4) to get}

\begin{equation} \mathbf{V}(t) = -\mathbf{L}\dfrac{d\mathbf{I}(t)}{dt} + j\omega_s \mathbf{LI}(t)\end{equation}

	\noindent \textit{where $\mathbf{V}(t)$ and $\mathbf{I}(t)$ are the dynamic phasor vectors corresponding to the physical time vectors $v(t)$ and $i(t)$, respectively.}'' \hfill\cite{zhangSynchronousMachineModeling2007}
\end{quotation}

	Therefore, we conclude that the theory presented in this thesis generalizes the results by both \cite{Venkatasubramanian1994} and \cite{zhangSynchronousMachineModeling2007}, by not requiring the signals considered to be in a low-pass spectrum or a fixed apparent frequency. This is illustrated in example \ref{example:rlc_dpt}, where the excitation \eqref{eq:example_voltage_freq_def} adopted has a infinitely wide spectrum yet the theory is able to deal with this excitation with ease. Because of this ease, this exact excitation is used throughout the thesis to reiterate how powerful this theory is, several times over. Furthermore, section \ref{sec:example_application} shows an example application where two excitations of infinite spectrum but different natures (one excitation is frequency modulated, the other is amplitude modulated) are adopted, and the theory is again able to operate these signals.

%-------------------------------------------------
\subsection{Consequences for Power Systems} %<<<2


	Particularly for Electric Power Systems, the complexity of Dynamic Phasor tools is also justified as they are known to considerably speed up the simulation times of differential models: for instance, \cite{laraRevisitingPowerSystems2024} states that ``\textit{Tools such as the EMTP [Electromagnetic Transients Program] work with instantaneous time variables and can continuously trace the evolution of the system state. These tools, however, require small discretization steps dictated by the need to trace the instantaneous values of all waveforms. This makes the EMTP unnecessarily slow to trace phenomena around the 60-Hz fundamental frequency}'', whereas the integration time steps are enlarged using the proposed SFA tehcnique, speeding up simulation time: ``\textit{The main advantages of the proposed shifted frequency analysis (SFA) model are realized when the frequencies in the simulation are close to 60 Hz, which allows, after frequency shifting, the use of large integration steps.}''

	Moreover, the fact that the apparent frequency must be ``chosen'' is also confusing, for a couple reasons. First, that this choice is arbitrary, and second, that a signal $x(t)$ might admit a phasorial representation against multiple (or even infinite) apparent frequency functions, thus one of such must be chosen. At this point one infers that the arbitrariness of $\omega(t)$ brings some problems at no benefit; in reality, it serves multiple purposes. In Power Systems, the many agents of the grid have a local measurement of frequency, and most will be equipped with frequency control dependend on active power output (known as Droop control) so that the frequency of the generated sinewave is transiently adjusted, therefore being a choice of the local agent by its controller; in this sense, the frequency signal $\omega(t)$ must be \textit{chosen}, hence why it is defined as arbitrary for now in the sense that the operator has to choose a signal, that is, the representation depends on the existence of some frequency signal that is preemptively defined.

	The problem of many (or infinite) possible apparent frequency choices is not new in the literature; for example, Venkatasubramanian debutes his linear operator-based approach by making an argument that for Electric Power Systems, the choice of the synchronous frequency is not only convenient but also plays on the well-definiteness of the problem of multiple possible frequencies:

\begin{quotation}
\textit{``In fact, it is easy to construct an infinite number of different time-varying phasors which all satisfy (3) [the definition of a Dynamic Phasor], but they are not of practical interest. The point is that in general, an explicit representation of the form $e_o$(t) [a generalized sinusoid at the frequency $\omega_c$] is not available for modulated signals, and the problem of finding the phasor is not mathematically meaningful unless we impose some assumptions to tighten the field of possible candidates to the practically interesting ones. For instance, it can be observed that the degenerate phasors (...) associated with the signal $e_o$(t) all have their bandwidths greater than or equal to $\omega_c$ [the carrier frequency] whereas the bandwidth of the degenerate phasors is strictly less than $\omega_c$. In other words, if we restrict the choice of phasors to those with bandwidths less than the carrier frequency $\omega_c$, then [the produced Dynamic Phasor] is the unique phasor associated with $e_o(t)$ and the problem is well-defined.}\hfill\pcite{Venkatasubramanian1994}
\end{quotation}
\vspace{3mm}
	
	Even then, as exposed in the introduction of this thesis, both the SFA technique by \cite{laraRevisitingPowerSystems2024} and the linear operator approach of \cite{Venkatasubramanian1994} require the signals under study to have a spectrum limited to a bandwidth around the synchronous frequency; however, in the representation proposed in this thesis and the entire theory that stems from it, no such requirement is made — reestated, the theory hereby proposed poses an expansion of such theories by requiring no conformities or restrictions on the characteristics of the signal.

	Considering this fact, this theory can model phenomena unavailable to these past theories by not only not restricting the signals but also not requiring a particular frequency value: for instance, a network with multiple agents will include many frequency signals, and each agent will represent the grid by a particular time-varying model that is used in its sensors, controllers and estimators. The definition hereby proposed aims to offer a ``wiggle room'', in the form of the arbitrariness of $\omega(t)$, so that the same grid can be modelled using multiple frequency signals. It will be proven in this thesis that the particular model of each agent is ``equivalent'' in some sense — which is only natural seen as the grid is the same for all agents after all. A ``common frequency'' representation can be chosen, however; in most Power System studies, the Center of Angle (CoA) is chosen as the pure average (or weighted average) of the frequency signals of the agents; ideally, the grid eventually achieves \textit{consensus} — loosely defined in Power Systems as the agents converging to a common frequency after some disturbance — thus reaching a steady-state value.

	This ``common frequency'' however can also be time-varying, prompting a wide and embracing definition. If the agents of the grid achieve consensus, then the grid achieves a steady-state frequency that deviates from the synchronous frequency depending on the load state of the system. The adjustment of this steady-state frequency generally depends on a collaborative and/or centralized control that adjusts the power setpoint of the agents, and this control acts on the slow bandwidth timescale — generally tens of seconds or even minutes. 

	Moreover, another interesting aspect of the arbitrariness of $\omega(t)$ comes from a modelling and a numerical standpoint and allowing particular choices of frequency signals at the discretion of the user, engineer, mathematician, or whoever is fortunate enough to use this theory. Seen as the representations of a particular linear system using two distinct frequency signals are equivalent, one asks what is the ``most convenient'' representation. Naturally, for modelling purposes, one might think it would be easier to adopt $\omega(t) = \omega_0$ the synchronous frequency, which makes sense in a ``slack'' or synchronous reference frame. Engineers interested in a simulatory and numerical approach will also ask what is the frequency signal that yields simpler differential models that can make simulation easier or faster by either reducing complexity or allowing for larger integration timesteps.

	By allowing any choice of apparent frequency, the theory proposed in this thesis embraces and generalizes the current literature definitions, like those of \cite{laraRevisitingPowerSystems2024} and \cite{Venkatasubramanian1994}, that define nonstationary sinusoids as always defined at the synchronous frequency. This definition is widespread in the Elecric Power System literature, and was standardized in the IEEE Standard C37.118.1-2011 where a nonstationary sinusoid is defined in page 6 as a signal 

\begin{equation} x(t) = X_m(t) \cos\left(2\pi \int f dt + \phi\right).\end{equation}

	The standard also defines a nominal frequency $f_0$ and a frequency deviation signal $g$:

	\textit{In the general case where the amplitude is a function of time $X_m(t)$ and the sinusoid frequency is also a function of time $f(t)$, we can define the function $g = f - f_0$ where $f_0$ is the nominal frequency and $g$ is the difference between the actual and nominal frequencies (note that $g$ will also be a function of time, e.g., $g(t) = f(t) - f_0$. The sinusoid can then be written as}

\begin{align}
	x(t) &= X_m(t) \cos\left(2\pi \int f dt + \phi\right) \nonumber\\[3mm]
	&= X_m(t) \cos\left[2\pi \int \left(g + f_0\right) dt + \phi\right] \nonumber\\[3mm]
	&= X_m(t) \cos\left[2\pi f_0 t + \left(2\pi \int gdt + \phi\right)\right] \label{eq:synchrophasor_time}
\end{align}

	\textit{The synchrophasor representation for this waveform is:}
\begin{equation} X(t) = \dfrac{X_m(t)}{\sqrt{2}} e^{j\left(2\pi \int gdt + \phi\right)} . \label{eq:synchrophasor_complex}\end{equation}

	Immediately one notices that choosing $\omega(t) = \omega_0 = 2\pi f_0$ constant on \eqref{eq:apparent_angle_def} yields this precise definition; as a consequence, the definition proposed generalizes this synchrophasor representation.

	Further, in page 8, the standard defines ``frequency'' as follows. Given a signal $x(t) = X_m(t)\cos\left(\theta(t)\right)$ measured by a Phasor Measurement Unit (PMU) or a realtime measurement device like a scope, the frequency is

\begin{equation} f(t) = \dfrac{1}{2\pi} \dfrac{d\theta}{dt} \label{eq:synchrophasor_complex_frequency}\end{equation}

	\noindent which differs significantly from the apparent frequency definition proposed. This is because the definition of the standard assumes that the phase $\phi$ is constant, where the proposed definition is more generalized. To this extent, we can define an equivalent definition of the \textbf{absolute frequency} $\eta$ as the derivative of the absolute angle $\theta$ in \eqref{eq:apparent_angle_def}:

\begin{equation} \eta(t) = \dfrac{d\theta}{dt}(t) = \omega(t) + \dfrac{d\phi}{dt}(t)\end{equation}

	\noindent which is equivalent to the frequency definition \eqref{eq:synchrophasor_complex_frequency} of the standard. This is therefore related to the apparent frequency and phase as

\begin{equation} \int_{t_0}^t \eta(s)ds = \phi(t) + \int_{t_0}^t \omega(s)ds \Rightarrow \phi(t) = \int_{t_0}^t \left[\eta(s) - \omega(s)\right]ds . \end{equation}

	Moreover, to make a frequency variation analysis the standard defines the Rate of Change of Frequency (ROCOF) quantity as

\begin{equation} \text{ROCOF}(t) = \dfrac{df(t)}{dt} .\end{equation}

	Then, because the standard defines synchrophasors as quantities computed in relation to a nominal frequency $f_0$, then the argument can be represented as $\theta(t) = 2\pi f_0 + \phi(t)$, and the frequency becomes $f(t) = f_0 + \Delta f(t)$, this latter term a frequency deviation and the ROCOF becomes

\begin{equation} \text{ROCOF}(t) = \dfrac{d}{dt} \left[\Delta f(t)\right].\end{equation}

%-------------------------------------------------
\section{Frequency effects on the Dynamic Phasor Transform}%<<<1

	The contributions of chapter \ref{chapter:choice_apparent_frequency} are entirely new in the literature; the main contribution, as outlined in the introduction, is the proof of the Quasi-Static Hypothesis, leading to and justifying Quasi-Static Models as per theorem \ref{theo:qsh_linear_circuits}.

	While the inner workings of this theorem are somewhat complicated as based on theorem \ref{theo:qsh_approx_nonlinivps} by \cite{Marva2012}, the application of the theorem seems rather intuitive to engineers: in essence, Quasi-Static Models are loosely defined as ``using classical phasor relationships to Dynamic Phasors'', which obviously greatly amenize the process of modelling.

	It must be noted that \cite{Venkatasubramanian1995a} shows a similar discussion on the proof of the QSH using Tikhonov's Theorem. The theorem presented is stronger, for two reasons: it also considers non-autonomous systems, and that allows it to use a more general model where forcings and frequencies are co-dependent. This makes the proof in \cite{Venkatasubramanian1995a} more limited with respect to the one shown here. That paper also discusses on the validity of the $\pi$ model (the Unified Model of subsection \ref{subsec:unified_model}), with the argument that the quasi-static approximation is naturally only as good as however well the model adopted can reflect the system dynamics but as voltages and currents on lines get quicker, electromagnetic modelling is needed to account for transmission line delay characteristics.

	Furthermore, the generalized Power System modelling of \cite{Venkatasubramanian1995a} also has simplifications, as it does not model transformations on the apparent frequency nor the dependence on frequency, forcings and states. Furthermore, the authors do not offer a phasorial theorem that explains Quasi-Static models. Moreover, the phasorial theory used is that of \cite{Venkatasubramanian1994}, which as discussed before, expects the generalized sinusoids involved to have limited spectrum.

	As such, the proof shown in this thesis not only adopts a more general and complete model of Power Systems but also uses the more generalized Dynamic Phasor theory that does not rely on specific spectrum qualities of the signals involved. As such, the proof shown here is inspired by the results of \cite{Venkatasubramanian1995a} but offers a more comprehensive and expanded proof and modelling.

	The effects of this approximation are thoroughly discussed in example \ref{example:rlc_timescales}, where the circuit parameters vary to show how a circuit with different timescales reacts to a fast-changing excitation. Section \ref{sec:omib_dynphasor_sim} shows the application of this theory to a Power System, illustrating how Quasi-Static Models fail to capture certain electromagnetic trasient phenomena of transmission grids and internal impedances of generators.

	Despite this main contribution of proving the Quasi-Static Hypothesis, this chapter also has other contributions to the overall theory of Dynamic Phasors. For instance, the chapter proves in theorem \ref{theorem:sols_are_nonst} that if a linear circuit is excited with generalized sinusoids at some frequency $\omega(t)$, then all currents and votages will also be sinusoids defined at the same frequency; furthermore, even if the excitations are defined at distinct frequencies, under mild requirements they can be written with respect to a common frequency.

	While apparently too theoretical, this chain of proofs has profound consequences in the Theory of Electrical Circuits. For instance, since all excitations, currents and voltages can be written in the same frequency, they can be geometrically drawn and compared in the same DQ frame; this justifies the commonly used phasorial diagrams for circuits. For instance, take the diagram of figure \ref{fig:dynamic_phasor_dqaxis_ibr}, where all quantities are drawn in a stationary real-imaginary frame. Because all quantities can be defined at the same frequency $\omega_P(t)$, then one can rotate the entire frame by $\psi(t)$, generating the phasorial diagram in the DQ frame \ref{fig:dynamic_phasor_dqaxis_ibr_dqframe}.

	Another contribution of this chapter is the fact that in a multi-frequency system (one where each agent inputs onto the system a forcing at a local particular frequency), the entire system can be modelled in a common frequency such that all signals can be reconstructed losslessly, as shown in theorems \ref{theo:homeomorphic_phasors} and \ref{theo:diff_freqs}, and illustrated in example \ref{example:diff_freqs}. This fact is widely used in Power Systems: most of the times the electrical grid is modelled at the synchronous frequency, and all agents forcings are rotated to this frequency so that all are modelled with respect to the same frequency. This is illustrated for instance in the example of subsection \ref{sec:omib_dynphasor_sim}, where the synchronous machine and the infinite bus input voltages at different frequencies but the system is modelled at the synchronous frequency.

%-------------------------------------------------
\section{About Dynamic Phasor Functionals}%<<<1

	While the Dynamic Phasor Theory of chapter \ref{chapter:dynamic_phasor_theory} generalizes concepts that are already timidly developed in the literature, Dynamic Phasor Functionals greatly enhance the application of the theory to produce intuitive and complete models of circuits and systems in nonstationary regimens.

	The concepts of these functionals was pointed at by Venkatasubramanian, Schättler and Zaborsky when they noted that the composition of the phasor transform — which they denoted $P(\cdot)$ — and the derivative generate yet another transformation in the complex domain:

\begin{quotation}
	\textit{``A special feature for the time-varying phasor transformation emerges when the time derivative operation is considered:}

\begin{equation} P\left(\dfrac{d}{dt}e(t)\right) = \dfrac{d}{dt}\vec{E}(t) + j\omega \vec{E}(t) \label{eq:venka_p_operator}\end{equation}

	\noindent\textit{the result simply follows from the definition of $P(\cdot)$ using the chain rule.}''\hfill\pcite{Venkatasubramanian1995a}
\end{quotation}

	However, this relationship was given as a property; the functional aspects were not considered. From this point onward, the entirety of the theory on Dynamic Phasor Functionals as developed in chapter \ref{chapter:dpos} is novel in the literature.

	The matter of fact is that these functionals, in some way, seem \textit{too good to be true}. They have very convenient algebraic properties that make them remarkably useful in modelling circuits and systems, and particularly Power Systems. Not only that, these functionals are also imbued with a topology through the norm definition \ref{def:dpnorm}, which allows producing idealized models of short and open circuits, as well as infinitely large open-loop gains — which was explored in the example of section \ref{sec:example_application}.

	More importantly, the algebraic properties of DPFs are also surprising; given the apparent sophistication of their definition, the functionals are ultimately easy to work with, being invertible, linearly combineable. The concepts of polynomials and matrices are also of great interest, because they allow the definition of Dynamic Impedances and the possibility of matrix network analysis in the Dynamic Phasor domain.

	One of the more happy aspects of Dynamic Phasor Functionals is that they are able to abstract the apparent frequency from the calculations and modelling. It is clear that without these functionals, if one were to use relationship \eqref{eq:venka_p_operator} for all circuit orders, they would inevitably be led to a rather concerning number of terms and combinations of the frequency and its derivatives. Whereas by using functionals these calculations are all abstracted and the frequency signal is only needed at the end of the model.

	The most important aspect of Dynamic Phasor Functionals for the Theory of Electric Circuits is, by far, the possibility of defining impedances in the Dynamic Phasor space, as per definition \ref{def:steinmetz_impedance}. This definition is very in line with the commonplace concept of impedances both in the classical phasors domain — where impedances are ratios of polynomials of $j\omega$ — but also impedances in the Laplace domain — where an impedance is a transfer function composed of ratios of polynomials of the Laplace frequency $s$.

	The inception of these impedance operators culminates with the Superposition Theorem (theorem \ref{theo:superposition}), which yields the Thèvenin and Norton theorems (theorems \ref{theo:thevenin} and \ref{theo:norton}). The importance of these theorems cannot be understated, because they mean that the circuit analysis in Dynamic Phasor domain is not at all different from the analysis engineers are already used to; with minimal adaptations, such as the ongoing notion that the quantities being time-varying phasors and functional operators, the modelling procedures and concepts are essentially the same as static phasors.

	It is clear that the reach and possibility of Dynamic Phasor Functionals is quite large, and they may be deployed to a large number of problems in a large number of areas. Particularly for Electrical Engineering, their application to signals, systems and controls cannot be understated. Of course, this author will develop further research into those topics.

%-------------------------------------------------
\section{About $\mu$ Transforms and the control theory in Dynamic Phasor Space}

	The control thery presented in chapter \ref{chapter:control_theory} is entirely new in the literature. The objective of this chapter is to essentially justify linear controllers in the Dynamic Phasor space, that is: there is a direct relationship between controllers in Dynamic Phasor domain and controllers in time domain, or equivalently, by controlling Dynamic Phasor quantities one also controls time quantities, and this process is justified.

	The great contribution of this thesis is to show the exact mathematical mechanism that generates this relationship between control realms. This mechanism is the $\mu$ Transform of \eqref{eq:mtransf_mu_def}, which is shown to be a generalization of the Laplace Transform in rotating complex frames. This can be seen by the simple fact that if the apparent frequency is null (the space does not rotate) then the $\mu$ Transform devolves exactly into the Laplace Transform, as well as the inverse transform devolves into the Inverse Laplace Transform as per theorem \ref{theo:xmu_reconst}.

	Albeit a generalization, the $\mu$ Transform retains the most convenient properties of the Laplace Transform: it has a Final Value Theorem (theorem \ref{theo:laplace_fvt}), it transforms differentials (DPFs) into a multiplication by the complex frequency $\mu$ (theorem \ref{theo:mu_transf_and_dpfs}) and it transforms rational systems into Transfer Functions (as per \eqref{eq:intuition_mutfs}).

	As a consequence, $\mu$ Transfer Functions ($\mu$TFs) also called Dynamic Phasor Transfer Functions (DPFTs) also generalize Laplace Transfer Functions while retaining interesting properties: a convolution operator is defineable (definition \ref{def:mut_convo}) and this operator makes $\mu$TFs a ring with respect to the space of functions (theorem \ref{theo:muT_conv_prod}); furthermore, the impulse distribution is the neutral element of the convolution (theorem \ref{theo:delta_neutral}). These facts culminate with the fact that the output of a linear system can be obtained by convolving the input with the impulse response (as per \ref{eq:impulse_response_convo}).

	However, the most important theorem of the chaper is theorem \ref{theo:bibo_mutfs}. This theorem basically defines that the concept of input-output stability, also named bounded input, bounded output (BIBO) stability is also available for $\mu$ TFs. The contribution of this theorem is that the construction of controllers using $\mu$TFs, like controllers on Laplace domain, can be made by engineering the transfer functions themselves instead of looking at the time responses directly; therefore, things like zeros-poles anlyses and stability are also maintained with minor adjustments.

	Further research is needed to deploy this theory to develop better controllers or to validate the ones that exist. Chapter  \ref{chapter:control_theory} shows that the current controller of figure \ref{fig:3p_curr_control}, example \ref{example:3p_eps_modelling} commonly used in IBR systems is not really convenient to use because of the amount of gain parameters, leading to difficult tuning. In contrast, the proposed substitute of figure \ref{fig:partial_blockmodel}, based on the proportional-integral equivalent controller in DP domain of figure \ref{fig:pi_utf_blockmodel}, is a much better candidate because its BIBO stability is guaranteed with the very simple requirement that the integral gain be positive; the rest of the tuning process can be undertaken using dynamic performance constraints. The example used loosely chosen values for the gains, but further research will be made to specify the tuning process, leading to better performance.

%-------------------------------------------------
\section{Conclusion} %<<<1

	The introduction of this thesis uses the Quasi-Static Hypothesis as a motivator for a sequence of faults in the literature of Power System modelling and control. All of the faults stem from the essential fact that in order to achieve Phasor Equivalent models of machines, the transmission grid, and the controllers employed, many approximations and assumptions are made, and extensively so, to the point of bringing into question if the models and simulations and controllers developed are valid, that is, if these elements do indeed yield verosimile results that mirror the signals and systems they intend to represent.

	The key concept is that the lack of a complete theory to represent generalized sinusoids forces the assumption that frequency swings are small in amplitude and slow in time. Under such assumption, the phasorial models \eqref{eq:machine_2a_model} and \eqref{eq:machine_2a_model_classical} are possible; coupled with the constant admittance model of the grid in \eqref{eq:multimachine_admittance}, one can model a transmission grid and its machines using approximate models that suppose (1) that the machines supply ``almost-sinusoidal'' voltages and currents and that (2) the grid circuit is much ``quicker'' than the frequency variations, so that it can be approximated for its steady-state sinusoidal behavior. This also allows the construction of power flow equations \eqref{eq:power_flow_eqs}, even though a clear and solid definition of complex power in nonstationary regimens is not available.

	Even though this assumption — formalized as the Quasi-Static Hypothesis or Modelling — seems reasonable, and for however important it is, a solid and straightforward proof that the models stemming from it are verosimile and indeed approximate quasistatic sinusoids in time is notable absent in the current literature. Moreover, for modern Power Systems this assumption is violated, requiring more involved models aimed specifically at the quicker transient phenomena these new systems can manifest.

	Moreover, the usual control systems built for Power Systems heavily draw from these approximations, like the controllers of figure \ref{fig:machine_model_controls}. However, due to the wide and deep approximations and assumptions, it becomes questionable if these controllers are really effective from a theoretical standpoint, and if their efficacy outside of the approximated models can be asserted.

	In the scope of these driving facts, this thesis achieves the initial task of offering a theory of Dynamic Phasors that allows for mathematical formalizations of the gaps in the literature that fundamentally cause the issues outlined. In a wide view, chapter \ref{chapter:dynamic_phasor_theory} dealing with the inception of Dynamic Phasor Theory formalizes the idea of generalized sinusoids as real signals as a generalization of static sinusoids by allowing time-varying amplitude, frequency and phase. It was shown, through a construction of several operators and functionals, that generalized sinusoids bear a bijectivie relationship — called the Dynamic Phasor Transform — with complex time functions called Dynamic Phasors. This allows, for instance, representing the widely used synchrophasors \eqref{eq:equivalent_emt_E}, \eqref{eq:equivalent_emt_X} and especially those as defined in the IEEE Standard C37.118.1-2011 \eqref{eq:synchrophasor_time}. Further, this chapter also shows that the Dynamic Phasor Transform achieves notions of complex, active and reactive power that generalize their static counterparts while maintaining close resemblance and physical interpretations. Therefore, this chapter successfuly achieves the first issue raised in the introduction, \textit{videlicet} the representation of generalized sinusoids as Dynamic Phasors with solid construction and physical meaning.

	It was shown that this theory allows for building models of power systems in Dynamic Phasor space, as shown by examples \ref{example:rlc_dpt} and \ref{example:rlc_dpt_power}, including three-phasor systems as in example \ref{example:3p_eps_modelling}. The models built are solid and produce phasorial quantities that losslessly reconstruct their respective signals in time, solving the second issue raised in the introduction as this allows for building models of power devices, in particular synchronous machines and transmission lines.

	In chapter \ref{chapter:choice_apparent_frequency}, it was shown that the Quasi-Static Modelling and the models derived from it indeed bear verosimilance, by proving the intuitive notion that if a circuit is quicker than its excitaions, the circuit behaves at an ``almost-sinusoidal'' state as proven by theorem \ref{theo:qsh_linear_circuits} and illustrated in example \ref{example:rlc_timescales}. Further, it was also shown that even if a circuit is imbued with several frequency or angle references, the models built in the different frames are equivalent in some way, as per theorem \ref{theo:diff_freqs}. Reestated, even if the models describe different phasors and models but they build the same time signals, as shown in example \ref{example:diff_freqs}. This shows that even though each agent in a multi-agent system has their own reference and frequency frames, their controls and models agree. This chapter justifies the constant admittance model \eqref{eq:multimachine_admittance} of power grids, as well as the power flow equations \eqref{eq:power_flow_eqs}, effectively justifying the common modelling used for power grids, as well as the validity of quasi-static models like \eqref{eq:machine_2a_model} and \eqref{eq:machine_2a_model_classical}. Furthermore, this chapter in section \ref{sec:freq_modelling_timescales} shows that any time-domain controlled system in nonstationary regimen is diffeomorphic to a phasor-domain controlled system, justifying phasorial-domain controllers for linear systems excited with generalized sinusoids.

	Further, chapter \ref{chapter:dpos} shows that the Dynamic Phasor Transform can be highly operationalized through a specific set of functionals in Dynamic Phasor space, called Dynamic Phasor Functionals (theorem \ref{theo:nth_order_relationship} and definition \ref{def:steinmetzoperator_revisited}). These functionals transform differentiation in time domain to very convenient and powerful algebraic structures (group, ring, field and vector space as proven in section \ref{subsec:notation_abuse}) that enable an entire development of circuit modelling and network analysis in Dynamic Phasor space, even without the Quasi-Static Hypothesis. By the advent of polynomials of such functionals one can define impedances in the Dynamic Phasor context (definition \ref{def:steinmetz_impedance}), and also matrices of such impedances, allowing for an admittance notation like that of the static case \eqref{eq:multimachine_admittance} but in a general case. Further, famous circuit modelling techniques have their Dynamic Phasor counterparts proven: Kirchoff's Laws (theorems \ref{theo:kirchoff_current} and \ref{theo:kirchoff_voltage}), the Superposition Principle (theorem \ref{theo:superposition}), Thèvenin's Theorem (theorem \ref{theo:thevenin}) and Norton's Theorem (theorem \ref{theo:norton}). These results show that a notion of complex admittances and network analysis is possible even if the Quasi-Static Hypothesis fails (see equation \eqref{eq:admittance_dpfs}), allowing to model modern power systems in a manner similar to the techniques already employed in classical systems.

	Finally, chapter \ref{chapter:control_theory} shows that a linear control theory is possible in Dynamic Phasor space, with some slight definitions and modifications. It is proven that an integral transform, named the Mu Transform or $\mu$T, is possible with very convenient properties — mainly that it highly resembles the Laplace Transform (as per definition \eqref{eq:mtransf_mu_def}). This transform has an inverse (theorem \ref{theo:xmu_reconst}) that allows Dynamic Phasors to be rebuilt from their transforms, like using complex poles (theorem \ref{theo:xmu_reconst_residue}). Further, Mu Transforms can produce Transfer Functions or $\mu$TFs (definition \ref{def:muT_TFs}) which again bear very close resemblance to Laplace Transfer Functions; mainly, $\mu$TFs of rational systems as stable if they are proper and Hurwitz Stable (theorem \ref{theo:bibo_mutfs}), the very same characteristic that makes Laplace Transfer functions useful. Thus, this chapter justifies linear controllers in generalized sinusoidal space, like the AVR, PSS and Droop controllers of figure \ref{fig:machine_model_controls} (which are generally designed and tuned using small-signal analyses) but can also produce better, more intuitive controllers than the current ones, as shown in subsection \ref{subsec:new_controller}.

	Ultimately, the Dynamic Phasor Theory proposed in this thesis proves to be a powerful and comprehensive theory that allows for modelling, control and simulation of electrical circuits in generalized sinusoidal regimens. Beyond Dynamic Phasor representation, the theory offers equivalents of modelling techniques and control theory that makes it applicable to a plethora of systems and circuits. The currently most used theories — the Short-Time Fourier Transform and the Hilbert Transform — are, in some way or the other, bereft of applicability as they do not fulfill one or more of the requirements initially set by Classical Phasors, while the current theory checks all the boxes.

	While this thesis offers a wide theory, the obvious challenge is applying the theory herein developed to engineering problems. Due to their intention, the examples showin in this thesis fulfill their purpose of showcasing the features of the theory develop, in so far as they illustrate its usage. Nevertheless, the examples are naturally simplistic in application and size. While it is obvious that while the theory developed can be used to model large systems, a dedicated simulatory software is needed to realize large-scale simulations in this framework.

	The theory also proves to be capable of producing models of Power Systems that can be used for simulation and stability analyses; regarding the ``classical'' Power Systems, even though the proof of the QSM shown in this thesis validates the customary synchronous machine models \eqref{eq:machine_2a_model} and \eqref{eq:machine_2a_model_classical}, it is still to be determined if these models remain the same if one uses the Dynamic Phasor Theory proposed. Much the same way, one wonders if the controllers used, like those in figure \ref{fig:machine_model_controls}, also remain or need to be adjusted in some way.

	Finally, it is also still to be determined what is the behavior of this theory when applied to nonlinear systems, especially because the development of the theory, as presented in this thesis, depends largely on the linear and time invariancy of the systems being studied. The Hartman-Gröbman Theorem guarantees that linearization of a nonlinear system around a hyperbolic equilibrium leads to an equivalent linearized system; this guarantees that the theory applicable to the linearized version around a hyperbolic equilibrium like most controllers are designed and tuned at, including AVRs, PSSs, PI controllers and so on. Nevertheless, this is only valid for small-signal perturbations. However, this fails when such nonlinear systems are subject to large disturbances — not a rare ocurrence in Power Systems.


%% ----------------------------------------------
% BIBLIOGRAPHY {{{1
% -----------------------------------------------
\bibliography{refs}

% -----------------------------------------------
% APPENDIXES {{{1
% -----------------------------------------------

%\part*{Appendixes}
%\appendix

%
%\input{./appendixes/cukSimulations.tex}

%\begin{customendmatter}\end{customendmatter}

\thispagestyle{empty}
% CUSTOM BACK MATTER <<<
\begin{figure}[h]
\centering
\vspace*{-2.6cm}
\makebox[\linewidth]{

\begin{tikzpicture}[transform shape,>={Stealth[inset=0mm,length=1.5mm,angle'=50]}]

% Define A4 cover size: 21cm wide, 29.7cm tall
\fill[darkblue] (0,0) rectangle (\paperwidth, \paperheight);

% Multiple sinewaves across the middle
\foreach \i in {0,...,15} {
    \draw[white, opacity=0.3, thick, samples=200, domain=0:21] 
    plot (\x, {17 + 1.5*sin(deg(0.5*\x) + \i*15)});
}

\node at (\paperwidth/2, 5) {\includegraphics[width=5cm]{../images/uniLogo_white.pdf}};
\end{tikzpicture}
}
\end{figure} %>>>
\cleardoublepage

\end{document}
